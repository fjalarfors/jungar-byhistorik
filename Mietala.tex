\jhchapter{Mietala, hemman Nr 9}

Den första bonden som omnämns på detta hemman var 1) Henrik Larsson år 1565. Det fanns vid den tiden ingen bebyggelse mellan Silvast och Wuoskoski. Den andra bonden på detta hemman var 2) Markus Larsson 1567--\allowbreak 1606. Han skattade år 1605 för 26 spannland odlad jord, mest av bönderna i Jeppo. I samband med att mantalsbestämningen infördes, skattade Mietala hemman för 1 mantal.

Missväxt- och hungerår omväxlande med hårda skatter och soldatutskrivningar gjorde, att mången bonde hamnade att lämna sitt hemman och ge plats åt en annan, oftast från finska Österbotten. De finska hemmansnamnen torde ha tillkommit genom dessa finnar.

Övriga hemmansägare:  3) Markus Markusson 1607--\allowbreak 1641, medåbo Grels Markusson 1616--\allowbreak 1624. 4) Sonen Olof Markusson 1643--\allowbreak 1657. 5) Sigfrid Andersson 1664--\allowbreak 1673. 6) Mårten Sigfridsson 1675--\allowbreak 1677. 7) Simon Hilli 1678--\allowbreak 1681 (från Hilli i Naarasluoma). 8) Matts Olofsson Eko 1683--\allowbreak 1686. 9) Hendrik Sigfridsson 1688--\allowbreak 1697. 10) Sonen Matts Henriksson 1698--\allowbreak 1713. 11) Henrik Persson 1723--.

Nämnas bör också  Zacharias Topelius far, doktor Topelius, som år 1819 köpte två hemmansdelar på Mietala, dvs Sandbacka och Södergård. Målsättningen var mönsterjordbruk för allmogen, vilket skulle tjäna hans målsättning för hälsovårdsarbetet. Hans död 1831 omintetgjorde dessa planer och hemmanen såldes bort.


Mietala hemman omfattas av vidstående karta nr \jhbold{14}.


<--- se KARTA nr 14 --->


\jhsubsection{Lägenheter på Mietala}



\jhhouse{Brännbacka Soldattorp, inkl. torpare}{9}{Mietala}{14}{100}

\jhnooccupant{}

Ansvar för soldattorpet Brännbacka hade hemmanen Keppo, Böös, Mietala och Karkaus. Vid 1752 års granskning framkom att stugans understa stockvarv var förruttnat, takved och näver förfallit av ålder. Ett nytt torpställe utsågs bredvid Palobäcken. Vid syneförrättning år 1791 var stugan och fähuset i gott skick. Boden skulle förses med nytt tak. Åkern på 1 tunnland var i behov av dikning. Kålland och brunn fanns vid torpet.

År 1910 var soldatåkerns tid förbi. Den delades mellan dåvarande intressenter Bösas och Grötas. Mietala var vid den tiden intressent i soldattorpet på Gunnar. Evert Norrgård, \textborn 1905, har berättat att stugan revs ca 1912. Han kom ihåg husgrunden och björkar som hade stått runt stugan. Torpare, som var skrivna på Mietala i början på 1800-talet, kan troligen ha bott här:
\begin{enumerate}
  \item Samuel Jacobsson Appelberg, \textborn 1776 i Wörå, vigd 1805 med Lisa Sigfridsdotter, \textborn 1783 i Socklot, åren 1817--\allowbreak 1844.
  \item Deras dotter Anna Stina, \textborn 1805, var gift med Matts Abrahamss. Katt, \textborn 1794 i Pedersöre. De var också skrivna på Mietala 1824--\allowbreak 1844.
  \item Torparen Jacob Jacobsson, \textborn 1773 i Pedersöre, och hustrun Caisa Pehrsdotter var redan
  1795 skrivna på Mietala.
  \item Pehr Jansson Lillrank, \textborn 1778 i Terjärv fanns med från 1824
\end{enumerate}

\jhbold{Soldaterna}, som bodde i torpet, var följande:
\begin{center}
  \begin{tabular}{l l p{0.68\textwidth}}
    \hline
    Simon Blinck & 1796--\allowbreak 1808 & Simon Simonsson Blinck, \textborn 15.10.1770 i Alavo, gift med pigan Maria Jonaedotter,  \textborn 1770 i Kauhava, 7 barn. Simon deltog i 1808-09 års krig, men lämnade av okänd anledning hären under hösten 1809. Efter kriget var Simon dräng på Kampas rustmästarställe i Överjeppo. Simon dog 04.07.1831. \\
    Hallongren & 1791--\allowbreak 1794 & Carl Magnus Hallongren, \textborn 08.09.1769 i Limingo, furir. 6 år gammal antogs han som minderårig volontär, 17 år gammal blev han befordrad till trumslagare, 1790 till furir. Rotelön från 1791 till 1795. Han deltog i Gustaf III:s krig 1788-90. Hustru Hedvig Helena Hoppström. \\
    Matts Blinck & 1773--\allowbreak 1790 & Matts Bengtsson Blinck, \textborn 25.03.1741 i Vetil, gift med bonddottern Maria Mattsdotter, \textborn 11.08.1738 på Knuts, Ytterjeppo, 7 barn. Matts deltog i Gustaf III:s krig 1788-90. Han dog på krigssjukhus 17.09.1790. \\
    Johan Blinck & 1768--\allowbreak 1773 & Johan Johansson Blinck, \textborn 05.08.1746 i Oravais, gift med soldatdotter Brita Isaacsdotter, \textborn 1746 i Vörå, 5 barn, varav äldsta sonen också blev soldat. Johan dog under kommendering till Helsingfors 1773. \\
    Jacob Blinck & 1760--\allowbreak 1763 & Jacob Blinck, \textborn ca 1738 i Österbotten, gift, men det finns inga uppgifter om hustru eller barn. Jacob torde ha dött under byggkommendering till Sveaborg. \\
    Hinric Nilsson & 1757--\allowbreak 1758 & Hinric Nilsson Blinck, gift med Sophia Jacobsdotter, två barn som båda var födda i Munsala. Under tiden i Kantlax kallades han Henric Klift. Hinric dog i Pommerska kriget år 1758. \\
    A. Lafwerbeck & 1743--\allowbreak 1757 & Anders Lafwerbeck, \textborn 1710 i Pedersöre, gift med Lisa Johansdr Sudd, född i Katternö. Paret hade 8 barn. Vid Pommern-inmönstringen 1757 fick Anders avsked som ``gammal och oduglig''. Hans ålder uppgavs då vara 58 år. Som avskedad bodde Anders på Böös samt hos svärsonen, soldaten Friman på Lussi, där han också dog 1768. \\
    J. Lafwerbeck & 1734--\allowbreak 1742 & Jacob Lafwebeck, \textborn 1705, födelsort okänd, gift (1) med Anna Johansdr, som dog 1739, gifte om sig (2) med bondeänkan Karin Johansdotter Kojola, \textborn 1704 på Tyni i Wuoskoski. Jacob deltog i Hattarnas krig 1741--42. Jacob dog i fält 06.09.1742 \\ \hline
  \end{tabular}
\end{center}



\jhhouse{Paloåkern}{9:91}{Mietala}{14}{1, 1a-d}


\jhhousepic{216-05794.jpg}{Kjell och Annette Forsgård}

\jhoccupant{Forsgård}{Kjell \& Annette}{1991--}
Kjell, \textborn 15.08.1966 i Jeppo, gifte sig 1994 med Annette  Nilsson, \textborn 24.11.1969 från Kronoby. Kjell är agrolog till utbildningen. Han övertog föräldrarnas lägenhet på Mietala år 1991. Annette är tandhygeinist och har under åren arbetat deltid vid sidan av jordbruket, f.t arbetar hon på Tandklinik Hedman/Nygård i Nykarleby.
\begin{jhchildren}
  \item \jhperson{\jhname[Fanny]{Forsgård, Fanny}}{15.01.1997}{}
  \item \jhperson{\jhname[Felix]{Forsgård, Felix}}{15.07.1998}{}
  \item \jhperson{\jhname[Filippa]{Forsgård, Filippa}}{21.08.2002}{}
\end{jhchildren}

Lägenheten består av 73 ha åkermark och 13 ha skog. Huvudnäring har varit potatisodling och kalvuppfödning - sistnämnda avslutades 2017. Yngve Forsgård byggde bostadshuset år 1963 och fähuset 1964. Bostadshuset förstorades och renoverades år 1996 av Kjell och Annette. År 1993 byggde Kjell en byggnad för kalvlösdrift, år 1997 med pappa Yngves hjälp en potatiskällare, år 2011 en maskinhall.

Kjell har varit aktiv inom olika föreningar, bl.a har han varit ordförande i ÖSP:s Jeppo-avdelning, ledamot i fullmäktige  samt ordförande för idrottsföreningen. Kjells helhjärtade insatser för byggandet av Måtarsstugan bör också nämnas.


\jhoccupant{Forsgård}{Yngve \& Berit}{1963--\allowbreak 1991}
Yngve, \textborn 06.01.1933 i Jeppo, gift med Berit Slangar, \textborn 21.12.1939 på Slangar i Jeppo.
\begin{jhchildren}
  \item \jhperson{\jhbold{\jhname[Kjell]{Forsgård, Kjell}}}{15.08.1966}{}
  \item \jhperson{\jhname[Monica]{Forsgård, Monica}}{28.11.1970}{}, socionom, gift Eklund
\end{jhchildren}
Berit hade tjänst på Jeppo Sparbank före giftermålet. Hon och Yngve övertog en tredjedel av hans föräldrars lägenhet. Senare köpte de till odlad jord för sitt jordbruk. Yngve byggde egnahemshuset och fähus på Paloåkern. Berit har bl.a varit principal i sparbanken, kassör i Pensionärs- och Marthaföreningen samt med i Hembygdsföreningens och Jeppo ungdomsförenings styrelse.

Till Yngves fritidsintresse hörde musiken. Som ung sjöng och spelade han i flera orkestrar, bl.a med sina bröder i orkestern Rythm Boys. Musikaliska uppdrag följde honom sedan livet ut. I flere år var han ledare för pensionärsföreningens sångare.

Yngve och Berit flyttade till radhuslägenhet vid Åkervägen 7 i samband med generationsväxlingen. Yngve \textdied 10.10.2013.



\jhhouse{Riåkern}{9:109}{Mietala}{14}{2, 2a}


\jhhousepic{218-05901.jpg}{Tony Kalijärvi och Janina Englund}

\jhoccupant{Kalijärvi \& Englund}{Tony \& Janina}{2016--}
Tony Kalijärvi, \textborn 15.04.1990 i Jeppo och Janina Englund, \textborn 11.01.1992 i Jakobstad, köpte gården år 2016. Tony arbetar på Mirka i Jeppo och Janina arbetar på Sale i Bennäs.


\jhoccupant{Norrgård}{Olof \& Dagmar}{1978--\allowbreak 2016}
Olof Norrgård, \textborn 24.10.1935 på Mietala, gift med Dagmar Sundell, \textborn *02.05.1935  i Jeppo. Olof har arbetat som minkfarmare och chaufför, Dagmar som butiksföreståndare och kontorist. En ny bostadsbyggnad uppfördes år 1978 av Oravais Hus och paret flyttade in. Garaget uppfördes år 1979 av Evert Norrgård, Olofs far.

Olofs fritidsintresse var musik, han spelade som ung i Jeppo hornorkester.

Olof \textdied 30.04.2014  ---  Dagmar \textdied 31.01.2016



\jhhouse{Kvarnfors}{9:77}{Mietala}{14}{102}


\jhhousepic{Mietala102-NorrgardEv.jpg}{Evert och Signe Norrgård}

\jhoccupant{Norrgård}{Evert \& Signe}{1951--\allowbreak 1990}
Evert Norrgård, \textborn 01.02.1905, gifte sig 06.11.1932 med Signe Gunell, \textborn 02.04.1906 i USA. Evert var byggnadssnickare och chaufför. Han har byggt ett tiotal egnahemshus i Jeppo. På 1950-talet startade han tillsammans med Karl Lindgren den \jhbold{första privata minkfarmen i Jeppo}, senare fanns också en tredje delägare, nämligen H Bäckman, farmchef på Keppo. Bäckman emigrerade senare till Australien. Farmen avslutades 1971 pga dåliga skinnpriser och Evert blev pensionär.

Evert var en naturvän, som sökte ovanliga blommor och växter i åbacken. Han kom ofta in med den första konvaljbuketten på våren. Signe var med i Marthaföreningen. Hon städade vid Gunnar småskola under ett par år.
\begin{jhchildren}
  \item \jhperson{\jhname[Ingmar]{Norrgård, Ingmar}}{08.06.1933}{}, tekniker, bor i Sverige
  \item \jhperson{\jhbold{\jhname[Olof]{Norrgård, Olof}}}{24.10.1935}{}, chaufför
  \item \jhperson{\jhname[Börje]{Norrgård, Börje}}{24.11.1938}{02.08.2008}, tekniker, Smedsby
  \item \jhperson{\jhname[Marléne]{Norrgård, Marléne}}{03.12.1943}{}, merkonom, bor i Larsmo
\end{jhchildren}
Evert byggde bostadshuset år 1951 samt ekonomibyggnaden 1954. Boningshuset uppfört i spirvirke med sågspån som isolering. Huset revs i januari 2010. Lägenheten har tillhört samma släkt sedan 1800-talet och övertogs av Evert efter fadern år 1923.

Signe \textdied 23.06.1988	---	Evert \textdied 31.07.1990



\jhhouse{Norrgård}{9:78}{Mietala}{14}{3, 3a}


\jhoccupant{Norrgård}{Bo \& Marita}{2014--}
Ägare till denna gård samt tomt är sedan år 2014 Bo och Marita Norrgård. Se närmare Mietala gård nr 5.\jhvspace{}


\jhhousepic{219-05795.jpg}{Rune Norrgård fram till 2014, nu Bo och Marita}

\jhoccupant{Norrgård}{Rune}{1973--\allowbreak 2014}
Rune Norrgård, \textborn 03.09.1949 på Mietala. Rune blev student från Nykarleby Samskola 1970 och studerade därefter språk och teologi vid Åbo akademi. År 1973 övergick hemmanet i hans namn. Han blev dock sjukpensionerad pga ohälsa och bodde ensam i hemgården till sin död, 27.01.2014.


\jhoccupant{Norrgård}{Lennart \& Edit}{1951--\allowbreak 1973}
Lennart Norrgård, \textborn 18.06.1908 på Mietala, gift med Edit Backlund,	\textborn 19.03.1910 på Måtar.
\begin{jhchildren}
  \item \jhperson{\jhname[Erik]{Norrgård, Erik}}{27.05.1936}{}, (Silvast 355 )
  \item \jhperson{\jhname[Göran]{Norrgård, Göran}}{23.08.1941}{}, bilmekaniker, bor i Sverige
  \item \jhperson{\jhname[Bo]{Norrgård, Bo}}{19.03.1945}{}, (Mietala 5)
  \item \jhperson{\jhbold{\jhname[Rune]{Norrgård, Rune}}}{03.09.1949}{}
\end{jhchildren}

Edit och Lennart var småbrukare på ena halvan av hans fars lägenhet. Lennarts yrkesverksamma liv präglades dock mer av de arbeten, som han genom sitt tekniska intresse och sin kreativitet innehade. Han arbetade bl.a som maskinist på Jungar mejeri. Skomakaruppdrag tog han emot fram till 1950-talet, då grävmaskinsarbeten upptog hans tid.

Bolaget ``Jeppo Gräv'' införskaffade en grävmaskin och Lennart samt Ensio Kula var de första maskinskötarna. Skomakarverktygen och -uppdragen övertogs av Uno Elenius. Lennart hade också en pärthyvel och en såg, som han körde runt med beroende på var uppdragen fanns. Nere vid älven hade han en smedja, som speciellt sommartid besöktes då verktyg behövde repareras.

Lennart byggde huset 1951. Tidigare bodde han med sin familj i hus \jhbold{nr 103}, nedan. I samma gård bodde brodern Evert och hans familj. Lennart var den första personen i Jeppo som lyfte en blygsam pension enligt lantbrukets nya pensionssystem.

Lennart \textdied 29.06.1980  ---  Edit \textdied 03.03.1982



\jhhouse{Norrgård}{9:49}{Mietala}{14}{103}


\jhoccupant{Österbottens amb.}{Hemslöjdsskola}{1951--\allowbreak 1952}
Österbottens ambulerande hemslöjdsskola var inhyst i huset efter att familjerna Norrgård byggt varsitt eget hus. Clarence Back från Kimo var lärare. Skolan hade 9 elever från orten: Valfrid Häggstrand, Ruben Eklöv, Uno Gunell, Evert Norrgård, Yngve Forsgård, Kurt Forsgård, Manne Strand, Gunnar Forsbacka och Jürgen Nylind. Man slöjdade köksinredningar, allmogebord, bänkar, sängar mm. Huset revs år 1958 av ägarna Lennart och Evert Norrgård.


\jhhousepic{Mietala103-Norrgard.jpg}{Gustaf och Anna Norrgård, hus 103}

\jhoccupant{Norrgård}{Gustaf \& Anna}{1898--\allowbreak 1923}
Bonden Gustaf Johan Jakobsson Norrgård, \textborn 01.12.1878 på Mietala, gift med Anna Simonsdotter Lavast, \textborn 11.04.1879.
\begin{jhchildren}
  \item \jhperson{\jhname[Anna Irene]{Norrgård, Anna Irene}}{12.03.1903}{}, till Åland som tjänarinna, gift där
  \item \jhperson{Gustav \jhbold{Evert}}{01.02.1905}{}
  \item \jhperson{\jhname[Valfrid]{Norrgård, Valfrid}}{20.04.1907}{03.05.1907}
  \item \jhperson{\jhbold{\jhname[Lennart]{Norrgård, Lennart}}}{18.06.1908}{}
  \item \jhperson{\jhname[Sigurd Johannes]{Norrgård, Sigurd Johannes}}{25.12.1910}{}, (Romar 36 )
  \item \jhperson{\jhname[Gerda Johanna]{Norrgård, Gerda Johanna}}{19.08.1913}{}, till Åland 1936, gifte sig där
\end{jhchildren}

Genom afhandling 24 november 1898 blev Gustaf Johan Jakobsson Mietala ägare till 1/18 mantal av Mietala hemman utav föräldrarna Johan Jakob Jakobsson Mietala och hustrun Johanna Andersdotter. Föräldrarna hade tre år tidigare överlämnat 1/18 mtl till brodern Anders.

Huset, som Gustaf fick överta, var byggt i Härmä, flyttades till Mietala nån gång i början på 1800-talet. Gustaf Norrgård deltog aktivt i den kommunala verksamheten, bl.a. var han åren 1910--\allowbreak 1915 ordförande i kommunalstämman. Han var också invald i kommunalfullmäktige och hörde till olika kommunala nämnder.

Den 07.06.1915 dog Anna och Gustav blev ensam med att ta hand om barnen. Irene, som var 12 år, fick axla husmorsansvaret i gården. Pappa Gustav kallades ut i kriget och blev skadad. Han dog då Irene var 20 år gammal, den 02.11.1923. Kvar i familjen fanns nu fem föräldralösa barn, varav ingen var myndig. De var utlämnade åt sig själva att sköta skola, hem och gård. Det var fattigt och många gånger hade de endast en rova att ta med sig till skolan. Gerda fick ett erkännande år 1978, då hon på morsdagen tilldelades Finlands Vita Ros.

Gustav hade 1923 överlåtit hemmandet till sönerna Evert och Lennart. Efter att ha bildat familj, bodde bröderna i var sin halva av gården.


\jhoccupant{Johansson}{Johan Jakob \& Anna}{1853--\allowbreak 1898}
Bondsonen Johan Jakob Johansson d.ä, \textborn 21.01.1831 på Mietala, vigd 06.11.1850 med bonddottern Anna Eriksdotter Fors, \textborn 06.11.1833 --- \textdied 19.03.1858. Johan Jakob ingick nytt äktenskap 11.11.1860 med Johanna Andersdotter Kauhajärvi, \textborn 28.08.1837, \textdied 28.11.1898. Johans tredje hustru var Maria Johansdotter,  \textborn 06.03.1844.
Anna framfödde fyra barn av vilka två dog i tidig ålder.
\begin{jhchildren}
  \item \jhperson{\jhname[Maria]{Johansson, Maria}}{25.10.1851}{}, gift med Johan Lindqvist
  \item \jhperson{\jhbold{\jhname[Johan Jakob]{Johansson, Johan Jakob}}}{24.10.1853}{}, (Forsgård)
  \item \jhperson{\jhname[Anna Sofia]{Johansson, Anna Sofia}}{09.09.1855}{1861}
  \item \jhperson{\jhname[Johanna]{Johansson, Johanna}}{11.01.1858}{1858}
\end{jhchildren}

Johanna födde åtta barn till familjen.
\begin{jhchildren}
  \item \jhperson{\jhbold{\jhname[Anders]{Johansson, Anders}} Wilhelm}{21.02.1861}{}, (Nord)
  \item \jhperson{\jhname[Isak]{Johansson, Isak}}{16.07.1862}{}
  \item \jhperson{\jhname[Sofia]{Johansson, Sofia}}{23.08.1863}{}
  \item \jhperson{\jhname[Simon]{Johansson, Simon}}{09.07.1866}{}
  \item \jhperson{\jhname[Johanna]{Johansson, Johanna}}{02.08.1870}{}
  \item \jhperson{\jhname[Wilhelmina]{Johansson, Wilhelmina}}{17.04.1873}{}
  \item \jhperson{\jhbold{\jhname[Gustaf]{Johansson, Gustaf}}}{01.12.1878}{}, (Norrgård)
  \item \jhperson{\jhname[Otto]{Johansson, Otto}}{01.11.1880}{20.12.1893}
\end{jhchildren}
Johan Jakob erhåller genom gåvobrev den 26 januari 1853 av sina föräldrar 1/9 dels mantal av Mietala hemman. I februari 1880 köpte makarna ett hemman av Simon Eriksson Mietala. Till detta köp hörde stugan samt hälften av alla andra hus. I juli 1884 brann stallet, ladan, en boda och ett loft ner. Om inte Otto von Essens brandspruta hämtats så snabbt, så hade en annan uthusbyggnad med all säkerhet blivit lågornas rov, enligt en notis i dagstidningen.

Jakob Mietala \textdied 23.05.1918, Maria flyttade året därpå till Jungar.


\jhoccupant{Henriksson}{Johan \& Maja Lisa}{1814--\allowbreak 1853}
Bds Johan Henriksson, \textborn 03.07.1793 på Mietala, vigdes 07.04.1815 med Greta Eriksdotter, \textborn 22.09.1793. Tillsammans fick de ett barn; Henrik, \textborn 04.09.1815.
Greta \textdied 09.11.1818.

Johan ingick nytt äktenskap med Maja Lisa Simonsdotter Jungar, \textborn 19.05.1799.
\begin{jhchildren}
  \item \jhperson{\jhname[Simon]{Henriksson, Simon}}{15.07.1820}{29.07.1820}
  \item \jhperson{\jhname[Greta]{Henriksson, Greta}}{19.06.1826}{}
  \item \jhperson{\jhname[Sanna Lisa]{Henriksson, Sanna Lisa}}{26.10.1828}{14.04.1867}
  \item \jhperson{\jhbold{\jhname[Johan Jakob]{Henriksson, Johan Jakob}}}{21.01.1831}{}
  \item \jhperson{\jhname[Maja]{Henriksson, Maja}}{22.04.1833}{}
  \item \jhperson{\jhname[Caisa]{Henriksson, Caisa}}{21.07.1835}{}, till Alahärmä
  \item \jhperson{\jhbold{\jhname[Sofia]{Henriksson, Sofia}}}{25.12.1837}{}
  \item \jhperson{\jhname[Johanna]{Henriksson, Johanna}}{01.11.1840}{}
\end{jhchildren}

År 1814 skrev Johan Jakobs mor och hennes andra man Carl Mietala över hemmanet åt sönerna Johan och Eric samt ansökte om klyvning av hemmanet. År 1836 köpte Johan 1/9 mantal av Mietala hemman av bonden Johan Thomasson Tolliko. I köpebrevet åläggs Johan att uppföra en stuga, 6 alnar i kvadrat, åt änkan Susanna Danielsdotter, Erik Henrikssons änka. Erik Henriksson, som var bror till Johan, dog 16.10.1833.

Uppgifter om när makarna Johan och Maja Lisa avlidit har inte hittats.


\jhoccupant{Henriksson}{Henrik \& Greta}{- 1814}
Henrik Henriksson, \textborn 11.03.1766 på Mietala, gift med Greta Mattsdr, \textborn 29.12.1766 i Kovjoki. Henrik dog 28.10.1798, Greta gifte om sig med Carl Eriksson Mietala, \textborn 21.10.1773.
\begin{jhchildren}
  \item \jhperson{\jhname[Mathias]{Henriksson, Mathias}}{30.01.1791}{04.02.1791}
  \item \jhperson{\jhname[Maria]{Henriksson, Maria}}{15.07.1792}{}
  \item \jhperson{\jhbold{\jhname[Johan]{Henriksson, Johan}}}{03.07.1793}{}
  \item \jhperson{\jhname[Jakob]{Henriksson, Jakob}}{06.09.1794}{19.09.1794}
  \item \jhperson{\jhname[Henrik]{Henriksson, Henrik}}{09.10.1795}{09.08.1796}
  \item \jhperson{\jhname[Anders]{Henriksson, Anders}}{08.12.1796}{}
  \item \jhperson{\jhbold{\jhname[Eric]{Henriksson, Eric}}}{17.01.1798}{}
\end{jhchildren}

År 1845 utgjordes Mietala av 2/3 mantal skatte- och 1/24 mantal kronojord. Samma år var Johan Henriksson samt medåboerna Johan Johansson Mietala och Anders Hansson gemensamma ägare till 1/3. År 1833 var Erik Henriksson och Johan Henriksson gemensamma ägare till 1/3 mantals skattehemman, år 1834 Johan Henriksson och Erik Henrikssons änka Susanna Danielsdotter.



\jhhouse{Mietala Torp}{9}{Mietala}{14}{101}


\jhoccupant{Nordqvist}{Johan \& Ida}{1909--\allowbreak 1923}
Johan Jakob Anders Vilhelmsson, \textborn 15.10.1885 på Mietala, gift med Ida Sofia Andersdotter Eklund, \textborn 22.12.1888 på Stenbacka.
\begin{jhchildren}
  \item \jhperson{\jhname[Mildrid Elvira]{Nordqvist, Mildrid Elvira}}{01.06.1910}{}
  \item \jhperson{\jhname[Anna Valdine]{Nordqvist, Anna Valdine}}{11.01.1912}{}
  \item \jhperson{\jhname[Signhild Linnea]{Nordqvist, Signhild Linnea}}{09.02.1913}{}
\end{jhchildren}

Johan reste till Amerika 17.11.1913. Ida bodde kvar med barnen på Mietala. Hon bakade och sålde bulla i hemmet. I november 1920 ansökte Ida om att få lösa in ett område tillhörande Edvard Mietala, men hon hade inte inlösningsrätt emedan hon inte bodde där, utan hade sitt torp på annan mark. Ida och barnen reste till mannen i Amerika i juli 1923.


\jhoccupant{Blink}{Herman \& Anna}{1888--\allowbreak 1903}
Backstugukarlen Herman Johan Pettersson Blink, \textborn 04.10.1842 i Jeppo, gift med Anna Henriksdotter, \textborn 01.05.1841 i Jeppo. Blink kom år 1888 från Alahärmä. Han är skriven under samma hemman som Ida Norrgård/Nordqvist. Besökt nattvard ännu 1903. År 1904 är de skrivna på Ruotsala. Sonen Gustaf (Lindström), \textborn 22.06.1882, reser till Amerika år 1900.


\jhoccupant{Slip}{Simon \& Maja}{1843--\allowbreak 1880}
Torparen Simon Hermansson Slip, \textborn 30.03.1810 i Purmo, gift med Maja Lena Johansdotter, \textborn 13.06.1810 i Purmo.
\begin{jhchildren}
  \item \jhperson{\jhname[Erik]{Slip, Erik}}{03.04.1832 i Purmo}{}
  \item \jhperson{\jhname[Beata Lena]{Slip, Beata Lena}}{14.09.1833 i Purmo}{}
  \item \jhperson{\jhname[Maja Lovisa]{Slip, Maja Lovisa}}{20.09.1838 i Purmo, till Alahärmä 1861}{}
  \item \jhperson{\jhname[Anna Lisa]{Slip, Anna Lisa}}{16.09.1842 i Purmo, till Alahärmä 1859}{}
  \item \jhperson{\jhname[Johan Jakob]{Slip, Johan Jakob}}{02.12.1845 i Jeppo}{1846}
  \item \jhperson{\jhname[Brita Caisa]{Slip, Brita Caisa}}{01.04.1847 i Jeppo}{}
\end{jhchildren}

Simon var torpare under Johan Jakob Johansson d.ä:s hemman. Han var dock en kort tid bonde, då han år 1858 köpte 1/12 hemman av bn Anders Andersson Mietala. År 1868 säljer han sin torplägenhet, vars kontrakt gjorts 1846, åt Patron von Essen med villkor att von Essen inte tillträder torplägenheten förrän efter tre år. Inom denna tid har Simon rätt att återlösa den. Simon dog samma år. Änkan Maja Lena bodde kvar på Mietala som inhyses, dock oklart om det är just i detta torp.

Simon \textdied 01.06.1868  ---  Maja Lena \textdied 1870 – 1886.



\jhhouse{Nord}{9:88}{Mietala}{14}{4, 4a}


\jhhousepic{222-05800.jpg}{Äldsta gården i Jungar by och i hela Jeppo}

\jhoccupant{Nord}{Sven \& Britta}{1989--}
Sven Nord, \textborn 21.04.1942 på Mietala, gifte sig 18.06.1977 med Britta Nymark, \textborn 06.10.1941 från Jakobstad. Makarna bor i Nykarleby och har inte bott i gården, som stått tom sedan Svens föräldrar dog. Sven skötte dock i många år lägenheten omfattande totalt 35 ha.

Enligt Runar Nyholm torde Nords gård vara den äldsta gården i Jeppo. ``Den har byggts på Jungar på 1600-talet, där den i ett par sekler utgjorde jungarsläktens stamgård – tolvmansgården. På 1800-talet byggdes en ny gård, och den gamla trotjänaren såldes åt Sven Nords förfäder, som har bebott den det senaste seklet. Efter vad en gammal man berättat, ville säljaren att gården skulle uppföras precis på samma sätt som den stått förut och med gaveln mot landsvägen. Rådet följdes till punkt och pricka, och därför kan man här se en exakt 1600-tals bondgård.'' En lång, smal och mörk farstu samt ett cirkelrunt litet fönster på gaveln mot öster är också tidstypiskt.


\jhoccupant{Nord}{Anders \& Elvira}{1947--\allowbreak 1989}
Anders Vilhelm Nord, \textborn 30.11.1899 på Mietala, gifte sig 08.11.1936 med Anni Elvira Skog, \textborn 04.09.1912.
\begin{jhchildren}
  \item \jhperson{\jhname[Viola]{Nord, Viola}}{21.04.1942}{15.08.2011}, gift med Boris Norrgård
  \item \jhperson{\jhbold{\jhname[Sven]{Nord, Sven}}}{21.04.1942}{}
  \item \jhperson{\jhname[Margareta]{Nord, Margareta}}{04.01.1945}{}, gift med Jan-Erik Nybyggar
\end{jhchildren}

Anders byggde ekonomibyggnaden år 1954. På lägenheten funnits fähus, stall, kärrlada, sädesbod. Dessa var byggda i slutet på 1800 och revs 1955. Elvira var bland de första hårfrisörskorna i Jeppo.

Anders \textdied 09.05.1973  ---  Elvira \textdied 18.02.1989


\jhoccupant{Jakobson}{Anders \& Anna}{1884--\allowbreak 1947}
Anders Vilhelm Jakobson, \textborn 21.02.1861 i Jeppo, gift med Anna Sanna Jakobsdotter, \textborn 22.07.1862.
\begin{jhchildren}
  \item \jhperson{\jhname[Johan Jakob]{Jakobson, Johan Jakob}}{15.10.1885}{}, till Amerika
  \item \jhperson{\jhname[Anders Vilhelm]{Jakobson, Anders Vilhelm}}{21.05.1887}{02.04.1890}
  \item \jhperson{\jhname[Otto]{Jakobson, Otto}}{03.09.1889}{06.01.1890}
  \item \jhperson{\jhname[Hilda Johanna]{Jakobson, Hilda Johanna}}{08.10.1893}{20.08.1894}
  \item \jhperson{\jhname[Hilda Johanna]{Jakobson, Hilda Johanna}}{01.12.1895}{17.04.1916}
  \item \jhperson{\jhname[Anna Juliana]{Jakobson, Anna Juliana}}{23.12.1897}{27.07.1980}
  \item \jhperson{\jhbold{\jhname[Anders]{Jakobson, Anders}} Vilhelm}{30.11.1899}{}
  \item \jhperson{\jhname[Signe Maria]{Jakobson, Signe Maria}}{04.11.1902}{13.11.1997}, g. Kytömäki
  \item \jhperson{\jhname[Sanni Sofia]{Jakobson, Sanni Sofia}}{01.12.1904}{05.01.1966}, g. Kronqvist
\end{jhchildren}

Anders och Anna får i sin ägo 1/18 mantal av Mietala hemman genom avhandling 16.03.1895 utav Anders föräldrar Johan Jakob Johansson och Johanna Andersdotter mot utlösen av andra syskon samt sytning åt föräldrar. Denna del hade föräldrarna handlat av Simon Eriksson Mietala (senare Sandbacka/Sandell) den 27 februari 1880. -- Uppgifter om Johan Jakob Johansson d.ä. under Mietala 103.

Anna \textdied 23.03.1943  ---  Anders \textdied 06.06.1947



\jhhouse{Dahl}{9:21}{Mietala}{14}{104}

\jhoccupant{Dahl}{Anna Lovisa}{1896--\allowbreak 1944}
Anna Lovisa, \textborn 24.08.1854 i Esse, gift med Jakob Forsman, \textborn 29.11.1852 på Grötas. Jakob och Anna Lovisa bodde första åren på Ruotsala tillsammans med Jakob Johansson Forsman och hans familj, senare var de skrivna under lösa personer.
\begin{jhchildren}
  \item \jhperson{\jhname[Anders Gustav]{Dahl, Anders Gustav}}{06.01.1881}{}
  \item \jhperson{\jhname[Johan Wilhelm]{Dahl, Johan Wilhelm}}{03.03.1884}{}
  \item \jhperson{\jhname[Mats Leander]{Dahl, Mats Leander}}{10.12.1886}{}
\end{jhchildren}

År 1896 flyttade Anna Lovisa in i torpstugan på Silta. Jakob torde tidigare ha emigrerat till USA, eftersom Anna 1892 är antecknad som frånskild. Jakob gifte om sig i USA och fick där 5 barn. De tre sönerna från giftet med Anna Lovisa emigrerade senare till USA. Anna Lovisa får en son, Wiktor, år 1889, som dör 1890.

Anna Lovisa gifte om sig år 1893 med änklingen, f.d. skarpskytten Anders Peltomaa, senare Ekoluoma eller Dahl, \textborn 21.7.1837 i Kortesjärvi. Han hade 7 barn från sitt tidigare äktenskap. År 1892 efter hustruns död flyttade han till Jeppo med yngsta dottern Anna Adolfina \textborn 13.04.1883.

Anders Dahl \textdied 26.11.1908.

Efter att torparlagen gett möjlighet till inlösen av backstuguområden, ansökte Anna Lovisa om att få lösa in sitt område, som var 3 ¾ kappland. Området tillhörde bonden Jacob Forsgård. Det fanns inget skriftligt kontrakt, men i.o.m att Anna Lovisa årligen utfört 2 dagsverk ansågs det inlösningsbart. Dagsverksskyldigheten upphörde i samband med betalningen.

Adolfina gifte sig med Elias Kangas, \textborn 1881 i Kuortane och de flyttade en hel del mellan torpen på Jungar (1897--09), Gunnar (1887-97) och Mietala. De bodde på Silta/Mietala tillsammans med Anna Lovisa åtminstone åren 1910-19. Elias emigrerade också till Amerika. Adolfina och Elias barn:
\begin{jhchildren}
  \item \jhperson{\jhname[Anders Elias]{Dahl, Anders Elias}}{26.07.1903}{}
  \item \jhperson{\jhbold{\jhname[Johannes Evald]{Dahl, Johannes Evald}}}{31.05.1905 på Jungar}{}
\end{jhchildren}

Adolfinas övriga barnaskara gjorde familjen försvarligt stor.
\begin{jhchildren}
  \item \jhperson{\jhname[Toivo]{Dahl, Toivo}}{1908}{}
  \item \jhperson{\jhname[Helvi]{Dahl, Helvi}}{1911}{}
  \item \jhperson{\jhname[Sulo]{Dahl, Sulo}}{1913}{}
  \item \jhperson{\jhname[Jaakko]{Dahl, Jaakko}}{1916}{}
  \item \jhperson{\jhname[Antti]{Dahl, Antti}}{1917}{}, (till Snappertuna 5.6.1941)
  \item \jhperson{\jhname[Taisto]{Dahl, Taisto}}{1920}{}
\end{jhchildren}

Johannes Evald kom att bo tillsammans med Anna Lovisa under många år. Efter att ha gift sig med Hilda Stenbacka,  \textborn 03.10.1908, bodde han med sin nya fru ännu en tid kvar på Silta (1921--\allowbreak 1930). De flyttade 1931 till Karleby. Evald var stationskarl och senare konduktör.

Anna Lovisa kallades ``Ståoanskon''. Som yngre arbetade hon som piga bl.a på Böös. Äldre personer kommer ihåg att hon tillverkade kvastar, som hon sålde. Man kommer ihåg henne komma dragande med kvastarna i en barnvagn. Bonden Fagerholm höll henne med en liten äng bredvid Björkvalls ria. Där hade hon fåren att beta.  Senare byttes denna äng ut till ett likadant område på Bomossan.

Adolfina \textdied 06.10.1927  ---  Anna Lovisa Dahl \textdied 19.03.1944


\jhoccupant{Björn}{Tobias \& Sanna}{1875--\allowbreak 1898}
Backstugukarlen Tobias Björn, \textborn 09.03.1856 i Sysmä, gift med Sanna Isaksdotter Löfqvist, \textborn 17.07.1844 i Alahärmä.
Från första giftermålet hade Sanna tre barn.
\begin{jhchildren}
  \item \jhperson{\jhname[Susanna Karl Gustafsdr]{Björn, Susanna Karl Gustafsdr}}{01.09.1870}{}, till Helsingfors 1890
  \item \jhperson{\jhname[Anna Lovisa Löfqvist]{Björn, Anna Lovisa Löfqvist}}{05.10.1874}{}, till Karstula 1892
  \item \jhperson{\jhname[Katrina Löfqvist]{Björn, Katrina Löfqvist}}{30.07.1877}{}, till Amerika 1896
\end{jhchildren}

Barn med Tobias: Amanda, \textborn 25.08.1886
Av kyrkböckerna får man den uppfattningen att Tobias Björn arbetat på annan ort. År 1888 flyttar han till Savolax, men hustrun och barnen besökte nattvard ännu 1897 och 1898. Vid denna tidpunkt har Sanna fått betyg till Amerika.


\jhoccupant{Heikfolk}{Gustaf \& Caisa}{1850--\allowbreak 1880}
Gustaf Isaksson Heikfolk, \textborn 08.11.1821 i Ytterjeppo, gift med Caisa Greta Eriksdotter, \textborn 22.12.1817 i Ytterjeppo.
\begin{jhchildren}
  \item \jhperson{\jhname[Susanna]{Heikfolk, Susanna}}{11.11.1843 i Ytterjeppo}{}
  \item \jhperson{\jhname[Catharina]{Heikfolk, Catharina}}{19.10.1845 i Kauhava}{}
  \item \jhperson{\jhname[Isak]{Heikfolk, Isak}}{12.08.1847 i Ytterjeppo}{}
  \item \jhperson{\jhname[Maria]{Heikfolk, Maria}}{30.08.1849  ''}{}
  \item \jhperson{\jhname[Erik]{Heikfolk, Erik}}{24.09.1851 på Mietala}{}
  \item \jhperson{\jhname[Johan Jakob]{Heikfolk, Johan Jakob}}{07.08.1854  ''}{}
  \item \jhperson{\jhname[Lovisa]{Heikfolk, Lovisa}}{03.09.1857  ''}{}
  \item \jhperson{\jhname[Anna]{Heikfolk, Anna}}{30.12.1859  ''}{}
  \item \jhperson{\jhname[Gustaf]{Heikfolk, Gustaf}}{22.10.1862  ''}{}
\end{jhchildren}
I köpekontrakt mellan Johan Jakob Johansson d.ä och sonen Johan Jakob (senare Forsgård) samt hans hustru Anna Isaksdotter år 1879 ``tryggas torparen Gustaf Isakssons torpkontrakt''.



\jhhouse{Dalabacka}{9:23}{Mietala}{14}{105}


\jhhousepic{Mietala105-Huht-a.jpg}{Antti och Signe Huhtas hus.}

\jhoccupant{Huhta}{Signe \& Antti}{1927--\allowbreak 1983}
Signe Maria Sandberg, \textborn 16.03.1902 på Mietala, gift 1927 med Antti Huhta, \textborn 25.10.1902 i Alahärmä. Före giftermålet arbetade Maria vid Jeppo yllespinneri på Kiitola. Antti kom till Jeppo som dräng, men arbetade även på järnvägen med reparationsarbeten och med skogs- och dikningsarbeten. De bodde i Marias föräldrahem på Silta. Gården bestod av 1 rum och kök. - Dage Häggstrand äger idag området, gården revs år 2000.
\begin{jhchildren}
  \item \jhperson{\jhname[Alvar]{Huhta, Alvar}}{14.09.1928, till Sverige}{}
  \item \jhperson{\jhname[Gunvor]{Huhta, Gunvor}}{07.01.1930, gift Widegren i Sverige}{}
  \item \jhperson{\jhname[Doris]{Huhta, Doris}}{10.06.1933, gift Storm i Sverige}{}
\end{jhchildren}

Antti \textdied 22.03.1966  ---  Maria \textdied 29.11.1983


\jhoccupant{Sandberg}{Mathias \& Anna}{1883--\allowbreak 1927}
Mathias (Matts) Jakobsson, \textborn 09.07.1852 på Mietala, gift med Anna Johansdotter, \textborn 20.08.1863 på Måtar. Matts var som ung dräng på Måtar, senare backstugusittare. Familjen livnärde sig på att ta del i böndernas arbete än här och än där. Barnen nedan födda på Mietala.
\begin{jhchildren}
  \item \jhperson{\jhname[Ida]{Sandberg, Ida}}{07.11.1884}{}, till Amerika 1902
  \item \jhperson{\jhname[Johan Jakob]{Sandberg, Johan Jakob}}{29.12.1886}{15.01.1887}
  \item \jhperson{\jhname[Mathias]{Sandberg, Mathias}}{24.01.1888}{04.02.1888}
  \item \jhperson{\jhname[Anna]{Sandberg, Anna}}{07.02.1889}{26.02.1889}
  \item \jhperson{\jhname[Johannes]{Sandberg, Johannes}}{16.10.1892}{}, till Amerika 1910
  \item \jhperson{\jhname[Anders]{Sandberg, Anders}}{12.10.1897}{25.07.1898}
\end{jhchildren}

Efter Annas död 03.12.1898 gifte Mathias om sig med Annas syster, Maria Johansdotter,  \textborn 24.05.1858. Ida var då 14 och Johannes 6 år. 18 år gammal begav sig Ida till Amerika. Hon uppmanades av sin far att hålla reda på biljetten eftersom hon rest så lite ``bara mellan Silta och Mietas''.

Barn med Maria: Signe Maria, \textborn 16.03.1902

Den 12.04.1921 höll legonämnden möte gällande Mats Sandbergs ansökan om inlösen av familjens backstuguområde på Silta. Bonden Erik Elenius var ägare till området. Det fanns inget skriftligt kontrakt, men då Matts sedan 40 år tillbaka innehaft området och till jordägaren erlagt 9 dagsverken årligen ansågs området inlösningsbart enligt torparlagen. Matts Sandberg betalade 525 mark och var därmed också fri från dagsverksskyldigheten.

I boken om Svenska Österbottens bebyggelse skrivs att bostaden är uppförd 1870 i trä under pärttak. Både Maria och Mathias dog år 1935; Maria \textdied 07.10.1935  ---  Mathias \textdied 14.05.1935.


\jhhouse{Svanbäck}{9:18}{Mietala}{14}{106}


\jhhousepic{Mietala106-Svanback.jpg}{Otto och Kaisa Svanbäck}

\jhoccupant{Svanbäck}{Otto \& Brita}{1887--\allowbreak 1900, 1914--\allowbreak 1939}
Otto Wilhelm Karlsson Svanbäck, \textborn 06.07.1860 i Tohmajärvi, gift med Brita Kaisa Mattsdotter Mietala, \textborn 10.07.1858 på Mietala.
\begin{jhchildren}
  \item \jhperson{\jhname[Hilda]{Svanbäck, Hilda}}{10.10.1881}{1899}
  \item \jhperson{\jhname[Karl Edvard]{Svanbäck, Karl Edvard}}{04.07.1883}{}
  \item \jhperson{\jhname[Otto Leonard]{Svanbäck, Otto Leonard}}{29.12.1886}{1913}
  \item \jhperson{\jhname[Johannes]{Svanbäck, Johannes}}{03.03.1888}{18.05.1888}
  \item \jhperson{\jhname[Oscar]{Svanbäck, Oscar}}{10.03.1889}{}, stationskarl, g. m. Ida Sannasdr Slangar (karta 8, nr 371)
  \item \jhperson{\jhname[Wendla]{Svanbäck, Wendla}}{30.04.1892}{1910}
  \item \jhperson{\jhname[Johannes]{Svanbäck, Johannes}}{25.04.1895}{}
  \item \jhperson{\jhname[Ellen Katrina]{Svanbäck, Ellen Katrina}}{01.05.1898}{1899}
  \item \jhperson{\jhname[Elmer]{Svanbäck, Elmer}}{29.03.1901}{06.02.1914}
\end{jhchildren}

Oscar bodde också här med sin familj före de flyttade till Silvast. Otto var skriven som backstugosittare på Mietala. Han arbetade som dräng, men också med andra arbeten, vilket framkommer av en tidningsartikel i Wasa Tidning 30.03.1890. I artikeln berättas om två arbetare, som var sysselsatta med stensprängning vid Jungarå fors, där en ny kvarn skulle anläggas. Den ena av dem var Otto Svanbäck. När de höll på att ladda ett skott brann detsamma av och träffade dem i ansiktet. Enligt tidningen miste Otto synen helt på ena ögat och ett finger på vänster hand blev söndersplittrat.

Den 22.10.1920 godkändes Ottos ansökan om inlösen av det 7 kappland stora backstuguområdet. Emil Mietala var då ägare till området. Före Otto kom med sin familj till Silta, bodde de i f.d. soldattorpet på Gunnar, dit de flyttade tillbaka år 1900. År 1914 flyttade de igen till Silta, där de kom att tillbringa sina sista år.

Brita Kaisa \textdied 12.11.1936  ---  Otto \textdied 29.06.1939


\jhoccupant{Bur}{Matts \& Kajsa}{1865--\allowbreak 1895}
Backstugusittare Matts Johansson Bur, \textborn 20.02.1831 i Purmo, gift med Kajsa Johansdotter, \textborn 03.04.1834 i Jeppo.
\begin{jhchildren}
  \item \jhperson{\jhname[Johannes]{Bur, Johannes}}{1834}{1860}
  \item \jhperson{\jhname[Brita Caisa]{Bur, Brita Caisa}}{10.07.1858}{}
  \item \jhperson{\jhname[Maria]{Bur, Maria}}{27.09.1861}{}
  \item \jhperson{\jhname[Sanna Lisa]{Bur, Sanna Lisa}}{01.06.1864}{}
  \item \jhperson{\jhname[Matts]{Bur, Matts}}{31.10.1867}{14.05.1908}
\end{jhchildren}

Kajsa \textdied 25.05.1897  ---  Matts \textdied 11.09.1901


\jhoccupant{Andersson}{Johan \& Brita}{1844--\allowbreak 1857}
Torparen Johan Erik Andersson,  \textborn 12.04.1812, gift med Brita Carlsdr, \textborn 11.03.1821.
\begin{jhchildren}
  \item \jhperson{\jhname[Maria Lovisa]{Andersson, Maria Lovisa}}{06.11.1847}{}
  \item \jhperson{\jhname[Sophia]{Andersson, Sophia}}{31.01.1853}{1856}
  \item \jhperson{\jhname[Anna]{Andersson, Anna}}{13.09.1854}{1856}
\end{jhchildren}

Johan dog 1856. Brita gifte om sig med Johan Hannula (Mietala 114). Torpet fanns på Silta, men det är osäkert om det var just detta torp. I ett torpkontrakt från 11 december 1844 nämns att torpet är beläget på Silta, att torparen ska göra ett dagsverke i månaden fr.o.m 1 nov 1845. En förutsättning för köpet är att köparen, dvs Johan Erik Andersson, inte får sälja torpet utan markägarnas eller doktorinnan Topelius' bifall. Markägarna var Johan Samuelsson och Jonas Mattsson Mjetala. Själva torpet ägdes av Topelius.



\jhhouse{Granlund}{9:19}{Mietala}{14}{108}


\jhoccupant{Ekström}{Edit}{1957--\allowbreak 1969}
Edit Ekström, \textborn 07.08.1895 på Mietala, bodde tillsammans med sin mor Sofia i gård nr 116, men flyttade till Nykarleby 1951. Till Silta flyttade hon efter 1957 och år 1969 till en pensionärsbostad i Silvast. Gunnar och Rune Forsbacka rev gården 1978 eller 1979.

Edit \textdied 26.02.1986


\jhhousepic{Mietala108-Granlund.jpg}{Rosa Granlund framför sitt hus}

\jhoccupant{Granlund}{Rosa \& Jonas}{1911--\allowbreak 1957}
Rosa Östman, \textborn 22.12.1880 på Lassila. Rosa gifte sig år 1902 i USA med Jonas Storkamp, senare Granlund,  \textborn 03.07.1874 i Purmo.
\begin{jhchildren}
  \item \jhperson{\jhname[Ingrid Elvira]{Granlund, Ingrid Elvira}}{1903}{}, flyttade till Vörå 1926
  \item \jhperson{\jhname[Johannes Arthur]{Granlund, Johannes Arthur}}{1904}{}
  \item \jhperson{\jhname[Theodor Valdemar]{Granlund, Theodor Valdemar}}{1906}{1919 i Jeppo}
  \item \jhperson{\jhname[Arvid Reinhold]{Granlund, Arvid Reinhold}}{1907}{1992 i Oravais}
  \item \jhperson{\jhname[Ellen Katarina]{Granlund, Ellen Katarina}}{1912}{}, flyttade t. Jakobstad 1931, senare Sverige
\end{jhchildren}

Efter att torparlagen trätt i kraft löste Rosa in det 5 kappland stora backstuguområdet 22.10.1920. Ägare var änkan Sofia Mietala.

Jonas \textdied 21.11.1919 i Jeppo  --- 	Rosa \textdied 08.07.1957


\jhoccupant{Backlund}{Matts \& Ida}{1899--\allowbreak 1911}
Matts Alfred Backlund, \textborn 20.12.1878 i Jeppo, vigd 08.10.1899 med Ida Böös, \textborn 14.09.1876 i Jeppo. Ida var dotter till Anders Jakobsson Böös och hustrun Anna Lovisa Isaksdotter. Som bröllopsgåva byggde Idas far åt familjen en liten stuga på Silta. Troligen är det samma stuga som Rosa Granlund senare bodde i. Matts var också skriven under Ekströms hemman. Matts emigrerade 1903 till Amerika. Där sysslade han bl.a med laxfiske. När han sparat ihop tillräckligt, köpte han en liten farm i Winlock, Washington, och hustrun reste dit med barnet 03.07.1911.
\begin{jhchildren}
  \item \jhperson{\jhname[Agnes Astrid]{Backlund, Agnes Astrid}}{29.08.1900 i Jeppo}{07.12.1904}
  \item \jhperson{\jhname[Sigrid Irene]{Backlund, Sigrid Irene}}{08.12.1901 i Jeppo}{}
  \item \jhperson{\jhname[Johannes Alfred]{Backlund, Johannes Alfred}}{29.04.1903 i Jeppo}{03.12.1904}
  \item \jhperson{\jhname[Elsie]{Backlund, Elsie}}{24.09.1912 i Winlock}{}
  \item \jhperson{\jhname[Edna]{Backlund, Edna}}{05.12.1915 i Winlock}{}
\end{jhchildren}



\jhhouse{Norrgård}{9:78}{Mietala}{14}{5}


\jhhousepic{220-05798.jpg}{Bo och Marita Norrgård}

\jhoccupant{Norrgård}{Bo \& Marita}{1979--}
Bo Norrgård, \textborn 19.03.1945  på Mietala hemman, gift med Marita Nygård, \textborn 18.09.1945 i Vörå.
\begin{jhchildren}
  \item \jhperson{\jhname[Kenneth]{Norrgård, Kenneth}}{09.06.1966}{}, företagare, bilförsäljare (se Mietala 6)
  \item \jhperson{\jhname[Yvonne]{Norrgård, Yvonne}}{15.03.1968}{}, organisationssekr för Föreningen Norden
  \item \jhperson{\jhname[Ronny]{Norrgård, Ronny}}{27.06.1977}{}, på KWH Mirka, huvudförtroendeman
\end{jhchildren}

Bo och Marita är pensionärer. Bo har i sin ungdom intresserat sig för tyngdlyftning och annan styrketräning och började träna tillsammans med Asko Linjanmäki i Rukkuskroken (se karta 5, nr 85) medan Asko bodde där med sin mamma. Senare flyttades träningsplatsen till Jungar folkskolas matsal.

Bo arbetade som förman på KWH Tidströms Farm på Keppo, Marita på Mirka. Bostadsbyggnaden uppfördes på samma tomt som Bos hemgård. Huset är ett elementhus från Oravais Hus och byggdes år 1979. Se också karta 14, gård nr 3.



\jhhouse{Villa Camilla}{9:87}{Mietala}{14}{6, 6a}


\jhhousepic{221-05799.jpg}{Kenneth och Elisa Norrgård}

\jhoccupant{Norrgård}{Kenneth \& Elisa}{1994--}
Kenneth Norrgård, \textborn 09.06.1966 på Mietala, gift med Elisa Korkiakangas, \textborn 15.06.1965 i Toholampi.
\begin{jhchildren}
  \item \jhperson{\jhname[Ann-Sofie]{Norrgård, Ann-Sofie}}{01.04.1991}{}, student, merkonom
  \item \jhperson{\jhname[Dan Otto]{Norrgård, Dan Otto}}{29.01.1993}{}, idrottsledare
  \item \jhperson{\jhname[Alicia]{Norrgård, Alicia}}{12.10.1995}{}, stud. utv.psyk., spelar i Jepokryddona
  \item \jhperson{\jhname[Cecilia]{Norrgård, Cecilia}}{16.03.1999}{}, skolelev, spelar fiol i Jepokryddona
  \item \jhperson{\jhname[Celina]{Norrgård, Celina}}{21.02.2001}{}, skolelev, spelar trombone i JUO
\end{jhchildren}

Kenneth har varit egen företagare, försäljare. Han arbetar nu som bilförsäljare på Nystedt bilaffär i Jakobstad. Han är examinerad år 1984 från husbyggnadslinjen vid Yrkesskolan i Jakobstad, numera Optima. Elisa arbetar som städare i Jeppo-Pensala skola. Elisa blev student 1984 vid Toholammen lukio och dimitterades år 1987 från Kokkolan Ammattikoulun parturilinja.

Egnahemshuset byggdes 1993--\allowbreak 1994 genom egen arbetsinsats samt med elementpaket från Myresjöhus i Finland.



\jhhouse{Häggstrand}{9:110}{Mietala}{14}{7}


\jhhousepic{223-05802.jpg}{Tommy Häggstrand}

\jhoccupant{Häggstrand}{Tommy \& Maria}{2015--}
Tommy Häggstrand,  \textborn 07.11.1983 på Mietala, gifte sig 2016 med Maria Eugenia Arellano från Mexiko. Tommy flyttade in i det nybyggda huset år 2015. Potatisodlingen sysselsätter makarna.

Palofors kvarnstuga fanns på den plats som Tommy byggt sitt hus på. Se närmare Mietala, gård nr 107.


\jhhouse{Paloforsen}{9}{Mietala}{14}{107}

\jhnooccupant{}

\jhbold{Kvarnen}

Kvarnen vid Paloforsen var en husbehovskvarn, som ägdes av bönderna på 	Mietala. Ännu år 1893 blev den pålagd ett skattöre. En kvarnstuga fanns kvar ännu i början på 1900-talet. Kvarnstugan finns dokumenterad på ett gammalt foto från 1909-10.

I en dagbok, som Isak Elenius, \textborn 1867, skrev,  står bl.a : ``Betalades till Paloforsqvarn 1 fm 92 pi den 20 November 1889. Malades vid Palofors qvarn 3 tunnor råg den 23  Nov 1889 då 3 kuggar gick av kugghjulet 	...byggt nytt vattenhjul i Palofors bolagsqvarn af bolagsmännen sjelf från 9 – 12 November 1891''. Enligt uppgifter från Historiken över Jeppo så torde kvarnen vid Paloforsen raserats före sekelskiftet.


\jhoccupant{Martois}{Elias \& Brita}{1857--\allowbreak 1861}
Mjölnaren Elias Johansson Martois, \textborn 21.01.1817 i Vörå, hustru Brita Isaksdotter, \textborn 02.12.1814 i Vörå. Mjölnaren är skriven på Mietala 1857–1861. Det är troligt att han var mjölnare vid denna kvarn. Familjen kom från Silvast och flyttade senare till Mjölnars.
\begin{jhchildren}
  \item \jhperson{\jhname[Johan]{Martois, Johan}}{1836}{}
  \item \jhperson{\jhname[Isak]{Martois, Isak}}{1838}{}
  \item \jhperson{\jhname[Mårten]{Martois, Mårten}}{1842 (i Vörå)}{}
  \item \jhperson{\jhname[Anders]{Martois, Anders}}{1845}{}
  \item \jhperson{\jhname[Greta]{Martois, Greta}}{1849}{}
  \item \jhperson{\jhname[Elias]{Martois, Elias}}{1851}{}
  \item \jhperson{\jhname[Brita Lisa]{Martois, Brita Lisa}}{1854 (i Överjeppo)}{}
  \item \jhperson{\jhname[Anna]{Martois, Anna}}{1857 på Mietala}{}
\end{jhchildren}


\jhbold{Siltabäcken och kvarnen}

Vid Mietala har också funnits en kvarn vid Siltabäcken, mellan landsvägen och älven. Enligt historiken torde den ha varit igång ännu i medlet av 1850-talet. Eventuellt har den dock funnits ännu på 1880-talet. Då Simon Eriksson (Sandbacka) år 1881 köpte hemmanet ingick kvarnställe och kvarn  i köpet. Det är troligt att det är just kvarnen vid Siltabäcken.



\jhhouse{Låglandet}{9:96}{Mietala}{14}{8}


\jhhousepic{224-05801.jpg}{Leif och Carola Elenius}

\jhoccupant{Elenius}{Leif \& Carola}{1991--}
Leif Elenius, \textborn 29.01.1961 i Jeppo, gifte sig år 1990 med Carola Wisén, \textborn 1962 i Kronoby. Leif är diplomingenjör och arbetar på Schur Flexibles Finland. Carola är filosofie magister och arbetar som lärare i Vasa.

\begin{jhchildren}
  \item \jhperson{\jhname[Susanna]{Elenius, Susanna}}{1991}{}
  \item \jhperson{\jhname[Sara]{Elenius, Sara}}{1992}{}
  \item \jhperson{\jhname[Theo]{Elenius, Theo}}{2001}{}
\end{jhchildren}

Huset används som fritidsbostad. Det byggdes med talkoarbete 1991--\allowbreak 1992. Huset byggdes av hyvlad stock från Finnwood.



\jhhouse{Häggstrand}{9:46}{Mietala}{14}{9, 9a-c}


\jhhousepic{225-05803.jpg}{Dage och Britt-Marie Häggstrand}

\jhoccupant{Häggstrand}{Dage \& Britt-Marie}{1976--}
Dage Häggstrand, \textborn 22.03.1953 i Jeppo, gift 1979 med Britt-Marie Engman, \textborn 30.05.1957 från Korsholm.
\begin{jhchildren}
  \item \jhperson{\jhname[Tommy]{Häggstrand, Tommy}}{07.11.1983 (Mietala 7)}{}
  \item \jhperson{\jhname[Mikaela]{Häggstrand, Mikaela}}{1986}{}
\end{jhchildren}

År 1976 övertog Dage sina föräldrars lägenhet och huvudnäringen var då mjölkproduktion. Idag har man inga djur utan potatisodlingen sysselsätter både Dage och Britt-Marie samt sonen Tommy.

Bostadsbyggnaden byggdes halvfärdigt (nr 9H) på andra sidan vägen år 1954, men flyttades 1955 till sin nuvarande plats. Det var Dages föräldrar som uppförde gården. År 1982 gjorde Dage en tillbyggnad, som möjliggjorde skilda lägenheter för föräldrarna samt Dages egen familj. Ekonomibyggnaden (f.d fähuset) uppfördes 1948 av Dages pappa Valfrid. Dage har senare uppfört en potatishall (9a).


\jhoccupant{Häggstrand}{Valfrid \& Anita}{1955--\allowbreak 1976}
Valfrid Häggstrand, \textborn 18.10.1914 i Jeppo, gift med Anita Lindholm, \textborn 12.07.1920 i Lappfjärd.

Barn: \jhbold{Dage}, \textborn 30.05.1953

Valfrid övertog lägenheten år 1945 efter modern, men nuvarande gård byggde han 1954--\allowbreak 1955. Efter att ha tillbringat en stor del av sin ungdomstid i krig, fortsatte Valfrid med jordbruket då det blev fred. Mjölkproduktion var huvudnäringen på lägenheten. Anita hade före giftermålet varit kontorist, bl.a på Jeppo-Oravais Handelslag i Jeppo. Hon var söndagsskollärare i många år. Till Valfrids fritidsintressen hörde bl.a. jakt.

Valfrid \textdied 10.07.1999  ---  Anita \textdied 15.03.2014


\jholdhouse{Gamla gården Häggstrand}{9:}{Mietala}{14}{109}

\jhoccupant{Häggstrand}{Valfrid \& Anita}{1945--\allowbreak 1955}
Valfrid Häggstrand, \textborn 18.10.1914 i Jeppo, vigd 1950 med Anita Lindholm, \textborn 12.07.1920 i Lappfjärd. Valfrid övertog lägenheten år 1945, byggde ny gård 1954--\allowbreak 1955 (nr 9H -> 9).\jhvspace{}


\jhoccupant{Häggstrand}{Jakob \& Katarina}{1896--\allowbreak 1945}
Jakob Mietala, senare Hägglund och Häggstrand, \textborn 21.01.1876, gift med Katarina Bärs, \textborn 05.02.1876 på Bärs i Jeppo.
\begin{jhchildren}
  \item \jhperson{\jhname[Ellen]{Häggstrand, Ellen}}{1902}{1905}
  \item \jhperson{\jhname[Saima]{Häggstrand, Saima}}{1904}{}, till Sverige
  \item \jhperson{\jhname[Ellen]{Häggstrand, Ellen}}{1906}{1938}
  \item \jhperson{\jhname[Helga]{Häggstrand, Helga}}{1907}{}, gift Back i Pensala
  \item \jhperson{\jhname[Frida]{Häggstrand, Frida}}{1909}{}, till Sverige
  \item \jhperson{\jhname[Linnea]{Häggstrand, Linnea}}{1912}{}, gift Jungar (Jungar 8)
  \item \jhperson{\jhbold{\jhname[Valfrid]{Häggstrand, Valfrid}}}{18.10.1914}{}
  \item \jhperson{\jhname[Runar]{Häggstrand, Runar}}{1916}{1953}
\end{jhchildren}

År 1903 överlämnade Jakobs föräldrar sin hemmansdel på Mietala till Jakob och Katarina. År 1917 emigrerade Jakob till USA, där han dog ett år senare, \textdied 17.10.1918, i Spanska sjukan. Katarina \textdied 13.07.1961.


\jhoccupant{Mietala}{Jakob \& Anna}{1875--\allowbreak 1896}
Johan Jakob Mietala, senare Forsgård, \textborn 24.10.1853 på Mietala, gift med Anna Isaksdotter Elenius, \textborn 23.12.1850.
\begin{jhchildren}
  \item \jhperson{\jhbold{\jhname[Jakob]{Mietala, Jakob}}}{21.01.1876}{}
  \item \jhperson{\jhname[Anna Lovisa]{Mietala, Anna Lovisa}}{01.08.1878}{}
  \item \jhperson{\jhname[Katrina]{Mietala, Katrina}}{29.01.1882}{}
  \item \jhperson{\jhname[Hilda Maria]{Mietala, Hilda Maria}}{09.07.1884}{}
  \item \jhperson{\jhname[Isak Joel]{Mietala, Isak Joel}}{05.02.1887}{}
  \item \jhperson{\jhname[Ida Johanna]{Mietala, Ida Johanna}}{23.06.1889}{}
  \item \jhperson{\jhname[Anders Gustav]{Mietala, Anders Gustav}}{13.01.1892}{}, (Forsgård, Mietala 10)
\end{jhchildren}

År 1879 fick makarna genom testamente av hans far och styvmor en 1/9 mantals hemmansdel. År 1896 köpte makarna två hemmansdelar om inalles 1/9 mantal, som en gång tillhört Jakobs farfar till år 1858. Samma år överlämnade de sitt gamla hemman åt äldsta sonen Jakob.

Anna \textdied 08.08.1925  ---  Jakob \textdied 18.11.1938



\jhhouse{Gunnar småskola, inkl. lärare}{9}{Mietala}{14}{111}


\jhhousepic{Mietala111-smaskola.jpg}{Gunnar småskola}

\jhoccupant{Småskolan}{Gunnar}{1919--\allowbreak 1966}
En småskola, i kyrklig regi, hölls i olika bondgårdar. Från 1905 och en tid framöver i Lennart Jungars framstuga. Senare också i olika bondgårdar fram till 1919 då ett småskolhus, genom ett privat initiativ av Johan Jungar, byggdes vid Mietala. Jeppo kommun köpte tomten av Jakob Forsgård. Vid läropliktens införande år 1921 fanns 19 barn i klasserna I och II.


\jhbold{Lärare} vid lägre folkskolan:

\jhoccupant{Granfors}{Saga}{1956--\allowbreak 1966}
Småskollärarinnan Saga Smedjebacka, \textborn 12.02.1910 i Terjärv gifte sig 04.10.1942 med Anders Rudolf Granfors, \textborn 06.03.1907 på Dalabacka i Åvist. Saga var efter centraliseringen av skolorna verksam som lärare i Kristinestad fram till sin pensionering.
Barn: Peter, \textborn 05.05.1950, ADB-planerare i Esbo


\jhoccupant{Åkerholm}{Ingeborg}{1922--\allowbreak 1956}
Småskollärarinnan Ingeborg Backlund, \textborn 29.01.1897 i Jeppo, gift med författaren Johannes Åkerholm, \textborn 16.04.1901 i Maxmo. Se karta 3, Romar nr 342.
\begin{jhchildren}
  \item \jhperson{\jhname[Ole Johannes]{Åkerholm, Ole Johannes}}{13.09.1932}{05.12.1934}
  \item \jhperson{\jhname[Ole Johannes]{Åkerholm, Ole Johannes}}{28.10.1935}{20.10.1936}
  \item \jhperson{\jhname[Solbritt Ingeborg]{Åkerholm, Solbritt Ingeborg}}{09.02.1938}{}
\end{jhchildren}


\jhoccupant{Blomqvist}{Katarina}{1918--\allowbreak 1922}
Katarina Storsved, \textborn 1891, gift med Axel Blomqvist, \textborn 1895 i Pensala.
\begin{jhchildren}
  \item \jhperson{\jhname[Gösta]{Blomqvist, Gösta}}{1920}{}
  \item \jhperson{\jhname[Eva]{Blomqvist, Eva}}{1922}{}
  \item \jhperson{\jhname[Gunhild]{Blomqvist, Gunhild}}{1927}{}
  \item \jhperson{\jhname[Gretel]{Blomqvist, Gretel}}{1929}{}
\end{jhchildren}



\jhhouse{Forsgård}{9:92}{Mietala}{14}{10, 10a, 110}


\jhhousepic{226-05804.jpg}{Lars Forsgård}

\jhoccupant{Forsgård}{Lars}{}
Lars Forsgård, \textborn 08.07.1936 i Jeppo, bor på hemgården i Mietala. Den odlade jorden, som hört till hemgården, har sålts till brorssönerna Per-Erik och Kjell Forsgård. Till lägenheten hör idag 9 ha skog. Byggnaderna, både bostads- och ekonomibyggnaden, är uppförda år 1926 av Lars' far, Anders Forsgård.


\jhoccupant{Forsgård}{Anders \& Helga}{1927--}
Anders Forsgård, \textborn 13.01.1892, gift med Helga Asplund, \textborn 22.09.1904 från Pensala.
\begin{jhchildren}
  \item \jhperson{\jhname[Kurt]{Forsgård, Kurt}}{19.11.1928}{}, (Mietala 11)
  \item \jhperson{\jhname[Paul]{Forsgård, Paul}}{26.06.1930}{}, f.d kantor i Nykarleby
  \item \jhperson{\jhname[Yngve]{Forsgård, Yngve}}{06.01.1933}{}, (Mietala  1)
  \item \jhperson{\jhbold{\jhname[Lars]{Forsgård, Lars}}}{08.07.1936}{}
  \item \jhperson{\jhname[Viking]{Forsgård, Viking}}{28.05.1943}{11.11.1959}
\end{jhchildren}

Av Anders föräldrar (se karta 14, nr 109) fick makarna överta den 1/9 mantal stora lägenheten på Mietala hemman. Den omfattade 74 ha, varav 36 ha odlad jord. Efter en brand 1925, då uthusbyggnaderna förstördes till grunden och  bostadsbyggnaden  brandskadades svårt, byggde Anders år 1926 upp dem igen.

Anders \textdied 16.10.1971  ---  Helga \textdied 14.06.1985


\jhpic{Mietala110-Forsgard.jpg}{Branden på Forsgård år 1925.}



\jhhouse{Forsgård}{9:108}{Mietala}{14}{11, 11a-d}


\jhhousepic{227-05805.jpg}{Per-Erik Forsgård, Mietas Js}

\jhoccupant{Mietas}{Js}{1987--}
Per-Erik Forsgård, \textborn 15.02.1960 i Jeppo, gifte sig 05.03.1988 med Åsa Höglund, \textborn 27.09.1965 i Jakobstad. Åsa var barnskötare och anställd vid Jeppo daghem. Hon dog 11.11.2012. Ägare blev då Mietas Jordbrukssammanslutning, dvs Per-Erik äger hälften, sönerna andra hälften.
\begin{jhchildren}
  \item \jhperson{\jhname[Simon]{Mietas, Simon}}{14.03.1989}{}
  \item \jhperson{\jhname[Sebastian]{Mietas, Sebastian}}{30.04.1993}{}
\end{jhchildren}

Per-Erik övertog jordbruket av sina föräldrar år 1987. Lägenheten består idag av 65 ha odlad jord och 66 ha skog. Huvudnäringen idag är kalvuppfödning och potatisodling.

Bostadsbyggnaden samt ekonomibyggnaden är uppförda 1955 och 1956 av Per-Eriks far Kurt. Huset renoverades senare av Per-Erik. Ett potatislager uppfördes 1971 samt tillbyggnad 1997, en maskinhall år 1986, samt byggnad för kall lösdrift för tjurar 1993.


\jhoccupant{Forsgård}{Kurt \& Ragni}{1955--\allowbreak 1988}
Kurt Forsgård, \textborn 19.11.1928 på Mietala, gift med Ragni Bäckstrand, \textborn 01.10.1928 på Romar hemman.
\begin{jhchildren}
  \item \jhperson{\jhname[Stig]{Forsgård, Stig}}{01.11.1956}{}, (Grötas nr 5)
  \item \jhperson{\jhbold{\jhname[Per-Erik]{Forsgård, Per-Erik}}}{15.02.1960}{}
\end{jhchildren}

Kurt och Ragni övertog en tredjedel av hans föräldrars lägenhet på Mietala och byggde åt sig år 1955 eget bostadshus och behövliga ekonomibyggnader. Huvudnäringen på lägenheten var i början mjölkproduktion. Kurt har varit ledamot i skogsnämnden och nämndeman vid jorddomstol. Ragni var en tid ordförande för Gunnar Marthakrets.

Kurt, Ragni och Stig flyttade till radhuslägenhet vid Åkervägen 7 i Silvast i samband med generationsväxling. Ragni \textdied 02.12.2014.



\jhhouse{Herrgård}{9:68}{Mietala}{14}{12, 12a-b}


\jhhousepic{228-05806.jpg}{Lars-Erik och Marita Elenius}

\jhoccupant{Elenius}{Lars-Erik \& Marita}{1980--}
Lars-Erik, \textborn 23.08.1951 på Mietala, gift år 1995 med Marita Nygård, \textborn 13.06.1953 i Jakobstad.
\begin{jhchildren}
  \item Enina, \textborn 1993, studerar vid Åbo Akademi, Vasa
\end{jhchildren}

Lars-Erik har gått i Korsholms jordbruksskola. Han övertog Herrgård	lägenhet år 1980 genom generationsskifte. Lägenheten består idag av 40 ha åker och 70 ha skog. På gården har bedrivits mjölk- och köttproduktion i många år. Sedan 2009 endast växtodling. Lars-Erik har varit styrelsemedlem i Jeppo ungdomsförening och i idrottsföreningen Minken, ledamot i lantbruksnämnden samt ordförande i ÖSP:s Jeppo-avdelning.

Marita har efter mellanskola arbetat på elfirma, på polisinrättningen i Jakobstad och senast på alarmcentralen i Jakobstad.


\jhoccupant{Elenius}{Martin \& Aldi}{1964--\allowbreak 1980}
Martin, \textborn 18.07.1916 på Mietala, gift 1950 med Aldi  Mattus, \textborn 03.06.1921 i Kimo, Oravais. Martin deltog i både vinter- och fortsättningskriget. Aldi var lotta i Oravais under kriget.
\begin{jhchildren}
  \item \jhperson{\jhbold{\jhname[Lars-Erik]{Elenius, Lars-Erik}}}{23.08.1951}{}, bonde
  \item \jhperson{\jhname[Birgitta]{Elenius, Birgitta}}{16.03.1953}{}, biokemist
  \item \jhperson{\jhname[Leif]{Elenius, Leif}}{29.01.1961}{}, diplomingenjör
\end{jhchildren}

År 1950 byggde Martin ny ekonomibyggnad för mjölkkor och många gamla byggnader revs på tomten. År 1964 byggdes nuvarande bostad och den gamla revs år 1966.

Martin tillhörde direktionen för Gunnar folkskola, samt förmyndarnämnden, prövningsnämnden samt idrottsföreningens styrelse. Han deltog också själv aktivt inom idrotten, främst i kastgrenarna. Han var också många år med i idrottsföreningens gymnastiklag. Aldi var aktiv inom Marthaföreningen, bl.a som ordförande för Gunnar Marthakrets.

Aldi \textdied 31.01.2000  ---  Martin \textdied 01.02.2007


\jholdhouse{Gamla gården Herrgård}{9:}{Mietala}{14}{112}


\jhhousepic{Mietala112-EleniusM.jpg}{Martin Elenius' gamla gård}

\jhoccupant{Elenius}{Martin}{1944--\allowbreak 1964}
Martin Elenius, \textborn 18.07.1916,  bodde åren 1950--\allowbreak 1964 med sin familj i gården, som enligt uppgift var byggd 1850. Det torde således varit Jakob Sandberg, som byggt gården. Gården var uppförd i trä med tegeltak. Se närmare om Martins familj under gård nr 12, ovan.

\jhoccupant{Elenius}{Erik \& Sanna}{1903--\allowbreak 1944}
Erik Johansson, senare med efternamnet Elenius,  \textborn 07.11.1870 på Jungarå hemman, gift med Sanna Mattsdotter,  \textborn 29.12.1872 på Bärs. År 1903 köpte Erik och Sanna Elenius Herrgård lägenhet på Mietala av Isak Sandberg Mietala. Erik och Sanna fick 11 barn, endast 3 av dem nådde vuxen ålder.
\begin{jhchildren}
  \item \jhperson{\jhname[Erik]{Elenius, Erik}}{31.01.1898}{1905}
  \item \jhperson{\jhname[Elna]{Elenius, Elna}}{04.02.1900}{1900}
  \item \jhperson{\jhname[Henrik]{Elenius, Henrik}}{23.01.1901}{1905}
  \item \jhperson{\jhname[Ellen]{Elenius, Ellen}}{06.02.1903}{1905}
  \item \jhperson{\jhname[Ellen]{Elenius, Ellen}}{09.05.1905}{}, gift Romar
  \item \jhperson{\jhname[Lea]{Elenius, Lea}}{01.07.1907}{1919}
  \item \jhperson{\jhname[Uno]{Elenius, Uno}}{19.07.1909}{1909}
  \item \jhperson{\jhname[Agda]{Elenius, Agda}}{21.10.1910}{1914}
  \item \jhperson{\jhname[Elna]{Elenius, Elna}}{14.02.1913}{1913}
  \item \jhperson{\jhbold{\jhname[Martin]{Elenius, Martin}}}{18.07.1916}{}
  \item \jhperson{\jhname[Arne]{Elenius, Arne}}{16.03.1919}{26.06.1944}, stupade i Lavola
\end{jhchildren}
Erik Elenius var före sitt giftermål 3 år soldat i den finsk-ryska militären i Vasa och Krasnoje i St.Petersburg. Erik hörde till den tredje finska skarpskyttebataljonen. Han har berättat att det tog en hel dag att marschera genom St. Petersburg (mer än 30 km), där han även fick se självaste kejsar Alexander III.

År 1940 köpte Erik och Sanna lägenheten Södergård på Mietala av Joel Elenius åt sin son Arne. Då Arne stupade i kriget 1944 kom brodern Martin Elenius att äga både Herrgård och Södergård lägenheter. Jfr Mietala, nr 118.

Sanna \textdied 09.03.1948  ---  Erik \textdied 26.04.1959


\jhoccupant{Sandberg}{Isak \& Sanna}{1864--\allowbreak 1903}
Isak Jakobsson Sandberg, \textborn 27.06.1846 på Mietala, vigd 18.04.1869 med Sanna Gabrielsdotter, \textborn 11.10.1844 på Bärs hemman.
\begin{jhchildren}
  \item \jhperson{\jhname[Jakob]{Sandberg, Jakob}}{11.03.1870}{1870}
  \item \jhperson{\jhname[Isak]{Sandberg, Isak}}{18.02.1871}{1878}
  \item \jhperson{\jhname[Maria]{Sandberg, Maria}}{07.02.1873}{1956}
  \item \jhperson{\jhname[Johannes]{Sandberg, Johannes}}{28.06.1875}{1917/1918}
  \item \jhperson{\jhname[Hilma]{Sandberg, Hilma}}{06.11.1878}{1965}
  \item \jhperson{\jhname[Katarina]{Sandberg, Katarina}}{01.05.1881}{1965}
  \item \jhperson{\jhname[Sanna]{Sandberg, Sanna}}{01.05.1881}{1967}
\end{jhchildren}

Isak övertog lägenheten om 1/12 mantal av sin mor år 1864. Han blev svårt misshandlad den 19 aug. 1906, då han var på väg hem från ett bröllop i byn. Han avled av skadorna 24.03.1907. Sanna dog 04.03.1903.


\jhoccupant{Sandberg}{Jakob \& Caisa}{1844--\allowbreak 1864}
Jakob Jakobsson Sandberg, \textborn 17.12.1803 på Keppo, gift med Caisa Jakobsdotter, \textborn 13.04.1819 på Grötas hemman.
\begin{jhchildren}
  \item \jhperson{\jhname[Johan Jakob]{Sandberg, Johan Jakob}}{04.03.1845}{1863}
  \item \jhperson{\jhbold{\jhname[Isak]{Sandberg, Isak}}}{27.06.1846}{}
  \item \jhperson{\jhname[Susanna]{Sandberg, Susanna}}{02.07.1848}{1848}
  \item \jhperson{\jhname[Mathias]{Sandberg, Mathias}}{09.07.1852}{1935}, Mietala 105
  \item \jhperson{\jhname[Brita Caisa]{Sandberg, Brita Caisa}}{1854}{1856}
  \item \jhperson{\jhname[Maria]{Sandberg, Maria}}{08.04.1857}{1891}
\end{jhchildren}

Jakob var några år sjöman. År 1844 köpte han Herrgård lägenhet på Mietala av Matts Johansson. Några månader senare gifte han sig med Caisa. Jakobs mor, Brita Sandberg, bodde sina sista år hos Jakob och hans familj.

Jakob \textdied 17.03.1857  ---  Caisa \textdied 14.07.1867


\jhoccupant{Johansson}{Mats \& Sanna}{- 1845}
Mats Johansson Mietala, \textborn 1819, hustrun Sanna Lisa Jöransdr, \textborn 02.09.1821 samt dottern Brita Maja,  \textborn 03.09.1844 flyttade till Tollikko i december 1845. Detta efter att ha sålt sitt hemman på Mietala. På Tollikko är de sedan skrivna som torpare.

På lägenheten fanns i början på 1800-talet mycket litet odlad jord och Mietala var förr i världen känd för fattigdom och kom därför att kallas för ``Fati-Mietas''. Stora familjer och litet odlad jord var väl en orsak. Det har berättats att på hemmanet, nångång på 1700--\allowbreak 1800-talet, funnits en trollkarl, som sades besitta stora övernaturliga krafter.



\jhhouse{Sandbacka}{9:103}{Mietala}{14}{13, 13a}


\jhhousepic{229-05807.jpg}{Sandells gård}

\jhoccupant{Sandell}{Peter}{1977--}
Peter Sandell, \textborn 17.11.1957 på Mietala är nuvarande ägare till gården, där också Peters mor Marita Sandell bor. Åkrar och skog tillhörande lägenheten är idag sålda. Före det drev Peter tillsammans med sin mor ett kreaturlöst jordbruk med potatisodling. Hemmansarealen var 24 ha åker och 32 ha skog. Bostadshuset är byggt av stock år 1935 av Maritas farfar Edvard. År 1957 byggdes huset till för att inrymma två lägenheter.


\jhoccupant{Sandell}{Marita \& Harry}{1957--\allowbreak 1977}
Marita  Sandbacka, \textborn 17.07.1938 på Mietala, vigd 1957 med Harry Sandell, \textborn 10.04.1934 på Måtar. Skilsmässa 1977.
\begin{jhchildren}
  \item \jhperson{\jhname[Peter]{Sandell, Peter}}{17.11.1957}{}
  \item \jhperson{\jhname[Susann]{Sandell, Susann}}{28.06.1965}{}, gift Forslund, bor i Norge
\end{jhchildren}
Marita och Harry övertog år 1957 hennes föräldrars lägenhet på Mietala. De var bland de första som började odla potatis i större skala i Jeppo.

ÖNSKEMÅL OM ATT INTE KOMMA MED I BOKEN FRAMFÖRT AV PETER SANDELL I SÅFALL KUNDE  DET SOM SKRIVITS OVAN LÄMNAS BORT


\jhoccupant{Sandbacka}{Edvin \& Hjördis}{1941--\allowbreak 1957}
Edvin Sandbacka, \textborn 20.05.1910 på Mietala, gifte sig med Hjördis Sandberg, \textborn 31.12.1912 på Finskas.
\begin{jhchildren}
  \item \jhperson{\jhname[Viva]{Sandbacka, Viva}}{21.06.1937}{31.01.1978}, g. Lönnvik
  \item \jhperson{\jhname[Marita]{Sandbacka, Marita}}{17.07.1938}{}
  \item \jhperson{\jhname[Karin]{Sandbacka, Karin}}{18.02.1945}{}, g. Kalijärvi
\end{jhchildren}

År 1941 övertog Edvin och Hjördis ena halvan av hans föräldrars lägenhet och 1945 köpte de resten av den.

Edvin \textdied 05.10.1955  ---  Hjördis \textdied  16.07.1983


\jholdhouse{Gamla gården Sandbacka}{9:}{Mietala}{14}{113 a}

\jhoccupant{Sandbacka}{Edvard \& Hilda}{1894--\allowbreak 1941}
Jakob Edvard Sandbacka, \textborn 06.01.1878 på Mietala, gift med Hilda Maria Andersdotter, \textborn 02.10.1876 på Grötas.
\begin{jhchildren}
  \item \jhperson{\jhname[Hilda Irene Katrina]{Sandbacka, Hilda Irene Katrina}}{10.02.1900}{20.01.1905}
  \item \jhperson{\jhname[Jakob Leonard]{Sandbacka, Jakob Leonard}}{27.07.1903}{03.02.1905}
  \item \jhperson{\jhname[Leander]{Sandbacka, Leander}}{27.11.1903}{22.09.1904}
  \item \jhperson{\jhname[Jenny Maria]{Sandbacka, Jenny Maria}}{20.08.1905}{}, gift Eklund
  \item \jhperson{\jhname[Elna Katrina]{Sandbacka, Elna Katrina}}{01.12.1907}{}, gift Grönlund
  \item \jhperson{\jhbold{\jhname[Johannes Edvin]{Sandbacka, Johannes Edvin}}}{20.05.1910}{}
  \item \jhperson{\jhname[Hilda Irene]{Sandbacka, Hilda Irene}}{01.11.1912}{}, gift Jungell
  \item \jhperson{\jhname[Jakob Runar]{Sandbacka, Jakob Runar}}{12.10.1916}{13.02.1940}, stupade vid Leipäsuo
  \item \jhperson{\jhname[Astrid Johanna]{Sandbacka, Astrid Johanna}}{24.01.1920}{}, gift Elenius
\end{jhchildren}
Gården, som eventuellt byggts av Simon Eriksson, var en rödmålad gård med långsidan mot vägen.

Hilda \textdied 05.05.1959  ---  Edvard \textdied 28.05.1959

\jhpic{Mietala113,116,118,15.jpg}{Gårdarna 113, 116, 118 och 15 på Mietala}


\jhoccupant{Eriksson}{Simon \& Kajsa}{1881--\allowbreak 1894}
Simon Eriksson,  \textborn 19.01.1831 i Purmo, gift med Kajsa Gustava Jakobsdr, \textborn 30.10.1830 i Munsala.
\begin{jhchildren}
  \item \jhperson{\jhname[Johan]{Eriksson, Johan}}{30.07.1864}{15.06.1890}
  \item \jhperson{\jhbold{\jhname[Simon]{Eriksson, Simon}}}{25.10.1870}{}
  \item \jhperson{\jhname[Katrina]{Eriksson, Katrina}}{06.12.1872}{}
  \item \jhperson{\jhname[Anna Gustava]{Eriksson, Anna Gustava}}{02.08.1874}{10.11.1893}
  \item \jhperson{\jhbold{\jhname[Jakob Edvard]{Eriksson, Jakob Edvard}}}{06.01.1878}{}
\end{jhchildren}

Edvards far, Simon, var född i Purmo, men kom som måg till Mietala. Av svärfadern fick makarna en lägenhet på Mietala. Den såldes och 1881 köpte Simon och Kajsa en annan lägenhet om 1/12 mantal på Mietala. I köpet ingick kvarnställe och kvarn vid Siltabäcken. Denna lägenhet överlämnades åt sönerna Edvard och Simon år 1894. Simon emigrerade och avled i USA, varvid Edvard köpte hans andel år 1905.

Kajsa \textdied 15.08.1894  ---	Simon \textdied 03.12.1910


\jholdhouse{Gamla gården Sandbacka}{9:}{Mietala}{14}{113 b}

Gården var en liten, gul gård med långsidan mot Åstrands (16, 116).	I gården bodde Edvards systrar, samt under kriget evakuerade från Karelen och Kemijärvi. Eventuellt byggdes denna gård redan på 1890-talet. I Simon Erikssons sytningskontrakt står nämligen ``vi ska bo i en gård... ''. Följande torpare var skrivna under detta hemman och bodde eventuellt i denna gård.


\jhoccupant{Mattsson}{Anders \& Greta}{1875--\allowbreak 1910}
Anders Mattsson, \textborn 30.11.1834 i Larsmo, gift med Greta Andersdotter, \textborn 04.10.1841 i Jeppo. I kyrkböckerna är Anders skriven under Sandells hemman under åren 1875--\allowbreak 1910. Tidigare bostad fanns på Tollikko.
\begin{jhchildren}
  \item \jhperson{\jhname[Johannes]{Mattsson, Johannes}}{24.06.1875}{}, vigd 1899 i  Minnesota m. Maria Lovisa Böös
  \item \jhperson{\jhname[Maria Sofia]{Mattsson, Maria Sofia}}{19.10.1879}{}
\end{jhchildren}


\jhoccupant{Stam}{Johan \& Anna}{1861--\allowbreak 1870}
Torparen Johan Eriksson Stam, \textborn 25.05.1837, gifte om sig 14.01.1861 med Anna Brita Andersdotter Mjetala,  \textborn 02.07.1838 i Larsmo. Hans första fru hade dött 1858.
\begin{jhchildren}
  \item \jhperson{\jhname[Erik]{Stam, Erik}}{22.08.1861}{}
  \item \jhperson{\jhname[Anders]{Stam, Anders}}{03.07.1863}{}
  \item \jhperson{\jhname[Anna Sanna]{Stam, Anna Sanna}}{14.08.1865}{}
  \item \jhperson{\jhname[Johan]{Stam, Johan}}{29.12.1867}{}
\end{jhchildren}

Johan och Anna bodde som inhyses på Mietala. De var skrivna under bonden Simon Eriksson. Johan dog 15.09.1870. Änkan Anna Brita flyttade med barnen till Böös, (Böös, gård nr 137).



\jhhouse{Potatishall}{9:75}{Mietala}{14}{14}


\jhoccupant{Elenius}{Lars-Erik}{2013--}
Lars-Erik Elenius (Mietala 12) köpte hallen år 2013.\jhvspace{}


\jhoccupant{Jeppo}{Potatis}{1979--\allowbreak 2013}
Företaget Jeppo Potatis köpte hallen år 1979. Hallen fungerade fortsättningsvis som potatislager.\jhvspace{}


\jhoccupant{Sandell}{Harry}{1974--\allowbreak 1979}
Harry Sandell (Mietala 13) byggde potatishallen år 1974 för eget behov som förvaringsutrymme för potatis.\jhvspace{}



\jhhouse{Södergård}{9}{Mietala}{14}{114}


\jhoccupant{Sandberg}{Isak \& Anna}{1868--\allowbreak 1898}
Isak Sandberg, \textborn 14.12.1839 på Lavast hemman, gift med Anna Greta Abramsdotter,  \textborn 13.03.1850 på Kärr hemman i Nykarleby lk.
\begin{jhchildren}
  \item \jhperson{\jhname[Johannes]{Sandberg, Johannes}}{19.02.1871}{}
  \item \jhperson{\jhname[Anna Lovisa]{Sandberg, Anna Lovisa}}{08.08.1872}{}
  \item \jhperson{\jhname[Ida Johanna]{Sandberg, Ida Johanna}}{28.01.1876}{}
  \item \jhperson{\jhname[Jakob]{Sandberg, Jakob}}{15.12.1877}{}
  \item \jhperson{\jhname[Hilda Maria]{Sandberg, Hilda Maria}}{20.12.1879}{}
  \item \jhperson{\jhname[Leander]{Sandberg, Leander}}{13.12.1881}{}
  \item \jhperson{\jhname[Emil]{Sandberg, Emil}}{23.12.1883}{}
\end{jhchildren}

Den 9 februari 1868 mottog Isak av sina föräldrar Södergård lägenhet. Han avled 09.01.1885 och Greta sålde det 1/12 mantal stora hemmanet på Mietala den 16.02.1898 till Johanna och Johan Gunnar. Hon klarade då inte mera av att sköta jordbruket. De två äldsta sönerna hade rest till Amerika, en dotter tagit tjänst och en dotter skulle gifta sig.

Hemmanet såldes på offentlig auktion. Anna Greta behöll 4 kpl ``utmed fätået mittemot Anders Mattssons boningsbyggnad''. Johanna och Johan Gunnar sålde gården vidare till sin son Johannes. (Mietala, gård nr 118)


\jhoccupant{Finskas}{Matts \& Greta}{1868--\allowbreak 1868}
Matts Jakobsson Finskas, \textborn 19.07.1810 på Keppo, gift med Greta Stina Karlsdotter,  \textborn 18.09.1814 på Kärr hemman i Nykarleby lk. Matts och Greta Stina fick 11 barn (1835--1857).

År 1868 köpte Matts' bror, Isak Sandberg, Södergård lägenhet åt Matts och Greta Stina. De lämnade dock lägenheten genast till sonen \jhbold{Isak} mot årlig sytning för Maria Östman samt sig själva.

Matts \textdied ca 1874  ---  Greta Stina \textdied 24.10.1893


\jhoccupant{Johansson}{Johan \& Brita}{1846--\allowbreak 1868}
Johan Erik Johansson Hannula, \textborn 20.04.1818 i Larsmo, gifte sig 10.09.1857 med Brita Carlsdotter, \textborn 11.03.1821. Brita varit gift tidigare med torparen Johan Erik Andersson (Mietala, gård nr 106). Barn med Hannula:
\begin{jhchildren}
  \item \jhperson{\jhname[Caisa Sophia]{Johansson, Caisa Sophia}}{23.07.1858}{02.08.1858}
  \item \jhperson{\jhname[Johannes Johan]{Johansson, Johannes Johan}}{21.06.1859 (Gunnar nr 125)}{}
  \item \jhperson{\jhname[Caisa Sophia]{Johansson, Caisa Sophia}}{15.08.1860}{}, dog som barn
  \item \jhperson{\jhname[Brita Caisa]{Johansson, Brita Caisa}}{01.08.1863}{20.07.1866}
\end{jhchildren}

Johan Hannula kom från Larsmo och köpte Södergård hemman år 1846. Han sålde det år 1868 och köpte ett hemman på Gunnar (125). Johan dog någon gång mellan åren 1881 och 1886, Brita dog 1904.



\jhhouse{Torppe}{9:112}{Mietala}{14}{15, 15a-b}


\jhhousepic{231-05809.jpg}{Göran Eklöf}

\jhoccupant{Eklöf}{Göran}{2011--}
Eklöf Göran, \textborn 10.01.1962 på Gunnar, köpte år 2011 Forsbackas bostadshus och ekonomiebyggnader med tillhörande tomtmark och hemåker. Bostadshuset är obebott idag. Gunborg och hennes familj äger ännu en del av de 0,0625 mantal av Mietala hemman som Elmer övertog.


\jhoccupant{Forsbacka}{Gunborg \& Gunnar}{1960--\allowbreak 2011}
Gunborg Ekström, \textborn 21.10.1929 på Mietala, gifte sig 24.04.1949 med Gunnar Forsbacka, \textborn 17.12.1927 på Forsbacka i Nykarleby. Gunborg och Gunnar var jordbrukare hela sitt verksamma arbetsliv, fr.o.m. 1960 som lägenhetsägare, först med mjölkproduktion och senare på 1970-talet med smågrisproduktion som huvudnäring. Lägenheten köpte de av Elmer och Jenny Ekström.
\begin{jhchildren}
  \item \jhperson{\jhname[Bjarne]{Forsbacka, Bjarne}}{27.08.1956}{}, dipl.ing
  \item \jhperson{\jhname[Rune]{Forsbacka, Rune}}{18.08.1959}{}, el.ing
\end{jhchildren}

Bostadshuset är byggt av Gunborgs far år 1925--\allowbreak 1926. En tillbyggnad gjordes senare så att Gunborg och Gunnar efter sitt
giftermål kunde få en egen lägenhet i huset. År 1965 installerades vattenburet värmesystem. Lägenheten omfattade 25 ha odlad jord och 16 ha skog samt 25 ha skog i Nykarleby.

Gunnar \textdied 01.02.2010  ---  Gunborg är på Hagalund.


\jhoccupant{Ekström}{Elmer \& Jenny}{1925--\allowbreak 1960}
Elmer Ekström, \textborn 28.02.1898 på Mietala, gifte sig 12.06.1921 med Jenny Nylund, \textborn 03.12.1899 i Nykarleby.
\begin{jhchildren}
  \item \jhperson{\jhname[Gunnar]{Ekström, Gunnar}}{10.08.1922}{10.08.1943}, stupade i Svir
  \item \jhperson{\jhbold{\jhname[Gunborg]{Ekström, Gunborg}}}{21.10.1927}{}
  \item \jhperson{\jhname[Gertrud]{Ekström, Gertrud}}{25.11.1934}{1977}, g. Romar (Romar nr 11)
\end{jhchildren}

Elmer övertog hälften av sina föräldrars lägenhet (Mietala 116), brodern Emil den andra hälften. Några av Elmers äldre bröder hade rest till  Amerika och år 1922 gjorde Elmer detsamma. År 1925 kom han tillbaka och kunde då bygga bostadshus åt sig och sin familj. Enligt köpebrev 22 september 1926 blir Elmer och Jenny ägare till 0,0625 dels mantal av Mietala hemman. Säljare är modern Lena Sofia.

Elmer \textdied 19.05.1969  ---  Jenny \textdied 06.08.1977



\jhhouse{Åstrand}{9:84}{Mietala}{14}{16, 16a-b}


\jhhousepic{230-05808.jpg}{Vivian och Simo Hautamäki}

\jhoccupant{Hautamäki}{Vivian \& Simo}{1990--}
Vivian Åstrand, \textborn 13.04.1957 på Mietala, gifte sig 13.08.1977 med Simo Hautamäki, \textborn 16.03.1955 i Jeppo. Simo har arbetat på Mirka fram till sin sjukpensionering. Vivian arbetar som kvalitetsassistent på Snellmans Köttförädling Ab. Spannmålsodling och  svinuppfödning sköttes vid sidan av förvärvsarbetet. Idag är marken utarrenderad. Vivian och Simo byggde nytt hus 1981 (Mietala 17). År 1990 bytte de hus med Vivians föräldrar.
\begin{jhchildren}
  \item \jhperson{\jhname[Anne]{Hautamäki, Anne}}{20.07.1978}{}, g. Eckerman, tradenom, på Paaluperustajat
  \item \jhperson{\jhname[Tomi]{Hautamäki, Tomi}}{26.10.1984}{}, arbetar som processkötare på Mirka
\end{jhchildren}


\jhoccupant{Åstrand}{Edvin \& Astrid}{1971--\allowbreak 1990}
Edvin, \textborn 30.10.1925 på Mietala, gift  18.06.1961 med Astrid Lillbacka, \textborn 10.11.1927 i Kimo, Oravais.
\begin{jhchildren}
  \item \jhperson{\jhbold{\jhname[Vivian]{Åstrand, Vivian}}}{13.04.1957}{}
  \item \jhperson{\jhname[Lilian]{Åstrand, Lilian}}{26.04.1965}{}, g. Ahlvik, arb. m. projektadm. på Wärtsilä i Vasa
\end{jhchildren}

Edvin och Astrid övertog hemmanet år 1962. I mitten på 60-talet hade man 9 kor och mjölkproduktion var huvudnäring. De  byggde nytt hus år 1971, på samma tomt där det gamla huset fanns. Stockarna från det gamla huset såldes till Solf.

Edvin \textdied 25.01.1998  ---  Astrid \textdied 21.08.1997


\jholdhouse{Gamla gården Ekström}{9:}{Mietala}{14}{116}

\jhoccupant{Åstrand}{Edvin \& Astrid}{1961--\allowbreak 1971}
Edvin, \textborn 30.10.1925 på Mietala, gift  18.06.1961 med Astrid Lillbacka, \textborn 10.11.1927 i Kimo, Oravais. Mera om Astrid och Edvin under gård nr 16.\jhvspace{}


\jhhousepic{Mietala116-Astrand.jpg}{Åstrands hus}

\jhoccupant{Åstrand}{Johannes \& Emilia}{1933--\allowbreak 1970}
Johannes, \textborn 18.09.1893 i Jeppo,  gift 13.07.1924 med Emilia Almberg, \textborn 19.07.1895 på Grötas.
\begin{jhchildren}
  \item \jhperson{\jhbold{\jhname[Edvin]{Åstrand, Edvin}}}{30.10.1925}{}
  \item \jhperson{\jhname[Margita]{Åstrand, Margita}}{04.02.1936}{21.03.1970}, gift Ek
\end{jhchildren}

Johannes och Emilia köpte hemmanet 22.03.1933 av Emil och Hilda Ekström. Samtidigt sålde de sin lägenhet på  Ruotsala till Emil och Hilda. I köpeavtalet ingick sytning till Emils mamma, som kom att bo i ena ändan av huset. Nya ekonomibyggnader byggdes 1946. Huset hade tidigare ägts av Emilias mor Maria Lovisa Andersdotter Mietalas bror Johan Andersson Mietala. Johannes och Emilia kunde köpa hemmanet och senare även tilläggsjord efter att Johannes arbetat i Amerika under två perioder.

Johannes \textdied 02.04.1963  ---  Emilia \textdied 26.05.1970


\jhoccupant{Ekström}{Emil \& Hilda}{1913--\allowbreak 1933}
Johan Emil Johansson Mietala, \textborn 16.06.1882 på Mietala, gifte sig 19.12.1909  med Hilda Karolina Mattsdotter Ström, \textborn 29.01.1888.
\begin{jhchildren}
  \item \jhperson{\jhname[Johannes Emil]{Ekström, Johannes Emil}}{28.07.1910}{}
  \item \jhperson{\jhname[Elmer Edvin]{Ekström, Elmer Edvin}}{13.03.1913}{}, el-mont., saknad e striderna v Summa 13.02.1940
  \item \jhperson{\jhname[Torsten Lennart]{Ekström, Torsten Lennart}}{09.11.1914}{}
  \item \jhperson{\jhname[Gunhild Maria]{Ekström, Gunhild Maria}}{05.01.1918}{}
  \item \jhperson{\jhname[Gunnar Wilhelm]{Ekström, Gunnar Wilhelm}}{05.12.1919}{}
  \item \jhperson{\jhname[Anders Werner]{Ekström, Anders Werner}}{30.06.1921}{}
  \item \jhperson{\jhname[Matts Robert]{Ekström, Matts Robert}}{09.07.1923}{}
  \item \jhperson{\jhname[Paul Olof]{Ekström, Paul Olof}}{26.12.1924}{}
  \item \jhperson{\jhname[Tor Bernhard]{Ekström, Tor Bernhard}}{19.02.1928}{}
  \item \jhperson{\jhname[Ruth Iris Gunvor]{Ekström, Ruth Iris Gunvor}}{31.08.1931}{}
\end{jhchildren}

Hälften av hemmanet, dvs 0,0616 mantal, övertogs av Emil år 1913, andra hälften övertogs senare av brodern Elmer. Emil sålde sin del av hemmanet åt Johannes Åstrand och flyttade med familjen till Silvast (Fors nr 397a).


\jhoccupant{Mietala}{Johan \& Sofia}{1892--\allowbreak 1913}
Johan Andersson Mietala, \textborn 23.12.1853, gift med Lena Sofia Simonsdr, \textborn 15.10.1859.
\begin{jhchildren}
  \item \jhperson{Johan \jhbold{Emil}}{16.06.1882}{}
  \item \jhperson{\jhname[Maria Lovisa]{Mietala, Maria Lovisa}}{01.12.1885}{}
  \item \jhperson{\jhname[Anders Gustaf]{Mietala, Anders Gustaf}}{05.01.1889}{27.06.1931 i USA}
  \item \jhperson{\jhname[Leander]{Mietala, Leander}}{18.11.1892}{29.10.1950 i USA}
  \item \jhperson{\jhname[Edit Emilia]{Mietala, Edit Emilia}}{07.08.1895 (Mietala 108)}{}
  \item \jhperson{\jhbold{\jhname[Elmer]{Mietala, Elmer}} Edvin}{28.02.1898 (Mietala nr 15)}{}
  \item \jhperson{\jhname[Ivar Lennart]{Mietala, Ivar Lennart}}{22.11.1900}{}
  \item \jhperson{\jhname[Artur]{Mietala, Artur}}{06.01.1904}{}
\end{jhchildren}

Bostadshuset byggdes år 1890, men om det var Johan, som byggde det är obekräftat. År 1892 köpte Johan 1/18 mantal av Wilhelmina och Gustaf Andersson. Wilhelmina var Sofias syster. Av föräldrarna Anders och Maria Lovisa erhöll de 1/16 mantal. Senare övertog Johan hela hemmanet, omfattande 1/8 mantal. I sytningskontraktet ingick ``särskildt boningsrum i framstugan med därofvan varande vindskammare'' för Johans föräldrar.

Johan \textdied 15.09.1913  ---  Sofia \textdied 12.06.1955


\jhoccupant{Johansson}{Anders \& Maria}{1850--\allowbreak 1892}
Husbonden Anders Johansson,  \textborn 01.05.1830 på Jungarå, gift med Maja Lovisa (Maria) Eriksdotter, \textborn 18.09.1834 i Jeppo.
\begin{jhchildren}
  \item \jhperson{\jhbold{\jhname[Johan]{Johansson, Johan}}}{23.12.1853}{}
  \item \jhperson{\jhname[Erik]{Johansson, Erik}}{20.02.1856}{}
  \item \jhperson{\jhname[Gustaf]{Johansson, Gustaf}}{11.09.1858}{}
  \item \jhperson{\jhname[Thilda]{Johansson, Thilda}}{28.04.1861}{}
  \item \jhperson{\jhname[Anton]{Johansson, Anton}}{16.10.1863}{}
  \item \jhperson{\jhname[Maria Lovisa]{Johansson, Maria Lovisa}}{16.12.1866}{}, g. m. Anders Eriksson (Grötas 111)
  \item \jhperson{\jhname[Albertina]{Johansson, Albertina}}{25.10.1867}{}
\end{jhchildren}

Anders är skriven som husbonde på Mietala åren 1865--\allowbreak 1895. Enligt avhandling 18 november 1850 har hemmanet kommit i hans ägo efter	``avlidne fadern'' Johan Johansson Mietala, \textborn 1787 på Mietala, gift med Brita Stina Johansdotter, \textborn 1797 i Härmä. Fadern hade dött 1850, modern 1846.	Vid Anders död 1896 bestod hemmanet av 1/8 mantal. Johan hade år 1892 erhållit 1/16 mantal.

Anders \textdied 07.02.1896  ---  Maja \textdied 19.03.1903



\jhhouse{Mangelstuga}{9}{Mietala}{14}{116a}

\jhoccupant{Gunnar}{Marthakrets Mangelstuga}{1932--\allowbreak 1979}
Gunnar Marthakrets bildades år 1923, då Jeppo Marthaförening delades i tre kretsar på grund av att avstånden till möten var för långa. Maria Liljeqvist blev kretsens första ordförande. Hennes man, som var lärare i byn, ledde Marthakören till sin död 1946. När kretsen bildades var 9 marthor med, men redan efter 3 år var 26 marthor inskrivna.

På 1930-talet skaffades vävstolar, serviser och mangel. 1932 köpte Gunnar Marthakrets sin mangel från Kvevlax. En enskild medlem lånade pengar till mangeln. Det var tänkt att mangeln skulle flyttas en gång/år inom området, men år 1936 beslöt man bygga ett hus för ändamålet. Huset byggdes på Åstrands tomt.

På 1970-talet blev behovet av denna mangel mindre, familjer skaffade egna manglar. Mangelhuset revs 27 april 1979. Mangeln förvarades en tid i ett uthus, men såldes senare.



\jhhouse{Finström}{9:101}{Mietala}{14}{17, 17a}


\jhhousepic{232-05882.jpg}{Mikael Finström}

\jhoccupant{Finström}{Mikael}{2000--}
Mikael Finström, \textborn 11.02.1963, gifte sig 25.06.2010 med Anne Waskinen, \textborn 03.08.1965 i Lahtis. Mikael arbetar som chaufför.\jhvspace{}


\jhoccupant{Åstrand}{Edvin \& Astrid}{1988--\allowbreak 2000}
Åstrand Edvin, \textborn 30.10.1925 på Mietala, gift med Astrid Lillbacka, \textborn 10.11.1927 från Kimo. (Se närmare gård nr 16). År 1988 bytte Edvin och Astrid bostad med dottern Vivian och hennes familj. Efter föräldrarnas död stod gården tom tills den såldes till Mikael Finström.


\jhoccupant{Hautamäki}{Vivian \& Simo}{1981--\allowbreak 1988}
Vivian Åstrand, \textborn 13.04.1957  på Mietala i Jeppo, gifte sig 13.08.1977 med Simo Hautamäki, \textborn 16.03.1955 i Jeppo. Vivian och Simo Hautamäki byggde bostadshuset år 1981 samt ekonomibyggnaden 1982. På tomten fanns då inga byggnader. Närmare uppgifter om familjen under Mietala, gård nr 16.



\jhhouse{Södergård}{9:61}{Mietala}{14}{18, 18s}


\jhhousepic{233-05813.jpg}{John och Pirkko Elenius}

\jhoccupant{Elenius}{John \& Pirkko}{1978--}
John, \textborn 26.01.1946 i Jeppo, gift med Pirkko Korkiamäki, \textborn 05.11.1944 i Kauhava. John och Pirkko övertog föräldrarnas lägenhet på Gunnar hemman år 1978. Nuvarande gård uppfördes 1963 av Johns far Bertel. John har odlat potatis och spannmål samt haft pälsfarm. Pirkko har fram till pensionering arbetat på Jeppo Potatis.
\begin{jhchildren}
  \item \jhperson{\jhname[Johanna]{Elenius, Johanna}}{22.02.1975}{}, ekon., vid Europeiska kommissionen i Bryssel
  \item \jhperson{\jhname[Christer]{Elenius, Christer}}{24.07.1976}{}, ekonom, bor i Vasa
\end{jhchildren}


\jhoccupant{Elenius}{Bertel \& Anni}{1963--\allowbreak 1978}
Elenius Bertel och Anni och deras barn bodde 1963--\allowbreak 1978 i denna gård. Se närmare gård nr 118.\jhvspace{}



\jhhouse{Södergård}{9:34}{Mietala}{14}{118}


\jhoccupant{Elenius}{Bertel \& Anni}{1945--\allowbreak 1963}
Bertel, \textborn 25.09.1914 på Gunnar, gift med Anni Björkvall, \textborn  22.04.1922 på Grötas hemman. Paret gifte sig under kriget, sommaren 1944. Bertel deltog i både vinter- och fortsättningskriget. År 1945 köpte de Södergård lägenhet på Mietala hemman av Martin Elenius. Lägenheten var ursprungligen tänkt för Martins son, som stupade i kriget. Anni och Bertel var jordbrukare, främst mjölkproducenter. Bertel moderniserade år 1958 ekonomibyggnaden och byggde 1963 ny egnahemsgård.
\begin{jhchildren}
  \item \jhperson{\jhbold{\jhname[John]{Elenius, John}}}{26.01.1946}{}
  \item \jhperson{\jhname[Gunnel]{Elenius, Gunnel}}{03.03.1951 (Gunnar 19)}{}
\end{jhchildren}



\jhoccupant{Elenius}{Erik \& Sanna}{1940--\allowbreak 1945}
Erik Elenius, \textborn 07.11.1870 i Jeppo, gift med Sanna Bärs, \textborn 29.12.1872 (Mietala 112), bodde med sin familj på Jungarå till år 1903, då han köpte Herrgård lägenhet på Mietala. Han köpte också i början på 1940-talet betydande delar av Södergård lägenhet. Han bodde dock aldrig på lägenheten, utan hade köpt den för sin son Arne, som stupade i kriget. Arnes bror Martin kom därför att äga Södergård lägenhet, som han sedan sålde till Bertel och Anni Elenius.


\jhoccupant{Elenius}{Lennart \& Elma}{1927--\allowbreak 1940}
Lennart, \textborn 08.10.1904 i Jeppo, gift med Elma Södergård, \textborn 01.03.1910 i Jeppo. Elma dog 26.01.1939 och Lennart ingick nytt äktenskap med Göta Forss, \textborn 12.03.1916  i Jeppo.
\begin{jhchildren}
  \item \jhperson{\jhname[Doris]{Elenius, Doris}}{24.07.1929, gift Björkell}{}
  \item \jhperson{\jhname[Gun-Lis]{Elenius, Gun-Lis}}{28.05.1941, sambo med Lars Nyberg}{}
  \item \jhperson{\jhname[Ann-Lis]{Elenius, Ann-Lis}}{24.02.1944, gift Brunnsberg}{}
  \item \jhperson{\jhname[Göran]{Elenius, Göran}}{18.09.1948 (Romar 24 )}{}
\end{jhchildren}

År 1927 hade Lennarts far, Joel, på auktion inropat Södergårds lägenhet och överlämnat den åt Lennart och Elma. Vid sidan av jordbruket idkade Lennart lastbilstransport av olika slag, främst grustransporter. Han hade även taxiverksamhet. Han var bl.a ledamot i direktionen för Gunnar folkskola, vägnämnden, Jungar Andelsmejeri samt ordförande i Jeppo Jaktvårdsförening. Efter Elmas död sålde Lennart sin gård och bosatte sig på Romar, gård 24.


\jhoccupant{Gunnar}{Johannes \& Hilda}{1908 – 1913(1927)}
Johannes Gunnar, \textborn 26.07.1879 i Jeppo, gifte sig 1906 med Hilda Jungarå, \textborn 25.03.1886 i Jeppo. Johannes hade i unga år varit på på arbetsförjänst i Amerika och Sydafrika. Hemkommen gifte han sig och köpte av sin far denna hemmansdel. Han antog samtidigt efternamnet Södergård.
\begin{jhchildren}
  \item \jhperson{\jhname[Johannes Ivar]{Gunnar, Johannes Ivar}}{08.08.1907, till Canada}{}
  \item \jhperson{\jhname[Linnea Katrina]{Gunnar, Linnea Katrina}}{01.06.1909, g. Stenbacka}{}
  \item \jhperson{\jhname[Ester Johanna]{Gunnar, Ester Johanna}}{03.11.1910, till Canada}{}
  \item \jhperson{\jhname[Anders Vilhelm]{Gunnar, Anders Vilhelm}}{07.09.1912, till Canada}{}
\end{jhchildren}

Den gamla gården torde vara byggd år 1909 av Johannes. Gruvsjukan ändade Johannes liv redan 23.09.1913. Änkan Hilda gifte om sig med drängen Juho Ojanperä, \textborn 19.03.1898 i Alajärvi. I detta äktenskap föddes fem barn.
\begin{jhchildren}
  \item \jhperson{\jhname[Ellen Ragnhild]{Gunnar, Ellen Ragnhild}}{06.07.1917, g. Vistbacka}{}
  \item \jhperson{\jhname[Gustav Lennart]{Gunnar, Gustav Lennart}}{23.11.1919}{}
  \item \jhperson{\jhname[Evert Runar]{Gunnar, Evert Runar}}{03.11.1921}{}
  \item \jhperson{\jhname[Elis Gunnar]{Gunnar, Elis Gunnar}}{03.11.1921}{24.05.1942}, stupade vid Svir
  \item \jhperson{\jhname[Ethel Alfhild]{Gunnar, Ethel Alfhild}}{17.11.1923, g. Vik}{}
\end{jhchildren}

År 1923 emigrerade Ojanperä till Kanada och Hilda lämnade ensam med barnen. Lungsoten ändade hennes liv 17.08.1927. Barnen skingrades, de yngsta togs om hand av släktingar och bekanta. Södergård såldes på auktion till bonden Joel Elenius.

Södergård hemmans historia fortsätter under Mietala, gård 114. Vi vet inte var tidigare hus stått, men eftersom det finns uppgifter att Isak Sandberg bott på platsen för gård 114, så antar vi att andra ägare av Södergård hemman bott där.
