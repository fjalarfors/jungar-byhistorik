%%%
% [chapter] Gunnar, hemman Nr 10
%
\jhchapter{Gunnar, hemman Nr 10}

Gunnar hemman kom med i skattelängderna år 1567. Hemmanet skattade för 1 mtl, från 1703  1/2 krono mtl. Det är det hemman där kronoskatt hängt längst med. Ännu år 1900 hade de då 4 hemmansägarna olika stor kronoskatt.

Per Jönsson innehade hemmanet 1567--\allowbreak 1600. Hans son Tomas Persson är antecknad som ägare 1602--\allowbreak 1631. År 1626 hade han ``2 ston, 6 kor, 2 kvigor, 8 får, 4 lamm, 1 svin, 2 kokalvar och 4 tunnland odlad jord''. År 1631 antecknas Gunnar som ``Lendzmanns'' hemman.

Gunnar hemman har sitt namn efter Matts Olofsson Gunnar, som blev ägare till hemmanet 1677 (1677--\allowbreak 1683). Han kom från Kunnari (Gunnar) i Naarasluoma by i Lappo. I rannsakningsjordeboken 1679 står bl.a att han har 2 kor och 1 elak häst, att hemmanet är förfallet, husen förfallna, skogen genom vådeld uppbrunnen, av åkern halvparten i linda. Matts varit i krigstjänst. 4 av sönerna blev också soldater. Sonen Gustav Mattsson (1686--\allowbreak 1706 samt mågen Markus Mattsson (1707--\allowbreak 1733) var ägare till hemmanet efter Matts. Markus Mattsson brukade jorden också efter stora ofreden.

Övriga ägare har varit: Jakob Tomasson 1634--\allowbreak 1664, Johan Jakobsson 1669--\allowbreak 1673, Anders Sigfridsson 1675,  Hindrik Sigfridsson 1676,  mågen Mårten Abrahamsson 1733--\allowbreak 1741, sonen Thomas Mårtensson 1742--\allowbreak 1774, Mickel Mattsson och förre ägarens son, Matts Thomasson, äger tillsammans hemmanet år 1791. Antal ägare till hemmanet varierar sedan mellan 2 och 5.


Gunnar hemman omfattas av vidstående karta nr \jhbold{15}.


<--- se KARTA nr 15 --->


%%%
% [subsection] Lägenheter på Gunnar
%
\jhsubsection{Lägenheter på Gunnar}



%%%
% [house] Nystrand
%
\jhhouse{Nystrand}{10:78}{Gunnar}{15}{20, 20a}


\jhhousepic{236-05815.jpg}{Göran Eklöv}

%%%
% [occupant] Eklöf
%
\jhoccupant{Eklöf}{\jhname[Göran]{Eklöf, Göran}}{1983--}
Göran Eklöf, \textborn 10.01.1962 på Gunnar, gifte sig 06.09.2003 med Susann Skog, \textborn 24.01.1972 i Kronoby. Susann dog i trafikolycka 15.06.2009.
\begin{jhchildren}
  \item \jhperson{\jhname[Anders]{Eklöf, Anders}}{10.03.2005}{}
  \item \jhperson{\jhname[Joel]{Eklöf, Joel}}{11.02.2008}{}
\end{jhchildren}

Göran övertog år 1984  föräldrarnas hemman på Gunnar. Gården tillhörde länge det fåtal i Jeppo som hade mjölkproduktion som huvudnäring, men upphörde med den år 2015. Idag arbetar Göran med kalvuppfödning.


%%%
% [occupant] Eklöf
%
\jhoccupant{Eklöf}{\jhname[Ruben]{Eklöf, Ruben} \& \jhname[Annikki]{Eklöf, Annikki}}{1955--\allowbreak 1983}
Ruben Eklöf,  \textborn 20.07.1927 på Gunnar, gifte sig 12.06.1955 med Annikki  Perälä,  \textborn 17.06.1934 i Evijärvi. Ruben dog 23.10.2015.
\begin{jhchildren}
  \item \jhperson{\jhname[Johan]{Eklöf, Johan}}{22.01.1956}{}
  \item \jhperson{\jhbold{\jhname[Göran]{Eklöf, Göran}}}{10.01.1962}{}
\end{jhchildren}

Ruben och Annikki övertog Rubens föräldrars lägenhet på Gunnar. Förutom mjölkproduktion, hade Ruben i 20 års tid en minkfarm. Farmen fick sin början genom att han fångade vildminkar. Minkmat köpte han av Evert Norrgård, som tillverkade minkfoder.


%%%
% [occupant] Eklöf
%
\jhoccupant{Eklöf}{\jhname[Joel]{Eklöf, Joel} \& \jhname[Helmi]{Eklöf, Helmi}}{1921--1955}
Joel Eklöf, \textborn 18.09.1894 på Gunnar, g m Helmi Strengell, \textborn 28.05.1900 på Jungar.
\begin{jhchildren}
  \item \jhperson{\jhname[Guladis]{Eklöf, Guladis}}{18.12.1923}{18.05.1973}, gift Lärka i Munsala
  \item \jhperson{\jhbold{\jhname[Ruben]{Eklöf, Ruben}}}{20.07.1927}{23.10.2015}
  \item \jhperson{\jhname[Birgit]{Eklöf, Birgit}}{22.07.1935}{}, gift Emet
\end{jhchildren}

Joel emigrerade till USA, där han arbetade några år. Han återvände hem år 1921 och övertog den andra delen av föräldrahemmanet. Han inledde arbetet på familjens nya hem med åtföljande nya ekonomiebyggnader. Under byggnadsarbetet hade man en murare från Härmä. Joel kunde inte finska och muraren inte svenska, men de pratade engelska sinsemellan eftersom båda varit till Amerika. På grund av svag hälsa dog Joel tidigt och hann inte flytta in i sitt nya hem.

Joel \textdied 03.09.1936  ---  Helmi \textdied 06.05.1985



%%%
% [house] Kronlund
%
\jhhouse{Kronlund}{10:28}{Gunnar}{15}{21, 21a-b}


\jhhousepic{235-05816.jpg}{Johan och Hans Kronlund. Huset är obebott.}

%%%
% [occupant] Kronlund
%
\jhoccupant{Kronlund}{\jhname[Johan]{Kronlund, Johan} \& \jhname[Hans]{Kronlund, Hans}}{1979--}
Bröderna Kronlund Johan, \textborn 22.12.1956, och Hans, \textborn 22.05.1958, övertog hemmanet 1979. Mjölkproduktionen, som var huvudnäring på gården från den tid föräldrarna skötte lantbrukslägenheten, upphörde 2010.

Johan började arbeta på Härmä Food, gifte sig, flyttade till radhuset Älvliden i Silvast (Grötas, gård nr 9).

Hans och hans sambo Gun-Helen Häggback, \textborn 26.03.1973 i Malax, bodde tillsammans med unga sonen Jakob, \textborn 30.01.2010, en tid i gården. De flyttade till Älvliden i januari år 2011, bor idag på Böös (gård nr 45). Gården på Gunnar är idag obebodd.


%%%
% [occupant] Kronlund
%
\jhoccupant{Kronlund}{\jhname[Gunnar]{Kronlund, Gunnar} \& \jhname[Edit]{Kronlund, Edit}}{1944--\allowbreak 1979}
Gunnar Kronlund, \textborn 02.10.1920 på Gunnar hemman, vigd 1954 med Edit Thors, \textborn 01.03.1920 i Bertby, Vörå. Gunnar gick i Korsholms lantmannaskola 1940--41. I samband med bouppteckningen efter Gunnars mor Hilda, skrevs hemmanet på Gunnar. Det var en lägenhet, som bestod av olika parceller. Gunnar och Edit köpte sedan tillskottsjord och fick också åkerjord från Edits hemgård.
\begin{jhchildren}
  \item \jhperson{\jhbold{\jhname[Johan]{Kronlund, Johan}}}{22.12.1956}{}, student (Grötas, gård nr 9)
  \item \jhperson{\jhbold{\jhname[Hans]{Kronlund, Hans}}}{22.05.1958}{}, lantbrukstekniker (Böös, gård 45)
  \item \jhperson{\jhname[Mats]{Kronlund, Mats}}{13.11.1960}{}, merkonom
\end{jhchildren}

Gunnar har varit kommunfullmäktig samt ledamot i Jeppo andelsmejeris styrelse och Gunnar folkskoldirektion.

Gunnar \textdied 04.03.1985  ---  Edit \textdied 12.01.2004



%%%
% [house] Präntagälon
%
\jhhouse{Präntagälon}{10:}{Gunnar}{15}{121a}


\jhhousepic{Gunnar121-EmilKronlund.jpg}{Kojonen-Kronlund}

%%%
% [occupant] Kojonen
%
\jhoccupant{Kojonen}{\jhname[Emil]{Kojonen, Emil} \& \jhname[Hilda]{Kojonen, Hilda}}{1913--\allowbreak 1935}
Emil Kojonen, senare Kronlund, \textborn 24.01.1887 på Kojonen hemman reste som 22-åring till USA, där han arbetade i koppargruvor i Butte, Montana. Efter tre år återvände han hem, satte sina besparingar i Nykarleby bank, som gick i konkurs. Själv var Emil lika fattig som före emigrationen. År 1913 gifte han sig med Hilda Johansdotter, \textborn 17.07.1887 på Gunnar.
\begin{jhchildren}
  \item \jhperson{\jhname[Johannes]{Kojonen, Johannes}}{23.01.1914}{}
  \item \jhperson{\jhname[Gunnar]{Kojonen, Gunnar}}{02.10.1920}{}
\end{jhchildren}

Makarna köpte det tidigare soldattorpet på Gunnar. Emil fick utkomst genom forsrensningsarbete, som pågick vid den tiden. Då äldsta sonen var tre år reste Emil på nytt på arbetsförtjänst till USA. Hemkommen satte han sina besparingar på bank. Den stora devalveringen kom och Emils besparingar blev ännu en gång värdelösa. Han arbetade nu som byggnadsarbetare, bl.a på Keppo och i Oravais.

Sommaren 1923 emigrerade Emil en tredje gång med två fasta beslut, nämligen att börja med husbyggnadsarbete och att genast placera sina besparingar i jordköp hemma i Jeppo. Han sände hem pengar och gav hustrun order om att köpa 4 ha av Johan Forsman. Sedan han kom hem köpte han ett annat skifte om 6 ha samt ett par mindre skogsskiften. En ny gård byggdes 1934.

Sonen Johannes prästvigdes 1939, var kyrkoherde i flere församlingar, arbetade som rektor bl.a vid  Evangeliska folkhögskolan i Jeppo och Vasa, erhöll prostetitel, gav ut flere publikationer. Sonen Gunnar övertog hemmanet 1944.

Hilda \textdied 13.09.1944  ---  Emil \textdied 17.02.1956


%%%
% [occupant] Svanbäck
%
\jhoccupant{Svanbäck}{\jhname[Oskar]{Svanbäck, Oskar} \& \jhname[Ida]{Svanbäck, Ida}}{1913}
Oskar Svanbäck, \textborn 10.03.1889 på Mietala hemman, gift med Ida Sannasdotter Slangar, \textborn 1885 på Slangar. Se närmare Mietala, gård nr 106. Enligt ett köpekontrakt daterat 17 april 1913 säljer Oskar och Ida Svanbäck en stugubyggnad och uthus åt Emil Kojonen. När stugan övergått i deras ägo förblir oklart.


%%%
% [occupant] Nyman
%
\jhoccupant{Nyman}{\jhname[Anders]{Nyman, Anders} \& \jhname[Maria]{Nyman, Maria}}{1910--\allowbreak 1913}
Anders Nyman, \textborn 26.11.1872 i Jeppo, vigd 02.02.1902 med Maria Eriksdotter Grötas, \textborn 26.06.1874 i Jeppo. Anders var skräddare och makarna, som var barnlösa, bodde först som backstugosittare på Jungarå, sedan på Gunnar. I kontraktet, som Anders gjorde år 1910 med Elias Gunnar, framkommer att Anders tidigare innehaft samt bebyggt det 4 3/4 kpl stora jordområdet. År 1913 flyttade Anders och Maria till Grötas, gård nr 112.


%%%
% [occupant] Soldattorp
%
\jhoccupant{Soldattorp}{\jhname[nr 141]{Soldattorp, nr 141}}{Norrbacka}
Soldattorp nr 141 Norrbacka finns utsatt på storskifteskartan från 1781 och av den framgår att torpet låg på Gunnar hemman. Den tidigare soldatåkern har under senare tider kallats ``Präntas-åkern''. Åkern gick i nord-sydlig riktning och sträckte sig från nuvarande Kronlunds gård ner till Gunnarsvägen. Enligt samma karta stod husen uppradade längs vägen vid åkerns södra ända. Om stugan fanns bland dem är osäkert. Enligt Johannes Kronlund (1914--86) var hans hem ett soldattorp.

I denna dokumentation är torpet placerat på samma plats som Kronlunds hemgård, som revs på 30-talet. Ansvar för detta torp hade Jungar Olof, Gunnar Marckus, Ekola Elias, Kojonen Johan och Krögarn Jacob. Vid syneförrättningen 1791 konstaterades att stugan och boden var i ``fullkomligt stånd''. Roten ålades att bygga upp fähuset, som var förfallet. Ladan skulle påtimras för att bli lika hög som fähuset. 1 tunnland åker hade roten samfällt vid torpet. Ett 2 kappland stort kålland fanns. Brunnen nyttjades samfällt med Matts Gunnar.


\jhbold{Torpets innehavare:}
\begin{center}
  \begin{tabular}{l p{0.8\textwidth}}
    \hline
    1796--\allowbreak 1808 & Johan Simonsson Wärn, \textborn 1772 i Lillkyro, hustru Anna Christina Henricsdotter, \textborn 1770 i Kelviå, 4 barn. Deltog i 1808-09 års krig, dog i fält. Anna Christina Wärn fick underhåll ur pensionsfonden för soldatänkor från 1812. Hon dog 1838. \\
    1791--\allowbreak 1795 & Carl Johan Lindbom, \textborn 1758 i Gamlakarleby, gift med pigan Anna Mattsdotter från Vörå, omgift med en soldatänka från Rödsö. 6 barn, 4 fosterbarn Tidigare plats Koskeby i Vörå. Johan befordrades till korpral 1791 och fick transport till detta torp. År 1795 fick han transport till nr 128 i Kyrkoby (Forsby). Deltog i Gustaf III:s krig 1788-90. \\
    1790--\allowbreak 1790 & Henric Ahlberg, död i fält. \\
    1779--\allowbreak 1790 & Johan Petter Unger, \textborn 1760 i Sverige, hustru Anna Ersdotter, \textborn 1758 i Kovjoki. Deltog i Gustaf III:s krig 1788-90. Dog på fältlasarett 1790. \\
    1774--\allowbreak 1779 & Eric Österdahl, \textborn 1753 i Nykarleby, hustru Caren Eliedotter, \textborn 1747 i Härmä, 6 barn, varav sonen Elias var soldat vid Kongl Wasa Regemente. Erics nästa placering var Wuoskoski. Deltog i Gustaf III:s krig. \\
    1773--\allowbreak 1774 & Petter Österdahl, transporterades redan 1774 till torp 115 Bröms i Munsala under namnet Vigg. \\
    1768--\allowbreak 1773 & Nils Österdahl, \textborn 1747 i Uppland, Magdalena Nilsdotter förklarades som hustru vid meddelandet om Österdahls död. Nils Österdahl dog under kommendering vid Sveaborg. Han hade 1771 fått 24 spö för subordinationsbrott. \\
    1749--\allowbreak 1763 & Hindrich Westerdahl/Österdahl, \textborn 1713 i Jakobstad, hustru Lisa Henricsdotter Tunnridar, \textborn 1708 i Pedersöre, 13 barn. Deltog i Pommerska kriget. \\
    1743--\allowbreak 1748 & Anders Knutsson Westerdahl, \textborn ca 1708, hustru Ewa Jonasdotter, \textborn 1704 i Lappajärvi, 7 barn, varav sonen Anders var trumslagare vid Livkompaniet. Han lär vara en av de Pommern-veteraner som hämtat hem och odlat potatis. Anders dog under byggkommendering vid Sveaborg. \\
    1734--\allowbreak 1742 & Mårten Mårtensson Westerdahl, \textborn ca 1697 i Livland, hustru Lisa Johansdotter, gift andra gången med Brita Eliaedotter. 3 barn. Deltog i Hattarnas krig 1741-42. \\
    \hline
  \end{tabular}
\end{center}


%%%
% [house] Präntagälon
%
\jhhouse{Präntagälon}{10:}{Gunnar}{15}{121b}


%%%
% [occupant] Gunnar
%
\jhoccupant{Gunnar}{\jhname[Johannes]{Gunnar, Johannes} \& \jhname[Johanna]{Gunnar, Johanna}}{1913--\allowbreak 1935}
Johannes och Johanna Gunnar (Gunnar, gård 125) bodde ev. i denna gård efter att ha överlämnat hemmanet åt sönerna Anders och Joel. Oklart när denna gård rivits, Ruben Eklöf, \textborn 1927, kom ihåg att det funnits en liten gård här.

Johannes \textdied 08.02.1922  ---  Johanna \textdied 20.04.1935



%%%
% [house] Åfält
%
\jhhouse{Åfält}{10:92}{Gunnar}{15}{22, 22a}


\jhhousepic{237-05817.jpg}{Mats Jakobsson och Malin Wiik}

%%%
% [occupant] Jakobsson \& Wiik
%
\jhoccupant{Jakobsson \& Wiik}{\jhname[Mats]{Jakobsson \& Wiik, Mats} \& \jhname[Malin]{Jakobsson \& Wiik, Malin}}{2016--}
Mats Jakobsson, \textborn 03.01.1980 på Tollikko och Malin Wiik, \textborn 11.05.1989 köpte huset i mars 2016 av Göran Gunnar och inledde en del ändringsarbeten på lägenheten. Mats är utbildad elektriker. Malin är utbildad sömmerska. Båda arbetar som ekologiska morots- och potatisodlare.

Till Mats intressen hör skidåkning. Han har deltagit i flera Vasa-lopp och ett år varit bästa finländare samt tävlat på FM-nivå i många år. Till Malins intressen hör handarbete.


%%%
% [occupant] Gunnar
%
\jhoccupant{Gunnar}{\jhname[Göran]{Gunnar, Göran}}{1985--\allowbreak 2016}
Göran Gunnar, \textborn 29.07.1960 på Gunnar hemman har sambott med Helena Välimäki, samlevnaden upphörde. Göran förlovade sig med Marja Porre och bor nu i Härmä.

Barn: Johnny, \textborn 09.08.1990, arbete på Östmans gård i Vörå, bor i Vasa

Göran är ekonomiemagister och arbetar på Företagstjänst Adviser. Göran uppförde bostadsbyggnaden 1985, år 1986 byggdes ekonomibyggnaden. År 2016 sålde han gården.



%%%
% [house] Renfält
%
\jhhouse{Renfält}{10:86}{Gunnar}{15}{23, 23a}


\jhhousepic{245-05827.jpg}{Yvonne och Heikki Holappa/hyrs av Ida och Kim Holmström}

%%%
% [occupant] Holappa
%
\jhoccupant{Holappa}{\jhname[Yvonne]{Holappa, Yvonne} \& \jhname[Heikki]{Holappa, Heikki}}{2008--}
Gården, byggd 1928, ägs idag av Yvonne, \textborn 26.03.1966 på Gunnar, gift med Heikki Holappa, \textborn 06.05. 1968. Lägenheten köpte de av av Ruth Fränti, f. Gunnar och Ulla Kujala, f. Gunnar. Bostaden är uthyrd sedan november 2008 och där bor nu:

Kim Holmström, \textborn 30.03.1979, gift med Ida Hermans, \textborn 22.02.1980.
\begin{jhchildren}
  \item \jhperson{\jhname[Rasmus]{Holappa, Rasmus}}{23.05.2006}{}
  \item \jhperson{\jhname[Oliver]{Holappa, Oliver}}{05.02.2008}{}
  \item \jhperson{\jhname[Edith]{Holappa, Edith}}{28.12.2011}{}
\end{jhchildren}


%%%
% [occupant] Gunnar
%
\jhoccupant{Gunnar}{\jhname[Lennart]{Gunnar, Lennart} \& \jhname[Maria]{Gunnar, Maria}}{1942--\allowbreak 1994}
Lennart, \textborn 26.07.1915 på Gunnar hemman, gift med Maria Laita, \textborn 10.03.1917 i Pelkkala.
\begin{jhchildren}
  \item \jhperson{\jhbold{\jhname[Rolf]{Gunnar, Rolf}}}{24.12.1939, jordbrukare}{}
  \item \jhperson{\jhname[Sven]{Gunnar, Sven}}{05.04.1941, maskinskötare}{}
  \item \jhperson{\jhname[Lars]{Gunnar, Lars}}{05.09.1944, tekniker}{}
  \item \jhperson{\jhname[Ruth]{Gunnar, Ruth}}{14.11.1950}{}, kontorist, g. Fränti
  \item \jhperson{\jhname[Stig]{Gunnar, Stig}}{20.10.1953, bankdirektör}{}
  \item \jhperson{\jhname[Ulla]{Gunnar, Ulla}}{13.07.1956, merkonom, g. Kujala}{}
\end{jhchildren}

\jhhousepic{Gunnar123-Lennart.jpg}{Lennart framför gården från 1928}

År 1942 övertog de gården efter Lennarts föräldrar. Lennarts far hade byggt gården 1928, ekonomibyggnaderna 1932. Lennart köpte sin första traktor 1937, en Fordson med stålhjul. Under kriget rådde det brist på bränsle, så han monterade ett gengasaggregat på traktorn. Det användes i många år. År 1952 köptes den första gummihjulsförsedda Ferguson traktorn. År 1959 säljer Lennart 1/6 mantal av Gunnar hemman till sonen Rolf (se Gunnar, nr 27).

Lennart hade många förtroendeuppdrag. Han var bl.a ledamot i Jeppo kommunalstyrelse, medlem i kommunfullmäktige samt nämndeman i tingsrätten, långvarig styrelseordförande fr. 1954 i Jeppo Kraft Alg samt styrelsemedlem i Jeppo Sparbank 1946--1981.

Lennart \textdied 10.04.1994  ---  Maria \textdied 11.01.1999


%%%
% [oldhouse] Renfält	Gamla gården
%
\jholdhouse{Renfält	Gamla gården}{10:86}{Gunnar}{15}{123}


\jhhousepic{Gunnar123-Anders.jpg}{Anders Gunnar byggde gården 1928}

%%%
% [occupant] Gunnar
%
\jhoccupant{Gunnar}{\jhname[Anders]{Gunnar, Anders} \& \jhname[Anna]{Gunnar, Anna}}{1867--\allowbreak 1907, 1885--\allowbreak 1942}
Anders Gunnar, \textborn 19.10.1865. Första hustru var Lovisa Gustafsdr, \textborn 29.10.1867 på Heikfolk. Hon dog 02.12.1906. Anders ingick nytt äktenskap med Anna Eriksdotter Nybyggar, \textborn 17.08.1880 på Back.
\begin{jhchildren}
  \item \jhperson{\jhname[Henrik]{Gunnar, Henrik}}{04.09.1892}{17.08.1892}
  \item \jhperson{\jhname[Anders]{Gunnar, Anders}}{11.08.1893}{14.08.1893}
  \item \jhperson{\jhname[Ester]{Gunnar, Ester}}{09.09.1894}{}, g. Eklöv (Gunnar 125)
  \item \jhperson{\jhname[Hilda]{Gunnar, Hilda}}{20.12.1899}{}, g. Bro (Gunnar 30)
  \item \jhperson{\jhname[Helga]{Gunnar, Helga}}{20.11.1905}{}, g. Enlund
  \item \jhperson{\jhname[Joel]{Gunnar, Joel}}{02.12.1906}{16.12.1906}
  \item \jhperson{\jhname[Ellen]{Gunnar, Ellen}}{14.07. 1909}{}, g. Sandvik
  \item \jhperson{\jhname[Signe]{Gunnar, Signe}}{12.07.1911}{}, g. Lundvik
  \item \jhperson{\jhname[Anders]{Gunnar, Anders}}{06.10.1913}{27.10.1913}
  \item \jhperson{\jhname[Johannes]{Gunnar, Johannes}}{06.10.1913}{16.10.1913}
  \item \jhperson{\jhbold{\jhname[Lennart]{Gunnar, Lennart}}}{26.07.1915}{}
\end{jhchildren}

Av Anders Eliasson köpte Anders Gunnar år 1885 en lägenhet på Gunnar, år 1893 fick han överta sina föräldrars lägenhet. Han byggde år 1928 en ny bostadsbyggnad, varvid den gamla byggnaden revs, och byggde ekonomibyggnaderna år 1932.	Anders var ett flertal år ledamot i Jeppo församlings boställsnämnd, Jeppo andelsmejeri, Svenska Folkpartiets lokalavdelning samt direktionen för Gunnar folkskola.

Anders \textdied 07.05.1944  ---  Anna \textdied 25.04.1965


%%%
% [occupant] Gunnar
%
\jhoccupant{Gunnar}{\jhname[Henrik]{Gunnar, Henrik} \& \jhname[Sanna]{Gunnar, Sanna}}{1852--\allowbreak 1893}
Henrik Henriksson Gunnar, \textborn 21.03.1839 på Gunnar hemman. Första hustru var Sanna Gustafsdr, \textborn 30.06.1844 i Köykkäri. Henrik var på arbetförtjänst några år i Amerika. Efter hemkomsten skiljde sig Henrik från sin första fru och ingick nytt äktenskap med Sanna Lisa Johansdr, \textborn 04.07.1835 på Knock hemman.
\begin{jhchildren}
  \item \jhperson{\jhname[Sanna]{Gunnar, Sanna}}{13.08.1861, g. Gunell (Ruotsala 121 )}{}
  \item \jhperson{\jhname[Sofia]{Gunnar, Sofia}}{08.02.1864, g. Sundqvist}{}
  \item \jhperson{\jhbold{\jhname[Anders]{Gunnar, Anders}}}{19.10.1865}{}
  \item \jhperson{\jhname[Anna Lovisa]{Gunnar, Anna Lovisa}}{31.08.1869, g. Nylind (Böös 152)}{}
  \item \jhperson{\jhname[Henrik]{Gunnar, Henrik}}{02.08.1871}{}
\end{jhchildren}

År 1852 fick Henrik hälften av föräldrarnas lägenhet, 1866 köpte han kusinen Johan Erikssons lägenhet, 21.04.1876 inropade han på auktion Johan Gustavsson Kaukos lägenhet efter att denne blivit ihjälstångad av en tjur året innan. Den sistnämnda lägenheten sålde han år 1872 till Elias Männikkö, senare Gunnar. Den 13.07.1868 sålde han till Johan och Brita Hannula den lägenhet som han köpt av Johan Eriksson ``utom	husen vid mangården'' (nuvarande Eklöv/Lund lägenhet).

Henrik \textdied 02.04.1917  ---  Sanna Lisa \textdied 17.01.1909


%%%
% [occupant] Roos
%
\jhoccupant{Roos}{\jhname[Samuel]{Roos, Samuel} \& \jhname[Maria]{Roos, Maria}}{1852--\allowbreak 1868}
Samuel Jöransson Roos, \textborn 09.01.1827 på Mietala, gifte sig med Maria Henriksdotter, \textborn 07.02.1834 på Gunnar hemman.
\begin{center}
  \begin{tabular}{l l l l l l}
    Henrik & \textborn 04.02.-53 & \textdied 1878 & Anna & \textborn 12.01.-55 & \textdied 1870 \\
    Sanna & \textborn 27.02.-57 & \textdied 1862 & Maja C & \textborn 11.03.-59, g Forsman & \\
    Samuel & \textborn 15.08.-61 & (Gröt 104) & Sanna & \textborn 18.12.-63, g Forsman & \\
    Georg & \textborn 24.09.-66 & \textdied 1878 & Lisa B & \textborn 16.08.-69 & \textdied 1869 \\
    Anna & \textborn 21.01.-71 & \textdied 1887 & Johannes & \textborn 30.05.-75 & \textdied 1896 \\
  \end{tabular}
\end{center}

Samuel kom från ett fattigt hem, där föräldrarna var tvungna att sälja sitt hem på Mietala. Barnen fick tidigt börja arbeta för att få mat för dagen. Samuel arbetade som dräng hos Johan Jakob Elenius på Smedsbacka.	Giftermålet innebar en stor förändring. Samuel och	Maria fick hälften av Marias föräldrars hemman på Gunnar år 1852. Samuel var redan då nämndeman och hade många uppdrag. Troligen bodde de i Marias föräldrahem fram till år 1868. Samuel och Maria köpte år 1860  1/8 mantals lägenhet på Tollikko av brodern Gabriel Jöransson. Senare, år 1868, sålde de hälften av densamma till 	Gabriel och andra hälften till Anders Gustaf Eriksson Tollikko och flyttade  till Ruotsala 118. Hemmansdelen på Gunnar såldes till Isak Thomasson Jungarå (Isak Elenius' farfar, Gunnar 124).


%%%
% [occupant] Eriksson
%
\jhoccupant{Eriksson}{\jhname[Henrik]{Eriksson, Henrik} \& \jhname[Sanna, Karolina]{Eriksson, Sanna, Karolina}}{ --1852}
Henrik Eriksson, \textborn 03.09.1803 på Gunnar hemman, gift med bndr Susanna Davidsdotter, \textborn 11.12.1803 på Bärs. Hon avled 06.10.1847. Henrik gifte om sig med bonddottern Karolina Isaksdotter, \textborn 11.03.1828 på Back hemman.
\begin{jhchildren}
  \item \jhperson{\jhname[Caisa]{Eriksson, Caisa}}{1829}{}
  \item \jhperson{\jhname[David]{Eriksson, David}}{1831}{}
  \item \jhperson{\jhbold{\jhname[Maria]{Eriksson, Maria}}}{1834}{}
  \item \jhperson{\jhname[Erik]{Eriksson, Erik}}{1836}{}
  \item \jhperson{\jhbold{\jhname[Henrik]{Eriksson, Henrik}}}{1839}{}
  \item \jhperson{\jhname[Gustaf]{Eriksson, Gustaf}}{1841}{}
  \item \jhperson{\jhname[Sanna]{Eriksson, Sanna}}{1846}{}
  \item \jhperson{\jhname[Katarina]{Eriksson, Katarina}}{1847}{}
  \item \jhperson{\jhname[Isak]{Eriksson, Isak}}{1850}{}
\end{jhchildren}

Henrik och Karolina överlät år 1852 hälften av sitt hemman till sonen Henrik, den andra hälften till dottern Maria. Av Henriks och Sannas gemensamma barn dog alla förutom Henrik och Maria som små. Henriks far Erik Eliasson Nystrand, \textborn 30.03.1765 på Keppo, är anfader till Lennart Gunnar, som senare brukade samma hemman.

Henrik \textdied 14.05.1852  ---  Karolina \textdied ca 1878



%%%
% [house] Gunnar
%
\jhhouse{Gunnar}{10:90}{Gunnar}{15}{24, 24a}


\jhhousepic{247-05829.jpg}{Bengt och Maire Elenius}

%%%
% [occupant] Elenius
%
\jhoccupant{Elenius}{\jhname[Bengt]{Elenius, Bengt}}{1975--}
Bengt Elenius, \textborn 05.02.1948 på Gunnar hemman, gifte sig år 1967 med Sirkka-Liisa Ievanen, \textborn 28.07.1947 i Kajana; frånskilda år 1988. Bengt gifte om sig 01.07.2009 med Maire Palo, \textborn 05.03.1948. Bostadshuset är byggt av Bengts far Uno år 1962. Ekonomibyggnaden är från 1958.
\begin{jhchildren}
  \item \jhperson{\jhname[Kim]{Elenius, Kim}}{09.05.1968 (Silvast 53)}{}
  \item \jhperson{\jhname[Niclas]{Elenius, Niclas}}{26.04.1973, har egen mekanisk verkstad, Mekanix}{}
  \item \jhperson{\jhname[Malin]{Elenius, Malin}}{15.01.1981, kanslist vid Nykarleby kraftverk}{}
\end{jhchildren}

Bengt övertog år 1975 föräldrarnas hemman på Gunnar. Det omfattade då 35 ha, varav 17 ha odlad jord. Vid sidan av jordbruket har Bengt haft tjänst på lantbrukscentralen i Jeppo och på Monäs Boden i Munsala, han har studerat till skogsmaskinschaufför och sedan 2008 har han arbetat som busschaufför.


%%%
% [occupant] Elenius
%
\jhoccupant{Elenius}{\jhname[Uno]{Elenius, Uno} \& \jhname[Aune]{Elenius, Aune}}{1962--\allowbreak 1975}
Uno Elenius, \textborn 08.09.1912 på Gunnar hemman, gifte sig 15.10.1939 med Aune Sipponen, \textborn 23.06.1916 på Pelkkala, som då hörde till Ruotsala i Jeppo. Närmare uppgifter om familjen under gård nr 124.


%%%
% [oldhouse] Gamla gården
%
\jholdhouse{Gamla gården}{10:90}{Gunnar}{15}{124}


%%%
% [occupant] Elenius
%
\jhoccupant{Elenius}{\jhname[Uno]{Elenius, Uno} \& \jhname[Aune]{Elenius, Aune}}{1945--\allowbreak 1962}
Uno Elenius, \textborn 08.09.1912 på Gunnar hemman, gifte sig 15.10.1939 med Aune Sipponen, \textborn 23.06.1916 på Pelkkala, som då hörde till Ruotsala i Jeppo.
\begin{jhchildren}
  \item \jhperson{\jhname[Pär]{Elenius, Pär}}{29.02.1940, (Tollikko 2)}{}
  \item \jhperson{\jhname[Tor]{Elenius, Tor}}{19.09.1941, täckdikningsentreprenör (Fors 98)}{}
  \item \jhperson{\jhname[Nils]{Elenius, Nils}}{31.08.1943, arbetat på KWH Mirka, (Romar 28)}{}
  \item \jhperson{\jhbold{\jhname[Bengt]{Elenius, Bengt}}}{05.02.1948}{}
\end{jhchildren}

Uno övertog 1945 hälften av föräldrarnas hemman på Gunnar. I boken ``Österbottens bebyggelse'', som utkom 1965, uppges att gården har 8 kor, 7 ungdjur och 2 får samt traktordrift. Makarna skötte även från 1950 telefoncentralen på Gunnar. Uno hade många kommunala förtroendeuppdrag, bl.a i skoldirektionen och som ordförande i socialnämnden. På sin ålders höst bodde makarna i Silvast, bl.a. Fors nr 98.

Uno \textdied 18.11.1996  ---  Aune \textdied 30.12.2014


\jhhousepic{Gunnar124-EleniusU.jpg}{Gamla gården, nr 124}

%%%
% [occupant] Elenius
%
\jhoccupant{Elenius}{\jhname[Isak]{Elenius, Isak} \& \jhname[Alina]{Elenius, Alina}}{1901--\allowbreak 1945}
Uno och Aune Elenius (se ovan) bodde åren 1939--\allowbreak 1941 med Unos föräldrar, Isak och Alina Elenius. Isak Mattsson, senare med efternamnet Elenius, \textborn 21.12.1867 på Jungarå, gift med Anna Lovisa Samuelsdotter, \textborn 27.12.1881. Hon dog i lungsot 25.04.1903. Deras son Joel var då 4 månader gammal.

År 1911 ingick Isak nytt äktenskap med mejerskan Alina Boholm, \textborn 31.12.1880, från Kronoby.
\begin{jhchildren}
  \item \jhperson{\jhname[Joel]{Elenius, Joel}}{22.12.1902}{}, till Kanada 1926, senare Sverige
  \item \jhperson{\jhbold{\jhname[Uno]{Elenius, Uno}}}{08.09.1912}{}
  \item \jhperson{\jhbold{\jhname[Bertel]{Elenius, Bertel}}}{25.09.1914 (Mietala 18)}{}
  \item \jhperson{\jhname[Hjördis]{Elenius, Hjördis}}{16.11.1916}{01.04.1938}
  \item \jhperson{\jhname[Olof]{Elenius, Olof}}{19.08.1920}{19.04.1942}, stupade vid Svir
  \item \jhperson{\jhname[Ines]{Elenius, Ines}}{24.05.1924 (Gunnar 26)}{}
\end{jhchildren}

Isak fick överta föräldrarnas lägenhet på Gunnar mot sytning åt dem samt utlösning åt yngsta brodern Johan. Lägenheten omfattade 70 ha, varav 30 ha var odlad jord. Isak var intresserad av läsning och var en trofast medlem i Svenska folkskolans vänner. Han hade många förtroendeuppdrag, men det var ändå främst som bonde som han utförde sitt livsverk. Det är många hektar kyttlandsmark han uppodlat, bränt mossa av, lerslagit och fått att bära skördar. Detta berättar han om i sin dagbok från 1889--\allowbreak 1896. Han berättar också om  hur grannarna på Gunnar bytte ägor, hur man tillsammans dikade, satte upp gärdesgårdar. I dagboken finns också utgifter och inkomster från tjärobruk. Var tjärdalen funnits nämns	inte.

Isak \textdied 13.03.1962  ---  Alina \textdied 24.02.1969


%%%
% [occupant] Isaksson
%
\jhoccupant{Isaksson}{\jhname[Matts]{Isaksson, Matts} \& \jhname[Kajsa]{Isaksson, Kajsa}}{1873--\allowbreak 1901}
Matts Isaksson, \textborn 10.01.1843 på Jungarå. Liksom brodern lämnade han bort efternamnet Elenius. Han gifte sig med Kajsa Simonsdotter,	\textborn 21.09.1844 på Jungar.
\begin{jhchildren}
  \item \jhperson{\jhbold{\jhname[Isak]{Isaksson, Isak}}}{21.12.1867}{}
  \item \jhperson{\jhname[Katarina]{Isaksson, Katarina}}{26.01.1871, g. Sjö (Böös 150)}{}
  \item \jhperson{\jhname[Anna Lovisa]{Isaksson, Anna Lovisa}}{14.09.1873, g. Nyström (Tollikko 101)}{}
  \item \jhperson{\jhname[Maria Vilhelmina]{Isaksson, Maria Vilhelmina}}{18.02.1876}{18.02.1876}
  \item \jhperson{\jhname[Ida Maria]{Isaksson, Ida Maria}}{12.06.1878, gift med Alfred Sandberg}{}
  \item \jhperson{\jhname[Hilda]{Isaksson, Hilda}}{20.04.1881, gift med Emil Sandberg}{}
  \item \jhperson{\jhname[Matts Joel]{Isaksson, Matts Joel}}{08.02.1884 (Tollikko 3)}{}
  \item \jhperson{\jhname[Johan Vilhelm]{Isaksson, Johan Vilhelm}}{06.04.1890 (Böös 143)}{}
\end{jhchildren}

År 1860 hade Matts' far Isak Elenius köpt 1/8 mantals lägenhet med åtföljande hushållskvarn i forsen nedanför mangården på Tollikko hemman och år 1867 en 47/384 mantals lägenhet på Gunnar. Dessa två fick Matts och Kajsa överta år 1873 och de flyttade då till Gunnar. Av odlad jord fanns då ännu inte mycket på någondera lägenheten. Genom träget arbete angreps odlingsbar mark med storgräfta och spade i händerna så att den odlade jorden småningom blev över 30 ha på Gunnar och 50 ha på Tollikko.

Matts Gunnar \textdied 02.11.1911  ---  Kajsa \textdied 16.03.1934


%%%
% [occupant] Eriksson
%
\jhoccupant{Eriksson}{\jhname[Henrik]{Eriksson, Henrik} \& \jhname[Albertina]{Eriksson, Albertina}}{1850--\allowbreak 1867}
Henrik Eriksson, \textborn 26.03.1826 på Gunnar hemman, gift med bndr Maria Albertina Johansdotter Elenius, \textborn 09.09.1828 på Jungarå.
\begin{jhchildren}
  \item \jhperson{\jhname[Susanna]{Eriksson, Susanna}}{1847}{}
  \item \jhperson{\jhname[Erik]{Eriksson, Erik}}{1850}{}
  \item \jhperson{\jhname[Johannes]{Eriksson, Johannes}}{1853}{}
  \item \jhperson{\jhname[Anna Sofia]{Eriksson, Anna Sofia}}{1856}{}
  \item \jhperson{\jhname[Henrik]{Eriksson, Henrik}}{1859}{}
  \item \jhperson{\jhname[Henrik]{Eriksson, Henrik}}{1861}{}
  \item \jhperson{\jhname[Gustaf]{Eriksson, Gustaf}}{1862}{}
  \item \jhperson{\jhname[Gustaf]{Eriksson, Gustaf}}{1864}{}
  \item \jhperson{\jhname[Albertina]{Eriksson, Albertina}}{1867}{}
  \item \jhperson{\jhname[Maria Matilda]{Eriksson, Maria Matilda}}{1871}{}
\end{jhchildren}

Henriks föräldrar ägde tillsammans med Henrik en större lägenhet på Gunnar. 1850 övertog Henrik och Albertina hälften av hans föräldrars hemman. År 1853 köpte Henrik tillsammans med brodern Johan 1/8 mantals lägenhet på Tollikko, men bytte samma år ut den mot 13/96 mtl på Gunnar. Enligt Johannes Kronlund rådde mycken ohälsa i hemmet och gården kunde inte hållas, utan måste säljas. År 1867 köpte bonden Isak Elenius hemmanet.

Henrik \textdied 07 augusti i Nykarleby  ---  Maria \textdied 05 december 1916 på Kuddnäs kommunalhem.



%%%
% [house] Lund
%
\jhhouse{Lund}{10:57}{Gunnar}{15}{25, 25a-c}


\jhhousepic{246-05828.jpg}{Anette och Kari Lund}

%%%
% [occupant] Lund
%
\jhoccupant{Lund}{\jhname[Anette]{Lund, Anette} \& \jhname[Kari]{Lund, Kari}}{1997--}
Anette Eklöv, \textborn 19.01.1968, gifte sig 24.07.1993 med Kari Lund, \textborn 01.08.1969 i Vasa. Kari är eltekniker och har arbetat på Jeppo Kraft Andelslag samt som underhållsingenjör vid Mirka. Anette är merkant till utbildningen. Hon 	har arbetat på Nykarleby bokföringsbyrå, Nyko Frys kontor, som skolgångsbiträde i Purmo, arbetar nu i Jeppo som eftisledare vid Jeppo-Pensala skolas eftis.
\begin{jhchildren}
  \item \jhperson{\jhname[Gabriella]{Lund, Gabriella}}{17.03.1994, studerar till kock vid Optima}{}
  \item \jhperson{\jhname[Fredrika]{Lund, Fredrika}}{23.01.1996, studerar till ingenjör vid Novia, Vasa}{}
  \item \jhperson{\jhname[Benjamin]{Lund, Benjamin}}{11.07.2001, studerande}{}
\end{jhchildren}

Gården har tillhört samma släkt sedan 1800-talet. År 1990 köpte Anette och brodern Bengt huset och tog över jordbruket. Föräldrarna Ragnar och Siv bodde kvar i huset till år 1997, då Anettes man, Kari, köpte Bengts andel av huset. Efter en brand renoverades gården och Anette och Kari flyttade in med sin familj.

År 2002 tog Anette över Bengts del av jordbruket. Huvudnäringen på gården är nu spannmålsodling och hö för familjens 2 hästar och 5 får. År 2003 gjordes en ombyggnad av huset till 1 ½ plan.


%%%
% [occupant] Eklöv
%
\jhoccupant{Eklöv}{\jhname[Ragnar]{Eklöv, Ragnar} \& \jhname[Siv]{Eklöv, Siv}}{1977--\allowbreak 1997}
Ragnar Eklöv, \textborn 10.06.1927, gift med Siv Julin, \textborn 09.10.1937. Ragnar och Siv bedrev jordbruk och boskapsskötsel på gården. Se gård 125. De flyttade år 1997 in i Edna Eklövs hus som hyresgäster och bodde kvar där till år 2002, då de köpte lägenhet vid Gökbrinken.

Ragnar \textdied 01.01.2004  ---  Siv \textdied 13.03.2013


%%%
% [oldhouse] Eklöv Gamla gården
%
\jholdhouse{Eklöv Gamla gården}{10:57}{Gunnar}{15}{125}


\jhhousepic{Gunnar125-RagnarEklov.jpg}{Vy över Eklövs gård, nr 125}

%%%
% [occupant] Eklöv
%
\jhoccupant{Eklöv}{\jhname[Ragnar]{Eklöv, Ragnar} \& \jhname[Siv]{Eklöv, Siv}}{1950--\allowbreak 1977}
Ragnar Eklöv, \textborn 10.06.1927, gift med Siv Julin, \textborn 09.10.1937. Ragnar och Siv bedrev jordbruk och boskapsskötsel på gården. År 1950 övertog Ragnar den andra halvan av sina föräldrars lägenhet på Gunnar hemman. Han byggde en ny ekonomibyggnad samma år. Efter giftermålet bodde familjen först i den gamla bondgården, men byggde år 1977 den gård som nu finns på lägenheten.
\begin{jhchildren}
  \item \jhperson{\jhname[Bengt]{Eklöv, Bengt}}{15.06.1961, enhetschef på Cash-in Konsulting Ab i Vasa}{}
  \item \jhperson{\jhbold{\jhname[Anette]{Eklöv, Anette}}}{19.01.1968}{}
  \item \jhperson{\jhname[Kristine]{Eklöv, Kristine}}{28.12.1973}{}, gift Paulin, familjedagvårdare i Vörå
\end{jhchildren}

Den gamla gården revs 1986. I boken Svenska Österbottens bebyggelse, som  kom ut 1965, uppräknas 11 kor, 12 ungdjur och 4 får på gården. År 1978 köpte Ragnar och Siv brodern Joels del av hemmanet. Huset blev kvar i Edna Eklövs ägo. Ragnars intresse för trav och hästar upptog så småningom en allt större del av arbetet på gården.


%%%
% [occupant] Eklöv
%
\jhoccupant{Eklöv}{\jhname[Anders]{Eklöv, Anders} \& \jhname[Ester]{Eklöv, Ester}}{1913--\allowbreak 1949}
Anders Gustav Eklöv, \textborn 29.09.1889, gift med Ester Andersdotter Gunnar, \textborn 09.09.1894 på Gunnar hemman. Anders och Ester övertog ena halvan av hans föräldrars lägenhet på Gunnar. De förstorade den något genom inköp, så att deras jordbruk kom att omfatta 57 ha, varav 27 ha var odlad jord. Brodern Joel fick andra halvan av hemmanet. Anders var kommunfullmäktig några år och ledamot i styrelsen för Jungar andelsmejeri.
\begin{jhchildren}
  \item \jhperson{\jhname[Signe]{Eklöv, Signe}}{1914}{1930}
  \item \jhperson{\jhname[Johannes]{Eklöv, Johannes}}{1914}{}, (Romar 34)
  \item \jhperson{\jhname[Anders]{Eklöv, Anders}}{1916}{1916}
  \item \jhperson{\jhbold{\jhname[Joel]{Eklöv, Joel}}}{1917 (Gunnar 33)}{}
  \item \jhperson{\jhname[Birger]{Eklöv, Birger}}{1919}{26.09.1941}, stupade i Poloviina
  \item \jhperson{\jhname[Ellen]{Eklöv, Ellen}}{1921, g. Warleman}{}
  \item \jhperson{\jhname[Agnes]{Eklöv, Agnes}}{1922, g. Ahlskog}{}
  \item \jhperson{\jhname[Ragnhild]{Eklöv, Ragnhild}}{1925, g. Grönlund}{}
  \item \jhperson{\jhbold{\jhname[Ragnar]{Eklöv, Ragnar}}}{1927}{}
  \item \jhperson{\jhname[Hellin]{Eklöv, Hellin}}{1928, g. Andersson}{}
  \item \jhperson{\jhname[Edna]{Eklöv, Edna}}{1932, g. Skrifvars}{}
  \item \jhperson{\jhname[Gunhild]{Eklöv, Gunhild}}{1935, g. Jansson}{}
\end{jhchildren}

Anders \textdied 13.10.1969  ---  Ester \textdied 08.09.1979


%%%
% [occupant] Gunnar
%
\jhoccupant{Gunnar}{\jhname[Johan]{Gunnar, Johan} \& \jhname[Johanna]{Gunnar, Johanna}}{1881--\allowbreak 1913}
Johannes Gunnar, \textborn 21.06.1859 på Södergård hemman, gift med Johanna Jungarå, \textborn 05.05.1861.
\begin{jhchildren}
  \item \jhperson{\jhbold{\jhname[Johannes]{Gunnar, Johannes}}}{26.07.1879 (Mietala 118)}{}
  \item \jhperson{\jhname[Ida Johanna]{Gunnar, Ida Johanna}}{16.05.1882, gift Strengell}{}
  \item \jhperson{\jhname[Anna Lovisa]{Gunnar, Anna Lovisa}}{16.05.1882, gift Rönnqvist}{}
  \item \jhperson{\jhname[Sanna Sofia]{Gunnar, Sanna Sofia}}{18.11.1884}{08.10.1917}
  \item \jhperson{\jhname[Hilda Maria]{Gunnar, Hilda Maria}}{17.07.1887, gift Kronlund (Gunnar 121)}{}
  \item \jhperson{\jhbold{\jhname[Anders]{Gunnar, Anders}} Gustav}{29.09.1889}{}
  \item \jhperson{\jhname[Hilma Katarina]{Gunnar, Hilma Katarina}}{10.03.1892}{10.03.1909}
  \item \jhperson{\jhbold{\jhname[Joel]{Gunnar, Isak Joel}}}{18.09.1894 (Gunnar 20)}{}
  \item \jhperson{\jhname[Ester Emilia]{Gunnar, Ester Emilia}}{17.11.1896}{01.12.1896}
  \item \jhperson{\jhname[Ester Irene]{Gunnar, Ester Irene}}{19.05.1899, gift Andersén}{}
  \item \jhperson{\jhname[Ellen Emilia]{Gunnar, Ellen Emilia}}{20.09.1902}{20.01.1905}
  \item \jhperson{\jhname[Ellen Emilia]{Gunnar, Ellen Emilia}}{05.05.1905}{18.05.1908}
\end{jhchildren}

Hemmanet på Gunnar hade Johannes far köpt år 1868. Johannes var enda sonen och fick 1881 överta den nya hemgården, som var ca 70 ha, varav endast ca 9 ha åker. Under hans tid uppodlades stora ängs- och mossmarker så att den odlade arealen fyrdubblades. I köpekontraktet ingick sytning till föräldrarna Johan och Brita samt förbindelse att ``föda och kläda afhändarenes fostersöner Gustaf och Johan Johanssöner Gunnar till dess de genomgått skriftskola...''. Deras far hade blivit ihjälstångad av en tjur, modern dött tidigare. I kontraktet nämns också en tjärdal, men var denna fanns, vet vi inte idag. Den bondgård som revs år 1986 var byggd av Johannes år 1892.

Johan \textdied 08.02.1922  ---  Johanna \textdied 20.04.1935


%%%
% [occupant] Hannula
%
\jhoccupant{Hannula}{\jhname[Johan]{Hannula, Johan} \& \jhname[Brita]{Hannula, Brita}}{1868--\allowbreak 1881}
Johan Erik Johansson Hannula, \textborn 20.04.1818 i Larsmo, vigd 1857 med änkan Brita Carlsdotter, \textborn 11.03.1821
\begin{jhchildren}
  \item \jhperson{\jhname[Maria Lovisa Johansdotter]{Hannula, Maria Lovisa Johansdotter}}{06.11.1847}{}
  \item \jhperson{\jhname[Caisa Sofia]{Hannula, Caisa Sofia}}{24.07.1858}{1858}
  \item \jhperson{\jhbold{\jhname[Johannes Johan]{Hannula, Johannes Johan}}}{21.06.1859}{}
  \item \jhperson{\jhname[Caisa Sofia]{Hannula, Caisa Sofia}}{1860}{1864}
  \item \jhperson{\jhname[Brita Kaisa]{Hannula, Brita Kaisa}}{01.08.1863}{20.07.1866}
\end{jhchildren}

Familjen Hannula kom från Larsmo och hade bott på Mietala sedan år 1846. De sålde hemmanet på Mietala år 1868 och köpte samma år 4/64 mantal på Gunnar, som Johan Eriksson Gunnar och hustrun Kaisa Johansdotter innehaft. Dessa flyttade med barnen till Hausjärvi och senare till Tammerfors. Johan Eriksson, vars farfar var Erik Eliasson Nystrand, var där skriven som banvakt. År 1869 köpte Johan och Brita 13/384 mantal av Henrik Henriksson Gunnar.

Brita \textdied 13.12.1904  ---  Johan \textdied 1881--\allowbreak 1886



%%%
% [house] Löte
%
\jhhouse{Löte}{10:40}{Gunnar}{15}{26, 26a}


\jhhousepic{244-05826.jpg}{Margit och Pekka Nikkanen}

%%%
% [occupant] Nikkanen
%
\jhoccupant{Nikkanen}{\jhname[Margit]{Nikkanen, Margit} \& \jhname[Pekka]{Nikkanen, Pekka}}{1984--}
Margit Broo, \textborn 06.03.1958 i Jeppo, gifte sig 11.04.1986 med Pekka Nikkanen, \textborn 27.08.1957 i Suonenjoki. Pekka är bilmekaniker, Margit arbetar på Mirka-Oravais.
\begin{jhchildren}
  \item \jhperson{\jhname[Jani]{Nikkanen, Jani}}{20.10.1986, montör}{}
  \item \jhperson{\jhname[Jukka]{Nikkanen, Jukka}}{22.08.1989, vvs-montör}{}
\end{jhchildren}

Margit och Pekka köpte huset 17.09.1984. I januari 1985 var renoveringen samt tilläggsbyggnaden färdig.


%%%
% [occupant] Elenius
%
\jhoccupant{Elenius}{\jhname[Ines]{Elenius, Ines}}{1945--\allowbreak 1984}
Ines Elenius, \textborn 24.05.1924, \textdied 05.01.2002. Hon utbildade sig till sömmerska, men blev sjuk redan i unga år. Hon skaffade sig stickmaskin, gick kurs i Tammerfors och började sticka åt folk. De första åren på Löte höll sig Ines och hennes föräldrar med några får, främst för ullens skull.

Vid bärtider såg man Ines mor, Alina, nästan varje dag fara till skogen. Hon sålde sedan bärskörden. Hon var också intresserad av fiske och tog sig ner för den branta åbacken ännu de sista levnadsåren. Ines flyttade till en radhuslägenhet på Åkervägen år 1978. Hon sålde huset 1984.

Isak \textdied 13.03.1962  ---  Alina \textdied 24.02.1969


%%%
% [occupant] Elenius
%
\jhoccupant{Elenius}{\jhname[Isak]{Elenius, Isak} \& \jhname[Alina]{Elenius, Alina}}{1941--\allowbreak 1945}
Isak, \textborn 21.12.1867 i Jeppo, gift med Alina Boholm, \textborn 31.12.1880 i Kronoby.

Barn: se närmare under Gunnar nr 24.

Isak och Alina köpte den 1 februari 1941 Söderstrand hemman av Hugo och Irene Ahlfors. Där byggde Isak en gård, dit han flyttade tillsammans med Alina och dottern Ines. År 1945 säljer Isak och Alina Elenius nämnda lägenhet till dottern Ines 	``med rättighet att få använda ett rum + kök under sin livstid''.



%%%
% [house] Renfelt
%
\jhhouse{Renfelt}{10:99}{Gunnar}{15}{27, 27a-c}


\jhhousepic{243-05825.jpg}{Rolf och Dorthy Gunnar / Yvonne och Heikki Holappa }

%%%
% [occupant] Holappa
%
\jhoccupant{Holappa}{\jhname[Yvonne]{Holappa, Yvonne} \& \jhname[Heikki]{Holappa, Heikki}}{1995--}
Yvonne Gunnar, \textborn 26.03.1966, merkonom, gift med Heikki Holappa, \textborn 06.05.1968, agrolog. År 1995 övertog Heikki och Yvonne Holappa lägenheten efter Yvonnes föräldrar, som bor kvar i huset. Yvonne och Heikki bor i Alahärmä.
\begin{jhchildren}
  \item \jhperson{\jhname[Tony]{Holappa, Tony}}{30.12.1991}{}
  \item \jhperson{\jhname[Annika]{Holappa, Annika}}{09.08.1993}{}
  \item \jhperson{\jhname[Robin]{Holappa, Robin}}{30.11.1995}{}
  \item \jhperson{\jhname[Linda]{Holappa, Linda}}{19.03.1998}{}
  \item \jhperson{\jhname[Ellen]{Holappa, Ellen}}{25.09.2007}{}
\end{jhchildren}

Yvonne arbetar på POP banken i Härmä, Heikki bygger hus till försäljning. Lägenheten består av 60 ha odlad jord och 50 ha skog. Huvudnäring på gården är potatisodling.


%%%
% [occupant] Gunnar
%
\jhoccupant{Gunnar}{\jhname[Rolf]{Gunnar, Rolf} \& \jhname[Dorthy]{Gunnar, Dorthy}}{1960--\allowbreak 1995}
Rolf Gunnar, \textborn 24.12.1939 på Gunnar hemman, gift med Dorthy Westin, \textborn 09.04.1939 på Grötas hemman.
\begin{jhchildren}
  \item \jhperson{\jhname[Göran]{Gunnar, Göran}}{29.07.1960, ekonomiemagister (Gunnar 22)}{}
  \item \jhperson{\jhbold{\jhname[Yvonne]{Gunnar, Yvonne}}}{26.03.1966}{}
\end{jhchildren}

År 1960 övertog Rolf och Dorthy en del av hans föräldrars lägenhet och några år senare resten (Gunnar, gård nr 23). En ny bostadsbyggnad uppfördes år 1960 samt ekonomibyggnad år 1961. Rolf har bl.a varit ledamot i hälsovårdsnämnden samt ordförande i Jeppo skogsvårdsförening.



%%%
% [house] Norrbacka
%
\jhhouse{Norrbacka}{10:6}{Gunnar}{15}{28, 28a}


\jhhousepic{239-05819.jpg}{Gustav Broo}

%%%
% [occupant] Broo
%
\jhoccupant{Broo}{\jhname[Gustav]{Broo, Gustav}}{1987--}
Gustav Broo,  \textborn 20.05.1959 i Jeppo, gifte sig 03.07.1982 med Anna-Maija Porre, \textborn 1962 i Alahärmä. Gustav Broo är jordbrukare och bor med sin familj i Alahärmä. Gården på Norrbacka, som numera står tom, uppfördes år 1961 av Gustavs farfar Jakob Broo.
\begin{jhchildren}
  \item \jhperson{\jhname[Arto]{Broo, Arto}}{26.01.1986}{}
  \item \jhperson{\jhname[Sonja]{Broo, Sonja}}{22.03.1988}{}
  \item \jhperson{\jhname[Tiina]{Broo, Tiina}}{14.03.1991}{}
  \item \jhperson{\jhname[Veikka]{Broo, Veikka}}{27.01.2004}{}
\end{jhchildren}


%%%
% [occupant] Broo
%
\jhoccupant{Broo}{\jhname[Rudolf]{Broo, Rudolf}}{1979--\allowbreak 1987}
Gustavs farbror Rudolf Tobias Broo, \textborn 14.04.1930 på Norrbacka, bodde i gården till 07.08.1987, då han dog. Rudolf var ogift. Han hade olika kortvariga arbetsförhållanden, i skogsarbete, på byggen etc.


%%%
% [occupant] Broo
%
\jhoccupant{Broo}{\jhname[Jakob]{Broo, Jakob} \& \jhname[Hilda]{Broo, Hilda}}{1961--\allowbreak 1979}
Gustavs farfar, Jakob Broo, \textborn 09.01.1886 i Jeppo, uppförde gården år 1961. Närmare uppgifter om Jakob och Hilda, se gård nr 131.


%%%
% [oldhouse] Norrbacka Gamla gården
%
\jholdhouse{Norrbacka Gamla gården}{10:6}{Gunnar}{15}{128}


%%%
% [occupant] Saikkonen
%
\jhoccupant{Saikkonen}{\jhname[Alexander]{Saikkonen, Alexander} \& \jhname[Sanna Lisa]{Saikkonen, Sanna Lisa}}{1918--\allowbreak 1944}
Alexander Karlsson Saikkonen, \textborn 10.9.1881 i Jyväskylä, gift med Sanna Lisa Mariasdotter, \textborn 05.10.1882 i Lehtimäki
\begin{jhchildren}
  \item \jhperson{\jhname[Vilho Elis]{Saikkonen, Vilho Elis}}{02.09.1908}{06.03.1927}, (Sannas son)
  \item \jhperson{\jhname[Olga Alina]{Saikkonen, Olga Alina}}{11.06.1910}{}
  \item \jhperson{\jhname[Otto Alexander]{Saikkonen, Otto Alexander}}{03.12.1911}{}, (år 1937 till Jyväskylä)
  \item \jhperson{\jhname[Kalle Johannes]{Saikkonen, Kalle Johannes}}{31.03.1914}{12.06.1914}
  \item \jhperson{\jhname[Tyyne Saima Elina]{Saikkonen, Tyyne Saima Elina}}{24.02.1916 i Laukaa}{}
  \item \jhperson{\jhname[Kalle Johannes]{Saikkonen, Kalle Johannes}}{1919}{}, (21.05.1935 till Jyväskylä)
  \item \jhperson{\jhname[Ida Maria]{Saikkonen, Ida Maria}}{1921}{}, (till Jyväskylä 1936)
\end{jhchildren}

Alexander med hustru och ett barn kom år 1909 från Gamlakarleby till Ruotsala i Jeppo. De flyttade någon gång efter 1918 till Norrbacka backstuguområde, som hörde till Tollikko, men ägdes av bönderna på Gunnar. Gården fanns på samma tomt som nuvarande gård nr 28 och revs av Jakob Broo.

Alexander hade före år 1925 köpt Edvard Mattssons gård på Hilli, flyttat den till Norrbacka och använt den som stall. Torde vara samma uthusbyggnad som nu finns kvar. År 1920 behandlades Alexander Saikkonens anhållan om att få lösa in backstuguområdet, men avslogs eftersom kontraktet blivit upprättat med Nikolai Taipaleenaho, som befann sig i Amerika. År 1922 togs ärendet upp igen i legonämnden. Alexander hade då skaffat fullmakt av Nikolai Taipaleenaho och nu godkändes ansökan. Alexander betalade 160 mark för det 4 kappland stora legoområdet. Bönderna som tills dess ägt området var Isak Elenius, Anders Gunnar, Elias och Johan Gunnar.

Alexander \textdied 09.12.1929  ---  Sanna Lisa \textdied 17.06.1944


%%%
% [occupant] Taipaleenaho
%
\jhoccupant{Taipaleenaho}{\jhname[Nikolai]{Taipaleenaho, Nikolai} \& \jhname[Aina]{Taipaleenaho, Aina}}{1887--\allowbreak 1909}
Backstugusittare Nikolai Mariasson Taipaleenaho, \textborn 29.06.1872 i Lehtimäki, gift med Aina Elin Johanna Jansson, \textborn 02.05.1875 i Jeppo.
\begin{jhchildren}
  \item \jhperson{\jhname[Johan Rudolf]{Taipaleenaho, Johan Rudolf}}{11.10.1897}{}
  \item \jhperson{\jhname[Anders Gustaf]{Taipaleenaho, Anders Gustaf}}{12.01.1899}{14.01.1900}
  \item \jhperson{\jhname[Gustaf Lennart]{Taipaleenaho, Gustaf Lennart}}{10.02.1901}{}
\end{jhchildren}

Familjen bodde endast några år i Jeppo, de kom från Ylihärmä och flyttade till Amerika 1904--\allowbreak 1909. Torde ha varit här ännu år 1904, eftersom det fanns en anteckning att Aina då besökt nattvarden.



%%%
% [house] Norrback
%
\jhhouse{Norrback}{11:8}{Tollikko}{15}{29, 29a-b}

Obs! att lägenheten står på Tollikko, men ingår på Gunnar, karta 15.

\jhhousepic{240-05820.jpg}{Eino Broos dödsbo}

%%%
% [occupant] Broo Einos
%
\jhoccupant{Broo Einos}{\jhname[dödsbo]{Broo Einos, dödsbo}}{2000--}
Gården står idag tom och används främst som fritidsbostad. Den ägs av Kustaa Eino Broos dödsbo, dvs Einos döttrar.


%%%
% [occupant] Broo
%
\jhoccupant{Broo}{\jhname[Eino]{Broo, Eino} \& \jhname[Aino]{Broo, Aino}}{1948--\allowbreak 2000}
Eino Broo, \textborn 20.09.1917 i Jeppo, gifte sig 12.10.1941 med Aino Finni, \textborn 18.05.1916 i Alahärmä.
\begin{jhchildren}
  \item \jhperson{\jhname[Eva]{Broo, Eva}}{11.05.1941, gift Pukkila}{}
  \item \jhperson{\jhname[Oili]{Broo, Oili}}{21.10.1945, gift Perä}{}
  \item \jhperson{\jhname[Ritva]{Broo, Ritva}}{20.10.1951, gift Nyberg}{}
  \item \jhperson{\jhname[Anneli]{Broo, Anneli}}{03.11.1955, gift Kalliosaari}{}
\end{jhchildren}
Eino och Aino hade bott på flere ställen i Silvast, bl.a i Samlingshuset, där Eino arbetade som gårdskarl. År 1947 revs Einos föräldrars hus på Norrbacka och 1948 byggde han nuvarande hus på samma plats och familjen flyttade från Silvast.

Både Eino och Aino arbetade på Keppo pälsfarm. Under några somrar (1955--\allowbreak 1960) hade Eino kioskförsäljning. Kiosken fanns invid landsvägen på Tollikko (karta 16, vid nr 2). Uthusen byggdes åren 1948--\allowbreak 1950 och huset grundrenoverades 1981.

Eino \textdied 16.03.2000  ---  Aino \textdied 14.04.2013

\jhpic{Gunnar29b-Eino Bro kiosk.jpg}{Eino Broos kiosk vid landsvägen på Tollikko; plats på karta 16}{0.5}

%%%
% [oldhouse] Norrback Gamla gården
%
\jholdhouse{Norrback Gamla gården}{11:8}{Tollikko}{15}{129}


%%%
% [occupant] Broo
%
\jhoccupant{Broo}{\jhname[Johannes]{Broo, Johannes} \& \jhname[Amanda]{Broo, Amanda}}{1916--\allowbreak 1947}
Johannes Gustavsson Broo, \textborn 15.03.1894 i Alahärmä, gifte sig 11.04.1915 med Amanda Saha, \textborn 09.10.1896, från Alahärmä.
\begin{jhchildren}
  \item \jhperson{\jhname[Lilja]{Broo, Lilja}}{05.11.1915, till Sverige}{}
  \item \jhperson{\jhbold{\jhname[Eino]{Broo, Eino}}}{20.09.1917}{}
  \item \jhperson{\jhname[Laina]{Broo, Laina}}{03.02.1921}{1925}
  \item \jhperson{\jhname[Eskil]{Broo, Eskil}}{30.06.1925, till Sverige}{}
\end{jhchildren}
Johannes Broo anhöll år 1920 om att få lösa in backstuguområdet, vars ägare var bönderna Joel Elenius och Johan Andersson Tollikko. Han löste in området och befriades från dagsverksskyldigheten. Huset revs 1947.

Amanda \textdied 04.05.1928  ---  Johannes \textdied 11.09.1970



%%%
% [occupant] Janson
%
\jhoccupant{Janson}{\jhname[Jakob]{Janson, Jakob} \& \jhname[Maria]{Janson, Maria}}{1890 – 1906}
Jakob Johansson Janson, f. 15.06.1837 i Munsala, gift med Maria Lovisa Johansdotter, \textborn 01.06.1838 i Munsala. Paret flyttade 1890 till Tollikko. Åren 1893-1895 torde Jakob varit i Amerika. År 1896 finns anteckning om att båda besökt nattvard. Maria Lovisa dör 1897. Jakob gifter om sig med Anna Sofia Yrjösdotter Ekola, \textborn 25.07.1869 i Alahärmä. Paret får två barn. Anna hade ett barn från tidigare.

Isak Elenius har i sin dagbok från 1893 skrivit följande: ``Norrback Jakobs hustru betalade tre mark femti penni till arende för norrback torp. Tre dagsverk om sommaren och ett om vintern...''. Troligen är det denna Jakob som bodde i Norrback torp.



%%%
% [occupant] Rönnqvist
%
\jhoccupant{Rönnqvist}{\jhname[Anders]{Rönnqvist, Anders} \& \jhname[Maria]{Rönnqvist, Maria}}{1870--\allowbreak 1892}
Anders Johan Rönnqvist, \textborn 1806 i Pedersöre, gift med Maja Lisa Johansdotter, \textborn 05.12.1814 i Pedersöre.
\begin{jhchildren}
  \item \jhperson{\jhname[Johan]{Rönnqvist, Johan}}{1837}{}
  \item \jhperson{\jhname[Maja Lena]{Rönnqvist, Maja Lena}}{1838}{}
  \item \jhperson{\jhname[Caisa Sofia]{Rönnqvist, Caisa Sofia}}{1841}{}
  \item \jhperson{\jhname[Anders]{Rönnqvist, Anders}}{1844}{}
  \item \jhperson{\jhname[Israel]{Rönnqvist, Israel}}{1847}{}
  \item \jhperson{\jhname[Anna]{Rönnqvist, Anna}}{1852}{}
  \item \jhperson{\jhname[Johanna]{Rönnqvist, Johanna}}{1855}{}
  \item \jhperson{\jhname[Lovisa]{Rönnqvist, Lovisa}}{1857}{}
\end{jhchildren}

De tre första barnen var födda i Pedersöre (Ytter-Purmo/Bur), de övriga på Kampinen i Jeppo. Pga avsaknad av kyrkböcker från denna tid förblir det oklart när familjen flyttade till Norrbacka Torp. Anders har benämningen torpare eller backstugukarl i kyrkböckerna, men i praktiken var han finsnickare. Han tillverkade bl.a predikstolen till Jeppo kyrka. Det påstås att han haft sin lilla flicka som modell, då han snidade änglahuvudet, ty likheten lär ha varit påfallande. Rönnqvist var också anlitad i Nykarleby stad före branden 1858. Ovanför apoteksdörren fanns esculapernas emblem, en slingrande orm, som Rönnqvist gjort. Den förstördes vid branden.

Anders hustru torde ha dött ca 1885. Då Anders dog var han änkling och troligen medellös. Den ``4 juni 1892 dog fattighjonet Anders Rönnqvist, 86 år, 3 månader, 21 dagar. Dödsorsak: ålderdom''.



%%%
% [occupant] Bjon
%
\jhoccupant{Bjon}{\jhname[Matts]{Bjon, Matts} \& \jhname[Sanna]{Bjon, Sanna}}{1856--}
Matts Mattsson Bjon, \textborn 04.02.1805 i Ytterjeppo, gift med Sanna Lisa Johansdotter, \textborn 04.12.1806 i Munsala. Inga gemensamma barn, men Sanna hade tre.
\begin{jhchildren}
  \item \jhperson{\jhname[Anna Andersdotter]{Bjon, Anna Andersdotter}}{1837}{}
  \item \jhperson{\jhname[Gustaf Andersson]{Bjon, Gustaf Andersson}}{1839}{}
  \item \jhperson{\jhname[Maria Andersdotter]{Bjon, Maria Andersdotter}}{1843}{}
  \item \jhperson{\jhname[Johan Andersson]{Bjon, Johan Andersson}}{1846}{}
\end{jhchildren}
På en auktion 1856 efter avlidne torparen Anders Johansson Tollikko hade Matts Mattsson det högsta anbudet. Han blev ägare till hela	Norrbacka torp samt hälften av Forshaga Torp. Oklart i vilket av dessa torp  familjen bott.


%%%
% [occupant] Bärs
%
\jhoccupant{Bärs}{\jhname[Carl]{Bärs, Carl} \& \jhname[Maria]{Bärs, Maria}}{1853--\allowbreak 1870}
Carl Gustafsson Bärs eller Norrbacka, \textborn 14.06.1823 i Ytterjeppo, gift med Maria Isaksdotter, \textborn 03.05.1823 på Back i Överjeppo.
\begin{jhchildren}
  \item \jhperson{\jhname[Maja Lovisa]{Bärs, Maja Lovisa}}{1847}{}
  \item \jhperson{\jhname[Isak]{Bärs, Isak}}{1851}{}
  \item \jhperson{\jhname[Brita Caisa]{Bärs, Brita Caisa}}{1853}{}
  \item \jhperson{\jhname[Gustaf]{Bärs, Gustaf}}{1855}{}
  \item \jhperson{\jhname[Carolina]{Bärs, Carolina}}{1858}{}
  \item \jhperson{\jhname[Johan]{Bärs, Johan}}{1862}{}
\end{jhchildren}

Punkt ett i Matts Mattssons torparkontrakt: ``Hela Norrbacka Torp med all hus och jord som Karl Gustafsson det innehaft...'', vilket skulle tyda på att denna familj bott på Nörrbackan. Familjen flyttade sedan till Gunnar hemman, ev har de flyttat till en annan stuga på Norrbacka. De övriga stugorna hörde/hör till Gunnar hemman.



%%%
% [house] Sandås
%
\jhhouse{Sandås}{10:38}{Gunnar}{15}{30, 30a}


\jhhousepic{238-05818.jpg}{Nuvarande ägare, Kjell Eklund}

%%%
% [occupant] Eklund
%
\jhoccupant{Eklund}{\jhname[Kjell]{Eklund, Kjell}}{1990--}
Kjell Eklund, \textborn 20.06.1967 i Jeppo, är nuvarande ägare till gården, som stått tom sedan Hilda på gamla dagar flyttade till dottern Gunvor på Stenbacka. Gården kom i Kjells ägo i samband med generationsskifte 1990.\jhvspace{}


%%%
% [occupant] Bro
%
\jhoccupant{Bro}{\jhname[Karl]{Bro, Karl} \& \jhname[Hilda]{Bro, Hilda}}{1934--\allowbreak 1990}
Hilda, \textborn 20.12.1899 i Jeppo, gift med Karl Bro, \textborn 29.10.1896 i Alahärmä. Karl kom som dräng till Hildas hemgård, men arbetade senare som mjölnare vid Tollikko kvarn samt vid järnvägen. Gården byggdes av Karl år 1934.

Barn: Gunvor, \textborn 10.06.1936, gift med Herbert Eklund på Stenbacka

Karl \textdied 30.03.1950  ---  Hilda \textdied 26.12.1993



%%%
% [house] Norrback
%
\jhhouse{Norrback}{10:11}{Gunnar}{15}{31, 31a}


\jhhousepic{241-05822.jpg}{Elof Broos dödsbo}

%%%
% [occupant] Broo Elofs
%
\jhoccupant{Broo Elofs}{\jhname[dödsbo]{Broo Elofs, dödsbo}}{1960--}
Elof Broo, \textborn 02.01.1934 på Norrbacka, gifte sig 1956 med Helena Kalliosaari, \textborn 27.02.1934 i Ylihärmä.

Elof övertog huset av sin far Jakob. Elof och Helena flyttade till Vasa 1956 där Elof arbetade på Bocks bryggeri. År 1957 flyttade de till Fagersta i Sverige. Elof arbetade i gruva.
\begin{jhchildren}
  \item \jhperson{\jhname[Ralf]{Broo Elofs, Ralf}}{14.02.1957 i Vasa, fabriksarbetare}{24.09.2001}
  \item \jhperson{\jhname[Margit]{Broo Elofs, Margit}}{06.03.1958 i Västanfors i Sverige (Gunnar 26)}{}
  \item \jhperson{\jhname[Gustav]{Broo Elofs, Gustav}}{20.05.1959 i Västanfors, Sverige, jordbrukare i Härmä}{}
  \item \jhperson{\jhname[Nils]{Broo Elofs, Nils}}{19.06.1961, i Jakobstad, byggnadsarbetare i Oravais}{}
\end{jhchildren}
År 1959 flyttade familjen till Jeppo, där Elof arbetade som byggnadsarbetare. Helena bor idag i radhus i Silvast och gården står tom. Elof byggde gården 1960, varvid den äldre gården revs. På tomten står ännu ladugården, som Jakob byggde. Elof förlängde den senare. På tomten finns också en lång stenmur, som Jakob byggde med hjälp av sina äldsta söner. Årtalet 1926 har hackats in på en av stenarna.

Elof \textdied 02.12.2004


%%%
% [oldhouse] Norrback Gamla gården
%
\jholdhouse{Norrback Gamla gården}{10:11}{Gunnar}{15}{131}


\jhhousepic{Gunnar131-JakobBroo.jpg}{Jakobs hus får ny granne 1960}

%%%
% [occupant] Broo
%
\jhoccupant{Broo}{\jhname[Jakob]{Broo, Jakob} \& \jhname[Hilda]{Broo, Hilda}}{1907--\allowbreak 1961}
Jakob Tobias Broo, \textborn 09.01.1886 i Jeppo, gifte sig 27.01.1907 med	Hilda Maria Nukala, \textborn 23.01.1889 i Alahärmä.

Jakob började sitt arbetsliv som dräng då han var 14 år, ett par år senare flyttade han till Jakobstad, där han bl.a arbetade som propslastare i hamnen. År 1906 flyttade han tillbaka till Jeppo, gifte sig och flyttade till Nörrbackan. Familjen bodde i ett hus, som var 16 m$^2$. Då Jakobs sjätte barn var fött, byggde han ut huset med en kammare, som var 4 x 4 m. Inga uppgifter finns om tidigare ägare till torpet.
\begin{jhchildren}
  \item \jhperson{\jhname[Aili Lovisa]{Broo, Aili Lovisa}}{18.10.1907}{21.10.1907}
  \item \jhperson{\jhname[Helmi Elisabet]{Broo, Helmi Elisabet}}{31.01.1909, till Kanada}{}
  \item \jhperson{\jhname[Jaakko Sanfrid]{Broo, Jaakko Sanfrid}}{03.10.1910 (Böös 151)}{}
  \item \jhperson{\jhname[Lilja Maria]{Broo, Lilja Maria}}{02.11.1912, till Sverige}{}
  \item \jhperson{\jhname[Lauri Johannes]{Broo, Lauri Johannes}}{04.05.1914 (Romar 31)}{}
  \item \jhperson{\jhname[Kalle Edvin]{Broo, Kalle Edvin}}{17.10.1915}{06.06.1922 (drunknade)}
  \item \jhperson{\jhname[Göta Emilia]{Broo, Göta Emilia}}{09.11.1917, till Sverige}{}
  \item \jhperson{\jhname[Anna Ester]{Broo, Anna Ester}}{01.10.1919, till Sverige}{}
  \item \jhperson{\jhname[Ellen Linnea]{Broo, Ellen Linnea}}{25.08.1921, g Åkerman i Munsala}{}
  \item \jhperson{\jhname[Kalle Edvin]{Broo, Kalle Edvin}}{17.08.1923, till Sverige}{}
  \item \jhperson{\jhname[Paul Robert]{Broo, Paul Robert}}{11.02.1926 (Fors 128)}{}
  \item \jhperson{\jhname[Rurik Wilhelm]{Broo, Rurik Wilhelm}}{11.02.1926 (Böös 51)}{}
  \item \jhperson{\jhname[Rakel Wikturia]{Broo, Rakel Wikturia}}{17.09.1927}{23.04.1956}
  \item \jhperson{\jhname[Rudolf Tobias]{Broo, Rudolf Tobias}}{14.04.1930 (Gunnar 28)}{}
  \item \jhperson{Ralf \jhbold{Elof}}{02.01.1934}{}
\end{jhchildren}
Jakob hade en tid ett litet jordbruk, men han var mest känd som slaktare. Jakob berättade inför sin 90-årsdag hur han cyklat med kalvar på pakethållaren. Det var nämligen vanligt förr att folk kallade på honom, när en ko kalvat. Man födde inte upp andra kalvar än de som skulle bli mjölkkor. Jakob slaktade på plats och kastade kalven sedan på pakethållaren. Köttet gick till Helsingfors och Gamlakarleby. Skinnen gick till Gamlakarleby. Jakob köpte alla slag av skinn, både från tama djur och vilda djur, ss ekorre, räv och t.o.m katt samt tagel från ko- och hästsvansar.

Efter torparlagens inträdande fick Jakob 1920 lösa in sin lägenhet på Norrbacka backstuguområde. Anders och Elias Gunnar varit ägare; 4 kappland av Elias mark och 3,3 kappland av Anders mark. Även den dagsverksskyldighet som Jakob haft till hemmansägarna upphörde.

Hilda \textdied 25.12.1972  ---  Jakob \textdied 02.11.1979



%%%
% [house] Ahlfors
%
\jhhouse{Ahlfors}{10:97}{Gunnar}{15}{32, 32a}


\jhhousepic{248-05830.jpg}{Kalle och Anki Häggblom}

%%%
% [occupant] Häggblom
%
\jhoccupant{Häggblom}{\jhname[Kalle]{Häggblom, Kalle} \& \jhname[Anki]{Häggblom, Anki}}{1986--}
Karl-Gustav Häggblom, \textborn 31.12.1959 på Gunnar hemman, gifte sig år 1983 med Anne-Kristine Nurmi, \textborn 02.10.1962 i Purmo. Karl-Gustav och Anne-Kristine övertog hemmanet 31.12.1985.

Karl-Gustav har varit pälsfarmare. Anki var hårfrisörska, men gick utbildning till närvårdare och arbetar nu på Nykarleby sjukhem.
\begin{jhchildren}
  \item \jhperson{\jhname[Jenny]{Häggblom, Jenny}}{15.07.1985, sömmerska}{}
  \item \jhperson{\jhname[Anna Josefin]{Häggblom, Anna Josefin}}{05.09.1988 (Silvast 44)}{}, sjukskötare
\end{jhchildren}


%%%
% [occupant] Häggblom
%
\jhoccupant{Häggblom}{\jhname[Paul]{Häggblom, Paul} \& \jhname[Alice]{Häggblom, Alice}}{1953--\allowbreak 1985}
Paul Häggblom, \textborn 28.11.1923 i Oravais, gift med Alice Ahlfors, \textborn 02.08.1923 på Grötas hemman.
\begin{jhchildren}
  \item \jhperson{\jhname[Kerstin]{Häggblom, Kerstin}}{08.03.1947, g. Nyby, pälsfarmare}{}
  \item \jhperson{\jhname[Håkan]{Häggblom, Håkan}}{16.09.1949}{05.06.1983}, försäljn.dir
  \item \jhperson{\jhname[Ralf]{Häggblom, Ralf}}{07.12.1953, dipl.ekon}{}
  \item \jhperson{\jhname[Leif]{Häggblom, Leif}}{27.07.1955, byggn.ing, pälsfarmare (Gunnar 34)}{}
  \item \jhperson{\jhbold{\jhname[Karl-Gustav]{Häggblom, Karl Gustav}}}{31.12.1959}{}
\end{jhchildren}

År 1953 övertog Paul och Alice hennes föräldrars lägenhet på Gunnar, vilken omfattade 54 ha, varav 19 ha var odlad jord. På 1960-talet hade man 9 kor, 8 ungdjur, 2 får, 4 svin. Paul byggde ny gård år 1967. Ekonomibyggnaden är från 1928 och uppförd av Alices far.

Alice dog 19.07.1979. Paul flyttade år 1986 till radhus på Åkervägen 7. År 2015 flyttade han till Hagalund. Han dog 07.06.2016.


%%%
% [oldhouse] Ahlfors Gamla gården
%
\jholdhouse{Ahlfors Gamla gården}{10:97}{Gunnar}{15}{132}


%%%
% [occupant] Häggblom
%
\jhoccupant{Häggblom}{\jhname[Paul]{Häggblom, Paul} \& \jhname[Alice]{Häggblom, Alice}}{1953--}
Paul Häggblom, \textborn 28.11.1923 i Oravais, gift med Alice Ahlfors, \textborn 02.08.1923 på Grötas hemman. Barn: Se gård nr 32.

Gården har tillhört samma släkt sedan 1850 och övertogs av Paul och Alice år 1953. Bostadsbyggnaden var uppförd 1850 i trä under pärttak med 4 rum och kök, moderniserad år 1935. Revs år 1967 då den nya gården blev färdig.


%%%
% [occupant] Ahlfors
%
\jhoccupant{Ahlfors}{\jhname[Hugo]{Ahlfors, Hugo} \& \jhname[Amanda]{Ahlfors, Amanda}}{1926--\allowbreak 1953}
Hugo Henriksson Ahlfors, \textborn 11.09.1899 på Grötas, gift 11.06.1922 med Amanda Gunnar, \textborn 10.01.1898 på Gunnar hemman.
\begin{jhchildren}
  \item \jhperson{\jhbold{\jhname[Alice]{Ahlfors, Alice}}}{02.08.1923}{}
  \item \jhperson{\jhname[Rut Gunda]{Ahlfors, Rut Gunda}}{21.10.1927}{14.08.1928}
\end{jhchildren}

Hugo och Amanda köpte den 26 januari 1926 av Amandas farfar Elias 0.1393 mantal av Gunnars hemman. Amanda Gunnar dog 06.07.1929. Hon hade i testamente önskat att brodern Axel, som vistades i Canada, skulle erhålla den s.k. Landsvägastugan och 2 tunnland jord. Därutöver skulle han inte behöva återbetala de resepengar som han fått. Dottern Lisi skulle ärva övriga delen av hemmanet. - Hugo dog 07.04.1954.


\jhhousepic{Gunnar132-EliasGunnar.jpg}{Elias Gunnar; ``Ella-gubben''}

%%%
% [occupant] Gunnar
%
\jhoccupant{Gunnar}{\jhname[Elias]{Gunnar, Elias} \& \jhname[Anna]{Gunnar, Anna}}{1872--\allowbreak 1926}
Elias Johansson Gunnar, \textborn 19.05.1840 på Fors, gift första gången 1865 med Anna Mattsdotter, \textborn 18.01.1840 på Lussi, \textdied 12.04.1902.
\begin{jhchildren}
  \item \jhperson{\jhname[Matts]{Gunnar, Matts}}{01.01.1866}{07.01.1913}
  \item \jhperson{\jhname[Johan]{Gunnar, Johan}}{10.09.1870}{06.03.1919}
  \item \jhperson{\jhname[Anders Gustav]{Gunnar, Anders Gustav}}{03.11.1877}{19.05.1902 (Gunnar 115)}
\end{jhchildren}

Elias och Anna fick år 1886 fastebrev på hemmanet, som var 107/768 dels mantal. De hade tidigare fått lagfart på 33/256 dels mantal. De hade år 1872 köpt 47/384 dels mantal av Anders Johansson Gunnar samt år 1876 köpt 13/384 dels mantal av Henrik Henriksson Gunnar. Ännu idag hör man talas om den legendariske ``Ella-gubben'', av vars efterlämnade föremål hembygdsgården fått en ansenlig del.

Elias gifte sig andra gången 19.10.1902 med änkan Kajsa Jakobsdotter Nylund, \textborn 20.05.1851 i Munsala, \textdied 10.11.1925. Kajsa hade två barn.
\begin{jhchildren}
  \item \jhperson{\jhname[Sanna Brita Henriksdotter]{Gunnar, Sanna Brita Henriksdotter}}{10.12.1881, flyttat till Alahärmä}{}
  \item \jhperson{\jhname[Katrina]{Gunnar, Katrina}}{28.02.1909}{}
\end{jhchildren}

Elias \textdied 15.11.1933


%%%
% [occupant] Johansson
%
\jhoccupant{Johansson}{\jhname[Anders]{Johansson, Anders} \& \jhname[Sanna]{Johansson, Sanna}}{1869--\allowbreak 1872}
Anders Johansson, \textborn 13.04.1826 på Kojola samt hustru Sanna And.dr, \textborn 27.01.1827 på Suorsholma.
\begin{jhchildren}
  \item \jhperson{\jhname[Lisa]{Johansson, Lisa}}{1856}{}
  \item \jhperson{\jhname[Sanna]{Johansson, Sanna}}{1859}{}
  \item \jhperson{\jhname[Anders]{Johansson, Anders}}{1864}{}
\end{jhchildren}

Anders och Sanna köpte denna hemmansdel 21 maj 1869. De kom från Alahärmä, men blev dock inte långvariga på Gunnar. Redan i april 1872 säljer de hemmanet till Elias, som vid den tidpunkten var torpare på Måtar.



%%%
% [oldhouse] Ahlfors
%
\jholdhouse{Ahlfors}{10:23}{Gunnar}{15}{115}


%%%
% [occupant] Gunnar
%
\jhoccupant{Gunnar}{\jhname[Axel]{Gunnar, Axel}}{1929--}
Denna stuga torde vara den s. k. landsvägastugan, som Amanda i sitt testamente nämner, då hon önskar att brodern Axel, som vistades i Canada, skall erhålla stugan. Amanda dog 06.07.1929. Det finns inga uppgifter om vem som eventuellt bott i stugan och när den revs.


%%%
% [occupant] Backlund
%
\jhoccupant{Backlund}{\jhname[Anders]{Backlund, Anders} \& \jhname[Maria]{Backlund, Maria}}{1893--\allowbreak 1929}
Backstugusittare Anders Backlund, \textborn 11.12.1860 på Jungar hemman, gift med Maria Lovisa Johansdotter, \textborn 18.02.1863.
\begin{jhchildren}
  \item \jhperson{\jhname[Anders Artur]{Backlund, Anders Artur}}{28.01.1897}{16.08.1904}
  \item \jhperson{\jhname[Karl Johan]{Backlund, Karl Johan}}{17.01.1899}{}
  \item \jhperson{\jhname[Erik Villiam]{Backlund, Erik Villiam}}{12.02.1900 (``Pränta Viljam'')}{}, Ruotsala 13
  \item \jhperson{\jhname[Hilda Maria]{Backlund, Hilda Maria}}{24.04.1902}{05.04.1905}
  \item \jhperson{\jhname[Anders Valfrid]{Backlund, Anders Valfrid}}{02.04.1905}{}
\end{jhchildren}

Anders är skriven som backstugusittare under Elias Gunnars hemman. Anders var timmerman, han flyttade från Jungar ev. redan 1893, den första anteckningen om nattvard är 1893. Anders tagit ut betyg till Amerika tre gånger, år 1888, 1892 och 1896. Anders dog 28.02.1905.

År 1929 flyttade änkan Maria Lovisa samt barnen Erik Villiam och Anders Valfrid till Jungar, Svartbacken. Maria Lovisa dog 22.09.1943.


%%%
% [occupant] Gunnar
%
\jhoccupant{Gunnar}{\jhname[Anders Gustav]{Gunnar, Anders Gustav}}{1897--\allowbreak 1902}
Anders Gustav Gunnar, \textborn 03.11.1877, gifte sig 1897 med Maria Sofia Pettersdotter Pensar, \textborn 14.06.1877 (Målar Petters dotter)
\begin{jhchildren}
  \item \jhperson{\jhbold{\jhname[Amanda]{Gunnar, Amanda}}}{10.01.1898}{}
  \item \jhperson{\jhname[Anders Viktor]{Gunnar, Anders Viktor}}{07.11.1899}{28.03.1918}
  \item \jhperson{\jhname[Axel Elias]{Gunnar, Axel Elias}}{14.01.1902}{}
\end{jhchildren}

Anders Gustav dog 19.05.1902. Vid bouppteckningen framkommer att familjen hade en ko, två får och två lamm. Enligt uppgift av släktingar arbetade Anders med att timra opp hus. Sonen Anders Viktor var en av fyra unga män från Jeppo, som stupade vid Tammerfors striderna 1918. Han hade deltagit i jägarutbildningen i Tyskland. Han var musikaliskt begåvad, spelade bl.a fiol.


%%%
% [occupant] Kaukos
%
\jhoccupant{Kaukos}{\jhname[Johan]{Kaukos, Johan} \& \jhname[Maria]{Kaukos, Maria}}{1869--\allowbreak 1877}
Johan Gustafsson Kaukos eller Gunnar, \textborn 05.09.1847, vigd 1869 med Maria Lovisa Johansdotter Gunnar, \textborn 12.02.1847 på Jungar. Majas broder var sexman på Jungar.
\begin{jhchildren}
  \item \jhperson{\jhname[Gustaf]{Kaukos, Gustaf}}{27.09.1869}{}
  \item \jhperson{\jhname[Johan]{Kaukos, Johan}}{24.03.1871}{}
  \item \jhperson{\jhname[Lovisa]{Kaukos, Lovisa}}{08.06.1872}{}
\end{jhchildren}

Johan och Maria köpte 18 maj 1869 av Isak Thomasson Jungarå 27/256 mantal på Gunnar. Isak hade köpt hemmanet av Henrik Eriksson Gunnar år 1867. Johan blev ihjälstångad av en tjur, han dog 01.12.1877. Hemmanet kom att köpas av Henrik Henriksson Gunnar. Hustrun torde ha dött tidigare. År 1874 säljer Johan på auktion egendom tillhörig de omyndiga barnen, dvs en häst, 4 kor, 2 kalvar och 10 får. Efter Johans död tas barnen om hand av grannarna, Johan och Brita Hannula. Johans familj torde ha bott i denna stuga.



%%%
% [house] Dahl
%
\jhhouse{Dahl}{10:55}{Gunnar}{15}{33, 33a}


\jhhousepic{250-05832.jpg}{Annikki Eklöf; Göran Eklöf}

%%%
% [occupant] Eklöf
%
\jhoccupant{Eklöf}{\jhname[Göran]{Eklöf, Göran}}{1994--}
Göran Eklöf, \textborn 10.01.1962, köpte år 1994 av Antero Bäckman  och Birgit Väisänen denna gård på Gunnar. År 2003 flyttade Görans föräldrar, Annikki och Ruben Eklöf till detta hus. Ruben dog 23.10.2015. Annikki bor ensam kvar i gården. Närmare uppgifter Gunnar nr 20.

1997--\allowbreak 2002: Ragnar och Siv Eklöv (Gunnar 25 ) bodde som hyresgäster i huset till år 2002, då de köpte lägenhet i radhuset Gökbrinken (Silvast 75).


%%%
% [occupant] Bäckman
%
\jhoccupant{Bäckman}{\jhname[Antero]{Bäckman, Antero}}{1988--\allowbreak 1994}
Antero Bäckman, \textborn 1958 i Ylistaro, gift med Birgit Väisänen, \textborn 1962 i Ylistaro. Efter skilsmässa köper Antero ifrågavarande gård och flyttar från Keppo, där familjen bodde.
\begin{jhchildren}
  \item \jhperson{\jhname[Berit]{Bäckman, Berit}}{1980 i Jeppo}{}
  \item \jhperson{\jhname[Tero]{Bäckman, Tero}}{1986 i Jeppo}{}
\end{jhchildren}


%%%
% [occupant] Dahlkvist
%
\jhoccupant{Dahlkvist}{\jhname[Lars]{Dahlkvist, Lars} \& \jhname[Ann-Christin]{Dahlkvist, Ann-Christin}}{1986--\allowbreak 1988}
Lars och Ann-Christin Dahlkvist, f. Levlin  köpte huset av Edna Eklöv år 1986. De sålde huset år 1988. Uppgifter om Lars och Ann-Christin och deras barn finns under Silvast nr 77.


%%%
% [occupant] Eklöv
%
\jhoccupant{Eklöv}{\jhname[Joel]{Eklöv, Joel} \& \jhname[Edna]{Eklöv, Edna}}{1948--\allowbreak 1975}
Joel Eklöv, \textborn 27.01.1917 på Gunnar, gift med Edna Westerlund, \textborn 03.06.1922 på Fors hemman (nr 396). År 1948 övertog Joel och Edna ena halvan av hans föräldrars lägenhet på Gunnar. Den omfattade 30 ha, varav 13 ha var odlad jord. Huvudnäring på gården var mjölkproduktion.

Bostadsbyggnaden är uppförd 1945 av Joels far, ekonomiebyggnaderna är från 1947. Efter Joels död bodde Ednas mor Amanda Westerlund tillsammans med Edna. År 1978 sålde Edna skogen och marken till brodern Ragnar. Efter Amandas död köpte Edna en radhuslägenhet i Silvast och flyttade dit (Bostad Ab Åkervägen).

Joel \textdied 21.08.1972  ---  Edna \textdied 02.06.1993


%%%
% [house] Ella
%
\jhhouse{Ella}{10:87}{Gunnar}{15}{34, 34a-b}


\jhhousepic{249-05831.jpg}{Leif och Birgitta Häggblom}

%%%
% [occupant] Häggblom
%
\jhoccupant{Häggblom}{\jhname[Leif]{Häggblom, Leif} \& \jhname[Birgitta]{Häggblom, Birgitta}}{1981--}
Leif, \textborn 27.07.1955 i Jeppo, gifte sig den 23.08.1980 med Birgitta Nymark, \textborn 22.03.1955 i Purmo. Familjens bostadsbyggnad uppfördes 1981, ekonomibyggnaderna 1980 (34a) och 1987 (34b). Birgitta är kontorist och Leif är pälsdjursfarmare. Leif har varit tidvis aktivt engagerad inom kommunalpolitiken.
\begin{jhchildren}
  \item \jhperson{\jhname[Johanna]{Häggblom, Johanna}}{22.06.1981}{}, ped.mag. språkbadslärare , Jakobstad
  \item \jhperson{\jhname[Martin]{Häggblom, Martin}}{21.04.1986}{}, dipl.ing vid Paptic Oy, Helsingfors
  \item \jhperson{\jhname[Mikael]{Häggblom, Mikael}}{29.09.1992}{}, dipl.ing.stud. i Helsingfors
\end{jhchildren}



%%%
% [house] SJ:s vaktstuga
%
\jhhouse{SJ:s vaktstuga}{10:}{Gunnar}{15}{133}


\jhhousepic{Gunnar117-Vaktstuga.jpg}{Vaktstugan vid Gunnar}

\jhbold{1891 – 1965}

``Första tiden från 1886 när banvakter anställdes, så var banvaktens granskningssträckas längd 5 km och den uträttades till fots. Efter det kom någon typ av spårgående sparkvagn, enligt banvakt V Wikström. Senare förlängdes granskningssträckan till 7 km, bangranskningen utfördes då med dressin. Banvakten hade 5-7 arbetare i arbete sommartid, vintertid något färre. När jag började var sträckan 7 km, den förlängdes dock då och då. Vid min pensionering var sträckan Härmä-Kronoby, inklusive Bennäs-Alholmen samt Kovjoki-Nykarleby. Jag hade dock inga arbetare. Av mig upptäckta fel och brister meddelade jag till den ansvarige banvakten, som i sin tur ansvarade för att adekvata reparationsåtgärder utfördes.'' (Paul Laxén)

På vaktstugetomterna fanns alltid en vaktstuga, som innefattade farstu, kök samt ett sovrum. I alla stugor var det högt till tak enligt järnvägens modell. Bastu/tvättstuga, fähus, vedlider, magasin och ``tuppen'' (utedass) fanns under samma tak. Det fanns utekällare med förrum, där man kunde elda om så behövdes. Dessutom fanns utrymme för dressinen och pumpdressin. Telefon fanns i samtliga banvaktsstugor. Allt vatten hämtades från en brunn.

Gunnar vaktstuga fanns på östra sidan om järnvägsspåret. Platsen är infogad på vidstående karta 15, nr 133, och bilden visar en välskött stuga med omgivning.

Följande banvakter har verkat och bott vid Gunnarskangan:
\begin{center}
  \begin{longtable}{l p{0.8\textwidth}}
    \hline
    1) & Anders Eriksson Almberg, \textborn 16.11.1868 --- \textdied 05.12.1936 (Grötas, 111). Anders började som banvakt vid Gunnarskangan 1891. Han slutade som banvakt 1929. \\

    2) & Jakob Vilhelm Jakobsson Wikström, \textborn 26.05.1891 --- \textdied 14.06.1969. Hustru Ester Sofia Isaksdotter, \textborn 12.09.1886. Barn: Else Viola, \textborn 1913, Ester Anna Alice, \textborn 1917, Jenny Emilia, \textborn 1920, Lennart Vilhelm, \textborn 1924. Jakob Vilhelm for till Nykarleby (Rijärv) och bodde där i banvaktsstugan. Familjen återvände till Jeppo 20.09.1922. \\

    3) & Tauno Reino Uksila, \textborn 1920 i Uleåborg, hustru Eeva Helena Viitala, \textborn 1922 i Riihimäki. Barn: Anja Eliina, \textborn 1940, Eeva Liisa, \textborn 1945, Olavi Väinö, \textborn 1947.  Familjen kom från Haapajärvi sommaren 1955, flyttade 27.12.1956 till Sääksmäki. \\

    4) & Kalle Hakala, hustru Helka. Barn: Ari. Inflyttade till Jeppo 1957, utflyttade 1962. \\

    5) & Paul Mack Laxén, \textborn 02.11.1931 (Grötas, nr 10). Familjen flyttade från banvaktsstugan vid Tapelbackan till Gunnarskangan 1962. Paul var sista banvakten, som bodde på Gunnar. Familjen flyttade år 1965 till Jeppo stationsområde. \\
    \hline
  \end{longtable}
\end{center}

Banvaktsstugan hyrdes ut till grävmaskinsförare, som arbetade åt järnvägen. De lastade grus på tåg, som transporterade gruset till järnvägen, där behov förelåg. Omkring ett tåg om dagen lastades (ca 30 vagnar). Denna typ av lastning skedde under några års tid.

Byggnaderna utauktionerades eftersom de ej mera behövdes. Matti Aromäki köpte vakthusbyggnaden.
