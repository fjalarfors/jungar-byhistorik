%%%
% [chapter] Skog, hemman Nr 1
%
\jhchapter{Skog, hemman Nr 1}

Före år 1831 hade hemmanet nr 32 i Nykarleby stads mantalslängd. Det gick då under namnet Skogsbyggare och nr 32 representerade dess plats i Överjeppo by och gränsade till Ytterjeppo by. Hemmanens antal längs i huvudsak älven i Överjeppo var 34 st, med Grötas som nr 1. Logiken i numreringen är för oss oklar, men fr.o.m. 1831 ändrar hemmanens mantalsnummer och Skogsbyggare får nu nr 1 och namnet ändras till enbart Skog. Hemmanet omfattade ½ mantal. Enligt släktforskarföreningen i Jakobstad innehades hemmanet mellan 1551--\allowbreak 1563 av Nils Larsson, 1565--\allowbreak 1607 av Mikkel Nilsson, 1609--\allowbreak 1633 av Mårten Mikkelsson, 1623--\allowbreak 1629 av Hans Hansson, 1629- ca.1653 av Carl Hindersson, 1657--\allowbreak 1673 av Matts Carlsson, 1675 av  Carl Mattsson, hu. Lisa Jönsdr., 1676--\allowbreak 1709 av Tomas Mattsson och hu 1 Karin Hansdr, hu 2 Lisa, medåbo bror Markus Mattsson 1677-97, 1710--\allowbreak 1713 av sonen Jakob Tomasson och 1723- av Hans Hansson, hu Maria.

Efter Stora ofreden genomförde Matthias Wörman år 1740 en lantmäteriförrättning och kartläggning längs ån med början från åmynningen i Nykarleby. På hans kartblad finns endast en gård registrerad. 1783 års mantalslängd av Nykarleby stad upptar också endast en innehavare med 5/18 mtl. Små åkertäppor var placerade i solskiftesform längs Romarbäcken. Det finns skäl att anta att Skogsbyggare (Skog) med sin placering en bit från älven var en bosättningsplats för backstugusittare under äldre tid. Från 1762 tilläts nämligen gifta legohjon att bygga backstugor och boningsrum på enskilda och samfällda ägor. Ändå har antalet bosatta på Skogsbyggare varit lågt. År 1807 fanns antecknat endast 4 personer och hemmanet innehades av personer från Romar.

Utmärkande för Skog, i ännu högre grad än för övriga hemman, är den mycket stora emigrationen från hemmansnummern. Stora syskonskaror emigrerade i sin helhet till USA och emellanåt följde också föräldrarna efter. Idag finns endast en (1) brukningsenhet kvar på hemmanet, som under hela 1900-talet hyste flera hemmansdelar. Däremot finns flera bostäder placerade på hemmanets mark.

Skog hemman omfattas av vidstående karta nr \jhbold{1}.


<--- se KARTA nr 1 --->


%%%
% [subsection] Lägenheter på Skog
%
\jhsubsection{Lägenheter på Skog}



%%%
% [house] Strandkanten
%
\jhhouse{Strandkanten}{1:43}{Skog}{1}{1+1a}


\jhhousepic{001-05522.jpg}{Kerstin Ljung-Sandvik och Sten Sandvik}

%%%
% [occupant] Ljung-Sandvik
%
\jhoccupant{Ljung-Sandvik}{\jhname[Kerstin]{Ljung-Sandvik, Kerstin} \& \jhname[Sandvik Sten]{Ljung-Sandvik, Sandvik Sten}}{1996--}
Kerstin Ljung Sandvik \textborn 13.06.1964 och maken Sten Sandvik \textborn 09.07.1958 köpte fastigheten 20.06.1996 av sterbhuset efter Harry Ljung. Makarna hade gift sig 12.11.1994

\begin{jhchildren}
  \item \jhperson{\jhname[Conny]{Ljung-Sandvik, Conny}}{12.03.1995}{}, är utbildad skogsmaskinförare
  \item \jhperson{\jhname[Fredrik]{Ljung-Sandvik, Fredrik}}{12.03.1995}{}, är utbildad kombinationsförare
\end{jhchildren}

Sten har arbetat på Jeppos största företag Mirka från 1992. Före det hade han arbete på Sjöholms minkfarm 1976--\allowbreak 1992.

Kerstin utbildade sig till Pedagogie magister och har haft flera tjänster. Hon startade som klasslärare i Ytterjeppo skola 1988--91. Därefter vidareutbildade hon sej 1991--92 och kom till Larsmo 1992 där hon stannade 1 år som lärare. Mellan åren 1993-96 fungerade hon som ambulerande speciallärare i Karleby för att från 1996 haft tjänsten som speciallärare i Larsmo.

Efter att Kerstin och Sten år 1996 köpt fastigheten och en del av hemmanets skog, såldes resten. Samma år revs den s.k. Morastugan och året därpå revs också ladugården.


%%%
% [occupant] Ljung
%
\jhoccupant{Ljung}{\jhname[Harry]{Ljung, Harry} \& \jhname[Inga]{Ljung, Inga}}{1959--\allowbreak 1996}
Harry Johannes Ljung, \textborn 05.02.1935 i Purmo, gifte sig 1955 med Inga Maria Lindström, \textborn 03.05.1933 i Oravais.

Harry utbildade sej tidigt till seminolog, ett yrke som på 50-talet var helt nytt inom den finska kreaturshållningen. Han fick 1956 jobb i östra Nyland, Strömfors, och familjen flyttade 1957 till Jakobstad. Fram till 1962 arbetade han som seminolog i Pedersöre och Larsmo. Därefter flyttade familjen till Jeppo där Harry och Inga övertog hemmanet av Harrys föräldrar Ellen och Johannes. Han fortsatte sitt arbete som uppskattad seminolog i Jeppo området fram till sin för tidiga död 1970. Den odlade jorden utarrenderades nu.

Makarna byggde 1964 ett nytt hus på gårdens tomt, men närmare ån än den gamla mangårdsbyggnaden. Denna s.k. Morastuga hade flyttats till platsen troligtvis 1905 och fungerat som släktens hem i två generationer.

Inga är utbildad barnskötare, men arbetade efter övertagandet av Harrys hemgård på jordbruket medan barnen växte upp. Efter makens död arbetade hon på Sundells bageri under 4 års tid. År 1974 började hon utbilda sej till barnträdgårdslärare och fick arbete inom barndagvården i Nykarleby där hon fungerade som administrativ ledare fram till sin pensionering år 1996 då hon flyttade till Jeppo centrum.

\begin{jhchildren}
  \item \jhperson{\jhname[Mikael]{Ljung, Mikael}}{06.05.1956}{}
  \item \jhperson{\jhname[Stefan]{Ljung, Stefan}}{20.08.1962}{}, bl.a. lärare på yrkesskolan i Närpes
  \item \jhperson{\jhbold{\jhname[Kerstin]{Ljung, Kerstin}}}{13.06.1964}{}
\end{jhchildren}

Harry Ljung \textdied 09.05.1970


\jhhousepic{Morastugan L.jpg}{Morastugan, nr 301}

%%%
% [occupant] Ljung
%
\jhoccupant{Ljung}{\jhname[Ellen]{Ljung, Ellen} \& \jhname[Johannes]{Ljung, Johannes}}{1939--\allowbreak 1959}
Ellen Alina Haga föddes 06.10.1907 och gifte sig  1934 med Paul Johannes Ljung, \textborn 24.05.1906 i Purmo. Efter giftermålet flyttade Ellen till Purmo där makarna bodde fram till den 26.04. 1939 då familjen flyttade till Jeppo för att där överta det hemman som Ellens föräldrar Johan och Hilda Haga den 11.04.1939 sålt till dem. Hemmanet var inte stort, drygt 10 ha odlad jord och något mera skog och på hemmanet odlades som på de flesta andra; hö, havre, råg och potatis. Genast efter kriget försöktes också med kalkonuppfödning vid sidan av de traditionella korna, men avslutades ganska snart.

En ny ladugård byggdes med den brist på byggnadsmaterial som var typiskt för efterkrigstiden. Johannes började för att stärka ekonomin fungera som ombudsman för Österbottens Kött till dess han insjuknade 1959 och Ellen fick en ännu större arbetsbörda.

\begin{jhchildren}
  \item \jhperson{\jhbold{\jhname[Harry]{Ljung, Harry}}}{05.02.1935}{09.05.1970}
  \item \jhperson{\jhname[Hans]{Ljung, Hans}}{15.01.1938}{2001}, i Sverige, verkat som lärare
  \item \jhperson{\jhname[Stig]{Ljung, Stig}}{07.11.1939}{}, Bl.a. chef för FPA:s byrå i Ekenäs och kommundirektör i Snappertuna
  \item \jhperson{\jhname[Greta]{Ljung, Greta}}{30.09.1941}{}, Bl.a. professor vid MIT-universitetet i Boston
  \item \jhperson{\jhname[Lisa]{Ljung, Lisa}}{11.11.1947}{}, Merkonom från handelsinstitutet i Karleby
\end{jhchildren}

Johannes \textdied 28.11.1968   ---    Ellen \textdied 28.05.1997


%%%
% [occupant] Haga
%
\jhoccupant{Haga}{\jhname[Hilda]{Haga, Hilda} \& \jhname[Johan]{Haga, Johan}}{1905--\allowbreak 1939}
Hilda Alina Skog, \textborn 15.05.1883, gifte sig den 06.08.1905 med Johan Mariasson Huhtala från Ylistaro. Han hette ursprungligen Juho Pikkutupa, men tog namnet Huhtala i samband med giftermålet med Hilda. Senare tog familjen namnet Haga för att deras barn skulle kunna gå i svensk skola. Johan, \textborn 05.04.1883 som faderlös, men sannolikt med ryskt påbrå.

Hilda var 8:e barnet till Anders Johansson Skog och hans första hustru Anna Kajsa Eriksdotter. När Hilda och Johan gifte sig 1905 utstyckades en hemmansdel från Skog hemman och bildade ett nytt hemman med bosättning vid älven dit en s.k. Morastuga flyttades. I äktenskapet föddes 12 barn:
\begin{jhchildren}
  \item \jhperson{\jhname[Johan Lennart]{Haga, Johan Lennart}}{12.03.1906}{18.07.1924}
  \item \jhperson{\jhbold{\jhname[Ellen Alina]{Haga, Ellen Alina}}}{06.10.1907}{28.05.1997}
  \item \jhperson{\jhname[Anders Evert]{Haga, Anders Evert}}{28.04.1909}{03.04.1969}
  \item \jhperson{\jhname[Edit Maria]{Haga, Edit Maria}}{18.04.1912}{}, till Karis
  \item \jhperson{\jhname[Gerda Katarina]{Haga, Gerda Katarina}}{10.11.1913}{29.12.1937}
  \item \jhperson{\jhname[Sylvi Linnea]{Haga, Sylvi Linnea}}{07.10.1915}{}, till Bromarv
  \item \jhperson{\jhname[Hilda Irene]{Haga, Hilda Irene}}{10.08.1917}{}, till Ekenäs
  \item \jhperson{\jhname[Elsa Anna Viola]{Haga, Elsa Anna Viola}}{06.01.1920}{}, till Sverige
  \item \jhperson{\jhname[Hjördis Valdine]{Haga, Hjördis Valdine}}{10.11.1921}{}, till Snappertuna
  \item \jhperson{\jhname[Erik Alfred]{Haga, Erik Alfred}}{14.05.1923}{04.12.2013}
  \item \jhperson{\jhname[Margit Susanna]{Haga, Margit Susanna}}{01.12.1925}{}, till Pernå
  \item \jhperson{\jhname[Johannes Evald]{Haga, Johannes Evald}}{04.12.1928}{}, till Snappertuna
\end{jhchildren}

Hilda Alina \textdied 19.01.1937  ---   Johan Mariasson \textdied 18.12.1943



%%%
% [house] Strandhagen
%
\jhhouse{Strandhagen}{1:57}{Skog}{1}{2}


\jhhousepic{002-05524.jpg}{Daniel Lindvall och Caroline Bäckstrand-Lindvall}

%%%
% [occupant] Lindvall
%
\jhoccupant{Lindvall}{\jhname[Daniel]{Lindvall, Daniel} \& \jhname[Caroline]{Lindvall, Caroline}}{2013-}
Daniel August Jakob Lindvall, \textborn 21.11.1979 i Nykarleby Forsby, gifte sig 14.02.2009 med Caroline Benita Louise Bäckstrand, \textborn 29.03.1983.

År 2012 köpte de delar av lägenheterna Ankar R:nr 1:54 och Haga R:nr 1:49 av Magnus Jungell och bildade lägenheten Strandhagen R:nr 1:57. Samma år påbörjades planeringen av ett nytt hus på den nybildade tomten. Utgångspunkten var ett av Teri-husens modeller, men efterhand utgicks från egen ritning och 2013 uppfördes huset, medan garaget byggdes året innan.

Daniel har utbildat sig till konditor. Han har arbetat 5 år på Nykarlebyföretaget Prevex och där avlagt yrkesexamen som plastmekaniker. Därefter har han också under 5 års tid arbetat på KWH-plast. 2010 övergick han till Mirka  och där genomgått en processutbildning och yrkesexamen inom kemi. För tillfället arbetar han nu på Mirkas avdelning i Jakobstad.

Caroline är utbildad massör och ergoterapeut. Efter examen och fram till 2007 arbetade hon på Hagalund i Nykarleby och sedan på Nykarleby sjukhem.

\begin{jhchildren}
  \item \jhperson{\jhname[Alizia Milja Louise]{Lindvall, Alizia Milja Louise}}{12.12.2009}{}
  \item \jhperson{\jhname[Leia Emilia Augustina]{Lindvall, Leia Emilia Augustina}}{01.06.2016}{}
\end{jhchildren}



%%%
% [oldhouse] Haga
%
\jholdhouse{Haga}{1:11}{Skog}{1}{300}

\jhhousepic{Haga Evert.jpg}{Evert och Agnes Haga, dottern Lisen g. Wikman}

%%%
% [occupant] Wikman
%
\jhoccupant{Wikman}{\jhname[Lisen]{Wikman, Lisen}}{1969--}
Lisen, \textborn 20.10.1944 i Kvevlax med efternamnet Kilvik. Hon adopterades av Evert och Agnes Haga som själva var barnlösa. Efter faderns död 1969 förblev egendomen ett dödsbo och jorden utarrenderades. Modern Agnes flyttade till Jakobstad 1977.

Efter detta årtal har ingen bott i huset, som förfallit och slutligen inköptes och revs 2013 av Magnus Jungell. Lisen gifte sig 05.07.1964 med Rolf Harry Wikman från Närpes, \textborn 30.09.1943. Familjen har bott i Närpes.


%%%
% [occupant] Haga
%
\jhoccupant{Haga}{\jhname[Evert]{Haga, Evert} \& \jhname[Agnes]{Haga, Agnes}}{1930-talet}
Evert Haga, \textborn 28.04.1909, gifte sig 07.07.1935  med Agnes Jungerstam, \textborn 15.04.1911. Lägenheten styckades ut från Skog skattehemman ägd av Hilda och Johan Haga (Huhtala) och bildade en  ny lägenhet på 1930-talet.

Till den nya tomten flyttades ett färdigt timrat hus från Holmen. Det syns på ett fotografi på sidan 282 i ``Historik över Jeppo''. Det var menat till ett hus åt bokhållaren på yllespinneriet vid Kiitola, \jhname[Johan Elis Sirén]{Sirén, Johan Elis}, men blev aldrig färdigt. Familjen flyttade 1922 till Gla:Karleby och det halvfärdiga huset köptes av Evert Haga för bortflyttning.Efter att timran plockats ner och och transporterats till tomten timrades det upp på nytt, men med en modifierad takform.

En ladugård byggdes också för några kor, svin och höns. Evert utökade inkomsterna med diversearbete bl.a. på sågen i Silvast och hjälpkarl på Jeppo-Oravais Handelslags lastbilar som försåg nejden med varor.

Evert \textdied 03.04.1969   ---    Agnes \textdied 21.11.1997



%%%
% [house] Ankar
%
\jhhouse{Ankar}{1:54}{Skog}{1}{4, 4a-4b}


\jhhousepic{136-05676.jpg}{Magnus Jungell och Kerstin Lindell}

%%%
% [occupant] Jungell
%
\jhoccupant{Jungell}{\jhname[Magnus]{Jungell, Magnus} \& \jhname[Lindell Kerstin]{Jungell, Lindell Kerstin}}{1983--}
Magnus Waldemar, \textborn 30.07.1955, sambo med Kerstin Lindell, \textborn 27.09.1960 i Bäckby, Esse, övertog hemmanet 28.02.1983. Han har genomgått Korsholms lantbruksskola och har satsat på mjölkproduktion fram till 2012, då denna produktionsform upphörde och ersattes av spannmålsproduktion.

Under åren har den odlade arealen kraftigt utökats från ca 15 ha till nuvarande ca 96 ha, dels genom inköp och dels genom nyodling. Också skogsarealen har genomgått samma utveckling, nu 135 ha. Fastigheten som tidigare utgjort en del av Skog R:nr 1 omfattar idag största delen av Skog R:nr 1.

%%%
% [occupant] Jungell
%
\jhoccupant{Jungell}{\jhname[Elis]{Jungell, Elis} \& \jhname[Dagmar]{Jungell, Dagmar}}{1962--\allowbreak 1983}
Elis Waldemar, \textborn 05.08.1924, gifte sig 22.05.1955 med Dagmar Alexandra Blomqvist från Kovjoki, \textborn 15.12.1918.

År 1962 den 21 maj hade Elis övertagit hemmanet och under efterkrigstidens uppbyggnadstid hade byggts ny ladugård år 1948 och nytt boningshus byggdes år 1972.

Det gamla bostadshuset som stått på samma tomt och flyttats från Stenbacken 1920 revs nu. Huset hade på Stenbacken stått strax söder om den nuvarande Hembygdsgården och den hade ibland använts för uppbevaring av avlidna och hävdades av många vara hemsökt av spöken, något Elis och familjen ändå inte upplevt.

1975 utvidgades ladugården västerut med en ny flygel för mjölkkorna.

\begin{jhchildren}
  \item \jhperson{\jhbold{\jhname[Magnus Waldemar]{Jungell, Magnus Waldemar}}}{30.07.1955}{}
  \item \jhperson{\jhname[Mona Elisabeth]{Jungell, Mona Elisabeth}}{16.03.1957}{}
\end{jhchildren}

Elis Waldemar Jungell \textdied 10.07.2006



%%%
% [oldhouse] Det gamla huset på Ankar
%
\jholdhouse{Det gamla huset på Ankar}{1:}{}{1}{304}


\jhhousepic{Magnus Jungells gla hus.jpg}{Jungells gamla hus}

%%%
% [occupant] Jungell
%
\jhoccupant{Jungell}{\jhname[Johannes]{Jungell, Johannes} \& \jhname[Anna]{Jungell, Anna}}{1900--\allowbreak 1962}
Johannes Jonasson, \textborn 04.11.1877 med efternamnet Romar, gifte sig med Katarina, \textborn 1881. Hon dog 1908 efter dotter Ellen Katarinas födelse 05.01.1908. I det äktenskapet föddes barnen:

\begin{jhchildren}
  \item \jhperson{\jhname[Johan Villiam]{Jungell, Johan Villiam}}{25.03.1903}{}
  \item \jhperson{\jhname[Anders Villiam]{Jungell, Anders Villiam}}{03.06.1906}{}
  \item \jhperson{\jhname[Ellen Katarina]{Jungell, Ellen Katarina}}{05.01.1908}{}
\end{jhchildren}

Johannes gifte om sig med Anna Backlund, \textborn 11.10.1886 i en av granngårdarna på Skog, strax på östra sidan av den alldeles nyss dragna järnvägen.

Hemmanet delades med yngre brodern Anders som 1930 sålde sin del till Joel Sandberg (se Skog nr 7). Namnbytet för bröderna från Romar till \jhbold{Jungell} skedde troligtvis i början av 1900-talet. Hemmanet hade övertagits efter fadern Jonas död år 1900 och ett bostadshus flyttades 1920 från Stenbacken för att härbärgera familjen.

\begin{jhchildren}
  \item \jhperson{\jhname[Johannes Edvin]{Jungell, Johannes Edvin}}{18.11.1920}{}, se Skog nr 5
  \item \jhperson{\jhname[Sven Erik]{Jungell, Sven Erik}}{07.04.1922}{}
  \item \jhperson{\jhbold{\jhname[Elis Waldemar]{Jungell, Elis Waldemar}}}{05.08.1924}{}
  \item \jhperson{\jhname[Anna Linnea]{Jungell, Anna Linnea}}{12.09.1927}{}
\end{jhchildren}

Johannes Jonasson \textdied 16.04.1949   ---   Anna Jakobsdr. \textdied 02.05.1965


%%%
% [occupant] Jungell
%
\jhoccupant{Jungell}{\jhname[Jonas]{Jungell, Jonas} \& \jhname[Lisa]{Jungell, Lisa}}{1880-talet--\allowbreak 1900}
Jonas' gärning på Skog blev rätt kortvarig då han avled redan år 1900, varefter hemmanet övertogs av äldste sonen Johannes.

\begin{jhchildren}
  \item \jhperson{\jhbold{\jhname[Johannes]{Jungell, Johannes}}}{04.11.1877}{}
  \item \jhperson{\jhname[Simon]{Jungell, Simon}}{25.06.1880}{}
  \item \jhperson{\jhname[Anders]{Jungell, Anders}}{26.07.1885}{}
  \item \jhperson{\jhname[Aina Susanna]{Jungell, Aina Susanna}}{17.02.1891}{}
  \item \jhperson{\jhname[Matias]{Jungell, Matias}}{18.01.1893}{}
  \item \jhperson{\jhname[Karl Emil]{Jungell, Karl Emil}}{25.02.1896}{13.01.1898}
  \item \jhperson{\jhname[Karl William]{Jungell, Karl William}}{05.05.1899}{}
\end{jhchildren}

Jonas \textdied 14.05.1900  ---   Lisa \textdied 10.09.1926



%%%
% [house] Ankar
%
\jhhouse{Ankar}{1:54}{Skog}{1}{5-5a}


\jhhousepic{139-05681.jpg}{Edvin Jungells hus}

%%%
% [occupant] Jungell
%
\jhoccupant{Jungell}{\jhname[Edvin]{Jungell, Edvin}}{1945--\allowbreak 2011}
Edvin Jungell, \textborn 18.11.1920, var äldst i syskonskaran. Han byggde bostadshuset efter kriget och levde ensam i detta hela sitt liv.

Han deltog naturligtvis som barn och ungdom i gårdens arbete fram till 18 års ålder då han fick arbete som ``krutpojke'' vid Kiitola ammunitionsfabrik hösten 1939. Arbetet gick ut på att tillsammans med 4 andra hjälpkarlar och en chaufför transportera krutlådor och övrigt material från järnvägsstationen till Kiitola, alternativt från Kiitola till de 3 lagerbyggnader som hyrts vid Keppo.

Krutet var förpackat i trälådor, 240 kg tunga. Lådorna skulle lyftas av 2 man i var ända och den 5:e personen skulle rycka in där det sviktade. Kraftprovet inträffade när lastbilen skulle lossas vid lastbryggan vid Kiitola då plattformen befann sig 80 cm högre än lastbilsflaket. Arbetskraft från Oravais deltog ofta i arbetet och det blev inte sällan en tävling mellan arbetslagen vilka som var starkast. Tillsammans med Birger Forslund, också han jeppobo, klarade de sig med den äran i konkurrensen. Edvin var nämligen mycket stark trots sin ganska spensliga kroppsbyggnad.

Ett exempel på detta kunde han ge då det på det s.k. ``låglandet'' på andra sidan landsvägen mittemot inkörsporten till fabriksområdet låg en stor artillerigranat öppet på marken. Ingen visste om den innehöll trotyl eller inte, men den sades väga 250 kg! Alla som trodde sig vara starka måste naturligtvis pröva sina krafter. P.g.a. sin form var den nästan omöjlig att få grepp om. Man måste stiga på knä och rulla granaten över ena handen och underarmen och knäppa ihop händerna bakom granaten om armarna räckte till. Därefter var det dags att pröva lyftet.

- Jag vet, att bara en finne och jag har fått upp den, berättar Edvin inte utan stolthet 70 år senare.

Den 12 mars, dagen före vapenstilleståndet efter vinterkriget skulle han infinna sig till Vasa för värnplikt. Edvin hade ända från födseln haft exeptionellt dålig syn och glasögon tjocka som flaskbottnar. P.g.a. sina ögonproblem kom han hem från värnpliktsutbildningen senhösten 1940. Före midsommaren -41 lastades han i oxvagnar tillsammans med andra reservister och transporterades ner mot Hangö, där en sovjetisk marinbas installerats efter vinterkriget. Där fick han fungera som ordonnans, vilket passade honom utmärkt, Han hade kondition som en häst och tyckte om att orientera sig fram även om det höll på att sluta illa. Då det var svårt att ta sej fram genom sly och taggtrådshinder beslöt han sig en sen kväll att använda järnvägsvallen, det var bekvämare så. Ingen kunde ju ändå se honom. Trodde han. Han såg mynningsflamman från en direktskjutande kanon innan han kände vinddraget bredvid vänster öra före han hörde knallen. ``Det var bara att kasta sig ner i järnvägsdiket då ju nästa skott kunde ha träffat'', berättar han. ``Inte fick vi heller gå in i staden sedan ryssarna jagats bort. Däremot passade nog ett svenskt frivilligkompani på att marschera in som första trupp och ta äran åt sig. Visste du det?''
(Anm. Det skedde på Mannerheims uppmaning och kanske det var en lycka då staden var kraftigt försåtsminerad.)

Efter kriget fick Edvin anställning på Jeppo-Oravais Handelslags magasin vid järnvägsstationen, ett magasin som nybyggdes på 1950-talet och kom att spela en betydande roll i ortens handel med alla de slag av jordbruksprodukter och byggnadsmaterial.

År 1949 beställde Edvin månadsjournalen ``Det Bästa'' från Sverige. Där läste han en gång en berättelse om 2 amerikanska marinkårssoldater som blivit torpederade under 2:a världskriget. De påstod att de hade drunknat om de inte hade haft kontaktlinser! Hade de haft vanliga glasögon, som hade flugit all världens väg vid explosionen, skulle de aldrig hittat de vrakdelar de kunde simma fram till och hålla sig flytande på fram till räddningen. ``Vad är kontaktlinser?'' undrade Edvin. ``Jag ringde till ögonläkare i Jakobstad, men ingen visste vad det var. Samma svar fick jag vid Vasa sjukhus. Småningom tog jag mod till mej och ringde Universitetssjukhuset i Helsingfors. En dam svarade att de nog hade hört om dem, men ingen hade gjort något liknande i Finland. Hon förenade till ansvarige professorn, som bekräftade att man inte gjort någonting sådant ännu i detta land.\newline
\indent  - Skulle Ni vara intresserad? \newline
\indent  - Visst skulle jag det, svarade jag.\newline
\indent  - Kan Ni komma hit nästa vecka?, frågade han.\newline
\indent  - Nästa vecka reste jag till Helsingfors och länge var jag där. Tårarna rann och jag såg ingenting. Man provade och provade och så småningom blev det bättre och till sist blev det riktigt bra! Senare har jag läst att det finns risk för blindhet när man blir gammal. Snart fyller jag 89 och på mitt vänstra öga är jag helt blind. På det högra har jag tunnelseende. Det går ännu här hemma, men blir jag helt blind blir det nog värre.''

Hans problem med synen begränsade ändå inte hans rörelsefrihet helt. Om han begav sej på en cykel- eller gångtur kunde han urskilja den vita randen vid vägens kant och mötte han en person kunde han identifiera denna på rösten.

Edvin var sin arbetsplats trogen hela sitt verksamma liv och han är tveklöst en av dem som lyft flest antal ton i detta land under sin livstid.

Ändå finns här en paradox. Samtidigt har han varit en av JIF:s trognaste och pålitligaste poängplockare i de klubbkamper som gick av stapeln under ett par decennier efter kriget. Hans specialitet var medeldistans och speciellt 1500 m. Efter en fysiskt tung arbetsdag där många skulle ha stupat av trötthet, cyklade han hem, tog med sej löpskorna, cyklade tillbaka till sportplanen och knäckte de flesta av sina motståndare med  hård jämn fart. Spurten var hans akilleshäl. -- Skidåkning uppskattade han också och deltog gärna i tävlingar, med framgång.

Edvin dog den 22.07.2011 på en plats där han trivdes bra – hemma i bastun.


%%%
% [house] Ankar
%
\jhhouse{Ankar}{1:54}{Skog}{1}{6, 6a}

%%%
% [occupant] Jungell
%
\jhoccupant{Jungell}{\jhname[Magnus]{Jungell, Magnus}}{2009--}
\jhperson{\jhname[Magnus Waldemar Jungell]{Jungell, Magnus Waldemar Jungell}}{30.07.1955}{}, inköpte år 2009 nämnda fastighet, som stått tom efter att Evert Skog avlidit 1981.\jhvspace{}


%%%
% [occupant] Sandberg
%
\jhoccupant{Sandberg}{\jhname[Kaj]{Sandberg, Kaj}}{1995--\allowbreak 2009}
Kaj Sandberg köpte fastigheten på 1990-talet och har tidvis utnyttjat den som förråd.\jhvspace{}


\jhhousepic{138-05680.jpg}{Evert Skogs hus, nu M. Jungells. Oklart när huset byggdes}

%%%
% [occupant] Skog
%
\jhoccupant{Skog}{\jhname[Evert]{Skog, Evert}}{1946--\allowbreak 1981}
\jhperson{\jhname[Anders Evert Skog]{Skog, Anders Evert}}{05.03.1912}{}, fick som ende sonen tidigt axla ansvaret för hemmanet och under kriget tillsammans med sin syster Ester. Efter att hon 01.02.1948 gift sig med Svante Haglund, \textborn 22.10.1915, bodde hon med sin familj i samma hus och här har barnen tillbringat sin barndom.
\begin{jhchildren}
  \item \jhperson{\jhname[Bo Anders]{Skog, Bo Anders}}{28.04.1948}{}
  \item \jhperson{\jhname[Stig Göran]{Skog, Stig Göran}}{18.04.1949}{}
  \item \jhperson{\jhname[Karl-erik]{Skog, Karl-erik}}{30.05.1951}{}
  \item \jhperson{\jhname[Barbro Ester Katarina]{Skog, Barbro Ester Katarina}}{30.08.1953}{}
  \item \jhperson{\jhname[Elvi Heldine Elisabeth]{Skog, Elvi Heldine Elisabeth}}{30.04.1958}{}
\end{jhchildren}

Familjen flyttade år 1963 till sitt inköpta hem vid Nylandsvägen (se Romar nr 30). Evert fortsatte efter denna tidpunkt ensam driva jordbruket i liten skala. Han dog 29.09.1981.


%%%
% [occupant] Skog
%
\jhoccupant{Skog}{\jhname[Thomas]{Skog, Thomas} \& \jhname[Katarina]{Skog, Katarina}}{1899--\allowbreak 1946}
\jhperson{\jhname[Thomas Thomasson]{Skog, Thomas Thomasson}}{15.03.1877}{} gifte sig 09.11.1897 med, \jhperson{Katarina Andersdotter Måtar}{01.11.1874}{}. Till en början bodde de som inhyses på Romar hos hans far Thomas Johansson innan de erhöll ½ av hemmanet på Skog, som fram till dess ägts av Fredrik Johanss och hans hustru Anna.

Livet på Skog var strävsamt och Thomas och Katarina fick 12 barn varav 6 dog i låg ålder:
\begin{jhchildren}
  \item \jhperson{\jhname[Irene]{Skog, Irene}}{1898}{1899}
  \item \jhperson{\jhname[Johannes]{Skog, Johannes}}{1905}{1905}
  \item \jhperson{\jhname[Johannes]{Skog, Johannes}}{1908}{1909}
  \item \jhperson{\jhname[Gunnar]{Skog, Gunnar}}{1910}{1913}
  \item \jhperson{\jhname[Thomas]{Skog, Thomas}}{1914}{1914}
  \item \jhperson{\jhname[Elis]{Skog, Elis}}{1916}{1916}
  \item \jhperson{\jhname[Ellen Katarina]{Skog, Ellen Katarina}}{28,10.1899}{}
  \item \jhperson{\jhname[Helga Maria]{Skog, Helga Maria}}{27.08.1907}{}
  \item \jhperson{\jhname[Edit Susanna]{Skog, Edit Susanna}}{06.04.1910}{}
  \item \jhperson{\jhbold{\jhname[Anders Evert]{Skog, Anders Evert}}}{05.03.1912}{}
  \item \jhperson{\jhname[Anna Elvira]{Skog, Anna Elvira}}{16.05.1913}{}
  \item \jhperson{\jhname[Ester Heldine]{Skog, Ester Heldine}}{09.05.1918}{}
\end{jhchildren}

Katarina \textdied 15.08.1929 --- Thomas \textdied 28.04.1946


%%%
% [occupant] Johansson
%
\jhoccupant{Johansson}{\jhname[Fredrik]{Johansson, Fredrik} \& \jhname[Anna]{Johansson, Anna}}{1880--\allowbreak 1899}
Fredrik Johansson Skog, \textborn 11.05.1853, gifte sig med Anna Eriksdr. Grötas,  \textborn 1850, och hade 1880 övertagit hemmanet av föräldrarna Johan och Maria Hansson, \textborn 1814 resp. 1817. Fredrik besökte 1892--\allowbreak 1894 USA, vilket småningom fick återverkningar för hela familjen.

I familjen föddes 10 barn av vilka 8 nådde vuxen ålder och de 2 yngsta, Emil \textborn 1891 och Aina \textborn 1893, dog i späd ålder.
\begin{jhchildren}
  \item \jhperson{\jhname[Johannes]{Johansson, Johannes}}{1874}{}
  \item \jhperson{\jhname[Fredrik]{Johansson, Fredrik}}{1875}{}
  \item \jhperson{\jhname[Erik]{Johansson, Erik}}{1877}{}
  \item \jhperson{\jhname[Anders]{Johansson, Anders}}{1879}{}
  \item \jhperson{\jhname[Joel]{Johansson, Joel}}{1882}{}
  \item \jhperson{\jhname[Anna]{Johansson, Anna}}{1884}{}
  \item \jhperson{\jhname[Ida]{Johansson, Ida}}{1886}{}
  \item \jhperson{\jhname[Selma]{Johansson, Selma}}{1888}{}
\end{jhchildren}

Medan de yngre barnen föddes drogs järnvägsspåret öster om bebyggelsen på Skog och delade hemmanets ägor. Det är ändå oklart om detta faktum inverkade på att ALLA familjens barn emigrerade till USA. Den 3 april 1909 ansökte också Fredrik och Anna om utresa. Tydligen insjuknade Anna därborta och hon står antecknad som död i Amerika 24.10.1909.

Fredrik återvände ensam och 18.03.1911 tog han ut lysning med \jhname[Helena Sofia Sundholm]{Sundholm, Helena Sofia} \textborn 14.10.1847 på Kaup. Den 3 april samma år gifte de sig och flyttade till Finskas nr 5 där de fram till sin död 1921, resp. 1930 levde som backstugusittare.



%%%
% [house] Hemfors
%
\jhhouse{Hemfors}{1:26}{Skog}{1}{7, 7a-7b}


\jhhousepic{137-05679.jpg}{Teemu och Jasmin Aalto bor i f.d. Sandbergs hus nära järnvägen}

%%%
% [occupant] Aalto
%
\jhoccupant{Aalto}{\jhname[Teemu]{Aalto, Teemu} \& \jhname[Jasmin]{Aalto, Jasmin}}{2009--}
Teemu Aalto, \textborn 20.01.1986, gift med Jasmin Viljaranta från J:stad, \textborn 19.12.1983, köpte fastigheten år 2009. I köpet ingick endast tomten inklusive byggnaderna.

Teemu är utbildad timmerman och har haft anställning bl.a. hos Brage Finskas och HT-Lasertekniikka i Alahärmä. För tillfället har han anställning vid Skaala Ikkuna i Alahärmä. Jasmin är utbildad kock och har arbetat på ABC-stationen i Oravais. Numera är hon innehavare av och driver Café Funkis.
\begin{jhchildren}
  \item \jhperson{\jhname[Jonny]{Aalto, Jonny}}{10.05.2006}{}
  \item \jhperson{\jhname[Jimi]{Aalto, Jimi}}{04.12.2008}{}
  \item \jhperson{\jhname[Jasper]{Aalto, Jasper}}{08.12.2010}{}
\end{jhchildren}


%%%
% [occupant] Sandberg
%
\jhoccupant{Sandberg}{\jhname[Kaj]{Sandberg, Kaj}}{1991--\allowbreak 2009}
Kaj Mats Sandberg,  \textborn 31.12.1967, övertog hemmanet år 1991, men ekonomiska svårigheter gjorde att det gick förlorat 2009.\jhvspace{}


%%%
% [occupant] Sandberg
%
\jhoccupant{Sandberg}{\jhname[Carl-Gustav]{Sandberg, Carl-Gustav} \& \jhname[Margareta]{Sandberg, Margareta}}{1967--\allowbreak 1991}
Carl-Gustav, \textborn 12.10.1935, gifte sig 04.11.1961 med Viola Margareta Wärnå från Esse, \textborn 22.11.1936. De övertog Hemfors lägenhet av Carl-Gustavs föräldrar år 1967. Lägenhetens huvudsakliga inriktning var mjölkproduktion vilken krävde makarnas fulla engagemang. Ladugården har renoverats och byggt ut i olika repriser under årens lopp.

Carl-Gustav har varit en samhällsinriktad person och deltagit i olika föreningars aktiviteter. Bl.a fungerade han under flera år som
ordförande för Jeppo lokalavdelning av ÖSP.
\begin{jhchildren}
  \item \jhperson{\jhname[Gunilla Margareta]{Sandberg, Gunilla Margareta}}{03.08.1964}{}
  \item \jhperson{\jhname[Birgitta Irene]{Sandberg, Birgitta Irene}}{02.04.1966}{}
  \item \jhperson{\jhbold{\jhname[Kaj Mats]{Sandberg, Kaj Mats}}}{31.12.1967}{}
  \item \jhperson{\jhname[Kristina Viveka]{Sandberg, Kristina Viveka}}{27.04.1970}{}
\end{jhchildren}



%%%
% [oldhouse] Hemfors
%
\jholdhouse{Hemfors}{1:5}{Skog}{1}{307}


%%%
% [occupant] Sandberg
%
\jhoccupant{Sandberg}{\jhname[Joel]{Sandberg, Joel} \& \jhname[Elna]{Sandberg, Elna}}{1930--\allowbreak 1967}
Joel Sandberg, \textborn 11.07.1900, gifte sig 1932 med Elna Irene Bäckstrand, \textborn 23.03.1908.

År 1930 köpte Joel Sandberg Hemfors lägenhet R:nr 1:5 av Anders Jungell. I köpet ingick de byggnader som fanns på lägenheten. Ny bostad byggdes åren 1959-60. Mjölkproduktion var huvudinriktning på gården, men Elna var också lärare på Jungar folkskola åren 1946-66 utöver de 2½ åren som vikarie under krigsåren.
\begin{jhchildren}
  \item \jhperson{\jhname[Carita Magdalena]{Sandberg, Carita Magdalena}}{23.03.1933}{}
  \item \jhperson{\jhbold{\jhname[Carl-Gustav]{Sandberg, Carl-Gustav}}}{12.10.1935}{}
  \item \jhperson{\jhname[Erik Markus]{Sandberg, Erik Markus}}{29.06.1945}{}
  \item \jhperson{\jhname[Martin Joel]{Sandberg, Martin Joel}}{29.06.1945}{}
\end{jhchildren}


%%%
% [occupant] Jungell
%
\jhoccupant{Jungell}{\jhname[Anders]{Jungell, Anders} \& \jhname[Olivia]{Jungell, Olivia}}{1920--\allowbreak 1930}
Anders Jungell, \textborn 26.07.1885 på Skog, gifte sig  29.04.1910 i USA med Olivia Ersfolk, \textborn 09.08.1887 i Övermark. Efter hemkomsten från USA uttogs lysning den 25 jan. 1920 för bekräftelse av äktenskapet. Nu delades hemmanet med äldre brodern Johannes varvid Anders och Olivia tog som fosterföräldrar hand om Johannes dotter från första giftet, Ellen Katarina, \textborn 05.01.1908, uppväxt upp hos farmor Lisa, som dog 1926 då Ellen var 18 år, gifte sig med Leander Sandberg (se Fors nr 407 och Silvast nr 84). Efter återkomsten från USA hade de  köpt  ett hus på Stenbacken som de plockade ner och flyttade till Skog.

1929 flyttade de till Silvast och 1930 sålde hemmanet till Joel Sandberg. I Silvast uppförde de den byggnad bredvid Ungdomslokalen som skulle bli det sista gästgiveriet i Jeppo (se Silvast nr 380).


%%%
% [occupant] Jungell (Romar)
%
\jhoccupant{Jungell (Romar)}{\jhname[Jonas]{Jungell (Romar), Jonas} \& \jhname[Lisa]{Jungell (Romar), Lisa}}{1880-talet-1900}
Jonas Simonsson Romar, \textborn 28.03.1851, gift med Lisa Johansdr., \textborn 22.11.1856. De har i mantalslängden för 1890 antagit efternamnet \jhbold{Jungell}. Jonas' gärning på Skog blev ganska kortvarig då han avled redan år 1900 (se nr 4).



%%%
% [house] Central
%
\jhhouse{Central}{1:9 / 1:4}{Skog / Haga}{4}{68, 368}


\jhhousepic{062-05596.jpg}{Den gamla bemannade centralen är ersatt med barack och ny teknik}

%%%
% [occupant] Telia/Sonera
%
\jhoccupant{Telia/Sonera}{\jhname[Finska staten]{Telia/Sonera, Finska staten}}{1957--}
Efter undersökningen 1954--\allowbreak 1955 och underhandlingar 1956--\allowbreak 1957 beslöt	Jeppo Telefon Ab att överlåta telefonbolaget till Post- \& Telegrafverket. Den 29.08.1957 blev därmed Finska staten ägare till fastigheten Central 1:9 med telefoncentralbyggnaden. Luftledningarna avskaffades och kabelsystem infördes. Samtidigt infördes en andra central, Gunnar, som betjänade de sydligaste delarna av Jungar och Överjeppo byar.

På centralen vid \jhbold{Sparvbacken} (Sparrbacken), se karta 4, turades 5-6 flickor om att förena samtalen till bygdens folk. Centralbyggnaden var kall och dragig. Hade man en flaska med vatten på golvet vintertid, frös vattnet till is. Post \& Telegrafverket ordnade med yllefiltar och filtstövlar till telefonisterna, för att skulle kunna hålla sig varma.

Den gamla centralbyggnaden (nr 368) revs och en telebyggnad och en telefonmast uppfördes på tomten. Telefonisterna fick äntligen arbeta i en mera ändamålsenlig och varm byggnad, med el-element, varmt vatten och toalett inomhus. När telefoncentralen i Pensala på 70-talet automatiserades, överfördes telefonisterna till Jeppo telefonanstalt, där 11 telefonister arbetade under de sista åren centralen var verksam.

Vev-telefonen ersattess 1972 av halvautomattelefon, man behövde bara lyfta på luren så svarade telefonisten och kopplade samtalet. Batteriet på väggen i hemmen kunde nu monteras ner. Efter 20 år, 1977, automatiserades telefonväsendet. Telefonisterna erbjöds arbete i Kauhava eller Vasa. Den som inte var villig att flytta till någon av dessa orter blev uppsagd från sin statliga tjänst, utan någon form av ersättning. Idag är telefon-, TV- och dataväsendet digitaliserat.


\jhhousepic{Jeppo telefoncentral 1968.png}{Jeppo telefoncentral 1968, nr 368}

%%%
% [occupant] Jeppo Telefon
%
\jhoccupant{Jeppo Telefon}{\jhname[Ab]{Jeppo Telefon, Ab}}{1921--\allowbreak 1957}
Den 19.07.1921 höll Jeppo Telefon Ab en extra bolagsstämma.	Närvarande var A. Sandqvist, R. Husberg, J. Wörlund, T. Westerlund,	O. Sundell, J.T. Backlund, A.W. Finskas, G. Liljeqvist och D. Jungar.	Beslöts att inköpa Anders Skogs ägande backstugområde av Skog hemman nr 1 i Jungar by. Köpebrevet upprättades 16.08.1921, för ett	pris på 6565 mark, för ett utbrutet område med boningsstuga och uthus. Telefoncentralen installerades i den inköpta gården. Anders Skog hade byggt en stuga på området och 1912 inlämnat ansökan om att få inlösa sitt område till självständig lägenhet, styckning slutförd 27.10.1920. År 1925 infördes i jordregistret Central 1:9 om 0,185 ha av lägenhet Haga 1:4 av Skog hemman.

Redan på 1890-talet började man i Jeppo undersöka möjligheterna till att få telefonförbindelser. Redan 1891 börjar telefonstolpar resa sig mellan Nykarleby och Jeppo och 1895 är telefonledningen färdig att anlitas. De första telefonerna hade prästen, postkontoret, Sundells Bageri, Modéns Handelsbutik, Keppo gård, Kiitola, Lantmannagillet, poliskonstapeln och Jungar folkskola. Före inbördeskriget var det glest med telefoner; år 1921 var antalet 15 abonnenter med 25 andelar.
Direktörerna under tiden för Jeppo Telefon Ab var:
\begin{center}
  \begin{tabular}{l l}
    \hline
    Namn & Period \\ \hline
    \jhname{Jungar, Daniel} & 1921--\allowbreak 1934 \\
    \jhname{Westerlund, Thure} & 1935--\allowbreak 1937 \\
    \jhname{Jungell, Anders} & 1938--\allowbreak 1941 \\
    \jhname{Wistbacka, Edvin} & 1942 \\
    \jhname{Ström, Georg} & 1943--\allowbreak 1945 \\
    \jhname{Jungerstam, E.W.} & 1946--\allowbreak 1948 \\
    \jhname{Backlund, Valter} & 1949--\allowbreak 1950 \\
    \jhname{Sundell, Torsten} & 1951--\allowbreak 1954 \\
    \jhname{Romar, Selim} & 1955--\allowbreak 1960 \\ \hline
  \end{tabular}
\end{center}



\jhhousepic{Jeppo telefoncentral 1968-C.jpg}{Christina Sundell, gift Simons, kopplar kontakterna}



\jhbold{Telefonisterna}

``Central-flickorna'' har varit många under årens lopp. De tre först anställda bodde till och med i huset. I slutet av 1940-talet infördes skiftesarbete med möjlighet att sova på natten men med krav på väckning vid telefonringningar. Centralföreståndarna tjänstgjorde vanligen i många år medan flickorna gjorde det i längre eller kortare perioder.

Telefoncentralen spelade en viktig roll i byns funktioner och kunde förmedla information som underlättade vardagslivet. Mången jeppobo minns ännu hur man kunde ringa centralen för att vid brandalarm få reda på var det brinner och då eventuellt skynda sig att bistå i släckningsarbetet, eller som fick hjälp med att spåra upp veterinären för ett akut gårdsbesök om denne råkade rörde sig i byn.

Personalen under den aktuella tiden:
\begin{center}
  \begin{tabular}{l l l}
    \hline
    Boende el. föreståndare & Telefonist & Telefonist\\ \hline
    \jhname{Mattsson, Sanna-Maria} & \jhname{Huhtala, Margaretha} & \jhname{Norrgård, Gun} \\
    \jhname{Kronlund, Göta} & \jhname{Sundell, Ulla} & \jhname{Back, Gun-Lis} \\
    \jhname{Sandqvist, Hjördis} & \jhname{Sandqvist, Eva} & \jhname{Simanainen, Pirkko} \\
    \jhname{Johansson, Else-Maj} & \jhname{Lindström, Gunnel} & \jhname{Elenius, Gun-Lis} \\
    \jhname{Sundell/Simons, Christina} & \jhname{Lindström, Margit} & \jhname{Strengell, Gundel} \\
    - & \jhname{Björkqvist, Anita} & \jhname{Norrgård, Eivor} \\
    - & \jhname{Björkqvist, Carita} & \jhname{Kennola, Anna-Liisa} \\
    \jhname{Sandqvist, Marita}, t:fonist & \jhname{Kula, Anita} & \jhname{Kennola, Marja-Liisa} \\
    \jhname{Sipponen, Hilkka}, t:fonist & \jhname{Bärs, Mona} & \jhname{Lassila, Gun-Britt} \\
    \jhname{Bro, Pirkko}, t:fonist & \jhname{Annanolli, Regina} & \jhname{Lassila, Birgitta} \\ \hline
  \end{tabular}
\end{center}
