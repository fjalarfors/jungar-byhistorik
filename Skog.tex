\jhchapter{Skog, hemman Nr 1}

Före år 1831 hade hemmanet nr 32 i Nykarleby stads mantalslängd. Det gick då under namnet Skogsbyggare och nr 32 representerade dess plats i Överjeppo by och gränsade till Ytterjeppo by. Hemmanens antal längs i huvudsak älven i Överjeppo var 34 st, med Grötas som nr 1. Logiken i numreringen är för oss oklar, men fr.o.m. 1831 ändrar hemmanens mantalsnummer och Skogsbyggare får nu nr 1 och namnet ändras till enbart Skog. Hemmanet omfattade ½ mantal. Enligt släktforskarföreningen i Jakobstad innehades hemmanet mellan 1551-1563 av Nils Larsson, 1565-1607 av Mikkel Nilsson, 1609-1633 av Mårten Mikkelsson, 1623-1629 av Hans Hansson, 1629- ca.1653 av Carl Hindersson, 1657- 1673 av Matts Carlsson, 1675 av  Carl Mattsson, hu. Lisa Jönsdr., 1676-1709 av Tomas Mattsson och hu 1 Karin Hansdr, hu 2 Lisa, medåbo bror Markus Mattsson 1677-97, 1710-1713 av sonen Jakob Tomasson och 1723- av Hans Hansson, hu Maria.

Efter Stora ofreden genomförde Matthias Wörman år 1740 en lantmäteriförrättning och kartläggning längs ån med början från åmynningen i Nykarleby. På hans kartblad finns endast en gård registrerad. 1783 års mantalslängd av Nykarleby stad upptar också endast en innehavare med 5/18 mtl. Små åkertäppor var placerade i solskiftesform längs Romarbäcken. Det finns skäl att anta att Skogsbyggare (Skog) med sin placering en bit från älven var en bosättningsplats för backstugusittare under äldre tid. Från 1762 tilläts nämligen gifta legohjon att bygga backstugor och boningsrum på enskilda och samfällda ägor. Ändå har antalet bosatta på Skogsbyggare varit lågt. År 1807 fanns antecknat endast 4 personer och hemmanet innehades av personer från Romar.

Utmärkande för Skog, i ännu högre grad än för övriga hemman, är den mycket stora emigrationen från hemmansnummern. Stora syskonskaror emigrerade i sin helhet till USA och emellanåt följde också föräldrarna efter. Idag finns endast en (1) brukningsenhet kvar på hemmanet, som under hela 1900-talet hyste flera hemmansdelar. Däremot finns flera bostäder placerade på hemmanets mark.

Skog hemman omfattas av vidstående karta nr \jhbold{1}.


\jhpic{Karta 1, Skog.png}{Karta 1 över Skog hemman, Nr 1}


\jhsubsection{Lägenheter på Skog}

\jhhouse{Strandkanten}{1:43}{Skog}{1}{1+1a}

\jhoccupant{Ljung-Sandvik}{Kerstin \& Sandvik Sten}{1996 -}

Kerstin Ljung Sandvik \textborn 13.06.1964 och maken Sten Sandvik \textborn 09.07.1958 köpte fastigheten 20.06.1996 av sterbhuset efter Harry Ljung. Makarna hade gift sig 12.11.1994

\begin{jhchildren}
  \item \jhperson{Conny}{12.03.1995}{}, är utbildad skogsmaskinförare
  \item \jhperson{Fredrik}{12.03.1995}{}, är utbildad kombinationsförare
\end{jhchildren}

\jhhousepic{001-05522.jpg}{Kerstin Ljung-Sandvik och Sten Sandvik}

Sten har arbetat på Jeppos största företag Mirka från 1992. Före det hade han arbete på Sjöholms minkfarm 1976-1992.

Kerstin utbildade sig till Pedagogie magister och har haft flera tjänster. Hon startade som klasslärare i Ytterjeppo skola 1988-91. Därefter vidareutbildade hon sej 1991-92 och kom till Larsmo 1992 där hon stannade 1 år som lärare. Mellan åren 1993-96 fungerade hon som ambulerande speciallärare i Karleby för att från 1996 haft tjänsten som speciallärare i Larsmo.

Efter att Kerstin och Sten år 1996 köpt fastigheten och en del av hemmanets skog, såldes resten. Samma år revs den s.k. Morastugan och året därpå revs också ladugården.


\jhoccupant{Ljung}{Harry \& Inga}{1959-1996}

Harry Johannes Ljung, \textborn 05.02.1935 i Purmo, gifte sig 1955 med Inga Maria Lindström, \textborn 03.0.1933 i Oravais.

Harry utbildade sej tidigt till seminolog, ett yrke som på 50-talet var helt nytt inom den finska kreaturshållningen. Han fick 1956 jobb i östra Nyland, Strömfors, och familjen flyttade 1957 till Jakobstad. Fram till 1962 arbetade han som seminolog i Pedersöre och Larsmo. Därefter flyttade familjen till Jeppo där Harry och Inga övertog hemmanet av Harrys föräldrar Ellen och Johannes. Han fortsatte sitt arbete som uppskattad seminolog i Jeppo området fram till sin för tidiga död 1970. Den odlade jorden utarrenderades nu.

Makarna byggde 1964 ett nytt hus på gårdens tomt, men närmare ån än den gamla mangårdsbyggnaden. Denna s.k. Morastuga hade flyttats till platsen troligtvis 1905 och fungerat som släktens hem i två generationer.

Inga är utbildad barnskötare, men arbetade efter övertagandet av Harrys hemgård på jordbruket medan barnen växte upp. Efter makens död arbetade hon på Sundells bageri under 4 års tid. År 1974 började hon utbilda sej till barnträdgårdslärare och fick arbete inom barndagvården i Nykarleby där hon fungerade som administrativ ledare fram till sin pensionering år 1996 då hon flyttade till Jeppo centrum.

\begin{jhchildren}
  \item \jhperson{Mikael}{06.05.1956}{}
  \item \jhperson{Stefan}{20.08.1962}{}, bl.a. lärare på yrkesskolan i Närpes
  \item \jhperson{\jhbold{Kerstin}}{13.06.1964}{}
\end{jhchildren}

Harry Ljung \textdied 09.05.1970


\jhoccupant{Ljung}{Ellen \& Johannes}{1939-1959}

Ellen Alina Haga föddes 06.10.1907 och gifte sig  1934 med Paul Johannes Ljung, \textborn 24.05.1906 i Purmo. Efter giftermålet flyttade Ellen till Purmo där makarna bodde fram till den 26.04. 1939 då familjen flyttade till Jeppo för att där överta det hemman som Ellens föräldrar Johan och Hilda Haga den 11.04.1939 sålt till dem. Hemmanet var inte stort, drygt 10 ha odlad jord och något mera skog och på hemmanet odlades som på de flesta andra; hö, havre, råg och potatis. Genast efter kriget försöktes också med kalkonuppfödning vid sidan av de traditionella korna, men avslutades ganska snart.

\jhhousepic{Morastugan L.jpg}{Morastugan}

En ny ladugård byggdes nu med den brist på byggnadsmaterial som var typiskt för efterkrigstiden. Johannes började för att stärka ekonomin fungera som ombudsman för Österbottens Kött till dess han insjuknade 1959 och Ellen fick en ännu större arbetsbörda.

\begin{jhchildren}
  \item \jhperson{\jhbold{Harry}}{05.02.1935}{09.05.1970}
  \item \jhperson{Hans}{15.01.1938}{2001}, i Sverige, verkat som lärare
  \item \jhperson{Stig}{07.11.1939}{}, Bl.a. chef för FPA:s byrå i Ekenäs och kommundirektör i Snappertuna
  \item \jhperson{Greta}{30.09.1941}{}, Bl.a. professor vid MIT-Bl.a. professor vid MIT-universitetet i Boston
  \item \jhperson{Lisa}{11.11.1947}{}, Merkonom från handelsinstitutet i Karleby
\end{jhchildren}

Johannes \textdied 28.11.1968   --    Ellen \textdied 28.05.1997


\jhoccupant{Haga}{Hilda \& Johan}{1905-1939}

Hilda Alina Skog, \textborn 15.05.1883, gifte sig den 06.08.1905 med Johan Mariasson Huhtala från Ylistaro. Han hette ursprungligen Juho Pikkutupa, men tog namnet Huhtala i samband med giftermålet med Hilda. Senare tog familjen namnet Haga för att deras barn skulle kunna gå i svensk skola. Johan, \textborn 05.04.1883 som faderlös, men sannolikt med ryskt påbrå.

Hilda var 8:e barnet till Anders Johansson Skog och hans första hustru Anna Kajsa Eriksdotter. När Hilda och Johan gifte sig 1905 utstyckades en hemmansdel från Skog hemman och bildade ett nytt hemman med bosättning vid älven dit en s.k. Morastuga flyttades. I äktenskapet föddes 12 barn:
\begin{jhchildren}
  \item \jhperson{Johan Lennart}{12.03.1906}{18.07.1924}
  \item \jhperson{\jhbold{Ellen Alina}}{06.10.1907}{28.05.1997}
  \item \jhperson{Anders Evert}{28.04.1909}{03.04.1969}
  \item \jhperson{Edit Maria}{18.04.1912}{}, till Karis
  \item \jhperson{Gerda Katarina}{10.11.1913}{29.12.1937}
  \item \jhperson{Sylvi Linnea}{07.10.1915}{}, till Bromarv
  \item \jhperson{Hilda Irene}{10.08.1917}{}, till Ekenäs
  \item \jhperson{Elsa Anna Viola}{06.01.1920}{}, till Sverige
  \item \jhperson{Hjördis Valdine}{10.11.1921}{}, till Snappertuna
  \item \jhperson{Erik Alfred}{14.05.1923}{04.12.2013}
  \item \jhperson{Margit Susanna}{01.12.1925}{}, till Pernå
  \item \jhperson{Johannes Evald}{04.12.1928}{}, till Snappertuna
\end{jhchildren}

Hilda Alina \textdied 19.01.1937  --   Johan Mariasson \textdied 18.12.1943


\jhhouse{Strandhagen}{1:57}{Skog}{1}{2}

\jhoccupant{Lindvall}{Daniel \& Bäckstrand-Lindvall Caroline}{2013-}

Daniel August Jakob Lindvall, \textborn 21. 11.1979 i Nykarleby Forsby, gifte sig 14.02.2009 med Caroline Benita Louise Bäckstrand, \textborn 29.03.1983.


\jhhousepic{002-05524.jpg}{Daniel Lindvall och Caroline Bäckstrand-Lindvall}

År 2012 köpte de delar av lägenheterna Ankar R:nr 1:54 och Haga R:nr 1:49 av Magnus Jungell och bildade lägenheten Strandhagen R:nr 1:57. Samma år påbörjades planeringen av ett nytt hus på den nybildade tomten. Utgångspunkten var ett av Teeri-husens modeller, men efterhand utgicks från egen ritning och 2013 uppfördes huset, medan garaget byggdes året innan.

Daniel har utbildat sig till konditor. Han har arbetat 5 år på Nykarlebyföretaget Prevex och där avlagt yrkesexamen som plastmekaniker. Därefter har han också under 5 års tid arbetat på KWH-plast. 2010 övergick han till Mirka  och där genomgått en processutbildning och yrkesexamen inom kemi. För tillfället arbetar han nu på Mirkas avdelning i Jakobstad.

Caroline är utbildad massör och ergoterapeut. Fram till 2007 arbetade hon på Nykarleby sjukhem och sedan dess på Hagalund i Nykarleby.

\begin{jhchildren}
  \item \jhperson{Alizia Milja Louise}{12.12.2009}{}
  \item \jhperson{Leia Emilia Augustina}{01.06.2016}{}
\end{jhchildren}


\jhhouse{Haga}{1:11}{Skog}{1}{300}

\jhoccupant{Wikman}{Lisen}{1969 -}

Lisen, \textborn 20.10.1944 i Kvevlax med efternamnet Kilvik. Hon adopterades av Evert och Agnes Haga som själva var barnlösa. Efter faderns död 1969 förblev egendomen ett dödsbo och jorden utarrenderades. Modern Agnes flyttade till Jakobstad 1977.

Efter detta årtal har ingen bott i huset, som förfallit och slutligen inköptes och revs 2013 av Magnus Jungell. Lisen gifte sig 05.07.1964 med Rolf Harry Wikman från Närpes, \textborn 30.09.1943. Familjen har bott i Närpes.


\jhoccupant{Haga}{Evert \& Agnes}{1930-talet}

Evert Haga, \textborn 28.04.1909, gifte sig 07.07.1935  med Agnes Jungerstam, \textborn 15.04.1911. Lägenheten styckades ut från Skog skattehemman ägd av Hilda och Johan Haga (Huhtala) och bildade en  ny lägenhet på 1930-talet.

Till den nya tomten flyttades ett färdigt timrat hus från Holmen. Det syns på ett fotografi på sidan 282 i ``Historik över Jeppo''. Det var menat till ett hus åt bokhållaren på yllespinneriet vid Kiitola, Johan Elis Sirén, men blev aldrig färdigt. Familjen flyttade 1922 till Gla:Karleby och det halvfärdiga huset köptes av Evert Haga för bortflyttning.Efter att timran plockats ner och och transporterats till tomten timrades det upp på nytt, men med en modifierad takform.

En ladugård byggdes också för några kor, svin och höns. Evert utökade inkomsterna med diversearbete bl.a. på sågen i Silvast och hjälpkarl på
Jeppo-Oravais Handelslags lastbilar som försåg nejden med varor.

Evert \textdied 03.04.1969   --    Agnes \textdied 21.11.1997


\jhhouse{Ankar}{1:54}{Skog}{1}{4, 4a-4b}

\jhoccupant{Jungell}{Magnus}{1983 -}

\jhhousepic{136-05676.jpg}{Magnus Jungell och Kerstin Lindell}

Magnus Waldemar, \textborn 30.07.1955, sambo med Kerstin Lindell, \textborn 27.09.1960 i Bäckby, Esse övertog hemmanet 28.02.1983. Han har genomgått Korsholms lantbruksskola och har satsat på mjölkproduktion fram till 2012 då denna produktionsform upphörde och ersattes av spannmålsproduktion.

Under åren har den odlade arealen kraftigt utökats från ca 15 ha till nuvarande ca 96 ha, dels genom inköp och dels genom nyodling. Också skogsarealen har genomgått samma utveckling, nu 135 ha. Fastigheten som tidigare utgjort en del av Skog R:nr 1 omfattar idag största delen av Skog R:nr 1.

\jhoccupant{Jungell}{Elis \& Dagmar}{1962-1983}

Elis Waldemar, \textborn 05.08.1924, gifte sig 22.05.1955 med Dagmar Alexandra Blomqvist från Kovjoki, \textborn 15.12.1918.

År 1962 den 21 maj hade Elis övertagit hemmanet och under efterkrigstidens uppbyggnadstid hade byggts ny ladugård år 1948 och nytt boningshus byggdes år 1972.

Det gamla bostadshuset som stått på samma tomt och flyttats från Stenbacken 1920 revs nu. Huset hade på Stenbacken stått strax söder om den nuvarande Hembygdsgården och den hade ibland använts för uppbevaring av avlidna och hävdades av många vara hemsökt av spöken, något Elis och familjen ändå inte upplevt.

1975 utvidgades ladugården västerut med en ny flygel för mjölkkorna.

\begin{jhchildren}
  \item \jhperson{\jhbold{Magnus Waldemar}}{30.07.1955}{}
  \item \jhperson{Mona Elisabeth}{16.03.1957}{}
\end{jhchildren}

Elis Waldemar Jungell \textdied 10.07.2006


\jhhouse{Det gamla huset på Ankar}{-}{Skog}{1}{304}

\jhoccupant{Jungell}{Johannes \& Anna}{1900-1962}

\jhhousepic{Magnus Jungells gla hus.jpg}{Jungells gamla hus}

Johannes Jonasson, \textborn 04.11.1877 med efternamnet Romar, gifte sig med Katarina, \textborn 1881. Hon dog 1908 efter dotter Ellen Katarinas födelse 05.01.1908. I det äktenskapet föddes barnen:

\begin{jhchildren}
  \item \jhperson{Johan Villiam}{25.03.1903}{}
  \item \jhperson{Anders Villiam}{03.06.1906}{}
  \item \jhperson{Ellen Katarina}{05.01.1908}{}
\end{jhchildren}

Johannes gifte om sig med Anna Backlund, \textborn 11.10.1886 i en av granngårdarna på Skog, strax på östra sidan av den alldeles nyss dragna järnvägen.

Hemmanet delades med yngre brodern Anders som 1930 sålde sin del till Joel Sandberg (se Skog nr 7). Namnbytet för bröderna från Romar -Jungell skedde troligtvis i början av 1900-talet.Hemmanet hade övertagits efter fadern Jonas död år 1900 och ett bostadshus flyttades 1920 från Stenbacken för att härbärgera familjen.

\begin{jhchildren}
  \item \jhperson{Johannes Edvin}{18.11.1920}{}, se Skog nr 5
  \item \jhperson{Sven Erik}{07.04.1922}{}
  \item \jhperson{\jhbold{Elis Waldemar}}{05.08.1924}{}
  \item \jhperson{Anna Linnea}{12.09.1927}{}
\end{jhchildren}

Johannes Jonasson \textdied 16.04.1949   --   Anna Jakobsdr. \textdied 02.05.1965


\jhoccupant{Jungell}{Jonas \& Lisa}{1880-talet-1900}

Jonas gärning på Skog blev rätt kortvarig då han avled redan år 1900, varefter hemmanet övertogs av äldste sonen Johannes.

\begin{jhchildren}
  \item \jhperson{Johannes}{04.11.1877}{}
  \item \jhperson{Simon}{25.06.1880}{}
  \item \jhperson{Anders}{26.07.1885}{}
  \item \jhperson{Aina Susanna}{17.02.1891}{}
  \item \jhperson{Matias}{18.01.1893}{}
  \item \jhperson{Karl Emil}{25.02.1896}{13.01.1898}
  \item \jhperson{Karl William}{05.05.1899}{}
\end{jhchildren}

Jonas \textdied 14.05.1900  --   Lisa \textdied 10.09.1926


\jhhouse{Ankar}{1:54}{Skog}{1}{5-5a}

\jhoccupant{Jungell}{Edvin}{1945-2011}

Edvin Jungell, \textborn 18.11.1920, var äldst i syskonskaran. Han byggde bostadshuset efter kriget och levde ensam i detta hela sitt liv.

\jhhousepic{139-05681.jpg}{Edvin Jungells hus}

Han deltog naturligtvis som barn och ungdom i gårdens arbete fram till 18 års ålder då han fick arbete som ``krutpojke'' vid Kiitola ammunitionsfabrik hösten 1939. Arbetet gick ut på att tillsammans med 4 andra hjälpkarlar och en chaufför transportera krutlådor och övrigt material från järnvägsstationen till Kiitola, alternativt från Kiitola till de 3 lagerbyggnader som hyrts vid Keppo.

Krutet var förpackat i trälådor, 240 kg tunga. Lådorna skulle lyftas av 2 man i var ända och den 5:e personen skulle rycka in där det sviktade. Kraftprovet inträffade när lastbilen skulle lossas vid lastbryggan vid Kiitola då plattformen befann sig 80 cm högre än lastbilsflaket. Arbetskraft från Oravais deltog ofta i arbetet och det blev inte sällan en tävling mellan arbetslagen vilka som var starkast. Tillsammans med Birger Forslund, också han jeppobo, klarade de sig med den äran i konkurrensen. Edvin var nämligen mycket stark trots sin ganska spensliga kroppsbyggnad.

Ett exempel på detta kunde han ge då det på det s.k. ``låglandet'' på andra sidan landsvägen mittemot inkörsporten till fabriksområdet låg en stor artillerigranat öppet på marken. Ingen visste om den innehöll trotyl eller inte, men den sades väga 250 kg! Alla som trodde sig vara starka måste naturligtvis pröva sina krafter. P.g.a. sin form var den nästan omöjlig att få grepp om. Man måste stiga på knä och rulla granaten över ena handen och underarmen och knäppa ihop händerna bakom granaten om armarna räckte till. Därefter var det dags att pröva lyftet.

- Jag vet, att bara en finne och jag har fått upp den, berättar Edvin inte utan stolthet 70 år senare.

Den 12 mars, dagen före vapenstilleståndet efter vinterkriget skulle han infinna sig till Vasa för värnplikt. Edvin hade ända från födseln haft exeptionellt dålig syn och glasögon tjocka som flaskbottnar. P.g.a. sina ögonproblem kom han hem från värnpliktsutbildningen senhösten 1940. Före midsommaren -41 lastades han i oxvagnar tillsammans med andra reservister och transporterades ner mot Hangö, där en sovjetisk marinbas installerats efter vinterkriget. Där fick han fungera som ordonnans, vilket passade honom utmärkt, Han hade kondition som en häst och tyckte om att orientera sig fram även om det höll på att sluta illa. Då det var svårt att ta sej fram genom sly och taggtrådshinder beslöt han sig en sen kväll att använda järnvägsvallen, det var bekvämare så. Ingen kunde ju ändå se honom. Trodde han. Han såg mynningsflamman från en direktskjutande kanon innan han kände vinddraget bredvid vänster öra före han hörde knallen. ``Det var bara att kasta sig ner i järnvägsdiket då ju nästa skott kunde ha träffat'', berättar han. ``Inte fick vi heller gå in i staden sedan ryssarna jagats bort. Däremot passade nog ett svenskt frivilligkompani på att marschera in som första trupp och ta äran åt sig. Visste du det?''
(Anm. Det skedde på Mannerheims uppmaning och kanske det var en lycka då staden var kraftigt försåtsminerad.)

Efter kriget fick Edvin anställning på Jeppo-Oravais Handelslags magasin vid järnvägsstationen, ett magasin som nybyggdes på 1950-talet och kom att spela en betydande roll i ortens handel med alla de slag av jordbruksprodukter och byggnadsmaterial.

År 1949 beställde Edvin månadsjournalen ``Det Bästa'' från Sverige. Där läste han en gång en berättelse om 2 amerikanska marinkårssoldater som blivit torpederade under 2:a världskriget. De påstod att de hade drunknat om de inte hade haft kontaktlinser! Hade de haft vanliga glasögon, som hade flugit all världens väg vid explosionen, skulle de aldrig hittat de vrakdelar de kunde simma fram till och hålla sig flytande på fram till räddningen. ``Vad är kontaktlinser?'' undrade Edvin. ``Jag ringde till ögonläkare i Jakobstad, men ingen visste vad det var. Samma svar fick jag vid Vasa sjukhus. Småningom tog jag mod till mej och ringde Universitetssjukhuset i Helsingfors. En dam svarade att de nog hade hört om dem, men ingen hade gjort något liknande i Finland. Hon förenade till ansvarige professorn, som bekräftade att man inte gjort någonting sådant ännu i detta land.
  - Skulle Ni vara intresserad?
  - Visst skulle jag det, svarade jag.
  - Kan Ni komma hit nästa vecka ?, frågade han.

- Nästa vecka reste jag till Helsingfors och länge var jag där. Tårarna rann och jag såg ingenting. Man provade och provade och så småningom blev det bättre och till sist blev det riktigt bra! Senare har jag läst att det finns risk för blindhet när man blir gammal. Snart fyller jag 89 0ch på mitt vänstra öga är jag helt blind. På det högra har jag tunnelseende. Det går ännu här hemma, men blir jag helt blind blir det nog värre.''

Hans problem med synen begränsade ändå inte hans rörelsefrihet helt. Om han begav sej på en cykel- eller gångtur kunde han urskilja den vita randen vid vägens kant och mötte han en person kunde han identifiera denna på rösten.

Edvin var sin arbetsplats trogen hela sitt verksamma liv och han är tveklöst en av dem som lyft flest antal ton i detta land under sin livstid.

Ändå finns här en paradox. Samtidigt har han varit en av JIF:s trognaste och pålitligaste poängplockare i de klubbkamper som gick av stapeln under ett par decennier efter kriget. Hans specialitet var medeldistans och speciellt 1500 m. Efter en fysiskt tung arbetsdag där många skulle ha stupat av trötthet, cyklade han hem, tog med sej löpskorna, cyklade tillbaka till sportplanen och knäckte de flesta av sina motståndare med  hård jämn fart. Spurten var hans akilleshäl. - Skidåkning uppskattade han också och deltog gärna i tävlingar med framgång.

Edvin dog den 22.07.2011 på en plats där han trivdes bra – hemma i bastun.


\jhhouse{Ankar}{1:54}{Skog}{1}{6, 6a}

\jhoccupant{Jungell}{Magnus}{2009 -}
\jhperson{Magnus Waldemar Jungell}{30.07.1955}{}, inköpte år 2009 nämnda fastighet, som stått tom efter att Evert Skog avlidit 1981.


\jhoccupant{Sandberg}{Kaj}{1995-2009}
Kaj Sandberg köpte fastigheten på 1990-talet och har tidvis utnyttjat den som förråd.


\jhhousepic{138-05680.jpg}{Evert Skogs hus, nu M. Jungells. Oklart när huset byggdes}

\jhoccupant{Skog}{Evert}{1946-1981}
\jhperson{Anders Evert Skog}{05.03.1912}{}, fick som ende sonen tidigt axla ansvaret för hemmanet och under kriget tillsammans med sin syster Ester. Efter att hon 01.02.1948 gift sig med Svante Haglund, \textborn 22.10.1915, bodde hon med sin familj i samma hus och här har barnen tillbringat sin barndom.
\begin{jhchildren}
  \item \jhperson{Bo Anders}{28.04.1948}{}
  \item \jhperson{Stig Göran}{18.04.1949}{}
  \item \jhperson{Karl-erik}{30.05.1951}{}
  \item \jhperson{Barbro Ester Katarina}{30.08.1953}{}
  \item \jhperson{Elvi Heldine Elisabeth}{30.04.1958}{}
\end{jhchildren}

Familjen flyttade år 1963 till sitt inköpta hem vid Nylandsvägen (se Romar nr 30). Evert fortsatte efter denna tidpunkt ensam driva jordbruket i liten skala. Han dog 29.09.1981.


\jhoccupant{Skog}{Thomas \& Katarina}{1899-1946}
\jhperson{Thomas Thomasson}{15.03.1877}{} gifte sig 09.11.1897 med ,\jhperson{Katarina Andersdr. Måtar}{01.11.1874}{}. Till en början bodde de som inhyses på Romar hos hans far Thomas Johansson innan de erhöll ½ av hemmanet på Skog, som fram till dess ägts av Fredrik Johanss. och hans hustru Anna.

Livet på Skog var strävsamt och Thomas och Katarina fick 12 barn varav 6 dog i låg ålder:
\begin{jhchildren}
  \item \jhperson{Irene}{1898}{1899}
  \item \jhperson{Johannes}{1905}{1905}
  \item \jhperson{Johannes}{1908}{1909}
  \item \jhperson{Gunnar}{1910}{1913}
  \item \jhperson{Thomas}{1914}{1914}
  \item \jhperson{Elis}{1916}{1916}
  \item \jhperson{Ellen Katarina}{28,10.1899}{}
  \item \jhperson{Helga Maria}{27.08.1907}{}
  \item \jhperson{Edit Susanna}{06.04.1910}{}
  \item \jhperson{\jhbold{Anders Evert}}{05.03.1912}{}
  \item \jhperson{Anna Elvira}{16.05.1913}{}
  \item \jhperson{Ester Heldine}{09.05.1918}{}
\end{jhchildren}

Katarina \textdied 15.08.1929 -- Thomas \textdied 28.04.1946

\jhoccupant{Johansson}{Fredrik \& Anna}{1880-1899}
Fredrik Johansson Skog, \textborn 11.05.1853, gifte sig med Anna Eriksdr. Grötas,  \textborn 1850, och hade 1880 övertagit hemmanet av föräldrarna Johan och Maria Hansson, \textborn 1814 resp. 1817. Fredrik besökte 1892-1894 USA, vilket småningom fick återverkningar för hela familjen.

I familjen föddes 10 barn av vilka 8 nådde vuxen ålder och de 2 yngsta, Emil \textborn 1891 och Aina \textborn 1893, dog i späd ålder.
\begin{jhchildren}
  \item \jhperson{Johannes}{1874}{}
  \item \jhperson{Fredrik}{1875}{}
  \item \jhperson{Erik}{1877}{}
  \item \jhperson{Anders}{1879}{}
  \item \jhperson{Joel}{1882}{}
  \item \jhperson{Anna}{1884}{}
  \item \jhperson{Ida}{1886}{}
  \item \jhperson{Selma}{1888}{}
\end{jhchildren}

Medan de yngre barnen föddes drogs järnvägsspåret öster om bebyggelsen på Skog och delade hemmanets ägor. Det är ändå oklart om detta faktum inverkade på att ALLA familjens barn emigrerade till USA. Den 3 april 1909 ansökte också Fredrik och Anna om utresa. Tydligen insjuknade Anna därborta och hon står antecknad som död i Amerika 24.10.1909.

Fredrik återvände ensam och 18.03.1911 tog han ut lysning med Helena Sofia Sundholm \textborn 14.10.1847 på Kaup. Den 3 april samma år gifte de sig och flyttade till Finskas nr 5 där de fram till sin död 1921, resp. 1930 levde som backstugusittare.


\jhhouse{Hemfors}{1:26}{Skog}{1}{7, 7a-7b}

\jhoccupant{Aalto}{Teemu \& Jasmine}{2009 -}

Teemu Aalto, \textborn 20.01.1986, gift med Jasmin Viljaranta från J:stad, \textborn 19.12.1983, köpte fastigheten år 2009. I köpet ingick endast tomten inklusive byggnaderna.


\jhhousepic{137-05679.jpg}{Teemu Aalto och Jasmin Viljaranta}

Teemu är utbildad timmerman och har haft anställning bl.a. hos Brage Finskas och HT-Lasertekniikka i Alahärmä. För tillfället har han anställning vid Skaala Ikkuna i Alahärmä. Jasmin är utbildad kock och arbetat på ABC-stationen i Oravais. Numera är hon innehavare av och driver Café Funkis.
\begin{jhchildren}
  \item \jhperson{Jonny}{10.05.2006}{}
  \item \jhperson{Jimi}{04.12.2008}{}
  \item \jhperson{Jasper}{08.12.2010}{}
\end{jhchildren}


\jhoccupant{Sandberg}{Kaj}{1991-2009}

Kaj Mats Sandberg,  \textborn 31.12.1967, övertog hemmanet år 1991, men ekonomiska svårigheter gjorde att det gick förlorat 2009.


\jhoccupant{Sandberg}{Carl-Gustav \& Margareta}{1967-1991}

Carl-Gustav, \textborn 12.10.1935, gifte sig 04.11.1961 med Viola Margareta Wärnå från Esse, \textborn 22.11.1936. De övertog Hemfors lägenhet av Carl-Gustavs föräldrar år 1967. Lägenhetens huvudsakliga inriktning var mjölkproduktion vilken krävde makarnas fulla engagemang. Ladugården har renoverats och byggt ut i olika repriser under årens lopp.

Carl-Gustav har varit en samhällsinriktad person och deltagit i olika föreningars aktiviteter. Bl.a fungerade han under flera år som
ordförande för Jeppo lokalavdelning av ÖSP.
\begin{jhchildren}
  \item \jhperson{Gunilla Margareta}{03.08.1964}{}
  \item \jhperson{Birgitta Irene}{02.04.1966}{}
  \item \jhperson{\jhbold{Kaj Mats}}{31.12.1967}{}
  \item \jhperson{Kristina Viveka}{27.04.1970}{}
\end{jhchildren}


\jhhouse{Hemfors}{1:5}{Skog}{1}{307}

\jhoccupant{Sandberg}{Joel \& Elna}{1930-1967}
Joel Sandberg, \textborn 11.07.1900, gifte sig 14.0.1932 med Elna Irene Bäckstrand, \textborn 23.03.1908.

År 1930 köpte Joel Sandberg Hemfors lägenhet R:nr 1:5 av Anders Jungell. I köpet ingick de byggnader som fanns på lägenheten. Ny bostad byggdes åren 1959-60. Mjölkproduktion var huvudinriktning på gården, men Elna var också lärare på Jungar folkskola åren 1946-66 utöver de 2½ åren som vikarie under krigsåren.
\begin{jhchildren}
  \item \jhperson{Carita Magdalena}{23.03.1933}{}
  \item \jhperson{\jhbold{Carl-Gustav}}{12.10.1935}{}
  \item \jhperson{Erik Markus}{29.06.1945}{}
  \item \jhperson{Martin Joel}{29.06.1945}{}
\end{jhchildren}


\jhoccupant{Jungell}{Anders \& Olivia}{1920-1930}

Anders Jungell, \textborn 26.07.1885 på Skog, gifte sig  29.04.1910 i USA med Olivia Ersfolk, \textborn 09.08.1887 i Övermark. Efter hemkomsten från USA uttogs lysning den 25 jan. 1920 för bekräftelse av äktenskapet. Nu delades hemmanet med äldre brodern Johannes varvid Anders och Olivia tog som fosterföräldrar hand om Johannes dotter från första giftet, Ellen Katarina, \textborn 05.01.1908, som senare gifte sig med Leander Sandberg (se nr 407).

1929 flyttade de till Silvast och 1930 sålde hemmanet till Joel Sandberg. I Silvast uppförde de den byggnad bredvid Ungdomslokalen som skulle bli det sista gästgiveriet i Jeppo (se nr 480).


\jhoccupant{Jungell (Romar)}{Jonas \& Lisa}{1880-talet-1900}

Jonas Simonsson Romar, \textborn 28.03.1851, gift med Lisa Johansdr., \textborn 22.11.1856. De har i mantalslängden för 1890 antagit efternamnet \jhbold{Jungell}. Jonas' gärning på Skog blev ganska kortvarig  då han avled redan år 1900 (se nr 4). Det är osäkert om han byggt detta hus.



\jhhouse{Central}{1:9 / 1:4}{Skog / Haga}{4}{68, 368}

\jhoccupant{Telia/Sonera}{Finska staten}{1957 -}

Efter undersökningen 1954-1955 och underhandlingar 1956-1957 beslöt	Jeppo Telefon Ab att överlåta telefonbolaget till Finska staten. Den 29.08.1957 blev därmed Finska staten ägare till fastigheten Central 1:9 med	telefoncentralbyggnaden. Luftledningarna avskaffades och kabelsystem infördes. Samtidigt infördes en andra central, som betjänade de sydligaste delarna av Jungar och Överjeppo byar.

Vid centralen vid \jhbold{Sparvbacken}, se karta 4, turades 5-6 flickor om att förena samtalen till bygdens folk.	Vev-telefonen ersätts 1972 och batteriet på väggen i hemmen kan monteras ner. Efter 20 år, 1977, automatiserades telefonväsendet. Den	gamla centralbyggnaden (nr 368) revs och en mindre telebyggnad och en telefonmast uppfördes på tomten. Idag är telefon-, TV- och dataväsendet digitaliserat.

\jhhousepic{062-05596.jpg}{Den gamla bemannade centralen är ersatt med barack och ny teknik}



\jhoccupant{Jeppo Telefon}{Ab}{1921-1957}

Den 19.07.1921 höll Jeppo Telefon Ab en extra bolagsstämma.	Närvarande var A. Sandqvist, R. Husberg, J. Wörlund, T. Westerlund,	O. Sundell, J.T. Backlund, A.W. Finskas, G. Liljeqvist och D. Jungar.	Beslöts att inköpa Anders Skogs ägande backstugområde av Skog hemman nr 1 i Jungar by. Köpebrevet upprättades 16.08.1921, för ett	pris på 6565 mark, för ett utbrutet område med boningsstuga och uthus. Telefoncentralen installerades i den inköpta gården. Anders Skog hade byggt en stuga på området och 1912 inlämnat ansökan om att få inlösa sitt område till självständig lägenhet, styckning slutförd 27.10.1920. År 1925 infördes i jordregistret Central 1:9 om 0,185 ha av lägenhet Haga 1:4  av Skog hemman.
\jhhousepic{Jeppo telefoncentral 1968.png}{Jeppo telefoncentral 1968, nr 368}
Redan på 1890-talet började man i Jeppo undersöka möjligheterna till att få telefonförbindelser. Redan 1891 börjar telefonstolpar resa sig mellan Nykarleby och Jeppo och 1895 är telefonledningen färdig att anlitas. De första telefonerna hade prästen, postkontoret, Sundells Bageri, Modéns Handelsbutik, Keppo gård, Kiitola, Lantmannagillet, poliskonstapeln och Jungar folkskola. Före inbördeskriget var det glest med telefoner; år 1921 var antalet 15 abonnenter med 25 andelar.
Direktörerna under tiden för Jeppo Telefon Ab var:
\begin{center}
  \begin{tabular}{l l}
    \hline
    Namn & Period \\ \hline
    Jungar Daniel & 1921-1934 \\
    Westerlund Thure & 1935-1937 \\
    Jungell Anders & 1938-1941 \\
    Wistbacka Edvin & 1942 \\
    Ström Georg & 1943-1945 \\
    Jungerstam E.W. & 1946-1948 \\
    Backlund Valter & 1949-1950 \\
    Sundell Torsten & 1951-1954 \\
    Romar Selim & 1955-1960 \\ \hline
  \end{tabular}
\end{center}

\jhhousepic{Jeppo telefoncentral 1968-C.jpg}{Christina Sundell/Simons kopplar kontakterna}
Telefonisterna, ``Central-flickorna'' har varit många under årens lopp. De tre första bodde i huset. I slutet av 1940-talet blev det skiftesarbete med möjlighet att sova på natten med väckning vid telefonringningar. Centralföreståndarna tjänstgjorde i många år medan -flickorna gjorde det i längre eller kortare perioder. En del av flickorna gifte sig under anställningstiden och fick nytt efternamn.
\begin{center}
  \begin{tabular}{l l l}
    \hline
    Boende el. föreståndare & Telefonist & Telefonist\\ \hline
    Mattsson Sanna-Maria & Huhtala Margaretha & Norrgård Gun \\
    Kronlund Göta & Sundell Ulla & Back Gun-Lis \\
    Sandqvist Hjördis & Sandqvist Eva & Simanainen Pirkko \\
    Johansson Else-Maj & Lindström Gunnel & Elenius Gun-Lis \\
    Sundell/Simons Christina & Lindström Margit & Strengell Gundel \\
    - & Björkqvist Anita & Kennola Anna-Liisa \\
    Sandqvist Marita t:fonist & Kula Anita & Kennola Marja-Liisa\\ \hline
  \end{tabular}
\end{center}



\jhhouse{Tvättstränderna vid ån}{0:0}{Alla}{1-20}{blått tecken}

Tvättmaskinerna i den form vi känner dem är inte gamla, men när de kom betydde de en revolution. Speciellt för kvinnorna.

Under århundraden tidigare hade klädtvätt inneburit ett helt annat slit. Visserligen var mängden material, antingen i form av tyg eller ylle, betydligt mindre än i dagens moderna samhälle, men mängden människor i ett hushåll var betydligt större. Likaså var materialet
inte lika lätthanterat och alltid mycket smutsigare.

Vintertid kokades kläderna  i en gryta hemmavid, placerades i en träså där en bärstång träddes genom handtagen och bars ner till ån. Väl där sattes den ner bredvid den vak som husbonden i bästa fall bemödat sig om att hugga opp. Ärmarna kavlades upp, kvinnorna steg på knä bredvid vaken med ett torrt plagg under knäna och sköljningen vidtog. De varma klädesplaggen doppades ner i det iskalla vattnet upprepade gånger, vreds ur och doppades på nytt och vreds ur. Gång efter gång. Plagg efter plagg. Kvinnornas armar blev rödare och rödare av växlingen mellan kallt och varmt. Men var man flera, vilket man ofta var, hördes skratt och glam från arbetet på isen. Och slutligen var det klart. Såstången träddes i och nöjda vandrade kvinnorna hemåt.

Men helst utfördes arbetet under sommartid. Det gjorde att tvätten sparades så länge som möjligt under vinterhalvåret. När isen gått och temperaturen stigit var det dags igen. Men nu kunde man inte tvätta som under vintern. Man måste finna en plats där det var lätt att ta sej ner längs slänten, samtidigt som det måste finnas en möjligast plan plats att vistas på. Det allmänna behovet av en sådan plats gjorde att det vid lantmäteriförrättningarna skapades servitut (ett juridiskt begrepp som ger en fastighet rätt att utnyttja en annan fastighet), där lämpliga platser för klädtvätt bestämdes. Inte endast platsen nere vid ån bildade servitutet, utan också vägen till platsen skiftades som ett servitut. Detta möjliggjorde att personer från ett eller flera hemmansnummer kunde utnyttja platsen utan att den ursprungliga markägaren kunde hindra detta.

Bykgrytorna anskaffades ofta gemensamt och placerades på en enkel eldstad av natursten. Ved måste man själv ha med, men närheten till vatten i obegränsade mängder gjorde arbetet lättare och roligare och en speciell doft spred sig från tvättstranden ut i omgivningen. När vintern närmade sig och den sista tvätten var över, stjälptes grytan upp och ner för att inte frysa sönder i väntan på en ny säsong.

Idag har dessa platser förlorat sin betydelse, och vid olika lantmäteriprocesser har deras servituträtt avlägsnats, och området på nytt infogats i det ursprungliga hemmanet. Platserna kvarstår ändå i folks minnen och de finns utpekade på de kartor som finns bifogade:
\begin{enumerate}
  \item Skog: stranden låg vid s.k. Jannes-juutån, nu utfylld strax söder om gård nr 2
  \item Romar: fanns i huvudsak nedanför Romar-Jannes hus, i höjd med Ralf o Svea Romar
  \item Silvast / Fors: nedanför Jeppo Krafts kraftstation, nu är den vackra stranden utfylld
  \item Grötas: nedanför och strax norrom torkrian
  \item Böös (Bösas)
  \item Jungar / Ruotsala
  \item Mietala / Gunnar
  \item Tollikko
\end{enumerate}
