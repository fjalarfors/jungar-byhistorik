\jhchapter{Foo}

\jhhouse{Kvarnbacken}{Kvarnbacken 3140}{Jungar by norra, hus 97 + 97 a-e 5.11.2013/21.2.2014/16.12.2016/25.12.2016}

\jhoccupant{Fors}{Fjalar \& Mayvor}{1995}

\jhperson{Fjalar Rainer}{11.04.1949}{} i Jeppo, agronom, gifte sig 26.07.1975 med \jhperson{Mayvor Margareta}{10.03.1949}{}, född Lillqvist, laborant, jordbrukardotter från Purmo.

\begin{jhchildren}
  \item \jhperson{Joakim Gustav Andreas}{16.02.1982}{}, dipl.ing., fam. i Sverige
  \item \jhperson{Daniel Rainer Alfred}{14.08.1985}{}, (dipl.ing.)
\end{jhchildren}

\jhhousepic{DSC05875.jpeg}{Kiitolavägen 15, år 2010. (Gård 97)}{97}

Sedan januari 1995 har paret Fors ordnat livet på denna lägenhet efter att först ha arrangerat, flyttat, sanerat och byggt till huset på Kraftgränd 4 (gård 97a). Åtgärderna gjordes innan familjen kunde dra upp bopålarna från Tottesund i Maxmo och placera dem vid Fjalars barndomshem på Kiitolavägen 15. Orsaken till flytten var att Fjalar sedan augusti 1990 innehade tjänsten som rektor vid Lannäslunds skolor (nu Optima Lannäslund) i Jakobstad och ville komma närmare sin arbetsplats.

Fjalars arbetsplatser: Korsholms skolor (linjeföreståndare 8 år), Finlands Pälsdjursuppfödares Förbund (försöksfarmchef Maxmo 5 år), Lannäslunds skolor (senare Lannäslundskolan Optima resp. Optima Lannäslund 19 år) och Projektkoordinator (Optima 4,5 år).

Mayvors dito: Laborant vid Schaumans fabriker, Östanlid sjukhus, preparator vid Tekniska högskolan i Otnäs, laborant vid utvecklings- laboratoriet vid Dickursby färgfabriker och vid Solf Centralbutiks spannmålslaboratorium, kontorist på Maxmo Försöksfarm samt byråfunktionär vid länsmanskansliet i Vörå.

Kvarnbacken 3140 bildades medels köp 5.8.1995 av delar av Broända 3109 och Strand 3130, inom vilka den tidigare Mejerivägen 314 ingått. Området är enhetligt på så vis att fastigheten i dagsläget omsluts av Kiitolavägen, Kraftgränd och Lappo å. På tomten finns i dag:
\begin{enumerate}
  \item en kvarnbyggnad (hus 97 d, uppförd vid sekelskiftet 1800-1900),
  \item ett bostadshus i trä/stock från 1800-talet (hus 97 a), som vid två tidpunkter flyttats (1960, 1994), restaurerats (1994) på Kraftgränd 4,
  \item ladugårdsbyggnad (1935, hus 97 e),
  \item gårdsbyggnad (ca 1950, hus 97 b),
  \item stock-/fritidshus (1999, hus 97 c), samt
  \item bostadshus i vitt tegel (hus 97, uppfört 1961-62) på Kiitolavägen 15.
\end{enumerate}
Fram till i dag pratar gårdsfolk och släktingar om ``Krymboas'', ett namn härlett ur benämningen ”kronobyboas”, efter Jakob Jakobsson, som bodde här vid slutet av 1600-talet, då man refererade till denna plats. - Hildur Fors bor från jan. -95 i huset på Kraftgränd 4.

Ett landmärke för lägenheten var flaggstången, som restes av Gustaf Fors och Leo Julin fredag kväll den 9 oktober 1936. Det var viktigt att få upp den i tid före bröllopet mellan syster Anna (Fors) och Leo Julin, vilket firades söndagen den 11 oktober. - Vid vigseln närvar 800 gäster, som kommit till kyrkan p.g.a. ryktet att Anna skulle bära den sällan sedda storkronan. Den kom ändå inte till användning p.g.a dess tyngd. - Den slanka furan hade med viss möda forslats genom krökarna från Kampasskogarna, liggande på dubbla stöttingar. Längden var då 22 m; senare förkortad från ”skatan och loman” i två repriser till ca 18 m. Flaggstången blickade över Silvast i 76 års tid, fram till natten 17-18 september 2012, då en kraftig stormby lade den till ro över fähustaket. Den 13.1.1997 startade paret Fors och Stig-Olof Lillqvist från Purmo

Kvarnbackens gymnastikförening r.f. för att ge egna och andras barn ordnade möjligheter till nyttig fysisk aktivitet. Då byn saknade goda utrymmen för redskapsgymnastik, tog Fjalar och Mayvor itu med att vintern 1997-98 ändra den gamla skulltorken/höladan till en värmeisolerad fungerande gymnastikhall, ”Arcas”, med tillgång till alla redskap som fordrades för träning och tävling i denna disciplin. Som mest hyste föreningen ett 50-tal aktiva barn från närregionen. För sitt engagemang valdes Mayvor år 2002 till ``Årets Jeppobo''. Under verksamhetssäsongen september-juni var Arcas i användning tre kvällar per vecka fram till hösten 2008.

Verksamheten inom s.k. tredje sektorn har också i övrigt intresserat på olika sätt. Fjalar blev medlem i Lions Club Jeppo hösten 1995 och har innehaft olika poster i den, bl.a. två perioder som president och tre dito som sekreterare. År 2003 var han med och initierade en nystart av Jeppo byaråd r.f.-Jepuan kyläyhdistys r.y., sedan hösten 2016 Jeppo byaförening r.f., vars ordförande han varit 2003-06 och 2009-17.

Fjalar och Mayvor fullföljde 2013-2014 byarådets plan att anlägga en vandringsled i Jeppo. ``Trådi'' kunde invigas 1 maj 2014 och paret fick senare Nykarleby stads idrottsdiplom ”Årets talkoarbetare 2014”.


\jhhouse{Broända gård}{foo}{bar}

\jhoccupant{Fors}{Gustaf \& Hildur}{1945 – 1995 (-79)}

\jhperson{Gustaf Alfred}{10.04.1949}{23.11.1992}, gifte sig den 21.10.1945 med \jhperson{Hildur Anna Maria}{17.10.1925}{}, född Engström från Jeppo (Lassila).
Barn:
\begin{jhchildren}
  \item \jhperson{Per-Håkan Gustaf}{18.09.1946}{}, instrumentmekaniker
  \item \jhperson{Christer Anders Alfred}{30.09.1947}{}, jordbrukare
  \item \jhperson{Fjalar Rainer}{11.04.1949}{}, agronom
  \item \jhperson{Kerstin Maria}{04.05.1951}{}, grafiker
\end{jhchildren}

Gustaf och Hildur övertog den 31.12.1945 en relativt modern lantbrukslägenhet (Broända gård) som Gustaf själv varit med om att bygga upp sedan tonåren, i gott samspel med sina fyra systrar, sin far Emil och mor Ida. Gustaf var intresserad av jordbruket, vilket bekräftas av att han gärna genomförde utbildningen i Gamla Vasa vid Korsholms lantmannaskola och tog med sig nyheter från den. Bland annat testade han majsodling och initierade en ladugårdstillbyggnad i form av AIV-torn för ensilage i anslutning till det 10 år gamla fähuset.

Gustafs medverkan i kriget 1939-44 blev, liksom för många andra i samma ålder, en ovälkommen parentes för den drivkraft som fyllde ynglingens väsen. Att komma helskinnad från fronten var ingen självklarhet, men en lycka i sig och för dem som väntade där hemma.

Lägenheten omfattade vid övertagandet ca 79 ha; trädgård 0,02 ha, åker 26,50 ha, skog 45,00 ha och övrig jord 7,48 ha. Av åkermarken var ca 7 ha täckdikad. Den relativt nya ladugården hade rum för 12 klavbundna mjölkkor, nödiga kättplatser, 3 hästar, en handfull s. k. baconsvin och några får för gårdens behov. Hönsskötseln hade avslutats tio år tidigare.

Vid övertagandet löstes systrarna Jenny, Gemima, Anna och Etel ut, varvid Mellanåkern, mellan mejeriet och Stenvall, samt en hustomt vid Harisloon för Gemima och Lennart Gustavsson, styckades från gården.

\jhpic{Broanda ForsIMGP2476.jpg}

Hemmanet utvecklades i takt med tiden. Inriktningen var mjölk- och foderproduktion, vilket bl.a. medförde behov av omändringar i befintliga gårdsbyggnader. En kalluftstork för spannmål inrättades ovanför redskapslidret (kärrladan). Höladan, mjölboden och vedlidret gav plats för en skulltork för hö. För att ge rum för dubbelt fler kor än vad ursprunget var planerat för, gjordes nödvändiga omdisponeringar i fähuset. Får, svin och hästar hade utmönstrats.

Mekaniseringen gjorde fortgående landvinningar, uttryckligen i och med inköpet av den första traktorn, en Ferguson (”Grålle”) år 1952.

Gustav engagerade sig under sin mest aktiva tid i organisationsliv och samhällsbygge. Han var ordförande i Jeppo Uf genast efter kriget, 1945. Senare satsade han mycket tid som ordförande i flera andra sammanhang; lokalavd. av Österbottens Svenska Producentförbund, Österbottens Kötts förvaltningsråd, Jeppo kommunalfullmäktige 1966-72, byggnadskommittén för Jeppo Kommunal- och hälsogård samt Jeppo Hembygdsförening.

Hildur har varit mångårig, intresserad medlem i Jeppo Martha- förening. Mode och inredning har varit andra intresseområden.

Vid generationsskiftet 12.09.1979 till Christer och Runa Fors (född Kåll), behöll Gustaf och Hildur den outbrutna delen av tomten mellan älven och ladugården på vilken bostadshuset står.

\jhoccupant{Rijf/Fors}{Emil \& Ida}{1911/-23 – 1945}

Emil Alfred Alfredsson Rijf/Fors, \textborn 15.5.1883 - \textdied 22.03.1955, gift med Ida Gustava Pettersson, \textborn 10.1.1886 - \textdied 08.081934, född Holm, Jeppo. Barn: a) Jenny Gustava, \textborn01.03.1909 - \textdied 07.07.1989 (Karlsson) Sverige
b) Gemima Alfrida, \textborn 08.07.1912 - \textdied 06.06.1999 (Gustavsson) c) Anna Viola, \textborn 29.10.1916 – \textdied 08.05.1967 (Julin)
d) Gustaf Alfred, \textborn 10.04.1919 - \textdied 23.11.1992
e) Etel Karin Alice, \textborn 28.02.1924 - \textdied 22.7.2009 (Albäck)

Emil var född i Alahärmä. Enligt uppgift var han två gånger till USA, 1902-04 och 1905. Den andra gången, i sällskap med Anders Jungarå och Johan Henriksson Grötas, reste han för att leta reda på sin far. Han hittade honom på en skogscamp i Ashland, Wisconsin, där Katarina Sandberg (gift Sikström; kallad Klockar-Katrin) kunde sammanföra fadern med sonen. De två hade arbe- tat en tid på samma camp utan att veta om varandra. Emil lär ha gett sin far ett kok stryk för att han lämnat

Infarten till gården ca 1935 (gård 397). T.h. familjen i sticket och beordrade honom därefter att resa syns gaveln av sytningsstugan (se gård 97a). hem.

Det är oklart i vilket skede Emil ändrade sitt efternamn från Rijf, via Johansson i Amerika, till Fors, men det bör ha skett ca 1910.

År 1911 köpte Emil av sin mor halva Strand skattehemman no 315 och ärvde den andra halvan. Tolv år senare, 1923, köpte han hela Broända no 36 av sin mor. De hittills sambrukade lägenheterna gick under namnet Broända gård. Nytt fähus för 12 kor byggdes år 1935; det står ännu kvar.

Emils yngre bror Johan (”Jukka”, ”Jan”, ”Krymboas-Janne”) var förmodligen tilltänkt som blivande hemmansägare på Krymboas. Liksom många andra gjorde han sin egen emigrantresa till Amerika, där han råkade ut för ett gruvras i (Vancouver-trakten) tillsammans med en Munsala-yngling. Olyckan höll honom instängd med sin döende kamrat i tre dygn, vilket kom att prägla Johans mentala hälsa i fortsättningen. Han skickades hem, där hans mor måste kvittera ut honom på farstutrappan. Johan bodde på gården fram till sin död.

\jhoccupant{Grötas/Rijf}{Sanna Lisa \& Alfred}{1904 – 1911/-23}

Sanna Lisa Karls(Johans-)dotter Grötas, \textborn 12.10.1857 – †22.02.1927 , gift med Alfred Johansson Rijf, \textborn 01.12.1856 – †23.09.1925 från Alahärmä.
Barn: a) Johannes Alfredsson Rijf, \textborn 08.12.1881 i Markkula, Alahärmä
b) Emil Alfredsson Rijf/Fors, \textborn 15.04.1883 i Alahärmä
c) Hilda Maria Alfredsdotter Rijf, \textborn 25.03.1885, gift Moisio d) Johan Valfrid Alfredsson Rijf, \textborn 10.06.1886 \textdied 29.11.1954
Sanna Lisa var uppvuxen på Fors hemman. Giftermålet förde henne till Markkula i Alahärmä, där svärfadern Johan Rijf, ingift från Hirvlax, hade ett litet torp. Torpet räckte inte till att försörja familjen, så han drygade ut utkomsten med sina skickliga kunskaper som bettsmed.
Sanna Lisa och Alfred flyttade snart efter Emils födelse till Sverige, där de förtjänade sitt levebröd i några års tid. Då beskedet från Jeppo kom, att hemmanet på Fors överförts på Sanna Lisa den 22.8.1904, styrdes flyttlasset tillbaka till Finland. Lägenheten på 7/96 mantal av Fors skattehemman (Broända) var en gåva av Sanna Lisas föräldrar, därför att hennes bror, kallad Stöipas-Erik, inte var betrodd arvtagare p.g.a. sitt missbruk av alkohol.
Maken Alfred, en hetlevrad person, lämnade år 1902 sin familj för ett mera äventyrligt liv i Amerika. Han återkom, men drog ånyo iväg i februari 1905, denna gång med dottern Hilda i sitt följe. Tilltaget verkade inte vara sanktionerat till alla delar och vistelsen blev heller inte långvarig, för sonen Emil reste efter, fann honom och såg till att han kom hem igen för att ta sitt ansvar för familjen och hemmanet.

Gården tjänade som gästgiveri ca 1891-1928. Den bistod med ungefär 200 skjutsar per år. En gäst har sänt sitt tack-kort: ”Til minde av ingeniør Gustav Keugl”. På bilden ses Johan Valfrid Alfredsson Rijf, okänd NN, Gustav Keugl,

\jhoccupant{Fors}{Karl Johan \& Greta Lisa}{1877/94 – 1904}

Sanna Lisa Karls-(Johans)dotter Rijf, Jenny Fors 3 år, smeden/gjutaren Karl Johan Abrahamsson Rijf och Ida Gustava Fors. Foto år 1912. (Gård 397)
Karl Johan Abrahamsson, \textborn 17.07.1834 (i Grötas) - \textdied 28.02.1918 (i Fors) gift med Greta Lisa Johansdotter, \textborn 03.09.1833 (i Soklot) - \textdied 04.03.1900 (i Fors), född Sundlin på Mannfors.
Barn: a) Sanna Lisa Karlsdotter Grötas, \textborn 12.10.1857 i Grötas
b) Karl Johannes Karlsson Fors, \textborn 19.04.1860 – \textdied 21.06.1867
c) Erik Karlsson Fors, \textborn 19.02.1863 - \textdied 24.3.1933 (Stöipas-Erik) d) Anna Sofia Karlsdotter Grötas, \textborn 07.08.1868
e) Anna Lovisa Karlsdotter Grötas, \textborn 08.07.1874 (gift Forsblom)

Karl Johan A. var en utomordentligt skicklig gjutare med gott rykte. I hans produktsortiment ingick bjällror, klockor, m.m. Han var även rutinerad, god smed.

Sonen Stöipas-Erik (c)) lärde sig smedens yrke. Han bodde på den plats där Johan Modén, och efter honom Eliel Brunell, senare bedrev butiksverksamhet. Erik skulle egentligen ha fått hemmanet, men på grund av hans problem med alkoholen gav föräldrarna lägenheten på Fors till dottern Sanna Lisa, gift med Alfred Rijf.

Karl Johan innehade lägenheten Broända från år 1877. Han blev m.a.o. den person som förenade lägenheterna Strand och Broända, varefter hemmanet hållits i samma släkte.

\jhoccupant{Forss}{Matts Nilsson}{1865 – 1877}
Fors/Broända----
Fors/Strand
Jeppo Kraft Alg
1952 – 1965
Brandt Evert Evert \& Saga 1948-1952
Ekström
Emil \& Hilda 1933 – 1948
Fors
Karl Johan 1894 – 1904
Forss
Thomas Johansson 1838 – 1865
Forss
Johan Thomasson 1826 – 1838
   Broända 36
Strand 315
3109 Gård 397
3130 Gård 397a

Andelslaget använde huset som kontor fram till 1956 och senare även som paus-/matrum och lager för allehanda elmaterial.
Anders Evert \textborn 1922 och Saga Linnea \textborn 1923, född Loman, gifte sig samma år som Evert blev ny disponent för andelslaget och paret fick sitt första barn.
Barn: a) Stig Björn Anders \textborn 1948 bor i Jakobstad
b) Hans Evert \textborn 1955 bor i Jakobstad
Paret med sin förstfödda bodde några år i huset men flyttade sedan till gård nr 93 för en kort tid innan lasset drog vidare till Jakobstad.
Emil var disponent på Jeppo Kvarn och Sågverksandelslag 1920-1948 (se Silvast kvarn) men flyttade in i huset först 1933 då han sålt sin del av hemmanet i Mietala (gård NNN) till Johannes Åstrand och flyttade med familjen till Fors.
Inga uppgifter om när huset, s.k. mjölnarstugan, byggdes finns att tillgå. Förmodligen var det en 1800-talsbyggnad. Den revs år 1992.
Karl Johan Abrahamsson \& Greta Lisa Johansdotter (se ovan) köpte lägenheten Strand på 7/96 mantal av Fors skattehemman 3 och kom således att föra samman Strand och Broända lägenheter. År 1904 skrev
Heikkilä
Erik Johansson 1894 – 1894
Männikkö
Mats
1887 – 1894
Lindén
Anders 1887 – 1886
Fors/Strand------
Jakobsson
Hans \& Susanna 1719 – ?
Jakobsson
Daniel \& Margareta 1713 – 1719
1713 –
1709 – 1712
Jakobsson
Jakob \& Maria 1696/-99 – 1709
Jakobsson
Jakob \& Margareta 1686 – 1696
Tomasson
Karl-Johan över lägenheten på dottern Sanna Lisa mot sytning fram till sin död 1918.
Erik, från Alahärmä, var i borgen för föregående Matts Männikkö. Matts rymde från sitt hemman en natt och sålunda kom Heikkilä att få det men sålde det vidare samma år.
Mats, från Alahärmä, köpte lägenheten av folkskolläraren Anders Lindén (1887).
Fors
Thomas Thomasson 1886 – 1886
Strand 315
Hans, gift med Susanna Mattsdotter ..... Hans var bror med Daniel (nedan)
Lassila
Jakob Jakobsson 1877 – 1886
   Johan 1694 –
1683 1675 – 1669 – 1664 – 1625 – 1619 –
1699
1681
1667 ca 1657
1624
Daniel, gift med Margareta Tomasdotter .....
Änkan till Matts Mågen Matts
Jakob är inflyttad från Kronoby, gift med Maria .....
Jakob köpte Jakob Jakobssons (nedan) del, 5/8 mtl, och sedan löst in Tomas (Hanssons) del. År 1699 äger han därmed hela Lillsilvast hemman. Jakob är förmodligen orsaken till att man ännu i dag talar om ”Krymboas” (kronobybons hemman)
Jakob, gift med Margareta Mattsdotter .....
Hemmanet är kluvet 1694 så att Jakob äger 5/8 mtl. Resterande del, 3/8 mtl, ägs i detta skede av nedanstående Johan Tomasson. Jakob säljer sin del 1696.
Johan äger 3/8 mtl, som han år 1699 säljer till kronobybon Jakob Jakobsson. [Man kan anta att Johan är son till Tomas Hansson (nedan)?]
Änkan till Tomas Hansson, Anna Mattsdotter Tomas Hansson
Änkan till Hans Olofsson
Hans Olofsson
Lars Nilsson Nils Mattsson
1592 – 1618 Matts Nilsson
Lillsilvast, 1 mtl
............................................................................................................................................................................
Minnesnoteringar:
------
2.2.1869 Arvsföreningens skiften av Elias Fors?? ------
Enligt en något osäker uppgift undertecknades år 1872 det första arrendekontraktet för en mjölkvarn på Fors. Den 1.12.1873 utfärdar guvernören för Wasa län (Carl Gustaf Fabian Wrede) rätt att uppföra
tullmjölkvarn med 2 par stenar. ------
Flora (och Hugo) Nyman köpte s.k. Kilen 351 på 9 ar den 31.5.1934 av Emil och Ida Fors. De byggde ett hus i rött tegel på tomten och bedrev handel där fram till hösten 1948. Den 11.6.1952 såldes fastigheten till Jepuan Tehtaat-Jeppo Fabriker. Jeppo Fabriker sålde i sin tur fastigheten till Mellersta Österbottens Andelskassa den 12.6.1953. Eliel Brunell avslutade nu sin butiksrörelse vid stationsvägen, då han blev föreståndare för Andelskassan och flyttde in i fastighetens bostadsutrymme med sin familj. Senare har Jeppo Blomma och Kemikalia (Alanen?) verkat en kort tid på platsen, som numera är Jeppo Krafts kontor och butik på Kiitolavägen 1.
