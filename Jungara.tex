\jhchapter{Jungarå, hemman Nr 12}

Jungarå hemman är det sydligaste hemmanet på Jeppo-sidan av Alahärmä-rån. Det har också märkts, i synnerhet tidigare generationer hade livligt umgänge över språkgränsen. Studenter som på 1876 åkte runt i Österbotten, konstaterade när de kom söderifrån, att här plötsligt talas svenska. Finska talades med svensk brytning och svenska med finsk brytning. Det var en hård dialekt, t.ex. ``ja’ a’ vori åsta schuuss mä märre ti Ståourgåln''.

Före 1755 fanns 35 hemman i Överjeppo by, Wåltti nr 35 var det sista. 1755 spjälktes det upp i flera och ``Krogen'' fick nummer 39. Från 1771 hette det Jungar nr 39, med beteckning 39Z på hemmanskartorna för att skilja det från Jungar hemman 6F, norr om Ruotsala. Själva namnet Jungarå började användas på 1830-talet när Jungarå blev hemmansnummer för ett hemman, som sträckte sig ända till Pedersöre rå, numera Kortesjärvi, en sträcka över 20 km. Storleken var 350 ha och man kan tänka sig att det var svårt att hålla alla vedtjuvar borta.

Älven var transportleden när vägarna eller stigarna var dåliga. Med start på 1500-och 1600 hundratalet blev tjäran en inkomstkälla för bönderna och tjäran transporterades längs med älven till uppköparna i Nykarleby, där den efter den s.k. tjärvräkningen såldes. Resan hem uppför forsarna var mödosam och komna till Jungarå var alla forsar avklarade. Där på östra sidan av älven, mitt emot holmsändan, fanns ett ``gästgiveri'', brännvinsbänneri och krog. Vägen på östra sidan av älven tog slut, en flottbro transporterade folk och hästar vidare till västra sidan, där vägen fortsatte. Krogen var m.a.o. placerad på rätt ställe.

Krögare under drygt 100 år:
\begin{enumerate}
  \item 1767--\allowbreak 1771, Mårten Jungar, Lisas far, var sista krögaren
  \item 1754--\allowbreak 1767, Dottern Lisa Jungar, \textborn 1727
  \item 1741--\allowbreak 1754, Mårten Jacobsson Jungar
  \item 1734--\allowbreak 1741, Jacob Jacobsson
  \item 1724--\allowbreak 1733, Jacob Mårtensson, son till krögare nr 2
  \item 1723--\allowbreak 1723, Knut Hindersson, andra gången
  \item 1714--\allowbreak 1722, Stora ofreden; Johan Isaksson (se nedan)
  \item 1704--\allowbreak 1713, sonen Henrik Knutsson fortsatte
  \item 1686--\allowbreak 1703, Knut Hindersson, gift med änkan Gertrud Olofsdotter
  \item 1677--\allowbreak 1685, Mårten Sigfridsson, gift med Gertrud Olofsdotter
  \item 1669--\allowbreak 1677, Gustaf krögare eller Månsson, den första kända krögaren
\end{enumerate}
Alla krögare var släkt från fadern eller modern till nästa generation. Undantaget åren 1714--22 då en Johan Isaksson hade övertagit verksamheten, han fick gå när de rätta ägarna kom tillbaka efter stora ofreden.

Om ``Det nya upplysta tidevarvet'' skrev Thomas Elenius år 1774 i sin fil.kand.-avhandling. Han var magister och kaplan i Nykarleby från 1778. År 1796 köpte han första delen av Jungar nr 39. I sin avhandling högaktade han åkerbrukaren som det högsta av yrken. När han själv blev jordabrukare, var det också i upplysningssyfte att lära och vidareförmedla ny kunskap. I magistersavhandlingen betonar han skolans betydelse för undervisningen av ungdomen. Efter hans död 1809 och hustrun Susannas död 1816 fortsatte 4 söner som jordbrukare.

1800-talets Jungarå var ett samhälle i samhället. Här fanns: jordbrukare, skomakare, skräddare, slaktare, handelsmän, mjölnare, smeder, bagare, färjkarlar, backstugosittare, torpare, gårdfarihandlare och inhysningar, drängar och pigor. I slutet på 1800-talet och inpå 1900-talet var emigrationen så stor, att mer än hälften av befolkningen åkte iväg, mest till Amerika. År 1890 fanns ca 50 personer på Jungarå.

Redan i slutet av 1700-talet betalades skatt för en kvarn i forsen, men först hundra år senare startade utbyggnaden av forsen med både kvarn och såg. Då var fallhöjden nästan fem meter, men efter två älvrensningar återstod en fallhöjd på ca en meter och vattenkraften var borta.

Av de 6 brukningsenheter som ännu på 1950-talet var igång, återstår nu 1 (en).


Jungarå hemman omfattas av vidstående karta nr \jhbold{17}.


<--- se KARTA nr 17 --->


\jhsubsection{Lägenheter på Jungarå}



\jhhouse{Dalabacka}{12:57}{Jungarå}{17}{12}


\jhhousepic{263-05848.jpg}{J-O o. M Elenius; Eskil och Svea Jungarå}

\jhoccupant{Elenius}{Jan-Olav \& Maaret}{}
Jan-Olav och Maaret köpte huset och tomten av Bo Jungarå 16.11.2010.\jhvspace{}


\jhoccupant{Jungarå}{Bo}{1997--\allowbreak 2010}
Bo fick åkermark och bostaden av sin far genom gåvobrev.\jhvspace{}


\jhoccupant{Jungarå}{Eskil \& Svea}{1947--\allowbreak 1997}
Eskil, \textborn 23.12.1921, gift 1945 med Svea Margareta, \textborn 01.09.1925, född Holster från Kimo.
\begin{jhchildren}
  \item \jhperson{Robert Eskil}{15.09.1947}{}
  \item \jhperson{Darlene Margaret}{19.05.1955}{02.02.2008}, född i Amerika
\end{jhchildren}

Hemkommen efter kriget (tjänstgjorde i IR-61 under ledning av Alpo 	Marttinen) när både fadern och modern dött, tog bröderna Eskil och Gunnar hand om jordbruket, som omfattade ca 25 ha odlad jord och 90 ha skog. Eskil gifte sig och bildade familj. Efter att bott i hemgården i 2 år byggde han ett nytt hus 1947 på andra sidan landsvägen. Nytt då var att ha luftcirkulation mellan rummen, eldstaden bestod av spis och vedhäll. Redan 1951 emigrerade familjen till Vancouver 	Canada och 4 år senare till Everett, Washington. Eskil arbetade först på fanérfabrik. Han hade talang för affärer och gjorde goda affärer i fastigheter i staden Everett, där Boeing hade sin flygplanstillverkning, och där innevånarantalet flerdubblades när tiderna var goda. 1966 köpte han andra halvan av sin farfarsfars hemman, som vid tidpunkten ägdes av Olof Dahlström. Eskil och Svea har varit till hemlandet många gånger.

Eskil \textdied 30.07.2001  ---  Svea \textdied 01.09.2011



\jhhouse{Sandström}{12:59}{Jungarå}{17}{109}


\jhoccupant{Sandström}{Anders \& Maria}{1919--\allowbreak 1928}
Anders, \textborn 26.11.1872, gift 1902 med Maria, \textborn 26.06.1874, född Eriksdotter Grötas. Anders var skräddare som sin far. Anders \textdied 09.09.1928  ---   Maria \textdied 31.10.1949.\jhvspace{}


\jhoccupant{Sandström}{Matts \& Lovisa}{1860--\allowbreak 1919}
Matts, \textborn 01.12.1835, gift med Greta Lovisa, \textborn 15.09.1833 i Purmo
\begin{jhchildren}
  \item \jhperson{Anna-Brita}{1859}{}
  \item \jhperson{Maria}{1861}{}
  \item \jhperson{Erik}{1863}{}
  \item \jhperson{Matts}{1866}{}
  \item \jhperson{Johannes}{1870}{}
  \item \jhperson{\jhbold{Anders} Gustaf}{26.11.1872}{09.09.1928}
  \item \jhperson{Samuel}{26.06.1875}{}, till stuga nr 119
  \item \jhperson{Wilhelm}{01.08.1878}{27.03.1940}, till stuga nr 118
  \item \jhperson{Leander}{01.08.1882}{}
\end{jhchildren}
Matts finns som backstugosittare på Jungarå sen 1860. Han var enligt Hilda Jungarå den första som köpte symaskin i Jeppo, före det gick han från gård till gård och på detta sätt livnärde han sin familj med 9 barn. Matts och Lovisa var kända för att vara ovanligt snälla mot barn, Lovisa kallades allmänt för skräddas-mamm.

Matts \textdied 04.04.1919  ---  Lovisa \textdied 01.10.1916


\jhhouse{Kassus}{12:57}{Jungarå}{17}{15, 15a-b}


\jhoccupant{Jungarå}{Olav}{2008--}
Olav köpte gården 2008 av Bo Jungarå som fått stugan och tomten av sin far Eskil via gåvobrev från 1998. Stugan har grundrenoverats och värms med luftvärmepump och spis. Bebos nu av Måns Jungarå och Ann-Charlotte Långkvist.


\jhhousepic{266-05852.jpg}{Måns Jungarå o. Ann-Charlotte Långkvist; Olav Jungarå}

\jhoccupant{Jungarå}{Eskil}{1946--\allowbreak 1998}
Stugan  beboddes av Kajsa Jungarå mellan åren 1946 och till sin död 13.04. 1951. I stugan bodde också Sigfrid Jungarå. Före kriget gick han i Uleåborg privata samskola. Under krigstiden blev han student, senare var han bl.a. och studerade i Helsingfors handelsinstitutet. På sina resor ner till Helsingfors hade han med sig produkter som han bytte till andra mera nödvändiga saker i dessa ransoneringstider. Bl.a. gick det att byta smör mot kopparledning, som behövdes när nya vattenhydrofonen skulle ha ström 1947.

Kajsa var farmor åt Sigfrid, Gunnar och Eskil och var en viktig person i familjen, efter att deras egna föräldrar dött redan tidigare. Alla tre bröderna emigrerade 1951, samma år dog Kajsa. Under 	krigstiden 1945 var huset bebott av Kemijärvbor: Aino och Kalervo Kärkelä med sonen Vilho, Helmi Haataja med barnen Anneli och Ossi. a = boda och lider, b = bastu.


\jhoccupant{Laxèn}{Johannes}{1930--\allowbreak 1944}
Byggmästar Johannes Laxèn, \textborn 30.01.1856, gift 1879 med Brita Johanna Pärnu från Kälviä, hon dog 1936. Efter en konkurs flyttade paret från Åggelby till Jeppo i början av 1920-talet. Under hans tid fick huset på Jungarå liggande spontad brädfodring. Han var en anlitad byggmästare i Tavastehus på 1880--\allowbreak 1910-talet med större och mindre projekt bl.a. i St Petersburg. Det slutade med konkurs. Sedan flytten till Jeppo på 1920-talet var han med och byggde bankhuset där Hab fanns (Silvast nr 70), samt tillbyggnaden på ``Sparrbackskolan''.

Johannes \textdied 03.01.1946


\jhoccupant{Källman}{Johan Edvard \& Anna Sofia}{1908--\allowbreak 1930}
Johan Edvard, \textborn 27.03.1884, gift med 1 Anna Sofia, \textborn 31.06.1883, född Jungarå. Paret gifte sig i Amerika år 1903.
\begin{jhchildren}
  \item \jhperson{Ester Sofia}{01.10.1904}{09.07.1991, född i USA}
  \item \jhperson{Artur Edvin}{15.09.1905}{28.08.1952, född i USA, emigrerade till USA}
  \item \jhperson{Valter Johannes}{30.01.1908}{18.04.1908, född på Jungarå}
\end{jhchildren}
Paret återvände till Jeppo med två barn. Anna Sofia dog 07.05.1908 i lungsot. Edvard gifte om sig 1909 med Ida Maria Nybyggar, \textborn 23.06.1883  -	\textdied 21.09.1953.
\begin{jhchildren}
  \item \jhperson{Hilda Maria}{21.12.1909}{21.01.1910}
  \item \jhperson{Hjördis}{21.05.1915}{26.02.1993, gift Fagerholm}
  \item \jhperson{Ingrid}{05.03.1911}{25.03.1999 gift Sundell}  OBS ordningsföljden!
\end{jhchildren}
Edvard fortsatte med sin fars yrke som mjölnare. Han var mjölnare på Jungarå kvarn och Tollikko kvarn, senare flyttade familjen till gård nr 390b (karta 6) i Silvast och han fortsatte som mjölnare på Silvast kvarn (nr 97d).

Ida \textdied 21.09.1953  ---  Edvard \textdied 07.02.1963


\jhoccupant{Källman}{Gustaf \& Kajsa}{ca 1890--\allowbreak 1908}
Gustaf Andersson, \textborn 29.10.1844, gift med Kajsa, \textborn 24.02.1847, född Dahl Dahlin, Bröms i Munsala.
\begin{jhchildren}
  \item \jhperson{Anders Gustaf}{09.12.1876}{}, g m Maria Lovisa Andersd. f 1884
  \item \jhperson{Matts Vilhelm}{10.04.1879}{15.08.1937}, g m Hilda-Sofia Grötas
  \item \jhperson{Maria Katarina}{13.09.1881}{}, g m Yrjö Ekola
  \item \jhperson{\jhbold{Johan Edvard}}{27.03.1884}{07.02.1963, gift 2 gånger}
  \item \jhperson{Hilma Lovisa}{19.09.1886}{}, till Amerika
  \item \jhperson{Emil}{09.07.1889}{}, till Amerika
  \item \jhperson{Ester Sofia}{07.11.1891}{20.11.1891}
\end{jhchildren}
J. A. Jungarå kallade huset för Kajsas stuga i början på 1900-talet. Gustaf dog 1904 och före det var han mjölnare på Tollikko och Jungarå kvarn. Kajsa kom som piga till Jeppo och hon gick i skriftskola på vinden i stuga nummer 120. Denna Källmanssläkt var spelmän och på alla tillställningar var någon med som spelman.



\jhhouse{Kassusbackan}{12:57}{Jungarå}{17}{114, 114a}


\jhoccupant{Koski}{Helmi Justina}{1967--\allowbreak 1968}
Helmi, \textborn 29.05.1913 i Muurala  ---  \textdied 26.03.1989 i Jeppo.

Barn: Kuisma Mikael \textborn 29.07.1950.

Dessa bodde på hyra i nämnda stuga från 29.05.1967 till augusti 1968. Stugan revs av Paul Laxèn i början på 1970-talet.


\jhhousepic{Jungara 114.jpg}{Både bageri och lanthandel i Ida Katarinas hus}

\jhoccupant{Gustafsson}{Ida Katarina}{1927--\allowbreak 1966}
Ida Katarina, \textborn 04.03.1889, född Backlund. Hon idkade butik- och Caférörelse från slutet av 20-talet, ännu i slutet av 1950-talet gick det att köpa lemonad och annat tillbehör. Före kriget på 1930-talet, när Sundells cafè i Silvast stängde på kvällen, cyklade ungdomen iväg till cafeet på jungarbackan och fortsatte umgås där. På reklamskylten på väggen stod drick Schmappis. Katrin höll sig med en ko. Sonen Lennart Gustafsson sålde tomt och stuga till Eskil Jungarå. a  = fähus.

\jhoccupant{Laxèn}{Lennart}{1916--\allowbreak 1927}
Lennart och syskonen bodde i huset tills han sålde det och själv flyttade till Härmä 1927, för en kortare tid, kom tillbaka till \jhbold{gård 123}.\jhvspace{}


\jhoccupant{Laxèn}{Gustaf \& Vilhelmina}{1885--\allowbreak 1916}
Gustav Laxèn, \textborn 19.05.1865, gift 1893 med Vilhelmina, \textborn 22.05.1862, född Kitula från Vörå.
\begin{jhchildren}
  \item \jhperson{Gustaf Lennart}{15.04.1894}{25.06.1894}
  \item \jhperson{Lisa Vilhelmina}{05.04.1895}{}, emigrerade till Amerika
  \item \jhperson{Axel Wiljam}{03.01.1897}{10.09.1915}
  \item \jhperson{Amanda Emilia}{19.11.1898}{22.01.1913}
  \item \jhperson{Maria Elvira}{08.12.1900}{}, flyttat till Jakobstad 30.09.1921
  \item \jhperson{Oskar Lennart}{25.11.1902}{03.06.1976, stuga nr 16}
  \item \jhperson{Hulda Matrosa}{26.07.1906}{}, emigrerade till Amerika
\end{jhchildren}
Gustaf Laxèns far Simon Oskar \textborn 14.02.1824 i Gamlakarleby, kom på 1840-talet som sågställare till Keppo. Gift med Lisa Davidsdotter \textborn 02.02.1828 i Härmä. Paret fick 11 barn och på 1880-talet flyttade de 	till Tollikko i ett hus som von Essen ägde. Gustaf var arbetsledare på Keppo, men flyttade till Tollikko och började baka vetebullar och idkade lanthandel 1890--\allowbreak 1908. 1909 gick affärsrörelsen i konkurs och Gustav reste till Amerika, under långa tider hade han ingen 	kontakt hem och dödförklarades 01.01.1939. Vilhelmina fick ensam ta hand om familjens sju barn. Hon dog 06.04.1916.



\jhhouse{Nurmo Sannas}{12:59}{Jungarå}{17}{115}

Nurmo Sannas stuga enligt J.A Jungarå karta. Man kan spekulera om huset byggdes som sytningsstuga åt Anna Helena Jungarå, senare Hägglund, född Forsberg, och kom från Nurmo. I så fall talar vi om Nurmos Anna. Hon blev änka efter andra äktenskapet 1845.



\jhhouse{Jungabackan}{12:47}{Jungarå}{17}{16, 16a}


\jhhousepic{265--05850.jpg}{Margit Granholm; tidigare Lennart och Fanny Laxén}

\jhoccupant{Granholm}{Margit}{1991--}
Margit Vilhelmina, \textborn 15.01.1929, flyttade till Sverige 1953, gifte sig och fick en dotter Stine, \textborn 01.09.1955. Hon arbetade först på syfabrik, sen som arbetare på Svenska Metallverken.

Margit var som ung med i Talkoungdomen och deras höstkampanj, året 1943 var lingonplockning. Plockningstiden var 2 veckor i september, Margit plockade 1340 kg och fick första pris och blev finsk mästare. Priset var en tysk damcykel och en resa till Helsingfors. Året därpå var hon också med och blev andra med 1153 kg lingon. Priset var en hornlös röd kokalv med namnet Essi. Kalven levererades till Jeppo station i en trälåda. Föräldrarna hade stor ekonomisk nytta av denna ko i många år. Talkopungdomens verksamhet var att hjälpa till i skördearbetet och andra göromål när det i krigstiden var brist på arbetsfolk.

Huset ägs numera av Margit, som bor Upplands Väsby, Sverige. Hon kommer varje sommar till stugan på Jungabackan. Också dottern och barnbarnen med familjer är gäster på Jungabackan varje sommar. Margits sambo Totte (eller Torsten Stomberg)  från Sverige, gjorde en grundlig renovering av huset samt gjorde en bastu i uthusbyggnaden åren 1991--\allowbreak 2005.


\jhhousepic{264-05851.jpg}{Gick under namnet Timperis stuga}

\jhoccupant{Laxèn}{Lennart \& Fanny}{1945--\allowbreak 1991}
Lennart, \textborn 25.11.1902, gift 1927 med Fanny Maria Gustavsdotter, \textborn 24.09.1901, född Bro i Jeppo. Lennart och Fanny byggde huset åt sig 1945 på samma tomt som den tidigare stugan, \jhbold{gård nr 116}. Lennart byggde ett fähus 1941 och hade en eller två kor till 1959. Uthusbyggnaden byggdes till i flera omgångar, en tillbyggnad bestod av ``Timperis'' stuga, som fungerade som ungkarlslya åt sönerna Paul och Manne. a = uthus med lider och bastu.

Lennart arbetade på järnvägen som banarbetare. Däremellan var han skogshuggare. Han deltog flera år i skogshuggar-Fm. Hedersdiplom av ministeriet för kommunikationsväsendet fick Lennart 1944 för huggning av klenvirke 44.92 m3 travad  kastved, detta gav 1 pris i mästerskapen i Jeppo. År 1945 gällde huggning av travat virke, Lennart högg 66,63 m3 lösmått brännved, också denna gången i ledning. Lennart var med på skyddskårens skjutövningar på 1920-talet och med i 	fortsättningskriget. Han var bl.a. en ivrig frimärkssamlare som ibland åkte till Helsingfors på frimärksauktioner.

Lennart \textdied	03.05.1976  ---  Fanny \textdied 23.04.1980.


\jholdhouse{Gamla gården på Jungabackan}{12:47}{Jungarå}{17}{116}


\jhhousepic{Jungara 116.jpg}{Nr 116, Jungarbackan}

\jhoccupant{Laxén}{Lennart \& Fanny}{1927--\allowbreak 1945}
Oskar Lennart, \textborn 25.11.1902 i Jeppo, gift 1927 med Fanny Maria Gustafsdotter, \textborn 24.09.1901, född Bro.
\begin{jhchildren}
  \item \jhperson{Margit Vilhelmina}{15.01.1929}{}, gift Granholm till Sverige 1953
  \item \jhperson{Manne Lennart}{30.03.1930}{09.10.1981, till Sverige 1951}
  \item \jhperson{Paul Mack}{03.11.1931, gift med Anneli Huhtala från Härmä}{}
\end{jhchildren}

Efter en kortare tids vistelse i Härmä flyttade Lennart och Fanny till den lilla stugan på ``jungarbackan'' ett stenkast från barndomshemmet. Stugan bestod av ett rum och en kall brädad del som användes som sovrum på sommaren. Här bodde familjen samt svärmor Anna-Lovisa Tollikko till sin död 13.07.1941. Mera om Lennart Laxèn gård nr 16, ovan.


\jhoccupant{Jansson}{Jacob}{1890--\allowbreak 1915}
Jacob, \textborn 15.06.1837 i Munsala, gift med Maria Lovisa, \textborn 01.06.1838 i Munsala. Maja dog 09.04.1897 och Jacob gifte om sig med Anna Sofia, \textborn 25.07.1869, född Yrjösdotter Ekola (mor hemma från granngården).
\begin{jhchildren}
  \item \jhperson{Anders}{1892, från första giftet}{}
  \item \jhperson{Ida Maria}{07.02.1901}{}
  \item \jhperson{Jacob Edvard}{29.02.1904}{28.02.1905}
\end{jhchildren}
Andra hustrun dog 12.12.1905. Jacob Jansson var gårdfarihandlare och gick runt i bygden med dottern Ida och sålde kammar tillverkade av horn. Jacob berättade ofta att han hade en duktig hustru och när Kajsa Jungarå frågade om han satte värde på henne svarade han: ``icke tillfyllest''.

Jacob \textdied 12.12.1915



\jhhouse{Slaktus}{12:59}{Jungarå}{17}{117}


\jhoccupant{Jungarå}{Anna-Sanna}{1909--\allowbreak 1923}
Efter makens och döttrarnas död tog Anna-Sanna hand om 4 föräldralösa barnbarn:
\begin{jhchildren}
  \item \jhperson{Ellen Sofia Viita}{08.08.1903}{03.03.1918}
  \item \jhperson{Johannes Sigfrid Jungar}{16.01.1909}{26.07.1922}
  \item \jhperson{Jenny Linnea Jungar}{24.07.1914}{21.02.1937}
  \item \jhperson{Gerda Sofia Jungar}{18.01.1916}{13.10.1919}
\end{jhchildren}
Anna Sanna avled i ålderdomsavtynande efter en kort tids svår sinnessjukdom 17.09.1923.


\jhoccupant{Andersson}{Erik \& Anna-Sanna}{1871--\allowbreak 1909}
Erik Andersson, \textborn 16.12.1848 på Grötas, gift 1871 med Anna Sanna, \textborn 17.10.1851 på Jungarå.
\begin{jhchildren}
  \item \jhperson{Erik}{20.04.1873, emigrerade i unga år, dödförklarad 1971}{}
  \item \jhperson{Anders Gustav}{22.03.1877, emigrerade, dödförklarad 1939}{}
  \item \jhperson{Anna Lovisa}{01.06.1881}{07.02.1906, gift Viita}
  \item \jhperson{Matts Leander}{02.03.1884, emigrerade i unga år}{}
  \item \jhperson{Johannes}{07.02.1886}{21.04.1912}
  \item \jhperson{Sanna Sofia}{22.01.1889}{19.12.1916, gift Rasmus}
\end{jhchildren}
Erik Andersson arbetade som ung på Keppo såg. Efter giftermålet bosatte sig makarna i en nybyggd stuga på Jungarå. Där verkade han som slaktare, köpte upp slakt i egna byn samt grannsocknarna. På lördagarna åkte han till marknaden i Nykarleby med sina slakteriprodukter. Han var också anlitad violinspelman på bröllop, men alkoholism bröt ner honom, han dog 12.11.1909.

BREVSIDOR?


\jhsubsection{Ett livsöde -  Anna Sanna Jungarå}

Ungefär 100 år efter att Thomas Elenius föddes, kom en flicka till världen som fick namnet Anna Susanna Jungarå. Hon föddes 17 oktober 1851. Hennes föräldrar var Anders Gustav Jungarå och Anna Greta Mietala (Forsgård). Farföräldrar till henne var Matts Elenius (prästpojke) och Anna Helena Forsberg. Anna Sannas farfar Matts hade tjänstgjort som bokhållare i Östermyra, Nurmo. Han flyttade tillbaka hem för att överta jordbruket på Jungarå efter sin far Thomas Elenius. På hemvägen så föddes Anna Sannas pappa, Anders Gustav, i landsvägsdiket vid Krouvi, nära intill Jungarå, men på andra sidan ån. Anders Gustavs far Matts Elenius dog ung i lungsot endast 33 år gammal.

Anna Sannas pappas, Anders Gustavs, jordbruk omfattade år 1856 av 241 ha varav 41 ha odlad jord. Anna Sanna föddes som andra barnet i en syskonskara på 9 barn. Hon var syster till Gustav d.ä., som är anfader till bla släkten Eskil Jungarå och Dahlströms. Hennes mor drabbades av tyfus och dog 42 år gammal. Då gifte fadern om sig med änkan till Johan Kaukos från Gunnar hemman, Anna Sanna Mickelsdotter.

Anna Sanna Jungarå fick växa upp i ett litet större jordbrukarhem och det innebar troligen att det fanns både mat och det mest nödvändiga människan behöver. Som 20 åring gifte hon sig med Erik Andersson, född 16 december 1848 på Grötas hemman i Jeppo. Erik arbetade som ung på Keppo såg. Paret fick 6 barn.
\begin{jhchildren}
  \item \jhperson{Erik}{20.4.1873}{}
  \item \jhperson{Anders}{22.3.1877}{}
  \item \jhperson{Anna-Lovisa}{1.6.1881}{}
  \item \jhperson{Matts Leander}{2.3.1884}{}
  \item \jhperson{Johannes}{7.2.1886}{}
  \item \jhperson{Sanna Sofia}{22.1.1889}{}
\end{jhchildren}
Det unga paret byggde åt sig en ny stuga på Jungarå, en bit från  hemgården på andra sidan vägen. Erik blev så småningom slaktare. Han färdades runt i trakten och köpte livdjur som han tog hem för slakt. Erik slaktade och sålde köttet på Nykarleby torg. Erik var den enda på Jungarbacken av backstugusittarna som hade häst, berättade Hilda Jungarå i en tidningsintervju. Han var ofta anlitad spelman på bröllop.

Eftersom det under dessa tider av 1800 och början av 1900 talet härjade allehanda sjukdomar, var människorna mera härdade inför sjukdom och död. Ändå förstår vi hur de våndades och led när de gång efter gång skulle bära sina barn och anhöriga till graven.


\jhbold{Lungsot eller tuberkulos}

Tuberkulos är en infektionssjukdom som sprids av tuberkelbakterien och angriper både människor och djur. Spridningen sker vanligen genom droppinfektion, ibland genom damm, smuts eller via födan(förr ofta via infekterad mjölk). Inkubationstiden är 6-8 veckor. Den första infektionshärden, primärhärden, är lungorna. Sjukdomen kan sedan spridas vidare till de flesta organ och vävnader. TBC, särskilt den kroniska lungtuberkulosen (lungsoten), var förr en förhärjande folksjukdom.

I spåren av 1800 talets industrialisering och urbanisering följde inte bara ekonomisk tillväxt och standardhöjning, utan också trångboddhet och slum, särskilt i större städer. Detta skapade i sin tur en gynnsam grogrund för tuberkulos. Det var först 1882, genom Robert Kochs odling av tuberkelbacillen och upptäckten av tubekulin, ett protein från tuberkelbaciller, som verkligt rationella insatser kunde göras mot tuberkulos. Han fick för denna insats 1905 års nobelpris i medicin. Man började isolera folk när man insåg att tuberkulos var en smittsam sjukdom.

Pastörisering av komjölk efter 1915 och kontroll av kreatursbesättningar med tuberkulintest var ett led i bekämpandet av sjukdomen. Efter det kom vaccinationer och skärmbildsundersökningar. År 1949 kom antibiotika som ett effektivt botemedel. Men när Anna Sanna och Erik fick sina barn mellan åren 1873 till 1886 hade all den nya kunskapen om bekämpandet av tuberkulos ännu inte nått till Prästas.

Erik Andersson arbetade som slaktare, han hade säkert inte den kunskap om hygien och renlighet som behövts, när man handskas med djur och djuravfall. Och så gick livet sin gilla gång. Barnen växte och det var hårda och fattiga tider och det berättades om landet i väst där man kunde skapa sig en bättre framtid. Också Anna Sannas pojkar ville pröva lyckan over there; Erik, Anders  och Matts Leander emigrerade till Amerika.
  
Dottern Anna Lovisa gifte sig som 22 åring med Emil Viita, född 17.9 1871, från Ylistaro. De bosatte sig vid Keppo gård där Emil hade sitt arbete. Glädje rådde säkert i hemmet över de tre barnen som föddes. Men så kom sjukdomen tuberkulos in i deras hem och både dottern Anna Lovisa och de två yngsta barnen dog 1906. Efter det här flyttade fadern tillbaka till Ylistaro. Nu fick Anna Sanna ta hand om barnbarnet Ellen Sofia som nu var 3 år gammal.

Erik var ofta anlitad som spelman på bröllop och det innebar också en hel del festande, som ledde till att han fick alkoholproblem. Så drabbades också han av lungsot och dog 12 november 1909. Av Anna Sannas sex barn var nu Johannes och Sanna Sofia kvar i Jeppo.

Anna Sannas son Johannes var endast 26 år gammal när han också avled i lungsot 21 april 1912. Dottern Sanna Sofia, född 22 januari 1899, var Anna Sannas yngsta och nu enda dotter. Hon hade varit fosterdotter hos moster Sofia och hennes man på Grötas. Nu hade hon hittat en bra karl i bonden på Jungar hemman, Matts Emil Rasmus, \textborn 8.3 1875.
De gifte sig och fick fem barn under åren 1909 till 1916 och Anna Sanna fick glädja sig över dotterns lycka och välgång.

Men livet var hårt mot familjen och Saima Linnea Jungar, 2 år, dog i mässling 1911 och Jenny Olivia Jungar, 1 år, dog i lungsot 1912. Sanna Sofia Jungar, eller Rasmus, insjuknade och dog i lungsot 19 december 1916 och hennes man Erik dog i hjärtslag 2.12 1917. Yngsta barnet, Gerda Sofia, dog i lungsot hos mormor Anna Sanna på Jungarå 1919. Ellen Sofia Viita, som flyttade till mormor Anna Sanna efter sin mors död, fick leva till 1918, dog då hon också avled i lungsot. Också dottern Sanna Sofias äldsta son, Johannes, fick leva till 13 år och dog 1922 hos sin mormor Anna Sanna.

Av dotter Sanna Sofias fem barn fanns bara Jenny Linnea kvar. Hon var född 24 juli 1914 och avlade mellanskola vid Nykarleby samskola och fortsatte till handelsskolan i Jakobstad, men även hon insjuknade i TBC och dog 23 år gammal.
Så blev det liv som Anna Sanna fick. Hon insjuknade inte själv i lungsot men det lidande hon fick utstå under sitt jordeliv var ovanligt hårt. Hon dog själv den 17 september 1923, vid 72 års ålder, av hjärnblödning efter en kort tids svår sinnessjukdom.

Den 26 maj 1925 och den 22 augusti 1925 kommer brev från Anna Sannas son Erik i Michigan till morbrodern Gustav på Prästas. Erik har fått höra om sin mors död och att han ärvt sin mors kvarlåtenskap. Han önskade att morbrodern höll en auktion eller sålde det som var kvar av hemmet och att han inte skulle frukta att ta betalt för allt besvär han haft med modern. Blir det pengar över så vill han att morbrodern sänder dem till Miners State bank i Iron River i Michigan USA.

Hur det blev med pengarna till Amerika får vi inte reda på. Men varje sommar blommar spirean på den plats där Anna Sanna en gång bodde.



\jhhouse{Brunsgård}{12:7}{Jungarå}{17}{18,18a}

\jhoccupant{Jungarå}{Olav \& Gunilla}{2005--}
Gården och tomten 0.12 ha köptes av Ingegärd Larsson år 2005, och hon i sin tur hade fått lägenheten av sin syster genom testamente.\jhvspace{}


\jhoccupant{Larsson}{Ingegärd}{1986--\allowbreak 2005}
Ingegärd, \textborn 23.10.1919, gifte sig 1949 med Tycko Larsson bosatta i Asphyttan, Värmland, Sverige. Halva gården fick hon genom gåvobrev av sina föräldrar redan 1945, andra halvan efter syster Hildas död.\jhvspace{}


\jhhousepic{270-05855.jpg}{Sticka-Hildas; O och G Jungarå}

\jhoccupant{Jungarå}{Hilda}{1964--\allowbreak 1986}
Hilda, \textborn 19.09.1899 på Jungarå, hon gick i Kronoby folkskola år 1920-21. År 1927 tog hon en kurs i maskinstickning i Vasa och ägnade sig åt att maskinsticka olika klädesplagg på beställning, såsom tröjor, strumpor m.m. Kundkretsen var stor, både från Härmä och Jeppo. Hon påstod att hon bara kunde ``stickafinsk''. Hon samlade i sitt hem på gamla saker till ett eget litet museum. Hon var bra insatt i gamla släkten och bosättningar. I Gunnar marthakrets var Hilda ordförande 1930-55. Hon var också tio år som ordförande i nykterhetsföreningen, styrelsemedlem i Jeppo marthaförening samt medlem i många andra föreningar.


\jhoccupant{Jungarå}{Gustav \& Lina}{1923--\allowbreak 1964}
Gustav och Karolina köpte gården redan 1923, de flyttade in med döttrarna \jhbold{Hilda} och \jhbold{Ingegärd} i gården efter kriget och samtidigt förstorades huset med två rum år 1945 när sonen Henrik övertog hemmanet och hemgården. Karolina dog 25.09.1949 och Gustav dog 03.04.1964. Ingegärd emigrerade till Sverige 1951.  a = lider och 	boda.


\jhoccupant{Taipale}{Viktor}{1911--\allowbreak 1923}
Arbetaren Viktor Taipale, \textborn 23.11.1878 i Korpilahti, inflyttad och vigd 08.07.1911 med Sanna Matilda Eriksdotter, \textborn 24.04.1876, född Forsberg på Jungarå.
\begin{jhchildren}
  \item \jhperson{Erik Viktor}{06.05.1912}{1941}, utflyttad till Virdois 1939, dog i strid
  \item \jhperson{Ahti Johannes}{21.06.1913}{29.08.1913}
  \item \jhperson{Marta Susanna}{26.07.1914}{}, flyttat med far till Jämsä 1944
  \item \jhperson{Hilda Maria}{02.08.1916}{28.08.1928}
  \item \jhperson{Fanny Wilhelmina}{02.03.1919}{}, utflyttad till Virrat 1939
\end{jhchildren}
Viktor besökte också Amerika före hustruns död.

Sanna Matilda \textdied 05.04.1922  ---  Viktor \textdied i  Haapamäki 1963.


\jhoccupant{Forsberg}{Erik}{1871--\allowbreak 1911}
Erik, \textborn 17.05 1842 på Jungarå, gift med Sanna, \textborn 27.10.1848, född Andersdotter på Måtar.
\begin{jhchildren}
  \item \jhperson{Anna}{18.12.1871}{19.08.1897, gift Vikman Terjärv, dog i USA}
  \item \jhperson{Johanna Fredrika}{19.09.1873}{}, betyg till Amerika 1893
  \item \jhperson{\jhbold{Sanna} Matilda}{24.04.1876}{05.04.1922}
  \item \jhperson{Anders Gustaf}{05.04.1879}{24.02.1903, dräng, gift, dog i USA}
  \item \jhperson{Erik}{24.01.1882}{}, betyg till Amerika 1899
  \item \jhperson{Johannes}{12.09.1884}{03.04.1903, betyg till Amerika 1902}
\end{jhchildren}
Erik Forsberg blev kallad ``fruns Erik'' efter sin mor, som kallades för frun. Under forsrensningen 1910-12 var kvarntomten nedanför Forsbergs stuga torgplats varannan torsdag när det var lönedags för de talrika arbetarna. Då var det marknad med försäljare från vida omkring, gårdfarihandlare och laukku-ryssar. Det blev säkert också ett extra klirr i Eriks kassa.

Uppgifter saknas, men stugan uppfördes troligen åt Erik sedan han gifte sig 07.07.1871.

Sanna \textdied 1913  ---  Erik \textdied 1914 på Jungarå.



\jhhouse{Sandström}{12:9}{Jungarå}{17}{118, 118a-b}


\jhhousepic{Jungara 118.png}{Asko och Marja-Liisa Linjamäki}

\jhoccupant{Linjamäki}{Asko \& Marja-Liisa}{1963--\allowbreak 2001}
Asko, \textborn 15.07.1939, köpte lägenheten 1963 och flyttade från Silvast (nr 85) in tillsammans med modern Siiri. Han gifte sig 1966 med Marja-Liisa, \textborn 1948, född Mäki från Alahärmä.

Barn: Merja Marianne \textborn 15.03.1967 gift Tikkamäki.

Asko arbetade på Keppo, han var en kraftkarl med många medaljer i styrkelyft.Tränandet gjorde han i en stocklada på tomten. Familjen flyttade 1968 till Lavast, senare Elsas och till Keppo. Siiri bodde några år in på 1970-talet i huset. Huset revs på 1990-talet och Olav Jungarå köpte tomten 2001. Tomten är numera åkermark och äges av Jungarå gårdssammanslutning.


\jhoccupant{Dahlström}{Ingmar \& Sally}{1958--\allowbreak 1963}
Ingmar, \textborn 29.12.1937 på Jungarå, gift 1957 med Sally, \textborn 26.09.1937, född Bro.
\begin{jhchildren}
  \item \jhperson{Stig Ingmar}{21.01.1958}{}
  \item \jhperson{Regina Karita}{22.03.1959}{}
  \item \jhperson{Alf Anders}{31.08.1960}{}
  \item \jhperson{Peter Johan}{18.10.1963, född i Silvast}{}
  \item \jhperson{Sten Gustav}{04.02.1966, född i Silvast}{}
\end{jhchildren}
Ingmar var i byggnadsarbete på Keppo. Ungdomarna var många på 1950-talet på Jungarå och gjorde en egen plan där de utövade olika sportgrenar bl.a. löpning och olika kastgrenar. Ingmar hade talang för kastgrenarna och i synnerhet kula blev hans specialitet. Han hade ÖID-rekordet många år i kula, som längst stötte han 14,91 m. Familjen flyttade 1963 till Silvast, Romar, karta 3, gård nr 35.


\jhoccupant{Sandström}{Vilhelm \& Adolfina}{1898--\allowbreak 1958}
Vilhelm, \textborn 01.02.1878 i Jeppo, gift 1. 1898 med Maria Holmström, \textborn 22.08.1872, \textdied 04.04.1922 i lungsot. Gifte om sig 1923 med Adolfina, \textborn 30.01.1888, född Granholm, alla barn från första giftet.
\begin{jhchildren}
  \item \jhperson{Matts Vilhelm}{27.12.1899}{27.03.1940}
  \item \jhperson{Elna Maria}{23.10.1903, gift Nylund}{}
  \item \jhperson{Hilda Katrina}{05.09.1906}{26.09.1982, gift Selin, till Canada}
  \item \jhperson{Erik Johan}{25.02.1910}{08.07.1910}
  \item \jhperson{Anders Edvin}{22.08.1911}{10.1987, till Canada} ??
  \item \jhperson{Anna Sofia}{27.07.1914}{24.01.1918}
\end{jhchildren}
Vilhelm var mjölnare på Tollikko och Jungarå kvarn, han var son till skräddaren Mats Sandström i gård nr 109, han dog 27.03.1940. Kvar i bostaden blev Adolfina eller ``Fina'' som hon kallades till sin död 07.02.1958. Hon idkade kammarhandel i bostaden liksom sin granne Katrin, så grannsämjan var inte den bästa alltid. I fähuset hade hon en ko. Bostaden bestod av 1 rum och ett litet kök. Systerdotter Maria Jungarå bodde i stuga nr 14. a = fähus, b = bastu.



\jhhouse{Johans}{12:62}{Jungarå}{17}{19, 19a}


\jhhousepic{271-05858.jpg}{Kim och Anna Jungarå}

\jhoccupant{Jungarå}{Kim \& Anna}{2008--}
Kim, \textborn 24.11.1979 i Jeppo, gift 2011 med Anna, \textborn 28.08.1986, född Leinonen i  Sundom.
\begin{jhchildren}
  \item \jhperson{Elsa Linnea Margareta}{23.05.2010}{}
  \item \jhperson{Alma Saga Alice}{06.02.2013}{}
  \item \jhperson{Edvin Karl-Gustav}{28.03.2016}{}
\end{jhchildren}

Kim är utbildad maskiningenjör, har arbetat på Citek i Vasa och som maskinansvarig på Jeppo potatis. 2014 övertog han tillsammans med brodern Måns jordbruket efter föräldrarna. Jordbrukssammanslutningen bedriver ekologisk rotfruktsodling, 	närmast morötter och husdjur i form av dikor. Kim köpte gården 2008 av Henriette Rydell och Louise Spillum, som är döttrar till Torsten Elenius. Gården har renoverats från grunden till en modern bostad. Nu bor familjen i det hus som Kims farfars farfar en gång byggde.


\jhoccupant{Rydell Henriette}{\& Spillum Louise}{2004--\allowbreak 2008}
Henriette bor i Victoria, Texas, och Louise bor i Atlanta, Georgia. Systrarna ärvde gården på ca 1 ha efter föräldrarnas död.\jhvspace{}


\jhoccupant{Elenius}{Torsten \& Astrid}{1972--\allowbreak 2004, 1945--\allowbreak 1957 }
Torsten, \textborn 01.07.1917 i Jeppo gift, 1944 med Astrid, \textborn 24.01.1920, född Sandbacka på Mietala.
\begin{jhchildren}
  \item \jhperson{\jhbold{Henriette} Christin}{25.05.1946, g Rydell, arbetat som flygvärdinna}{}
  \item \jhperson{Charlott \jhbold{Louise}}{06.05.1952, g Spillum, arbetar på kontor}{}
\end{jhchildren}
\jhbold{1/2}
Torsten och Astrid övertog hans föräldrars halva lägenhet 1945 som 	omfattade 44 ha varav 16 ha odlad jord. Han byggde en ny ladugård i tegel 1947. Torsten var tekniskt intresserad och byggde bl.a en potatiströska, traktor köpte han redan 1952. 1957 flyttade familjen till Port Artur, Canada, och 1963 flyttade de till Everett, Washington i USA. Torsten arbetade på fanérfabrik och Astrid var sömmerska. Som pensionärer flyttade de till Peachtree City i Georgia. Efter Astrids död flyttade Torsten till Victoria i Texas.


\jhoccupant{Broo}{Svea \& Sanfrid}{1955--\allowbreak 1974}
Svea, \textborn 24.09.1908 i Jeppo, gift 1936 med Sanfrid, \textborn 03.10.1910, född Broo.
\begin{jhchildren}
  \item \jhperson{Sally}{26.09.1937, gift med Ingmar Dahlström}{}
  \item \jhperson{Mary}{19.12.1941, gift med Pauli Risikko}{}
  \item \jhperson{\jhbold{Herman}}{03.07.1945}{2012}, gift 1966 med Paula Inkeri Ylinen
\end{jhchildren}

Herman och Paula Inkeri Ylinen, \textborn 02.02.1947, fick tre barn.
\begin{jhchildren}
  \item \jhperson{Jari Kristian}{14.10.1967 på Jungarå}{}
  \item \jhperson{Kirsi Johanna}{05.07.1969 på Svartbackan}{}
  \item \jhperson{Kati Maarit}{17.11.1972 på Silvast}{}
\end{jhchildren}

I början av äktenskapet bodde familjen på Böös hemman. Sanfrid arbetade på järnvägen och i skogen. 1955 köpte de halva Sveas hemgård på Jungarå av Ragni och Gunnar Finne och blev jordbrukare. År 1967 skaffade Herman skogstraktor och började köra ut virke på förtjänst, sen korna sålts bort hade de en tid grisar. Herman och 	Paula flyttade år våren 1969 till Svartbacken, gård nr ?.  Sanfrid och Svea flyttade till Jungar, gård nr ?, hösten 1968. Hemmanet sålde de till staten år 1973.


\jhoccupant{Finne}{Ragni \& Gunnar}{1954--\allowbreak 1955}
\jhbold{1/2}
Ragni, \textborn 03.11.1919 i Jeppo, gift 1939 med Gunnar Finne, \textborn 06.05.1916 i Markby. Ragni och Gunnar kom hem från Sverige 1954 och övertog hennes mors hemmansdel. Redan året därpå sålde de jordbruket till  hennes faster och återvände till Sverige. De hade dottern Gun Ellen Viola, \textborn 12.06.1952.


\jhoccupant{Elenius}{Johannes \& Ellen}{1920--\allowbreak 1948}
Johannes, \textborn 23.11.1893 i Jeppo, gift 1916 med Ellen, \textborn 18.06.1895, född Romar.
\begin{jhchildren}
  \item \jhperson{\jhbold{Torsten} Johannes}{01.07.1917}{2004}
  \item \jhperson{\jhbold{Ragni} Ellen}{03.11.1919}{}, gift med Gunnar Finne
\end{jhchildren}
Johannes ändrade tillnamnet tillbaka till ursprunget Elenius. Han övertog hälften av sina föräldrars hemman, som bestod av 44 ha varav 16 ha odlad jord. Johannes var intresserad av travsport och ägde en stor och kraftig avelshingst, som var känd vida omkring (Hampus). Efter Johannes död 19.12.1948 ägdes halva hemmanet av Ellen tills hon sålde det till sin dotter.


\jhoccupant{Jungarå}{Johan \& Johanna}{1899--\allowbreak 1920}
Johan, \textborn 30.03.1866 i Jeppo, gift 1887 med Johanna, \textborn 22.11.1867, född Löfgren på Böös.
\begin{jhchildren}
  \item \jhperson{Johannes Eliel}{05.01.1892}{26.06.1892}
  \item \jhperson{\jhbold{Johannes} Edvin}{23.11.1893}{19.12.1948}
  \item \jhperson{Anders Artur}{21.05.1896}{27.12.1961, gård nr 14}
  \item \jhperson{Jakob Eliel}{12.10.1901}{27.01.1984, emigrerade till Canada}
  \item \jhperson{Helfrid Johanna}{24.05.1906}{07.10.1907}
  \item \jhperson{\jhbold{Svea} Johanna}{24.09.1908}{20.05.1990, gift Broo}
\end{jhchildren}
Genom testamente 1898 överlämnade fadern Johan Jungarå sitt hemman till bröderna Johan, Matts och Gustav så de fick 5/72 mantal var. Efter Matts' död året därpå delades hans del till Johan och Gustav. På detta sätt blev Johan ägare till halva hemmanet år 1900, hemmansdelen var 110 ha, varav 20 ha odlad jord. Johan och Johanna blev ensam ägare till den stora mangården, som fadern byggde åt två av sönerna två år tidigare.


\jhoccupant{Jungarå}{Matts \& Sanna}{1898--\allowbreak 1899}
Matts, \textborn 31.01.1873, gift 1896 med Sanna Sofia, \textborn 11.08.1876, född Finskas.
\begin{jhchildren}
  \item \jhperson{Ester Emilia}{09.04.1896}{09.04.1896}
  \item \jhperson{Hilda Sofia}{12.05.1898}{05.11.1924, gift med Paul Ståhl}
\end{jhchildren}
Matts hade liksom sin bror Erik tjänat den ryska militären tre år i Vasa och med manöver i Krasnaje Selo-lägret i Petersburg, där fick han början till lungsot, och han dog 30.09.1899. 1998 byggde fadern ett gemensamt hus åt Matts och Johan med familjer och de hade genom testamente fått av hemgårdshemmanet en tredjedel var. Hemmanet såldes till bröderna Gustav och Johan Jungarå efter Matts död. Änkan Sanna Sofia gifte om sig 1901 med Erik Gustav Stenbacka.



\jhhouse{Dalabacken}{12:10}{Jungarå}{17}{119, 119a}


\jhoccupant{Strand}{Maria}{1913--\allowbreak 1959}
Maria Rasmus, \textborn 30.12.1879, senare med tillnamnet Strand, bodde ogift tillsammans med syskonbarnen. De emigrerade anefter de uppnått vuxen ålder. Marias brorsdotter Jenny Linnea, \textborn 24.07.1914, tog hon hand om sedan hennes föräldrar Matts Emil, med tillnamnet Rasmus, senare Jungar, dog 03.12.1917 och Sanna Sofia dog 19.12.1916, från	gård nr 117. Jenny Linnea gick ut Nykarleby samskola och fortsatte till handelskolan i Jakobstad, men även hon insjuknade i Tbc och dog 23 år gammal den 21.06.1937. ``Rasmu Mari'', eller Samuls Mari, dog 07.03.1959. a = stenfähus.

Huset revs av Paul Laxèn.

\jhoccupant{Sandström}{Samuel \& Emma}{1907--\allowbreak 1913}
Samuel, \textborn 26.02.1875, gift med Emma Lovisa, \textborn 03.09.1872, född Andersdotter Rasmus från Terjärv. Föräldrarna kom via Rasmus i Gamlakarleby till Jungar gård nr (119).
\begin{jhchildren}
  \item \jhperson{Emil Edvard}{03.09.1902 i Jeppo}{}
  \item \jhperson{Samuel Edvin}{30.01.1904 i Michigan}{}
  \item \jhperson{Matts Elmer}{17.05.1905 i Visconsin}{}
  \item \jhperson{Anders Edvin}{07.05.1907}{}
  \item \jhperson{Johannes Evert}{04.10.1908}{}
  \item \jhperson{Elsi Elvina}{28.11.1910}{13.11.1913}
\end{jhchildren}
Samuel var första gången till Amerika 1892, kom tillbaka några år senare och gift sig med Emma Lovisa, \textborn 03.09.1872, född Rasmus i Terjärv. Samuel och Emma hade betyg till Amerika 1903, men kom tillbaka efter några år. Emma dog redan 1913 och Samuel emigrerade på nytt samma år. Efter Emmas död flyttade syster Maria hem från Amerika - nu med efternamnet Strand - till gården och tog hand om barnen, som alla	efter att de gått ut skriftskolan emigrerade till Amerika. Maria kallades för Samuls Mari, hon var född 30.12.1879. Hon var också en av de starka ensamstående kvinnorna som höll sig med en ko.


\jhoccupant{Luoma}{Gustav \& Anna-Brita}{1890--\allowbreak 1907}
Gustaf, \textborn 1859 gift med Anna Brita, \textborn 1859, född Sandström. Gård nr 109
\begin{jhchildren}
  \item \jhperson{Hilma}{1881}{}
  \item \jhperson{Vendla}{1884}{}
  \item \jhperson{Gustav Jarl}{1886}{}
  \item \jhperson{Johannes Victor}{1891}{}
  \item \jhperson{Signe Emilia}{1893}{}
  \item \jhperson{Ida Lydia}{1896}{}
  \item \jhperson{Anna Amanda}{1989}{}
  \item \jhperson{Otto Wilhelm}{1900}{}
  \item \jhperson{Valfrid Immanuel}{1903}{}
\end{jhchildren}
Gustav Luoma var född i Härmä och gift med skräddarens dotter Anna-Brita. Han var skomakare och hade ett eget rum för sig där han sydde skor. För honom var det också att gå från gård till gård. Förr var det sed att bönderna hade hudar av kor, hästar och svin som garvades till läder. Av dessa hudar sydde skomakaren pjäxstövlar, pjäxor och kängor. Hela familjen emigrerade till Amerika 1907.



\jhhouse{Jungarå}{12:69}{Jungarå}{17}{20, 20a-b}

\jhbold{Thomas Elenius bostad}
BILD!

\jhoccupant{Jungarå}{Olav \& Gunilla}{2006-}
Olav och Gunilla köpte tomten, bostaden och befintliga bodar omfattande 0.5 ha av Bo Jungarå år 2006. Största delen av Eskil Jungarås hemman köptes på 1990 talet, totalt ca 110 ha skog och 20 ha odlad jord. Det sista av hemmanet, med undantag av en stuga, köptes 2008. Andra köpare var bl.a Edgar och Dorita Jungarå.


\jhoccupant{Jungarå}{Bo}{1996--\allowbreak 2006}
Bo Eskil, \textborn 15.09.1947, fick genom gåva 1996 hemgårdstomten, samt en annan del av lägenheten. Systern Darlene fick också en del av lägenheten som gåva. Olav Jungarå har köpt båda delarna.


\jhhousepic{272-05856.jpg}{En kohage som andas historia}

\jhoccupant{Jungarå}{Eskil \& Gunnar}{1938--\allowbreak 1996}
Bröderna Eskil och Gunnar övertog hemmanet efter sin far Sigfrid, som dog 1938. Sigfrid gick i Uleåborgs svenska samskola blev student under kriget och studerade efter kriget i Helsingfors handelsinstitut bl.a. engelska, tyska, ryska och ekonomi; han blev ut dimmiterad 1948. Han emigrerade till USA 1951.

Under kriget var Gunnar jordbrukare, sen brukade Eskil och Gunnar jordbruket gemensamt tills Eskil emigrerade med sin familj 1951, först till Kanada och sen till USA. Året därpå emigrerade också Gunnar.

Sigfrid hade rest till Olympia på västkusten, där hans farbror Willam var delägare i en kooperativ fanérfabrik. År 1920 satsade ca 100 personer mest emigranter från Finland och Sverige \$ 500 var för att starta en fanerfabrik. Virke fanns det gott om, kunnandet var det sämre med, men de kom igång och redan 1927 producerade de ca 12 \% av USA:s totala fanérproduktion. Bill Gustafsson, som han kallade sig i Amerika, var föreståndare i 30 år för Olympia Vaneer. När han emigrerade i början på 1900-talet hade redan tre syskon och ett 15-tal kusiner emigrerat från Prästas. Mera om Eskil \jhbold{gård nr 12}.


\jhoccupant{Jungarå}{Isak \& Anna-Sofia}{1916--\allowbreak 1938}
Isak, \textborn 28.03.1891, gifte sig 1915 med Anna-Sofia, \textborn 23.04.1893, född Montin på Måtar hemman.
\begin{jhchildren}
  \item \jhperson{Runar Joel}{21.10.1915}{08.08.1916 }
  \item \jhperson{Anders Runar}{21.07.1917}{22.07.1917}
  \item \jhperson{Gustav Sigfrid}{01.06.1919}{31.05.1996, till Amerika}
  \item \jhperson{\jhbold{Eskil} Johannes}{23.12.1921}{30.07.2001, till Amerika}
  \item \jhperson{Agda Elise}{27.09.1924}{30.11.1983, g m Sigurd Björklund}
  \item \jhperson{\jhbold{Gunnar} Bernhard}{20.06.1928, till Amerika}{}
  \item \jhperson{z}{z}{}
\end{jhchildren}
Isak var en storväxt och kraftfull karl. Han övertog sina föräldrars jordbruk 1916, först halva senare andra halvan, som omfattade ca 22 ha odlat och 98 ha skog. I början på 1930-talet byggde de ett nytt fähus i cementtegel med hölada på vinden. a = fähus, b = bodar och redskapslider.

Isak \textdied 16.06.1938  ---  Anna Sofia \textdied 03.12.1945


\jhoccupant{Jungarå}{Gustaf \& Katarina}{1880--\allowbreak 1916}
Gustaf Jungarå d.ä., \textborn 28.06.1859, gift med Katarina, \textborn 19.12.1864, född Jungar på Jungar hemman.
\begin{jhchildren}
  \item \jhperson{Anna-Sofia}{30.06.1883}{07.05.1908, g m E. Källman, gård 15}
  \item \jhperson{Hilda Katarina}{25.03.1886}{17.08.1927 g m 1 J Södergård, 2 Jukka Ojanperä}
  \item \jhperson{Anders Gustaf}{15.05.1888}{27.04.1925, till Amerika}
  \item \jhperson{\jhbold{Isak} Joel}{28.03.1891}{16.06.1938}
  \item \jhperson{Johan Edvard}{19.09.1893}{02.02.1962, till Rundt hemman}
  \item \jhperson{Ester Emilia}{12.05.1896}{27.05.1896}
  \item \jhperson{Matts William}{03.06.1897}{16.01.1956, till Amerika}
  \item \jhperson{Artur Edvin}{16.12.1899}{11.03.1903}
  \item \jhperson{Ester Emilia}{08.2.1903}{26.9.1980, g m Alfred Ruotsala/ Elenius}
  \item \jhperson{Edit Maria}{29.09.1905}{03.08.1977, g m Ivar Vestin}
\end{jhchildren}
Gustaf d.ä. övertog halva hemmanet efter sin far 1880.  Katarina eller Kajsa, som hon kallades, gick 2 år i Tollikko folkskola med goda betyg i bl.a. räkning. Kajsa levde längre än både mannen, sonen och sonhustrun, därför blev hon som en mor för barnbarnen.

Gustaf \textdied 18.10.1930  ---  Kajsa \textdied 13.04.1951


\jhoccupant{Jungarå}{Anders Gustaf \& Anna-Greta}{1845--\allowbreak 1880}
Anders Gustaf, \textborn 22.10.1828 i Jeppo, gift 1848 med 1) Anna-Greta Mietala, \textborn 19.06.1826--\textdied 06.09.1868 i tyfus, 2) med Anna Sanna, \textborn 26.02.1839--\textdied 09.10.1906, född Mickelsdotter från Laggars i Kantlax. Hennes tidigare man var Johan Kaukos från Gunnar. Alla barnen från första giftet.
\begin{jhchildren}
  \item \jhperson{Johannes}{17.06.1849}{1870-talet}
  \item \jhperson{Anna Susanna}{17.10.1851}{17.09.1923, g m Erik Andersson}
  \item \jhperson{Samuel}{02.11.1854}{21.06.1902, blev jordbrukare på ena halvan}
  \item \jhperson{Maria}{18.11.1856}{15.06.1882, g m Erik Henriksson Åvist, Nabba}
  \item \jhperson{\jhbold{Gustaf}}{28.06.1859}{18.10.1930}
  \item \jhperson{Johanna}{05.05.1861}{20.04.1935, g m Johannes Gunnar}
  \item \jhperson{Matts}{29.07.1863}{03.10.1905, jordbrukare på Böös och Ryss}
  \item \jhperson{Sofia}{19.12.1865}{19.05.1903, g m Erik Tollikko}
  \item \jhperson{Jakob}{09.05.1868}{07.09.1868}
  \item \jhperson{z}{z}{}
\end{jhchildren}
Anders Gustaf föddes i landsvägsdiket vid Krouvi under färden från Nurmo till Jungarå, där hans far hade tagit över halva hemmanet. Fadern dog dock redan när Anders Gustaf var 4 år. Han blev småningom jordbrukare och köpte sina bröders hemmansdelar för 785 rubel. På 1850-talet var hemmanet 241 ha. Ända till 1850-talet bildade gårdarna nr 120 och nr 20 en gemensam gårdsplan. Detta ändrades i och med att Anders Gustaf kortade av prästgården samtidigt som den svängdes 90 grader mot vägen. Före gården svängdes fanns en vattenbrunn i östra ändan (där bl.a. ett barn drunknade). I västra ändan av gården fanns en stor jordkällare. Infarten till gårdstomten gick genom en port i stallbyggnaderna, som fanns längs med vägen.

År 1879 gifte Anders Gustaf om sig med änkan Anna Sanna efter bonden på Gunnar hemman, Johan Kaukos, som blev ihjälstångad av en tjur.

Anders Gustaf \textdied 06.02.1894  ---  Anna-Sanna \textdied 09.10.1906 på Gunnar, där hon bodde som änka och sytningshjon.


\jhoccupant{Elenius}{Anna-Helena}{1832--\allowbreak 1845}
Anna Helena, \textborn 08.04.1804, gifte om sig efter Matts' död år 1839 med Erik Hägglund, \textborn 1805 i Pedesöre. Erik flyttade in i Elenius hus, enligt tingsprotokoll uppstod snart tråkiga tvister mellan honom och de omyndiga barnens rättsbevakare.
\begin{jhchildren}
  \item \jhperson{Johanna Fredrika}{1839}{1897, bodde med systern på Karkaus}
  \item \jhperson{Sanna Brita}{1841}{}, g m Johan Jungar, blev ägare till Karkaus hemman
  \item \jhperson{Erik Forsberg}{1842}{1913}, g m Sanna Andersdotter på Måtar
  \item \jhperson{z}{z}{}
\end{jhchildren}


\jhoccupant{Elenius}{Matts \& Anna-Helena}{1828--\allowbreak 1832}
Matts, \textborn 29.04.1899--\textdied 23.06.1832 på Lillollas, gift 1826 med smedsdottern Anna-Helena, \textborn 08.10.1804, född Forsberg på Kimo Bruk.
\begin{jhchildren}
  \item \jhperson{Johan Samuel}{19.05.1827}{21.09.1902}
  \item \jhperson{\jhbold{Anders Gustaf}}{22.10.1828}{06.02.1894}
  \item \jhperson{Matts}{06.02.1832}{04.02.1857}
  \item \jhperson{z}{z}{}
\end{jhchildren}
Matts var 10 år när fadern Thomas Elenius dog. Året därpå började han i Nykarleby Pedagogi tillsammans med bröderna Isak och Gustav. Brodern Samuel tillsammans med halvbrodern Göran Roos började i Vasa Trivialskola. 16 år gammal dog hans mor, som dagen före sin död testamenterade halva hemmanet till Isak, andra halvan till Gustav, äldsta sonen Johan fick Sandbacka hemman på Mietala. Till Göran Roos hade Susanna Elenius år 1812 köpt Herrgård hemman på Mietala. Samuel blev student 1816 och kaplan 1819.

Matts studerade och var butiksbiträde i Gamlakarleby och från 1821 var han bokhållare på Östermyra bruk. Han flyttade som sagt hem 20.10.1828 för att överta en del av hemgårdshemmanet. På färden från Nurmo fanns förutom föräldrarna den 1,5 åriga sonen Johan Samuel och Anna Helenas syster Brita. Den 22.10 föddes familjens andra barn i landsvägsdiket vid 	Krouvi, en halv kilometer från hemmet de var på väg till. Han fick namnet Anders Gustaf. Hemmanet köpte han av brodern 	Gustaf. Han betalade 1650 riksdaler för det åt brodern, som i sin tur blev bondmåg på Slangar. Matts dog drygt 3 år senare den 23.06.1832.


\jhoccupant{Elenius}{Gustav}{1816--\allowbreak 1828}
Efter moderns död var Gustav endast 13 år när han genom testamente blev ägare till halva hemmanet. Hur jordbruket sköttes, av drängar och pigor eller av förmyndare, är obekant. År 1828 sålde han hemmanet till brodern och blev själv jordbrukare på hustruns hemgård på Slangar.


\jhoccupant{Elenius}{Susanna}{1809--\allowbreak 1816}
När Susanna blev änka måste hon flytta bort från kaplansbostället Lillollas, som de bott i parallellt med Jungarå. År 1812 köpte hon Herrgård hemman på Mietala till sonen Göran Roos för 500 riksdaler. Han gifte sig 1816 med bonddottern Anna Henriksson från Gunnar. Samma år, den 11.09.1816, dog Susanna i lungsot.


\jhoccupant{Elenius}{Thomas \& Susanna}{1796--\allowbreak 1809}
Thomas, \textborn 10.12.1749 i Tervajoki, gift 1794 med Susanna, \textborn 17.12.1767 i Nykarleby. Susanna var först gift med länsskrivaren Mårten Roos och fick sonen Göran, \textborn 08.08.1790. Samma år dog maken. Flicknamnet var Paulin och släkten var handelsborgare i 	Nykarleby.
\begin{jhchildren}
  \item \jhperson{Johan Jakob}{19.05.1795}{16.01.1868, flyttade 1816 till Sandbacka}
  \item \jhperson{Samuel}{10.10.1796}{04.06.1828, kaplan i Kalajoki}
  \item \jhperson{Matts}{29.04.1799}{23.06.1832, bokhållare i Östermyra 1821--\allowbreak 1828}
  \item \jhperson{Isak}{07.07.1801}{08.11.1873}
  \item \jhperson{\jhbold{Gustav}}{27.08.1803}{19.06.1881}
  \item \jhperson{Michael}{14.09.1804}{15.09.1804}
  \item \jhperson{Israel}{10.12.1807}{11.04.1808}
  \item \jhperson{Aron}{10.12.1807}{06.03.1808}
\end{jhchildren}
Thomas föddes med namnet Helenius, tog senare namnet Elenius. Han började i Vasa trivialskola 9 febr 1762, blev student 1771. Året därpå, 1772, svor han trohetseden inför konung Gustav III. Fil. kand. blev han 1773, med en avhandlingen ``Det värde varuti ekonomien blivit hållen av åtskilliga gamla folkslag'' med Pehr Kalm som opponent. Följande år, 1774, blev han fil.mag. med sitt verk under titeln ``Apoqrismi Institutionem juventus in scholis'' eller ``Om skolans betydelse för utbildningen av ungdomen'', med Jacob Arnt Carp som opponent. Förutom svenska och finska fick han lära sig latin och franska under sina fem år i Åbo. Han prästvigdes 1777 och tillträdde genast kaplansadjunttjänsten i Nykarleby. Ordinarie kaplan blev han 1780.

Han bodde som ungkarl på Lillollas kaplansboställe fram till 1794. Efter giftermålet bodde makarna kvar på Lillollas och först när den nya mangården blev klar i sekelskiftet (som fortfarande står kvar, nr 20), flyttade de mera varaktigt till Jungarå. Som kaplan hade han ett stort distrikt att besöka. Från Wuoskoski i Alahärmä till Nykarleby var det lång väg, förflyttningen skedde med häst, mest ridande, som det står i sägnerna ``med sin ståtliga hingst''. För att i någon mån minska på arbetsbördan hade han en egen prästvigd adjunkt anställd i början av 1800-talet. Prästerna på 1700-talet var inte bara själasörjare utan hade som målsättning att utveckla bygden och jordbruket i synnerhet. Det kan också utläsas ur hans avhandlingar.

Intresset för jordbruk hade säkert en andel i att han köpte egna 	hemman. Jordbruket på Lillollas bestod av 35 odlad jord och 165 ha skog. År	1796 köpte han 1/6 mantal av Jungar 39 för 500 riksdaler, 1800 	köpte han 1/6 mantal på Bjon för 277 riksdaler, 1804 köpte han 1/12 	mantal av Lisa och Johan Mattsson på Jungar 39 för 666 riksdaler. Sista fjärdedelen av Jungar 39 köpte han år 1807 av en annan Johan Mattsson (systerson till den förra) för 555 riksdaler. År 1806 köpte han 	Sandbacka hemman på Mietala för 333 riksdaler.

Hemmanet på Jungarå uppgick till 450 ha, odlade jorden 48 ha? Huset, som blev deras hem, byggdes i början av 1800-talet. Redan när han skrev sin fil.kand.-examen högaktade han jordabruket. Han fick praktisera sina färdigheter som jordbrukare när han blev kaplan och tydligen gick affärerna bra, eftersom han fortsatte att köpa jord. Enligt berättelser från sitt barnbarn i Monå, tröskades på Prästas hundra riar varje år.

Efter slaget vid Juthas den 13 september 1808, skulle en grupp kosacker ta sig över Nykarleby älv till den östra sidan. Då ingen bro fanns, vandrade de söderut längs landsvägen fram till Ryss hemman. Där skulle de enligt sägnen ha rivit ned en riebyggnad och av 	virket gjort en flottbro och så kommit över älven. Ett av deras mål skall ha varit ``Prästas'', som Jungarå alltfort kallades, men hette också ``Krogen''. Prästen skall ha blivit varskodd och låtit sin 13-åriga son Johan  föra åtta hästar i säkerhet till Pittkant skogsäng ca 6 km österut från gården. Hemma hos sig höll han sin högt avhållna 	hingst. Då kosackerna kom till gården lade de märke till den vackra hingsten och beslöt att ta den med sig. Men Elenius ville förstås förhindra detta och en häftig ordväxling och slagsmål uppstod, med den följd att hästen togs och Elenius blev sårad och dog 8 månader senare, den 25.05.1809 i ``feber''. Det finns andra versioner på den här sägnen, men innebörden och utgången är densamma.



\jhhouse{Prästbacka}{12:40}{Jungarå}{17}{21, 21a}


\jhoccupant{Jungarå}{Gårdsammanslutning}{2014--}
Jungarå gårdsammanslutning ägs av Kim och Måns Jungarå. Gårdens inriktning är dikor och morotsodling. Morötterna lagras i eget kyllager, omändrat från verkstad år 2012 och tillbyggt med packeri. Under vintern sorteras och packas morötterna, som odlas på ca 6 ha. Under sommarens rensningssäsong finns ca 12 anställda, resten av året 2-3 st. Arbetarna är hemma från Ukraina. År 2016 byggdes ett nytt dikostall för 50 kor. Gården drivs ekologiskt, både djurhållningen och växtodlingen. a = fähus, b = morotslager, c = fd pannrum, biogas,  d = dikostall.


\jhhousepic{Jungara nr 21.jpg}{Olav och Gunilla Jungarå}

\jhoccupant{Jungarå}{Olav \& Gunilla}{1979--\allowbreak 2014}
Olav, \textborn 29.06.1955 i Jeppo, gifte sig 21.07.1979 med Gunilla, \textborn	30.04.1956, född Björklund från Jussila i Munsala.
\begin{jhchildren}
  \item \jhperson{\jhbold{Kim} Olav Mikael}{24.11.1979, utbildad maskin- och metallingenjör}{}
  \item \jhperson{Sofie Caroline}{07.04.1982, utbildad till ekonomiemagister}{}
  \item \jhperson{\jhbold{Måns} Carl Kristian}{27.10.1986, utb. bilmontör, skogsmaskinförare}{}
\end{jhchildren}
Olav studerade vid Korsholms lantmannaskola 1972--73 och övertog 1979 hemgården efter sina föräldrar. Gården bestod då av 35 ha odlad jord och 50 ha skog. Makarna ändrade inriktning på gården från mjölkproduktion till köttproduktion samtidigt som ekonomibyggnaden förstorades med plats för 150 ungnöt. Efter inträdet i EU -95, satsades mera på dikor. Gården odlas ekologiskt sen 1992 och omfattas idag av 100 ha odlad jord och 220 ha skog.

Genom sitt intresse för nytänkande byggde Olav 1984 en biogasanläggning för värmeproduktion. Den togs ur drift efter ändrad produktionsinriktning. Bostaden och sonen Kims bostadshus uppvärms idag från en gemensam flisvärmecentral. Olav har också entrepenadverksamhet med bl.a skogsprocessor och grävmaskiner. Han har suttit bl.a i ÖK:s fullmäktige och förvaltningsråd, i ÖSL:s styrelse, lantmannagillet och i Jeppo lokalavdelning av Ösp.

Gunilla studerade 1976--78 till barnträdsgårdslärare i Jakobstad och arbetar sedan 1993 i Jeppo församling som dagklubbsledare. Hon deltar även i nötproduktionen och jordbruksarbetet på gården.


\jhoccupant{Jungarå}{Henrik \& Gunhild}{1945--\allowbreak 1979}
Henrik och Gunhild byggde gården 1970. Tomten Rn:12:40 där gården finns, köptes av Olof Dahlström år 1967. Fähuset från 1956 renoverades och förstorades 1968 och en kalluftstork byggdes. År 1979 förstorades bostaden åt föräldrarna, 1989 flyttade Henrik och Gunhild till en lägenhet på Åkervägen i Jeppo.

Henrik \textdied 05.12.1993  ---  Gunhild \textdied 04.01.2004. Mera på \jhbold{gård nr 120}.



\jhhouse{Prästas}{12:35}{Jungarå}{17}{120, 120a-c}


\jhhousepic{H Jungara 120.png}{Henrik och Gunhild Jungarå}

\jhoccupant{Jungarå}{Henrik \& Gunhild}{1945--\allowbreak 1979}
Henrik, \textborn 09.08.1913 i Jeppo, gifte sig 15.06.1947 med Gunhild, \textborn 11.03.1917, född Sundvik från Hirvlax, Munsala.
\begin{jhchildren}
  \item \jhperson{\jhbold{Olav}}{29.06.1955}{}
  \item \jhperson{Kennet}{23.04.1960}{08.02.1965, dog i traktorolycka}
\end{jhchildren}
Henrik gick i Evangeliska folkhögskolan på Keppo gård 1932--\allowbreak 1933 samt Korsholms lantmannaskola åren 1934-35. Han var med i vinterkriget och i fortsättningskriget sammanlagt 6 år. Henrik övertog 1945 hälften av föräldrarnas gård på 52 ha, varav 24 ha odlad jord. Han var med i olika lokala styrelser och föreningar, bl.a. kommunstyrelsen.

Gunhild studerade vid Breidablick ambulerande folkhögskola, samt vid Svenska teoretiska mejeriskolan i Vasa 1942-43. Hon hade tjänst bl.a på Jakobstad mjölkcentral (när Jakobstad blev bombat 22.2.1943) och Jeppo Andelsmejeri från 5.6 1944. Vid en tragisk olycka 30.9.1944 brände sig disponenten R Liljeqvist på en exploderande blåslampa med dödlig utgång 2 veckor senare. Före en ny disponent var vald var Gunhild tillförordnad disponent några veckor.

Efter giftermålet 1947 deltog båda gemensamt i skötseln av gården, den gamla hemgården från 1700-talet och ekonomiebyggnaden var omoderna och gårdstomten alltför liten för utbyggnad av ett nytt fähus. När granntomten var till salu köpte Henrik den av Olof Dahlström (Silvast nr 50). På tomten fanns ett fähus från 1956 som moderniserades 1968 och fick plats för gårdens mjölkbesättning. På lägenheten uppfördes 1970 en ny mangård, gård nr 21. a = fähus, stall, boda, kärrlada och hölada, b = bastu, c = trösklada.


\jhoccupant{Jungarå}{Gustav \& Lina}{1898--\allowbreak 1945}
Gustav, \textborn 28.04.1875 i Jeppo, gifte sig 1896 med Karolina, \textborn 19.09.1875 på Back i Jeppo.
\begin{jhchildren}
  \item \jhperson{Gustav Lennart}{12.10.1897}{17.10.1897}
  \item \jhperson{Elna Maria}{21.09.1898}{26.09.1897}?
  \item \jhperson{Hilda Irene}{19.09.1899}{17.08.1986}
  \item \jhperson{Johannes Artur}{09.02.1902}{16.01.1995}
  \item \jhperson{Elna Maria}{04.05.1904}{23.10.1990, gift med H Sandberg}
  \item \jhperson{Ester Alina}{21.10.1906}{20.01.1996, gift med V Sandberg}
  \item \jhperson{Henrik Uriel}{25.12.1908}{08.11.1909}
  \item \jhperson{Saima Olivia}{24.09.1910}{12.08.1992, gift med Martin Kolam}
  \item \jhperson{\jhbold{Henrik} Olof}{09.08.1913}{05.12.1993}
  \item \jhperson{Gerda Viola}{23.07.1916}{29.07.2016, gift med Uno Gunell}
  \item \jhperson{Anna Ingegärd}{23.10.1919}{16.12.2013, g m Tycko Larsson fr Asphyttan Sverige}
\end{jhchildren}
Gustav och Karolina mottog en tredjedel av hans föräldrahemman 	på Jungarå 1898 och fick två år senare köpa hälften av broder Matts 	hemmansdel. De hade då 22 ha odlad jord och ca 90 ha skogsmark. Gustav var en kraftfull man och under sin tid som jordbrukare nyodlade han med yxa, järnspett och spade 35 tunnland kärr- och mossmark till åker. Gustavs stora intresse var hästar och travsport, många är priserna från både vinter- och sommartravtävlingar. Till hans hobbies hörde också träslöjd. Hustrun Karolina var den första kvinnliga eleven som blev dimitterad från Jungar folkskola . ``Vi barn 	hade god hjälp av mors skolgång'', berättade dottern Hilda Jungarå. Karolina var också den första allmoge som bar brudslöja. Karolina var en tid mejerska vid det av Elias Lönnqvist nystartade mejeriet i Ekola, Alahärmä.

Lina \textdied 25.09.1949  ---  Gustav \textdied 03.04.1964


\jhoccupant{Jungarå}{Johan \& Anna Sanna}{1872--\allowbreak 1898}
Johan, \textborn 18.08.1838 i Jeppo, gift 1861 med Anna Sanna, \textborn 02.07.1843, född Kackur på Hilli i Ytterjeppo.
\begin{jhchildren}
  \item \jhperson{Isak}{16.07.1862}{06.09.1944, senare Ruotsala}
  \item \jhperson{Anders}{16.08.1864}{24.06.1865}
  \item \jhperson{Johan}{30.03.1866}{03.06.1935}
  \item \jhperson{Katarina}{08.04.1868}{06.06.1950, gift Jungar}
  \item \jhperson{Erik}{07.11.1870}{26.04.1959, senare Elenius}
  \item \jhperson{Matts}{31.01.1873}{30.09.1899}
  \item \jhperson{\jhbold{Gustav}}{28.04.1875}{03.04.1864}
  \item \jhperson{Anna-Sanna}{28.071878}{25.03.1949, gift Finskas/ Lundvik}
  \item \jhperson{Henrik}{07.10.1880}{23.03.1962, senare Jungell}
  \item \jhperson{Ida Maria}{05.01.1882}{26.09.1962, gift Julin}
  \item \jhperson{Hilda Johanna}{08.07.1885}{18.10.1914, gift Elenius}
\end{jhchildren}
Johan och Anna-Sanna övertog hans hemgård drygt tio år efter giftermålet. År 1856 omfattade jordbruket 218 ha, varav 39 ha odlad jord och 179 ha odlingsbar mark och skog. Dessutom hörde en kvarn i forsen till gården. Johan och hans broder Matts (Gunnar) fick tillsammans med hemmanen av sin far en hel del lån att reda upp. Som energiska bönder var de skuldfria efter tio år. Åt två söner byggde han en stor mangård 1898. Av Isaks 6 söner blev alla 	bönder på egna jordbruk.

Sytningskontrakt 8 okt 1901: ``Jag Johan Isaksson Jungarå som genom gåva af denna dag enligt villkor som i övrigt finnes däri 	upptagna att mina söner Johannes Matts och Gustaf Johanssöner samt deras hustrur givit mitt egande Jungarå 5/24dels mantal skattehemman No 12 i Jungar by af Jeppo kapell, har jag till lifstids sytning från hemmanet åt mig förbehållit nedannämnda förmåner och persedlar årligen: 1. Till boningshus framstugan med därpå befintliga vind samt en boda, färdig huggen och hemkörd ved till bränsle efter behof. 2. Nio hektoliter råg, tre hektoliter korn ren och strid säd. 3. Fem liter sötmjölk dagligen, men om mjölken på något sätt förfalskas så är jag berättigad till pengar efter 15 p litern. 4. Två fullvuxna får med lång ull, tio kilo färsk fläsk, en hektoliter salt, en fjärding strömming, två par manspjäxor. 5. För sädes- och potatisodlingen. Färi strandskiftet som skall av hemmansägaren gödas, brukas och stängas. Häst och redskap till och från kyrkan och andra nödiga resor. Av skog får jag begagna mig till mitt eget behof. Trettio mark i kontanter årligen. 6. Skötsel på min sjuksäng och efter döden en hederlig begrafning. Då jag med döden avfgot skall min boningsstuga och boda tillfalla den hemmansägare som bekommer min gamla stuga.''


\jhoccupant{Elenius}{Isak \& Caisa}{1816--\allowbreak 1872}
Isak, \textborn 07.07.1801 i Jeppo, gift 1825 med Caisa, \textborn 20.10.1807, född Gunnar i Jeppo.
\begin{jhchildren}
  \item \jhperson{Thomas}{27.09.1826}{09.09.1833, dog i rödsot}
  \item \jhperson{Erik}{03.11.1830}{10.09.1833, dog i rödsot}
  \item \jhperson{Isak}{17.07.1833}{27.04.1854}
  \item \jhperson{Maja-Caisa}{12.12.1835}{26.02.1874, gift Klockars}
  \item \jhperson{\jhbold{Johan}}{18.08.1838}{22.04.1913}
  \item \jhperson{Matts}{10.01.1843}{02.11.1911}
  \item \jhperson{Susanna}{18.01.1846}{1885}
  \item \jhperson{Gustav}{09.01.1849}{13.01.1849}
  \item \jhperson{Anna}{23.12.1850}{08.08.1938, gift Mietala}
\end{jhchildren}
Isak Elenius antog efternamnet Jungarå sedan Jungar Nr 39 på 1830-talet officiellt ändrats till Jungarå. Han blev ägare till hälften av Jungarå hemman via ett testamente av sin mor, änkefru Elenius, som är daterad 10 sept 1816, en dag före hennes död. Isak var då 15 år. Isak blev med åren en energisk jordbrukare och kvarnägare, som förstorade den odlade arealen. 1894 fanns på gården 3 hästar, 15 kor, 1 kalv, 1 gris och 16 får. Det var det praktiska som gällde, likaså när det gällde inredning i bostaden - inte var faderns franska rokokomöblemang praktiskt, stolarna var låga o.s.v., varför möblemanget eldades upp.

Han iklädde sig många gånger den otacksamma uppgiften som borgensman för sina släktingar. År 1860 blev han ägare till 1/8 dels mantal av Tollikko hemman, 1866 en del av Gunnar hemman, 1867 sin svågers hemman på Gunnar, 1868 blev han tvungen 	som borgenär för smeden Jansson på 	Nygårds (smeden Jansson gift med halvbroderns dotter) i Nykarleby överta all hans 	fasta och lösa egendom och betala rådman Lybäck 1102 mk och 17 penni finsk mynt!	År 1868 ytterligare borgen för en 1/8 del av Tollikko. Caisa och Isak upprättade ett testamente kort före hennes död vilket inleds med orden: 	``Som vi alla äro vandringsmän och ej veta dagen ej heller stunden döden kallar oss hädan, som ännu mera av den inbördes kärlek vi oss emellan under vår sammanlefnad haft, så hava vi idag under moget betänkande och av fri vilja osv.''. Det skrevs på 	Jungarå 11 december 1851. Isak var en av de pådrivande och ledande personerna vid byggandet av Jeppo kyrka, som blev 	färdig 1861.

Caisa \textdied 13.12.1851  ---  Isak \textdied 08.11.1873

Gården, som under den tid Thomas Elenius var jordbrukare inhyste drängarna och pigorna, blev troligen utbyggd till den storlek gården hade under Elenius period. Gamla delen bestod endast av det som senare kallades för framstugan. Huset revs av Olav Jungarå 1981.



\jhbold{Jungar hemman nr 39 i Överjeppo}, \jhbold{Johan \& Lisa Mattsson (Jungar)}

Johan och Lisa sålde 1/6 dels mantal av Jungar nr 39 år 1804 till Thomas Elenius för 666 riksdaler. I det köpet ingick också förutom sytningar även mjölkvarn, kvarnställe och sjuderi. Denna Lisa (dotter till krögar-Lisa) gick i Nykarleby trivialskola 1796 (troligen uppmuntrad av Thomas Elenius). Sista delen, 1/12 dels mantal, sålde de 1807 för 555 riksdaler.

Första delen 1/6 mantal av Jungar nr 39 köpte Thomas Elenius av Jonas Moberg för 500 riksdaler. Moberg igen hade via 	inteckning övertagit hemmansdelen 1783 av Lisa Jungar den yngre \textborn 1760. Hon och hennes syskon Erik \textborn 1751, Maria \textborn 1753 och Marias son Johan fick 1/4 var av föräldrarna.

Den gamla sädesboden från 1778 (bild) och den mindre gårdsbyggnaden var troligen boställe för någon av brukarna. Själva krogen fanns några hundra meter längre söderut där älven grenar sig och det ställe där alla forsar ner till Nykarleby började. Blomstringstiden för krogen och bränneriet var i slutet av 1600-talet och första halvan av 1700-talet. Då var trafiken på älven livlig, handeln med tjära i Nykarleby från trakter ända från Kuortane och Lappo gick sommartid längs älven förbi Jungar Krog. Flotten eller pontonen hade egen skötare långt in på 1800-talet, den s.k. färjkarlen. Vägen på östra sidan tog slut vid Jungarå, så det gällde att komma till andra sidan med fortskaffningsmedel häst och kärra. Vintertid var det isvägen mot Lappo som användes, också då var krogen med tillhörande härbärge strategiskt rätt placerad.

\jhhousepic{Jungaraboda 120.jpg}{Sädesboden står numera vid hembygdsmuséet}

Jungar Krog hade inte det bästa ryktet. I ett klagomål till kyrkorådet i Nykarleby 1723 berättas om hur det alla kvällar, ja till och med under högmässotid, levdes ett vilt liv med skrål, svordomar och kortspel vid krogen. Under stora ofreden tid 1714--\allowbreak 1722 var krögaren i Sverige och en Johan Isaksson hade övertagit krogverksamheten. De resandes bekvämlighet var det också si och så med, rum fanns endast i boningsstugan och en liten kammare i gaveln. Kanske var det under hans tid som krögare som ryktet blev dåligt. Den tidigare ägarens son, Jacob Mårtensson, blev enligt ett tingsprotokoll 17 okt 1722 rättmätig ägare till krogen och Isaksson avstod densamma.

Dottern Lisa blev krögare 1755, var gift med Johan Johansson Jungar. Johan Bladh, som övertog Keppo 1759, hade stora planer för området. Han ansökte också om tillstånd att öppna en krog på västra sidan av älven mittemot Jungar Krog. Johan Jungar väckte åtal mot Johan Blath och frågan behandlades vid tinget i Nykarleby 19 nov 1765. Där upplästes en av Jungar signerad besvärsskrift, i vilken han berättar att krogen varit i svärfaderns släkts ägo sedan 1677, och att han nu är den fjärde generationen. Jungar säger vidare att han nu i femton års tid förestått krogen och färjan ``till allas nöje och tjänst, men ändå har Bladh på andra sidan av ån mittemot krogen och färjstället på en liten ängsstig honom till stort förfång byggt en krog utan att han efterhört Jungars samtycke''.

Bladhs krog hade erlagt ``minuteringsavgiften'' allt sen 1762 och hade landshövdingens tillstånd till detta krogställe. Målet avvisades och Bladhs krog fick stå kvar.



\jhbold{Jungar Krog}

Se även inledande text om Jungarå hemman.

\begin{enumerate}
  \item Sista krögaren var Lisa Jungar, \textborn 1727, gift med Johan Johansson Jungar, \textborn 1724, fick 15 barn, 12 levde till vuxen ålder. Härifrån härstammar bl.a. släkten  Storgårds (Isotalo) i Härmä och i närområdet finns många ättlingar till samma släkte. Lisa var krögare 1755-67. Lisas far Mårten Jacobsson fortsatte till 1771.
  \item Mårten Jacobsson, \textborn 1707, gift med Sofia, \textborn 1694, krögare 1741-54. Han var sonson till Knut.
  \item Knut Hindersson, gift 2 med Elsa Tohmasdr och sonen Jacob Jacobsson, gift med Brita Arvidsdr. Krögare 1734-41.
  \item Jacob Mårtensson, gift med Brita Arvidsdr. Krögare 1725-33.
  \item Under stora ofreden kom en Johan Isaksson till stället. Enligt tingsutslag 1722 återfick de gamla ägarna äganderätten och Isaksson lämnade stället.
  \item Henrik Knutsson gift med Maria. Krögare 1704-13.
  \item Knut Hindersson, gift 1 med Gertrud Olofsdr. Krögare 1686--\allowbreak 1703 och 1723.
  \item Mårten Sigfridsson, gift med Gertrud Olofsdr. Krögare 1677--\allowbreak 1685 samt kronogästgifvare, paret kom från Mietala, hustrun gifte sig med Knut 1685.
  \item Gustaf Krögare är den som först nämns i handlingar som krögare år 1669. De var alla från samma släkte.
\end{enumerate}

Om hemmanet står det 1678, när åbon Simon Hilli kom att tillträda detta kronohemman som delägare, att det är alldeles förfallet, åkrar halfparten i linda oduglig och sandig vordt, engarna mestadels skogsgångna, inga hus, vargarna genom ruttna fähusväggar i vinter inmundigat sig och alla fåren uppätit, har ej mer än 2 kor och en elak häst, Simon är bråttfällig.



\jhhouse{Kvarnplats}{12:4}{Jungarå}{17}{121, 121a-d}


\jhbold{Jungarå Kvarn}

Enligt gamla dokument fanns på hemmanet vattenhjulskvarn redan på 1700-talet.

Den 2 september 1879 undertecknade Johan Isaksson Jungarå ett köpebrev om utarrendering av kvarnställe och halva Jungarå fors och dessutom 10 kappland för kvarnplats och gårdstomt med vägrätt. Arrende årligen: 3 tunnor spannmål, hälften råg och hälften korn, dessutom skall kvarnägaren till hemmet mala allt vad som hemmanet fordrar, till mjöl, gryn, skräd m.m.

Bonden Anders Gustaf Mattsson Jungarå, som ägde andra halvan, avsäger sig sin rätt i Jungarå fors samt till uppdämning av vattnet till mjölnaren Matts Wilhelm Harald eller Forslund mot en årlig ersättning om fri malning av 25 tunnor 	spannmål. Undertecknandet är 2 september 1882. Arrendetiden var femtio år.

\jhhousepic{Jungara kvarn 121}{Jungarå Kvarn}

Redan 1884 hade borgesmannen Sjökapten J.J.Rank övertagit äganderätten till kvarnen, då hade han varit kvarnägare i Jungar redan 10 år. 1886 sålde han kvarnarna och återgick till befattningen som kapten. Omnämnd i Wasa tidning 1893, hade han anlänt som kapten på barkskeppet Concordia med full last salt från Medelhavet till Santos Brasilien 1891. Två år vistades Concordia i Santos, orsaken var en mellan kaptenen i och för rederiet förd process mot lastmottagaren handelshuset Zerenner Bulöv \& co. Denna process kostade den åldrige skeppsbefälhavaren mycket besvär till följd av ett bristfälligt ``konnisement''. Då detta inte skedde innan skeppets 19 ``liggdagar'' gått till ända, vidtog kapten Rank erfoderliga åtgärder för lastens bortaukionering. Lasten bestående av salt såldes också på auktion, men handelshuset lyckades utverka lossningsförbud. Concordia blev liggande i Santos, skeppet som inte var kopprad för full last, hade inom kort - de delar som ej skyddats - blivit genomborrade av mask. Rank väckte anspråk på ersättning för skadan, vilket också erkändes. Med skadeersättningen på 98.000 milires kunde de undvika utmätning. Tiden gick och kaptenen som kämpade för rederiets bästa dök under i gula febern. Processens viktigaste person var borta och rederiet fick sälja skeppet som var förtärt av maskar och röta för en spottstyver. Rederiet Wasa rederibolag fick inte en endaste milires från Santos, den mest osunda av de osunda brasilianska hamnstäderna. Ett rederis uppgång och fall.

Sedan sågverksdisponenten N.F.Nordman genom transporthandling av den 10 november 1886 tillhandlat sig Julius Ranks ägda besittningsrätt till kvarnställe och kvarn, har ägarna till detta hemman, Johan Jungarå och Gustav Jungarå d.ä., överenskommit om försäljning av kvarnstället och forsrättigheter, samt fri malningsrätt. Summan är 1000 mark, dessutom skall kvarnägaren årligen betala 25 mark i ``afgäld''. Underteknad vid Keppo den 19 januari 1890.


\jhhousepic{jungara fors 1890}{Planeringskarta 1890 för Jungarå fors och kvarn}



Efter Nordmans övertagande byggdes kvarnstuga (försäkrad med lösegendom till 2865 mark). År 1890 uppfördes en smedja och ett stenfähus på tomten. Kvarnen brann ner 24 oktober 1890. Den var försäkrad till 7120 mark. Redan 1891 var kvarnen uppbyggd på nytt och bestod liksom tidigare av 3 par stenar och grynverk.

``Vintern 1897 har inspektör Nordman, som förut är ägare till en av kommunens största mjölkvarnar belägen i Jungarå fors, under loppet av denna vinter låtit uppföra ett sågverk invid nämnda kvarn. Arbetet lär redan vara så långt medhunnet att numera endast återstår att inpassa innanredet, vilket inom några veckor torde vara undangjort	då sågen omedelbart därpå försättas i verksamhet, för vilket ändamål en mängd stock släpats till platsen.'' Utdrag ur tidningen 2 mars 1897.

Åren 1895--96 rensades älven i mindre skala utan större resultat för att minska översvämningarna uppströms. Kronan blev ersättningsskyldig till tre vattenverks ägare i Jeppo för förkortad höst- och vårflod. Till J von Essen för Kiitola kvarn och Keppo schoddyfabrik 5625 mark, till Nordman och Johan Jungarå för Tollikko kvarn 3408 mark och 33 penni, till Nordman för Jungarå kvarn 4375 mark. Åren framöver malde kvarnstenarna vidare tills det på nytt blev tal om rensning av älven.

\jhhousepic{jungara fors 1893}{Planeringskarta 1893 för Jungarå fors och kvarn}

1 februari 1908 hade en värderingsnämnd tillsatts för att värdera Nordmans kvarn- och såganläggning med anledning till expropiering. Värderingsmännen kom till en summa av 44300 mark för vattenkraften jämte såg och kvarn. Summan godkändes av väg och vattenöverstyrelsen och köpebrevet undertecknades 10 febr 1910. Då hade Nils Fredrik Nordman dött, kvarnbolagen gått i konkurs och ärendet sköttes av bankdirektör Erik Söderström samt fröknarna Wilma, Agda och Edit Nordman.

Enligt köpevillkoren skulle 9000 mark innehållas, tills säljaren styrkt sin äganderätt till hela forsen eller överenskommit med övriga eventuella delägare i forsen om äganderättens överlåtande till överstyrelsen. Målet var uppe i häradsrätten angående bättre rätt till andel i Jungarå fors för jordägarna, därtill gjordes anspråk på ränta på kapitalet. Det igen betydde att summan 9000 mark deponerades i Nykarleby aktiebank av väg och vatten. När ärendet var klart och förlikning skett hade banken, Nykarleby aktiebank gått i konkurs. Till följd därav återfick överstyrelsen av det deponerade beloppet jämte ränta 5467 mark och 31 penni eller 50 procent. Största delen av summan betalades till kärandeparten i rättstvisten. Senaten blev inkopplad och gav tillstånd att resterande obetalda summa 4713 mark 63 penni skulle tas från anslaget för Lappo älvs regleringsarbeten åt nämnda parter till slutbetalning. I och med expropieringen upphörde verksamheten och Överstyrelsen för väg och vatten var ägare till kvarnstället med dithörande byggnader.

\jhpic{Forsrensning bild 60.jpg}{Mödosam forsrensning med män och hästar i fysiskt arbete.}

16 april 1914, efter slutförandet av årensningen, sålde Väg och vatten parcellområdet om 0,164ha till kronlänsman Erik Söderström för 76 mark. Sågbyggnaden såldes till Charles von Essen, kvarnbyggnaden till Matts Malkamäki samt kvarnstugan och barack såldes till kiertokauppias Kaarlo Alexi Sarkanen.

3 april 1915 sålde Söderström (skötte Nordmans konkursbo) kvarnplatsen till Sarkanen för 152 mark. 30 maj 1917 köpte Gustav Jungarå kvarnplatsen för 1200 mark. Dottern Hilda fick tomten genom testamente, hon i sin tur testamenterade åt bröderna Johannes och Henrik, som sålde sina delar till Olav Jungarå. Cirkeln är sluten och tomten är i samma släkts ägo på nytt.

Kvarnen revs 1915 likaså sågbyggnaden, a = kvarnstuga, b = magasin, c = smedja, d = stall/fähus. Alla andra byggnader revs av G Jungarå.

Mjölnare:
\begin{enumerate}
  \item Matts Wilhelm Harald
  \item Gustav Andersson Källman
  \item Johan Mattsson Hägglöf
  \item Wilhelm Sandström
  \item Emanuell Hägglöf
  \item Edvard Källman
\end{enumerate}



\jhhouse{Storhagen}{12:39}{Jungarå}{17}{22, 22a-b}


\jhhousepic{274-05863.jpg}{Ralf och Malin Dahlström}

\jhoccupant{Dahlström}{Ralf \& Malin}{2007--}
Ralf, \textborn 08.05.1972 i Jeppo, gifte sig 08.09.2007 med Malin Birgitta, \textborn 12.12.1970, född Storlund i Vasa.
\begin{jhchildren}
  \item \jhperson{Emanuel Storlund}{30.07.1990}{}, i Vasa
  \item \jhperson{Rebecka}{03.04.2009}{}
  \item \jhperson{Alexandra}{04.06.2012}{}
\end{jhchildren}

Ralf arbetar på KWH Mirka som montör, han är också privatföretagare med egen metallverkstad. Verkstaden byggdes omkring 2001 och inrymmer olika slags svarvar och metallbearbetningsmaskiner. På gården har senare byggts ett flislager, verkstaden och bostaden uppvärms med flis. Bostaden förstorades 2011 sen Malin och Ralf flyttat in. Malin är utbildad storhushållsföreståndare, har	arbetat som värdinna/kock på KWH-koncernen. a = pannrum, b = verkstad, c =lager.


\jhoccupant{Dahlström}{Erik \& Hilja}{1958--\allowbreak 2007}
Erik, \textborn 09.04.1936 i Jeppo, gifte sig 1959 med Hilja, \textborn 17.08.1938, född Kari i Alahärmä.
\begin{jhchildren}
  \item \jhperson{Helena}{03.01.1968,  gift Glasberg Maxmo}{}
  \item \jhperson{\jhbold{Ralf}}{08.05.1972}{} ???född
\end{jhchildren}
Erik byggde bostaden ett stenkast från hemgården (gård nr 122) år	1958 och ekonomibyggnaden 1959. Bostaden har tillbyggts år 1975 samtidigt installerades centralvärme. Erik har under hela sin yrkesverksamma tid arbetat som långtradarchaufför, en lång tid på Keppo i utrikestrafik och några år på Thors Trans i  Oravais.

Hilja har också arbetat på Keppo och Mirka. När Ralf övertog gården flyttade Erik och Hilja till en lägenhet på Åkervägen i Silvast, bostad 5a nr 2.



\jhhouse{Jungarå}{12:23}{Jungarå}{17}{122, 122a-b}


\jhhousepic{Jungara 122.jpg}{Dahlströms gård}

\jhoccupant{Dahlström}{Olof \& Sirkka}{1964--\allowbreak 1966}
Olof, \textborn 06.10.1940 i Jeppo, gift 1959 med Sirkka, \textborn 22.05.1938, född Sipponen från Pelkkala.
\begin{jhchildren}
  \item \jhperson{Sven Olof}{23.09.1960}{}
  \item \jhperson{Carina Viola}{20.05.1962}{}
  \item \jhperson{Anne Maria}{03.03.1966}{}
\end{jhchildren}

Olof övertog hemmanet efter sina föräldrar 1962, Före det arbetade han som stationskarl på järnvägen. Som jordbrukare började makarna med mjölkpoduktion. Olof ändrade inriktning och skaffade en större traktor som  han jobbade på vägbyggen med samt utförde dikesplogningar. 1966 sålde han hemmanet till Eskil Jungarå och 1967 fähuset och tomten till Henrik Jungarå. Olof med familj flyttade till Böös 1967 gård nr (-). Samma år revs huset.  a = fähus, stallbyggnad och foderlada, b = bastu.


\jhoccupant{Dahlström}{Anders \& Gerda}{1902--\allowbreak 1964}
Anders, \textborn 27.12.1879 i Jeppo, gift 1932 med Gerda, \textborn 01.08.1911, född Mäenpää på Fors Hemman.
\begin{jhchildren}
  \item \jhperson{Anita Maria}{22.04.1934, gift med Gösta Holgerson från Sverige}{}
  \item \jhperson{Erik Anders}{09.04.1936, gård nr 23}{}
  \item \jhperson{Rolf Ingmar}{20.12.1937, gård nr 118}{}
  \item \jhperson{Gurli Carita}{31.05.1939, gift med Tauno Pukkila}{}
  \item \jhperson{Isak \jhbold{Olof}}{06.10.1940, Silvast nr 50}{}
  \item \jhperson{Vilho Wiljam}{12.01.1942}{08.04.1942}
  \item \jhperson{Runar Alfons}{17.07.1944}{}, bor i Nykarleby
\end{jhchildren}
Anders föddes Jungarå men tog namnet Dahlström. Som barn blev han invalidiserad med en krökt vänsterfot. Trots handikappet skötte han ett stort hemman på 22 ha odlad jord och 90 ha skog. 1956 byggdes med sönernas hjälp ett nytt fähus i tegel med hölada och sädesboda. Anders och Gerda bytte stuga (gård nr--) med Olof när han övertog hemmanet.


\jhoccupant{Jungarå}{Samuel \& Sanna}{1880--\allowbreak 1902}
Samuel, \textborn 02.11.1854 i Jeppo, gift 1. 1872 med Sanna Jungar (Strengell), \textborn 27.01.1854, 2. 1878 med Sanna Jungar, \textborn 28.11.1857, född på Svartbacken.
\begin{jhchildren}
  \item \jhperson{\jhbold{Anders} Gustav}{27.12.1879}{01.09.1969}
  \item \jhperson{Anna Lovisa}{27.11.1881}{25.04.1903, gift med Isak Elenius}
  \item \jhperson{Ida Maria}{31.05.1884}{15.01.1975, gift med Antti Tyni}
  \item \jhperson{Sanna Sofia}{14.04.1886}{20.01.1906}
  \item \jhperson{Hilda Katarina}{09.02.1888}{16.07.1929, gift med Anders Lindèn}
  \item \jhperson{Edla Johanna}{07.06.1890}{13.11.1962, gift med Albert Kaukos}
  \item \jhperson{Johannes}{30.09.1892}{09.05.1894}
  \item \jhperson{Ester Emilia}{21.01.1895}{01.05.1930 gift med Artur Hill USA}
  \item \jhperson{Signe Elisabet}{03.03.1897}{}, gift J Blomqvist, till Amerika
  \item \jhperson{Selma Irene}{23.07.1899}{11.02.1983 ,gift med Erik Gran}
  \item \jhperson{Johannes}{18.10.1901}{22.07.1902}
\end{jhchildren}
År 1880 delades hemmanet i två delar och Samuel blev jordbrukare på ena halvan. Han byggde åt sig och familjen ny bostad och fähus och andra nödvändiga byggnader, ett stenkast från fädernegården.

Samuel \textdied i lungsot 21.06.1902  ---  Sanna \textdied 23.05.1944



\jhhouse{Jungarå}{12:37}{Jungarå}{17}{23, 23a-b}


\jhoccupant{Jungarå}{Dorita}{2013--}
Dorita Jungarå, \textborn 04.03.1937, född Julin på Slangar. Dorita sköter ensam om gården sen hon blev änka 2013.\jhvspace{}


\jhhousepic{Dorita.jpeg}{Dorita och Edgar Jungarå}

\jhoccupant{Jungarå}{Edgar \& Dorita}{1958--\allowbreak 2013}
Edgar, \textborn 26.11.1929 i Jeppo, gifte sig 17.08.1958 med Dorita, \textborn 04.03.1937, född Julin på Slangar i Jeppo.
Edgar gick i Kronoby folkhögskola åren 1950-51 och i Korsholms lantmannaskola 1954. Dorita gick i Kronoby folkhögskola 1954-55. De övertog halva av hans föräldrars hemmansdel 1958 andra halvan 1968, omfattande 50 ha, varav 22 odlad jord.

Under tiden som jordbrukare har både jord- och skogsarealen förstorats. På gården bedrevs mjölkproduktion. Mangården förstorades 1958 och nytt fähus uppfördes 1969. Mjölkproduktionen avslutades på 1980-talet och Edgar och Dorita fortsatte med köttproduktion. År 1993 såldes åkerjorden till kusinen Olav. Efteråt fortsatte de med skogsbruket.

\jhpic{JA Edgar o Dorita.jpg}{Vårbruket förbereds}

De har båda varit aktiva i föreningslivet och politiken. Edgar har bl.a. varit ledamot i kommunstyrelsen i Jeppo i 20 år, 18 år som ledamot i stadsfullmäktige i Nykarleby, 20 år i skattenämnden, 11 år i styrelsen för Nykarleby sjukhem. Dorita har varit aktiv i marthaföreningen, hembyggdsföreningen och Röda korset. m.m. a = fähus, foderlada,garage, b = bastu.


\jhoccupant{Jungarå}{Johannes \& Hellin}{1928--\allowbreak 1958}
Johannes, \textborn 09.02.1902 i Jeppo, gifte sig 1928 med Hellin, \textborn 07.09.1903, född Sandberg från Finskas i Jeppo.
\begin{jhchildren}
  \item \jhperson{\jhbold{Edgar} Johannes}{26.11.1929}{26.01.2013}
  \item \jhperson{Kurt Ingmar}{08.06.1932,  lärare i Pargas}{}
  \item \jhperson{Siv Regina Susann}{21.03.1944, gift Hinkkanen, posttjänsteman}{}
\end{jhchildren}
Johannes gick i Kronoby folkhögskola åren 1920-21 och Korsholms lantmannaskola 1926-27. År 1928, samma år som han gifte sig, övertog makarna halva hemgården efter Johannes föräldrar, samt byggde de också en ny bostad 200 m från barndomshemmet invid älven. År 1929 byggde de en eknomibyggnad i trä och tegel.

Vid sidan om jordbruket hade Johannes ett stort intresse för politik och samhällsfrågor. Redan 1931 var han med och bildade Jeppo skogsägarförening, som ombildades till Jeppo skogsvårdsförening 1936. Han var styrelseledamot i många föreningar och organisationer, idrottsföreningen, ungdomsföreningen, lantmannagillet, andelsmejeriet, handelslaget, Enighetens förvaltningsråd, stiftande medlem av Jeppo ÖSP 1945, i ÖSP:s förbundsstyrelse 23 år, och i dess verkställande utskott, regionförbundet, 35 år i Jeppo kommunfullmäktige, 14 år som dess ordförande. Han ställde upp i riksdagsvalet redan 1948, men blev inte invald då. Vid följande försök 10 år senare blev han invald och var riksdagsman åren 1958-66 samt elektor i presidentvalet 1962 för återval av Kekkonen.

\jhpic{JA o Hellin.jpg}{Riksdagsman JA Jungarå med fru Hellin i bekant miljö}

Under sin tid som riksdagsman arbetade han för jordbrukarnas trygghet och trivsel. Lantbruksbefolkningen bör ha sådana villkor att de kan få en tryggad utkomst. Att åstadkomma eller försvara detta i tider när primärnäringarna inte stod så högt i kurs krävde mycket arbete. Han var med både under den s.k. nattfrosten 1958, när förbindelserna österut bröts och Virolainens regering avgick. År 1961 var nästa kris den s.k. notkrisen. Under sin tid arbetade han för förbättring av vägarna och för att få en centralskola till Jeppo. Projektet godkändes av fullmäktige och bygget kunde startas, taklagsfest hölls 02.10.1965, Hösten 1966 startade skolåret i den nya skolan.

Hellin gick i Breidablick folkhögskola 1921-22, hon  var i ungdomsföreningens styrelse och aktiv i Martharörelsen. Som hustru till en man med många uppdrag fick hon mera ansvar över hemmet och dess skötsel.

Johannes intresse för trav- och hästsporten var viktig i yngre dar, också idrottsintresset, särskilt skidsport, som han själv hållit på med och hade många medaljer från. Vikten av att själv hålla sig i form bidrog säkert till att han var i bra kondition till hög ålder.

Hans intresse att bidra och upplysa med tidningsinlägg kan exemplifieras med utdrag från 1927:

\jhbold{Bonden och hans trånga existensmöjligheter.}
`` I anledning av den kris, som bonden för närvarande har att kämpa med vill undertecknad ha anfört följande. Hur är det beställt här i vårt förut så fattiga land. Här ser det ut som vore landets lösen att importera, ty här leves blott på importerade varor. Oräkneliga mängder spannmål, potatis och industrialster införas. Vart seglar våra pengar genom en dylik hushållning, om icke till utlandet, vilket tydligen bevisas genom den iögonfallande dåliga handelsbalansen.
För den jordbrukande befolkningen i synnerhet visar sig läget mycket allvarsamt, ja nästa hopplöst.''


\jhbold{En jordbrukares vårbetraktelse.} 1927
``Våren har kommit. Icke allenast naturen föryngras, utan även människan vaknar  till nytt liv, när vårens intåg börjar göra sig gällande. Men hava vi icke också skäl att vakna till nytt liv samt med glädje hälsa denna tid av året. Vi veta, att den kulna vintern, de mörka och dystra vinterkvällarna, för en kortare tid nödgas ge vika hör en varmare och skönare tid. Vi ha de senaste veckorna fått bevittna, huru den ena jordbiten efter den andra stuckit fram undan det smältande snötäcket. Vi se huru våra lövträd stå färdiga att iklädda sig sin gröna, vackra sommarskrud, huru gräset frodas och marken börjar skifta färg. Allt har fått liv efter månaders lång vinterdvala. Vi har skäl att jubla och instämma med diktaren Malmström: Gläd dig min vän åt livet, och se ej trumpen ut. Ej blev det åt oss givet till sorg och klagotjut.''

 Hellin \textdied 16.09.1989  ---  Johannes \textdied 16.01.1995



\jhhouse{Vaitserila}{10:84}{Tollikko}{i bokens början}{24, 24a}


\jhhousepic{Alfthan.jpeg}{Jaktstugan vid ån har hyst många högt uppsatta gäster}

\jhoccupant{Alfthan}{Olof}{1965--\allowbreak 2000 }
Olof Alfthan, professor i urologi och intresserad jägare och hunduppfödare, \textborn 29.08.1925.

Det började inte så bra när Gunnar Almberg kom med polis när han 	var på olovlig jakt på andra sidan järnvägen på Jeppos jaktmarker. Efter det tog han kontakt och anhöll om jakträtt och från och med 1968 hade han jaktkort på Jeppo jaktföreningsmarker; sedan tidigare hade han jaktkort på finska föreningars marker söderöver.

Olof köpte tomten 1965 på älvbranten söder om Prästaskangan på gränsen till Alahärmä av Tuure Antila. Han skaffade en gammal sädesbod, som timrades upp på tomten. Senare byggde han en bastu, vilket är en självklarhet för en finländare, och efter några år byggdes det ett hus för övernattande gäster. Gästerna var många, både finska och utländska under årens lopp. Bl.a. kan nämnas bergsråden Rosenlew och Ehrnrooth. Övernattning och mat i jaktstugan var mycket enkelt, men det är den jaktliga gemenskapen som var viktigast vid sådana	här tillfällen.

Det var jakt med stående fågelhundar (hundarna springer sicksack framför jägaren och stannar och markerar riktningen på bytet med tassen), närmast pointer, som var hans stora intresse. Ute på åkrarna är det rapphöns och fasan som jagas. För att få mera vilt att jaga, hade han med sig fasaner på höstarna. På vårarna placerades ruvande fåglar ut. Rapphöns sände han med ``blåtåget'', då hade Sigurd Norrgård och Jan-Erik Nybyggar fått besked om vart de skulle föra paren. De var parade två och två som det skall vara för att det skall bli ungar. Sigurd och Jan-Erik förde ut fåglarna i flera års tid.

Jan-Erik Nybyggar berättar att jaktföreningen var ditbjuden flera gånger och det blev en hel del trivsamma kvällar i stugan. Olof hade väldigt lätt att få kontakt med folk. Med bl.a. Rajalas-``flickorna'' i grannstugan hade han så bra kontakt, att när han fyllde jämna år så var de där med uppvaktning. Han hjälpte också Sigurd Norrgård när han blev sjuk så att han måste opereras. Han var inte den enda Jeppobon som han hjälpte till läkare. Som presidentens livmedicus 	opererade han även Kekkonen för ``gubbsjukan'' som det kallades	tidigare.

Kekkonen kom också och uppvaktade Olof Alfthan när han fyllde 50 	år hösten 1975. Då var det även jakt på en förutbestämd rutt med inte alltför stora diken som hinder. Kekkonen kom med tåg och hade egen vagn vid järnvägsstationen i Jeppo. Dit bjöd han in representanter från jaktföreningarna där de jagade. Övriga föreningar var finska, men när vi satt i vagnen så talade Kekkonen svenska med oss, berättar Nybyggar.

Olof Alfthan var Finlands första professor i urologi, men ändå kom cancern som en obehaglig överraskning för honom, den var för långt gången när den upptäcktes. Jan-Erik minns så väl när han sade, hade jag vetat något tidigare så hade det varit möjligt att åtgärda, men nu är det för sent. Han ordnade en avskedsfest (nån gång juli-aug 1999) med mat och dryck för inbjudna, när det stod klart att inget mera kunde göras. Som avslutning steg han upp för att tacka för allt som 	varit och vad vi varit med om, men när han skulle avsluta brast det för honom.

Numera ägs huset av sonen Tom. a = bastu, b = övernattningslya.

Olof \textborn 29.08.1925 --- \textdied 29.01.2000.



\jhchapter{Samhällsservice i närmiljön}


\jhhouse{Tvättstränderna vid ån}{0:0}{Alla}{1-17}{blå markering}

\jhnooccupant{}

Tvättmaskinerna i den form vi känner dem är inte gamla, men när de kom betydde de en revolution --- speciellt för kvinnorna.

Under århundraden tidigare hade klädtvätt inneburit ett helt annat slit. Visserligen var mängden material, antingen i form av tyg eller ylle, betydligt mindre än i dagens moderna samhälle, men mängden människor i ett hushåll var betydligt större. Likaså var materialet inte lika lätthanterat och alltid mycket smutsigare.

Vintertid kokades kläderna  i en gryta hemmavid, placerades i en träså där en bärstång träddes genom handtagen och bars ner till ån. Väl där sattes den ner bredvid den vak som husbonden i bästa fall bemödat sig om att hugga opp. Ärmarna kavlades upp, kvinnorna steg på knä bredvid vaken med ett torrt plagg under knäna och sköljningen vidtog. De varma klädesplaggen doppades ner i det iskalla vattnet upprepade gånger, vreds ur och doppades på nytt och vreds ur. Gång efter gång. Plagg efter plagg. Kvinnornas armar blev rödare och rödare av växlingen mellan kallt och varmt. Men var man flera, vilket man ofta var, hördes skratt och glam från arbetet på isen. Och slutligen var det klart. Såstången träddes i och nöjda vandrade kvinnorna hemåt.

Men helst utfördes arbetet under sommartid. Det gjorde att tvätten sparades så länge som möjligt under vinterhalvåret. När isen gått och temperaturen stigit var det dags igen. Men nu kunde man inte tvätta som under vintern. Man måste finna en plats där det var lätt att ta sej ner längs slänten, samtidigt som det måste finnas en möjligast plan plats att vistas på. Det allmänna behovet av en sådan plats gjorde att det vid lantmäteriförrättningarna skapades servitut (ett juridiskt begrepp som ger en fastighet rätt att utnyttja en annan fastighet), där lämpliga platser för klädtvätt bestämdes. Inte endast platsen nere vid ån bildade servitutet, utan också vägen till platsen skiftades som ett servitut. Detta möjliggjorde att personer från ett eller flera hemmansnummer kunde utnyttja platsen utan att den ursprungliga markägaren kunde hindra detta.

Bykgrytorna anskaffades ofta gemensamt och placerades på en enkel eldstad av natursten. Ved måste man själv ha med, men närheten till vatten i obegränsade mängder gjorde arbetet lättare och roligare och en speciell doft spred sig från tvättstranden ut i omgivningen. När vintern närmade sig och den sista tvätten var över, stjälptes grytan upp och ner för att inte frysa sönder i väntan på en ny säsong.

Idag har dessa platser förlorat sin betydelse, och vid olika lantmäteriprocesser har deras servituträtt avlägsnats varvid området på nytt infogats i det ursprungliga hemmanet. Platserna kvarstår ändå i folks minnen och de är utsatta med \jhbold{blå} markering på de kartor som finns bifogade:

\begin{enumerate}
  \item Skog: Karta 1, stranden låg vid s.k. Jannes-juutån, nu utfylld strax söder om gård nr 2
  \item Romar: Karta 2, fanns i huvudsak nedanför Romar-Jannes hus, i höjd med Ralf o Svea Romar, nr 10
  \item Silvast / Fors: Karta 6, nedanför Jeppo Krafts kraftstation, den vackra stranden nu utfylld
  \item Grötas: Karta 11, nedanför och strax norrom torkrian, uppströms Kennola, nr 31
  \item Böös (Bösas)
  \item Jungar / Ruotsala: Karta 13, se Perus, nr 211, uppströms kvarnen
  \item Mietala / Gunnar
  \item Tollikko
  \item Jungarå
\end{enumerate}


\jhsection{Väglaget för Gunnar – Lavast väg}

Att ``vatten förenar och land åtskiljer'' är kanske en sanning som gäller i många sammanhang där det varit lättare att färdas över vatten än över mer eller mindre väglöst land. Dock inte i Jeppo. Här har Lappo å/Nykarleby älv, bitvis förgrenad i två älvfåror, mera utgjort ett hinder än en förbindelse. Tvärtrafiken över älven har alltid varit besvärlig och under tider av islossning eller högt vattenflöde t.o.m. omöjlig. De höga och branta stränderna har genom århundradena tvingat befolkningen att söka lösningar på de platser där naturen var mest medgörlig för överfart. Detta har inneburit långa omvägar för att nå sin ägande mark, som kanske fanns på bara 50 meters avstånd tvärs över ån.

Efter att flottbron, som byggts vid Krouvi färjställe på 1890-talet, upphörde i samband med forsrensningen 1910-12 vid Jungarå fors, blev situationen för bönderna från Gunnar, Tollikko och Jungarå extra besvärlig, då de skulle ta sig över till ägorna på Holmen. På 1940-talet byggdes därför en körbro över den östra älvfåran vid Gunnar. Den placerades på stora stenar och bockar på älvens botten.

Bron plockades upp på våren före islossningen, det var ett mödosamt tilltag. Att åter lägga ut den efter vårflödet krävde ett par dagars dagsverken per hemman. Så rullade det på i 20 år med intressenternas insatser, som också innefattade underhållet av den väg som förband bockbron vid Gunnars med bockbron vid Keppo, byggd 1910, och fram till landsvägen på älvens västra sida vid Lavast. Emellertid var det svårt att underhålla vägen och den var ställvis i dåligt skick. Ny lagstiftning öppnade nu en möjlighet att erhålla både kommunalt och statligt bidrag för enskilda vägar, men det förutsatte att ett väglag med juridiskt ansvar skulle finnas.

På självständighetsdagen den 6 dec. 1965 hölls under bonden Leo Elenius ledning ett konstituerande möte i syfte att bilda ett väglag för vägsträckningen Gunnar--Lavast. Femton (15) intressenter deltog och beslöt att bilda ett väglag för nämnda väg. Två dagar senare hölls ett första styrelsesammanträde med Leo Elenius som ordf., Erik L. Back viceordf., Paul Björkqvist kassör, Harry Sandell sekr. och som styrelseledamot Toivo Lavast.

Den 1 juli 1966 beslöts anhålla om det första kommunala bidraget. Och 3 mars följande år ansöktes om statsbidrag för underhållet och Brynolf Kula och Rolf Gunnar fick ansvar som broövervakare vid utläggningen och upptagningen av bockbron. Vid Keppo hade en ny betongbro med tillräcklig bärighet invigts och tagits i bruk året innan.

\jhpic{Gunnar bockbro 2.jpeg}{``Och det hände sig anno 1971 att vårflödet och islossningen satte in på allvar den 27 januari''; JT 28.1.-71. En 45 hk:s traktor i halkan var till begränsad hjälp, manskapet drog lika bra.}

Vid årsmötet 15 mars 1971 gavs styrelsen i uppdrag att undersöka möjligheterna att erhålla stats- och kommunalt bidrag för eventuellt byggande av en fast bro vid Gunnar. Man hade fått nys om att en utrangerad fackverksbro om 30 m kunde fås för ca. 2500 mk och vid ett extra insatt medlemsmöte 1 febr. 1972 beslöts att inlämna en offert om 3500 mk på en 40 m lång del av den bro som i Kvevlax skulle ersättas av en större  betongbro över Kyro älv. Efter en del hopande och roende kring inbegärda offerter för byggande av 2 landfästen av betong, fick slutligen bygdens son, byggmästare Sune Jungerstam, vid ett sammanträde den 29 juni 1973, i uppdrag att utföra arbetet.

Den 3 aug. 1973 anländer bron från Kvevlax med specialtransport och den 28 sept. lyfts den på plats. Den 21 nov. hålls slutsyn på brobygget. Nu var det definitivt slut med bockbrons årliga bekymmer, men den ökande trafiken gjorde att intresset för en rak vägförbindelse mellan bron vid Gunnar och den år 1965 nybyggda bron vid Keppo dök upp i diskussionerna.

Med Erik L. Back som ny styrelseordf. gick väglaget in för att låta uppgöra en plan för denna nya vägsträckning tvärs över Holmen. Den utarbetades av byggmästare Karl-Oskar Norrback (uppväxt på Silvast, nr 19) och var klar till årsmötet 28 dec 1976. Det kom dock att dröja 10 år innan planen förverkligades. I början av 1990-talet anhängiggjordes också ett behov av att höja bron vid Gunnars, en åtgärd som genomfördes 1991 när bron höjdes med 1.5 m. Väglagets ekonomi var nu ansträngd och småningom övertog staden ansvaret för vägsträckan i medlet av 1990-talet i likhet med andra motsvarande vägar. Sedermera har ansvaret för vägsträckan övertagits av statliga bolag, vilket inte upplevs som någon förbättring.



\jhsection{Byggning 113 --- Jungarå Vattenandelslag u.t.}

Tiden efter kriget började moderniseringen och på landsbygden aktualisera saker som vattenförsörjningen. Det hade skötts tidigare med hink och ämbar från en brunn nära bostaden eller fähuset.

Nu blev det modernt (och kanske förmånligt) med stockledningar, så och på Jungarå. 3 januari 1948 hade gårdsägarna ett konstituerande möte för bildande av Jungarå vattenandelslag u.t. På samma möte antogs stadgarna för nämnda andelslag. Till medlemmar i styrelsen valdes: J.A. Jungarå, Torsten Elenius och Ivar Jungell. Suppleanter Eskil Jungarå och Henrik Jungarå. Revisorer Leo Elenius och Hilda Jungarå. Som ordförande utsågs Torsten Elenius, sekr. och disponent J.A. Jungarå.


\jhhousepic{Vattenandelslaget.jpg}{Att borra och skarva stockledning kan inte vara lätt}


Under vårens lopp hölls 7 protokollförda möten och då beslöts allt 	från stockanskaffningen (som Anders Dahlström skötte) till utauktioneringen av grävningen för vattenledningen. Ledningsdiket, 1250 m långt, delades upp i korta avsnitt, som auktionerades ut till den som tog det billigast. Djupet skulle vara 1,5 m och bredden på bottnen minst 35 cm, stenar större än vad två personer orkar lyfta	avlägsnas genom sprängning.

Brunnsgrävningen skulle påbörjas den 10:de mars, alla deltog, den utfördes på 152 timmar, 2 dar och krafter gemensamt. Den 24 maj började Anders Storbjörk från Kronoby borra stockarna, som var 6 m långa med en toppdiameter om minst 6 tum. Pris, 36.132 mark.

I dessa tider när tillstånd måste sökas, var det inte så enkelt att anskaffa material; spikar och cement var på licens, ett parti galvanrör kunde inte beviljas, hydrofonen var ett problem och bristvara. Sigfrid Jungarå som bodde i H:fors lovade förmedla kopparledning till kraftströmmen till ett pris av 350 mk per kilo, det ansågs dyrt och andelslaget inköpte inte mera än behövliga 150 kg (en del smörhandel var inblandad i den affären). Ett problem uppstod när det visade sig att ledningen var för fin, då 	blev det att tvinna ihop flera trådar, till detta gick en hel dag för 7 personer.

Under försommaren pågick grävningsarbetet (med handkraft) intensivt och framskred bra trots ett skyfall som vattenfyllde de färdigt grävda dikena med ras som följd. Henrik och Johannes åkte till sångfesten och lämnade bedrövelsen. I början på augusti var stockarna på plats och igen fyllnadsarbetet på börjades, vilket 	också bortauktionerades. På stockarna fylldes en halvmeter 	stenfritt jordlager, därefter inga större stenar i diket än vad man med skyffeln kunde lyfta.

Arbetet i fähusen och bostäderna var nästa sak. Röråtgång: 11/4 tums = 277 meter , 3/4 tumsrör = 157,5 m , 12 slasktrattar jämte stanklås à 1.813 mk, 12 tvättlavoarer jämte stanklås à 4.745 mk, krökar, t-stycken och kranar. Kostnadsberäkningen gick på 1.107.740 mk.

Bidrag söktes från Lantbruksstyrelsen i Helsingfors för byggande av vattenledningar och slaskledningar på landsbygden åt Jungarå vattenandelslag u.t.,  men det blev avslag bl.a. på grund av att arbetet påbörjats.

Medlemmar: Vid starten 11 medlemmar varav 8 var djurgårdar. Kornas antal var 50 och hästarna 19. Djurantalet var grunden för fördelningen av utgifterna.

Medlemsavgifter: Inträdesavgift 10.000 mk, 4 andelar à 1.000 mk för varje ko och häst. För familj som inte är jordbrukare tecknas 5 andelar. Med dessa summor finansierades ca halva kostnaden, resten var lån och medlemmars inbetalningar.

Redan 1951 tömdes ett fähus när ägarna emigrerade, 1957 hände samma sak, en utträdde, i början på 1960-talet tömdes ytterligare ett, i slutet  på  60-talet var 3 djurgårdar kvar. Korna blev fler, hästarna var borta.

Styrelsemötet 17 januari 1969 beslöt att det är mest ändamålsenligt med tanke på framtiden att gå med i det planerade vattenbolaget Keppo vattenandelslag. Detta efter 21 år med många läckage och arbetstimmar av medlemmarna och i slutändan ganska järnhaltigt vatten, men ändå ett viktigt steg i utvecklingen.




\jhsection{Gunnar Ayrshiretjurförening r.f.}

\jhhousepic{Till tjuren.jpg}{Trilskande ko är tungledd}

Bildades hösten 1931. Gustaf Jungarå sålde ayrshire tjuren Pricent Lex (A6034) för 4.000 mk till 	föreningen. Medlemmar fanns mellan Jungarå och Böös. Betäckningsavgift 40 mk för medlemmar och 60 mk för icke	medlemmar. 	Verksamheten fick statsstöd, 1.300 mk första året.

År 1931 utfördes 10 betäckningar, år 1932 utfördes 43 betäckningar, år 1933 utfördes 76 betäckningar. 	Verksamheten flyttade bort från Jungarå på 1934 till annan skötare. 1935 betäcktes 90 kor, 29 besättningar var med, tjuren började tröttna, men av avkomman hade både mjölkmängden ökat och fetthalten stigit. Nästa tjur som anskaffades var Bellman av Keppo (A7537). 1937 anskaffades Konduktören,  en ungtjur som såldes på grund av ekonomiska skäl under krigsvintern 1940.

J.A. Jungarå var med i styrelsen från starten och skrev årsberättelserna till 1943. Det året fanns 40 medlemmar och antalet betäckta kor var 182 st. Verksamheten fortsatte till början 	på 1960-talet.



\jhsection{Jungarå-Tollikko tröskbolag}

\jhhousepic{Troskbolaget.jpg}{Agri 500 på hjul}

Tröskverket och motor anskaffades våren 1944. Lån för anskaffningen 20.000 mk upptogs i Helsingfors Aktiebank. Betalades bort året därpå. Tröskverket användes av 15 bönder. Ett nytt tröskverk, Agri 500 på hjul, anskaffades 1950 till ett pris av 140.600 mark. Bolaget delades och blev mindre, antalet delägare sjönk till 5 stycken.

Att få igång tröskverksmotorn var en vetenskap, därför anlitades en maskinist, flera höstar var Evert Norrgård maskinist, på 50-talet var bl.a. Edgar Jungarå, Ruben Jungar, Torsten Elenius, Erik Dahlström maskinister.
Jungarå Tröskbolag upphörde i början av 1960-talet när skördetröskorna kom med i bilden.



\jhsection{Jungarå Brobolag}

\jhhousepic[pic:brobolaget]{Brobolaget.jpg}{Ett bolag i vardande}

Vid konstituerande möte 17 mars 1949 beslöt de närvarande att bilda ett bolag benämnt Jungarå Brobolag. Dess ändamål är att bygga en bro på betongkistor med trälock över älven vid Jungarå för att underlätta 		 bebrukningen av de skiften som ligger på holmen. Bron skall byggas på samma plats som hängbron under  forsutgrävningstiden. Bygget skulle påbörjas på sommaren av delägarna Eskil Jungarå 50\%, Torsten Elenius 25\%, Henrik Jungarå 12,5\% och J.A. Jungarå 12,5\%. Brofästena byggdes, men efteråt torkade projektet in.

Bild \ref{pic:brobolaget} visar Artur Jungarå på gångbron över älven vid Jungarå 1936.\jhvspace{7}



\jhsection{Forsrensningen}

\jhhousepic{Ahlblad.jpg}{Ingenjör Waldemar Ahlblad}

Ända till slutet av 1800-talet strömmade forsen fritt i Jungarå, själva forsen börjar strax nedanom älvens förgrening i två fåror, som bildar holmen som är sex kilometer lång. Jungaråforsen, eller Prästasforsen, finns i den östra älvfåran och är 400 m lång. Fallhöjden var före årensningen ca 6m, enligt sägnen hördes forsen dån ända till ``kyton'' på vårarna, en  sträcka på 2,5 km. Hela Lappo älv är 165 km lång och nederbördsområdet är 4619 km$^2$.

När jordbruket fick större betydelse på 1800-talet började bönderna, som odlade de stora låglänta markerna i Alahärmä, Ylihärmä och Lappo, klaga på de årligen återkommande översvämningarna. År 1881 anhöll dessa kommuner hos kejserliga senaten om behovet av rensning av älvens forsar. Efter fem års väntan, den 9 april 1894, fick man från Vasa läns guvernör meddelande, att anhållan blivit godkänd. Efter ytterligare 2 år var ingenjörerna redo att starta arbetet. Denna rensning av forsar pågick i 3 år och större delen av arbetet gjordes uppströms från Jeppo. Kostnaden för rensningen var 18.223 mk och 22 p. Det motsvarade lön för 5217 arbetsdagar.

Rensningen var så obetydlig att översvämningarna 	fortsatte, kanhända värre än förut, eftersom stora myrmarker dikades uppströms. Trots det ålades kronan att år 1897 ersätta kvarnägare för ``förkortad vårflod''. Jungarå kvarn erhöll 4375 mk, Tollikko kvarn 3408 mk; båda dessa summor erhöll ägarna J Nordman och Johan Jungarå. J von Essen, ägare till Kiitola kvarn och schoddyfabriken i Jeppo, erhöll 5625 mk. Ersättningarna var totalt 13708 mk 33p.

T.ex 1902 på sommaren, mitt i höbärgningen, kom ett skyfall som pågick i flera dagar och vattnet steg på de lägsta höängarna till 3 m:s höjd. Missnöjet var stort och dåvarande prästen i Lappo, Wilhelm Malmivaara, som också var invald i riksdagen, lämna in en anhållan 1907 om förnyad älvrensning. Resultatet var dock magert, men han återkom 1908 med en ny anhållan och hade då hjälp av ingenjör Waldemar Ahlblad från överstyrelsen 	för Väg och Vatten. Den 12 oktober 1909 bestämde kejserliga senaten att rensning av Lappo älv skulle påbörjas. För arbetet hade anslagits 2.200.000 mk och redan för innevarande år hade anslagits 200.000 mk, som de beräknade skulle räcka för att få bukt med översvämningarna.

Redan på vintern 1909 gjorde W Ahlblad anbud på virke för rensnings- och 	fördämningsarbetet, bl.a 4 stockar 9 m och 6”,62 stockar 6,5 m långa, 72 st 5 m långa, 50 st 3,5 m långa, 1000 lm 7” stock, 68 tolfts 11/2” x 7-8” 17 fot, 85 tolfts 11/4 x 6-7” 18 fot, 280 tolfts 3 x 7-8” 15 fots plankor (en tolft = 12 stycken).

Karl Reinhold von Willebrand var chef för detta stora projekt, närmast underlydande var fem byggmästare, bland dem med ett större ansvar var Valdemar Ahlblad. Redan 1908 fanns tillsatt en värderingsnämnd, som skulle värdera Jungarå fors. Nämndens ordförande var V Ahlblad. Värdet på vattenkraften jämte såg och kvarn med tillhörande byggnader och dammanläggningar värderades till 44.300 mk.

\jhpic{Forsutgravning1912 Stenholms fors.jpeg}{Rensningen i Stenholms fors fordrade manhaftiga krafter.}

Arbetet påbörjades nedströms vid Jungarå forsen där de största 	hindren fanns. År 1910 var arbetet i full gång, kvarnen sågen och dammanläggningarna inlöstes och revs. Arbetet utfördes till största delen med handkraft och med hjälp av hästar, som drog vagnar längs smalspårig järnväg. Den breda älvfåran fylldes upp 	på båda sidorna med det utgrävda materialet. Nya motordrivna pumpar och ånglok hade beställts från USA och år 1911 var Baldvin Lokomotive Works ånglok installerade. Spårvidden var 60 cm och vikten 6 ton. Tågen hade skilda bromskarlar och växelpojkar och lokförarna kom från östra Finland och de höll lilleputtarna skinade rena i det annars smutsiga arbetet. Också VR s lokförare på stambanan var intresserade av småloken och 	signalerade ``kukkokiekuu'' när de åkte förbi och småtågen svarade.

\jhpic{Forsrensning 4.jpg}{Spårväg och lastvagnar nere på botten av åfåran.}

Alla hus längs älven var försedda med luckor för fönstren för att undvika splitter från sprängningarna. De större stenarna lyftes med vajertaljor, endera till stranden eller på tågvagnarna. De beräknade mängderna sten och grus som  skulle forslas bort var 30.000 m$^3$. Forsrensningen pågick 1910--12 och Keppogrenen påbörjades året därpå.

Våren 1910, den 24.04, kom islossningen igång och isen förde med sig spårvägen och en stor del plankor och träbockar följde med ut till havet. Delar av spårvägen kunde räddas. Vid Keppo hade fabrikör Grönroos utfört dyra fördämnings- och kajbyggnader, vilka till stor del blev förstörda.

Varannan veckas torsdag var ``avlöningsdag'' för de anställda, som var närmare 600 st som mest. Då hölls marknad på kvarntomten med försäljare av alla de slag från när och fjärran, både gårdsfarihandlare och laukku-ryssar. Hilda berättade att en bodde hos dem, han kom med häst och vagn fullastad med varor från Kajana en annan bodde hos ``adesto''. Varorna kunde vara allt från skorpor och bulla, mjölprodukter och kläder till allsorts kramhandel. Också underhållning i form av fiolspel och sång. Hon berättade att ibland kunde de få en 10-pennis slant att köpa snask för. År 1914, när det mesta var klart, hade 156.518 m$^3$ mull, grus och sten grävts ur älvfåran.




Låt oss avsluta med denna berättelse:

\jhsection{Dalabackan}

Dalabackan finns vid Prästas i Jeppo. Dalabackan har antagligen fått namnet av att där fanns många tjärdals- och kolamilor. Där fanns även den omtalade ``dalagränen''. Om den kunde tala kunde den berätta mycket.
Den stod där stor och majestätisk redan när kolamilorna var i bruk. Den stod granne med bombskyddet, som användes vid flygalarm under krigstiden. Den stod så nära kolamilan att ``dalagränen'' fattade eld och skadades, men kolamilsmännen ``dalmästana'' hann släcka elden och granen fortsatte att växa. Sista tjärdalen brändes på Jungarå år 1907.

I början av 1930 talet kom några ungdomar på idén att nu skall vi ha högsta påskbrasan i Jeppo, varvid en av ungdomarna klättrade upp till granens topp med en långhalmskärve och tände eld på kärven där uppe i toppen. Visst var det en brasa som syntes långt vida omkring, toppen dog men granen fortsatte ändock att växa. Den blev så tjock under sin livstid att det behövdes 2 män för att omfamna granen.

På 1940 talet inristades av någon 30 cm höga initialer HS samt en pil på granens stam, granen läkte dock sina sår med kåda. Under sin levnadstid såg granen många generationer barn växa upp, mogna, åldras och dö bort vid Prästas, men granen den levde vidare.

Vilka mängder frön har inte granen spridit ut under sin levnadstid för att ge liv till nya granar, ända tills höststormen var för mäktig med stormbyar som fällde granen. Den som en gång i tiden var så mäktig och stolt föll med ett brak till marken någon gång kring 2010.

Vilken samhällelig utveckling har inte `dalagränen'' fått uppleva under sin livstid. Vilken samhällelig utveckling får inte dagens växande ``dalagrän'' uppleva och hur ser samhället Jeppo ut då när den granens tid är till ända? Vore det inte intressant att kunna glutta in i den framtiden?

Paul Laxén 10.10.2014
