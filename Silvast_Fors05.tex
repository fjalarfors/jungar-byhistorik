



%%%
% [house] Statens Järnvägars stationsområde plus anställda och boende
%
\jhhouse{Statens Järnvägars stationsområde, anställda, boende}{893-871-2-2, 893-871-2-3, 893-408-1-9}{SJ}{8}{71}

Området utgör stomlägenhet 4:5 av Silvast hemman och omfattar 5,21 ha.

\jhnooccupant{}

\jhhousepic{054-05586.jpg}{Stationshuset, en stilig byggnad}

Stationsområdet, dess utformning och behov fick sin början i och med dragningen av järnvägen genom Jeppo i medlet av 1880-talet. Det var ingen självklarhet att det skulle bli så. När frågan om järnvägens förlängning norrut först dök upp, uttalades oro för att den skulle bli en ödemarksbana ``för bonden Paavo och hans släktingar'' långt från de folkrika kusttrakterna och dess städers behov. Det blev som ofta en kompromiss.

Frågan behandlades i mars 1884 vid ett sammanträde mellan representanter från Nykarleby stad, Jeppo, Munsala och Nykarleby landskommun. Planerna på att placera en station i Jeppo föreföll vara accepterat av alla, likaså dess placering på Silvast hemman, men från stadens sida var man inte villig att placera någon station i Kovjoki. Hellre då i Sorvist dit avståndet från staden var 6 1/3 verst. Till Kovjoki var det 10 verst. Kovjokiborna ville inte bygga någon ny väg till Sorvist, inte heller något nytt gästgiveri. Stadsborna blev irriterade och sa att om inte Sorvist blev placeringsort fick också Kovjoki bli utan! Kanske kunde Jeppo station byggas lite längre norrut mot Ytterjeppo, Ojala eller Löv och betjäna både Jeppo och Nykarleby? Planeringens ursprungliga förslag blev ändå rådande och Nykarleby fick istället en smalspårig järnväg. (ÖP)

\jhhousepic{Jeppo station.JPG}{Stationshuset från öster (Wikipedia)}

Arbetet med järnvägen föregicks och genomfördes med en samtidig lantmäteriprocess av stora mått, där statens anspråk på mark möttes av en naturlig ovilja från markägarna att avstå sitt ägande, samtidigt som redan då kunde skönjas en oro för den försämring av rörelse i väst-östlig riktning en järnväg skulle medföra. Naturligtvis drog markägarna det kortare strået och processen rullade vidare av sin egen tyngd med lantmäteriförrättning, värdering och inlösen av de områden staten, d.v.s det Finska Storfurstendömet behövde.

29 okt år 1886, efter en häpnadsväckande kort byggtid, rullade det första tåget genom Jeppo. Enligt Brita Sandlin, som då var 6 år, var det endast loket och en vagn. Den 1 nov skedde den officiella invigningen av hållplatsen. Det skulle för alltid prägla livet i bygden på både ont och gott, med kollektiv övervikt för det senare. Stationshuset uppfördes som hållplats enligt arkitekt Knut Nylanders typritningar för Uleåborgsbanans byggnader. Den upphöjdes till station år 1888 och tillbyggdes med en tvärgående flygel år 1908.

I stark kontrast till dagens tänkande var smått vackert och framför allt funktionellt. Järnvägsstationer skulle placeras med tillräckligt korta avstånd för att på bästa möjliga sätt kunna erbjuda invånarna samfärdsel inom Storfurstendömet. Stambanan Helsingfors-Uleåborg nådde en relativ närhet till kusten just i Jeppo för att därefter dra vidare norrut upp längs kustlinjen.

En lika viktig aspekt när det gäller tätheten av stationer var behovet av att med jämna mellanrum tillfredsställa lokomotivens behov av bränsle och vatten. Alla stationer och stoppställen kunde naturligtvis inte begåvas med en sådan möjlighet, men Jeppo station hade turen att i planeringen utpekas som en sådan plats. Det skulle ge massor av arbetsplatser under nästan 80 år.

Stationen skulle inte enbart fungera som ett stoppställe för av och påstigning av passagerare, eller påfyllning av bränsle (ved och torv) och vatten, även om detta var en väsentlig del av stationens funktioner. Ända fram till efter kriget kastades bränslet upp i lokets tanderbehållare för hand, av stressade män. Men nu skulle en hög lastningsbrygga resa sig för det fasta bränslet. I ca 2 m³:s vagnar på smalspårig räls drogs veden med vajerspel upp längs sluttningen till plattformen där flera vagnar hade rum samtidigt. När tågen kom in för påfyllning av bränsle, stannade loket exakt nedanför den kraftiga fällplåt som nu vevades ner över tandern. Sidoväggen på vedvagnarna sparkades loss och vedklabbarna rasade med ett dån ner i tanderförrådet på loket. Vagn efter vagn tömdes tills lokets förråd var fullt, medan småpojkar från denna spännande plats kunde kika in i lokets innanmäten och beundra rökplymerna som steg mot himlen när tåget drog igång.

Innan ånglokens pensionering brann den höga plattformen ner år 1965 och resterna revs bort. Vattentornet på bild \ref{pic:sjtorn}, karta 6, nr 420, revs senhösten 1962.

Stationen skulle också ge möjlighet för ett spirande näringslivs transporter i lika hög grad som de agrara näringarnas leverans av insatsmedel och färdiga produkter. Därför växte av- och pålastningsplattformar eller bryggor upp på bägge sidor om spåren. Byggnader för mellanlagring och senare behandling av varor och produkter uppfördes på den östra sidan av området, där stickspår kunde ta emot tågsätt eller enskilda vagnar för lastning och lossning. Här byggdes också en massiv lastbrygga av sten. Kort sagt; Jeppo järnvägsstation med sitt huvudspår och 3 stickspår var en livlig plats med många aktiviteter.

En faktor som kraftigt bidrog till detta var den kejserliga senatens beslut att genast överföra postgången till järnvägen då sträckan mellan Östermyra (nuv. Seinäjoki) och Gamlakarleby (nuv. Karleby) var  klar. Orter som begåvades med postexpeditioner var Kauhava, Härmä, Jeppo och Kronoby. Via Jeppo postexpedition hanterades post och postförsändelser till och från Oravais, Munsala, Nykarleby stad, och delvis Nykarleby landskommun. Stationsinspektörerna var ansvariga föreståndare för postexpeditionerna.

Från en blygsam start den 1 dec. 1885 med 230 försändelser och inte en enda ankommen tidning det året, blev posttåget småningom en fläkt från stora världen med brev, paket och tidningar, både inhemska och utländska i mängder, flera hundratusen per år! Folk visste när posttågen skulle anlända och ofta var perrongen full av människor för att möta det.

Att stationen placerades i Silvast, på i huvudsak Silvast hemman R:nr 4, skulle också få långt gående konsekvenser för socknens infrastruktur och bebyggelse. Stenbacken med kyrkan hade fram till dess varit socknens naturliga centrum, men i och med järnvägens dragning öster om älven, förflyttade sej socknens bebyggelse och utbyggda servicefunktioner flyttade efter. Småningom blev Silvast socknens centrum.

Järnvägsstationens dynamik och omsättning på anställda personer märks också i kyrkböckerna, där stationsområdet fått en egen avdelning. Att här redogöra för alla under mer än 100 års verksamhet är kanske varken möjligt eller meningsfullt, men för de centrala personerna på stationsområdet byggdes förutom stationsbyggnaden (nr 71) också bostadshus. Vem som bodde var och när är inte alltid lätt att finna ut, men att ta del av alla de människor som passerat revy genom decennierna, ger en bild av den rörlighet järnvägens folk underkastats under statsjärnvägarnas nu 130 åriga historia genom Jeppo.

Persontrafiken på Jeppo upphörde 1991 och godstrafiken rönte samma öde år 2002. Resenärperrongen revs 2008 och numera fungerar stationsområdet enbart som mötesplats för passerande tåg.


\jhsubsubsection{Stationsinspektörer}

Det var naturligt att stationsinspektörerna bodde i stationens s.k. ``Parkhus'', nr 371, men inte alla. Faktum är att tjänsteinnehavaren kunde välja att bo på annat ställe och därmed ge någon annan möjlighet att ockupera huset. Att själva stationshuset i slutet av 1880-talet var nära att brinna upp hade för en kort tid kunnat förändra detta faktum, men byggnaden kunde räddas. Inspektörerna var i tur och ordning:
\begin{center}
  \begin{longtable}{l l l}
    \hline
    Boehm Eugen Teodor & \textborn 1853 & 1885--\allowbreak 1886 \\
    von Essen Edvard Wilhelm & \textborn 1858 & 1886--\allowbreak 1896 \\
    Lindqvist August Konstantin & \textborn 1862 & 1896--\allowbreak 1902 \\
    Nikander Ernst Josef Wilhelm & \textborn 1865 & 1902--\allowbreak 1907 \\
    Åström Karl Viktor & \textborn 1869 & 1907--\allowbreak 1919 \\
    Nordström Erik Wilhelm & \textborn 1868 & 1919--\allowbreak 1930 \\
    Lindström Emil Vilhelm & \textborn 1879 & 1930--\allowbreak 1945 \\
    Klockars Frans Emil & \textborn 1885 & 1945--\allowbreak 1952 \\
    Stråhlmann Bertel Bernhard & \textborn 1905 & 1953--\allowbreak 1960 \\
    Hästbacka Hilding Johannes & \textborn 1909 & 1960--\allowbreak 1966 \\
    Kronlöv John Rafael & \textborn 1924 & 1967--\allowbreak 1988 \\
    \hline
  \end{longtable}
\end{center}

\jhbold{1)} Boehm Eugen Teodor, \textborn 1853, tillträdde tjänsten som stationsinspektör vid 32 års ålder år 1885. Han kom att stanna på posten ca 1 år, men hans personalier gick förlorade när Jeppo Församlings prästgård brann ner 1887 tillsammans med den unga församlingens kyrkböcker.

\jhbold{2)} von Essen Edvard Wilhelm, \textborn 18.09.1858 i Uleåborg. Hustrun Agnes Amanda Nordgren, \textborn 20.10.1866.
Barnen föddes i Jeppo.
\begin{jhchildren}
  \item \jhperson{Bertel Wilhelm}{21.08.1889}{}
  \item \jhperson{Torsten Harald}{13.01.1892}{}, stupade 03.04.1918 i Tammerfors
  \item \jhperson{Toini Margit}{07.06.1894}{}
\end{jhchildren}
Pigan Ida Fredriksdr. Ylitee, \textborn 05.11.1870 i Ruovesi. kom 1892 från Alahärmä. Pigan Ida Kristina Holm, \textborn 11.01.1872 i Siikais, kom 1896 från Björneborg. Den 26 nov. 1896 flyttade familjen till Storkyro efter 10 år i Jeppo.

\jhbold{3)} Lindqvist August Konstantin, \textborn 12.08.1862 i Nurmijärvi. Hustrun Hanna Schmidt var född i Tavastkyro den 13.02. 1865.
\begin{jhchildren}
  \item \jhperson{Karin Ollilia}{03.06.1893 i Sievi}{12.05.1897}
  \item \jhperson{Ebba Matilda}{10.01.1895 i Sievi}{}
  \item \jhperson{Hanna Ingrid}{11.09.1896}{}
  \item \jhperson{Karin}{15.09.1898}{}
  \item \jhperson{August Alvar}{10.05.1900}{16.05.1900}
\end{jhchildren}
År 1902 flyttade familjen bort från Jeppo efter 6 års tjänstgöring.


\jhpic[pic:lojakt]{Lojakt.jpeg}{Jürgen von Essen och stationsinspektör Nikander vid Jeppo station visar upp de ``sköna lodjuren'' från Purmo.}{0.6}

\jhbold{4)} Nikander Ernst Josef Vilhelm, \textborn  19.03.1865 i Simo. Hustrun Johanna Maria Suuppa, \textborn 10.04.1874 i Birkala.
\begin{jhchildren}
  \item \jhperson{Ernst Oscar Heikki Adolf}{14.11.1893 i St. Petersburg}{}
  \item \jhperson{Paavo}{19.02.1901 i Alahärmä}{}
  \item \jhperson{Sylvi}{15.08.1902 i Jeppo}{}
\end{jhchildren}
Familjen flyttade från Jeppo år 1907 efter 5 års tjänstgöring. År 1904 var Nikander på ``den sista lodjursjakten vid Lostenen'' i Purmo tillsammans med ing. Jürgen von Essen, som 12 år senare blev nedskjuten vid sitt hem på Kiitola av ryska gendarmer. Bytet från lodjursjakten blev 3 ståtliga djur. För bildbeviset stod ortens enda yrkesfotograf, stationskarl Oskar Skantz, se bild \ref{pic:lojakt}.

\jhbold{5)} Åström Karl Viktor, \textborn 07.10.1869 i Gamlakarleby. Hustrun Ellen Anna Löfman, \textborn 06.02.1870 i Gla. Karleby.
\begin{jhchildren}
  \item \jhperson{Ellen Ebba}{22.02.1896  i Gla. Karleby}{}
  \item \jhperson{Viktor Åke}{22.02.1896 ''  ''}{}
  \item \jhperson{Karin}{19.10.1900 i Åbo}{}
  \item \jhperson{Lars Erik}{10.10.1902 i Åbo}{}
\end{jhchildren}
Familjen flyttade från Jeppo till Neder-Torneå 20.03.1919 efter 12 års tjänstgöring.

\jhbold{6)} Nordström Erik Vilhelm, \textborn 02.04.1868 i Malax och hustrun Emilia Sofia Smulter, \textborn 31.01.1869 anlände från Vasa 03.11. 1919.
\begin{jhchildren}
  \item \jhperson{Torsten Edvin}{20.05. 1897 i Vasa}{} Senare chef för Vasa vägdistrikt
  \item \jhperson{Erik Vilhelm}{29.07.1906 i Vasa}{}
\end{jhchildren}
Familjen flyttade från Jeppo till Kannus 24.10.1929 efter 10 tjänsteår.

\jhbold{7)} Lindström Emil Vilhelm, \textborn 22.05.1879 i St.Petersburg, och hustrun Hanna Agneta Joh.dr. Westman, \textborn 24.12.1881 i Sjundeå, anlände 14.05.1930 från Kangasala. Hustrun dog 24.08.1936.
\begin{jhchildren}
  \item \jhperson{Gretel Ingeborg Agneta}{02.09.1912 i Helsingfors}{}
  \item \jhperson{Hedvig Hildegun Anita}{18.03.1919 i Tammerfors}{}
\end{jhchildren}
Den 27.02.1941 anlände från Vasa hushållerskan Amanda Wilhelmina Jokela, \textborn Mäkelä 19.12.1888 i Bjerno(Bjärnå). Familjen flyttade till Nykarleby innan tjänstgöringen avslutades 1945 efter 15 tjänsteår. Emil Lindström var alltså stationsinspektör under de turbulenta krigsåren.


\jhpic{Lok.jpg}{Ånglok på stationen med stolta lokförare och järnvägsarbetare}


\jhbold{8)} Klockars Frans Emil, \textborn 22.12.1885 i Lappfjärd. Hustrun Leventine Viktoria Fredrice Jansson, \textborn 26.09.1888 på Vårdö, Åland. Inga barn finns antecknade. Tjänstgöringen  i Jeppo avslutades 1952 efter 7 tjänsteår.

\jhbold{9)} Stråhlmann Bertel Bernhard, ``Pipi'' kallad, \textborn 24.10.1905 i Helsingfors. Hustru Rose-Marie Adelaide Hagman, \textborn 04.05.1919 i Eno. De anlände 29.07. 1953.
Barn: Adoptivdottern Yrsa, \textborn 23.01.1947 i Lappajärvi. Födelsenamnet var Marja Lisa Virtanen (Far \= Uno). Yrsa, med efternamnet Häggström, har senare skrivit om sin barndom och sina upplevelser i bl.a. Jeppo, i en självbiografisk bok med titeln: ``Man måste ha nio liv''.
Tjänstgöringen i Jeppo avslutades 01.09.1960 efter 7 tjänsteår. Han var sista stationsinspektören som hade stationsbyggnaden som sin bostad.

\jhbold{10)} Hästbacka Hilding Johannes, \textborn 20.12.1909 i Nedervetil. Hustrun Ruth Margareta Sandler, \textborn 14.08.1911 i Munsala. Hilding Hästbacka kom till stationen som bokhållare 02.12.1939 och paret bodde i det hus (nr 371) som i tiden uppförts i stationsparkens norra ända. Han utnämndes till stationsinspektör 1960, men fortsatte att bo i parkhuset, som småningom började kallas Hästbackas hus. Inga barn finns antecknade. Hilding Hästbacka kvarstod som stationsinspektör till slutet av 1966 och fick därmed 6 inspektörsår i Jeppo. Han flyttade sannolikt tillbaka till Nedervetil.

\jhbold{11)} Kronlöf John Rafael, \textborn 08.11.1924 i Oravais. Hustrun Mary Isakas, \textborn 18.03.1929 i Vörå. Rafael och Mary kom från Vörå 21.07.1949 och Rafael med familj inkluderande fadern Johan Kronlöf, \textborn  25.11.1875, kom att bosätta sig i f.d. Fredrik och Ellen Thors hus (Fors nr 19) för att därefter flytta till Helsingfors Aktiebanks lokal. Fadern dog 02.02.1959.

Tjänstgöringen i Jeppo som stationsinspektör startade 01.01.1967. Till en början ökade frakterna ännu en tid, främst från Oravais fabrik,  men småningom ändrade transporterna karaktär till att allt mer ske med lastbilar. Från början av 1960-talet gjorde dieselloken sitt intåg och ångloken trängdes undan. Allt färre stannade och tankade ved och vatten och i och med att en ny och effektiv eldriven vattenpump installerades vid ån 1960, hade också vattentornet (nr 420, se bild under Fors, nr 405), som byggts 1884, tjänat ut. Vattenröret flyttades nu mellan spår 1 och 2 och den 29 sept 1962 säljs tornet för bortrivning för 22.500 mk till Svante Haglund.

\jhpic{Stationens vattentorn och vedbrygga ca 1953.jpg}{Med stationens vattentorn och vedbrygga i fonden tar Curt, Bruno och Gundel Strengell på harven en konstpaus bakom märren Stella och valacken John Deere. Året ca 1953.}

Tre år senare, 1965, antänds vedbryggan av en gnista från ett ånglok och dess södra ända brinner ner. Eftersom tiden höll på att rinna ut för ånglokstrafiken byggdes den inte upp på nytt. Hösten 1980 infaller den största stationsruschen på 35 år på stationen. Orsaken är att samtidigt som fryståg, med fisk från Vladivostok för leverans till  Keppo  och andra pälsfoderkök i nejden anländer till stationen, startar  Oravais Elementhus sin export av elementhus till ryska Karelen, med lastning på Jeppo station. Hundratals tågvagnar skall på kort tid lastas, bilda tågset och växlas ut från stationen. Verksamheten är intensiv för en kort tid, men efteråt fortsätter nedmonteringen av stationens byggnadsbestånd och den 30 maj 1985 säljs de gamla personalbostäderna och jordkällarna för rivning. De blev billiga. Det stora boningshuset gick för 1000 mk, ett uthus för 350 mk, två  andra för 100 mk vardera, likaså bastu/bagarstuga för 100 mk. De två jordkällarna gick på köpet.

Fram till 1984 hade ännu 4 personer turats om med trafikstyrningen och hösten 1983 uppmanades folk att fortsättningsvis använda järnvägen då 3 dagliga persontåg i båda riktningarna ännu stannar! Men 1982 elektrifierades bansträckan upp till Gamlakarleby och bara ett par år senare infördes fjärrdirigering av tågtrafiken, vilket innebar att det tidigare självklara arbetet för stationspersonalen att välja vilket spår inkommande tåg skulle slussas in på försvann. Då huvudspåret och endast ett sidospår var elektrifierade, innebar detta att dieseldrivna växellok måste sköta om de vagnar eller tågsätt som skulle in på sidospåren för lastning eller lossning. Det blev ett osmidigt system som ytterligare bidrog till att minska frakterna. Godstrafiken upphörde i sin helhet 2002 och plattformen revs 2008.

När Rafael Kronlöf den 31 maj 1988 avgick med pension, blev han den sista stationsinspektören i Jeppo. När familjen anlände 1949 var ännu 18 man sysselsatta med att kapa ved åt ångloken, där åtgången ofta översteg 100 m³ per dygn och ett rekord på 160 m³ dygn! Det fanns 6-7 stationskarlar plus 4 anställda på kontoret och ca 10 personer sommartid anställda i den s.k. ``ståpparåickån''. 1984 försvann de 4 sista stationskarlarna. 1988 när Rafael tillsammans med hustrun Mary flyttade till Vörå, var allt detta över. Stationen blev obemannad och persontrafiken upphörde 1991.
\begin{jhchildren}
  \item \jhperson{Lisbeth Maria}{05.07.1951}{}
  \item \jhperson{Stig Erik}{04.04.1954}{}
  \item \jhperson{Tom}{19.02.1957}{}
\end{jhchildren}

\jhbold{12)} Semskar Bo Erik, \textborn 28.02.1932 i Jakobstad. Hustrun Raija Maija Ilona, \textborn 07.10.1935 i Kortesjärvi. Bo Erik flyttade till Jeppo från Nykarleby 27.10.1961, men redan 1959 hade han varit järnvägstjänsteman på stationen. Åren 1961-70 bodde han i stationsbyggnaden. Han gifte sig 1964. Under åren i Jeppo föddes dottern Nina, \textborn 19.01.1966. Efter 7 tjänsteår flyttade familjen till Kållby.

\jhbold{13)} Heimovirta Heino, \textborn 06.08.1932 i Kortesjärvi. Hustrun Maire, \textborn 03.06.1934 i Alahärmä. Han fungerade som stationskarl åren 1971--\allowbreak 1975 , sedan det mesta av verksamheten vid stationen avslutats. Han bodde under sin tid i Jeppo i Stationshuset och bosatte sig efter tjänstgöringen i Köukkäri. Barn: Kari, Harri, Hannu och Pekka.

Numera susar de flesta av tågen snabbt förbi Jeppo station, men då och då finns behov av att styra in tåg på det närmaste sidospåret p.g.a. mötande trafik. Att tajmningen är viktig märks när det tåg som styrts till sidospåret inte ens hinner stanna helt innan huvudspårets tåg susat förbi, allt styrt från Seinäjoki, en stationsort med stark koppling till Jeppo. Mera därom på annan plats.


%%%
% [oldhouse] Parkhuset
%
\jholdhouse{Parkhuset}{893-}{SJ}{8}{371}

\jhhousepic{Parkhuset.jpg}{Parkhuset t.v. med det röda uthuset utanför. Vyn fotograferad från hus nr 70.}
\jhnooccupant{}

När huset uppfördes har inte kunnat fastställas. Vid järnvägens grundande byggdes bostäder till de ledande tjänstmännen, men exakt när har inte gått att utröna. Att detta hus under tiden tjänat som bostad för i huvudsak stationsinspektören är troligt. Bokhållare, järnvägstjänsteman, stins eller stationsinspektör hade i grunden samma utbildning. För tjänstgöring på de mindre stationerna grundade sig utbildningen på lyceum, mellanskola eller motsvarande. För de stora stationernas stationsinspektörer fordrades högre utbildning grundad på studentexamen och för samtliga språkintyg. Att bli utnämnd till stationsinspektör var ett karriärsteg, men grundutbildningen var i sak den samma.


\jhsubsubsection{Tjänstemän som bott i någon av områdets bostäder}

Uppgiften att reda ut vem som bott var i områdets bostäder har varit i det närmaste omöjlig. Här följer ändå en förteckning över de människor som enligt kyrkböckerna förlagts till stationsplatsen. Uppläggningen avviker från bokens huvudsakliga ordning, d.v.s. från senaste till äldsta innehavare; här är det i det närmaste tvärtom.


Järnvägsbokhållare Johan Elis Sirén, \textborn 15.04.1885 i Gla. Karleby och hustrun Marta Maria Montin, \textborn 20.05.1882 i Kuopio anlände från Kaskö 15.11.1917. År 1921 flyttade de till Fors hemmansnr. och året därpå till Holmen. Senare under år 1922 flyttade familjen till Gla. Karleby.
\begin{jhchildren}
  \item \jhperson{Jarl Erik}{07.02.1910 i Nykarleby}{}
  \item \jhperson{Saga Ingegärd}{24.07.1911 i Nykarleby}{}
\end{jhchildren}


Banmästare Matts Eliasson Salo \textborn 07.01.1856 i Ullava kom redan 1887 till Jeppo från Valkeala. Hustrun Maria Stina Andersdr. \textborn 1859 i Rautio.
\begin{jhchildren}
  \item \jhperson{Matts Frithiof}{09.09.1875 i Lohteå}{}
  \item \jhperson{Ivar Valfrid}{01.07.1887 i Jeppo}{}
  \item \jhperson{Arvo Ilmari}{31.03.1890 i Jeppo}{}
\end{jhchildren}
Familjen flyttade till Lappo 21.01.1891.


Banmästare Gustaf Adolf Lundqvist, \textborn 03.12. 1858 i Pedersöre. År 1890 kom han med sin familj till Jeppo omfattande hustrun Maja Kajsa Lars Joh. dr., \textborn 04.03.1853 i Pedersöre, och 4 barn. Ytterligare 2 barn föddes i Jeppo innan Gustaf dog 27.08.1902. Änkan flyttade till Pedersöre med sina barn 10.03.1903.
\begin{jhchildren}
  \item \jhperson{Maria Sofia}{12.02.1880 i Seinäjoki}{}
  \item \jhperson{Hilma Erika}{16.06.1881 i Seinäjoki}{}
  \item \jhperson{Hildur Ingeborg}{01.12.1887 i Kronoby}{}
  \item \jhperson{Ture Adolf Alexander}{08.05.1890 i Kronoby}{}
  \item \jhperson{Joel}{06.06.1893 i Jeppo}{}
  \item \jhperson{Uno Edvin}{05.02.1895 i Jeppo}{}
\end{jhchildren}


Banmästare Karl Alexander Rosendal \textborn 02.08.1867 i Tusby. Vid 30 års ålder anlände han med sin familj från Helsingfors 1897. Hustrun Anna Sofia Silén, \textborn 24.01.1867 i Tusby. Parets fem första barn föddes i Helsingfors, de övriga i Jeppo.
\begin{jhchildren}
  \item \jhperson{Karl Vilhelm}{06.05.1889}{}
  \item \jhperson{Elin Sofia}{11.09.1890}{}
  \item \jhperson{Artur Alexander}{02.09.1892}{}
  \item \jhperson{Signe Dagmar}{09.10.1894}{}
  \item \jhperson{Oscar Valdemar}{26.07.1896}{}
  \item \jhperson{Alfhild Elisabeth}{01.05.1900}{}
  \item \jhperson{John Kristian}{18.03.1901}{}
  \item \jhperson{Ebba Maria}{09.12.1902}{}
  \item \jhperson{Gunnar Ludvig}{07.05.1905}{}
\end{jhchildren}
Den 30.01.1908 flyttade familjen till Nurmijärvi.

Banmästare Johannes Andersson, \textborn 26.09.1866 i Pedersöre, gift med Maria Sofia Månsdr., också hon från Pedersöre, \textborn 20.11.1872. Familjen anlände 1908 från Jakobstad. Johannes blev änkling 23.03.1914 då Maria dog. De fick 7 barn.
\begin{jhchildren}
  \item \jhperson{Johan Adolf}{06.11.1892 i Pedersöre}{}
  \item \jhperson{Frans Artur}{06.09.1895 i Jakobstad}{}
  \item \jhperson{Bruno Valter}{06.10.1897 i Jakobstad}{}
  \item \jhperson{Ragnar Rudolf}{14.01.1905 i Jakobstad}{}
  \item \jhperson{Alva Elise}{14.06.1907 i Gla. Karleby}{}
  \item \jhperson{Else Ingeborg}{27.12.1909  i Jeppo}{}
  \item \jhperson{Anna Linnea}{04.05.1912 i Jeppo}{}
\end{jhchildren}
Av barnen skulle 3 söner följa i faderns fotspår med anställning inom järnvägen; Artur, Bruno och Ragnar.


Banmästare Väinö Eerik Eerikinp. Lehtinen, \textborn 16.01.1902 i Keuru. Gift 1932 med Martta Vilhelmiina Priita Josefinantytär Krigsholm, \textborn 08.07.1908 i Rovaniemi, därifrån familjen anlände 09.01.1934.
\begin{jhchildren}
  \item \jhperson{Reijo Raafael}{15.10.1932 i Rovaniemi}{07.11.1936}
  \item \jhperson{Pirkko Tuulikki}{18.09.1934 i Jeppo}{}
\end{jhchildren}
Familjen flyttade till Pojo 27.02.1937.


Banmästaren Jaakko Juhonp. Niemelä, \textborn 10.07.1899 i Seinäjoki. Gift 12.08.1928 med Anna Heinänen \textborn 08.10.1902 i Lauhaa. Familjen anlände från Seinäjoki 25.06.1937.
\begin{jhchildren}
  \item \jhperson{Eila Inkeri}{21.10.1930 i Norrmark}{}
  \item \jhperson{Kauko Kalervo}{11.10.1932 i Norrmark}{}
\end{jhchildren}
Familjen flyttade till  Lundo 02.12. 1943.


Banmästarna V. Lehtinen och J. Niemelä var bägge starkt engagerade i starten av den finska skolan i Jeppo. De barn som följde med den familj som flyttade mellan Statsjärnvägarnas olika driftspunkter hade ofta finska som modersmål. Att få undervisningen på eget modersmål var viktigt och bidrog samtidigt till att också Jeppos övriga finskspråkiga befolkning kunde frångå den tidigare ``kiertokoulu'' (ambulerande skola) och få en egen skola.


Banmästare Eino Johannes Merasto, \textborn 14.10.1904 i Kiuruvesi. Gift 10.11.1934 med Siiri Matilainen, \textborn 06.03.1901 i Pihtipudas. Familjen anlände 17.07. 1944 från Längelmäki.
\begin{jhchildren}
  \item \jhperson{Erkki Aulis}{13.02.1935 i Ristijärvi}{}
  \item \jhperson{Kaisa Anneli}{20.05.1937   ''}{}
  \item \jhperson{Risto Juhani}{12.02.1940 i Kiuruvesi}{}
\end{jhchildren}
Familjen flyttade till St. Mickel 06.07.1948.


Banmästare/byggmästare Erik Edvin Romar (far:Anders), \textborn 10.07.1896 i Jeppo. Gift 12.11.1916 med Alexandra Åvist, \textborn 22.10.1895 i Purmo. Familjen anlände från Valtimo 29.07.1948.
\begin{jhchildren}
  \item \jhperson{Eva Maria}{01.09.1920 i Nykarleby}{}
  \item \jhperson{Åke Edvin}{02.05.1927    ''}{} Till Borgå 02.01.1949
  \item \jhperson{Sven Erik}{22.07.1929    ''}{}
\end{jhchildren}


Byggmästare Arvo Johannes Sallinen *19.03. 1919 i Siikajoki. Gift 17.07.
1949 med Aili Amanda Koskela *02.08. 1920 i Pudasjärvi. De anlände 03.10. 1960 från Kaskö.
\begin{jhchildren}
  \item \jhperson{Seppo Tapio}{13.08.1950 i Uleåborg}{}
  \item \jhperson{Veli Jukka}{18.09.1951      ''}{}
  \item \jhperson{Paavo Sakari}{15.10.1953    ''}{}
  \item \jhperson{Ritva Anneli}{25.04.1959 i Kaskö}{}
\end{jhchildren}
Familjen flyttade till Kauhava 18.09.1965.


Telegrafisten Thor Harald Forsen, \textborn 04.03.1884 i Gla.Karleby, anlände från Tusby år 1900. Till äkta sammanvigda 16.09.1906 med Anna Vasilijeff, som var grek. katolik. Inga barn antecknade.


Telegrafisten Rufus Arvid Fritjof Bergroth, \textborn 28.08.1881 i Hammarland. Han anlände från Kauhava 1902. Den 11.03.1903 gifte han sig med Saima Sirén, \textborn 20.09.1874 i Nykarleby. Paret flyttade till Tusby 19.04.1906.


Telegrafisten Urilho Benjamin Lehtonen, \textborn 29.05.1888 i Uleåborg, kom 04.07.1914 från Uleåborg. Hustrun Saima Katarina Lahdenperä, \textborn 30.09.1887 i Uleåborg.
\begin{jhchildren}
  \item \jhperson{Kerttu Kaarina}{04.05.1912 i Rovaniemi}{}
  \item \jhperson{Maija Sylvia}{12.08.1913 i Uleåborg}{}
  \item \jhperson{Jorma Wilho}{10.06.1915 i Jeppo}{08.11.1982}
  \item \jhperson{Lauri Antero}{23.01.1917 i Jeppo}{}
\end{jhchildren}
Den 03.11. 1918 flyttade familjen tillbaka till Uleåborg.

\jhpic{Arbetare Vilhelm Wikstrom, Paul Laxen, Eino Broo 1950-t.jpeg}{Dressinen, ett behändigt rälsgående fortskaffningsmedel vid förflyttning längs banvallen. Vilhelm Wikström, Paul Laxén och Eino Broo är ute på uppdrag på 1950-talet.}{0.7}

Telegrafisten Johan Verner Wahlberg, \textborn 01.08.1889 i Jakobstad, kom från Jakobstad 17.01.1918.


Telegrafisten Carl Wilhelm Fredriksson Micklin, \textborn 11.05.1898 i Pulkkila. Gifte sig 25.11.1923 med Elin Astrid Gustavsson, \textborn  06.06.1902 i Jakobstad. De anlände 03.07.1925 från Pedersöre.
Barn: Raoul Vilhelm, \textborn 28.05.1927 i Jakobstad.


Telegrafisten Karl Johan Erik Kuni (Karlsson), \textborn 31.03.1900 i Gamla Karleby. Gift 23.06.1928 med hustrun Sigrid Gunhild Ottosdr. Strandberg, \textborn 28.11.1906 i Närpes. Familjen anlände 08.02.1933 från Närpes.
Barn: Rolf Eric, \textborn 19.10.1928 i Närpes och Greta Helena, \textborn 16.08.1933.


Järnvägsbokhållaren Herman Väinö Hermansson, \textborn *14.04.1888 i Gamla Karleby. Gift 05.04.1915 med Aina Johanna Westerlund från Nykarleby, \textborn 30.11.1890. Familjen anlände 04.03.1921 från Nykarleby.
\begin{jhchildren}
  \item \jhperson{Elna Johanna}{06.03.1916 i Nykarleby}{}
  \item \jhperson{Alice Maria}{14.07.1919       ''}{}
  \item \jhperson{Astrid Charlotta}{29.11.1920  ''}{}
\end{jhchildren}


Järnvägsbokhållaren Herman Thorvald Eklund, \textborn 10.06.1895 i Helsingfors. Gift 20.11.1921 med Avena Emesentia Salminen, \textborn 13.08.1897 i Oulais. Familjen anlände 29.12.1927 från Gamla Karleby.
\begin{jhchildren}
  \item \jhperson{Gunvor Avena}{31.12.1922 i Uleåborg}{}
  \item \jhperson{Avena Kristina}{20.06.1925 i Jakobstad}{}
  \item \jhperson{Tordis Ingeborg}{13.08.1926 i Gamla Karleby}{}
\end{jhchildren}


Järnvägsbokhållare Paul Edvard Lindqvist, \textborn 07.07.1895 i Helsingfors. Gift 25.11.1921 med Lempi Ananiasdr. Lindell, \textborn 20.03.1898 i Multiala. Makarna anlände från Ruskeala 05.06.1937. Samma dag som Vinterkriget tog slut, 13.03. 1940, dog Paul Edvard. Änkan Lempi flyttade 20.05 1941 till Gamla Karleby.


Järnvägsbokhållare Sigurd Reinhold Leonard Lindqvist, \textborn 03.08.1895 i Snappertuna. Gift 04.12.1927 med Lempi Julia Engman (f.d. Mäkelä), \textborn 1907 i Helsingfors. De flyttade till Hangö 19.03.1943.


Järnvägsfunktionär Toini Emilsdr. Rautavaara, \textborn 23.11.1903 i Kaukalo. Kom från Alahärmä 20.10.1936 och flyttade till Nurmo 17.03.1938.


Järnvägstjänstemannen Paul Runar Holmlund, \textborn 09.12.1923 i Seinäjoki. Gift med Anna Blondine Berg, \textborn 08.02.1925 i Solf. De anlände från Vasa 16.07.1949.
\begin{jhchildren}
  \item \jhperson{Johan Erik}{18.10.1949}{}
  \item \jhperson{Ann Kristine}{14.07.1952}{}
\end{jhchildren}
Familjen flyttade till Kaskö 09.01.1960.


Postfunktionär Kerttu Albertina Emilsdr. Rautavaara, \textborn 01.05.1910 i Kaukalo. Kom från Alahärmä 20.10.1936 och flyttade till Kokkola 03.11.1939.


Folkskollärarinnan Anni Linnea Joh.dr. Andersson, \textborn 04.05.1912 i Jeppo (Banmästare Johannes Anderssons dotter), anlände 05.11.1935 från Pedersöre och flyttade ett knappt år senare till Jakobstad, 20.08.1936.


Servererskan Inger Martina Teräs, \textborn 23.05. 1922 i Munsala. Hon kom till Jeppo 03.03.1944 från Helsingfors. Den 25.03 samma år gifte hon sig med elektromontören Anders Werner Ekström, \textborn 30.06.1921 i Jeppo.
\begin{jhchildren}
  \item \jhperson{Bror Håkan}{04.12.1944 i Jakobstad}{}
  \item \jhperson{Gunvi Margaretha}{19.07.1948 i Jeppo}{}
\end{jhchildren}
Familjen bosatte sig senare på Fors skattehemman och flyttade 08.07.1949 till Korsholm.


Postfunktionär Elna Sofia Sandler, \textborn 26.11.1920 i Munsala, kom därifrån 09.02.1948. Hon flyttade vidare till Vasa 09.01.1951.

Stationskarlen Jakob Johansson Kling, \textborn 14.02.1859 i Storkyro, anlände till Jeppo 1888 från Ylistaro. Hustrun Susanna Johansdr., \textborn 19.06.1856 i Storkyro.
\begin{jhchildren}
  \item \jhperson{Fanny Justina}{15.07.1888 i Jeppo}{23.06.1889}
  \item \jhperson{Yrjö Wilhelm}{13.04.1890     ''}{}
\end{jhchildren}
Jakob Kling \textdied 15.07.1891. Änkan med sonen Yrjö flyttade 05.08.1891till Storkyro.


Stationskarlen Erik Johan Simonsson Back, \textborn 20.11.1867 i Jeppo. Hustrun Maria Sofia Isaksdr., \textborn 21.03.1867 i Munsala.
Barn: Anna Edith, \textborn 06.08.1893 i Jeppo
Familjen flyttade 10.11.1893 till Hiitala. Anteckn.: Godkänd lott nr 6.


Stationskarlen Matts Mattsson Helin, \textborn 07.03.1863 i Ylistaro, anlände 1891 från Storkyro. Hustrun Lisa Mattsdr., \textborn 02.05.1861 i Storkyro.
\begin{jhchildren}
  \item \jhperson{Jakob Arthur}{19.03.1886 i Storkyro}{}
  \item \jhperson{Ryydi Adolf}{09.11.1892      ''}{}
\end{jhchildren}
Familjen flyttade till Kronoby 10.12.1893. Godkänd lott nr 21.


Stationskarlen Gustaf Isaksson Lehtinen, \textborn 03.08.1870 i Täysä, anlände från Uleåborg 1894. Hustrun Katarina Jakobsdr., \textborn 23.01.1859.
\begin{jhchildren}
  \item \jhperson{Ellen Johanna}{27.09.1893 i Uleåborg}{}
  \item \jhperson{Wäinö Arnold}{23.09.1894 i Jeppo}{}
\end{jhchildren}


Stationskarlen Anders Simonsson Jokimäki, \textborn 06.03.1876 i Haapavesi, kom 1895 från Uleåborg. Han flyttade till Sievi 21.01.1897.


Stationskarlen Sameli Mattsson Hakala, \textborn 25.07.1871 i Ilmajoki, anlände till Jeppo från Seinäjoki 1898. Hustrun Kajsa Serafia Jaak dr., \textborn 02.03.1872 i Seinäjoki. Paret flyttade till Ylivieska 30.03.1899.


Stationskarlen Otto Svahn, \textborn 01.07.1862 i Kelviå. Anlände till Jeppo från Lappo 1899. Hustrun Vilhelmina Aronsdr., \textborn 22.05.1860 i Etseri.
\begin{jhchildren}
  \item \jhperson{Alma Vilhelmina}{26.09.1885 i Kelviå}{}
  \item \jhperson{Helmi Maria}{24.05.1887  i Lappo}{}
  \item \jhperson{Ellen Sofia}{19.02.1889       ''}{}
  \item \jhperson{Jalo Ilmari}{20.03.1891       ''}{}
  \item \jhperson{Gustaf Emil}{23.12.1892       ''}{}
  \item \jhperson{Hulda Siviä}{11.09.1895        ''}{}
  \item \jhperson{Otto Armas}{04.04.1897       ''}{}
  \item \jhperson{Marta Elfrida}{07.09.1899      ''}{}
  \item \jhperson{Aaron Abraham}{02.09.1901 i Jeppo}{}
  \item \jhperson{Arvo Albert}{28.02.1904       ''}{}
  \item \jhperson{Vilho Adolf}{24.12.1905       ''}{}
\end{jhchildren}
Familjen flyttade 17.09.1929 till Kannus efter 30 år i Jeppo. Otto var en av de första personerna som hade ansvaret för vattenpumpstationen vid ån.


Stationskarlen Jakob Elias Jakobsson Ojaniemi, \textborn 15.08.1887 i Vihanti. Anlände 1905 och flyttade vidare till Viborg 26.02.1908.


Stationskarlen Heikki Laurinpoika Seppänen, \textborn 09.07.1883 i Utajärvi, kom från Uleåborg. Han flyttade till Kemi 27.11.1907.


Stationskarlen Anders Andersson Mattila, \textborn 22.09.1877 i Alahärmä. Gifte sig med Kajsa Sofia Thomasdr., \textborn 02.12.1879 i Seinäjoki.
\begin{jhchildren}
  \item \jhperson{Mauriz}{10.07.1909  i Jeppo}{}
  \item \jhperson{Mirja}{30.04.1912       ''}{}
\end{jhchildren}
Familjen flyttade 1914.


Stationskarlen Oscar Emil Skantz, \textborn 03.03.1876 i St. Petersburg. Anlände till Jeppo 23.12.1909 från Uleåborg. Hans hustru Urda Therese Lönnqvist, \textborn 06.11.1885 i Lovisa, anlände därifrån 27.04.1910. Oscar var en av de personer som tidigt intresserade sig för fotografering. Han står längst till höger på ett foto som han själv tog hos Johan och Sanna Kajsa Backlund på Romar (se Romar hemman nr 312).
\begin{jhchildren}
  \item \jhperson{Rose Maj}{03.12.1910 i Jeppo}{}
  \item \jhperson{Erling Arnold}{04.07.1912      ''}{}
  \item \jhperson{Bernhardine Therese}{26.01.1915    ''}{}
\end{jhchildren}
Familjen flyttade 23.02. 1917 till Nedertorneå. Oscar \textdied 02.12.1938 ---  Urda \textdied 19.12.1955


Stationskarlen (Banvakten) Emil Joh.sson. Helsius, \textborn 07.06.1894 i Alahärmä, därifrån han anlände 29.11.1917. Hustrun Vendla Maria Granlund, \textborn 29.07.1890 i Jeppo.
Emil \textdied 10.04.1924  ---  Vendla \textdied 16.10.1931.

\jhpic{Transport fr Jeppo station till Keppo 1930-t Viljam Stenbacka.JPG}{Viljam Stenbacka med ett arbetslag för transport från stationen till Keppo gård på 1930-talet.}{0.8}

Stationskarlen Frans Artur Andersson, \textborn 06.09.1895 i Jakobstad (fader banmästare Johannes Andersson). Hustrun Maria Alice Mattsdr. Sandvik, \textborn 28.12.1897 i Terjärv. De anlände till Jeppo 1920 från Terjärv.
\begin{jhchildren}
  \item \jhperson{Ragni Alice}{18.10.1920 i Jeppo}{}
  \item \jhperson{Nils Ola}{04.07.1923 i Jakobstad}{}
\end{jhchildren}
Arthur \textdied 05.06.1983 i Jeppo  ---  Maria \textdied 19.05.1998 i en ålder av 100 år.


Stationskarlen Oscar William Ottosson Svanbäck, \textborn 10.03.1889 i Jeppo. Gift med Ida Johanna Sannasdr., \textborn 12.05.1885 i Jeppo.
\begin{jhchildren}
  \item \jhperson{Leo William}{19.07.1909 i Jeppo}{}
  \item \jhperson{Oscar Edvin}{11.05.1912       ''}{}
  \item \jhperson{Birger Waldemar}{13.07.1913       ''}{}
  \item \jhperson{Signe Johanna}{24.10.1915       ''}{}
  \item \jhperson{Svea Ingeborg}{27.05.1917       ''}{}
  \item \jhperson{Erik Runar}{12.05.1919       ''}{}
  \item \jhperson{Saga Katarina}{15.12.1920       ''}{}
\end{jhchildren}
Familjen flyttade till Pedersöre 16.02.1921.


Järnvägsarbetaren Väinö Anders Valfrid Sanfridsson Dahlbacka, \textborn 14.11.1902 i Gla. Karleby, gift med Martha Elfrida Ottosson Svanh (se Otto Svahn), \textborn 07.09.1899 i Lappo. De anlände från Gla. Karleby 25.02.1922.
\begin{jhchildren}
  \item \jhperson{Majlis}{21.03.1922 i  Jeppo}{}
  \item \jhperson{Olof Valfrid}{26.01.1924 i Gla. Karleby}{}
  \item \jhperson{Holger Olof}{*25.07.1925          ''}{}
\end{jhchildren}
Maken Väinö avled 28.08.1927 och änkan med barnen flyttade 07.06.1930 till Gla. Karleby.


Stationskarlen Edvin Johannes Stenbacka (Stenvall), \textborn 07.06.1891 i Jeppo. Fadern var banvakten Isak Stenbacka, \textborn 18.01.1852. Gift 09.08.1914 med Maria Angelia Mattsdr., \textborn 29.10.1891 i Munsala, därifrån hon flyttade 08.07.1914, strax innan bröllopet.
\begin{jhchildren}
  \item \jhperson{Gerda Maria}{13.02.1915 i Jeppo}{}
  \item \jhperson{Anna Teresia}{14.12.1916 i Jakobstad}{}
  \item \jhperson{Erik Edvin}{27.07.1918 i Jeppo}{03.03.1919}
  \item \jhperson{Erik Edvin}{17.09.1921 i Jakobstad}{}
\end{jhchildren}
Familjen flyttade till Kronoby 29.12.1923.


Stationskarlen Johan Sivert And.son Nygrann (Nygren), \textborn 03.06.1892 i Gla. Karleby socken därifrån han anlände 30.09.1919. Gifte sig med Edit Cecilia Simonsson Björk, \textborn 03.10.1895 i Kronoby.
Barn: Göran Wilhelm, \textborn 22.05.1922 i Jeppo.
Familjen flyttade till Pedersöre 27.12.1923.


Stationskarlen Paul August Ståhl, \textborn 03.12.1900 i Nykarleby. Gifte sig 06.08.1922 med Hilda Sofia Stenbacka, \textborn 12.05.1898 i Jeppo. Han kom 08.11.1924, tre dagar efter sin hustrus död 05.11.1924. Han vigdes 24.01.1926 vid sin 2:a hustru Agnes Maria Elielsdr. Sandberg, \textborn 04.11.1905 i Jeppo.
\begin{jhchildren}
  \item \jhperson{Paul Olof}{08.05.1923 i Pedersöre}{14.10.1925}
  \item \jhperson{Magnhild Birgitta}{30.12.1926 i Jakobstad}{}
  \item \jhperson{Maj Britt Regina}{21.09.1928          ''}{}
  \item \jhperson{Erna Veneta}{23.02.1932 i Jeppo}{}
  \item \jhperson{Stig Fjalar}{21.10.1937        ''}{}
\end{jhchildren}


Stationskarlen Otto Armas Ottos. Svahn, \textborn 04.04.1897 i Lappo. Gift 05.09.1926 med Emma Selma Joh.dr Karhinen, \textborn 24.06.1897 i Kyyjärvi. De anlände från Gla.Karleby 20.01.1927.


Stationskarlen Juho Konstantin Savolainen, \textborn 07.07.1897 i Töysä. Gift 20.05. 1922 med Elna Eliina Lempinen, \textborn 29.06.1898 i Ähtäri. Familjen anlände från Alahärmä 04.12.1935 och flyttade småningom till Holmen på Fors skattehemman som rågranne till Bönehuset. Hustrun \textdied 02.12.1955.

Barn: Kaija Tuulikki, \textborn 14.05.1927 i Vasa. Hon gifte sig 05.06.1949 med Karjalainen och flyttade till Puolanka 30.12.1949. Hon blev änka 26.05.1951 och återkom 17.12.1952 därifrån med dottern Tuula Elina Karjalainen, \textborn 15.02.1950 i Kajana. Den 16.01.1953 gifte hon sig med folkskolläraren Valto Virtanen, \textborn 31.08.1924 i Åbo. De flyttade till Polvijärvi 30.11.1953.


Stationsarbetaren Lennart Vikström, \textborn 02.12.1924 i banvaktsstugan vid Gunnar, gifte sig med Elsa Englund, \textborn 25.05.1927 i Bennäs. Familjen bodde i bostad nr 3 i det röda hyreshusets norra del. Lennart var stationstjänsteman och hade länge ansvaret för de försändelser som kom till och från den godsterminal som fanns vid stationens lastbrygga.
\begin{jhchildren}
  \item \jhperson{Carita Viveca}{02.07.1951}{}, gift med Ingmar Brännäs, P:öre
  \item \jhperson{Yvonne Kristina}{26.12.1957}{}
\end{jhchildren}


Stationskarlen Thure Bernhard Ekholm, \textborn 09.05.1899 i Åbo. Gift 08.07.1921 med Aina Ekman, \textborn 24.02.1898 i St. Petersburg. Familjen kom från Vasa 03.03.1939. I en stor barnaskara föddes de flesta i Vasa, endast de två yngsta i Jeppo.
\begin{jhchildren}
  \item \jhperson{Eino Bernhard Ferdinand}{12.06.1922}{}
  \item \jhperson{Gerhard Rolf Rafael}{29.12.1923}{}
  \item \jhperson{Ruben Clarence Ronald}{02.04.1926}{}
  \item \jhperson{Rosa Jeanne Margareta}{26.02.1929}{}
  \item \jhperson{Alfons Klaus Sellfrid}{17.08.1931}{}
  \item \jhperson{Sylve Dagmar Ingeborg}{14.12.1932}{}
  \item \jhperson{Rainer Osvald Alarik}{02.10.1934}{}
  \item \jhperson{Anita Savina Violet}{08.12.1935}{}
  \item \jhperson{Lars Viktor Konrad}{27.10.1937}{}
  \item \jhperson{Erik Jan Fredrik}{28.08.1939}{}
  \item \jhperson{Sven Jarl Arnold}{18.01.1941}{}
\end{jhchildren}

Banarbetaren Toivo Kalliosaari, \textborn 01.11.1930, gift med Elli Eliisa Perämäki, \textborn 12.07. 1930 i Alahärmä, bodde i ``Parkhuset'' fram till 1959, när han tillsammans med brodern Tarmo köpte det hus som ägdes av Elmer och Leander Sandberg och Edvin Jungell och finns beskrivet under Romar nr 28.

\jhhousepic{Stationsuthus.png}{Uthus i parken}

Tågklarerare Ulf Nybäck, \textborn 23.05.1942 i Porkkala, Degerby, uppvuxen i Nykarleby, där hans far var stins sedan 1949. Gift 1968 med Gunilla Åkerholm från Katternö, \textborn 02.07.1947. Han bodde i ``Parkhuset'' vid stationen under sin tjänstgöring. Han fungerade som tågklarerare i Jeppo tillsammans med Rafael Kronlöf, Hilding Nygård och Bo Erik Semskar åren 1967-68. Året innan hans tid i Jeppo noterades nytt trafikrekord med 54 tåg på ett dygn den 22 mars 1966!

Under sin tid vid järnvägen påminner Ulf sej när det första dieseldrivna tåget anlände år 1961. Trafiken fortsatte jämsides med ångloken fram till 1968 då de sista ångloksdrivna tågen från Rautaruukki i Brahestad, på väg till Helsingfors, stannade i Jeppo för vattenpåfyllning. Detta gällde den reguljära trafiken. Ånglok stannade i Jeppo ännu de första åren på 1970-talet.

Ulf flyttade med fru Gunilla till Nykarleby 1969 i samband med minskningen av personal inom VR. Innan bodde de en kort tid i den nya brandstationens bostadslokal invid kommungården i Jeppo.


Banförman Paul Mack Laxén, \textborn 03.11.1931 i Jeppo. Gifte sig 08.09.1957 med Eevi Anneli Huhtala, \textborn 23.12.1937 i Alahärmä.  Familjen kom 26.03.1960 från Nykarleby lk.
Barn: Cay Paul Ingemar, \textborn 13.08.1958 i Nyk.lk.

Paul och Anneli byggde sig ett nytt hus på det bostadsplanerade området i Jeppo år 1973  och använde då som stomme det timmer som köpts av SJ i samband med att den banvaktsstuga som fanns vid Tapelbacken revs. Den stugan hade familjen varit stationerad i åren 1960 –1963 då Paul fungerat som den sista banvakten i distriktet medan ångloken ännu körde.

Leif-ole Nygård, \textborn 1951, gifte sig med Rita Moisio, \textborn 1952 i Surahammar, Sverige, bodde i huset under 1972--1973. Son vid tidpunkten i Parkhuset: Tyrone, \textborn 1971 i Ytterjeppo.

Iiskola Jorma, \textborn 24.05 i Jaala och Brita, \textborn 19.10.1939 i Karis, hade under en tid ``Parkhuset'' 1 som sin bostad år 1974. De hade 1970 rest ut som missionärer till Atemo missionsstation i Kenya, där Jorma till en början blev biträdande chef för ungdomsarbetet inom LCK, den kenyanska lutherska kyrkan och Brita fungerade som lärare. Jorma var missionär i Kenya i 12 år och Brita 9 år.




\jhpic{Stationskarta Jeppo.png}{SJ:s situationsplan över stationsområdet, magasinen och t.o.m. placeringen av Jungar folkskola längst t.h.}




\jhpic{Vy stationsplan 1955.JPG}{Vy över stationsområdet 1955}




%%%
% [house] Ab Jepograin Oy
%
\jhhouse{Ab Jepograin Oy}{893-871-2-3}{Stationen}{8}{118, 118a-b}


%%%
% [occupant] Bolag
%
\jhoccupant{Bolag}{\jhname[Lindborg]{Bolag, Lindborg} \& \jhname[Strengell]{Bolag, Strengell}}{2016--}
Fastigheten inköptes hösten 2016 av de lokala odlarna Matts Lindborg med sonen Erik och Glenn Strengell med sonen Jakob som ägare, vilka under ett gemensamt bolag, ``Jepograin'' fortsätter användningen av fastigheten för spannmålshantering och som oljeslageri.


\jhhousepic{Jepograin.jpg}{Jepograin har etablerat sig, nr 118}

%%%
% [occupant] Jeppo Food
%
\jhoccupant{Jeppo Food}{\jhname[Oy/Ab]{Jeppo Food, Oy/Ab}}{1989--\allowbreak 2016}
Oy Jeppo Food Ab, hade bildats 1988 som ett odlarägt aktiebolag med handlanden John Nyman som drivande kraft (John Nyman var för övrigt också den som tog initiativet och drev på bildandet av Jeppo Potatis med början under 1975).

Stiftande medlemmar i bolaget var: John Nyman, Christer Fors, Börje Lindborg, Stig-Johan Jungar, Greger Bränn, Leif-Ole Romar, Johan Stenfors, Erik Elenius, Kurt Stenvall och Stig Jakobsson. Stiftelseurkunden undertecknades 11 nov. 1988. Den första styrelsen bestod av John Nyman ordf., Christer Fors viceordf., Johan Stenfors, Leif-Ole Romar och Magnus Finnila.

Avsikten var bl.a. att stabilisera och stimulera handeln med spannmål för i huvudsak Nykarlebynejden efter att den turbulenta växlingen av huvudaktörer på spannmålsmarknaden börjat upplevas hämmande för odlarna att få sin spannmål såld och tillgången på småfrön säkrad. På försommaren 1989 erbjöds plötsligt möjligheten att av Waasko handelslag köpa den spannmålsanläggning (nr 118) och det magasin, som byggts 1947 resp. 1956 av Jeppo-Oravais handelslag vid stationen (se närmare nr 57).

\jhhousepic{146-05599.jpg}{Spannmålsmagasinet från spårsidan}

Genom köpet fick Jeppo Food en flygande start och sökandet av en tomt i närheten av Mirka avbröts. Detta år och följande års skörd av spannmål var ovanligt stor och inköpen av de stora mängderna krävde i initialskedet stora krediter. Samtidigt gick den finländska  ekonomin in i den nu välkända ``laman'' 1990-93 när Bnp sjönk med 13 \%.  Nils Sandvik som följt med köpet som s.k.``gammal arbetstagare'', fick Tor Holm som ny Vd. Hans tid i företaget blev rätt kort och 1992 hade han ersatts av Yrjö Långnabba samtidigt som Nils Sandviks tjänst avslutats. ``Laman'' innebar också ekonomiska problem för det unga företaget, men med hjälp av den nyvalde Vd:n Yrjö Långnabbas enträgna arbete och planering lyckades företaget rida ut stormen och situationen stabiliserades. Småningom fattades beslut om att vid sidan om spannmålshandel, torkning, utsädessortering, småfröförsäljning och bränsleförsäljning utöka verksamheten med att pressa rapsolja för livsmedelsindustrin och detaljhandeln. Utrymmen färdigställdes, utrustning för oljepressning anskaffades och produktionen startade. Initiala problem uppstod, men med hjälp av kompetent rådgivning av bl.a. tekn. Harry Björkqvist från Korsholm, uppnåddes en högkvalitativ produkt, som vann gillande hos konsumenterna under produktnamnet ``Livette''.

Att efter Yrjö Långnabbas pensionering år 2012 finna en ny och kompetent Vd för företaget visade sej överraskande svårt och efter flera misslyckade försök på den punkten började företaget tappa fart och svårigheter uppstå. Omsättningen gick ner med 60 \% under åren 2014--16 och verksamheten avslutades sommaren 2016, varefter fastigheten såldes.


%%%
% [occupant] Jeppo-Oravais Hlg
%
\jhoccupant{Jeppo-Oravais Hlg}{\jhname[Andelsringen-Waasko]{Jeppo-Oravais Hlg, Andelsringen-Waasko}}{1947--\allowbreak 1989}
Handeln med spannmål, gödsel och andra förnödenheter var som de flesta andra produkter reglementerade under krigsåren. Staten var av nödvändighet och tvång intresserad av att ha kontroll över vilka mängder spannmål som fanns att tillgå. Det hade en historia tillbaka till 1928 när Statens spannmålsförråd grundades, i första hand för att trygga spannmålsförsörjningen för statens anläggningar och försvarsmakten. Efter krigets slut 1944 hade lärdomen blivit, att på statens ansvar låg att se till att tillräckliga mängder lagrades för att klara svängningar i tillgången på spannmål förorsakade av vädret eller internationella konflikter.

Av denna anledning sökte Statens spannmålsförråd samarbetspartner för att inte självt bli tvunget att bygga lagringskapacitet för all den mängd som lagen krävde; 400 milj kg brödsäd, 200 milj kg fodersäd, 100 milj kg utsädesspannmål och 1,2 milj kg vallväxtfrö. Samtidigt kunde man med regionala samarbetspartner geografiskt sprida lagringsmängderna. Handelslaget gick därför in för att 1947 bygga ett spannmålsmagasin som förutom för ett eget behov samtidigt fungerade som ombud för Statens spannmålsförråd.

Det gamla magasinet längst söderut på stationens lastningsområde (se nr 419, revs på 1960-talet) användes då fortfarande som förnödenhetslager. År 1956 beslöts att bygga ut det nya spannmålsmagasinet norrut med en tillbyggnad i två våningar, som gjorde det möjligt att koncentrera verksamheten. Härifrån levererades kalk, konstgödsel, utsäde och foderspannmål, vetekli, kraftfoder, cement, byggnadskalk och tegel. Hit hämtades foderspannmål, brödsäd, potatis och utsäde för försäljning. Odlarnas eget utsäde sorterades mest under vårvintern. Vartefter tiden gick utökades sortimentet. Här var Edvin Jungell ansvarig person i flera årtionden, se särmare Skog, nr 5.


%%%
% [occupant] F:ma
%
\jhoccupant{F:ma}{\jhname[Backlund]{F:ma, Backlund}}{1938--}
Det tidigare s.k. \jhbold{Backlundska magasinet}, nr 118b, som uppförts före kriget av firma Backlund från Vasa, hade nu köpts. Det stora magasinet användes av f:ma Backlund främst för handel med hö och till en del halm och potatis. Helge Back med hjälp av brodern Anders skötte kommersen, som var rätt livlig under och strax efter krigen, när de statliga utskrivningarna av foder skulle levereras och skickas iväg per järnväg. Produkterna levererades till magasinet från närområdet. Fanns de längre bort, skötte Anders om transporten med firmans lastbil till stationen.

\jhhousepic{JepograinB.jpg}{Magasin med plåtfasad, nr 118b + 118a}
Magasinet utbyggdes i mitten av 1980-talet med en ny plåthall för lagring av bl.a storsäckar, som nu vunnit insteg på marknaden. Likaså för varor som lagrades på lastpallar och krävde ett jämnt golv. Nu uppfördes också en större plåtsilo för spannmål på yttre sidan om huvudmagasinets södra ända. Den drivande kraften var i detta skede Vd:n Bengt Sandvik.

I de ursprungliga magasinbyggnaderna arbetade till en början Ivar Vestin och Edvin Jungell. Senare Erik Norrgård, Lars Enlund och Nils Sandvik i olika perioder. Det gamla magasinet längst söderut, som ursprungligen ägts av Jeppo-Oravais Handelslag, revs på 1980-talet.



%%%
% [house] Oravais Fabriks Magasin
%
\jhhouse{Oravais Fabriks Magasin}{893-871-2-3}{Stationen}{8}{119}

\jhhousepic{OFmagasin.jpg}{Oravais Fabriks magasin, nr 119}

%%%
% [occupant] KWH-
%
\jhoccupant{KWH-}{koncernen}{}

\jhbold{1920-talet}
I  början 1920-talet byggdes detta magasin för Oravais Fabriks behov. Ända sedan järnvägens tillkomst var Oravais Fabrik en av de stora kunderna vid stationen. Det är idag svårt att föreställa sej att ännu de första åren på 1930-talet kunde det anlända hästkaravaner omfattande upp till 50 ekipage med svarta glänsande seldon till stationen från Oravais. År 1930 antecknas att den första lastbilen anlänt med frakt. Efter den tidpunkten minskade hästtransporterna raskt.

För att smidigare kunna ta emot och avsända leveranserna byggdes detta magasin invid det yttersta sidospåret. Med åren klämdes det in mellan Jeppo-Oravais Handelslags spannmålsmagasin och Backlunds magasin. Det har under 1980-talet tidvis varit i användning av privata kortvariga hyresgäster, men har länge fört en mycket tynande tillvaro vad gäller utnyttjandet.



%%%
% [house] K-Lantbrukscentralen John Nyman Kb
%
\jhhouse{K-Lantbrukscentralen John Nyman Kb}{893-871-2-3}{Stationen}{8}{120}

\jhhousepic{K-Nyman.jpg}{John Nyman driver verksamhet i fd Varma Andelshandels magasin}

%%%
% [occupant] Nyman
%
\jhoccupant{Nyman}{\jhname[John]{Nyman, John}}{1975--}
John Nyman, som 1967 öppnat ``Fyrens snabbköp'' i Silvast (se nr 92) köpte 1975 Varmas magasin vid stationsområdet. Det s.k. ``saltlagret'' i magasinets södra ända (nr 420) revs efter en tid ner p.g.a. sitt dåliga skick. Efter att han startat handel med lantbruksprodukter i det hus som i tiden blev ortens sista gästgiveri och inrymt bl.a. John Lasséns kolonialvaruaffär, uppstod ett behov av större utrymmen. Magasinet vid stationen fungerade som lager, men det var opraktiskt att ha själva affären i centrum och en del av varorna på annat håll. 1984 beslöts att renovera  och inreda magasinet till en affärslokal. Sven Jungell blev anställd att utföra jobbet. Samtidigt byggdes en ny lagerhall i direkt anslutning till affärslokalen och den 15 mars 1985 hölls öppning av de nya utrymmena. Lantbruksaffären i centrum av Silvast stängdes. I dessutrymmen öppnade fru Gunilla en affär för blommor, trädgård och inredning.

De nya affärs- och lagerutrymmena medgav ett större sortiment och en livligare handel med konstgödsel, utsäde, byggnadsvaror och maskiner av alla de slag. Den Lantbrukscentral som John öppnat i Jakobstad något år tidigare, avslutades och verksamheten flyttade nu till Jeppo. Den istället nyöppnade affären Kesmotors i Jakobstad, kom att samarbeta med affären i Jeppo, i främsta hand gällande personalen. Under denna tid hade John anställda expediter. Torbjörn Backlund, Nils Strand och Bengt Elenius avlöste varandra under olika perioder.

1989 valde John Nyman att avsluta Lantbrukscentralens verksamhet för egen del och i stället satsa på Kesmotors i Jakobstad. Verksamheten såldes till f:ma Kjellman, men fastigheten behölls i egen ägo. Efter något år flyttade emellertid Kjellman till Esse för att där expandera  och hela fastigheten tömdes. Sedan dess har speciellt lagerhallen använts till uppbevaring av de båtar som hör till John Nymans affärsrörelse.


%%%
% [occupant] Varma
%
\jhoccupant{Varma}{\jhname[Andelshandel]{Varma, Andelshandel}}{1930-talet-1975}
När Varma Andelshandel, tillhörande OTK-gruppen, öppnade sin affär vid vägkorsningen till Oravais i Silvast (se nr 391) på 1930-talet, blev det nödvändigt att skaffa sig större utrymmen för handeln med agrara produkter och byggnadsmaterial. Ett nytt magasin byggdes vid stationen. Den södra delen av byggnaden byggdes i två plan, medan den norra delen byggdes i ett plan. Den södra delen med sina trappor upp till andra våningen användes främst för säckvaror och den norra delen oftast som mellanlager för höbalar. Lösningen upplevdes som obekväm och opraktisk. Från mellanlagret lastades höbalarna på järnvägsvagnar för transport, mest till norra Finland. Magasinet kom att tjäna Varmas behov tills verksamheten avslutades 1967. Byggnaden stod därefter oanvänd tills John Nyman köpte den 1975.



%%%
% [oldhouse] Statens Järnvägars snickeri
%
\jholdhouse{Statens Järnvägars snickeri}{893-871-2-3}{Stationen}{8}{418}

\jhnooccupant{}

\jhbold{1920 – 1955}

På östra sidan av stationsområdet, i höjd med den lastbrygga som fanns vid huvudspårets västra sida, uppfördes på 1920-talet en byggnad, som kom att fungera som såväl snickeri för SJ:s verksamhet som magasin för arbetsteamet ``ståpparåikkån''. Snickeriet fanns i husets norra ända. Det var avsett att förse Statens Järnvägars behov av olika slag av snickeriprodukter. Det må sedan ha varit inredning till kontor eller vänthallar eller bänkar för utomhusbruk. Produkterna levererades till hela landet. Här arbetade Elis Kennola och Evert Lindström (se nr 352x och 52) i det välutrustade snickeriet vad gällde handverktyg. Vad modernare snickerimaskiner anbelangar var det kanske sämre ställt.

Snickeriet antändes av ett blixtnedslag i mitten av 1950-talet och brann ner till grunden. Det återuppbyggdes aldrig och Evert och Elis fick andra arbetsuppgifter hörande till järnvägen. Bl.a. fick de delta i den ständiga dygnsberedskapen att hålla ångmaskinen startklar vid järnvägens pumphus nere vid ån (nr 405).



%%%
% [house] Statens Järnvägars magasin
%
\jhhouse{Statens Järnvägars magasin}{893-871-2-3}{Stationen}{8}{418b}

\jhnooccupant{}

\jhbold{Ca 1920-ca 1960}

För den kontinuerliga verksamheten vid järnvägen, om det sedan gällde vid stationen eller längs spåret i bägge riktningar, behövdes ett magasin för att uppbevara all den materiel och utrustning som kom till användning. Magasinet inhystes i samma byggnad som snickeriet, men i dess södra ända. Det var ett enkelt rödmålat magasin som revs på 1960-talet. Slipers och järnvägsskenor lagrades under bar himmel.



%%%
% [house] Sportplanen
%
\jhhouse{Sportplanen}{4:111}{Silvast}{8}{121}


\jhhousepic{145-05689.jpg}{Järnvallens servicehus}

%%%
% [occupant] Nykarleby
%
\jhoccupant{Nykarleby}{\jhname[Stad]{Nykarleby, Stad}}{1975--}
Efter kommunalsammanslagningen 1975 överfördes den av JIF ägda Sportplanen på Nykarleby stad. Årsmötet som hölls 11 dec. 1976 beslöt överlåta äganderätten till staden på villkor att planen hålls i sådant skick  att både fotbolls- och friidrottsträningen kan fortgå som tidigare. Likaså krävdes att tyngdlyftningsrummet vid gamla Jungar skola skulle följa med i avtalet; ett som det senare skulle visa sig viktigt yrkande. Det utgjorde en argumenteringsgrund som senare skulle bidra till byggandet av en ny konditionshall.

Den 4 juli 1987 brann läktaren, som  haft plats för 350 personer. Branden förlöpte snabbt och allt som återstod var eftersläckning och städning inför följande dags fotbollsmatch. Branden hade sin orsak i ungdomars tjuvrökande. Den ersattes av en ny, men betydligt mindre version.


%%%
% [occupant] Jeppo
%
\jhoccupant{Jeppo}{\jhname[Kommun]{Jeppo, Kommun}}{1974--\allowbreak 1975}
Den 1 jan. 1974 övertog kommunen ansvaret och kostnaderna för skötseln av området. Sportplanen hade då i nästan 40 år förvaltats och skötts på frivillig bas av idrottsföreningen. Äganderätten kvarstod hos föreningen.\jhvspace{}


%%%
% [occupant] Jeppo
%
\jhoccupant{Jeppo}{\jhname[Idrottsförening]{Jeppo, Idrottsförening}}{1937--\allowbreak 1974}
Den 31.12.1937 sålde Jeppo Ungdomsförening, genom Sven Häll, Sportplanen till Jeppo Idrottsförening (JIF), genom Sven Jungar. Området hade fram till dess förvaltats av Ungdomsföreningens egen idrottssektion, sedan föreningen den 16 febr. 1935 undertecknat ett köpebrev med Axel Silfvast gällande ca 2,5 tunnland öster om järnvägen.

Efter att JIF övertagit sportplanen, inramades den med en granhäck. En stor del av verksamheten avstannade under kriget. År 1946 framfördes  en tanke om att bredda sportplanen norrut, men inte förrän 1951 kunde man förverkliga tanken, när ett nytt köpebrev med Vilhelm Silfvast kunde undertecknas den 17 okt. Det betydde en förstoring på 3036 m² och nu kunde också planerna på en ny läktare förverkligas, placerad på planens norra långsida. En stockinsamling genomfördes 1951. Med Paul Sandströms lastbil och med pålastning för hand av Tauno Stenvik, Erik Stenvall och Ingmar Björkvik kom sej stockarna till sågen. En läktare för 300 personer fordrade mycket virke.

Läktaren ritades av byggmästare Georg Romar och uppfördes av Ensio Kula med hjälp av brodern Vilhelm Kula. Den ursprungliga granhäcken mot norr måste till en början avlägsnas där läktaren skulle placeras och en ny planteras vid den nya rågränsen. Samtidigt kunde man nu utöka löparbanornas antal från 4 till 6. Fyllnadsmaterial kördes från Slangar med häst. Med enkel utrustning jämnades materialet ut. Den första blygsamma läktaren med omklädningshytter i planens östra ända kunde nu avlägsnas; den hade gjort sitt.

I detta sammanhang började också uttalas en önskan om bättre beläggning på banorna. Efter en tid bestämde man sej för kolstybb, ett material som inte var det bästa, men det fick duga. Kolstybben kom i öppna järnvägsvagnar, lastades på hästkärror och portionerades ut på banorna för att därefter jämnas ut. En handdragen sladd gjorde den sista utjämningen och slutligen vältades banorna med en ca 2000 kg tung handdragen betongvält. En åtgärd som upprepades varje år.

På 1950-talet jämnades också innerplanen och gjordes mera lämplig för fotboll. Sportplanen var nu redo för många idrottsfester, som berett idrottsintresserade jeppobor mycken glädje. Under hela denna intensiva uppbyggnadstid var talkoandan för det mesta utomordentlig. Men det behövdes eldsjälar och bland dem kan vi nämna: Ivar Almberg, Sven Jungar, Runar Nyholm, Tauno Stenvik, Edvin Jungell, Ingmar Björkvik, Hilding Nygård, Holger Snickars, Helge Asplund, Paul Björkqvist, John Lassén och Brynolf Kula.

Efter kraftansträngningen 1951 kunde man vid årsmötet 1953 med tillfredsställelse konstatera att sportplanen var i det närmaste klar och Ingmar Björkvik blev avtackad med pokal för sin stora insats! Svårigheterna att få styrelseposterna besatta, gjorde att man 1966 för första gången diskuterade frågan om att överlåta sportplanen till Jeppo kommun, men det dröjde till 1 jan. 1974 innan detta blev verklighet.


%%%
% [occupant] Jeppo
%
\jhoccupant{Jeppo}{\jhname[ungdomsförening]{Jeppo, ungdomsförening}}{1935--\allowbreak 1937}
Ungdomsföreningen undertecknade den 16 febr. 1935 ett köpebrev med Axel Silfvast om inköp av 2,5 tunnland mark för grundande av sportplan. Föreningens idrottssektion stod fram till 1937 för sportplanens grundläggande.



%%%
% [house] Smedgärde \& Sjöblom
%
\jhhouse{Smedgärde \& Sjöblom}{4:119 resp. 4:51}{Silvast}{8}{122}
Styckad av stomlägenhet Norrgård 4:8

\jhhousepic{144-05688.jpg}{Ralf och Marita Linder}

%%%
% [occupant] Linder
%
\jhoccupant{Linder}{\jhname[Ralf]{Linder, Ralf} \& \jhname[Marita]{Linder, Marita}}{1990--}
År 1990 övertog Ralf Erik Linder, \textborn 26.11.1954 på Silvast, jordbruket och sitt barndomshem Smedgärde 4:119 på Lövbackvägen, och samtidigt sin farbror Alfreds lägenhet på Böös hemman. Han flyttade hem med sin sambo, merkonom Marita Kåla, \textborn 29.09.1959 i Karleby lkm. Ralf blev student 1975 och arbetade därefter under 8 år som sjöman.

På 1980-talet träffade han Marita. Ralf och Marita har köpt mera jord, som nu omfattar 32 ha odlad jord och 25 ha skog. Dessutom arrenderar de 20 ha odlad jord. De har specialiserat sig på potatisodling. År 1993 byggde de ett potatislager och 1997 en plåthall. Bostadshuset grundrenoverades 1998--\allowbreak 1999. Marita och Ralf vigdes 1995.
\begin{jhchildren}
  \item \jhperson{Charlotta	Louice}{08.06.1990}{}, Jstads Handelsläroverk
  \item \jhperson{\jhname[Ellinor Tina Alice]{Linder, Ellinor Tina Alice}}{06.04.1993}{}, Jstads Gymnasium
\end{jhchildren}
Marita har aktivt deltagit i politiken, som styrelsemedlem inom SFP Jeppo, fullmäktigeledamot och styrelsemedlem i Nykarleby stad samt medlem i flera nämnder. Förutom i det egna packeriet har Marita arbetat ca 2 år som bokförare på Norlic, Munsala bokföringsbyrå.


%%%
% [occupant] Linder
%
\jhoccupant{Linder}{\jhname[Runar]{Linder, Runar} \& \jhname[Vera]{Linder, Vera}}{1946--\allowbreak 1990}
År 1946 köpte Runar Emil Linder, \textborn 14.07.1919 på Böös hemman, ett jordområde i Silvast vid Lövbackvägen utgörande en del av lägenhet Mellangård	4:53. Säljare var Fredrik och Ellen Thors. År 1952 köpte Runar och Vera lägenhet Sjöblom 4:51 av K.W. Sjöblom. Båda lägenheterna utgör en del av stomlägenhet Norrgård 4:8. Runar gifte sig 1946 med Vera Valdine Norrgård, \textborn 22.03.1922 i Böle, Korsholm. De byggde ett bostadshus, som blev inflyttningsklart 1949 samt 1950 ett 	fähus med förråd och garage. Efter kriget fick Runar tjänst på järnvägen i Jeppo som växel- och stationskarl på stationsmagasinet. Runar och Vera tyckte om att dansa. De fick tre barn.
\begin{jhchildren}
  \item \jhperson{\jhbold{\jhname[Ralf]{Linder, Ralf}} Erik Richard}{26.11.1954}{}, student 1975
  \item \jhperson{\jhname[Bjarne Mikael]{Linder, Bjarne Mikael}}{29.10.1955}{}, student 1974
  \item \jhperson{\jhname[Mona Gunilla]{Linder, Mona Gunilla}}{29.10.1955}{}, merkonom 1975
\end{jhchildren}
Bjarne och Mona är tvillingar. Bjarne har arbetat som telegrafist och sjöman och bor numera som pensionär i Sundom. Mona är merkonom och är sambo med Greger Engström född i Ytteresse. De har en son, Oliver. Mona arbetar som löneräknare på Jakobstads Social- och Hälsovårdsverk och tidigare på stadskansliet i Nykarleby. Familjen bor i Nykarleby.
Runar och Vera flyttade 1990 till en hyreslokal i ``Jeppo-stugan'', på 	Åkervägen. Runar dog 20.06.2001 och Vera levde som änka i bostaden till 2012, då hon flyttade till Hagalund. Vera dog den 10.10.2015.
