
KARTA nr 7 placeras här ------>


\jhhouse{Kotimökki}{3:38}{Fors}{7}{129}

\jhoccupant{Asplund}{Benny \& Åsa}{2016 -}
Benny Asplund, \textborn 21.02.1972, gift med Åsa Frilund, har fr.o.m 2016 övertagit fastigheten, men familjen bor i Lippjärvi, Nykarleby.


\jhoccupant{Asplund}{Dödsbo}{2014-2016}
Som sterbhus har huset stått obebott.


\jhoccupant{Asplund}{Helge \& Gurli}{1957-2014}
Helge Johannes Asplund, \textborn 29.01.1933 i Pensala, gifte sig med Gurli Anita Juslin, \textborn 15.01.1935 i Jeppo. De köpte huset av Erik och Hjördis Haga år 1957.

Helge började som 13 åring arbeta på skinnfabriken vid Kiitola. När den lades ner fungerade han till en början som diversearbetare och arbetade med manuell skogsdikning innan han fick anställning inom det på orten nystartade företaget Mirka, ett företag han var trogen fram till sin pensionering. Gurli har haft hand om skötseln av hemmet och familjen med de 3 sönerna. Därtill har hon arbetat med städning vid både Posten och Mirka.

\jhhousepic{Asplund.jpeg}{}

\begin{jhchildren}
  \item \jhperson{Bengt}{16.03.1954}{}, har länge jobbat som montör på Jeppo Kraft, bor i Nykarleby
  \item \jhperson{Boris}{30.07.1959}{}, bor i Sverige
  \item \jhperson{Benny}{21.02.1972}{}, se ovan
\end{jhchildren}

Helge \textdied 17.03.2014  --  Gurli \textdied 10.07.2016


\jhoccupant{Haga}{Erik \& Hjördis}{1951-1957}
Erik Alfred Haga, \textborn 14.05.1923 på Skog, gifte sig 10.04.1944 med Hjördis Margareta Backlund, \textborn 13.11.1922 i Kimo. Under en permission den 1 maj 1943 träffade han sin blivande hustru Hjördis på ``danslavan'' i Gränden. Hjördis hade då sedan år 1942 varit anställd som kontorsbiträde på Handelslaget.

Efter att ha legat på militärsjukhuset i Vasa, åkte Erik i juni 1944 till Stockholm för att delta i en sex månaders kurs för mekaniker, en kurs som vände sig till krigsskadade finländare. I huset bodde då en familj som evakuerats från Kemijärvi. Den bodde kvar till sommaren 1945. Därefter flyttade familjen Haga tillbaka till huset.

Efter kriget köpte Erik hemmansdelar av sina syskon. Tillsammans med Hjördis drev han ett litet jordbruk och under en period satsade de på  hönseri. Samtidigt arbetade Erik på Kiitola pälsberederi. År 1952 köpte han sin första taxibil av märket Austin. Hjördis arbetade under åren 1955-1957 på Andelsbanken. 1957 såldes huset på Holmen. Erik och Hjördis köpte Åkermarks fastighet på stationsvägen vilken kom att inrymma Kemikalia, affär och medecinskåp, se nr 77.
\begin{jhchildren}
  \item \jhperson{Eivor Inga-Lisa}{17.10.1944}{}
  \item \jhperson{Monica Elisabeth}{28.12.1948}{}
  \item \jhperson{Ingmar Erik Johan}{08.08.1951}{}
  \item \jhperson{z}{z}{}
\end{jhchildren}


\jhoccupant{Haga}{Erik, Margit \& Evald}{1943-1951}
Erik Alfred Haga, \textborn 14.05.1923,  Margit Susanna Haga, \textborn 01.12.1925 och brodern Johannes Evald Haga, \textborn 04.12.1928, övertog fastigheten efter faderns död 18.12.1943. Erik och Hjördis köpte sedan fastigheten 01.12.1951.


\jhoccupant{Haga}{Johan}{1939-1943}
Marg
Johan Mariasson Haga, \textborn 05.04.1883 i Ylistaro, men bosatt på Skog, köpte den 18.06.1939 en tomt på Holmen, tillhörande Fors hemman nr 3 av Torsten Forss och köpesumman var 6000 mk. Huset timrades upp och kom att tjäna som bostad också åt de tre barnen Erik, Margit och Evald. Erik blev inkallad i kriget år 1942 och sårades i mars 1943. Margit och Evald studerade på annat ort.

Johan dog 18.12.1943 i huset på Holmen och de tre syskonen övertog fastigheten.



\jhhouse{Kotimökki 2}{3:124}{Fors}{7}{329}

\jhoccupant{Savolainen}{Juho \& Elna}{1936- ..}
Stationskarlen Juho Konstantin Savolainen, \textborn 07.07.1897 i Töysä, gifte sig 20.05.1922 med Elma Eliina Lempinen, \textborn 29.06.1898 i Ähtäri. De anlände från Alahärmä 04.12.1935 och bodde en tid på stationsområdet innan Juho inköpte ett hus på Holmen som byggts av Erik Nyman, svärson till Mårten och Brita Kajsa Theel och där dessa bott på sin ålderdom. Erik Nyman sålde sin lägenhet 1929 till Henrik Johansson Jungell, som överlät denna till sonen Birger Jungell.

Det hus som Mårten och Brita Kajsa bott i, köptes småningom av Juho Savolainen och timrades upp på den från Torsten Forss 16.11.1936 inköpta tomten bredvid bönehuset. Här bodde Juho till sin död. Hans hustru avled 1955 och deras dotter Kaija, \textborn 14.05.1927 växte upp här. Sedan Kaija gift sig 1949 flyttade hon till Puolanka, men återvände som änka därifrån 17.12.1952 med sin dotter Tuula, \textborn 15.02.1950. De bodde då med Juho och Elma tills de flyttade till Polvijärvi ett år senare.

Juho \textdied 08.10.1972  --  Elma \textdied 02.12.1955



\jhhouse{ Åbrant}{3:45}{Fors}{7}{130}

\jhoccupant{Jääskeläinen}{Adam}{2004 -}
Adam Jääskeläinen, \textborn 27.10.1978 i Nykarleby, Brännon, köpte fastigheten i början av 2004. Adam är metallarbetare och arbetar på Mirkas verkstad.
Barn: Astrid Malinen, \textborn 2009


\jhoccupant{Dahlskog}{Bengt \& Linnea}{1975-2004}
Bengt Fredrik Dahlskog, \textborn 01.12.1932 i Kronoby, gifte sig 1965 med Märta Linnea Häggqvist, \textborn 02.11.1934 i Lappfors,Esse.

Bengt utbildade sig till forsttekniker och var anställd hos Willhelm Schaumann Ab som virkesutdrivare inom Jeppo och Munsala fram till 1982. Efter det anställdes han av Jeppo Skogsvårdsförening inom vilken han var verksamhetsledare fram till sin pensionering. I okt 1975 köpte de huset. Se nedan.

\jhhousepic{Nyvik.jpeg}{}

Linnea blev utdimitterad folkskollärarinna från Ekenäs seminarium år 1958. Hon tjänstgjorde till en början vid Kållby folkskola, året 1967-1968 vid Evangeliska Folkhögskolan i Hangö som ämneslärare för att därefter återvända till Kållby fram till 1974 då hon flyttade till Jeppo centrumskolas lågstadium, där hon verkade fram till pensioneringen. Många är de jeppoelever som i henne haft sin första lärare.
\begin{jhchildren}
  \item \jhperson{Östen Gerhard}{06.07.1968}{}
  \item \jhperson{Åsa Linnea}{31.01.1970}{}
  \item \jhperson{Malin Johanna}{05.09.1977}{}
\end{jhchildren}

Bengt avled i Jakobstad 02.12.2016.


\jhoccupant{Nyvik}{Bertel \& Maj-Lis}{1960-1975}
Bertel Simon Vilhelm Nyvik, \textborn 07.08.1932 på Slangar, gifte sig som 23-åring med Maj-Lis Asta Alina Sandberg, \textborn 15.02.1935 på Finskas. År 1960 inköptes tomten på Holmen, alldeles intill Silvast åbro, och ett nytt hus uppfördes 1963.

Bertel är utbildad till forsttekniker och har tjänstgjort inom Ab Wilhelm Schaumann som anskaffningstekniker. Maj-Lis har under lång tid varit huvudbokförare på Nykarleby stads ekonomikontor. De sålde huset i okt 1975 till Bengt och Linnea Dahlskog.



\jhhouse{ Älvbo}{3:101, 3:102}{Fors}{7}{131}

\jhoccupant{Norrback}{Britten \& Thomas}{1993 -}
Thomas Gustav Norrback, \textborn 16.04.1959 i Sideby, gifte sig med Britten Viola Enlund, \textborn 04.12.1956 på Fors. Familjen köpte huset år 1993  av Åke och Ragnhild Grönlund.

Thomas var från 4 års ålder tillsammans med sina föräldrar Gustav och Märta Norrback på missionsfältet i Kenya. 1973 återvände familjen och läsåret 1974/75 gick han i Evangeliska Folkhögskolan i Vasa där han lärde känna sin blivande fru. Han fortsatte studierna på Österbottens svenska centralyrkesskola, billinjen. Därefter fortsatte han vid Vasa Tekniska läroanstalt och efter utbildningens slut var han t.f. lärare i metallslöjd vid Smedsby högstadieskola och vid yrkesskolan i Vasa. 1984 blev han föreståndare för Vasa svenska församlings lägergård i Alskat. År 2000 fick han arbete på Mirka i Jeppo och familjen flyttade nu till det hus som de sedan 1993 haft som sommarbostad.

\jhhousepic{BrittenNorrback.jpeg}{}

Britten har genomgått en barnavårdskurs i Helsingfors och en hjälpskötarutbildning i Vasa. På Vasa Centralsjukhus har hon haft vikariat och tjänst på Roparnäs sjukhus barndaghem. Efter flytten till Jeppo har hon också  arbetat på daghemmet i Jeppo. Som gift har hon också haft dagbarn.
\begin{jhchildren}
  \item \jhperson{André Gustav}{21.06.1982}{}
  \item \jhperson{Bernt Johan}{16.04.1985}{}
  \item \jhperson{Elisabeth Berit}{24.05.1995}{}
\end{jhchildren}


\jhoccupant{Grönlund}{Åke \& Ragnhild}{1975-1993}
Tor Åke Vilhelm Grönlund, \textborn 09.05.1921 på Heikfolk, gifte sig 1945 med Göta Ragnhild Eklöv, \textborn 21.08.1925 på Gunnar. Åke och Ragnhild har varit jordbrukare på hans hemgård på Ojala fram till 1975, då sonen Stig, \textborn 12.01.1947, övertog jordbruket. Samma år köpte de detta hus, ägt av änkan Saga Nyvik. De bodde i huset fram till 1993, då de sålde det till Thomas och Britten Norrback. Därefter flyttade de in i ett av radhusen på byggplaneområdet i centrum av Jeppo.


\jhoccupant{Nyvik}{Vilhelm \& Saga}{1963-1975}
Anders Vilhelm Nyvik, \textborn 05.12.1891 i Kantlax, gifte sig 1928 med Saga Johanna Julin, \textborn 15.02.1904 på Mjölnars hemman. Samma år blev de jordbrukare på Slangar fram till 1937, då de köpte en egen lägenhet i Ytterjeppo.

1963 byggde sonen Bertel och hans hustru Maj-Lis nytt hus på inköpt tomt på Holmen och nu passade Vilhelm och Saga på att köpa detta grannhus som var till salu. År 1967, den 1 april, avled Vilhelm och Saga fortsatte att som änka bo i huset fram till 1975, då sonen Bertel med Maj-Lis sålde sitt hus till Bengt och Linnea Dahlskog och flyttade till Nykarleby. Saga följde efter till Nykarleby, där hon avled den 22.02.1996.


\jhoccupant{Häger}{Ingrid \& Alfred}{1955-1963}
Ingrid Maria Lindström, \textborn Westerlund 03.05.1892 på Fors, hade 29.10.1950 ingått sitt 2:a äktenskap med Gustav Alfred Häger, \textborn 04.01.1881 i Nykarleby lk. De var änka och änkling på respektive håll och de vigdes i Tavelsjö, Sverige, där Ingrid vistats hos sin dotter Elsa och erhållit svenskt medborgarskap. Efter giftermålet flyttade de till Nyk.by lk. och år 1955 flyttade de till Jeppo och det hus hon bott i med sin 1:a, nu avlidna man.

Alfred avled 23.08.1961 och Ingrid var på nytt änka. Hon fortsatte livet med att hjälpa andra människor som behövde en hjälpande hand. Hon   skötte   länge ``uppdraget'' att elda och hålla varmt i bönehuset inför olika samlingar så länge krafterna räckte. När Bertel och Maj-Lis skulle börja bygga sitt nya hus, föreslog hon att de istället skulle köpa hennes. I stället blev det Bertels föräldrar som köpte huset. Ingrid flyttade till Nykarleby, där hon dog 01.09.1967.


\jhoccupant{Forss}{Eliel \& Ingrid}{1910-1955}
Elias Eliel Forss (senare Lindström), \textborn 13.10.1888 på Fors, gifte sig 10.03.1912 med Ingrid Maria Westerlund, \textborn 03.05.1892  på Fors.

Eliel byggde i början av 1900-talet det lilla huset. Efter giftermålet och den 1:a dotterns födelse, reste han till Australien 1913 och kom tillbaka 1919. Han fick då för första gången se sin 2:a dotter. Efter att sett ytterligare 2 döttrar komma till världen, reste han till USA 1923, där han dog 17.08.1948 utan att återse fosterlandet.
\begin{jhchildren}
  \item \jhperson{Elsa Erika}{06.10.1912}{}
  \item \jhperson{Elvi Maria}{16.11.1913}{}
  \item \jhperson{Ines Ingrid}{11.11.1920}{}
  \item \jhperson{Iris Edit}{01.09.1923}{}
\end{jhchildren}



\jhhouse{Forsgård}{3:12}{Fors}{7}{100, 100a-b}

\jhoccupant{Forss}{Leif \& Ingeborg}{1975 -}
Leif Olof Forss, \textborn 03.06.1949 i Jeppo, gifte sig 22.07.1972 med Ingeborg Dahlqvist, \textborn 27.06.1951 i Purmo.
\begin{jhchildren}
  \item \jhperson{Gunilla Marianne}{1973}{}, boende i Purmo
  \item \jhperson{Annika Bernice}{1978}{}, boende i Nykarleby
  \item \jhperson{Benita Ingeborg}{z}{}, boende i Vasa
\end{jhchildren}

\jhhousepic{148-05729.jpg}{}

Gården har varit i släktens ägo sedan början av 1800 -talet. Den övertogs av nuvarande ägarna 1975 samtidigt som ett nytt fähus byggdes för utvidgad mjölkproduktion. Vid sidan av jordbruket jobbade Leif vid sågen, som ägdes av Jeppo Kraft. Innan jordbruket avyttrades 2004, hade mjölkproduktionen lagts ner och makarna övergick till förvärvsarbete; Leif inom byggnadsbranchen och Ingeborg inom socialvården. Den gamla hemgården (nr 400), som fanns på tomten från tidig 1800-tal, revs 2011, stockarna togs tillvara av dottern Annika, plockades upp på nytt och används nu som sommarstuga ute i Nykarleby skärgård.


\jhoccupant{Forss}{Erik \& Cecilia}{1937-1975}
Erik Evald Forss, \textborn 19.10.1915 i Jeppo, gifte sig 1943 med Cecilia Maria Sandvik, \textborn 22.02.1915 i Purmo, Sexsjö.
\begin{jhchildren}
  \item \jhperson{Carl-Erik}{1944}{}
  \item \jhperson{\jhbold{Leif Olof}}{1949}{}
  \item \jhperson{Gerd Margareta}{1951}{}
  \item \jhperson{Nils Gottfrid}{1952}{}
\end{jhchildren}

Erik deltog i fortsättningskriget 1941-1944. Han var mycket intresserad av hästar och travsport vid sidan om skötseln av jordbruket. Största framgången hade han med  stoet ``Murtis-Liisa'', som 1953 slog finskt rekord över 1 km med tiden 1.36.8. Den första traktorn köptes till gården 1954. På gården fanns Avelsföreningens tjurar åren 1940-50. Från gården skötte man flera år intransporten av mjölk med traktor och vagn till mejeriet.

Erik \textdied 06.09.1990  --  Cecilia \textdied 27.01.1992


---> Den gamla gården på \jhbold{Forsgård} 3:121,revs 2011. Fors hemman, karta 7, nr \jhbold{400}

\jhoccupant{Forss}{Edvard \& Ida}{1916-1937}
Erik Edvard Forss, \textborn 11.04.1892, gifte sig med Ida Sofia Bärs, \textborn 20.09.1885.
\begin{jhchildren}
  \item \jhperson{Ellen}{13.11.1914}{}
  \item \jhperson{\jhbold{Erik Evald}}{19.10.1915}{}
  \item \jhperson{Leo}{25.03.1920}{10.7.1920 pga leversjukdom}
\end{jhchildren}

Edvard var bonde på Forss hemman men hade också intresse för biodling och trädgårdsodling. Även byggnadsarbete hörde till hans intressen. Han var med i verksamheten för bönehuset, där han också var styrelsemedlem.

Edvard \textdied 21.12.1954  --  Sofia \textdied 17.11.1967


\jhoccupant{Thomass. Forss}{Elias \& Sofia}{1864-1916}
Elias Forss, \textborn 08.01.1841, gifte sig 10.6.1866 med Sofia Jakobsdotter, \textborn 23.02.1847.
\begin{jhchildren}
  \item \jhperson{Anna Lovisa}{07.06.1867}{}
  \item \jhperson{Katarina}{24.02.1869}{04.02.1870 till följd tandsprickning}
  \item \jhperson{Sanna Lisa}{23.01.1871}{}
  \item \jhperson{\jhbold{Johannes Jakob}}{14.12.1874}{}
  \item \jhperson{Thomas Wilhelm}{15.06.1878}{}
  \item \jhperson{Anders Emil}{25.07.1880}{}
  \item \jhperson{Selma Maria}{02.03.1884}{}
  \item \jhperson{Johanna Sofia}{03.12.1886}{}
  \item \jhperson{Elias Eliel}{12.10.1888}{}
  \item \jhperson{\jhbold{Erik Edvard}}{*11.04.1892}{}
\end{jhchildren}

Elias hade tillfälle att gå två år i Nykarleby elementarskola och förvärvade sig därigenom färdighet i skrivning och räkning, som han senare i livet skulle få stor användning för. Redan som ung fick han ta över hemmanet på Fors. Det hade varit kronohemman, men blivit inlöst till skattehemman redan 13.10.1790. Elias arbetade tidigt och sent på sitt jordbruk, med framgång. Det var ett tungt arbete eftersom inga jordbruksmaskiner då ännu fanns.

Elias var djupt religiös och offrade mycken tid och energi på söndagskoleverksamheten, den tidens första skolundervisning. Denna verksamhet gav senare upphov till en privat folkskola i Jeppo, nämligen Jungar folkskola. Han var även verksam som nämndeman i Nykarlebyting. Elias hade även fallenhet för intellektuellt arbete. Han skrev sägner och gav dem rimkrönikans form. Det mesta rörde händelser under ofredstider. Tyvärr har hans anteckningar till följd av utlåning försvunnit. Han skrev även korrespondenser (kåserier) i ortstidningen Österbottniska Posten under signaturen -s-s. Han anlitades  ofta vid bouppteckningar och upprättande av andra handskrifter samt vid lantmäteriförrättningar.

Farfadern Johan Thomasson (se koppling till Broända 3:6 på nr 397) sålde fastigheten till Elias 12.9.1864. Elias blev en trotjänare och stod vid rodret hela 52 år, en ansenlig tidsperiod som är svåröverträffad.

Enligt sägnen var den allra första som bosatte sig på Fors, eller Lill-Silvast som det kallades förut,  kommen från  Kronoby. Därför  kallas nu platsen ``Krymboas'' d.v.s kronobyboas. Var den första bosättningen fanns har inte kunnat fastställas. Den första som i mantalslängden antecknats som kommen från Kronoby är Jakob Jakobsson. 1699-1709 äger han hela hemmanet, som dock varit bebyggt och brukat i minst 100 år före det.



\jhhouse{Jaskari}{3:26}{Fors}{7}{101, 101a}

\jhoccupant{Jaskari}{Jarkko \& Regina}{1994 -}
Jarkko, \textborn 1969 från Härmä, gifte sig 02.07.1989 med Regina Hinkkanen, \textborn 1967 från Jeppo.
\begin{jhchildren}
  \item \jhperson{Tomi}{08.12.1989, boende i Jeppo}{}
  \item \jhperson{Tina}{05.06.1992, boende i Härmä}{}
\end{jhchildren}

\jhhousepic{150-05874.jpg}{}

Familjen Jaskari köpte och flyttade in i huset 1.7.1994. Jarkko jobbar som serviceingenjör och Regina är sjukskötare och arbetar på Nykarleby bäddavdelning. De har i flera repriser renoverat huset och tillhörande uthus. Tomi är utbildad artesan och arbetar på Kannustalo i Oravais. Tina som, har studerat på handelsskolan arbetar på Citymarket i Jakobstad.


\jhoccupant{Nybyggar}{Alf}{1988-1994}
Gården köptes år 1988 av Alf Nybyggar, \textborn 19.03.1956. Han sålde den 1994. Alf arbetar som långtradarchaufför.


\jhoccupant{Löv}{Jarl \& Emilia}{1939-1988}
Jarl Axel Löv, \textborn 01.07.1908, gifte sig 10.10.1939 med Emilia Södergård från Böös, \textborn 07.07.1911.
\begin{jhchildren}
  \item \jhperson{Ragnhild}{23.03.1942}{}, gift Solgård, boende i Sverige
  \item \jhperson{Gunvor}{10.03.1951}{}, gift Andersson boende i Sverige
  \item \jhperson{Roger}{09.08.1952}{}, gift m. Gunlis, boende i Jakobstad
\end{jhchildren}

Bostadsbyggnaden uppförde Jarl 1952 efter att samma år ha rivit den tidigare byggnaden (nr 401) där Jarls mor Johanna och syster Selma delat hushåll. Det sägs  att Johanna och Selmas far, Elias Thomasson Forss (se fastighet nr 100), byggde ett hus åt flickorna för att Selma var ogift och Johanna var ensam med sonen Jarl efter att hennes man Isak emigrerat till Amerika och sedermera brutit kontakten till hemlandet. Hans öde i Amerika är okänt, han hördes aldrig mera av.

--->  Hus \jhbold{nr 401}
Den ursprungliga stugan på tomten bestod av ett kök och en sal och på mitten ett sovrum och tambur. Byggnadsåret för detta hus är okänt, men om det som sägs ovan är riktigt, torde huset ha timrats upp i början av 1900-talet.

Tomten har under åren förstorats två gånger via köp av Elmer Fors vartefter ekonomibyggnad uppfördes och tillbyggdes. Till lägenheten hörde ingen ärvd jord eller skogsmark. Jarl arbetade på Kiitola spinneri och senare som eldare i pannrummet, därefter blev han maskinmästare på mejeriet. Järnvägsstationerna i Jeppo och Karleby har också varit hans arbetsplatser. Han arbetade även vid pumpstationen vid ån, därifrån vatten pumpades till vattentornet bredvid stationen.

Tack vare sitt stora hästintresse köpte Jarl med tiden upp jord och byggde upp ett småbruk, som han först skötte på fritiden och som så småningom blev hans huvudsyssla. Han arbetade även kortare perioder på sågen.

Johanna, ovan, \textdied 1960  --  Selma, ovan, \textdied 1966
Jarl \textdied 07.09.1985  --  Emilia \textdied 09.09.2001



\jhhouse{Vallgård}{3:137}{Fors}{7}{102, 102a}

\jhoccupant{Strengell}{Bruno \& Siv}{1974 -}
Nils Bruno Strengell, \textborn 31.01.1946, gifte sig 02.11.1968 med Siv Inger Pärus, \textborn 15.07.1946 i Kimo, Oravais. År 1974 övertog Bruno och Siv hemmanet och Ellen och Valdemar bereddes bostad i Sikströms hus (se nr 99). Det hus som Bruno och Siv nu flyttade in i hade köpts i Oravais (Bertils) 1947. Där plockades det ner, transporterades till Jeppo och återuppbyggdes. Till sin hjälp hade Valdemar sin svärfar Edvard Forss och ytterligare Edvard Sandbacka. Ekonomiebyggnaden uppfördes 1950 och gav plats för ett halvt dussin mjölkkor.

\jhhousepic{151-05707.jpg}{}

En eldsvåda förstörde ekonomiebyggnaden 1987. Den byggdes upp på nytt och äggproduktionen återupptogs, men efter att verksamheten på jordbruket avslutades 1995, upphörde äggproduktionen året därpå. Åkerarealen är nu avyttrad, men skogen kvarstår i makarnas ägo.

Bruno är utbildad maskinmontör. Han har arbetat heltid, vid sidan om jordbruket, som chaufför/montör vid olika transportföretag. Från 1996 fram till sin pensionering var han montör vid KWH-Mirka.

Siv är utbildad frisör och vårdare. Hon arbetade på lägenheten fram till dess att verksamheten lades ner. Från 1996 fram till pensionen har hon varit anställd på Hagalunds pensionärscenter i Nykarleby.
\begin{jhchildren}
  \item \jhperson{Harriet Ann-Louise}{28.06.1970}{}, Ped.mag., Optima samkommun.
  \item \jhperson{Fredrik Kim Bruno}{14.02.1976}{}, Trafiklärare vid Ajovarma
\end{jhchildren}


\jhoccupant{Strengell}{Valde \& Ellen}{1947-1974}
Paul Valdemar Strengell, \textborn 06.09.1906, gifte sig vid midsommar 1937 med Ellen Sofia Forss, \textborn 13.11.1914 på Fors. De bodde till en början på Jungar (Ruotsala nr 26) innan de via bageriverksamhet på Silvast (nr 98) år 1947 uppförde åt sig en bostad och 1950 en ekonomiebyggnad samt övertog en del av Edvard Forss' hemman.

Valde hade efter hemkomsten från Canada 1935 en tid arbetat som mjölnare på Silvast kvarn tillsammans med Edvard Källman, men efter köpet av Sigurd och Heldine Jungars bageri i Silvast 1938 blev den verksamheten deras levebröd fram till 1947.
\begin{jhchildren}
  \item \jhperson{Curt Valdemar}{27.01.1938}{}
  \item \jhperson{Roy Erik}{03.09.1939}{}
  \item \jhperson{Bengt Roger}{26.05.1941}{}
  \item \jhperson{\jhbold{Nils Bruno}}{31.01.1946}{}
  \item \jhperson{Gundel Helena}{31.01.1948}{}
  \item \jhperson{Bo Rune}{19.04.1956}{}
\end{jhchildren}

Valdemar \textdied 24.03.1993  --  Ellen \textdied 01.06.2012



\jhhouse{Forss}{3:36}{Fors}{7}{108, 108a-b}

\jhoccupant{Forss}{Gunnar}{t.v.}
Gunnar Forss, \textborn 28.11.1942, ansvarar numera för fastigheten. Gunnar bor i Sibbo.


\jhoccupant{Forss}{Johannes \& Sirkka-Liisa}{1973 -}
Kurt Johannes (Johan) Forss, \textborn 18.11.1935, gifte sig 10.10.1972 med Sirkka-Liisa Tikkala, \textborn 09.04.1940 i Jurva. År 1973 övertogs hemmanet från Johans föräldrar Torsten och Annie Forss. Ladugården renoverades, men de nya kvotbestämmelserna gjorde att båsplatserna inte kunde fyllas, varför produktionen efter en tid avslutades. År 1992 avyttrades största delen av åkermarken medan skogsmarken stannade i familjens ägo.

\jhhousepic{098-05706.jpg}{}

I 17 år har Johan fungerat som brandchef i Jeppo. När Lions Club bildades i Jeppo 1966 var han en av de stiftande medlemmarna. Han har under lång tid arbetat på Jeppo Food som inhoppare under högsäsongerna. Samma funktion har han haft som t.f. vaktmästare vid Jeppo skola, kommunalgården, brandstationen och vid bostäderna för pensionärerna. Också Greger Nygårds firma har utnyttjat hans tjänster, speciellt sommartid.

Sirkka-Liisa hade innan hon flyttade till Jeppo varit städerska vid finska flicklyceet i Vasa i 12 år och vid Jeppo skola i 8 år. I Vasa var hon ofta anlitad vid hovrättspresidentens bjudningar.
\begin{jhchildren}
  \item \jhperson{Ria}{23.10.1972}{}
  \item \jhperson{Kalle}{28.03.1980}{}
\end{jhchildren}


\jhoccupant{Forss}{Torsten \& Annie}{1945-1973}
Torsten Ludvig Forss, \textborn 04.03.1910, gifte sig 08.07.1934 med Annie Maria Lindström, \textborn 17.01.1910. Efter kriget fick Torsten och Annie överta fastigheten 1945.

Huset hade byggts 1935 åt Torstens bror Elmer, som hade planer på att i det väl tilltagna huset öppna en dörr mot landsvägen och i ett rum innanför inreda och utrusta ett apotek. Men kriget kom emellan och Elmer hann aldrig bo i huset eller fullfölja sina planer. Efter att under kriget verkat inom sanitärtrupperna, utbildade han sig vid social- och kommunalhögskolan och har därefter arbetat som vårddirektör i Grankulla, Hangö och Borgå. I Jeppo övertog han sin gamla hemgård vid åkanten, intill åbron på Fors och en del av den närmast belägna åkermarken.

Under kriget hade farmor Maria Lovisa ,familjen Henry och Alli Törnqvist och familjen Göta och Lennart Elenius bott i huset. År 1945 påbörjades byggandet av ny ladugård för 6 kor och 2 hästar, för att senare också inhysa grisar.
\begin{jhchildren}
  \item \jhperson{\jhbold{Kurt Johannes}}{18.11.1935}{}
  \item \jhperson{Greta Marianne}{21.01.1938}{}
  \item \jhperson{Nils Gunnar}{28.11.1942}{}
  \item \jhperson{Maja Linnea}{04.07.1945}{}
\end{jhchildren}


\jhoccupant{Forss}{Elmer}{1935-1945}
Elmer Forss, \textborn 28.03.1912, byggde huset på hemmanets jord år 1935 för att år 1945 byta fastighet med sin bror Torsten.


---> Lägenhet R:nr 3:46 av Fors hemman, karta 7, nr \jhbold{408}     --> BILD

\jhoccupant{Forss}{Elmer}{1945-1959}
Elmer Forss (se ovan) var den sista ägaren till det gamla huset vid ån, för två familjer, som revs i två omgångar. Elmers hemgård utgjordes av den norra ändan närmast åbron över till Holmen. Här hade hans far Johannes Jakob Eliasson Forss, \textborn 14.12.1874, den 05.07.1896 gift sig med Maria Lovisa Finskas, \textborn 18.12.1874, och övertagit ½ hemmanet efter Elias Thomasson Forss, \textborn 08.01.1841 och hans hustru Sofia Johansdr. Heikfolk, \textborn 23.02.1847 (se nr 400).

Johannes (Johan) blev med tiden utnämnd till poliskonstapel i Jeppo, men efter att bägge benen blev amputerade under 1900-talets första decennium, blev han tvungen att avsäga sej uppdraget. Efter detta rörde han sej utomhus med tjocka stoppade läderdynor under benstumparna. Han var starkt engagerad i byggandet av bron över ån till Holmen 1910-11.
Johan Jakob och Maria Lovisa fick sex barn.
\begin{jhchildren}
  \item \jhperson{Fanny}{19.12.1899}{}, gift Jungar
  \item \jhperson{Johannes}{03.03.1902}{}
  \item \jhperson{Ester}{10.08.1907}{}, gift Jungar
  \item \jhperson{\jhbold{Torsten}}{04.03.1910}{}
  \item \jhperson{\jhbold{Elmer}}{28.03.1912}{}
  \item \jhperson{Göta}{12.03.1916}{}, gift Elenius
\end{jhchildren}

Huset, som delades av två familjer, byggdes sannolikt på tidigt 1800-tal. Denna norra ändan revs i medlet av 1960-talet, medan den södra
ändan revs 2011. Hemmanet hade av kungliga kammarkollegiet år 1790 försäkrats inköpt av kronan och var nu ett s.k. skattehemman.

Johan Jakob \textdied 13.05.1935  --  Maria Lovisa \textdied 24.11. 1950



\jhhouse{Jeppo Krafts såg}{3:61}{Fors}{7}{109, 109a}

\jhbold{Tiden 1954 -}

Sågen i Silvast hade ursprungligen byggts 1902 alldeles i närheten av den gamla kvarnen av stock, som stod kvar till 1920. Gamla foton ger vid handen att sågen var placerad så att insamlingsrännan kunde riktas mot ån för att stockarna lättare kunde styras in i sågen. Både kvarnen och sågen fick sin kraft från samma turbiner, vilket tidvis kunde leda till tvister. Sågen stod något ``omlott'' med kvarnen för att
kraftöverföringen mellan byggnaderna skulle vara möjlig.

I vilket fall innebar sågens placering en påtaglig belastning på omgivningen och verksamheten, som vi kan läsa i ``Historik över Jeppo'' : ``Kvarnen, sågen och mejeriet samsades om inkörsväg. Den var tungt belastad, speciellt på vårkanten, och blev snart en enda lervälling, som av de stora kärrhjulen östes upp över mjölkkannorna. Stockar som under vintern lagts på snön vid vägkanten, kunde vid töväder rulla ner på vägen och hindra framkomligheten. Om hästkusken inte orkade få upp en sådan stock igen, rullade han den till vägens mitt så att kärrhjulen kunde passera på vardera sidan om den. Vägen var ofta vansklig att färdas på, också för hästarna, för många stockar kunde samtidigt ligga utspridda i den djupa leran......man blev nödsakad att inreda ett allmänt utedass för folk som av olika orsaker måste vänta på att få sina såg-, kvarn-, mejeriärenden uträttade''. Till yttermera förtret kunde seldonen gå sönder och lånet från någon av granngårdarna återlämnades ofta också de söndriga.

Det sågade virket travades i stora travar norrom sågen ända fram till platsen för pärthyveln. Området var kort sagt belastat. Situationen blev bättre när det nya mejeriet flyttats till annan plats och togs i bruk 1933, men problemen kvarstod.

I samband med att Jeppo Kraftandelslag beslöt att restaurera och modernisera kvarnen i börja av 1950-talet, beslöts samtidigt efter en rätt lång juridisk armbrytning att sågen skulle flyttas bort. Andelslaget undertecknade ett köpebrev med kommunen 29.12.1951 om inköp av ett markområde, tillhörande Södergård Rno 3:41 av Fors skattehemman, närmare järnvägen. Området var 15605 m² stort. Vid en extra andelsstämma den 17 jan. 1952 godkändes köpet.

\jhhousepic{100c-05640}{}

Erik L Back fick långt gångna fullmakter att verkställa beslutet och byggandet av en ny sågbyggnad påbörjades. En nyare ram köptes och andelslaget fick både byggnadsvirke och dagsverken till skänks av medlemmarna, men anläggandet av den nya ramsågen blev ändå att kosta 3 milj. mk. Trots detta var beslutet om flytten otvivelaktigt riktigt. Den intensiva byggverksamheten i samhället, som tagit sin början efter krigen, innebar en stor efterfrågan av sågat virke och köerna till sågen var tidvis långa. Samtidigt utnyttjades sågens kapacitet under sommarhalvåret av Jeppo Skogsandelslag, som genom denna verksamhet kunde saluföra sina medlemmars virke i förädlad form. Stora travar med sågat virke reste sig öster om sågen, klara för export. Ansvarig för denna verksamhet var Verner Sjö och andelslagets disponent Bernhard Fogström. Åtskilliga 1000-tal stockar passerade genom sågramen varje år och Skogsandelslaget byggde 2 lagerhallar för det virke som torkat i stora staplar på utsidan, staplar som matades med färskt sågat virke längs en smalspårig järnväg som ledde från sågen.

Allteftersom byggtekniken och materialanvändningen förändrades i samhället, minskade efterfrågan på sågat virke. Samtidigt var det svårt att klara enhetskostnaderna hos små sågar. Verksamheten började visa minussiffror och på 1980-talet upphörde driften med ramsågen. I dess ställe byggdes en mindre cirkelsåg, som kunde betjänas av en enda man. I början sköttes den av Sven-Olof Forsbacka och sedan av Leif Forss, för att efter år 2005 helt upphöra med driften. Efterfrågan hade upphört. En epok var till ända.

I dagsläget används sågbyggnaden som lagerutrymme för Jeppo Kraft Alg, (d.v.s. den har fått samma funktion som kvarnen hade åren 1975-98).



\jhhouse{Klövernötarens Hus}{3:61}{Fors}{7}{110}

\jhoccupant{Jeppo}{Lantmannagille}{t.v.}
År 1934 hade Jeppo Lantmannagille anskaffat en enkel maskin. Till följd av det ökande intresset för odling av klöver, till vilken inspiration  hämtades från A.I. Virtanens forskning kring klöverns proteininnehåll, uppstod behov att få en klövernötare till  orten. Den naturliga organisationen att köpa en sådan maskin var förstås Lantmannagillet, en  organisation  av  bönder för praktisk hjälp och information i det dagliga livet. Gillet var också den organisation som i början av 1900-talet hade en egen butik i Silvast.
\jhhousepic{Klovernotare.jpeg}{Växtligheten kommer tätt inpå ett tomt hus.}
Klövernötarens funktion var enkel. Den skulle rispa sönder skalet på klöverfröet och så skonsamt som möjligt få hylsorna runt fröet att lossna och därigenom öka fröets grobarhet. Naturligt har fröet så tjockt skal att groningstiden kan bli lång och resultatet osäkert. Klöverfrö fanns förstås att köpa, men det har alltid varit dyrt och då det gick att odla själv var det intressantare. Det förutsatte att fröet kunde ``nötas''. Gillets medlemmar hade möjlighet att mot en avgift använda maskinen till sitt hemodlade utsäde. Den hade en skötare som säsongvis tog hand om driften, oftast under sen höst och på vårvintern.

Ett hus för uppbevaring och användning av nötaren stod till en början på ``Svinvallen'', ett samfällt område för Fors skattehemman nära järnvägsstationen. 1955 beslöt gillet att ett nytt hus skulle byggas. Det blev uppfört våren 1956 av Ingmar Björkvik, Edvin  Bäckstrand och Olof Bäckstrand i närheten av Jeppo Krafts nyflyttade såg i Silvast. Efter att nötaren fått sitt nya hus, minskade behovet av dess tjänster rätt snabbt, då tillgången på kvävegödselmedel blev större och kunde ersätta den osäkrare klöverodlingen.



\jhhouse{Brandstationen}{3:133}{Fors}{7}{111}

\jhoccupant{Nygård}{Greger}{2011 -}
År 2011 inköpte Greger Nygård (se nr 107) brandstationsbyggnaden av Nykarleby Stad. Byggnaden har sedan dess delvist hyrts av staden
till den del brandverket behövt utrymme för den brandbil som finns stationerad i Jeppo. Till övriga delar har Greger använt stationen som
garage för sitt företags traktorer och bostaden har hyrts ut bl.a till Helge Bergman, \textborn 1933, och Asko Linjamäki, \textborn 1939 i Vörå, med hustrun Maria (Mäki), \textborn 1948 i Jeppo.

\jhhousepic{101-05641.jpg}{}

\jhoccupant{Nykarleby}{Stad}{1975-2011}
Efter kommunsammanslagningen fram till 2011 har staden ägt och i stort sett använt brandstationen för brandväsendet, som det ursprungligt var planerat, men placeringen av tankbilen har tidvis skiftat. Oy Mirka Ab:s fabriksfastigheter har krävt ett geografiskt nära placerat utryckningsfordon.


\jhoccupant{Jeppo}{Kommun}{1968-1975}
I slutet av år 1968 färdigställdes brandstationen i samband med byggandet av såväl ny kommungård som nya pensionärsbostäder. Brandstationen medgav en standardförhöjning vad gäller beredskapen för utryckningar och tornet för att torka slangarna var välkommet. Tornet var ursprungligen utrustat med sirén på taket, men efter ett år med 14 falska alarm, förorsakade av automatiken, kopplades den bort.

I samma byggnad som brandstationen färdigställdes samtidigt en bostad för den fastighetsskötare/vaktmästare, som var tilltänkt att sköta det kommunala fastighetsbeståndet. Här bodde till en början Ulf och Gunilla Nybäck innan Ulf år 1969 flyttade från anställningen som tågklarerare vid stationen till Nykarleby. Här bodde också Sven-Olof och Lise-Maj Forsbackas familj (se nr …... Åkervägen) och därefter Erik Östmans familj, den senare också släckningschef i Jeppo.



\jhhouse{Kommungården}{893-408-3-133}{Fors}{7}{112}

\jhoccupant{FAB-Kommungården}{Jeppo}{2011 -}
Nykarleby stad ansåg sig ha alltför stor fastighetsmassa, både att äga och att underhålla. År 2011 beslöt staden att göra en värdering av den tidigare kommungården i Jeppo och den brandstation som  byggts samtidigt. Båda objekten blev utbjudna för försäljning. Ett i bildande varande fastighetsbolag, FAB Kommungården Jeppo, med Greger Nygård och Johnny Rönnqvist som likvärdiga ägare, inköpte den tidigare kommungården. Brandstationen (se nr 111) blev köpt av Greger Nygård ensam.

\jhhousepic{102-05644.jpg}{}

Efter ägobytet har kommungårdens lokaliteter för Lantbrukskansliet, som nu administreras av Pedersöre kommun, överförts på denna som hyrestagare från 2013. Nykarleby stad har  hyrt utrymmen för hälsorådgivningen i de lokaler som använts sedan tidigare. Nykarleby stad har också hyrt utrymmen för hemvårdens personal, men som år 2016 fick ny plats i en tidigare lärarbostad i radhuset på skolområdet.

Övre våningen har omdisponerats och delvis byggts om till bostäder, som hyrs ut, den senaste tiden till asylsökande flyktingar. Detta boende
administreras av Oravais flyktingförläggning och hyrestagare är Vörå kommun. Nedre våningens bostäder och i källarvåningen inredda rum hyrs också ut, företrädesvis till utländsk arbetskraft.


\jhoccupant{Nykarleby}{stad}{1975-2011}
Nykarleby stad övertog i samband med kommunsammanslagningen i Nykarlebynejden fr.o.m. 1.1.1975 äganderätten till fastigheten. Genom åren har fastigheten använts som ett sidokansli till stadsförvaltningen. Lantbrukskansliets funktioner stationerades hit och likaså avbytarservicen. Dessa har senare fått andra huvudmän;
\begin{enumerate}
  \item Lantbrukskansliet överfördes 2013 till Pedersöre kommun, men kvarstår i fastigheten.
  \item Avbytarservicen i övre  våningen överfördes genom LPA:s beslut till Vörå 2008 och flyttade bort.
  \item Hälsorådgivningen och också barnrådgivningen har under denna tid disponerat över sina utrymmen som ursprungligt var planerat.
  \item Övre våningens sammanträdesrum har genom åren också använts till annan aktivitet, som till exempel ``Musiklekis'' och mötesplats för olika föreningssammankomster.
\end{enumerate}


\jhoccupant{Jeppo}{Kommun}{1968-1975}
Beslutet om att bygga ett nytt kommunhus med start 1967 gladde säkert många. Det hus på Holmen som använts av kommunen sedan 1937 hade gjort sitt. I det huset skulle trängas sessionssal, bibliotek och kommunalkansli. Dessutom saknade det brandsäkert arkiv, vilket gjorde att kommunen blev tvungen att hyra in sig i andras arkiv och det var opraktiskt.

\jhhousepic{Jeppo-vapen.jpg}{}

Huvudentrepenör blev byggföretaget Häggblom \& Hyövälti och som platschef skulle fungera Lars-Erik Lövdahl. Den 25 okt. 1967 hade grundningsarbetena startat. En extremt kall vinter satte käppar i hjulen och försenade byggandet, men med insats av en större arbetsstyrka kunde arbetet forceras och kommungården kunde tas i bruk som sammanträdesplats den 30.10.1968. Hela byggnadskroppen skulle innehålla kommunalkansli, hälsovård, bostäder för anställda (här bodde bl.a. hälsosystern Tea Fogström med familj och barnmorskan Lisbeth Pantolin med familj på övre våningen), sammanträdesplats och andra mötesrum i andra våningen, arkiv och servicefunktioner i källaren.

Samtidigt byggdes ny brandstation inklusive bostadsdel och flera efterlängtade pensionärslägenheter i olika storlekar.

Allt var nytt och ståtligt och den 20 mars 1969 anländer landshövding Martti Viitanen till Jeppo för att högtidligt inviga den stora satsningen tillsammans med stolta jeppofäder. Hela komplexet var otvivelaktigt det mest förnäma i nejden. Det hade kostat 100 milj. gamla mark eller 1 milj. nymark.



\jhhouse{Jeppo-Pensala skola}{893-50-5012-1}{Fors}{7}{104}

\jhoccupant{Nykarleby}{stad}{1975 -} inkl. \jhbold{Jeppo Kommun} 1966-1975
Efter en process kring centraliseringen av skolväsendet, som i många stycken var likartad den vi ser idag, beslöt kommunfullmäktige i Jeppo den 12.12.1960 att Jeppo skulle utgöra ett enda skoldistrikt i stället för som tidigare 4 stycken; Jungar, Gunnar, Måtar och Bärs. Den 23 okt. 1963 hade planeringen av en ny centrumskola avancerat så långt att ett uppgjort byggnadsprogram godkändes och som Skolstyrelsen sedan godkände några veckor före jul.

\jhhousepic{103-05642.jpg}{Nya daghemmet i förgrunden och den omdisponerade skolan från 1966 i fonden}

Arkitekt Erik von Ungern Sternberg, som fått uppdraget att rita skolan, kom på besök efter årsskiftet för att bekanta sej med platsen där skolan skulle placeras, den gamla ``Sågtomten''. Här, på den nu av kommunen ägda tomten på Fors hemman nr 3, hade tidigare stått en ångsåg ägd av Norra Trävaru i Österbotten med eget järnvägsstickspår anslutet till stationen. (foto)

Nu tillsattes en byggnadskommitté med bonden Karl Sandås som ordförande och lärare Olof Lillqvist som sekr. Midsommarveckan 1964 kom plötsligt det oönskade beskedet från Undervisningsministeriet att statsstöd inte skulle erhållas för bygget! Ny anhållan insändes i augusti och med hjälp av skolinspektor Åbonde, kommunminister Grels Teir och Jeppos egen riksdagsman J.A. Jungarå lyckades man nu uppnå ett positivt resultat.

Vid sitt första sammanträde 1965,  den 18 jan., valde kommunens fullmäktige byggnadsbyrå Berg \& Backlund från Vasa som huvudentrepenör för bygget. Redan veckan därpå var bygget igång. Efter ett intensivt byggande kunde taklagsfesten hållas redan efter drygt 8 mån. När skolstarten skedde den 26 aug. 1966 kunde lärare och elever från de tidigare skoldistrikten ta ibruk en splitterny och durabel skola.

Till skolan anslöts också kommunens bibliotek. Samtidigt byggdes också radhusbostäder för lärarna närmare ån. Skolan har genom åren genomgått smärre reparationer, men efter att Nykarleby stad beslutat om Pensala skolas indragning och elevernas förflyttning till Jeppo Centrumskola, uppstod genast utrymmesproblem och under 2015-16 har skolbyggnaden byggts till med nya fina utrymmen för kommunal dagvård och ett till klassrum för skolans behov. Samtidigt flyttades biblioteket till tidigare barnträdgården och de friställda utrymmena tillsammans med en del av kökets krympta  utrymmen togs ibruk för förskolans behov. Allt blev klart till det nya läsåret hösten 2016 och kommundelen kan nu erbjuda den uppväxande ungdomen goda utrymmen för både lärande och fritid.

Vid inflyttningen i skolan 1966 arbetade lärarna: Sven Jungar 1966-68, Elna Sandberg 1966-69, Runar Nyholm 1966-73, Agda Norrback 1966-74 och Olof Lillqvist 1966-86.

Följande tjänsteinnehavare har jobbat vidare i ett viktigt arbete:
\begin{center}
  \begin{tabular}{l l l l}
    \hline
    Åke Lillas & 1968-70 & Ellen Nygård & 1970-71 \\
    Nanna-Lisa Finskas & 1971-2007 &  Linnea Dahlskog & 1974-94 \\
    Ole Nordström & 1973-1997 & Stefan Stubb & 1986-91 \\
    Anders Wingren & 1993-2001 & Nancy Jakobsson/Achrén & 1996- \\
    Susanne Sandin & 1997-     & Stina Viklund & 1989-91 \\
    Anders Vingren & 1992-2001 & Stina Viklund & 1993-97 \\
    Anne-Marie Stenfors Kronqvist & 1998 - & Mayvor Grönroos Lillkung & 1999- \\
    John Semskar & 2001- & Maria Ståhl & 2003- \\
    \hline
  \end{tabular}
\end{center}

Utöver dess lärare, som verkat över längre anställningsförhållanden, har många speciallärare och timlärare verkat under kortare tider, ofta
endast ett läsår och emellanåt endast en termin eller t.o.m. kortare tider. Det har ofta handlat om vikariat:
\begin{center}
  \begin{tabular}{l l l}
    \hline
    Brita Iiskola & Linvi Nylund & Beatrice Grandén \\
    Helen Lönnqvist & Myrtel Nyholm & Berit Edman \\
    Svanhild Erikslund & Birgitta Ferm & Gunilla Norrgård \\
    Carita Hermans & Christina Simons & Carita Östberg \\
    Kristine Bergström & Kerstin Nysten & Agneta Sjöblom \\
    Ann-Louise Strengell & Ulrika Sjölind & Kirsi Broo/Saari \\
    Anja Häggman & Heli Lehto & Marica Enlund \\
    Tua Söderbacka & Solveig Liljeström & Majvor Palmberg \\
    Camilla Björk & Eva Wester & Susanna Östman \\
    Karin Nilsström & Helena Havulehto & Sara Jansson \\
    Maria Lillrank & Ann-Christine Elenius & Camilla Fellman \\
    Kaj Finne & Anne-Marie Virtanen & Susann Björk \\
    \hline
  \end{tabular}
\end{center}

En skola utan bespisning vill vi inte ha och städat ska det va'! Därför riktas stor tacksamhet till följande personal:
\begin{center}
  \begin{tabular}{l l l l}
    \hline
    Köksa i skolan & År & Städare har varit & År \\
    Lise-Maj Forsbacka & 1966-77 & Ragnborg Nygård & 1966-89 \\
    Barbro Renlund & 1977-87 & Sirkka-Lisa Forss & 1990-94/98-04 \\
    Siv-Britt Furu & 1986- &  Elisa Norrgård & 2005- \\
    - & - & Lisen Aspnäs & 2015- \\
    Kökshjälpsbiträden: & . & . & \\
    Gudrun Julin & 1986-90 & Helena Forss & 1990-96 \\
    Ing-Britt Forss & 1996-2009 & Carola Nyman & 2009-15 \\
    Julia Dahlqvist & 2009-11 & Eva Domis & 2011-14 \\
    Michaela Lindholm & 2015- & . & .  \\
    \hline
  \end{tabular}
\end{center}

Ansvaret för underhåll och säkerhet har legat på vaktmästare/gårdskarl/släckningschef :
\begin{center}
  \begin{tabular}{l l l l}
    \hline
    Ruben Nygård & 1966-76 & Johan Back & 1976-80 \\
    Håkan Julin & 1980-85 & Erik Östman & 1985-94 \\
    Carl-Erik Forss & 1994-2009 & Karl-Erik Blom & 2009-11 \\
    Ingvar Jåfs & 2011 & . & .  \\
    \hline
  \end{tabular}
\end{center}

Vartefter skolan ändrat karaktär, har också nya grupper av personal vunnit insteg i skolans värld:
\begin{center}
  \begin{tabular}{l l l l}
    \hline
    Förskolan: & . & . & . \\
    Susanna Markkula förest. & 2000-14 & Kristina Forsbacka & 2000-09 \\
    Jonina Häggblom  '' & 2014- & Marica Sandvik vik. & 2015 \\
    Gisela Neuman vik '' & 2015-16 & . & . \\
    Skolkuratorer: & . & . & . \\
    Johanna Dahlskog/Haglund & 2006-12 & Anna Ingo & 2008-09 \\
    Malin Muukkonen & 2007-08 & Sara Höglund & 2013-15 \\
    Sandra Svedlund & 2015-16 & . & . \\
    Hälsovårdare: & . & . & . \\
    Ann-Louise Sundsten & 2005-14 & Maily Ahlin & 2008-09 \\
    Ghita Wikblom & 2015-16 & . & . \\
    Psykolog: & . & Sofia Hedman & 2005-11/15-16 \\
    Skolgångsbiträden: & . & . & . \\
    Eva-Maj Sandvik & 1991-92 & Liudmyla Gudenko & 2014- \\
    Maj-Britt Hietamäki & 2014- & Stephan Esselbach & 2014- \\
    \hline
  \end{tabular}
\end{center}

Skolans lärarbostäder utnyttjades till en början i huvudsak av lärarna, men hyrdes an efter också till andra. År 2010 flyttade förskolans verksamhet in i en av lokalerna, kallad ``Lillstugan'', med föreståndaren Susanna Markkula och fortsatte med sin verksamhet här tills det var dags för nästa flytt 2012 till färdigställda utrymmen vägg i vägg med biblioteket. Efter Susannas pensionering 2014 fortsatte verksamheten ännu ett år i samma utrymmen, men nu med Jonina Häggblom som föreståndare. 2015 var det på nytt dags att flytta för ett nytt byggnadsingrepp och nu till ett färdigställt klassrum i skolan. Sedan det nya \jhbold{Jeppo Daghem} blivit klart 2016 och dess klientel flyttat till sin rätta plats, flyttade \jhbold{förskolan} tillbaka till sin plats i anslutning till det nu tömda biblioteket med t.f. föreståndare  Gisela Neumann.

\jhbold{Gruppfamiljedagvården} har verkat i Lillstugan åren 1999-2010 med Kristina Julin och Anna Elenius som personal. En del av daghemmet var utlokaliserat till Lillstugan åren 2013-14 för 5-åringar och 2014-2015 för 1-3 åringar. Personalen bestod då av barnträdgårdslärare Desirée Fridlund och barnskötare Marina Björklund.

När det nya daghemmet var klart, nybyggt i anslutning till skolan, startade verksamheten där i jan. 2016. Daghemsföreståndare är barnträdgårdslärare Gun Åkerlund. Barnträdgårdslärare är Desirée Fridlund, Anna Stennabba och Jenny Sorjonen. Som barnskötare fungerar Tanja Björklund, Marina Björklund och Virpi Elenius. Lilian Julin är dagvårdare.

Efter att de kommunala investeringarna var klara på skolområdet 2016, flyttade också \jhbold{hemvårdens kansli} från sina hyrda utrymmen i den tidigare kommunalgården till en av de lediga lokalerna som finns i de tidigare lärarbostädernas radhus. Flytten inleddes 2017.

\jhbold{Biblioteket 1966-2016}

I samma byggnadskropp som skolan inrymdes också kommunens bibliotek. Det hade tidigare funnits i samma hus som den gamla kommunalgården på Holmen. Nya rymligare utrymmen kunde nu tas i användning och dess anslutning till skolans funktioner blev en klart stimulerande lösning för bokintresset då biblioteket kunde besökas också på rasterna.

Bibliotikarier har varit:
\begin{enumerate}
  \item Sven Jungar, 1966-71
  \item Runar Nyholm, 1971-76
  \item Karin Elenius, 1976-
  \item Anna-Lisa Dahlsten, 1982-83
  \item Susanna Kung
  \item Johanna Häggblom
  \item Lotta Karlström
\end{enumerate}


\jhhouse{Biblioteket}{893-50-5012-4}{Fors}{7}{105}

\jhoccupant{Nykarleby}{stad}{2016 -}
Jeppo kommundels bibliotek flyttade våren 2016 från de utrymmen det sedan år 1966 disponerat i Centrumskolan, till barnträdgårdens nu lediga fastighet. I och med detta fick biblioteket större utrymmen och biblioteket öppnade den 23 maj nyrenoverat och fräscht.

\jhhousepic{106-05643.jpg}{}


\jhoccupant{Barndaghemmet}{x}{1984-2014}
I och med att barnträdgårdens utrymmen, där den verkat sedan 1972 i tidigare Jungar skolas ena lärarbostad, alltmer ansågs gamla och otidsenliga, beslöt Nykarleby stad att bygga ett nytt daghem i närheten av Jeppo skola (Centrumskolan) på den tomt som sedan 1885 varit säte för statsjärnvägarnas pumpstation för distribution av vatten till järnvägen. En byggnadskommitté under ledning av fullmäktige Edgar Jungarå tillsattes och tillsammans med personalen besöktes olika nyare byggda barnträdgårdar/daghem i närheten. Efter att planeringen  blivit klar, startades bygget våren 1984 och fastigheten var inflyttningsklar till hösten till kostnaden 630.000 mk. Invigning hölls 15.11.1984.

Nu fick personal och barn flytta in i nya ändamålsenliga utrymmen anpassade för just barnens behov. Till en början fungerade fastigheten  som daghem för 6 år fyllda barn, men i och med förändringar i lagstiftningen, som år 2000 införde begreppet förskola, fick barnen mellan 1 och 5 år samsas med 6-åringarna, som nu gick i förskola.

2010 flyttade förskolans barn till en av de tidigare lärarbostäderna och 2012 till utrymmen i anslutning av det ursprungliga biblioteket. Personalen vid förskolan/daghemmet /eftis har varit:
\begin{enumerate}
  \item Susanna Markkula 1984-2010, föreståndare, (2010-14 förskola)
  \item Gudrun Julin, 1984-1995, daghem
  \item Barbro Nordling, daghem
  \item Åsa Forsgård 1996-2009, daghem
  \item Kristina Forsbacka 1999-2014, daghem/eftis
  \item Kristina Julin 2010-2014, daghem
  \item Anna Lundvik 2009-2016, föreståndare 2010
  \item Ulla Linder 2009, daghem
  \item Virpi Elenius 2009-2016, daghem
  \item Gun Åkerlund 2010-2016, föreståndare 2010 -
\end{enumerate}



\jhhouse{Konditionshallen}{893-50-5012-3}{Fors}{7}{106}

\jhoccupant{Nykarleby}{stad}{1986 -}
När Jeppo idrottsförening hösten 1976 beslöt att överföra ägandet av sportplanen till Nykarleby stad, yrkades samtidigt att det utrymme i gamla Jungar skola, som användes för styrketräning, skulle följa med. Utrymmet hade i tiden något renoverats, men det var kallt vintertid och uppvärmningskostnaderna föll på föreningen. Hösten 1979 kom uppvärmningssystemet i oskick och påpekades åt idrottsnämnden att det låg på deras bord att åtgärda problemet.

\jhhousepic{107-05672.jpg}{}

År 1981 hade kostnaderna för uppvärmningen blivit en påtaglig belastning och utrymmets skick upplevdes vara uselt och i januari 1982 uttalades en önskan att ordna en krafthall på annan plats att hyra. Blickarna riktades i första hand mot ungdomslokalen. I april anser man inom föreningen att också utrustningen är bristfällig och fullmäktige Jarl Romar lovar att i förtroendemannakretsar försöka finna förståelse för behovet. JIF hade ju bland regionens bästa juniorer inom flera grenar och behovet av goda träningsmöjligheter för dem var nödvändiga.

Redan på hösten 1982 börjar ärendet röra på sig och ett stormöte med olika nämnder och ledande tjänstemän hålls för att få en lösning. Under tiden besluter föreningen att anskaffa förnyade redskap och maskiner till den existerande lokalen. Tanken på en tillbyggnad av Uf-lokalen får tummen ner av stadsstyrelsen våren 1983 med motiveringen att staden inte äger Uf-lokalen. Blickarna riktas nu mot skolans garage och för att sätta tryck i processen besluts att uppgöra en besökslista i krafthallen för att påvisa besöksmängden.

I slutet av aug. 1984 godkänns ritningarna till en ny konditionshall men diskussionen kärvar till sig kring kostnaderna. Vad får den kosta? En halv miljon mark blir taket för kostnaderna och bygget startar i september 1985. I årsberättelsen för 1985 utbrister föreningen glatt: ``Äntligen händer det. Den efterlängtade konditionshallen är under byggnad. Våra tyngdlyftare, bodybildare och andra viktlyftare får drägliga förhållanden, kanske de bästa i hela nejden. Vi får hoppas att hallen skall bli flitigt använd av både proffs och amatörer alla dar.'' Staden äger byggnaden, men lovar inte att stå för utrustning och inredning.

Invigning hålls hösten 1986 samtidigt som JIF firar sitt 50-årsjubileum. Hallen inreds med befintliga redskap. Nät och skydd sätts upp för att tillåta kastträning inomhus. I början av 2000-talet visar Mirka intresse för konditionshallens möjligheter och erbjuder sig att stå för en komplett ny redskapsutrustning i utbyte mot att deras personal utan ersättning kan använda den för fysisk träning; ett rbjudande som faller i god jord och blir gällande. Idag är konditionshallen en uppskattad plats för både barn, ungdom, idrottande personer och människor långt upp i åldrarna.


\jhhouse{Ångsågen och Såggården}{893-50-5012-3}{Fors}{7}{406}

\jhbold{Tiden 1907-1966}
År 1907 anlände från Alavo kapten Daniel Roos, \textborn 13.12.1846. Han skulle bli verkställande direktör för Norra Trävarubolagets nya ångsåg (nr 406), som byggdes på Fors hemman i Silvast. Ägare var Labbart \& Schauman. Med hjälp av duglig lokal arbetskraft byggdes sågen, utrustad med två ramar och driven med ångmotor i en separat tegelbyggnad.

Likaså byggdes en stor husbyggnad, den s.k. Såggården (nr 406a), i närheten av landsvägen för direktör, personal och kontor. Ett drygt 500 m långt järnvägsspår byggdes med anslutning till stationens vedplan så att järnvägsvagnar nu kunde bogseras ända fram till sågen och växlas in till båda ramarnas utkast på var sin sida. På detta sätt kunde man lasta olika sortiment direkt på järnvägsvagnarna utan mellansortering. En stor del av stockarna vinschades upp direkt från ån, där de samlats ihop av särskild personal.


\jhpic{Saha 1907.JPG}{Vy över ångsågen med såggården t.h. i fonden. Platsen är densamma som Jeppo-Pensala skola ockuperar i dag}




\jhpic{Saha ca 1907.JPG}{Här syns stickspåret med kontakt via dåvarande vedplan till stamjärnvägen}




Efter att verksamheten kommit igång ersattes kapten Roos av Karl Fredrik Spolander, \textborn 24.10.1858 i Kronoby. Han var Vd för bolaget åren 1911-13. Under tiden hamnade bolaget i ekonomiska svårigheter och verksamheten avslutades 1913, företaget gick i konkurs 1914. Anläggningen revs och flyttades till Alholmen i Jakobstad.

I juni 1921 körde man sand från den tidigare banvallen, som hade lett till ångsågen, och med hästtransport flyttades delar av den som fyllnadsmaterial till bönehusets gårdsplan. Området köptes nu av Erik Nordström. Huset på tomten revs inte ner. Hösten 1916 överenskommer Evangeliska Unga i Jeppo med Erik Nordström  om att för en månatlig hyra om 5 mk per månad disponera den s.k. Såggården som samlingslokal, medan byggandet av bönehuset färdigställs. Efter en tid inköptes huset av kommunen och kom att fungera som ett hem för åldringar, av endel kallat för ``fattighus'', fram till några år efter kriget. De sista invånarna flyttades därefter till Östervall åldringshem i Nykarleby. En av dem var ``Kola- Maanu''.

En kort tid bodde familjen Alexander Sandqvist i huset 1936/37, men blev p.g.a. plågan med vägglöss tvungen att flytta bort till Östermans hus i närheten (se Fors hemman, karta---, nr 409).

Hösten 1948 kom ett direktiv från skolstyrelsen om inrättande av en VII:e klass utöver de tidigare VI i folkskolan. Denna nya klass måste till en början tränga in sig i befintliga utrymmen, men fr.o.m. 1950 hade kommunen skapat utrymme för alla 7:e klassister här i såggårdens fastighet. Som ny lärare anställdes Bengt Erik Bro, \textborn 09.03.1927 i Övermark, och han anlände 30.12.1950. Han bosatte sig i skolfastigheten, där också jägarfältväbeln/disponenten Carl Didrik von Essen, \textborn 25.12.1875, bodde tillsammans med sin hustru Martha Bredelow, \textborn 18.08.1890 i Magdeburg. De hade gift sej 06.08.1920 och bott på flera platser i socknen.

Bengt Bro gifte sig under tiden i Jeppo med Eila Helmi Eleonora Liljeqvist, som var änka efter mejeridisponenten Robert Liljeqvist, vilken
avlidit 12.10.1944 efter en arbetsolycka i mejeriet. Eila hade barnen Karin Margareta, \textborn 17.10.1941 och Carita Helena, \textborn 26.09.1943, med sig in i giftermålet. I äktenskapet med Bengt Bro föddes Kristina Viola, \textborn 18.03.1952, och Krister Erik Johannes, \textborn 09.08.1953. Bengt Bro stannade i sin tjänst på medborgarskolan fram till 1956, då han ersattes av Edvin Johannes Westin, \textborn 29.10.1904, gift med Elna Alina, \textborn 11.07.1907 i Purmo.
\begin{jhchildren}
  \item \jhperson{Astrid Linnea}{06.07.1941}{}
  \item \jhperson{Valdemar}{21.03.1943}{16.03.1991}
  \item \jhperson{Solveig}{21.08.1947}{}
\end{jhchildren}

Edvin förblev lärare för 7.e och  8:e klassen under deras kvarvarande tid denna byggnad. I olika ämnen utnyttjades också distriktskolornas lärare, såsom Sven Jungar, Runar Nyholm, Olof Lillqvist, Holger Snickars och Jan-Olof Fagerudd. Den sista terminen blev våren 1966.

Huset var tänkt att rivas i och med att den nya skolan blivit färdig. Hösten 1966, när alla kommunens klasser 1-6 samlades i den nybyggda skolan, blev det för medborgarskolans elever i klasserna 7-8 att flytta till den tömda Jungar folkskola. Edvin följde med och avslutade (se nr 342) sin lärargärning 1967. Ny föreståndare blev Helge Hedman, som verkade i f.d. Jungar skola till 1973, då den nya skolereformen koncentrerade högstadiet till Nykarleby. ``Såggården'' stod kvar ännu år 1971, men revs därefter.



\jhhouse{``Juutilainen''}{893-50-5012-1}{Fors}{7}{402}

\jhoccupant{Juutilainen}{Johanna}{1939-1956}
August Alexander Juutilainen, \textborn 02.09.1868 i Sulkava, gifte sig 10.12.1899 med Johanna Karlsdr. Monthen, \textborn 25.07.1871 i Tammela. August blev anställd på Kiitola som färgmästare i början av 1900-talet och familjen byggde sej ett hus på Finskasbacken,(hus nr 39 på sid.136 ``Överjeppo 2''). August dog 11.05.1925 och dödsboet sålde huset 1929 till makarna Joel och Anna Rönnqvist.

\jhhousepic{Juutilainen?.JPG}{Barack vid ångsågen. Är detta Juutilainens?}

Änkan Johanna gifte om sej 05.06.1932 med mjölnaren på Fors, Johan Westerlund, \textborn 11.07.1874 i Oravais. Han hade varit änkling sedan 1929 och Johanna flyttade efter giftermålet tillsammans med sin nya make. Efter Johan Westerlunds död 04.11.1939 flyttade änkan Johanna till detta hus där hon dog 17.10.1956. I Juutilainens äktenskap föddes fem barn.
\begin{jhchildren}
  \item \jhperson{Ragnar August}{12.06.1900}{}
  \item \jhperson{Reino}{1905}{}
  \item \jhperson{Marta Sylvia}{13.02.1907}{}
  \item \jhperson{Meeri Johanna}{10.01.1910}{}
  \item \jhperson{Walto Werner}{29.02.1912}{}
\end{jhchildren}
Efter Johannas död köptes fastigheten av Aarne Waltteri Wikström, \textborn 31.05.1913. Här bodde han med sin syster Ida Wiik. En eldsvåda förstörde uthuset 1963 där paret hade en ko. Aarne avled 13.07.1976.

Fastigheten hade redan tidigare köpts av Jeppo kommun inför det nya skolbygget. Under skolans byggnadstid användes huset som arbetsplatskontor och revs efter att skolan färdigställts hösten 1966. När det byggdes och av vem har vi inte kunnat klargöra.



\jhhouse{Statens Järnvägars pumpstation}{893-50-5012-4}{Fors}{7}{405}

\jhbold{Tiden 1885 – 1980}
När järnvägen byggdes 1885 och sträckningen genom Jeppo var fastställd och placeringen för stationen bestämts, fanns det två saker som måste ordnas för att ångloken skulle fungera; vatten och ved. Ved kunde med hästtransport köras till stationen, men vattnet måste pumpas. Gjutjärnsrör grävdes ner som vattenledning ända till ån och ett pumphus byggdes. Det utrustades med en ångmaskinsdriven vattenpump. Nere vid ån byggdes en brunn, som skulle säkerställa vattentillgången också vid lågvatten.

\jhpic{SJpumphus.JPG}{Pumphuset nere vid älvbrinken försåg vattentornet (karta 6, nr 419) och därmed ångloken med behövligt vatten.}

I anslutning till pumphuset byggdes en bostad för personalen och ett vedförråd för ångmaskinens drift. All ved skulle sågas för hand och mängden var inte alldeles liten då pumpen hela tiden skulle vara redo för drift och övervakning skedde dygnet runt. Från 1944 kördes färdigt kapad ved från stationsområdet till pumpstationen.

Vem som var den första att tjänstgöra här är okänt, men från 1899 tillträdde Otto Svan tjänsten som ansvarig och fortsatte så i 27 år fram till 1926 då han pensionerades. Han flyttade till Kannus den 17 sept 1929 efter 30 år i Jeppo och i familjen hade fötts 11 barn.(se närmare Stationsområdet).

Otto Svan efterträddes av Ragnar Andersson, kallad ``Pump-Ragnar''. Ragnars familj härstammade från Pedersöre och fadern var banvakt/banmästare. De anlände 9 dec.1908 och tre av sönerna fick småningom anställning inom järnvägen; Arthur, Bruno och lillebror Ragnar, \textborn 14.01.1905 i Pedersöre. Han förblev ogift men trogen pumpstationen åren 1926-1960. Sistnämnda år pensionerades han samtidigt som pumpstationen blev eldriven och automatiserad. Evert Lindström blev samtidigt flyttad till andra uppgifter (se nr 52).

Framförhållningen inom SJ var inte den bästa. En ny flera hundra kubikmeter stor brunn byggdes 1959 i åkanten, till stora kostnader. Den blev gjuten i betong och indelad i 3 separata mindre brunnar för sedimentutfällning. Samtidigt förnyades vattenledningen mellan ån och vattentornet vid stationen. Det första dieseltåget körde förbi Jeppo 1961! En ny era höll på att ta över; först diesel, sedan elektricitet.

\jhhousepic{SJVattentornet.JPG}{Vattentornet på 1950-talet intill vedplanen vid järnvägen.}


Ragnar förblev bosatt i  pumpstationens bostad också en tid efter att pumpen tagits ur bruk 1968. Pumpstationen revs för att bereda plats för den nya barnträdgården, som skulle placerades på denna tomt 1984.



\jhhouse{ Östermans hus}{893-50-5012-4}{Fors}{7}{409}

\jhoccupant{Österman}{Frans \& Johanna}{1915-1955}
Frans Oskar Österman, \textborn 09.05.1867 i Tammerfors, gifte sig 03.01.1901 med Johanna Johansdr. Rajala, \textborn 21.02.1876 i Ylihärmä.
Familjen anlände 8 febr. 1913 från Vörå till Mjölnars. Frans var maskinist på Kiitola yllespinneri. 1915 byggde han detta hus på Fors skattehemman för sin familj. Huset stod på högra sidan om vägen ner till VR:s pumphus vid ån, alldeles i närheten av den s.k. Såggården. På ett märkligt sätt decimerades familjen inom loppet av drygt 7 år.

\jhhousepic{Frans Oskars.JPG}{Huset fick ingen lång historia.}

\begin{jhchildren}
  \item \jhperson{Väinö Oskar}{15.11.1901 i Vörå}{23.07.1938}
  \item \jhperson{Paavo Armas}{04.07.1907 i Nurmo}{02.08.1936}, stationskarl
  \item \jhperson{Lauri Waldemar}{18.04.1910 i Jeppo}{11.02.1932}, till Kanada
  \item \jhperson{Aarne Johannes}{27.07.1912 i Jeppo}{30.12.1935}, till J:stad
  \item \jhperson{Saima Johanna}{06.05.1916 i Jeppo}{}
\end{jhchildren}

Frans Oskar \textdied 19.07.1931  --  Johanna \textdied 09.01.1938

Av en familj på 7 personer vid midsommar 1931, återstod endast en person efter den 23 juli 1938. Familjen Alexander Sandqvist flyttade in i huset 1937 och stannade till 1939.  Carl Didrik von Essen med hustru Martha bodde en kort tid i huset. Här har sedan dess bott hälsosyster Valborg Stadigs familj när de kom från Oravais i jan. 1951, men deras vistelse blev kort innan de flyttade till den kommunala hälsogården bredvid (se nr 408).

Nya hyresgäster blev nu Harry och Svea Grahns familj som stannade till 1957. Huset var kallt att bo i och lockade inte hyresgäster i nämnvärd grad varför det jämnades med marken i början av 1960-talet.



\jhhouse{Gamla hälsogården}{893-50-5012-3}{Fors}{7}{410}

\jhbold{Åren 1956 - 1966}
Sedan skolklass VII fr.o.m. hösten 1948 infördes av skolstyrelsen och trängdes in i de befintliga skolorna, framstod behovet av en lösning på utrymmesproblemet som akut. Klass VII och senare klass VIII från de fyra skoldistrikten centraliserades till  den s.k. ``Såggården'' i Silvast. Olika praktiska läroämnen behövde ytterligare utrymmen och på andra sidan vägen, som ledde ner till VR:s pumpstation vid ån, stod nu den tidigare hälsogården tillgänglig och hit flyttades bl.a. undervisningen i huslig ekonomi, en lösning som användes fram till 1962, varefter huset användes som kök och matsal för skolan.

\jhbold{1930 – 1956}
Efter att den sista utbyggnaden vid Jungar högre folkskola var klar med nya utrymmen för småskolan, kunde kommunens hälsovård disponera över den fastighet som tidigare hyst småskolan i Silvast. Här kunde nu sjuksköterskan Selma Johanna Wiis, \textborn 28.07.1888, flytta in. Hon hade anlänt till Jeppo den 11.08.1917 från Kronoby innan Finland ännu blivit självständigt, men där den nya kommunallagen, antagen 1917, skulle träda i kraft 1918 och påbjuda att kommunen i högre grad skulle ansvara för social- och hälsofrågor.

Selma Wiis skulle bli en känd person i kommunen genom sin långvariga insats i hälsoarbetet. När fastigheten blev tillgänglig förlade hon också sitt boende hit, sida vid sida med mottagningen. Selma dog 05.02.1938.

Hit kom också barnmorskan Edit Jolanda Bernas, \textborn 01.04.1896 i Närpes. Hon kom till Jeppo 20.09.1924 och gjorde här sin insats för väntande mödrar fram till 12.04.1956, då hon flyttade tillbaka till Närpes.

Sjukskötare Edit Cecilia Bergström, \textborn 01.01.1893 i Vasa, anlände 1937 och återvände 1965 till Vasa. En kortare tjänstgöring i Jeppo hade hälsosyster Ingrid Aili Maria Boström, \textborn Pirilä 10.07.1921 i Munsala. Hon tjänstgjorde 03.11.1947-02.10.1950, då hon flyttade till Dragsfjärd.

Hälsosystern Valborg Salome Herrgård, \textborn 04.05.1910, hade gift sig 01.08.1948 med Kurt Helmer Stadig, \textborn 29.07.1924. Familjen kom  från Oravais 5 jan. 1951 och stannade till 20 maj 1958, då familjen flyttade till Jakobstad. Barnen Urda Gunborg föddes 1950 och Ulf Helmer 1953. I början av tiden i Jeppo bodde de i Östermans hus (se 409).
