\jhchapter{Jungar by}

\jhsection{Jungar-namnet}

Om man i öppna nätbaserade kunskapskällor slår in ordet Jungar, befinner man sig plötsligt i Nepal i Rolpa-distriktet eller  i västra inre Mongoliet vid stranden av Gula floden. Bägge alternativen angivna med exakta koordinater.

Så långt borta skall vi emellertid inte söka namnets ursprung. Varifrån namnet här-stammar är likväl höljt i dunkel och flera alternativa förslag finns, alla mer eller mindre sannolika. Ralf Karsten menar i sin bok ”Svensk Bygd i Österbotten II”, att namnet Jungar har ett okänt ursprung. Kan det kanske ha kommit från ”jungare”, som var benämningen på en förskärarkniv eller av ”Junkhärra”, ”junkar”, d.v.s. ett begrepp för ung ädling ?

Vad man i alla fall tror sig veta är att namnet Jungar i äldre tid endast användes som hemmansnamn  eller släktnamn. Det skrevs då Jungaren och det dyker bland annat upp i skrift i en tingsnotis från ett Pedersöreting  13-14 augusti 1606: ”Fältes Marcus Hansson i Jungaren Saak 3 mr för åurkan Han hafr giordt på Michel Olofssons eng i Jäpu”.

På Jungar hemman om 1 mantal var husbonden vid den tiden Hans Larsson (1582-1607), varför det måste ha varit hans son Marcus Hansson som senare tillträdde hem-manet (1608-1627) som hade stått för nämnda åverkan.  Michel Olofsson som nämns som kärande i målet var hemmansägare över Tollikko omfattande 1 mantal (1600-1631).

Namnet har skrivits också på annat sätt, vilket framgår av att det år 1773 i kammararkivet i Stock-holm skrevs ”Ljungo”. Karsten och också andra forskare föreslår att ändelsen -en kan ha sitt ursprung i ändelsen -vin (kort i-ljud), som enligt nordisk eller norsk tradition är en benämning för en gräsrik slätt. Professor Lars Huldén anser dock i likhet med rikssvenska handböcker att belägg för denna ändelse saknas i Finland. Vi får alltså nöja oss med att vi inte vet med säkerhet.

Även om det inte dyker upp i skrift förrän på 1500-1600 -talen, utesluter inte detta att namnet funnits, låt vara i talad form. Vi kan tro att det följt med ända sedan de första personerna för första gången slog sig ner  just här.
