
<--- se KARTA nr 9 --->


\jhhouse{Lindström}{4:118}{Silvast/stoml. Lindström 4:7}{9}{123}


\jhhousepic{142-05684.jpg}{Finskas Daniel}

\jhoccupant{Finskas}{Daniel}{2014--}
Vid generationsskifte 2014 övertog Daniel Bo Alexander Finskas, \textborn	22.04.1989 på Silvast hemman, hemgården Lindström 4:118 på Lövbacken med potatis- och sädesodling, samt kalkonuppfödning samt 50 ha skog. Samtidigt köpte han av sin morfar 150 ha skog och 30 ha odlad jord. Jordbruket omfattar 110 ha odlad jord plus 10 ha arrenderad jord samt skogsbruket 200 ha. Daniel har studerat till elmontör vid yrkesskolan Optima i Jakobstad och till ingenjör inom produktionsekonomi vid Yrkeshögskolan Novia i Vasa.


\jhoccupant{Finskas}{Bo-Göran \& Ann-Sofi}{1980--\allowbreak 2014}
Bo-Göran fick sin lantbruksutbildning vid Korsholms lantmannaskola. Vid generationsskifte på hösten 1979 övertog Bo-Göran Finskas, \textborn 02.10.1957 på Silvast, Lövbacken, hemmanet med mjölkproduktion och potatisodling, 31 ha odlad jord och 80 ha skog. Bo-Göran och Ann-Sofi Lönnvik, \textborn 25.07.1960 på Kaup hemman, gifte sig 22.11.1980.

År 1988 avslutade Bo-Göran och Ann-Sofi mjölkproduktionen. De ändrade om fähuset till ett hus för värphöns för äggproduktion. De specialiserade sig på spannmåls- och potatisodling. År 2001 byggde Bo-Göran ett fjäderfähus för uppfödning av kalkonbroiler. Bostadshuset tillbyggdes och grundrenoverades år 1990.

Bo-Göran ökade under 2000-talet via köp och nyodling den odlade jordarealen till 80 ha plus 40 ha arrendejord, samt via köp av skogslägenheter ökat arealen av skog till 200 ha. När 	barnen blev större studerade Ann-Sofi 1991--\allowbreak 1993 till merkonom vid 	Jakobstads handelsläroverk. Ann-Sofi arbetar sedan 1994 som exportsekreterare på Oy KWH-Mirka Ab.
\begin{jhchildren}
  \item \jhperson{\jhname[Heidi Sofia]{Finskas, Heidi Sofia}}{31.08.1981}{}, pol.mag., Oslo, g.m. Björn Nyman
  \item \jhperson{\jhname[Julia Terese]{Finskas, Julia Terese}}{05.07.1983}{}, ped.mag., Mariefred, sambo Koskinen
  \item \jhperson{\jhname[Lina Charlotte]{Finskas, Lina Charlotte}}{26.07.1985}{}, socionom, sambo Käld
  \item \jhperson{\jhbold{\jhname[Daniel]{Finskas, Daniel}}}{22.04.1989}{}
\end{jhchildren}
Bo-Göran och Ann-Sofi köpte en lokal i Jakobstad, dit de flyttade i juli 2015. Bo-Göran sysslar fortfarande med skogsbruk på 150 ha.


\jhoccupant{Finskas}{Yngve \& Torhild}{1946--\allowbreak 1981}
År 1945 gifte sig Torhild Lindén, \textborn 23.12.1922 på Silvast hemman och Tor Yngve Finskas, \textborn 04.02.1920 på Finskas hemman. Vid generationsskifte 1946 av Lindström skattelägenhet 4:7 av Silvast hemman, övertog Yngve och Torhild en stor del av lägenheten.

År 1946 byggde de sitt bostadshus samt fähus och stall med tillhörande övriga ekonomibyggnader i skogen på Lövbacken. De hade mjölkkor, hästar, får och grisar. Yngve med bror Leo Finskas och svågern Edvin Sandås bildade år 1952 ett bolag och anskaffade sin första traktor med jordbruksredskap, bland annat en bogserad skördetröska. Efter några år köptes en självgående skördetröska. Yngve var en driftig jordbrukare och under årens lopp utökades den odlade jorden samt skogsarealen.
\begin{jhchildren}
  \item \jhperson{\jhname[Birgitta Elisabet]{Finskas, Birgitta Elisabet}}{30.10.1948}{}, biblotikarie, Kista, Sverige, pens.
  \item \jhperson{\jhname[Nils Anders]{Finskas, Nils Anders}}{18.11.1949}{}, agronom, lantbruksrådgivare, Vasa, pens.
  \item \jhperson{\jhname[Berit Viola Susanna]{Finskas, Berit Viola Susanna}}{06.10.1952}{}, hjälpsköterska, Jeppo
  \item \jhperson{\jhbold{\jhname[Bo-Göran]{Finskas, Bo-Göran}} Leander}{02.10.1957}{}, se ovan
\end{jhchildren}
Som pensionärer flyttade Yngve och Torhild till lokal 2 i Bostads Ab Gökbrinken-Asunto Oy på Stationsvägen 17. Yngve dog hastigt där hemma den 06.08.2000, och Torhild 11.02.2015 hemma hos Berit och Magnus på Kaup.



\jhhouse{Väster \& Söder}{4:98 resp. 4:102}{Silvast/stoml. Norrgård 4:8}{9}{124}


\jhhousepic{143-05685.jpg}{Kim Karvonen}

\jhoccupant{Karvonen}{Kim}{2013--}
Ovannämnda fastigheter c:a 4600 m2 med bostadsbyggnad och uthus köptes 31.12.2012 av metallmekaniker Kim Henrik Karvonen, \textborn 12.07.1990 i Jeppo. Säljare var hans mor Maria. Kim arbetar på KWH-Mirka Ab:s serviceverkstad i Jeppo. Kim är varmansförare och intresserad av gamla bilar. År 2013--\allowbreak 2014 byggde han en bilhall mot nordost på tomten.


\jhoccupant{Kula}{Maria}{1996--\allowbreak 2012}
Maria Karvonen, född Kula, \textborn 16.03.1962, övertog år 1996 fastigheterna	Väster 4:98 och Söder 4:102 av sina föräldrar Elof och Ruth Kula. Efter studentexamen utbildade sig Maria till läkarsekreterare och gifte sig med Jari Karvonen, \textborn 1962 i Jakobstad. Jari är arbetsledare på Mirka.
\begin{jhchildren}
  \item \jhperson{\jhname[Jan Niklas]{Kula, Jan Niklas}}{18.04.1982 i Jakobstad}{}, bor i Nykarleby
  \item \jhperson{\jhbold{\jhname[Kim]{Kula, Kim}} Henrik}{12.07.1990 i Jeppo}{}
  \item \jhperson{\jhname[Johanna Maria]{Kula, Johanna Maria}}{22.07.1995 i Jeppo}{}, arb. m kundservice på Mirka
\end{jhchildren}
Sonen Nicklas gifte sig 2008 med Ann Sundqvist 13.05.1982 i Nykarleby. Maria och Jari tog ut skilsmässa år 2008 och Maria har återtagit flicknamnet Kula och bor på Älvliden i en av stadens hyreslokaler.


\jhoccupant{Kula}{Elof \& Ruth}{1968--\allowbreak 1996}
Elof Kula, \textborn 1936 på Kampas, gifte sig 1957 med Ruth Achrén, \textborn 1936 i Vörå.  År 1959 hyrde de på Lavast-backen Sandviks affär, som de bedrev fram till 1962. De flyttade 1962 som hyresgäster till bröderna Sandströms hus på Lövbacken, som de köpte 1968. Elof utbildade sig 1966--\allowbreak 1968 vid Vasa Yrkesskola till överelmontör. Han arbetade därefter på Jeppo Kraft Andelslag fram till pensioneringen. Elofs fritidsintressen har varit jakt och stövarhundar. Ruth hade gått i Vörå Folkhögskola och Borgå Folkakademi. Hon arbetade på Sundells Bageri som butiksbiträde 1969--\allowbreak 1975 och därefter på Fyrens Snabbköp med John Nyman och slutligen på KPO, därifrån hon gick i pension. Ruth var intresserad av musik och spelade och undervisade barn i pianospel. Elof och Ruth grundrenoverade och förstorade bostadshuset 1971--\allowbreak 1973.
\begin{jhchildren}
  \item \jhperson{\jhname[Stefan Nils]{Kula, Stefan Nils}}{31.05.1958 på Lavast}{}
  \item \jhperson{Åsel \jhbold{Maria}}{16.03.1962 på Lavast}{}
\end{jhchildren}
Maria var ½ år vid flyttningen till Lövbacken, där hon bodde fram till 2012. Stefan har studerat vid Åbo Akademi och är rektor för Topeliusgymnasiet i Nykarleby. Han är gift med Greta, född Karlsson 16.09.1958 i Nykarleby. De har tre barn, Simon, Anna och Towe.	När Elofs och Ruths dotter Maria år 1996 övertog fastigheten flyttade de till stadens hyreshus,  Älvliden.  Ruth dog 18.12.2011 bruten av en svår sjukdom.


\jhoccupant{Sandström}{Runar}{1954--\allowbreak 1968}
År 1954 köpte elmontör Isak Runar Sandström, \textborn 1932 på Fors hemman, fastigheten Väster 4:98 av lägenhet Mellangård 4:53. Säljare var brodern Paul Sandström. Brodern Jarl Uno Sandström erhöll rättighet till halva boningshuset att förfoga över. Uno flyttade i december 1957 med familj till Kållby.

Under sin vistelse i Sverige skadades Runar när han föll till marken från en elstolpe. Senare fick han tjänst som elmontör på Vasa flygfält. Runar gifte sig 11.12.1955 med Kaisa Hiekkavirta från Voltti och de fick en son Lars Runar, \textborn 10.05.1956 i Kauhava. Familjen flyttade 02.05.1957 till Oravais. Lars heter numera Lauri Hiekkavirta. När familjen besökte Vilhelm och Rut i Bonäs den 14.06.1972 drunknade Runar i sjön.

\jhbold{Hyresgäst}
Posttjänsteman Marith Lise Lotten Snickars, född Risberg, \textborn 20.07.1930 i Pörtom, och hennes son Mikael Leif Erik Åbacka, \textborn 11.11.1955 i Närpes, hyrde Unos lokal från 1958 till 29.06.1962, då de återvände till Pörtom.


\jhoccupant{Sandström}{Paul \& Uno}{1945--\allowbreak 1954}
År 1945 köpte Paul Sandström, \textborn 1926 på Fors samfällighet, tomten Söder 4:102 av lägenhet Nord 4:55, som utgjorde en del av stomlägenhet Norrgård 4:8 av Silvast hemman 4. Uno Sandström, \textborn 1928, köpte samtidigt granntomten Väster 4:98 av skattelägenhet Mellangård 4:53,som även utgjorde en del av stomlägenhet Norrgård 4:8.  Säljare var föräldrarna Alfred och Ida Sandström, som köpte lägenheten Nord 4:55, år 1925, om 0,0082 mantal av Silvast hemman. Lagfart 09.09.1932. Tidigare ägare Karl-Henrik Storgård. Lägenheten Väster 4:98 köpte de 17.02.1945 av Fredrik Thors.

Paul och Uno byggde tillsammans ett bostadshus med två lokaler av stockar på stående från en ria riven i Oravais. Alfred och Ida samt Runar bodde i den norra lokalen och Paul och Uno i den södra delen av huset. Alfred dog 1951 och senare som änka bodde mor Ida med Paul i den norra lokalen och Uno med familj i den södra lokalen.

Uno gifte sig 1953 med Magdalena Forsman, \textborn 19.02.1930  i Purmo.
\begin{jhchildren}
  \item \jhperson{\jhname[Gun-Maj Beatrice]{Sandström, Gun-Maj Beatrice}}{22.05.1954}{}
  \item \jhperson{\jhname[Helena Margareta]{Sandström, Helena Margareta}}{28.09.1955}{}
  \item \jhperson{\jhname[Christian]{Sandström, Christian}}{29.10.1972}{}
\end{jhchildren}
Flickorna föddes när de bodde på Lövbacken i Jeppo och Christian i Vasa. Uno var stationskarl och i december 1957 flyttade familjen till Kållby. År 1961 erhöll han ett hedersbetyg från SJ och samma år kom han in till konduktörskolan i Helsingfors. Den 13.06.1962 fick han tjänst som konduktör på SJ med möjlighet att välja stationeringsort. Han valde Vasa, där de fortsättningsvis bor.



\jhhouse{Lillström}{4:207}{Silvast/stoml. Norrgård 4:8}{9}{125}


\jhoccupant{Dödsbo}{Lillström}{1961--}
\jhvspace{}


\jhhousepic{141-05683.jpg}{Lillströms hus}

\jhoccupant{Lillström}{Edvin \& Signe}{1938--\allowbreak 1961}
Johannes Edvin Lillström, \textborn 21.11.1898 på Lillas hemman och hans hustru Signe Lovisa Sandberg, \textborn 18.10.1906 i Purmo, köpte den 01.08.1938 ett jordområde på Lövbacken om 0,0010 mantal av stomlägenhet Norrgård 4:49. De var nygifta och byggde ett bostadshus och ett mindre uthus. Edvin planerade att bygga och starta en rävfarm på området, men planen verkställdes inte på grund av att kriget kom emellan. Edvin arbetade i Jakobstad en kortare tid under kriget och på statsjärnvägarnas vedplan i Jeppo invid stationen. Edvin och Signe fick 4 barn.
\begin{jhchildren}
  \item \jhperson{\jhname[Kurt Johannes]{Lillström, Kurt Johannes}}{25.02.1940}{29.12.1981}
  \item \jhperson{\jhname[Per Mattias]{Lillström, Per Mattias}}{25.02.1940}{}
  \item \jhperson{\jhname[Klas Erik]{Lillström, Klas Erik}}{07.11.1941}{}
  \item \jhperson{\jhname[Stig Greger]{Lillström, Stig Greger}}{08.03.1944}{10.02.2012}
\end{jhchildren}

Kurt adopterades av Edvins bror, Viktor Lillström och hans hustru Ellen på Bärs hemman. Kurt gick 1959--\allowbreak 1960	i Lannäslunds jordbruksskola i Jakobstad. År 1964 gifte han sig med Marita Nylund, \textborn 24.11.1942 i Sundby,	Pedersöre. De övertog hemmanet efter Kurts föräldrar Viktor och Ellen och fick sonen Greger, \textborn 02.07.1966, som numera är en driftig jordbrukare på hemmanet. Kurt dog i en tragisk olyckshändelse vid vedcirkeln 29.12.1981.

Per dimitterades från bilavdelningen vid centralyrkesskola i Vasa och har därefter haft tjänst vid Haldin \& Rose i Jakobstad. År 1964 gifte han sig med Gundel Juslin, \textborn	24.11.1942 på Måtar hemman. De har en dotter Pia, \textborn 03.04.1969. Familjen bor i ett egnahemshus i Lövö, Bennäs.

Klas gifte sig med Lisen Björklund från Munsala. De flyttade till Södertälje i Sverige och har tre barn.

Stig blev järnvägstjänsteman i Bennäs. Han gifte sig 1975 med Britta Stoor, \textborn 08.12.1952 på Ryss hemman i Nykarleby lk. De har två barn. Stig dog hastigt den 10.02.2012.

Edvin \textdied 28.09.1961  ---  Signe \textdied 27.07.1996. Bostadshuset står kvar på Lövbacken.



\jhhouse{Stenvall \& Öständ}{4:198 resp. 4:197}{Silvast}{9}{126, 123}
Lägenheterna brutna från stoml. Bäckstrand 4:6 resp. Nygård 4:9


\jhhousepic{140-05682.jpg}{Göran och Gunborg Stenvall}

\jhoccupant{Stenvall}{Göran \& Gunborg}{1971-}
Den 07.04.1971 köpte Göran Edvin och Gunborg Carita Stenvall av Görans föräldrar, Ragnar och Aino Stenvall,  Lövbacken skiftet med bostadsbyggnad och uthus. År 1985 inlöste Göran och Gunborg farbror Åke Stenvalls tomt och byggde där ett maskinstall.

Göran föddes i Jakobstad under kriget, \textborn 16.10.1944 och Gunborg Enström, \textborn 21.12.1948 på Lassila i Jeppo. Vid 16 år reste Göran till Stockholm till sin moster Anne för att söka arbete. Han blev kvar i nästan 10 år. Gunborg blev merkonom 1968 och flyttade i januari 1969 med Göran till Sverige. De gifte sig 1970 i Jeppo och återvände från Sverige 1971.

Göran har efter hemkomsten varit egen företagare inom bilmåleri till år 2004 samt odlat potatis till år 2011. Gunborg var familjedagvårdare 1974--\allowbreak 1986, arbetade på Jeppo Postkontor 1987--\allowbreak 1993 och på Jeppo-Potatis Ab 1994--\allowbreak 2010,då hon pensionerades.

När Göran och Gunborg 1971 övertog bostadshuset hade det stått tomt från hösten 1968 då Ragnar och Aino Stenvall flyttade till Jakobstad. Göran och Gunborg grundrenoverade huset 1975 och förstorade bostadshuset 1985.
Göran och Gunborgs barn är födda på Lövbacken (via BB).
\begin{jhchildren}
  \item \jhperson{\jhname[Carina Anna-Maria]{Stenvall, Carina Anna-Maria}}{19.09.1972}{}, Nykarleby, sjukskötare på Hvc Jstad
  \item \jhperson{\jhname[Matias Lars Göran]{Stenvall, Matias Lars Göran}}{28.02.1975}{}, ekonom, numera Rikander, Nyköping
  \item \jhperson{\jhname[Sofia Linda-Maria]{Stenvall, Sofia Linda-Maria}}{28.03.1982}{}, fil. och ekon.mag., på Herrfors, Jstad
\end{jhchildren}


\jhoccupant{Stenvall}{Ragnar \& Aino}{1946--\allowbreak 1971}
Ragnar Edvin Stenvall, \textborn 06.08.1919 på Nygård 4:9, Silvast, gifte sig 1944 i Jakobstad med affärsbiträdet Aino, född Koukkuluoma, \textborn 11.02.1918 på Silvast. De flyttade efter kriget 1945 från Jakobstad	och byggde åt sig bostadshuset på Lövbacken på en tomt av Nygård 4:9 och Bäckstrand 4:33. De flyttade in i huset 1946.

Den 20.02.1951 vid syskonens försäljning av Nygård 4:9 till brodern Paul, erhöll Ragnar Lövbacka-skiftet, som bestod av delar av Nygård 4:9 och Bäckstrand 4:33. De hade några kor och byggde en ny ladugård 1961 och köpte tilläggsmark. Ragnar arbetade på järnvägen i Jeppo som växelkarl och på stationsmagasinet. På grund av förändringar inom SJ blev Ragnars arbetsplats i Jakobstad och Aino och Ragnar flyttade hösten 1968 dit.
\begin{jhchildren}
  \item \jhperson{\jhname[Göran Edvin]{Stenvall, Göran Edvin}}{16.10.1944 i Jakobstad}{}
  \item \jhperson{\jhname[Eva Katarina]{Stenvall, Eva Katarina}}{30.11.1946 i Jeppo}{18.07.1997 i Stockholm}
\end{jhchildren}
Eva gifte sig med Tapio Lehtinen från Jakobstad och fick sonen Toni Lehtinen 02.10.1966. Efter skilsmässa gifte Eva sig med Jonathan Graham och flyttade till Florida. Eva fick Parkinsons sjukdom och dog 18.07.1997 i Stockholm. Efter kremering blev hon begraven i Jeppo.
Ragnar \textdied 27.06.2004  ---  Aino \textdied 19.07.2011 i Jakobstad. De är begravna i Jeppo.



\jhhouse{Timmerbacken/Nybonde}{3:126 -> 3:150}{Fors}{9}{127, 127a-b}


\jhhousepic{147-05880.jpg}{Patrik och Henriette Fors}

\jhoccupant{Fors}{Patrik \& Henriette}{2003--}
Patrik Anders, \textborn 02.01.1972 i Jeppo, gift med Henriette Kristensen, \textborn 12.01.1977 i Horsens, Danmark.
\begin{jhchildren}
  \item \jhperson{\jhname[Andreas Gustav]{Fors, Andreas Gustav}}{2002 i Kerteminde, Danmark}{}
  \item \jhperson{\jhname[Simeon Eric]{Fors, Simeon Eric}}{2004 i Jeppo}{}
  \item \jhperson{\jhname[Marius Arvid]{Fors, Marius Arvid}}{2007 i Jeppo}{}
  \item \jhperson{\jhname[Iben Christine]{Fors, Iben Christine}}{2010 i Jeppo}{}
  \item \jhperson{\jhname[Kristian Anders]{Fors, Kristian Anders}}{2011 i Jeppo}{}
\end{jhchildren}

Patrik och Henriette träffades i Australien 1997 som deltagare i ett internationellt program för ungdomar. Efter avslutat program bosatte sig Patrik i Danmark 1999, där de gifte sig år 2000. År 2003 flyttade de till Finland och övertog Patriks föräldrars jordbruk, där en av  lägenheterna är ovan nämnda Nybonde 3:126, på vilken bostad och driftcentrum är beläget. Driften fortsätter som specialiserad mjölkproduktion med ca. 60 mjölkkor + ungdjur och med en egen areal på ca 80 ha odlad jord och drygt 100 ha i odling.


\jhoccupant{Fors}{Christer \& Runa}{1970--\allowbreak 2003}

Christer Anders Alfred, \textborn 30.09.1947 i Jeppo, gift 31.10.1970 med Runa Gudrun Karolina Kåll, \textborn 27.08.1946 i Sundby, Pedersöre.
\begin{jhchildren}
  \item \jhperson{\jhbold{\jhname[Patrik]{Fors, Patrik}} Anders}{02.01.1972 i Jeppo}{}
  \item \jhperson{\jhname[Jenny Maria]{Fors, Jenny Maria}}{18.01.1974 i Jeppo}{}, bor i Chicago
  \item \jhperson{\jhname[Magdalena Carolina]{Fors, Magdalena Carolina}}{07.83.1983 i Jeppo}{}, bor i Närpes
  \item \jhperson{\jhname[Annichen Linnea Josefin]{Fors, Annichen Linnea Josefin}}{27.05.1987 i Jeppo}{}, bor i Vasa
\end{jhchildren}

År 1958 hade de tidigare hemmansägarna, Ernst och Agnes Westerlund med familj, emigrerat till U.S.A. De utarrenderade i samband med detta sitt jordbruk till familjen Holger Sandvik, \textborn 1917, och Alice Sundström, \textborn 1930 från Sexsjö i Purmo. År 1966 dog familjefadern och kvar på arrendehemmanet stannade änkan med fem barn och Holgers mor Maria, \textborn 1886, \textdied 1968.
\begin{jhchildren}
  \item \jhperson{\jhname[Eivor]{Sandvik, Eivor}}{1951}{}
  \item \jhperson{\jhname[Max]{Sandvik, Max}}{1952}{}
  \item \jhperson{\jhname[Bo]{Sandvik, Bo}}{1954}{}
  \item \jhperson{\jhname[Bengt]{Sandvik, Bengt}}{1956}{}
  \item \jhperson{\jhname[Sten]{Sandvik, Sten}}{1958}{}
\end{jhchildren}
De fortsatte arbetet på lägenheten med 7 kor och ungdjur och klarade uppgiften med hjälp av släkt och vänner.

År 1970 kom Ernst Westerlund tillbaka till hemlandet i akt och  mening att sälja det mesta av sin egendom. Budet gick i första hand till familjen Sandvik, men de avstod. Uppgiften upplevdes för stor. Mitt i sommaren anordnades en offentlig auktion på gårdsplanen vid Timmerbacken. Som mäklare fungerade poliskonstapel Tauno Stenvik. Christer Fors ropade in den övervägande delen av den fasta egendomen, såsom jord, skog, bostad och ekonomiebyggnader. Den skarpaste konkurrensen uppstod bl.a om den del av den odlade jorden som låg närmast kommuncentrum nära järnvägen, då Jeppo kommun visade intresse av att placera ett kommunalt område för  pälsdjursfarmning där.

Produktionen byggdes nu upp kring en mjölkproduktionsenhet omfattande 60 mjölkkor med utlokaliserad nyrekrytering hos annan jordbrukare, nämligen Runas bror bosatt i Sundby. Besättningen har länge varit en av de största i svenska Österbotten och samtidigt en av de första i nejden som övergick till lösdrift. Efter att i slutet av 1979 ha köpt hemmanet Broända gård, bestående av i huvudsak de båda lägenheterna Broända 3:54 och Strand 3:89 m.fl., av föräldrarna Gustaf och Hildur Fors, flyttades största  delen av detta driftcentrums verksamhet inne i Silvast centrum till Timmerbacken, öster om järnvägen. Det gamla driftcentrumet inklusive tomtmark inne i Silvast, såldes 1995 till Christers bror Fjalar och hans hustru Mayvor Fors och bildade tillsammans med bostadshuset inklusive tomt, Kvarnbacken 3:140 (se Fors  nr 97). Den egna arealen uppgick nu till ca 45 ha odlad jord och har sedan dess avsevärt utökats.

Runa Fors är utbildad hemvårdarinna och har tjänstgjort i såväl Kronoby som Jeppo kommuner. Christer Fors har genomgått två-årig jordbruksutbildning i Korsholms lantbruksskolor och en påbyggnadsutbildning i Odense, Danmark, benämnd LD-tekniker (Lantbruksdiplom tekniker).

Utöver arbetet på det egna jordbruket tillsammans med sin hustru, har han varit engagerad i många förtroendeuppdrag, både nationellt, regionalt och lokalt, bl.a.:
\begin{enumerate}
  \item Styrelseordförande för Centrallaget Enigheten
  \item Styrelseordförande för Mejeriandelslaget Milka
  \item Styrelseordförande för lokalavdelningen av ÖSP
  \item Ordförande för hälsovårdsnämnden i Jeppo
  \item Ordförande för lantbruksnämnden i Nykarleby stad
  \item Medlem i MTK:s Maitovaliokunta. Medverkat vid bildandet.
  \item Medlem i Nykarleby stads hälsovårdsnämnd
  \item Medlem i Nykarleby kyrkliga samfällighets olika organ
  \item Fullmäktigeledamot i Nykarleby församling
  \item Ordförande i Jeppo församlings ekonomisektion
  \item Medlem i SLEF:s lokalavdelnings styrelse i Jeppo
  \item Styrelsemedlem i Jeppo Food, där han medverkat vid bildandet
  \item Styrelsemedlem i Sparbanken Deposita
  \item Medlem i hembygds- och pensionärföreningens styrelse
  \item Revisor i företag och föreningar
\end{enumerate}


\jhoccupant{Westerlund}{Ernst \& Agnes}{1945--\allowbreak 1970}
Efter vinterkriget, hösten 1940, avlider Eric Westerlund, mångårig ordförande för Jeppo Andelsmejeri och far till Ernst Westerlund. Hemgården (nr 396) befann sig då mitt i Silvast centrum och sköttes under fortsättningskriget av änkan Amanda och de hemmavarande barnen.	Först när kriget var slut 1944 och Ernst återvände hem från fronten, kunde han med hustrun Agnes överta hemmanet och på allvar ta itu med framtiden. De gjorde ett radikalt beslut; de tänkte flytta ut från Silvast centrum, över järnvägen och ytterligare 500 m österut, upp till Timmerbacken. Där började de med att bygga en ny ladugård 1945 och året efteråt uppfördes nytt boningshus. Ett nytt driftcentrum med bostad är skapat, borta från byns trängsel.

Emellertid började efter ca 10 år andra influenser göra sig gällande. Amerika började, med hjälp av släktingar, bli alltmer lockande och 1958 tar familjen det slutgiltiga beslutet att emigrera till U.S.A. Som ovan framgår arrenderades hemmanet ut till familjen Sandvik, för att slutligen säljas 1970.
\begin{jhchildren}
  \item \jhperson{\jhname[Greta]{Westerlund, Greta}}{1942}{}, lever i USA
  \item \jhperson{\jhname[Gunnevi]{Westerlund, Gunnevi}}{1947}{}, lever i USA
  \item \jhperson{\jhname[Gun-Britt]{Westerlund, Gun-Britt}}{1949}{}, lever i USA
\end{jhchildren}
Ernst, \textborn 15.11.1913--\textdied 13.12.1992 i USA  ---  Agnes(Lindfors), \textborn 21.01.1914--\textdied 19.11.2001 i USA
Urnorna med stoftet efter både Ernst och Agnes är gravsatta i Jeppo.
