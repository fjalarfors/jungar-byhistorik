
KARTA nr 9 hit --->


\jhhouse{Timmerbacken/Nybonde}{3:126 -> 3:150}{Fors}{9}{127, 127a-b}

\jhoccupant{Fors}{Patrik \& Henriette}{2003 -}
Patrik Anders, \textborn 02.01.1972 i Jeppo, gift med Henriette Kristensen, \textborn 12.01.1977 i Horsens, Danmark.

\jhhousepic{147-05880.jpg}{}

\begin{jhchildren}
  \item \jhperson{Andreas Gustav}{2002 i Kerteminde, Danmark}{}
  \item \jhperson{Simeon Eric}{2004 i Jeppo}{}
  \item \jhperson{Marius Arvid}{2007 i Jeppo}{}
  \item \jhperson{Iben Christine}{2010 i Jeppo}{}
  \item \jhperson{Kristian Anders}{2011 i Jeppo}{}
\end{jhchildren}
Patrik och Henriette träffades i Australien 1997 som deltagare i ett internationellt program för ungdomar. Efter avslutat program bosatte sig Patrik i Danmark 1999, där de gifte sig år 2000. År 2003 flyttade de till Finland och övertog Patriks föräldrars jordbruk, där en av  lägenheterna är ovan nämnda Nybonde 3:126, på vilken bostad och driftcentrum är beläget. Driften fortsätter som specialiserad mjölkproduktion med ca. 60 mjölkkor + ungdjur och med en egen areal på ca 80 ha odlad jord och drygt 100 ha i odling.


\jhoccupant{Fors}{Christer \& Runa}{1970-2003}

Christer Anders Alfred, \textborn 30.09.1947 i Jeppo, gift 31.10.1970 med Runa Gudrun Karolina Kåll, \textborn 27.08.1946 i Sundby, Pedersöre.
\begin{jhchildren}
  \item \jhperson{\jhbold{Patrik} Anders}{02.01.1972 i Jeppo}{}
  \item \jhperson{Jenny Maria}{18.01.1974 i Jeppo}{}, bor i Chicago
  \item \jhperson{Magdalena Carolina}{07.83.1983 i Jeppo}{}, bor i Närpes
  \item \jhperson{Annichen Linnea Josefin}{27.05.1987 i Jeppo}{}, bor i Vasa
\end{jhchildren}

År 1958 hade de tidigare hemmansägarna, Ernst och Agnes Westerlund med familj, emigrerat till U.S.A. De utarrenderade i samband med detta sitt jordbruk till familjen Holger Sandvik, \textborn 1917, och Alice Sundström, \textborn 1930 från Sexsjö i Purmo. År 1966 dog familjefadern och kvar på arrendehemmanet stannade änkan med fem barn och Holgers mor Maria, \textborn 1886, \textdied 1968.
\begin{jhchildren}
  \item \jhperson{Eivor}{1951}{}
  \item \jhperson{Max}{1952}{}
  \item \jhperson{Bo}{1954}{}
  \item \jhperson{Bengt}{1956}{}
  \item \jhperson{Sten}{1958}{}
\end{jhchildren}
De fortsatte arbetet på lägenheten med 7 kor och ungdjur och klarade uppgiften med hjälp av släkt och vänner.

År 1970 kom Ernst Westerlund tillbaka till hemlandet i akt och  mening att sälja det mesta av sin egendom. Budet gick i första hand till familjen Sandvik, men de avstod. Uppgiften upplevdes för stor. Mitt i sommaren anordnades en offentlig auktion på gårdsplanen vid Timmerbacken. Som mäklare fungerade poliskonstapel Tauno Stenvik. Christer Fors ropade in den övervägande delen av den fasta egendomen, såsom jord, skog, bostad och ekonomiebyggnader. Den skarpaste konkurrensen uppstod bl.a om den del av den odlade jorden som låg närmast kommuncentrum nära järnvägen, då Jeppo kommun visade intresse av att placera ett kommunalt område för  pälsdjursfarmning där.

Produktionen byggdes nu upp kring en mjölkproduktionsenhet omfattande 60 mjölkkor med utlokaliserad nyrekrytering hos annan jordbrukare, nämligen Runas bror bosatt i Sundby. Besättningen har länge varit en av de största i svenska Österbotten och samtidigt en av de första i nejden som övergick till lösdrift. Efter att i slutet av 1979 ha köpt hemmanet Broända gård, bestående av i huvudsak de båda lägenheterna Broända 3:54 och Strand 3:89 m.fl., av föräldrarna Gustaf och Hildur Fors, flyttades största  delen av detta driftcentrums verksamhet inne i Silvast centrum till Timmerbacken, öster om järnvägen. Det gamla driftcentrumet inklusive tomtmark inne i Silvast, såldes 1995 till Christers bror Fjalar och hans hustru Mayvor Fors och bildade tillsammans med bostadshuset inklusive tomt, Kvarnbacken 3:140 (se Fors  nr 97). Den egna arealen uppgick nu till ca 45 ha odlad jord och har sedan dess avsevärt utökats.

Runa Fors är utbildad hemvårdarinna och har tjänstgjort i såväl Kronoby som Jeppo kommuner. Christer Fors har genomgått två-årig jordbruksutbildning i Korsholms lantbruksskolor och en påbyggnadsutbildning i Odense, Danmark, benämnd LD-tekniker(Lantbruksdiplom tekniker).

Utöver arbetet på det egna jordbruket tillsammans med sin hustru, har han varit engagerad i många förtroendeuppdrag, både nationellt, regionalt och lokalt, bl.a.:
\begin{enumerate}
  \item Styrelseordförande för Centrallaget Enigheten
  \item Styrelseordförande för Mejeriandelslaget Milka
  \item Styrelseordförande för lokalavdelningen av ÖSP
  \item Ordförande för hälsovårdsnämnden i Jeppo
  \item Ordförande för lantbruksnämnden i Nykarleby stad
  \item Medlem i MTK:s Maitovaliokunta. Medverkat vid bildandet.
  \item Medlem i Nykarleby stads hälsovårdsnämnd
  \item Medlem i Nykarleby kyrkliga samfällighets olika organ
  \item Fullmäktigeledamot i Nykarleby församling
  \item Ordförande i Jeppo församlings ekonomisektion
  \item Medlem i SLEF:s lokalavdelnings styrelse i Jeppo
  \item Styrelsemedlem i Jeppo Food, där han medverkat vid bildandet
  \item Styrelsemedlem i Sparbanken Deposita
  \item Medlem i hembygds- och pensionärföreningens styrelse
  \item Revisor i företag och föreningar
\end{enumerate}


\jhoccupant{Westerlund}{Ernst \& Agnes}{1945-1970}
Efter vinterkriget, hösten 1940, avlider Eric Westerlund, mångårig ordförande för Jeppo Andelsmejeri och far till Ernst Westerlund. Hemgården (nr 396) befann sig då mitt i Silvast centrum och sköttes under fortsättningskriget av änkan Amanda och de hemmavarande barnen.	Först när kriget var slut 1944 och Ernst återvände hem från fronten, kunde han med hustrun Agnes överta hemmanet och på allvar ta itu med framtiden. De gjorde ett radikalt beslut; de tänkte flytta ut från Silvast centrum, över järnvägen och ytterligare 500 m österut, upp till Timmerbacken. Där började de med att bygga en ny ladugård 1945 och året efteråt uppfördes nytt boningshus. Ett nytt driftcentrum med bostad är skapat, borta från byns trängsel.

Emellertid började efter ca 10 år andra influenser göra sig gällande. Amerika började, med hjälp av släktingar, bli alltmer lockande och 1958 tar familjen det slutgiltiga beslutet att emigrera till U.S.A. Som ovan framgår arrenderades hemmanet ut till familjen Sandvik, för att slutligen säljas 1970.
\begin{jhchildren}
  \item \jhperson{Greta}{1942}{}, lever i USA
  \item \jhperson{Gunnevi}{1947}{}, lever i USA
  \item \jhperson{Gun-Britt}{1949}{}, lever i USA
\end{jhchildren}
Ernst, \textborn 15.11.1913 - \textdied 13.12.1992 i USA
Agnes(Lindfors), \textborn 21.01.1914 - \textdied 19.11.2001 i USA
Urnorna med stoftet efter både Ernst och Agnes är gravsatta i Jeppo.
