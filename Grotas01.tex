
\jhchapter{Grötas, hemman Nr 5}

Ohls menar att namnet Grötas har ett fornnordiskt ursprung. Den första delen utgörs av grjöt \= sten. Den andra delen är a \= å. Grötas skulle alltså betyda ``stenån''. Också Karsten har samma uppfattning; Grötas kommer från gröt, gröut (stamform graut) \= massa av småsten. Det låter sig sägas, som så ofta när det gäller ortnamn. Frågan inställer sig varför vi inte finner detta namn på flera ställen längs ån där det finns massor av stenansamlingar? Det är frestande att i sammanhanget återge allmogens uppfattningar om ursprunget till namnet, sannolikt tillkomna efter Stora ofreden, berättade med glimten i ögat;
\jhbold{a)} Folket från Grötas var ivriga jägare och var flera dagar borta och jagade. När de kommo hem, plägade de säga: ``Nu skall vi ha oss något lite gröt'' och så fingo de namnet Grötas.
\jhbold{b)} En gång gick gubben på Ojas upp längs åliden och kom till Grötas. Där kokade man just gröt, och därav gav han namnet Grötas åt gården.

Uppfattningen att Grötas är den äldsta bosättningen i Jeppo härstammar från samma sägner. Enligt dessa ``började den första svensken i Jeppo bo på Grötas, det äldsta stället i Jeppo. Han var jägare. På den nummern har därefter alltid bott jägare. Byggningen stod på åbacken. En dag såg karlen en vit tvet (huggspån) komma drivande längs ån. Han gick då uppåt för att söka sin granne och träffade honom vid Heikkilä i Alahärmä åtta km högre upp. Han klandrade ändå den andre för att han hade byggt sig för nära!'' Var och en får avgöra sanningshalten i dessa berättelser, men av någon anledning fick Grötas hemman nummer 1 (ett) när storskiftet genomfördes i slutet av 1700-talet. På initiativ av överdirektören i den kungliga lantmäteristyrelsen, Jacob Faggot, beslöt nämligen riksdagen 1755-56 att ett storskifte skulle genomföras i Finland för att förbättra jordbrukets produktionsmöjligheter. En mätning och kartläggning måste ligga till grund för denna process och den genomfördes redan 1740 av bl.a. Matthias Wörgren (se Skog).

År 1783 ägdes Grötas av Mats 21/64 mtl, Daniel 7/32 mtl och  ..?. 7/32 mtl., men innan dess av bl.a. Sigfrid Persson 1592--\allowbreak 1625, Erik Sigfridsson 1625--\allowbreak 1653, Sigfrid Eriksson  1657--\allowbreak 1673, Erik Sigfridsson med hu. Lisa Tomasdr. 1675--\allowbreak 1694, sonen  Anders Eriksson med hu. Karin Andersson 1695--\allowbreak 1697, svågern Erik Markusson med hu. Brita Eriksdr. 1698--\allowbreak 1713 och som medåbo dennes svåger Mickel Hansson med hu. I Elsa och hu. II Gertrud från 1699.

Vägen genom Grötas och söderut gick tidigare längre österut. Halvvägs mellan ån och dagens järnväg mynnade den ut till dagens vägsträckning strax före Bösas bäck. Längs denna väg växte bebyggelsen fram. Åsen kallades då `Furukangan'', beväxt med ståtlig tallskog därifrån timmer till Nykarleby nya kyrka, invigd 1:a advent 1708, sägs ha avverkats och flottats ner längs älven. Sannolikt fanns ändå den allra första vägen längs åkanten då skogsbrynet inte låg långt borta. Grötas var i medlet av 1800-talet det folkrikaste hemmanet i Jungar by med ca 120 personer. Grötas har också i något sammanhang haft en relation till staden Nykarleby då på dess mark finns ett område kallat ``stadsbohagen''. Kanhända har det sin bakgrund i den landslag av år 1442 där det stadgades, att taverner eller gästgiverier skulle inrättas vid allmänna vägar på två, eller högst två och en halv mils avstånd från varandra. Redan under första delen av 1500-talet var allmogen i Österbotten underkastade detta åtagande och oftast var det herremän som skulle skjutsas. De som reste i konungens och befallningsmans ärenden, skulle få ``mat och öl'' och ``bliva befordrade med skjutshästar och andra skäliga nödtorfter gratis''. Att vägra ledde till åtal. och böter.  I vilket fall anses Grötas vara det hemman som tidigast härbärgerade ett gästgiveri öster om älven, och det omnämns redan 1635. Den troliga platsen var i närheten av där Ahlfors hemman finns idag. Förmodligen fanns en gång i tiden ett område, nu på järnvägens östra sida, där stadens befolkning vid behov kunde rasta sina hästar under resa, d.v.s. Stadsbohagen.

På Grötas stod också soldattorp 139, som upprätthölls av roten; Lawast, Lillsilfvast, Grötas och Rämäcki. Det har genom tider kallats Långängen, Langängen och Lawast (se mera under Soldattorp Nr 139 Grötas).

Insprängt i Grötas geografiska område finns också delar av hemmanen Holm, Skog, Romar, Fors och  Silvast. Efter 1831 fick Grötas nummer 5 och har tydligen alltid varit ett skattehemman och till ingen del ett kronohemman (Kronhemman \= ägt, uthyrt av kronan).

Idag finns också den tätaste bebyggelsen i Jeppo på Grötas hemman. Det är det  markområde som i början av 1970-talet köptes för att i första hand erbjuda ett expanderande Mirka en ny industritomt. Så blev det nu inte och istället planerades området väster om landsvägen till ett kommunalt bostadsområde för såväl egnahem som radhus och detsamma skedde med området öster om landsvägen. Området gränsar i norr till Fors hemman.

På 1950-talet fanns 8 aktiva jordbrukare på Grötas hemman. Idag finns endast en (1).

Grötas hemman omfattas av kartorna \jhbold{nr 10-11}.


<--- se KARTA nr 10 --->


\jhsubsection{Lägenheter på Grötas}



\jhhouse{Soldattorp Nr 139}{5:147}{Grötas}{10}{777, 777b}
Detta torp upprätthölls av soldatroten bestående av hemmanen Lawast R:nr 10, Lillsilfwast R:nr 25, Grötas R:nr 1 och Rämäcki R:nr 17. Det utgjorde vad som då stadgades av 4 ¼ rotemantal. Under olika tider har det benämnts Långängen år 1737, Langängen 1752 och Lafwast från 1775.

Själva torpet fanns på Grötas hemmans mark, sannolikt alldeles norr om Bror och Siv Westins hus idag. Torpets åker, ``Kliftens lindor'', fanns ett stycke norrut på Lillsilfwast mark (Fors) och är nu till stor del under asfalt. Dels under vägen mot Oravais, dels under tennisplanen och konditionshallen.

Det är oklart om roten, som skulle uppgå till 4 ¼ rmtl för att kunna försörja en soldatfamilj, vid introduktionen av indelningsverket alltid byggde upp ett nytt torp eller om gamla stugor kunde tas i användning. Påfallande ofta förefaller det som om torpens skick, vid den syn som hölls 15 år efter starten, var dåligt. Oftast var de nedersta stockvarven ruttna och fönstren blyfallna. Vid syneförrättning 1752 hade också detta torps stuga rivits ner och byggts upp på nytt med färska nedre stockvarv. Försett med ett nytt skurtak ansågs den ``i stånd till alla delar''. Samma vår hade rotesällarna byggt ett nytt fähus med nytt tak, medan ladans takved var rutten och taket läckte. Det lovade roten åtgärda, likaså en ny farstu mellan fähus och lada. Boden fann nåd som varande i försvarligt skick och låsförsedd.

Åkern fanns för hela roten och varande i fullt bruk. På Lafwast var en gärdesgård i dåligt skick. Ängarna som skulle ge foder till husdjuren var delvis ``av inttet wärde'' och kompenserades istället med ett stycke skattejord till dess en större äng hunnit röjas och iståndsättas. Vid nästa antecknade syn år 1791 var husen och uthusen i gott skick. Åkern omfattande 1 tunnland fanns tillgänglig för torpet, likaså ett 2 kappland stort kålland. Ängarna ``wälhäfdade'' och ansågs kunna avkasta stipulerade 12 skrindor hö. En ny brunn skulle grävas under hösten. Tydligen ansågs åvattnet redan då olämpligt att dricka för folk, även om det fanns nära till hands.

Klift var ett gångbart namn på de första knektarna som innehade  torpet:

\begin{enumerate}
  \item Anders Andersson Klift, antagen 1713. Ankom ca 1734. Kvarstod till 1741. Befordrad till korpral N:o 1.
  \item Matts Stephansson Klift, 1742 – 1743. Död hemma på roten efter att ha kommit hem svårt misshandlad under fångenskap i krigets slutskede och där ådragit sig svår diarré.
  \item Matts Eliasson Klift, 1745 – 1758. Död i Pommern.
  \item Hans Larsson Klift, 1758 – 1779. Föravskedad. Rymt?
  \item Johan Persson Klift, 1779 – 1789. Transport till Nr 131 i Ytterjeppo.
  \item Gustaf Mattsson Torn, 1789 – 1791. Byte till Nr 113 i Munsala.
  \item Jakob Johansson Högbom, 1791 – 1804. Död hemma på roten.
  \item Isaac Henricsson Skön,  1805 – 1808 ? Stupade i kriget.
\end{enumerate}

Efter fredsslutet i Fredrikshamn 17.09.1809 höll ryssarna kvar militär i Finland ända till 1881, då landet fick egen militär. Inkvarteringen blev nu istället för knekthållet en börda för bönderna i de socknar, där de ryska soldaterna var förlagda. Husrum skulle upplåtas för officerare och manskap. Stall skulle hänvisas för deras hästar och betesmarker sommartid. Ved skulle fram till manskapet och exercis- och lägerplats för truppernas behov upplåtas. Huruvida dessa pålagor drabbade Jeppobygden är inte dokumenterade. Däremot tvingades allmogen att underhålla de gamla torpen, vilket väckte motstånd. Det sista torpet avvecklades i Jeppo först 1909. Det var torp Nr 137 , ``And'as torp''.

När indelningsverket avvecklades 1810 fick officerare och även underofficerarna bibehålla sina boställen och sin lön. Detta gällde bland annat Kamppinens boställe. Men soldaterna miste sina torp och massor av invalider, soldater, soldatänkor och uttjänta soldater fick t.o.m. ta till tiggarstaven. Hård var deras lott.



\jhhouse{Strengell, inkl. hyresgäster}{5:105}{Grötas}{10}{3}


\jhhousepic{109-05646.jpg}{Elvi Strengells hus}

\jhoccupant{Strengell}{Elvi}{1984--}
Elvi Strengell, \textborn 25.08.1939 på Jungar hemman, lät bygga denna gård år 1984. Huset beställdes från Puutalo, som drevs av brodern Olof i Jakobstad. Bröderna Olof och Ruben byggde huset. Huset är byggt på stadens arrendetomt.

Elvi, som bor i hemgården (Jungar 7), hyr ut huset. Bl.a dessa hyresgäster har bott i gården:
\begin{enumerate}
  \item Jennifer Vikström, \textborn 1993 i Lappfors, hyrt huset sedan 2014. Hon har tidigare arbetat inom hemservice, arbetar nu på Florahemmet i Nykarleby.
  \item Max Ström, \textborn 1958
  \item Kai Nyberg, \textborn 1969 (Grötas 6)
  \item Vesa Niskanen, \textborn 1944, restaurangkock, Gunnevi Granbacka, \textborn 1941, pensionär (Romar 26)
  \item Regina  och Jarkko Jaskari med barn (Fors 101)
  \item Familjen Ari, Sorjonen, \textborn 1960 och Riitta Marjanen, \textborn 1963, (fr. Silvast nr 59/lokal 3, KPO)
\end{enumerate}



\jhhouse{Norrhagen, inkl. boende}{5:94, 5:105}{Grötas}{10}{4a, 4b, 4c}


\jhhousepic{118-05655.jpg}{Lövgränd 4a}

Bostads Ab Jeppo Lövgränd var det första blandhuset i Nykarleby, dvs där finns både hyresbostäder, statsbelånade bostäder och hårdvalutabostäder. Bostäderna var inflyttningsklara i slutet på september 1985. Namn på alla hyresgäster har inte kunnat fås.\jhvspace{}

\jhhousepic{117-05654.JPG}{Lövgränd 4b}

\jhvspace[10]

\begin{center}
  \begin{longtable}{l p{0.8\textwidth}}
    \hline
    4a, bostad 1 & Gustav Julin, f. 1938, hustru Gudrun, f. Sandin 1935, flyttat in 1995. Det var meningen att Gustav skulle överta  hemgårdens hemman på Slangar. Som ung körde han Jeppo Grävs traktor Fordson Major med dikesplog under 2 år, ett arbete som ofta skedde i senhöstens regn och kyla, utan hytt att sitta i. Sådana fanns inte under tidigt 1950-tal. I febr. 1957 råkade han ut för en arbetsolycka i Utskogen, som gjorde att han låg medvetslös en vecka.  Olyckan ändrade fullständigt framtidsplanerna.

    Han fick år 1961 anställning på Schaumans fabrik i virkets inkommande tvätteri, en mycket fuktig och ohälsosam miljö, som han måste lämna efter ett år. Därefter fungerade han som försäkringsinspektör fram till aug. 1965.

    Vid återkomsten till Jeppo köpte makarna Gunnar och Nanna Holms gård på Holmen och familjen flyttade in efter slutförd renovering. Gustav var därefter anställd på Mirka fram till sjukpensioneringen 1971.

    Gustav har haft många förtroendeuppdrag:

    -4H-ledare i slutet av 1960-talet i 3 år

    -Medlem i brandnämnden,dess ordf. efter kommunsammanslagningen fram t. 1991

    -Medlem i stadsstyrelsen under 14 år, åren 1987-2001

    -Ordf. i Pensionärsföreningen under 4 år

    -Ordf. i Jakobstadsnejdens reumaförening under 7 år

    -Medlem i Nykarleby stads handikapråd från starten

    -Medlem i styrelsen för Svenska Hörselskadade i Nykarlebynejden

    -Medlem i styrelsen för Jeppo lokalavd. av SFP

    -Ordf. i bostadsbolaget sedan 1995

    Gudrun Julin var en tidig initiativtagare till öppnandet av en barnträdgård i Jeppo. Det skedde 1972 i f.d. Jungar skola. Hon skötte bl.a. mattransporterna mellan skolans kök och barnträdgården. Åren 1984 -1995 fungerade hon som personal i det nya daghemmet och var också kökshjälpsbiträde vid skolans kök 1986 -90. --- Stig och Inger Lindgren (Ruotsala 13), köpte lägenhet 1986 \\
    4a, bostad 2 & Pirkko Bro, f. 1950 köpt och flyttat in i lägenheten 1986 \\
    4a, bostad 3 & Hans Vesterholm,  2014  - (Tapani Ikola, Lappo äger lägenheten). -- Olav Skoglund. --	Erkki Hirvi, Kälviä, ägare 2003--\allowbreak 2007. --	Jari Kalijärvi, f. 1966, 1986 – 2003 (Romar 29) \\
    4b, bostad 4 & Ingmar och Sally Dahlström (Romar 35), flyttat in 1997. -- Gunfrid Kuoppala, f. Enlund, packare, sambo med Erkki Kuoppala, f. 1940, timmerman. Barn: Pia, f. 1975. Gunfrid till bostad 8 år 1997. \\
    4b, bostad 5 & Christina Liljeqvist, f. 1955 i Petalax, arbetar på Mirka, flyttat in 2000. Nykarleby bostäder ägare t.o.m 2006. hyresgäster bl.a: Martin och Aldi Elenius, 1992--\allowbreak 1999 (Mietala 12) \\
    4b, bostad 6 & Nykarleby Bostäder äger lägenheten, hyresgäster bl.a: Abdi Ahmed Muhamed flyttat in i maj 2017. -- Ralf Broman 2017. -- Tina Jaskari 2014, från Fors 101. -- Nina Särkiniemi 2013, från Ruotsala 30. -- Gurli Lindén och Eva-Stina Byggmästar, författare från Sverige 2004--2006, till Silvast 75--76 -- Erland Sundell (Silvast 74) \\
    4c, bostad 7 & Nykarleby Bostäder, hyresgäster bl.a: Toivo Kalliosaari, f. 1930 i Alahärmä, hustrun Aino, f. Hyyppä dog 26.11.2008. Se närmare Romar gård 328. -- Zaida Nyström, f. 24.01.1908 flyttade som pensionär till Jeppo. Hon hade arbetat på hotell i Jakobstad. Hon dog 08.02.2000. \\
    4c, bostad 8 & Nykarleby Bostäder är ägare till lägenheten. Gunfrid Enlund, f. 20.01.1935 hyrt sedan 1997. -- Ingmar och Sally Dahlström (Romar 35), 1996--\allowbreak 1997. -- Nina Nyberg, (Grötas 122).	-- Astrid Viklund (1909--\allowbreak 1992), hyrde lägenheten några år. \\
    4c, bostad 9 & Kim Häggblom, f. 1986 i Pensala, gift med Daria, f. Melnikova 1991. Kim köpte lägenheten  2011. Han arbetar på Mirka, Daria studerande. -- Matias Harald, bodde i lägenheten före Häggbloms. -- Thomas och Johanna Sandberg, åren 2000--\allowbreak 2003.	-- Veli-Pekka Aho, f. 1973, elmontör, hustru Annu, f. Leppälä 1972, konditionsskötare bodde här 1995  (Silvast 63 ). -- Stig-Johan och Johanna Back. -- Jarmo och Marina Saviaro (Grötas 23). -- Lindén Christoffer. -- Saima Romar,  1986 – 1993 (Romar 12). -- Tidigare ägare  Nykarleby Bostäder. \\
    \hline
  \end{longtable}
\end{center}



\jhhouse{Norrhagen/Rosas, inkl. boende}{5:94, 5:104}{Grötas}{10}{5a, 5b}


\jhhousepic{114-05651.jpg}{Åkervägen}

\jhoccupant{Bostads Ab}{Jeppo Åkervägen}{1978--}
Bostads Ab Jeppo Åkervägen består av 2 radhus, som byggdes år 1978 av Byggnadsbyrå Kaustinen \& Kattil. Det var svårt att få köpare till aktielägenheterna och hösten 1977 löser Nykarleby stad in tre lägenheter. Endast tre lägenheter var då sålda och bygget skulle inte ha kunnat påbörjas. Staden köpte lägenheterna för pengar som  i budgeten var reserverade för pensionärsbostäder, ett projekt som man inte fått bostadslån till. KWH-koncernen reserverade också lägenheter för sina arbetstagare. Nedan situationen 2017; uppgifter om alla hyresgäster har inte kunnat fås.

\begin{center}
  \begin{longtable}{l p{0.8\textwidth}}
    \hline
    5\jhbold{a}   Bostad 1 & Leif Häggblom (Gunnar 34), ägare, hyr ut lägenheten åt -- Linda Sjö och Hans Ekroth (Silvast 75--76). -- Paul Häggblom, 1986--\allowbreak 2015 (Gunnar 32). -- Karl Gustav och Anne-Kristine Häggblom, 1984--\allowbreak 1986, till Gunnar 32. -- Barbro Back med barn, 1978--\allowbreak 1984 \\
    5a  Bostad 2 & Erik och Hilja Dahlström, från Jungarå nr 22, år 2008. -- André och Helena Norrback samt dottern Amanda år 2005--\allowbreak 2007. -- Herbert och Gunvor Eklund, 1993--\allowbreak 2001 (barnen nuv. ägare). -- Erik Holm och Jenny Ylinen, 1981--\allowbreak 1992. -- Familjen Dackemo från Danmark t.o.m 1981. -- Olav, Gunilla och Kim Jungarå 1979--\allowbreak 1980 (Jungarå 21). -- Nykarleby stad var ägare åren 1978--\allowbreak 1981, 1992--\allowbreak 1993. \\
    5a Bostad 3 & Erik och Ann-Britt Elenius, köpt 2009, flyttar in i december 2016. -- Anders och Charlotta Elenius, flyttar in 2009, 2016 till Tollikko 7. -- Ronny Norrgård med familj bodde till 2009, flyttade till Holmen. -- Judit Sarelin, \textborn 23.12.1923, \textdied 21.02.2017, på Åkervägen 2002--\allowbreak 2008. -- Familjen Guy och Ann-Maj Häger, 1988--\allowbreak 2002, till Stenbacken. -- Heikki Wakkuri. -- Familjen Saunala, ca 1984. -- Familjen Sven-Erik och Vailet Furu från Sverige. -- Oravais fabrik, senare KWH-koncernen Ab, ägare. \\
    5a  Bostad 41 & Stig Forsgård, bokförare och ansvarig byråchef på Norlic i Jeppo köpt lägenhet 1988, då han samt föräldrarna Kurt och Ragni flyttar in i lägenheten. Ragni dog 02.12.2014. (Mietala 11). -- Per-Erik och Åsa Forsgård köpte lägenheten 1983     (Mietala 11). -- Bo-Gustav och Gunnel Nybyggar, 1978 – 1984, arbetade på Mirka \\
    5\jhbold{b} Bostad 5 & Gunnel Elenius köpt och flyttat in år 2012 (Gunnar 19). -- Stig och Solveig Grönlund blev ägare år 2008. Lägenheten uthyrdes kortare perioder. -- Ragnhild och Åke Grönlund blev ägare och flyttade in 1993. -- Edna Eklöv (Gunnar 33) köpte lägenheten men flyttade in efter moderns död, år 1984. Edna hade hyrt ut lägenheten till Helga Kula, \textborn 1909, som då flyttade till Pastellen. (Grötas 6) \\
    5b  Bostad 6 & Rune Elenius, \textborn 1950, t. slippappersarbetare vid Mirka, fr.o.m  2001. -- Leo och Helmi Elenius (Tollikko 11) köpte lägenheten 1978, men hyrde ut den de första åren till Selma Grahn, \textborn 1899, \textdied 1983. \\
    5b Bostad 7 & Yngve och Berit Forsgård flyttade in 1996, ägare sedan 1991. Yngve dog 10.10.2013. (Mietala 1) -- Kjell och senare Kjell och Annette Forsgård, 1991--\allowbreak 1996. -- Jorma Hankonen, 1988--\allowbreak 1991. -- Bo Andersson och Marlene Nylund. Bo arbetade på Mirka, Marlene inom hemservice, 1982--\allowbreak 1988. -- Christer och Harriet Jungerstam, 1979--\allowbreak 1981 (Jungar 28). -- Nykarleby stad köpte denna lägenhet. \\
    5b  Bostad 8 & Inga Ljung köpte lägenheten av KWH-koncernen 1996 (Skog nr 1). -- Veli-Pekka Aho med familj (Silvast 63). -- Siv och Bror Vestin, 1994--\allowbreak 1996 (Grötas 7). -- Mats Kock med familj, flyttade till Nykarleby. -- Nils Ahlvik med familj (juli 1984-aug 1988). -- Granbäck Yvonne. -- Familjen Bäckman. -- KWH-koncernen köpte lägenheten, hyrde ut till arbetstagare. \\
    5b Bostad 9 & Christian Sjölind, Kronoby, köpt lägenheten 2016, hyr ut fr.o.m 2017. -- Maria Mitrakova, 2013--\allowbreak 2016, med  sonen Yuri. -- Matti och Elisabeth Suurholma ägare sedan 2006. Hyresgäster bl.a. -- Maria Viklund. -- Matts Norrgård (Romar 15). -- Ines Elenius, 1978--\allowbreak 2002(Gunnar 26). \\
    5b Bostad 10 & Barbro Mushendwa, \textborn 10.08.1943 på Finskas, vigd 24.06.1978 med Joshua Mushendwa, \textborn 19.01.1944 i Bukaba i Tanzania Barn: Siima,\textborn 25.03.1979, Sune, \textborn 23.08.1980, Eric, \textborn 23.01.1984. Barbro har varit i Finska missionens tjänst och kom till Tanzania som missionär 1973. Joshua har studerat till arkitekt i Danmark. I mitten av 80-talet startade Barbro och maken Joshua en egen resebyrå, senare skötte Barbro ensam resebyrån efter att Joshua öppnat ett eget arkitektkontor. Familjen har sitt hem i Arusha i Tanzania. Lägenheten används främst som extra hem då Barbro besöker hemlandet. -- Ralf Stenvik, ägare efter Tuovi Salo. Lägenheten stod tom många år. -- Tuovi Salo, f. Lahti flyttade till Jeppo då hon blev pensionär. \\
    5b Bostad 11 & Olav Jungarå, ägare sedan 2014, hyr ut lägenheten till Tobias Elenius, \textborn 1993, arbetar på jordbrukslägenhet, Sandra Vikström, \textborn 1995, merkonom. Tobias och Sandra flyttat in 2015. -- Stig, Greta och Inga Ljung samt Lisa Teir-Siltanen köpte lägenheten 1978, dit Ellen Ljung (Skog, 1) flyttade. Efter Ellens död 28.05.1997 hyrdes lägenheten ut fr.o.m 2004 till Ernst Lindvall och fr.o.m 2007 till Jani Nikkanen. \\
    \hline
  \end{longtable}
\end{center}



\jhhouse{Norrhagen/Rosas, inkl. boende}{5:94, 5:104}{Grötas}{10}{6a, 6b}


\jhhousepic{135-05673.jpg}{Pastellen}

\jhoccupant{Fastighets Ab}{Jeppo Pastellen}{}
Åkervägen 9 Fastighets Ab Jeppo Pastellen består av två radhus med hyresbostäder. Husen byggdes år 1982. Nykarleby bostäder är ägare idag. Några av nuvarande och tidigare hyresgäster presenteras nedan:\jhvspace{}

\begin{center}
  \begin{longtable}{l p{0.8\textwidth}}
    \hline
    6\jhbold{a}, bostad 1 & Maj-Lis, \textborn 15.12.1937 och Eino, \textborn 04.08.1934, Simanainen flyttade in i september 1996. Eino dog 05.11.2012. Maj-Lis bor ensam kvar. -- Valter Backlund, på 1990-talet. -- Gunnel Elenius-Laitala, 1989, (Gunnar 19).	-- Annie Forss, 1982--\allowbreak 1988 (Fors 108).  \\
    6a, bostad 2 & Kai Nyberg, \textborn 1969, arbete Mirka. -- Hugo och Linnea Saha. -- Jens Björklund, taxichaufför. -- Gerda Perus, 1989--\allowbreak 1994 (Jungar 211). \\
    6a, bostad 3 & Eve Juurma, \textborn 1967 och Janis Mitt, \textborn 1972 i Estland, flyttat in 2016. -- Jaskari Tomi. -- Pennanen Salme. -- Tuula Määttä och Reijo Niiranen. -- Pia Kuoppala. \\
    6a, bostad 4 & Lägenheten tom i maj 2017. -- Tanja Ylenius, från Kangasala, 2017. -- Erik och Ann-Britt Elenius, 2014--\allowbreak 2016, till Grötas 5a. -- Familjen Jonni och Jasmin Aalto (Skog 4). -- Saviaro Satu  (Grötas 19). -- Familjen Veli-Pekka Aho (Silvast 63). -- Siv Källman och Kristian Hautanen 1982--\allowbreak 1994 (Ruotsala 15). \\
    6a, bostad 5 & Jaana Leppävuori, \textborn 1961 i Vetil, flyttat in i april 2015. -- Jan-Ola och Maret Elenius (Jungarå). -- Siv och Bror Vestin, 1991--\allowbreak 1994, flyttade till Åkervägen 7. -- Lea Jungerstam, från Ruotsala 25. -- Thomas och Lilian Lindén 1982--. \\
    6\jhbold{b}, bostad 1 & Stina Österlund, \textborn 1992, flyttat in i januari 2017, arbetar på JEPO. -- Jungell. -- Raija Houtsonen. \\
    6b, bostad 2 & Alice Sandvik, flyttat in 2011, från Silvast 59. -- Tomi Hautamäki, 2004--\allowbreak 2005 (Mietala 16). \\
    6b, bostad 3 & Hautanen Kristian, flyttat in 2004. \\
    6b, bostad 4 & Lägenheten tom i maj 2017.	-- Rand, född i Estland. -- Sorjonen Jonna. -- Markus Laitala, 2006--\allowbreak 2008 (Mietala 19). -- Anni Julin. \\
    6b, bostad 5 & Mati Piht, \textborn 1968 i Estland, 2016--. -- Selma Kula, från Åkervägen 7. -- Kurt Hongisto med familj, 1987--1989 (Silvast 92). \\
    6b, bostad 6 & Ellinor Sarin, \textborn 1993, flyttat in 2016. -- Rurik Broo, (Böös 51). -- Siri Finskas. -- Hjördis Sandås (Böös 45). \\
    \hline
  \end{longtable}
\end{center}



\jhhouse{Kipari}{5:}{Grötas}{10}{100}


\jhoccupant{Kipari}{Antti \& Eeli}{1931--\allowbreak 1937}
Bröderna Antti, \textborn 22.01.1907 i Jeppo, och Eeli, \textborn 30.11.1909 i Sievi, flyttade år 1931 tillbaka till Jeppo till den gård, där familjen tidigare bott. I januari 1937 flyttade Antti till Pedersöre, Eeli Edvin flyttade samtidigt till Gamlakarleby.

Som ombud för sina bröder ansöker systern Martta år 1936 om att få lösa in området som Emil Björkqvist varit ägare till, men området tillhörde nu Johannes Westin. P.g.a avsaknad av skriftligt kontrakt samt andra oklarheter kunde ansökan inte beviljas. Bland legonämndprotokollen finns dock en ansökan om inlösen av området gjord i augusti 1927 av Antti Kipari, men ingen behandling av ärendet hittas.
Finns inga uppgifter om stugan stod obebodd åren 1913 – 1931.


\jhoccupant{Kipari}{Antti \& Aina}{1897--\allowbreak 1913}
Antti (Anders), \textborn 04.09.1859 i Kortesjärvi, gift med Aina Maria Johansdr, \textborn 25.03.1874 i Sievi. Åren 1897--\allowbreak 1909 är familjen skriven i Jeppo med 7 barn, varav de två första och de två sista är födda i Sievi.
\begin{jhchildren}
  \item \jhperson{\jhname[Johan Teofilus]{Kipari, Johan Teofilus}}{13.10.1897}{07.10.1905}
  \item \jhperson{\jhname[Martta Karolina]{Kipari, Martta Karolina}}{20.05.1900}{}, gift Enkvist
  \item \jhperson{\jhname[Naoma Dagmar]{Kipari, Naoma Dagmar}}{12.01.1903}{13.02.1903}
  \item \jhperson{\jhname[Sylvi Maria]{Kipari, Sylvi Maria}}{07.10.1904}{07.07.1924}
  \item \jhperson{\jhname[Anders Evert]{Kipari, Anders Evert}}{22.01.1907}{}
  \item \jhperson{\jhname[Eeli Edvin]{Kipari, Eeli Edvin}}{30.11.1909}{}
  \item \jhperson{\jhname[Viljo Ilmari]{Kipari, Viljo Ilmari}}{28.05.1912}{}
\end{jhchildren}
Hela familjen flyttar till Sievi i oktober 1913. I kyrkböcker finns inga anteckningar om att familjen flyttat tillbaka, men Sylvi Maria, som dog år 1924, är begravd i Jeppo.


\jhoccupant{Kipari}{Antti \& Anna}{1895--\allowbreak 1910}
Backstugukarlen Antti Mariasson Kipari, \textborn 20.08.1837 i Kortesjärvi, gift med Anna Lovisa Mattsdr, \textborn 09.10.1834 i Kortesjärvi

Barn: \jhbold{Antti}, \textborn 04.09.1859 i Kortesjärvi

Antti var backstugukarl och stugan fanns på Gustaf Hansson Ryss' hemmansdel, senare Samuel Roos. Eventuellt byggde Antti stugan. Den 18 oktober 1902 bjöds gården, fähus och lador ut på frivillig auktion. Troligen köpte sonen Antti gården, eftersom en ny frivillig auktion hölls 1906 och då var ägare Anders Kipari d.y. Ytterligare 3 år senare, nämligen 14.05.1909 utbjöds ``en mindre boningsstuga jämte 3 kpl åkermark, 1 st ko, 1 st årsgammal kalv, samt diverse möbel och husgerådssaker. Vid samma tillfälle Anders Kiparis d.ä. tillhörig diverse möbel''. Antti och Anna flyttade tillbaka till Kortesjärvi år 1910. Deras son och hans familj bodde kvar i Jeppo.



\jhhouse{Westin}{5:27}{Grötas}{10}{101}


\jhhousepic{Ivar Westin 1933.jpg}{Ivar och Edit Westin}

\jhoccupant{Westin}{Ivar \& Edit}{1933--\allowbreak 1969}
På grund av den nya vägsträckningen över Holmen, inlöste Väg- och vattenbyggnadsdistriktet jordområdet med byggnader 1969 av Ivar och Edit Westin. Ivar rev husen.

Anders Ivar Westin, \textborn 21.04.1906 i Amerika, gift 22.02.1931 med Edit Maria Jungarå \textborn 29.09.1905 (Jungarå, nr 120). De köpte tomten av Ivars föräldrar 1933 och byggde gård och ekonomiebyggnader. Ivar var på arbetsförtjänst till Amerika 1926--\allowbreak 1930 och 1934--\allowbreak 1937. Efter hemkomsten 1934 hyrde de ut bostaden i Jeppo och köpte hus i Jakobstad. Ivar arbetade där åt Ab Haldin \& Rose. De återvände  med familjen till Jeppo 1945, sålde huset i Jakobstad år 1946. Ivar arbetade åt Jeppo-Oravais Handelslag tills han blev pensionär. Makarna var också småbrukare. Efter försäljningen av bostaden, köpte de bostad i Silvast (Silvast, nr 93). Edit dog i trafikolycka 03.08.1977 och Ivar dog 12.09.1991.
\begin{jhchildren}
  \item \jhperson{\jhname[Doris Regina]{Westin, Doris Regina}}{02.12.1931}{}, gift Lundvik (Silvast nr 79)
  \item \jhperson{\jhname[Dorthy]{Westin, Dorthy}}{19.04.1939}{}, gift Gunnar (Gunnar, nr 27)
\end{jhchildren}



\jhhouse{Solbacka}{5:18}{Grötas}{10}{102}

\jhhousepic{WWikstrom.jpg}{Sigrid och William Wikström}

\jhoccupant{Nygård}{Melita \& Evald}{1976--\allowbreak 1981}
Nykarleby stad köpte tomten med byggnader den 21.08.1981 av Melita Gunvor Johanna Nygård \textborn Wikström 22.02.1928, och resten av lägenheten sålde hon åt köpvilliga jordbrukare. Melita var gift med Karl Evald Nygård, \textborn  12.08.1927 i Pensala, han dog 2013. De har båda arbetat åt Jeppo-Oravais Handelslag och var en lång tid ansvariga för Ytterjeppo filial. Efter pensioneringen köpte de lägenhet i Nykarleby centrum. Melita \textdied 05.02.2017.


\jhoccupant{Wikström}{William \& Sigrid}{1926--\allowbreak 1976}
Senast boende var William Wikström, \textborn 11.09.1902 (nr 103), gift 1926 med Sigrid Johanna Ekblad, \textborn 17.10.1906. De köpte lägenheten av Williams svåger Mantere år 1926, med cykelförsäljning och reparationsverkstad som William fortsatte med. Under åren förstorade de ekonomiebyggnaderna och köpte tilläggsjord. De var småbrukare och företagare.

William ägde också dåvarande Kilen 3:51 (nuvarande Jeppo Krafts kontor) från 1967 fram till sin död (se Fors nr 96). William dog 24.04.1976, Sigrid flyttade till pensionärshusen, hon dog 19.09.1984.

Barn:\jhbold{Melita} Gunvor Johanna, \textborn 27.02.1928 ---  \textdied 2017.


\jhoccupant{Mantere}{Mattias \& Sanna}{1915--\allowbreak 1926}                    .
Mattias Eemeli Mantere, \textborn 17.04.1888 i Alahärmä, gifte sig 1911 med Sanna Maija Wikström, \textborn 10.07.1890 (nr 103), som dog 18.08.1928. Mantere gifte om sig med Sannas syster Anna Sofia, \textborn 09.03.1893, som blivit änka efter Leander Lundqvist från Lassila. Mantere hade år 1915 genom  muntligt avtal med sin svärfar, Gustav Wikström (nr 103), erhållit området där han uppförde bostadshus med verkstad och uthus. Mantere började med cykelförsäljning och reparation. Efter försäljningen av verksamheten 1926, flyttade familjen till Nykarleby och fortsatte där med samma verksamhet.
\begin{jhchildren}
  \item \jhperson{\jhname[August Eemeli]{Mantere, August Eemeli}}{07.01.1912}{03.02.1925}
  \item \jhperson{\jhname[Auni Aili Maria]{Mantere, Auni Aili Maria}}{29.09.1914}{1925}
  \item \jhperson{\jhname[Aina Juliana]{Mantere, Aina Juliana}}{31.03.1917}{}
  \item \jhperson{\jhname[Elna Johanna]{Mantere, Elna Johanna}}{23.01.1920}{}
  \item \jhperson{\jhname[Johannes Wilhelm]{Mantere, Johannes Wilhelm}}{22.06.1922}{01.08.1940}
  \item \jhperson{\jhname[Elna Alice]{Mantere, Elna Alice}}{12.09.1924}{29.03.1925}
  \item \jhperson{\jhname[Göta Helmi Alice]{Mantere, Göta Helmi Alice}}{21.07.1926}{}
  \item \jhperson{\jhname[Veikko Eemeli]{Mantere, Veikko Eemeli}}{13.08.1928 i Nykarleby}{26.06.1932}
\end{jhchildren}

Mantere Matttias \textdied 04.01.1962  ---  Anna Sofia \textdied 18.06.1980, de är begravda i Jeppo.



\jhhouse{Wikström}{5:163}{Grötas}{10}{103}


\jhhousepic{Marita K.jpg}{Gustav och Adolfina Wikström m. fam. på Gustavs födelsedag}

\jhoccupant{Köykkä}{Marita \& Heikki}{}
Bostadshuset revs 1998,  ekonomiebyggnaderna på 1980-talet. Lägenheten ägs av Marita Köykkä Wikström, \textborn 02.05.1949, gift med Heikki Köykkä, \textborn 1949. De är pensionärer, bor i Vasa. Marita är merkonom ,jobbat på Citec i Vasa och Heikki som byggmästare åt Vasa stad. De har sålt en del av jordlägenheten och arrenderat ut resten.
Barn: Mika, \textborn 1975


\jhoccupant{Wikström}{Edvin \&  Bertta, Senja}{1948--\allowbreak 1973}
Urho Edvin Wikström, \textborn 16.05.1916, gift 1947 med Bertta Julia Laita, \textborn 03.09.1914 i Alahärmä, övertog vid arvskiftet 1948 tillsammans med Edvins syster Senja Juliana, \textborn 21.01.1905, bostadshuset och ekonomiebyggnaderna, samt det mesta av jordlägenheten. Inhyses var systern Ida, som gifte sig Wiik 1941, flyttade till Jakobstad 1947, samt Volmar, \textborn 1937, son till brodern Aarne, som blev änkling 1940. Volmar gick i finska skolan i Silvast, flyttade till sin far i Seinäjoki 1953. Edvin deltog i båda krigen.

De byggde nytt fähus 1948, brukade lägenheten tills de blev pensionärer 1973 och flyttade till pensionärshus i Jeppo. Edvin och Bertta flyttade till Vasa 1980. Senja \textdied 15.01.1991.

Barn: \jhbold{Doris Marita Helena}, \textborn 02.05.1949, gift \jhname[Köykkä]{Wikström, Doris Marita Helena}, bor i Vasa.

Edvin \textdied 25.04.1994  ---  Bertta \textdied 07.12.2007.



\jhoccupant{Grötas/Wikström}{Gustav \& Adolfiina}{1897--\allowbreak 1948}
Gustav Johan Grötas, senare Wikström, \textborn 31.07.1863 i Alahärmä, gift med Maria Adolfiina, \textborn 17.06.1869 i Alahärmä, ärvde ½ lägenheten med bostadshus och ekonomiebyggnader 1897 av fadern Johan Gustavssons dödsbo. De brukade lägenheten tillsammans med barnen och förstorade bostadshuset.
\begin{jhchildren}
  \item \jhperson{\jhname[Hanna Adolfiina]{Grötas/Wikström, Hanna Adolfiina}}{06.08.1887}{}, gift Myllykoski (nr 106)
  \item \jhperson{\jhname[Sanna Maija]{Grötas/Wikström, Sanna Maija}}{10.07.1890}{}, gift Mantere (nr 102)
  \item \jhperson{\jhname[Anna Sofia]{Grötas/Wikström, Anna Sofia}}{09.03.1893}{}, gift 1. Lundqvist, 2. Mantere (nr 102)
  \item \jhperson{\jhname[Johannes Einar]{Grötas/Wikström, Johannes Einar}}{12.11.1895}{}, till Amerika 1913
  \item \jhperson{\jhname[Gustav]{Grötas/Wikström, Gustav}}{30.12.1897}{}, till Amerika 1923
  \item \jhperson{\jhname[Nikolai]{Grötas/Wikström, Nikolai}}{24.03.1900}{}, g. m Saimi Laurila \textborn 1896, bodde på Kaupp,  i Kokkola
  \item \jhperson{\jhname[Viktor]{Grötas/Wikström, Viktor}}{24.03.1900}{21.06.1900}
  \item \jhperson{\jhname[William]{Grötas/Wikström, William}}{11.09.1907}{}, (nr 102)
  \item \jhperson{\jhbold{\jhname[Senja Juliana]{Grötas/Wikström, Senja Juliana}}}{21.01.1905}{15.01.1991}, ogift
  \item \jhperson{\jhname[Ida Katarina]{Grötas/Wikström, Ida Katarina}}{04.05.1907}{}, gift Wiik, Jakobstad
  \item \jhperson{\jhname[Aarne Emeli]{Grötas/Wikström, Aarne Emeli}}{30.08.1909}{09.09.1909}
  \item \jhperson{\jhname[Helmi Aina Karin]{Grötas/Wikström, Helmi Aina Karin}}{15.08.1910}{25.05.2001}, gift Sandell i Jeppo
  \item \jhperson{\jhname[Aarne Valtteri]{Grötas/Wikström, Aarne Valtteri}}{30.05.1913}{13.07.1976}, gift, änkling 1940
  \item \jhperson{\jhbold{\jhname[Urho Edvin]{Grötas/Wikström, Urho Edvin}}}{18.05.1916}{25.04.1994}, gift m Bertta
\end{jhchildren}

Gustav \textdied 22.03.1943  ---  Adolfiina \textdied 06.12.1951


\jhoccupant{Paalanen/Grötas}{Johan \& Johanna}{1886--\allowbreak 1897}
Johan Gustavsson Paalanen/Grötas, \textborn 19.09.1833 i Alahärmä, och hustrun Johanna Kristiina Johansdt., \textborn 03.01.1837 i Alahärmä, köpte lägenheten 7/64 dels mantal av Grötas skattehemman  nr 5, 1886 av bönderna Karl Johansson Grötas och Johan Johansson Grötas. Enligt brandförsäkringen bestod byggnaderna av ett boningshus, en spannmålsboda, en mindre spannmålsbod, en bod med loft, ett stall jämte lada under samma tak, ett fähus, tvenne foderlador, ett vedlider en badstuga samt en ribyggnad. De idkade jordbruk med kreaturskötsel. Alla barn var födda i Alahärmä.
\begin{jhchildren}
  \item \jhperson{\jhbold{\jhname[Johan Gustav]{Paalanen/Grötas, Johan Gustav}}}{31.07.1863}{}
  \item \jhperson{\jhname[Adolfiina]{Paalanen/Grötas, Adolfiina}}{05.03.1866}{}, gift Samuel Roos (nr 104)
  \item \jhperson{\jhname[Johan]{Paalanen/Grötas, Johan}}{24.06.1869}{}, tog namnet Salo (Silvast nr 365)
  \item \jhperson{\jhname[Salomon]{Paalanen/Grötas, Salomon}}{31.10.1873}{1912}, gift, till Amerika 1890
  \item \jhperson{\jhname[Viktor]{Paalanen/Grötas, Viktor}}{15.09.1876}{}
  \item \jhperson{\jhname[Nikolai]{Paalanen/Grötas, Nikolai}}{18.07.1879}{}, tog namnet Kennola (nr 131)
\end{jhchildren}
Johan Gustav \textdied 04.01.1897  ---  Johanna Kristiina \textdied 18.12.1900



\jhhouse{Roos}{5:85 och 5:30}{Grötas}{10}{104}


\jhhousepic{Roos-104.jpg}{Johannes och Maria Roos}

\jhoccupant{Roos}{Johannes \& Maria}{1933--\allowbreak 1969}
Johannes Samuelsson Roos, \textborn 05.09.1891 på Ruotsala hemman, gifte sig 30.07.1916 med Maria Fredrika Romar, \textborn 10.08.1889 på Romar.
\begin{jhchildren}
  \item \jhperson{\jhname[Gunhild]{Roos, Gunhild}}{23.07.1917}{}, gift med Oswald Sandberg
  \item \jhperson{\jhname[Göta]{Roos, Göta}}{31.07.1919}{}, gift med Bruno Westerlund
  \item \jhperson{\jhname[Margareta]{Roos, Margareta}}{29.04.1921}{}, gift med Mats Löfberg
  \item \jhperson{\jhname[Svea]{Roos, Svea}}{23.04.1925}{}, gift med Harry Grahn (Silvast 50)
  \item \jhperson{\jhname[Märta]{Roos, Märta}}{12.09.1930}{}, gift med Erik Johansson
\end{jhchildren}

De första åren efter makarnas giftermål bodde de på Romar, gård nr 22. På 1920-talet var Johannes två gånger på arbetsförtjänst till Amerika. År 1934 köpte de hemgårdslägenheten på Grötas och blev bönder.

Johannes avled 28.12.1942. Efter makens död bedrev Maria jordbruket ensam fram till 16.07.1949, då äldsta dottern Gunhild och hennes man Oswald Sandberg köpte Roos lägenhet. Maria bodde i huset fram till år 1969, då hon flyttade till Jakobstad. Hon avled 24.10.1978.

Bostadshuset och ekonomiebyggnaderna revs på 1980-talet av Elof Broo. Raul och Maija Saviaro köpte tomten år 1990.


\jhoccupant{Roos}{Samuel \& Adolfina}{1895--\allowbreak 1933}
Samuel Roos, \textborn 05.08.1861 på Gunnar hemman, gifte sig år 1889 med Adolfina Johansdotter Grötas, \textborn 05.03.1866 i Alahärmä.
\begin{jhchildren}
  \item \jhperson{\jhname[Maria]{Roos, Maria}}{10.02.1890}{}, gift med Iisak Högbacka
  \item \jhperson{\jhbold{\jhname[Johannes]{Roos, Johannes}}}{05.09.1891}{}
  \item \jhperson{\jhname[Gustaf]{Roos, Gustaf}}{02.10.1894}{21.11.1894}
  \item \jhperson{\jhname[Anna Lovisa]{Roos, Anna Lovisa}}{10.01.1895}{06.02.1896}
  \item \jhperson{\jhname[Aina]{Roos, Aina}}{26.12.1898}{}, gift Myllymäki, till Canada 1930
  \item \jhperson{\jhname[Signe]{Roos, Signe}}{20.04.1903}{}, gift Kling
  \item \jhperson{\jhname[Gustaf]{Roos, Gustaf}}{12.05.1905}{}
  \item \jhperson{\jhname[Elis]{Roos, Elis}}{06.02.1907}{}
  \item \jhperson{\jhname[Arne]{Roos, Arne}}{22.08.1909}{}
\end{jhchildren}

Makarna var till en början skrivna på Ruotsala, men efter några års	arbetsförtjänst i USA köpte Samuel den 4 januari 1895 av Sofia Jakobsdotter Grötas och Gustaf Hansson en 21/256 mantals lägenhet	på Grötas och blev bonde på Grötas. Samuel Roos hade en del förtroendeuppdrag, bl.a. var han ordförande för kommunalnämnden. I många år var han också orgeltrampare i Jeppo kyrka.

Samuel \textdied 16.02.1945  ---  Adolfina \textdied 24.01.1941


\jhoccupant{Ryss}{Gustaf \& Sofia}{1890--\allowbreak 1895}
Gustaf Hansson Ryss, \textborn 01.11.1861 i Pedersöre, gift med Sofia Jakobsdotter, \textborn 06.12.1856 på Grötas. De tre första barnen föddes i Nykarleby, övriga i Jeppo.
\begin{jhchildren}
  \item \jhperson{\jhname[Johan Jakob]{Ryss, Johan Jakob}}{15.05.1882}{}
  \item \jhperson{\jhname[Anna Erika]{Ryss, Anna Erika}}{30.07.1883}{}
  \item \jhperson{\jhname[Hilma Maria]{Ryss, Hilma Maria}}{03.11.1885}{}
  \item \jhperson{\jhname[Viktor]{Ryss, Viktor}}{03.03.1887}{}
  \item \jhperson{\jhname[Anders Gustaf]{Ryss, Anders Gustaf}}{15.02.1891}{}
  \item \jhperson{\jhname[Otto Vilhelm]{Ryss, Otto Vilhelm}}{20.02.1892}{}
  \item \jhperson{\jhname[Selma Johanna]{Ryss, Selma Johanna}}{03.11.1893}{}
\end{jhchildren}

Gustaf var slaktare förutom perioden då han var bonde på detta hemman. Han kallades ``Slaktar-Gust''. Gustaf och Sofia sålde 1895 lägenheten till Samuel Roos och flyttade till Nykarleby, Forsby. Sofia var dotter till Stenbacka smeden, som tillverkade ljuskronor, lås, gångjärn och port till Jeppo kyrka.


\jhoccupant{Ekoluoma}{Anders \& Caisa}{1881--\allowbreak 1890}
Anders Ekoluoma, \textborn 07.02.1843 i Alahärmä, gift med Caisa Mattsdr. Sandberg, \textborn 06.06.1846 på Finskas hemman.
\begin{jhchildren}
  \item \jhperson{\jhname[Matts]{Ekoluoma, Matts}}{08.02.1880}{}
  \item \jhperson{\jhname[Sanna Maria]{Ekoluoma, Sanna Maria}}{16.04.1881}{}
  \item \jhperson{\jhname[Oskar]{Ekoluoma, Oskar}}{03.04.1884}{}
  \item \jhperson{\jhname[Adolfina]{Ekoluoma, Adolfina}}{14.05.1888}{}
\end{jhchildren}

Den 22 juli 1881 sålde Carl Jonathan von Essen sin lägenhet på Grötas till Anders och Caisa samt till Caisas bror Carl. Carl behöll sin andel av lägenheten, senare Nyberg. Caisa avled 20 augusti 1889. Följande år sålde Anders deras del av lägenheten till Gustaf Hansson Stenbacka och flyttade till Alahärmä. Det finns inga uppgifter om när bostadshuset samt övriga byggnader byggts, men troligen bodde Anders och Caisa på lägenheten.


\jhoccupant{von Essen}{Karl Jonathan}{1879--\allowbreak 1881}
Kontoristen Jonathan von Essen, \textborn 07.10.1851 på Keppo, var några år ägare till detta hemman. Han köpte det på auktion 7 juli 1879. Ägare till hemmanet var Johan Jakobsson West och Isak Isaksson Pensalas konkursmassa.


\jhoccupant{Andersson}{Matts \& Caisa}{1877--\allowbreak 1879}
Matts Andersson, \textborn 15.11.1826 på Keppo, gifte sig 1847 med Caisa Lisa Jacobsdotter, \textborn 06.03.1824. Matts Andersson Keppo eller Grötas köpte hemmanet på auktion 27 oktober 1877. Hemmanet hade ägts av Johan Johansson Grötas. Matts sålde efter ca 1 år hemmanet till \jhname[Johan West]{West, Johan} och \jhname[Isak Pensala]{Pensala, Isak}.


\jhoccupant{Johansson}{Johan \& Sanna}{1867--\allowbreak 1877}
Johan Johansson, \textborn 12.01.1842 i Jeppo, gift med Susanna Andersdr. Forss, \textborn 21.01.1845 på Fors hemman.
\begin{jhchildren}
  \item \jhperson{\jhname[Johannes]{Johansson, Johannes}}{30.11.1863}{}
  \item \jhperson{\jhname[Anders Wilhelm]{Johansson, Anders Wilhelm}}{05.01.1865}{}
  \item \jhperson{\jhname[Maria Catarina]{Johansson, Maria Catarina}}{28.04.1866}{}
  \item \jhperson{\jhname[Anna Lovisa]{Johansson, Anna Lovisa}}{27.01.1870}{}
  \item \jhperson{\jhname[Sanna Brita]{Johansson, Sanna Brita}}{05.04.1871}{}
  \item \jhperson{\jhname[Anders Emil]{Johansson, Anders Emil}}{08.04.1873}{}
\end{jhchildren}

Johan fått denna lägenhet i sin ägo år 1867 av sina föräldrar Johan	Danielsson Grötas och Katarina (Caisa) Andersdotter.


\jhoccupant{Danielsson}{Johan \& Caisa}{- 1867}
Johan Danielsson Grötas, \textborn 01.11.1811 i Jungarby, hustru Caisa Andersdotter, \textborn 12.01.1842.



\jhhouse{Eklund}{5:7}{Grötas}{10}{109}


\jhoccupant{Eklund}{Selma \& Anna}{1915--\allowbreak 1969}
Selma Eklund, \textborn 27.05.1891 på Grötas. Selma var ensamstående och bodde med sin mamma Anna. Selma var sömmerska. Hon arbetade bl.a vid Lönnqvists stickeri en tid, vid Almbergs som piga, på Kiitola på 20-talet. Senare arbetade hon som hushållerska och vårdare av en äldre, sjuklig man i Jeppo. Selma flyttade år 1969 till det nya pensionärshuset.

I december 1921 behandlade legonämnden i Jeppo Selmas mors, Anna Eklunds ansökan om att få lösa in sin bostadslägenhet omfattande 0.0003 mtl. År 1927 fick hon lagfart på bostadslägenheten, som Anders och Anna år 1913 genom köpebrev undantagit  åt sig själva då de sålde sin lägenhet. Raul Saviaro (nr 23, nr 21) köpte tomten år 1970, rev bostadshuset och byggde uthus åt sig själv på tomten.

Anna \textdied 02.11.1936  ---  Selma \textdied 08.09.1974


\jhoccupant{Eklund}{Anders \& Anna}{1913--\allowbreak 1915}
Backstugukarlen, tidigare bonden Anders Eriksson Ruotsala, senare Grötas, \textborn 05.03.1854 på Ruotsala, gift med Anna Gustafsdotter, \textborn 12.12.1851.
\begin{jhchildren}
  \item \jhperson{\jhname[Hilda Maria]{Eklund, Hilda Maria}}{07.09.1879}{}
  \item \jhperson{\jhname[Anna Lovisa]{Eklund, Anna Lovisa}}{05.12.1882}{}, g Westin
  \item \jhperson{\jhname[Johan Emil]{Eklund, Johan Emil}}{23.09.1884}{}
  \item \jhperson{\jhname[Anders Gustaf]{Eklund, Anders Gustaf}}{18.12.1886}{}
  \item \jhperson{\jhname[Ida Sofia]{Eklund, Ida Sofia}}{22.12.1888}{}, gift med Johan Jakob Andersson Mietala
  \item \jhperson{\jhbold{\jhname[Selma]{Eklund, Selma}}}{27.05.1891}{}
  \item \jhperson{\jhname[Erik Alfred]{Eklund, Erik Alfred}}{20.11.1896}{29.11.1896}
\end{jhchildren}

Anders och Anna köpte lägenhet på Grötas år 1894, men sålde denna år 1913 till dottern Anna Lovisa och hennes man Johannes Westin, då de kom hem från Amerika. I köpekontraktet undantog de för egen del fyra kappland åkerjord i 50 års tid. Anders och Anna samt barnen Selma, Anders Gustaf och Johan Emil bodde kvar i en sytningsstuga på lägenheten. Sonen Anders emigrerade till Amerika 1914, sonen Johan hade farit tidigare. Åren 1921–30 är han skriven som inhysing tillsammans med sin hustru Maria Johanna Israelsdotter Sandås, \textborn 24.06.1883 i Munsala. De fick en son Emil Edvin, \textborn 10.09.1911 i Amerika.

Anders Eklund \textdied 03.02.1915  ---  Anna \textdied 02.11.1936



\jhhouse{Långlindan}{5-16}{Grötas}{10}{106}


\jhhousepic{Grotas106-Saviaro.jpg}{Adolfiina Myllykoski}

\jhoccupant{Myllykoski}{Jakob \& Adolfiina}{1926--\allowbreak 1969}
Huset och ekonomiebyggnaderna revs 1971 av Raul Saviaro. Senast boende var änkan Hanna Adolfiina Myllykoski, \textborn  06.08.1887 (nr 103), gift med Jakob Mattsson, \textborn 26.12.1880 i Alahärmä, \textborn 25.05.1910. Änkan Adolfiina(Fiina) köpte huset och byggnaderna av kommunen på auktion 1926 för 405 mk. År 1931 fick hon inlösa backstugo-området som underlydde Samuel Roos hemman. Fiina arbetade bl.a. på Jeppo ullspinneri och som hjälp åt jordbrukare, hon dog 30.03 1969.  Barn: \jhbold{Anni} Aino Maria, \textborn 18.08.1910, gift Saviaro.


\jhoccupant{Saviaro}{Anni \& Antti}{1929--\allowbreak 1953}
Anni gifte sig 1929 med Anders (Antti) Gustav Saviaro, \textborn 01.05.1907. De bodde tillsammans med änkan Fiina i huset tills de flyttade till nr 105 år 1953. Antti arbetade åt Statens järnvägar tills han blev pensionerad, tillsammans med hustrun Anni bedrev de småbruk, som ung hade också Anni  arbetat vid Jeppo ullspinneri.
\begin{jhchildren}
  \item \jhperson{\jhname[Harry Anders Wilhelm]{Saviaro, Harry Anders Wilhelm}}{08.02.1932}{1985 i Jakobstad}, gift
  \item \jhperson{\jhname[Raul Gustav]{Saviaro, Raul Gustav}}{25.01.1936}{}, nr 23 och 21
\end{jhchildren}


\jhoccupant{Niemi}{Jaakko \& Lisa}{1909--\allowbreak 1925}
Enligt köpekontrakt av den 30.08.1909, köper Jaakko Niemi, \textborn 25.05.1884 i Ylistaro, gift med Lisa Johanna Tomasdr., \textborn 24.05.1883 i Jeppo, boningsstugan, fähus och boda uppförda på Samuel Roos arrendejord, arrendetid 50 år, av änkan Anna-Sofia Eriksson. Efter att föräldrarna Niemi dött - Lisa 1922 och Jaakko 1925 - omhändertogs barnen av kommunen och byggnaderna tillföll kommunen och såldes på auktion.
\begin{jhchildren}
  \item \jhperson{\jhname[Edit Johanna]{Niemi, Edit Johanna}}{10.06.1908}{14.10.1920}
  \item \jhperson{\jhname[Ester Sofia]{Niemi, Ester Sofia}}{18.07.1909}{}
  \item \jhperson{\jhname[Agnes Sylvia]{Niemi, Agnes Sylvia}}{18.07.1916}{}
  \item \jhperson{\jhname[Anna Linea]{Niemi, Anna Linea}}{26.02.1918}{16.03.1918}
  \item \jhperson{\jhname[Inga Maria]{Niemi, Inga Maria}}{27.10.1919}{15.03.1920}
\end{jhchildren}


\jhoccupant{Eriksson}{Anna-Sofia}{?-1909}
Närmare uppgifter saknas.\jhvspace{}



\jhhouse{Åbacken}{5:145}{Grötas}{10}{7}


\jhhousepic{128-05666.jpg}{Bror och Siv Westin}

\jhoccupant{Westin}{Bror \& Siv}{1996--}
Bror Henrik Westin, \textborn 04.04.1953 (nr 8), gifte sig 12.07.1991 med Siv Marianne Sundqvist, \textborn 03.05.1966 i Öja, köpte 1995 tomten med föräldrarnas bostadshus (nr 8). De byggde nytt bostadshus bredvid 1996. Bror har arbetat som arbetsledare på KWH-Mirka och är idag deltidspensionär. Siv är församlingsmästare på Jeppo kapellförsamling.
\begin{jhchildren}
  \item \jhperson{\jhname[Ola Henrik]{Westin, Ola Henrik}}{22.04.1993}{}, bor i Nykarleby, maskinskötare på Mirka
  \item \jhperson{\jhname[Lucas André]{Westin, Lucas André}}{13.05.1997}{}, arbetar på Mirka
\end{jhchildren}



\jhhouse{Westin}{5:32}{Grötas}{10}{8}

\jhoccupant{Westin}{Bror \& Siv}{1995--}
Huset obebott och ägs idag av Bror och Siv Westin (nr7).\jhvspace{}


\jhhousepic{129-05667}{Obebott. Senaste hushåll var Valfrid och Ingrid Westin}

\jhoccupant{Westin}{Valfrid \& Ingrid}{1955--\allowbreak 1995}
Henrik Valfrid Westin, \textborn 21.05.1910 i Jeppo, gift 03.03 1940 med Ingrid Linnea Lillqvist, \textborn 28.04.1911 i Purmo. Valfrid deltog i krigen. Han flyttade till Jakobstad 1935 där han arbetade på Cellulosafabriken. År 1948 vid arvskiftet övertog Valfrid och brodern Elis ½ lägenheten var av föräldrarna. Valfrid och Ingrid fortsatte med jordbruket, byggde nytt bostadshus 1955, på samma område vid älven.

I mitten på 1970-talet avslutade de jordbruksverksamheten och sålde ut. Valfrid jobbade också som byggnadsarbetare och Ingrid i pälsningsarbete på Keppo. Valfrid dog 27.12 1988 och Ingrid dog 06.04 2002, de sista fem åren bodde hon i  pensionärshus. Sonen Bror Henrik köpte bostadshuset med tomt 1995.
\begin{jhchildren}
  \item \jhperson{\jhname[Magnus Valter Johannes]{Westin, Magnus Valter Johannes}}{07.05.1940}{}, gift, bor Sverige
  \item \jhperson{\jhname[Lars Valdemar]{Westin, Lars Valdemar}}{15.09.1941}{}, gift, bor i Sverige
  \item \jhperson{\jhname[Sven Yngve]{Westin, Sven Yngve}}{25.11.1946}{},gift, bor i Jakobstad
  \item \jhperson{\jhname[Anita Linnea]{Westin, Anita Linnea}}{05.06.1950}{}, gift Sundström bor i Jakobstad
  \item \jhperson{\jhbold{\jhname[Bror]{Westin, Bror}} Henrik}{04.04.1953}{}, gift (nr7)
  \item \jhperson{\jhname[Rut Maria]{Westin, Rut Maria}}{29.11.1954}{}, sambo, bor i Jakobstad
\end{jhchildren}



\jhhouse{Westin}{5:27}{Grötas}{10}{108}


\jhhousepic[pic:westin]{Elis Westin.jpg}{Elis Westin}

\jhoccupant{Westin}{Elis}{1948--\allowbreak 1981}
Bostadshuset och ekonomiebyggnaderna revs i mitten av 1980-talet. Senast boende var Elis Verner Westin, \textborn 29.08.1912, ogift. Han och brodern Henrik Valfrid med hustrun Ingrid Linnea övertog ½ lägenheten var vid arvskiftet 1948. Valfrid och Ingrid byggde nytt bostadshus åt familjen 1955 (nr 8). Elis arbetade hela sitt liv med jordbruk och kreatur. Deltog i krigen.

I huset bodde en tid tre generationer innan huset vid ån byggdes år 1955. Elis \textdied 27.10.1981.


\jhoccupant{Westin}{Johan \& Anna-Lovisa}{1933--\allowbreak 1948}
Johan Henrik och Anna-Lovisa med familj flyttade bostadshuset (nr 107, karta 11) till den här nya platsen efter storskiftet 1932--33 samt byggde nya ekonomiebyggnader. De var jordbrukare med kreatursskötsel. De överlät hemmanet 1948. Barn och övriga uppgifter, se nr 107.

Anna-Lovisa \textdied 31.12 1965  ---  Johan Henrik \textdied 20.06 1967



\jhhouse{Älvliden, inkl. boende}{893-50-5003-1 \& 893-50-5001-5}{Grötas}{10}{9}


\jhhousepic{134-05671.jpg}{Älvliden 10, från åsidan}

\jhoccupant{Fastighets Ab}{Älvliden}{1974--}

Sedan medlet av 1960-talet hade den kommunala investeringtakten i olika fastigheter varit hög. En ny Centrumskola byggdes på Sågtomten och blev klar 1966. Kort därefter startade planeringen av ett nytt hus för den kommunala förvaltningen. I samma veva skulle en ny brandstation byggas med tillhörande bostad och därtill beslöts att också ett pensionärshus med 7 lägenheter skulle uppföras alldeles i närheten av kommunalhuset. Allt detta blev klart i slutet av 1968 och i mars 1969 kunde landshövdingen Martti Viitanen inviga allt det nya.

Men det var inte slut med detta. Keppos expansion inom pälsnäringen samtidigt som Mirka kommit ordentligt igång med sin verksamhet vid Kiitola, gjorde att behovet av bostäder för arbetskraft, som pendlade in till orten växte. Sommaren 1971 inköptes delar av Roos lägenhet på Grötas av kommunen. Det hade två syften; att kunna påvisa tomtmark för Mirka, som hade börjat signalera om ett sådant behov för en ny hall för sin expanderande verksamhet och samtidigt säkerställa ett markområde för egnahemshus och eventuella radhus. Markköpet av Roos lägenhet fördelade sig på östra och västra sidan om vägen mot Voltti. Den östra skulle i första hand reserveras för Mirkas behov och den västra för bostäder.

\jhhousepic{131-05895.jpg}{Älvliden 7}

I nov. 1971 besluter kommunfullmäktige att godkänna planerna på att bygga ett hyreshus vid älven på det inköpta området. I budgeten intas ett anslag om 20.000 mk för mätning och kartläggning under år 1972. I mars 1972 väljs ingenjörsbyrån Jord \& Vatten från Lappo att utföra både generalplan och delplan för området. Det resulterar i en delplan som ger rum för 8 egnahemshus och ett större radhus. Länsarkitekten, som anlitas i sammanhanget, anser att samtidigt som byggplan uppgörs för centrumområdet, kunde också Furubacken flygfotograferas och anslutas. Redan före midsommar fattas beslut om att tomter kan delas ut på det nya området till hösten. I oktober 1972 fattas beslut om att till våren 1973 gräva ner 1300 m vattenledning och 300m avloppsledning på området. Avloppsledningen tänker man senare ansluta till eventuellt eget reningsverk. Fullmäktige meddelas att ansökan om ett bostadslån inlämnats till bostadsstyrelsen.

I november 1973 startar bygget av radhusprojektet, som omfattar 15 lägenheter i 2 huskroppar omfattande totalt 830 m² till en beräknad kostnad om 878.000 mk. Kommunen äger 84 \% av aktierna i det nya bolaget, som får namnet ``Fastighets Ab Älvliden''. Resten av aktierna delas mellan Ab Keppo Oy och Jeppo Församling. Bostadsstyrelsen har beviljat 330.000 mk i bostadslån, men höjer senare summan till 405000 mk.

Den 14 juli 1974 firas taklagsfest på det nya bygget och projektets ordf., bankdir. Johan Stenfors, kan meddela att de 15 lägenheterna borde vara klara för inflyttning i slutet av juli månad.

Lägenheterna blev snabbt besatta med hyresgäster och omsättningen bland hyresgästerna har genom åren varit stor. Det betyder att möjligheten att identifiera dem alla har visat sig omöjligt, då hyresmatrikeln inte finns tillgänglig.

År 1995 genomfördes en fusion mellan de bostadsaktiebolag som fanns inom staden och där staden var majoritetsägare. Den 12 april 1995 undertecknades fusionsavtalet av Älvlidens styrelseordförande Johan Stenfors. I denna sista styrelse för Ab Älvliden Kiinteistö Oy satt också Börje Lindborg och Göran Westerlund, med suppl. Edgar Jungarå, Alf Back och Jarl Roos. I nuläget äger staden 96,92\% av aktierna och KWH 3,08\% av det nya bostadsaktiebolaget Nykarleby Bostäder.
Hyresgäster genom åren har varit bl.a.:

\begin{center}
  \begin{longtable}{l l}
    \hline
    Elof \& Ruth Kula &  Kim Brännkärr \\
    Adele Nygård & Ari Tupeli \\
    Daniel Back \& ….......Itkonen & Saga Nygård \\
    Johan \& Göta Kronlund & Maria \& Henrik Bergström \\
    Elisa Norrgård \& Maria \& Jari Karvonen \\
    Matias Elenius & Janne Kujala \\
    Johanna Karvonen & Jarmo Hänninen \\
    Marjatta Perälä & Tobias Julin \& Linda Söderlund \\
    Camilla Back \& Ida Lavast & Alf \& Päivi Dahlström \\
    Peter \& Merja Dahlström & Matias Julin \&  ?? \\
    Matts \& Ida MarieNorrgård & Åsa Kulla \\
    Henry \& Eira Back & Rune Elenius \\
    Pirkko Alaharja & Bo Strengell \& Yvonne Granbäck \\
    Ester Ahlström & ..Lindén \\
    Martti \& Hilja Aho & Ann-Louise Strengell \\
    Roger Nybäck & Jorma \& Brita Iiskola \\
    Hans \& Agnes Backman & Elisa Hilli \& Jorma Alaranta \\
    Östen \& Helga Dahlskog & Markku \& Seija Aalto \\
    Veikko Kontio & Håkan \& Barbro Back \\
    Vivan \& Johan Back & Trygve \& Maria Sirén \\
    Greta Eklöv & Göta Elenius \\
    Ingrid Cederström & Senja Vikström \\
    Ellen Jungerstam & Sigrid Vikström \\
    Margit Romar & Alice Ahlfors \\
    Veijo Mäkinen & Leif \& Vivan Andersson \\
    Kaj \& Linda Andersson & Liselott Back \\
    Edvin \& Ruth Bäckstrand & John \& Pirkko Elenius \\
    Torolf \& Yvonne Emaus & Tom \& Benita Bäckstrand \\
    Kuisma \& Helmi Koski & Adolfina \& Lennart Laxén \\
    Gunfrid Kuoppala & Sören \& Britt Marie Lawast \\
    Ragnar \& Britta Jungell & Birgitta Kronqvist \\
    Juha-Matti Kujala & Guy \& Ann-Maj Häger \\
    Runar Törnqvist & Curt \& Margareta Sandberg \\
    Martin \& Anette Sandberg & Kustaa \& Edit Ylinen \\
    Moritz \& Birgitta Nygård & Elna Julin \\
    Saima Romar & Kaija Salminen \& Kyrönperä \\
    Anna Renvall \& …......... & Tapanimäki \\
    Tore \& Ulla Wallin & Gottfrid \& Marianne Lundvik \\
    Conny Bäckstrand \& Pia Jakobsson & Esko \& Terttu Jussila \\
    Hans Kronlund \& Gun-Helen Häggback & - \\
    \hline
  \end{longtable}
\end{center}



\jhhouse{Norrlid}{5:96}{Grötas}{10}{10}


\jhhousepic{132-05896.jpg}{Paul och Anneli Laxén}

\jhoccupant{Laxén}{Paul \& Anneli}{1973--}
Paul Laxén, \textborn 03.11.1931, gifte sig 1957 med Anneli Huhtala, \textborn 23.12.1937. Paul är pensionerad banförman, Anneli bageriarbetare.

Barn: \jhname[Cay]{Laxén, Cay}, \textborn 13.08.1958, arbetat/arbetar med olika uppgifter inom SJ.

Paul och Anneli köpte tomt av Jeppo kommun. Banvaktseran var slut och banvaktsstugorna såldes på auktion. Paul köpte Tapelbackastugan och använde det han kunde från stugan, bl.a timret, som timrades på stående. Väggarna av tegel gömmer den gamla banvaktsstugan. Byggår var 1973.

Parets liv är i många stycken präglat av järnvägen, vilket bekräftas av det diplom Paul förärades med vederbörliga signaturer och dateringen i Helsingfors, i järnvägsstyrelsen den 28 februari 1981: ``Då Ni Banförman Paul Mach Laxén efter 25 år i statsjärnvägarnas tjänst fullgjort Edert värv, frambär järnvägsstyrelsen härmed till Eder sin erkänsla och sitt bästa tack för det arbete Ni troget utfört till fromma för järnvägarna och därmed för hela landet. Må detta diplom vara ett bevis på det värde och den betydelse järnvägsstyrelsen tillmäter Eder livsgärning.''

Paul har med stort intresse engagerat sig i och bidragit till tillkomsten av denna bok.



\jhhouse{Union}{5:95}{Grötas}{10}{11}


\jhhousepic{154-05708.jpg}{Håkan och Lenelis Cederström}

\jhoccupant{Cederström}{Håkan \& Lenelis}{2003--}
Håkan, vvs-montör, och Lenelis, byråsekreterare och närvårdare (Ruotsala nr 20) köpte lägenheten 2003 av Boris Lindén och BSB-Mekan. Redan följande år inrättade Lenelis en present- och blombod 	i kafédelen. Den var verksam 2004--\allowbreak 2007. År 2011 renoverade man sagda butik och förvandlade den till	kontor/bostad. Då sonen Jonas och hans sambo Paulina Peltorinne	(Silvast-Fors, karta 5, nr 70) flyttade hem från Sverige, kunde det unga paret ordna sitt boende här under åren 2012--\allowbreak 2014. Parets son, Noah Cederström, föddes 19.07.2012.
Hallarna används f.n. som garage, verkstad, lagerrum och flisförråd.

\jhbold{Hyresgäster under perioden 1975--\allowbreak 2016}


\jhoccupant{Källan}{förening}{2016--}
Föreningen flyttade in som hyresgäst år 2016 i samband med att dess tidigare utrymme vägg i vägg med Café Funkis saneras för att ingå i en förstorad kafélokal.\jhvspace{}


\jhoccupant{Norrback}{Bernt}{2006}
Under 2006 hyrde Bernt in sig med sin spirande verksamhet kring bilreparationer och motorrenoveringar. Se karta 6, nr 87.\jhvspace{}


\jhoccupant{KGF}{r.f.}{1997}
Sommaren -97 inledde nybildade Kvarnbackens Gymnastikförening r.f. inredningsarbete i Union-stationens serviceutrymmen. Man hade fått lov att utrusta och utnyttja västra hallen som tillfällig träningslokal. Den var råkall och dragig för lättklädda gymnaster, men fick till en början duga som provisorium under den varma tiden på året. Föreningens första	``byuppvisning'' i sju discipliner (parterr, bom, ringar, räck, barr, hopp 	och bygelhäst) för en nyfiken publik skedde med framgång i denna miljö, alltmedan en permanent hall (Arcas) färdigställdes i höladan på Kiitolavägen 15, karta 6, nr 97e.


\jhoccupant{Kesko/}{K-Lantbruk}{}
Företaget med handelsman \jhname[Tomas Kjellman]{Kjellman, Tomas} var hyresgäst en kort period i början på 1990-talet innan han s.s. drog sig till Esse. Se även Silvast, karta 8, nr 120.\jhvspace{}


\jhoccupant{Karvonen}{Jari \& Jyrki}{1988--\allowbreak 1992}
Bröderna Jari och Jyrki Karvonen drev Union-stationen tillsammans med verksamhetskonceptet bensinförsäljning och bilreparationer. Kafeterian var inte öppen.\jhvspace{}


\jhoccupant{Broman}{Ralf}{1982--\allowbreak 1988}
Ralf köpte verksamheten 1982 och fortsatte med bensinförsäljningen, sålde oljor, däck och utförde däckreparationer m.m. Som ensam företagare med enbart enstaka hjälpande inhoppare blev arbetsbördan dryg.	Den underlättades inte heller p.g.a. den brand som den 3 oktober 1984 höll på att ödelägga verkstaden totalt. Eld hade uppstått i isoleringsmaterial, som BSB Mekan hade lagrat på utsidan vid västra	gaveln, och därefter spridit sig via nocken vidare in i hallen.


\jhoccupant{Grandén}{Tage}{1979--\allowbreak 1981}
Tage var en person som verkade vara född med skiftnyckel och skruvmejsel i sina händer. Han var inte rädd för nya utmaningar i all sorts mekanik och behövde en verkstad för att få utlopp för sina talanger. I verkstaden och bränsleförsäljningen jobbade \jhname[Håkan Back]{Back, Håkan}, \jhname[David Nybyggar]{Nybyggar, David}, \jhname[Bror Fors]{Fors, Bror} och Lars Dahlqvist. Tage etablerade sig i egna 	utrymmen i Gränden år 1982.


\jhoccupant{Hankkija}{Labor}{1975--\allowbreak 1979}
Via avtal i december -81 kunde Hankkija, Labor och Pellervo fortsätta använda verkstadsfunktionerna vid Union. Företaget hyrde in sig med förnödenhetsbutik och fältförsäljning av traktorer, tröskor o.a. Tom Bäckstrand skötte butiken medan \jhname[Tore Wallin]{Wallin, Tore}, \jhname[Bengt Back]{Back, Bengt} och \jhname[Johan Nylund]{Nylund, Johan} dejourerade kl. 8-10 på förmiddagarna innan de drog ut i fält som försäljare av maskiner och anläggningar. Verksamheten levde förhållandevis trångt. Efter fyra år flyttade delar av den till finska skolan på Stationsvägen (karta 5,  nr 361).


\jhbold{Ursprunget}


\jhoccupant{Lindén}{Boris \& Gurli}{1969--\allowbreak 2003}
Lägenheten på 6008 m$^2$ inköptes år 1969 av Uno Nyberg, R:nr 5:29. Verkstaden med tillhörande kafé och bensinmack, ``Union'', började	byggas senhösten 1969 när det redan fanns ordentligt med snö på marken. Boris' bror \jhname[Börje]{Lindén, Börje} deltog också i byggnadsarbetet. Den 375 m$^2$ stora lokaliteten togs i bruk redan följande år då Boris och Gurli (karta 6, nr 94) inledde verksamheten.

Gurli skötte kaféet och bokföringen. Boris och två anställda, Håkan Back från orten och \jhname[Matti Riikonen]{Riikonen, Matti} från Oravais, tog hand om verkstaden och bensinstationen. Den 23 oktober 1970 annonserades följande i Österbottniska Posten:


\jhpic{Unionannons.png}{Annons i Österbottniska Posten (ÖP)}


Verksamheten var från början livlig, eftersom den nya omfartsvägen, Riks 19, byggdes då. Under den här perioden rätades tillika stambanan ut vid Skog, norr om centrum, vilket medförde diverse reparations- och servicejobb på den tunga utrustningen.

Efter 1973 kom Gurlis författarskap mera i fokus och ett kafébiträde blev anställt. 'Farmor' Anni Lindén var även inkopplad i den skötseln. Sönerna Thomas och Andreas jobbade som bensinförsäljare under sina sommarlov. Säsongjobb förslog även åt andra intresserade; t.ex. Nils Forss fick här en början på sin arbetshistoria. Redan i byggnadsskedet anställdes Håkan Cederström från Ytterjeppo och senare hans blivande fru, Lenelis Lindgren, Jeppo, som skötte kaféet 1973--\allowbreak 1974; det var här de träffades och romansen fick sin början.	Också andra personer har skött kaféet i kortare perioder.

Verkstaden gav bröderna Boris, Stig och Börje Lindén möjligheter att testa nya idéer. Det var här man monterade ihop sin första prototyp	av gödsellastare avsedd för pälsfarmer. Det nystartade företaget Ab	BSB-Mekan Oy, vilket infördes i handelsregistret den 27.4.1979 (Yrityshaku), har därefter utvecklats med en alldeles egen historia och med verksamheten förlagd till Nykarleby.

Boris dog, bruten av sjukdom, den 07.08.2017 i Nykarleby.



\jhhouse{Sydliden}{5:98}{Grötas}{10}{12, 12a}


\jhhousepic{133-05670.jpg}{Paula Broo}

\jhoccupant{Broo}{Paula \& Herman}{1973--}
Herman Broo, \textborn 03.07.1945 på Böös, gifte sig 1966 med Paula Ylinen, \textborn 02.02.1947 i Alahärmä.
\begin{jhchildren}
  \item \jhperson{\jhname[Jari]{Broo, Jari}}{14.10.1967}{}, elmontör
  \item \jhperson{\jhname[Kirsi]{Broo, Kirsi}}{05.07.1969}{}, ambulerande speciallärare i Nykarleby
  \item \jhperson{\jhname[Kati]{Broo, Kati}}{17.11.1972}{}, speciallärare i Nykarleby
\end{jhchildren}

Herman köpte tomt av kommunen år 1973 och uppförde själv sin gård. Herman har varit privatföretagare hela livet. Han började som skogshuggare med häst och släde med sin far. Småningom skaffades traktor med vinsch och han fick kontrakt med Jeppo skogsandelslag. År 1969 skaffades en virkesutkörningsmaskin. Sedan dess arbetade han som skogsmaskinsentreprenör för Österbottens Trä/UPM Kymmene fram till sin pensionering år 2008.

Av intresse för trä och timmer skaffade Herman en timmersvarv, med vilken han har svarvat bl.a lekstugor, garage, bastun och grillhus i stock. Paula började arbeta för Nykarleby stad som städerska år 1976. Hon har städat kommunalgården, biblioteket samt konditionshallen i Jeppo i 34 år. Paula utbildade sig till anstaltsvårdare år 2006. Hon gick i pension 2010. Under åren 1980--\allowbreak 1989 hade familjen egen pälsfarm med rävar.

Herman \textdied 30.01.2012



\jhhouse{Sollid}{5:99}{Grötas}{10}{13}


\jhhousepic{127-05664.jpg}{Greger och Kristina Forsbacka}

\jhoccupant{Forsbacka}{Greger \& Kristina}{1990--}
Greger Forsbacka, \textborn 1947 i Terjärv, gifte sig 1974 med Kristina Sundqvist, \textborn 1951 i Terjärv. Tidigare boningsort Oravais.
\begin{jhchildren}
  \item \jhperson{\jhname[Malin]{Forsbacka, Malin}}{1982}{}, arbetar på Wärtsilä
  \item \jhperson{\jhname[Jonas]{Forsbacka, Jonas}}{1986}{}, arbetar på Jeppo Potatis
\end{jhchildren}

Familjen köpte huset efter Ragnar och Elvi Lindgren år 1990. Greger har arbetat som butiksföreståndare på Sale, Kristina som familjedagvårdare.


\jhoccupant{Lindgren}{Ragnar \& Elvi}{1974--\allowbreak 1988}
Ragnar Lindgren, \textborn 21.05.1924 i Tenala, gift med Elvi Almberg, \textborn 13.03.1921 på Grötas. Ragnar var pälsdjursuppfödare, Elvi arbetade vid Mirka. Ragnar köpte tomten i juli 1973 av Jeppo kommun, han uppförde bostadsbyggnaden år 1974. Huset är ett s.k. elementhus med fasad av trä och tegel.

Ragnar \textdied 27.04.1986  ---  Elvi \textborn 01.06.1988



\jhhouse{Björklid}{5:100}{Grötas}{10}{14}


\jhhousepic{126-05663.jpg}{Carl-Erik och Ing-Britt Forss}

\jhoccupant{Forss}{Carl-Erik \& Ing-Britt}{1973--}
Carl-Erik Fors, \textborn 01.10.1944 på Fors hemman, gifte sig 20.11.1966 med Ing-Britt Lövbacka, \textborn 07.03.1946 i Vörå.
\begin{jhchildren}
  \item \jhperson{\jhname[Christel]{Forss, Christel}}{1967}{}, stud.merkonom, arbetar på Fpa
  \item \jhperson{\jhname[Sven-Erik]{Forss, Sven-Erik}}{1968}{}, tel.montör, arbetar på Anvia
\end{jhchildren}

Familjen varit bosatt i Jeppo sedan 1967. Carl-Erik har jobbat som elmontör på Jeppo Kraft och L. Jacobson, sammanlagt 28 år. Från 1992 som gårdskarl och släckningschef åt Nykarleby stad i Jeppo, 18 år fram till pensioneringen. Fritiden har ägnats åt olika förtroendeuppdrag.Ing-Britt har jobbat som hemhjälpare i 16 år. Skolade om sig 1989 till sysselsättningsledare. De senaste 13 åren arbetade hon som köksbiträde på Jeppo skola.

Carl-Erik och Ing-Britt uppförde nuvarande bostad år 1973. Kommunen, som inköpt området av Oswald och Gunhild Sandberg, hade planerat området för egnahemshus och radhus. Området är en del av Roos lägenhet.


\jhbold{Roos},  R:nr 5:85

\jhoccupant{Sandberg}{Oswald \& Gunhild}{1949}
Oswald, \textborn 26.09.1911 på Finskas, gifte sig 1938 med Gunhild Roos, \textborn 23.07.1917 på Grötas. Oswald övertog hälften av hemmanet på Finskas. 16.07.1949 köpte han Roos lägenhet Rn: 5:85 av svärmor Maria Roos. Roos hemman såldes senare till kommunen.



\jhhouse{Rosas}{5:104}{Grötas}{10}{15}


\jhhousepic{124-05661.jpg}{Veikko och Gunnel Kontio}

\jhoccupant{Kontio}{Veikko \& Gunnel}{1980-}
Veikko Kontio, \textborn 02.03.1941 i Alahärmä, gifte sig 06.12.1970 med Gunnel Bergström, \textborn 22.03.1946 i Nykarleby. Veikko har arbetat som långtradarchaufför. Gunnel arbetade som hemdagvårdare. Hon dog 10.08.1989.
\begin{jhchildren}
  \item \jhperson{\jhname[Katarina]{Kontio, Katarina}}{16.03.1971}{},  g. Pirkman, arbetar på TW Laskentapalvelut
  \item \jhperson{\jhname[Anna-Lena]{Kontio, Anna-Lena}}{28.08.1973}{}, arbetar på Mirka
\end{jhchildren}

Veikko byggde huset på stadens arrendetomt. Han har byggt det mesta på huset själv. Byggår var 1980. Idag bor Veikko och dottern Anna-Lena i huset.



\jhhouse{Forsbacka}{5:101}{Grötas}{10}{16}


\jhhousepic{122-05659.jpg}{Sven-Olof och Lise-Maj Forsbacka}

\jhoccupant{Forsbacka}{Sven-Olof \& Lise-Maj}{1977--}
Sven-Olof Forsbacka, \textborn 20.08.1948 i Nykarleby, gifte sig 1969 med Lise-Maj Back, \textborn 24.09.1946 på Back i Jeppo.
\begin{jhchildren}
  \item \jhperson{\jhname[Jan-Mikael]{Forsbacka, Jan-Mikael}}{1972}{}, el.ingenjör
  \item \jhperson{\jhname[Bernt-Ola]{Forsbacka, Bernt-Ola}}{1974}{}, bygg- och vägmästare, Rakennus-Bygg SF, Vd.
  \item \jhperson{\jhname[Ann-Charlotte]{Forsbacka, Ann-Charlotte}}{1981}{}, lektor, frisörlinjen v. Yrkesakademin Öb
\end{jhchildren}

Sven-Olof och Lise-Maj, som numera är pensionärer, hade tidigare egen pälsdjursfarm. Lise-Maj arbetade också som köksa vid Jeppo skola. Sven-Olof har också arbetat på Jeppo Kraft, tidvis som ansvarig för den nya cirkelsågen. Huset byggdes år 1977 i egen regi, dels med element från Oravais Hus. Till fritidsintressen hör fiske.



\jhhouse{Björkskog}{5:102}{Grötas}{10}{17, 17a}


\jhhousepic{120-05657.jpg}{John och Pirkko Elenius}

\jhoccupant{Elenius}{John \& Pirkko}{1977--}
John Elenius, \textborn 26.01.1946 på Mietala, gift med Pirkko Korkiamäki, \textborn 05.11.1944 i Kauhava.
\begin{jhchildren}
  \item \jhperson{\jhname[Johanna]{Elenius, Johanna}}{22.02.1975}{}
  \item \jhperson{\jhname[Christer]{Elenius, Christer}}{24.07.1976}{}
\end{jhchildren}

Mer uppgifter om familjen under Mietala nr 18. John och Pirkko köpte bostadstomten av Nykarleby stad och byggde år 1977 sitt hus. Väggarna beställdes som element från Heikius Hus. Övrigt byggnadsarbete gjorde John själv med hjälp av pappa Bertel samt Elof Broo.

I samband med generationsväxling på jordbruket godkändes inte att bostaden var så långt från hemmanet, vilket resulterade i att Johns föräldrar och Johns familj bytte bostäder. Bertel och Anni bodde som pensionärer i huset. Bertel tillbringade mycket tid ute i skog och mark. Han var en ivrig fiskare och plockade bär varje sommar. Han dog 07.10.2006. Anni lever ensam kvar i huset.



\jhhouse{Ahlfors}{5:173}{Grötas}{10}{18}


\jhhousepic{153-05709.jpg}{Caroline Ahlfors och Tommy Karlsson}

\jhoccupant{Ahlfors}{Caroline \& Tommy}{2009--}
Ägare Stefan Ahlfors (Grötas 20), köpte lägenheten med byggnader av Bruno och Agnes Ahlfors 1997. Boende är Caroline  Maria Ahlfors, \textborn 11.05.1984 (nr 20), sambo med Tommy Andreas Karlsson, \textborn 08.02.1982 i Korsholm, de flyttade in år 2009.

Caroline är utbildad sjuksköterska, Tommy har teknisk utbildning, arbetar på Wärtsilä i Vasa. De har grundreparerat och förstorat gården.
\begin{jhchildren}
  \item \jhperson{\jhname[Sigge Anders]{Ahlfors, Sigge Anders}}{18.08.2011}{}
  \item \jhperson{\jhname[Molly Maria]{Ahlfors, Molly Maria}}{24.05.2015}{}
\end{jhchildren}


\jhoccupant{Ahlfors}{Bruno \& Agnes}{1954--\allowbreak 1997}
Henrik Bruno Ahlfors, \textborn 12.06.1921 (nr 20), gifte sig 1946 med Agnes Seppänen, \textborn 21.12.1920 i Salla. De köpte bostadstomten av Brunos mors dödsbo 1950 och tilläggsjord av brodern Tor 1955. Makarna byggde gården och ekonomiebyggnaderna 1952--54 samt hönshus 1955. Bruno hade reparationsverkstad i Silvast samt försäljning av motorcyklar, velocipeder och radioapparater. Han deltog i fortsättningskriget. Familjen emigrerade 1960 till Sverige.  Lägenheten med byggnader har varit uthyrd  åt bl.a. Raul och Maija Saviaro, Olof Laxéns familj, Martti Ahos familj och Pekka och Margit Nikkanen under åren 1961--\allowbreak 1997.
\begin{jhchildren}
  \item \jhperson{\jhname[Ulla Karin]{Ahlfors, Ulla Karin}}{10.12.1947}{}, gift, bor i Sverige
  \item \jhperson{\jhname[Krister Henrik]{Ahlfors, Krister Henrik}}{21.05.1952}{}, gift, bor i Sverige
\end{jhchildren}

Bruno \textdied 14.10.2008  ---  Agnes \textdied 09.05.2005, båda i Sverige



\jhhouse{Ahlfors}{5:173}{Grötas}{10}{20}


\jhhousepic{152-05710.jpg}{Stefan och Monica Ahlfors}

\jhoccupant{Ahlfors}{Stefan \& Monica}{1975--}
Ägare och boende Stefan Anders Ahlfors, \textborn 25.05.1952 och sambo Monica Ulla Alice Litens, \textborn 06.08.1953 i Nykarleby. Stefan övertog lägenheten 1975 av sina föräldrar, 35 ha skog och 17 ha odlat. I dag har han ca 100 ha odlat plus en del arrendejord. Föräldrarna hade avslutat mjölkproduktionen. Stefan började med svinuppfödning, avslutade den 1984, satsade på sädes- och potatisodling. Han byggde en sädestork 1988 samt förstorade ekonomiebyggnaderna år 2000, köpte Björkvalls svinhus (nr 20b), byggde om det till potatislager 2015.
Stefan rev det gamla bostadshuset och uppförde nytt på samma tomt 1985.
\begin{jhchildren}
  \item \jhperson{\jhname[Paul Kenneth]{Ahlfors, Paul Kenneth}}{20.01.1977}{}, agrolog, bor i Norge,chaufför
  \item \jhperson{\jhname[Carolina  Maria]{Ahlfors, Carolina  Maria}}{11.05.1984}{}, sjuksköterska (nr 18)
\end{jhchildren}


\jhhousepic{Birger Ahlfors.jpg}{Birger och Alice Ahlfors, huset numera rivet}

\jhoccupant{Ahlfors}{Birger \& Alice}{1953--\allowbreak 1975}
Anders Birger, \textborn 04.03.1919, gift 1945 med Maria Alice Lillbacka, \textborn 14.07.1926 i Oravais. Efter giftermålet ombyggdes boningshuset till två lägenheter och hemmanet omhändertogs tillsamman med änkan Anna Lovisa och bröderna, tills arvsskiftet 1953. Birger köpte tillbaka bröderna Tors och och Pauls arvslotter och utvidgade ekonomiebyggnaderna.

Birger tjänade landet i båda krigen, han lär ha varit en orädd matkusk. Birger och Alice var mjölkbönder.
\begin{jhchildren}
  \item \jhperson{\jhname[Solbritt Alice]{Ahlfors, Solbritt Alice}}{30.10.1945}{}, gift Visti, bor i Sverige
  \item \jhperson{\jhbold{\jhname[Stefan]{Ahlfors, Stefan}} Anders}{25.05 1952}{}
\end{jhchildren}

Birger \textdied 22.04.1979  ---  Alice flyttade till Älvbranten 1983, \textdied 14.08.2012


\jhoccupant{Ahlfors}{Axel \& Anna-Lovisa}{1926--\allowbreak 1953}
Axel Vilho Henriksson Grötas, tog namnet Ahlfors, \textborn 26.11.1892 i Amerika, gift 1915 med Anna-Lovisa Andersdt. Grötas, \textborn 22.12.1890 (Grötas, nr 111). Övertog hemmanet 1926, de var bönder med sädesodling och boskapsskötsel.
\begin{jhchildren}
  \item \jhperson{\jhbold{\jhname[Anders Birger]{Ahlfors, Anders Birger}}}{04.03.1919}{}
  \item \jhperson{\jhname[Henrik Bruno]{Ahlfors, Henrik Bruno}}{12.06.1921}{}, (nr 18)
  \item \jhperson{\jhname[Helge Axel]{Ahlfors, Helge Axel}}{15.01.1925}{15.03.1998}, ogift till Canada 1951
  \item \jhperson{\jhname[Paul Magnus]{Ahlfors, Paul Magnus}}{16.04.1927}{1994}, gift, flyttade till Jakobstad
  \item \jhperson{\jhname[Tor Erik]{Ahlfors, Tor Erik}}{12.11.1931}{2008}, gift, flyttade till Jakobstad
\end{jhchildren}

Axel \textdied 30.05.1942  ---  Anna-Lovisa \textdied 07.09.1954


\jhoccupant{Grötas}{Henrik \& Sofia}{1895--\allowbreak 1926}
Henrik Johansson Grötas, \textdied 04.05.1860  i Alahärmä, gift med Sofia Johanna Jakobsdt Mietala, \textborn 02.08.1864, ansöker om lagfart på 7/64 del mantal av Grötas skattehemman nr 5  1895, föräldrarna var säljare. Före det hade de varit på arbetsförtjänst i Amerika. De fyra första barnen var födda i Amerika. Makarna var jordbrukare med kreatur och sädesodling.
\begin{jhchildren}
  \item \jhperson{\jhname[Johan Jakob]{Grötas, Johan Jakob}}{04.06.1885 i Amerika}{}, på nytt till Amerika 1910
  \item \jhperson{\jhname[Henrik Joel]{Grötas, Henrik Joel}}{26.12.1889 i Amerika}{}, på nytt till Amerika
  \item \jhperson{\jhbold{\jhname[Axel Wilho]{Grötas, Axel Wilho}}}{26.04.1892 i Amerika}{30.05.1942}
  \item \jhperson{\jhname[Anna Sofia]{Grötas, Anna Sofia}}{22.10.1893 i Amerika}{18.12.1897}
  \item \jhperson{\jhname[Hanna Johanna]{Grötas, Hanna Johanna}}{04.11.1895 i Jeppo}{08.11.1897}
  \item \jhperson{\jhname[Otto Henrik]{Grötas, Otto Henrik}}{19.06.1898 i Jeppo}{25.12.1898}
  \item \jhperson{\jhname[Johannes Artur]{Grötas, Johannes Artur}}{19.06.1898 i Jeppo}{24.12.1898}
  \item \jhperson{\jhname[Hugo Selim]{Grötas, Hugo Selim}}{11.09.1909}{11.09.1954}, (Gunnar, nr 31b)
\end{jhchildren}

Henrik Johan \textdied 15.12.1909  ---  Sofia Johanna \textdied 23.11.1930


\jhoccupant{Grötas/Tyni}{Johan}{- 1895}
Johan Henriksson  Grötas/Tyni f. 22.02. 1833 i Alahärmä, gift med Anna Gustavsdt.f 03. 08.. 1838 i Alahärmä. De har ett fastebrev från 11.09 1886. Makarna var jordbrukare. Samtliga barn föddes i Alahärmä.
\begin{jhchildren}
  \item \jhperson{\jhbold{\jhname[Henrik Johan]{Grötas/Tyni, Henrik Johan}}}{04.05.1860}{}
  \item \jhperson{\jhname[Vilhelmina]{Grötas/Tyni, Vilhelmina}}{22.12.1861}{}
  \item \jhperson{\jhname[Johan]{Grötas/Tyni, Johan}}{31.12.1863}{}
  \item \jhperson{\jhname[Gustav]{Grötas/Tyni, Gustav}}{07.12.1865}{}
  \item \jhperson{\jhname[Adolfiina]{Grötas/Tyni, Adolfiina}}{21.04.1870}{}
  \item \jhperson{\jhname[Sanna Maria]{Grötas/Tyni, Sanna Maria}}{23.12.1872}{}
  \item \jhperson{\jhname[Oscar]{Grötas/Tyni, Oscar}}{12.11.1874}{}
  \item \jhperson{\jhname[Amanda]{Grötas/Tyni, Amanda}}{06.04.1877}{}
\end{jhchildren}

Johan \textdied 12.01.1901  ---  Anna \textdied 16.01.1902



\jhhouse{Vägskiftet}{5:127}{Grötas}{10}{21}


\jhhousepic{159-05712.jpg}{Maija och Raul Saviaro}

\jhoccupant{Saviaro}{Maija \& Raul}{1993--}
Ägare och boende änkan Maija Amalia Saviaro, \textborn Kojonen 19.03.1935 i Alahärmä, som tillsammans med maken Raul Gustav Saviaro, \textborn 25.01.1935, sålde sitt bostadshus till sonen (nr 23) och byggde nytt bostadshus 1993, på tomten de hade köpt 1990 av Maria Roos dödsbo.

Barn: se nr 23. Raul \textdied 04.07.2011.

Lorem ipsum dolor sit amet, consectetur adipiscing elit, sed do eiusmod tempor incididunt ut labore et dolore magna aliqua. Ut enim ad minim veniam, quis nostrud exercitation ullamco laboris nisi ut aliquip ex ea commodo consequat. Duis aute irure dolor in reprehenderit in voluptate velit esse cillum dolore eu fugiat nulla pariatur. Excepteur sint occaecat cupidatat non proident, sunt in culpa qui officia deserunt mollit anim id est laborum. % FIXME




\jhhouse{Lind}{5:123}{Grötas}{10}{23}


\jhhousepic{158-05711.jpg}{Jarmo och Marina Saviaro}

\jhoccupant{Saviaro}{Jarmo \& Marina}{1993--}
Ägare och boende Jarmo Saviaro, \textborn 04.05.1963, gift med Marina Elisabeth, \textborn 07.06.1968 i Esbo. De köpte huset av Jarmos föräldrar 1993. Jarmo arbetar på KWH-Mirka som arbetsledare och Marina som försäljare på K-Rauta i Alahärmä.
\begin{jhchildren}
  \item \jhperson{\jhname[Henri Anders]{Saviaro, Henri Anders}}{10.04.1990}{}, VVS-montör
  \item \jhperson{\jhname[Kristian Raul Erik]{Saviaro, Kristian Raul Erik}}{12.05.1992}{}, bilmekaniker
  \item \jhperson{\jhname[Mikael Gustav]{Saviaro, Mikael Gustav}}{22.04.1997}{}, bilmekaniker
\end{jhchildren}
Marina har flyttat och bor numera i Alahärmä.


\jhoccupant{Saviaro}{Raul \& Maija}{1968--\allowbreak 1993}
Raul Gustav, \textborn 25.01 1936 i Jeppo, och makan Maija Amalia Kojonen, \textborn 19.03.1935 i Alahärmä, köpte bostadstomten 1968 av Edvin Wikströms dödsbo och byggde bostadshuset. År 1970 köpte de Selma Eklunds lägenhet (nr 109), rev husen och byggde ekonomiebyggnaden. Raul var byggnadsarbetare på Keppo pälsdjursfarm i 30 år, därtill bedrev han tillsammans med hustrun bl.a. småbruk, svin- och pälsdjursfarm. Sonen Jarmo köpte lägenheten 1993. Raul och Maija byggde nytt hus (nr 21).
\begin{jhchildren}
  \item \jhperson{\jhname[Jari Gustav]{Saviaro, Jari Gustav}}{10.04.1960}{}, se nr 19
  \item \jhperson{\jhbold{\jhname[Jarmo Juhani]{Saviaro, Jarmo Juhani}}}{04.05.1963}{}
\end{jhchildren}



\jhhouse{Sundholm}{5:9}{Grötas}{10}{24}


\jhoccupant{Nyberg}{Uno \& Leena}{2005--}
Nuvarande ägare Uno Nyberg och makan Leena köpte lägenheten 2005 av Hans Nybyggar. De hade bott på hyra i gården från år 1998, då de flyttade från hus nr 22. Efter köpet har Uno reparerat byggnaderna.\jhvspace{}


\jhhousepic{157-05715.jpg}{Uno och Leena Nyberg}

\jhoccupant{Nybyggar}{Hans \& Anette}{1987--\allowbreak 2005}
Hans Nybygggar, \textborn 08.02.1966 i Jeppo, sambo med Anette Margareta Johansson, \textborn 06.09.1964 i Göteborg, köpte lägenheten av Risto Yliaho, som köpte nr 34. Hans har varit truckförare på Jeppo Potatis och lastbilschaufför åt olika transportbolag både i Finland och Sverige. Paret har ett barn, Johan David, \textborn 13.05.1992. De separerade och Hans flyttade till Sverige, där han gifte sig 1994 och fick två barn.

Anette är också långtradarchaufför, hon flyttade till (Mietala --).  Lägenheten på Grötas uthyrdes till Uno Nyberg 1998, och han köpte den år 2005.
Hans \textdied år 2007.


\jhoccupant{Yliaho}{Risto \& Susanne}{1986--\allowbreak 1987}
Yliaho Risto med familj köpte lägenheten av Gunnar Söderlund år 1986, bodde i huset två år, tills de köpte Olof Dahlströms lägenhet, nr 34.\jhvspace{}


\jhoccupant{Söderlund}{Gunnar \& Rauha}{1950--\allowbreak 1986}
Gunnar Söderlund, \textborn 17.12.1925 i Esse, köpte lägenheten av Adolfiina (Fiina) Rintalas dödsbo 1950. Han gifte sig med Rauha Maria Perikanta, \textborn 19.03.1934 i Sordavala. Gunnar var skomakare till yrket. Han arbetade på Keppos pälsdjursfarm och senare på KWH-Mirka. Han byggde hönshus 1957, förstorade bostaden och rev den gamla uthuslängan. Åren 1964--\allowbreak 1970 bodde de i Jakobstad, där han arbetade som gårdskarl. De återvände till Grötas 1970, stannade ett år och flyttade 1971 till Vasa, där han fortsatte  att arbeta som  gårdskarl åt ett fastighetsbolag. Nu är han pensionär och bor i Vasa.

Gunnar deltog i fortsättningskriget, blev sårad och är krigsinvalid. Som veteran blev han år 2016 tilldelad Finlands Vita Ros orden medalj. Makan Rauha dog 2001 i Vasa.

Barn: Yngve Elof, \textborn  30.06 1963 i Jeppo, bor i Vasa.

Gunnar hade hyrt ut huset under sin vistelse i Jakobstad till bl.a. Tor och Tulikki Elenius (Fors nr 98), Esko och Kaarin Kalijärvi (Romar nr 29), Marja-Liisa och Mikko Erkinheimo, senare 1974--76 till Curt och Monica Elenius och 1977--85 till Ann-Christin och Lars Dahlkvist (Silvast nr 77).


\jhoccupant{Rintala}{Jaakko \& Fiina}{1940--\allowbreak 1950}
Skräddaren Jaakko Gustavsson Rintala, \textborn 07.01.1887 i Nurmo, gift med Adolfiina (Fiina) Kaarna, \textborn 01.07.1887 i Lappo, tidigare gift Halme. Jaakko och Fiina  köpte lägenheten 1936 av Viktor Sundholm men flyttade in först 1940. Fiina hade 3 barn från sitt första äktenskap. Efter försäljningen av lägenheten 1950 flyttade Fiina till (Silv.     ).
\begin{jhchildren}
  \item \jhperson{\jhname[Wellamo]{Rintala, Wellamo}}{10.06.1912}{}, frisörska på Silvast, emigr. t. Sverige
  \item \jhperson{\jhname[Osmo]{Rintala, Osmo}}{23.03.1916}{}, emigrerade till Australien
  \item \jhperson{\jhname[Oiva Jaakko]{Rintala, Oiva Jaakko}}{22.09.1918}{}, gift (Silvast      )
\end{jhchildren}
Jaakko \textdied 13.02.1943

\jhoccupant{Sundholm}{Viktor \& Ellen}{1926--\allowbreak 1940}
Viktor Johannes Sundholm, \textborn 07.09.1903 i Nykarleby, gifte sig 1931 med Ellen Anna Alfhild Westerlund, \textborn 11.01.1906 på Silvast. De köpte lägenheten Sundholm 5:9 med fastigheter utgörande 0,0009 mantal av Grötas skattehemman 5, den 01.03.1926 av Viktors mormor, änkan Anna-Lovisa Jakobsson, för 10.000 mark.

Viktor var diversearbetare. Ellen dog 18.02.1940 och deras son Magnus, \textborn 09.03.1938, dog samma dag. Viktor flyttade till Gamlakarleby, där han dog.


\jhoccupant{Jakobsson}{Erik \& Lovisa}{- 1926}
Backstugusittare Erik Johan Jakobsson, \textborn 01.11.1851 i Jeppo, gift med Anna-Lovisa Andersdotter, \textborn 16.06.1849 i Jeppo. År 1921 löste änkan Lovisa in området av Anders Fagerholm och Anselm Grötas.
Maken Erik \textdied 15.08.1916  ---  Anna Lovisa \textdied 08.05.1926.
\begin{jhchildren}
  \item \jhperson{\jhname[Johannes]{Jakobsson, Johannes}}{31.12.1877}{}
  \item \jhperson{\jhname[Maria-Sofia]{Jakobsson, Maria-Sofia}}{01.05.1879}{1965 på Finskas}, gift med J.Andersson
  \item \jhperson{\jhname[Emil]{Jakobsson, Emil}}{12.01.1882}{}
  \item \jhperson{\jhname[Jakob]{Jakobsson, Jakob}}{28.04.1886}{}
  \item \jhperson{\jhname[Anna-Lovisa]{Jakobsson, Anna-Lovisa}}{28.05.1891}{}
\end{jhchildren}

Enligt  mantalslängden ägde Gustav Jakobsson Böös torpet 1870--\allowbreak 1906 och i mantalslängden 1910 finns banvakten Anders Grötas som ägare av torpet enligt köpebrev 1906. Torparna har dock haft kontrakt med ägarna av bakstuguområdet.



\jhhouse{Björkqvist}{5:32}{Grötas}{10}{126b-a}


\jhhousepic{Ida Bjorkqvist.jpeg}{Ida Björkqvists hus vid vägen. Jakob och Brita Sandlin på hemväg efter besök?}

\jhoccupant{Björkqvist}{Ida}{1956--\allowbreak 1970}
Bostadshuset nr \jhbold{126b} revs 1982 av Martti Aho, som köpt tomten och husen av Artur och Siviä Björkqvist 1979. Senast boende var änkan Ida Maria Björkqvist, hon dog 24.12.1973.\jhvspace{}


\jhoccupant{Björkqvist}{Ivar \& Vieno}{1932--\allowbreak 1956}
Ivar Björkqvist \textborn 06.08.1912, och makan Vieno Aili Johanna Grötas, \textborn 10.05.1911 (nr 131) gifte sig 1932 och flyttade in i bostaden år 1933. De byggde nytt bostadshus (Grötas, nr 32) år 1954.
\begin{jhchildren}
  \item \jhperson{\jhname[Ragne Valdine]{Björkqvist, Ragne Valdine}}{01.05.1935}{11.03.2013}, gift Kujanpää i Jakobstad
  \item \jhperson{\jhname[Anita Lisbet]{Björkqvist, Anita Lisbet}}{22.08.1936}{}, gift Stenbacka, Jeppo
  \item \jhperson{\jhname[Paul Ivar]{Björkqvist, Paul Ivar}}{09.03.1938}{}, gift med Carita, född Ström, Jeppo
\end{jhchildren}


\jhoccupant{Björkqvist}{Emil \& Ida}{1910--\allowbreak 1932}
Emil Johansson Grötas, tog namnet Björkqvist, \textborn 28.03.1884 (nr 107), och makan Ida Maria Fagerholm (Grötas, nr 128), \textborn 18.03.1883, gift 1905, hade tillhandlat sig Emils föräldrars jordlägenhet år 1909 och av brodern Johannes Henrik år 1910, 39 ha varav 13 ha odlad. De byggde bostadshus och ekonomiebyggnader. De var jordbrukare.

Emil var på arbetsförtjänst till Amerika 1925. År 1932 byggde de nytt bostadshus på samma tomt, sonen Ivar med fru flyttade in 1933.
\begin{jhchildren}
  \item \jhperson{\jhname[Johannes]{Björkqvist, Johannes}}{24.06.1906}{1974}, ogift, emigrerade till Canada
  \item \jhperson{\jhname[Anders William]{Björkqvist, Anders William}}{15.01.1910}{1941}, ogift
  \item \jhperson{\jhname[Ivar]{Björkqvist, Ivar}}{06.08.1912}{}, (nr 131)
  \item \jhperson{\jhbold{\jhname[Artur Selim]{Björkqvist, Artur Selim}}}{07.09.1915}{}, (nr 126a)
  \item \jhperson{\jhname[Ingrid Maria]{Björkqvist, Ingrid Maria}}{11.04.1918}{}, gift Eriksson Soklot
\end{jhchildren}


\jhhouse{Björkqvist}{5:32}{Grötas}{10}{126a} på samma lägenhet


\jhhousepic{Artur Bjorkqvist.jpg}{Artur och Siviä Björkqvist}

\jhoccupant{Björkqvist}{Artur \& Siviä}{1955--\allowbreak 1975}
Bostadshuset revs av Martti Aho 1984, som köpt tomten 1979. Senast boende i huset var Artur Selim Björkqvist, \textborn  07.09.1915, gift med Siviä Rintala, \textborn 04.11.1924 i Jurva. Artur övertog lägenheten 1955, före det var han på arbetsförtjänst i Canada under åren 1952--\allowbreak 1955. Artur deltog i båda krigen, blev sårad i slutstriderna.  Han var  en trotjänare i föreningslivet och idrotten, både som aktiv och talkoarbetare, man minns honom bäst som fotbollsmålvakt.  Familjen avslutade jordbruket och sålde till staten och kommunen 1975. De köpte sig en lägenhet i Jakobstad och flyttade dit. Tomten med två bostadshus och ekonomiebyggnader i Grötas sålde de till Martti Aho 1979.
Barn: fosterbarn Hannu

Artur \textdied 13.01.1983, Siviä gifte om sig, flyttade till Vörå där hon dog.


\jhoccupant{Björkqvist}{Emil \& Ida}{1932--\allowbreak 1955}
Anders Emil med hustrun Ida Maria och barnen (nr 126b) byggde bostadshuset och förstorade ekonomiebyggnaderna 1932--33. Enligt jordregistret 1935 ägde de 48 ha varav 22 ha odlat.
Barn: Se nr 126 b ovan.

Emil \textdied 02.02.1941  ---  Ida Maria \textdied 24.12.1973



\jhhouse{Farmen}{.893-408-15-2}{Grötas}{10}{26}


\jhhousepic{160-05716.jpg}{Martti och Hilkka Aho}

\jhoccupant{Aho}{Martti \& Hilkka}{1979--}
Ägare och boende Martti Aho, \textborn 1951 i Jeppo, gift 1971 med Hilkka Linnanmäki, \textborn 1952 i Alajärvi. De köpte bostadstomten med två bostadshus och ekonomiebyggnader (nr 126a,126b) av Artur och Siviä Björkqvist år 1979 och byggde nytt bostadshus år 1981. Ekonomiebyggnaderna byggdes om, de sysslade några år med kalkonuppfödning. Martti har jobbat som lastbilschaufför och Hilkka är utbildad kokerska och har arbetat bl.a. som hemhjälpare.
\begin{jhchildren}
  \item \jhperson{\jhname[Veli Pekka]{Aho, Veli Pekka}}{27.01.1973,  chaufför/elektriker}{}
  \item \jhperson{\jhname[Johanna Marjaana]{Aho, Johanna Marjaana}}{02.04.1974, hälsovårdare}{}
  \item \jhperson{\jhname[Marika Pauliina]{Aho, Marika Pauliina}}{04.07.1975,  närvårdare}{}
  \item \jhperson{\jhname[Minna Hannele]{Aho, Minna Hannele}}{06.10.1976,  sömmerska}{}
  \item \jhperson{\jhname[Janne Antero]{Aho, Janne Antero}}{23.01.1978,  byggnadsarbetare}{}
  \item \jhperson{\jhname[Elina Kristiina]{Aho, Elina Kristiina}}{28.06.1979,  konditor}{}
  \item \jhperson{\jhname[Elisa Henna Maaria]{Aho, Elisa Henna Maaria}}{15.03.1981,  laborant}{}
  \item \jhperson{\jhname[Katri Marjut Inkeri]{Aho, Katri Marjut Inkeri}}{29.08.1982,  merkonom}{}
  \item \jhperson{\jhname[Kaisa Annika]{Aho, Kaisa Annika}}{05.12.1983,  merkonom}{}
  \item \jhperson{\jhname[Mikko Petteri]{Aho, Mikko Petteri}}{29.09.1985,  byggnadsarbetare}{}
  \item \jhperson{\jhname[Jukka Tapio Mikael]{Aho, Jukka Tapio Mikael}}{11.12.1987,  bilmekaniker}{}
  \item \jhperson{\jhname[Vesa Jaakko Juhana]{Aho, Vesa Jaakko Juhana}}{03.05.1989,  byggnadsarbetare}{}
  \item \jhperson{\jhname[Antti Valtteri]{Aho, Antti Valtteri}}{01.11.1990,  svetsare}{}
  \item \jhperson{\jhname[Joonas Kristian Samuel]{Aho, Joonas Kristian Samuel}}{03.01.1994,  studerande}{}
  \item \jhperson{\jhname[Jussi Markus Henrikki]{Aho, Jussi Markus Henrikki}}{10.10.1996,  studerande}{}
\end{jhchildren}



<--- se KARTA nr 11 --->




\jhhouse{Nyberg}{5:152}{Grötas}{11}{22}


\jhoccupant{Nyberg}{Sven}{1996--}
Sven Uno Matias, \textborn 11.09.1970 i Jeppo, köpte bostadshuset med tomt och byggnader 1996 av fadern Uno. Sven rev bostadshuset 2015 och uppförde nytt på samma tomt. Sven är ogift, jobbar som arbetsledare på Finskas Br, Ab.\jhvspace{}


\jhhousepic{155-05714.jpg}{Sven Nyberg}

\jhoccupant{Nyberg}{Uno \& Leena}{1966--\allowbreak 1996}
Uno Elis, \textborn 25.08.1943 i Jeppo. Vid arvsskiftet 1966 tog han över lägenheten med byggnader. Han gifte sig 1969 med Leena Järvenpää, \textborn 26.12.1948 i Alahärmä. Uno, tillsammans med brodern Leo, som gifte sig med Hjördis Häggman från Vörå och flyttade 1965 till Jakobstad, skötte om  dödsboet under tiden 1959--\allowbreak 1966. De rev bort den gamla uthuslängan med stall och bod och byggde en ny. Uno avslutade boskapsuppfödningen i början på 1970-talet och satsade på svin. Han byggde om fähuset och renoverade bostadshuset. Problem med hälsovårdsmyndigheterna gjorde att svinuppfödningen avslutades. År 1993 avslutades jordbruket, allt såldes förutom bostaden och ekonomiebyggnaderna med omkringliggande jordområde. Uno var många år i tillfälligt arbete på Keppo pälsdjursfarm samt åt Johan Slangar på svinfarm. Förutom jordbruksarbete var makan Leena postutdelare 32 år i Jeppo och senare i Nykarleby stad, tills hon blev pensionerad. Uno och Leena flyttade till annat hus (nr 24) 1998, som de köpte 2005.
\begin{jhchildren}
  \item \jhperson{\jhname[Kaj Jaakko Emil]{Nyberg, Kaj Jaakko Emil}}{13.07.1969, arb. på KWH-Mirka}{}
  \item \jhperson{\jhbold{\jhname[Sven]{Nyberg, Sven}} Uno Matias}{11.09.1970, arb. på Finskas Br. Ab}{}
  \item \jhperson{\jhname[Nina Sisko Maria]{Nyberg, Nina Sisko Maria}}{04.08.1976, gift, Ek Pensala}{}
\end{jhchildren}


\jhhousepic{Nyberg o Westin Johannes.jpg}{Huset Nyberg t.v. och gaveln av Westin Johannes t.h.}

\jhoccupant{Nyberg}{Emil \& Aino, Leander}{1935--\allowbreak 1959}
Emil Nyberg, \textborn 11.01.1892 i Jeppo och brodern Leander, \textborn 03.02.1884 i Jeppo, övertog hemmanet av fadern Carl år 1935. Arealen var 58 ha, varav 18 ha odlad. Emil gifte sig 1939, med änkan Aino Adolfiina Sipponen, \textborn 26.01.1898 i Alahärmä. Lägenheten var en typisk jordbrukslägenhet med spannmålsodling, boskap, hästar, får och grisar.
\begin{jhchildren}
  \item \jhperson{\jhname[Anders Leo]{Nyberg, Anders Leo}}{13.12.1940, gift, bor i Jakobstad}{}
  \item \jhperson{\jhbold{\jhname[Uno]{Nyberg, Uno}} Elis}{25.08.1943, flyttade till nr 24}{}
\end{jhchildren}

Makan Ainos barn från första äktenskapet, Holkkola:
\begin{jhchildren}
  \item \jhperson{\jhname[Heino Jakob]{Holkkola, Heino Jakob}}{27.03.1928}{16.06.2008}, begr. i Veberöd, Skåne
  \item \jhperson{\jhname[Tauno Esaias]{Holkkola, Tauno Esaias}}{14.02.1930}{30.10.2007}, '' i Staffanstorp, Skåne
  \item \jhperson{\jhname[Mauno Johannes]{Holkkola, Mauno Johannes}}{12.11.1932}{}, i Karlskoga
\end{jhchildren}

Aino dog 26.08.1954. Efter hennes död inlöste maken Emil med sönerna Leo och Uno år 1955 arvslotterna av Ainos barn från  första äktenskapet.

Emil \textdied 26.08.1959  ---  Leander \textdied 19.07.1962, ogift.


\jhoccupant{Nyberg}{Carl \&  Anna-Sofia}{1881--\allowbreak 1935}
Carl Mattson Sandberg, \textborn 08.02.1854 på Finskas, tog namnet Nyberg. Efter några års arbetsförtjänst i Amerika, köpte han vid hemkomsten år 1881, tillsammans med systern Caisa och hennes man Anders Abrahamsson Ekoluoma, en lägenhet på Grötas 21/128 mantal av Grötas skattehemman nr. 5 av Jonathan von Essen. Det finns inga uppgifter hur man disponerade bostadshuset och övriga byggnader, troligtvis bodde Anders och Caisa i lägenheten (nr 104) och Carl byggde nytt. Carls syster Caisa dog 1889, följande år sålde maken Anders Ekoluoma deras andel av lägenheten till Gustav Hansson Ryss och flyttade själv med familj till Alahärmä. Samuel Roos köpte år 1895 Gustav Hansson Ryss' del (se nr 104).

År 1883 gifte sig Carl med  Anna-Lovisa Simonsdr. från Pedersöre, hon dog 1888. Carl ingick nytt äktenskap med Anna-Sofia Ryss, \textborn 30.10.1857 i Pedersöre. De var bönder. Enligt hörsägen lär Carl ha varit en bra berättare och lärare för Grötas' ungkarlar. Av familjens barn är det äldsta fött i första giftet, resten i andra giftet.
\begin{jhchildren}
  \item \jhperson{\jhbold{\jhname[Leander]{Nyberg, Leander}}}{03.02.1884}{19.07.1962}, ogift
  \item \jhperson{\jhbold{\jhname[Emil]{Nyberg, Emil}}}{11.01.1892}{26.08.1959}
  \item \jhperson{\jhname[Johannes]{Nyberg, Johannes}}{25.12.1894}{1895}
  \item \jhperson{\jhname[Anna-Lovisa]{Nyberg, Anna-Lovisa}}{21.11.1896}{1983}, ogift, dövstum
\end{jhchildren}

Carl \textdied 13.02.1938  ---  Anna-Sofia \textdied 23.08.1936


\jhoccupant{von Essen}{Jonathan}{1879--\allowbreak 1881}
Se Grötas nr 104 för mera information.\jhvspace{}



\jhhouse{Westin}{5:8}{Grötas}{11}{107}


\jhoccupant{Westin}{Johan \& Anna-Lovisa}{1910--\allowbreak 1933}
Huset som ses i bild \ref{pic:westin} var deras bostad. Senast inneboende var Johan Henrik Johansson Grötas, senare Westin, \textborn 25.12.1881, gifte sig 1903 med Anna Lovisa Andersdt. Eklund, \textborn 05.12.1882 (karta 10, nr 109). Johan och Anna-Lovisa vistades i Amerika åren 1903--09 och Johan ensam 1910--\allowbreak 1911. År 1910 tillhandlade sig Johan hemgården och brodern Emil jordlägenheten. Enligt köpebrev av den 22.04.1913, köpte makarna 0,0821 mantal av Grötas skattehemman nr 5 (d.v.s. nr 109), säljare är Anna Lovisas föräldrar Anders Eriksson och hustrun Anna Gustavsdt. Ruotsala. Efter storskiftet 1932--\allowbreak 1933 flyttade familjen bostadshuset och byggde nya ekonomiebyggnader på ny plats (karta 10, nr 108). Familjen med hemmavarande barn och kreatur flyttade med.
\begin{jhchildren}
  \item \jhperson{\jhname[Johannes Edvin]{Westin, Johannes Edvin}}{29.10.1904 i USA}{1983}, fsk.lär. Purmo, Jeppo
  \item \jhperson{\jhname[Anders Ivar]{Westin, Anders Ivar}}{21.04.1906 i Amerika (nr 101)}{}
  \item \jhperson{\jhname[Ester Maria]{Westin, Ester Maria}}{30.06.1908 i Amerika}{10.12.1909}
  \item \jhperson{\jhbold{\jhname[Henrik Valfrid]{Westin, Henrik Valfrid}}}{21.05.1910 i Jeppo (nr 8, 108)}{}
  \item \jhperson{\jhbold{\jhname[Elis Verner]{Westin, Elis Verner}}}{29.08.1912 i Jeppo ogift (nr 108)}{}
  \item \jhperson{\jhname[Agnes Elvira]{Westin, Agnes Elvira}}{11.12.1914 i Jeppo, gift Sandberg}{2006}
  \item \jhperson{\jhname[Erik Runar]{Westin, Erik Runar}}{19.05.1917 i Jeppo, gift i Ingå}{16.11.1968}
  \item \jhperson{\jhname[Artur Evert]{Westin, Artur Evert}}{13.05.1920 i Jeppo, sambo i Sverige}{}
  \item \jhperson{\jhname[Lennart Wilhelm]{Westin, Lennart Wilhelm}}{02.11.1922 i Jeppo, g., frånsk., lektor i Borgå}{}
  \item \jhperson{\jhname[Hjördis Irene]{Westin, Hjördis Irene}}{29.01.1926 i Jeppo, gift Finskas}{15.08.2013}
\end{jhchildren}


\jhoccupant{Grötas}{Johan \& Maria}{1894--\allowbreak 1910}
Johan Henriksson Grötas, \textborn 19.07.1859, gift 1879 med Maria Danielsdt. Levelä, \textborn 14.01.1854. Johan och Maria blev bönder på Grötas sedan han övertagit hemmanet av morbrodern Erik efter dennes död 1894. Ett år senare köpte de, änkan Sofia Gustavsdt. Tollikos 61/768 mantal av Grötas 5.  Efter överlåtande av egendomen 1910, flyttade de med yngsta sonen William till torpet (nr 113). Johan var på arbetsförtjänst till Amerika år 1887.
\begin{jhchildren}
  \item \jhperson{\jhname[Brita Sofia]{Grötas, Brita Sofia}}{29.01.1880}{}, gift Sandlin (nr 127)
  \item \jhperson{\jhbold{\jhname[Johan Henrik]{Grötas, Johan Henrik}}}{25.12.1881}{}, (nr 108, 107)
  \item \jhperson{\jhbold{\jhname[Anders Emil]{Grötas, Anders Emil}}}{28.03.1884}{}, (nr 112,113)
  \item \jhperson{\jhname[Erik William]{Grötas, Erik William}}{24.08.1886}{1892}
  \item \jhperson{\jhname[August William]{Grötas, August William}}{21.07.1896}{}, (nr 113)
\end{jhchildren}

Johan Henriksson \textdied 09.04.1911  ---  Maria \textdied 12.07.1942


\jhoccupant{Grötas}{Johannes \& Sanna Lisa}{1880--\allowbreak 1894}
Johannes Eriksson Grötas, \textborn 07.06.1860 på Grötas, son till Erik Olof Olofsson, gift med Sanna Lisa Andersdt. Silvast, \textborn 23.09.1863, köpte år 1880 61/168 mantal av Grötas 5 av fadern Erik Olofsson,  familjen emigrerade till Amerika 1894.
Barn: Anna-Lovisa, \textborn 23.08.1893 i Jeppo, \textdied 1977 i Amerika

Johannes \textdied 1905  ---  Sanna-Lisa \textdied 1908 i Amerika


\jhoccupant{Grötas}{Erik \& Anna-Maria}{1855--\allowbreak 1880}
Erik Olofsson Grötas, \textborn 06.05.1835, son till bonden Olof Olofsson Grötas, gift med Anna-Maria Johansdt. Öhman, \textborn 03.10.1837 på Silvast.

Bröderna Erik och Matts övertog hemmanet år 1855 av fadern Olof. (Samtidigt tryggas systern Brita-Lenas och hennes man torparen Henrik Johanssson vid sitt torparkontrakt av år 1854.) Brodern Matts gifte sig med Maria Andersdotter Skog, \textborn 11.08.1837, de bosatte sig på Skog. Erik bosatte sig på hemmanet och köpte till jord 1870 av Anders Anderson Grötas 61/768 mantal.

Hustrun Anna-Maria \textdied före år 1887  ---  Erik \textdied 1894. Alla barn emigrerade till Amerika. Efter Eriks död övertogs hemmanet av systersonen Johan Henriksson Grötas.
\begin{jhchildren}
  \item \jhperson{\jhname[Anna-Lena]{Grötas, Anna-Lena}}{12.09.1858}{01.10.1859}
  \item \jhperson{\jhbold{\jhname[Johannes]{Grötas, Johannes}}}{07.06.1860}{1905 i Amerika}
  \item \jhperson{\jhname[Erik]{Grötas, Erik}}{25.09.1862}{07.04.1889 i Amerika}
  \item \jhperson{\jhname[Matts]{Grötas, Matts}}{06.06.1865}{}, gift, till Amerika 1890
  \item \jhperson{\jhname[Anna-Lovisa]{Grötas, Anna-Lovisa}}{23.10.1867}{17.12.1867}
  \item \jhperson{\jhname[Anders Gustav]{Grötas, Anders Gustav}}{11.05.1871 ?}{}
  \item \jhperson{\jhname[Jakob Wilhelm]{Grötas, Jakob Wilhelm}}{05.10.1876}{},dog i Amerika,ogift
\end{jhchildren}


\jhoccupant{Grötas}{Olof \& Caisa-Greta, Maria, Magdalena}{1824--\allowbreak 1855}
Olof Olofsson Grötas, \textborn 25.05.1802 på Backhemming i Soklot, köpte en lägenhet på 5/32 mantal av Grötas 5, år 1824 av Henrik Danielsson och hustrun Maria Simonsdt. Olof var gift 3 ggr.: 1-gifte med Caisa-Greta Henriksdt. Grötas, \textborn 28.08.1804, \textdied 24.02.1828, 2-gifte med Maria Eriksdt. Flod, \textborn 17.05.1811, \textdied 07.12.1830, 3-gifte med Magdalena Mattsdt. Pesonen, \textborn 06.10.1806, \textdied 20.06.1854.

De var jordbrukare. Av vem och när bostaden med ekonomiebyggnader byggts vet man ej. Olof \textdied 02.01.1867 på Grötas.
\begin{jhchildren}
  \item \jhperson{\jhname[Erik]{Grötas, Erik}}{05.09.1826}{07.08.1828}, 1:a giftet
  \item \jhperson{\jhname[Henrik]{Grötas, Henrik}}{17.12.1827}{07.08.1828}, 1:a giftet
  \item \jhperson{\jhbold{\jhname[Brita-Lena]{Grötas, Brita-Lena}}}{07.09.1829}{}, 2:a giftet, g m Henrik Johansson (nr 120)
  \item \jhperson{\jhname[Maija-Lovisa]{Grötas, Maija-Lovisa}}{08.12.1832}{dog 01.05.1833}, 3-giftet
  \item \jhperson{\jhname[Matts]{Grötas, Matts}}{11.12.1833}{23.03.1867 på Skog i Jeppo}, 3-giftet
  \item \jhperson{\jhbold{\jhname[Erik]{Grötas, Erik}}}{06.05.1835}{19.04.1894}, 3-giftet
  \item \jhperson{\jhname[Anna]{Grötas, Anna}}{04.11.1837}{06.02.1838}, 3-giftet
  \item \jhperson{\jhname[Lisa]{Grötas, Lisa}}{02.08.1841, gift Eklund i Lassila}{09.06.1930}, 3-giftet
  \item \jhperson{\jhname[Sofia]{Grötas, Sofia}}{11.04.1847}{}, emigrerade till Sverige, 3-giftet
\end{jhchildren}



\jhhouse{Forslund}{5:14}{Grötas}{11}{105}


\jhhousepic{Saviaro 105.jpeg}{Anders och Anni Saviaro}

\jhoccupant{Saviaro}{Anders \& Anni}{1953--\allowbreak 1985}
Huset och ekonomiebyggnaderna revs 1996 och 1993 av Raul Saviaro. Senast boende var Anders (Antti) Gustav Saviaro och Anni Aino (nr 106). Johan Forsblom hade testamenterat egendomen till dem 1952. De reparerade bostaden och förstorade ekonomiebyggnaden. Sonen Raul övertog lägenheten 1977.

Anders \textdied 06.10.1974 --- Anni \textdied 25.05.1985


\jhoccupant{Forsblom}{Johan \& Hilda}{1910--\allowbreak 1952}
Johan Abramhamsson Forsblom, \textborn 07.08.1875 i Nykarleby, gift 1903 med Hilda Maria Andersdt., \textborn 07.09.1879 (nr 109). Makarna var på arbetsförtjänst i Amerika 1904--\allowbreak 1910, fyra av deras barn var födda i Amerika. Enlig legonämndens protokoll av den 13.04 1928, har Johan Forsblom fått inlösa gårdstomten, som han hade gjort arredekontrakt  på  med Anders Emil och Ida Maria Grötas (Björkqvist), 07 maj 1910. Johan byggde bostadshus och ekonomiebyggnaderna, han var småbrukare och byggnadsarbetare.
\begin{jhchildren}
  \item \jhperson{\jhname[Signe Maria]{Forsblom, Signe Maria}}{18.03.1905 i Amerika}{1922}
  \item \jhperson{\jhname[Anna Elvira]{Forsblom, Anna Elvira}}{10.12.1906 i Amerika }{1920}
  \item \jhperson{\jhname[Johannes Richard]{Forsblom, Johannes Richard}}{15.01.1909 i Amerika}{1927}
  \item \jhperson{\jhname[Anders Ivar]{Forsblom, Anders Ivar}}{19.09.1910 i Amerika}{1933}
  \item \jhperson{\jhname[Edit Johanna]{Forsblom, Edit Johanna}}{24.08.1912 i Jakobstad}{10.03.1943}
\end{jhchildren}

Makan Hilda Maria \textdied 20.06.1916 vid 36 års ålder  ---  Johan \textdied 13.05.1952



\jhhouse{Mellangård}{5:151}{Grötas}{11}{25}


\jhhousepic{Grotas25-SandlinS.jpg}{Selma begrundar utsikten}

\jhoccupant{Sandlin}{Jessica}{2012--}
Jessica Sandlin, \textborn 21.11.1983  i Jakobstad, fick 13.09.2012 huset i gåva av Hilda och Edit Sandlin. Huset är idag obebott.\jhvspace{}


\jhoccupant{Sandlin}{Johannes, Hilda \& Edit}{1965--\allowbreak 2012}
Johannes Sandlin, \textborn 03.05.1903 på Grötas, övertog tillsammans med systrarna Hilda, \textborn 27.06.1909 och Edit, \textborn 21.07.1916 gården efter Selmas död.\jhvspace{}


\jhoccupant{Sandlin}{Selma}{1920--\allowbreak 1965}
Selma Andersdotter Sandlin, \textborn 13.03.1893 på Grötas, förblev ogift. Selmas far byggde huset åt Selma, som försörjde sig genom att sticka strumpor, tröjor m.m. åt folk. Hon fick troligen en början till yrket, då hon arbetade hos Lönnqvists. Selma flyttade många gånger, men hade kvar sin stuga på Grötas. I folkmun kallades hon ``Sticka-Selma''.

Då brodern Emil kom hem från USA bodde han först med Selma, men köpte sedan en gård i Nykarleby, dit båda flyttade. Den 1 maj 1936 flyttade de tillbaka till Jeppo efter att ha köpt Bäck lägenhet på Fors. År 1937 dog Emil. Ett år senare inköper brorssonen Johannes med Selma som andra ägare gästgiveriet i Silvast (karta 5, nr 380). Där bodde hon i övre våningen tills gästgiveriet lades ner. Hon flyttade då tillbaka till sin stuga på Grötas, där hon bodde till sin död.

Fastrarna Edit och Hilda bodde i Selmas gård de perioder som gården stod tom. Också farmor Brita vistades mycket i stugan. Selma \textdied 11.05.1965.



\jhhouse{Mellangård}{5:151}{Grötas}{11}{27, 27a och 127}


\jhhousepic{163-05719.jpg}{Hilda, Edit och Johannes Sandlin}

\jhoccupant{Sandlin}{Jessica}{2012--}
Jessica Sandlin, \textborn 21.11.1983 i Jakobstad fick 13.09.2012 huset i gåva av Hilda och Edit Sandlin. Huset fungerar idag som kontor och lager för Jessicas far, Sten Sandlin. Skogen och marken hade tidigare sålts.\jhvspace{}


\jhoccupant{Sandlin}{Johannes, Hilda \& Edit}{1962--\allowbreak 2012}
Johannes Sandlin, \textborn 03.05.1903 på Grötas övertog tillsammans med systrarna Hilda, \textborn 27.06.1909 och Edit, \textborn 21.07.1916 den ena halvan av föräldrarnas lägenhet på Grötas hemman. Denna omfattade 33 ha, varav 11 ha var odlad jord. Johannes uppförde en ny gård i tegel år 1962. Ekonomibyggnaden är från 1800-talet i cement med asbesttak.

Johannes var ett antal år kommunfullmäktig samt styrelseledamot i Jeppo sparbank.

Johannes \textdied 11.10.1973  ---  Edit \textdied 03.12.2010  --  Hilda \textdied 12.11.2011.


\jholdhouse{Mellangård}{5:151}{Grötas}{11}{127}


\jhhousepic{Grotas127-SandlinJ.jpg}{T.v. Forsblom 105, vita gården är Selma Sandlin och Johannes' gård t.h.}

\jhoccupant{Sandlin}{Johannes}{1944--\allowbreak 1962}
Johannes Sandlin övertog gården efter fadern år 1944. Johannes och systrarna Sigrid och Hilda var också ägare till gästgiveriet i Jeppo. Gården har tillhört samma släkt sedan 1800-talet. Oklart vem som byggde denna gård, men det är troligt att det var Anders Jakobsson.


\jhoccupant{Sandlin}{Jakob \& Brita}{1900--\allowbreak 1944}
Jakob Andersson, senare med efternamnet Sandlin, \textborn 01.08.1871 på Grötas, vigd 10.03.1901 med Brita Johansdotter, \textborn 29.01.1880 på Grötas.
\begin{jhchildren}
  \item \jhperson{\jhname[Fanny Maria]{Sandlin, Fanny Maria}}{18.10.1901}{03.01.1913}
  \item \jhperson{\jhbold{\jhname[Johannes]{Sandlin, Johannes}}}{03.05.1903}{}
  \item \jhperson{\jhname[Sigrid]{Sandlin, Sigrid}}{07.10.1905}{}, gift Svanbäck
  \item \jhperson{\jhname[Emil]{Sandlin, Emil}}{12.04.1907}{08.01.1912 i difteri}
  \item \jhperson{\jhname[Hilda]{Sandlin, Hilda}}{27.06.1909}{}
  \item \jhperson{\jhname[Einar]{Sandlin, Einar}}{13.09.1911}{}, smed, till Jakobstad
  \item \jhperson{\jhbold{\jhname[Emil]{Sandlin, Emil}}}{05.05.1914}{}, (Böös 46)
  \item \jhperson{\jhname[Edit]{Sandlin, Edit}}{21.07.1916}{}
  \item \jhperson{\jhname[Lydia]{Sandlin, Lydia}}{28.04.1919}{25.07.1948}
  \item \jhperson{\jhname[Lennart]{Sandlin, Lennart}}{12.11.1921}{18.07.1941}
\end{jhchildren}

År 1900 övertog Jakob sina föräldrars lägenhet om  21/256 mantal på Grötas hemman. Under flere år var Jakob ledamot i Jeppo kommunalnämnd och i direktionen för Jungar folkskola.

Jakob \textdied 25.01.1965  ---  Brita \textdied 18.04.1979, 99 år gammal


\jhoccupant{Jakobsson}{Anders \& Maria}{1879--\allowbreak 1900}
Anders Jakobsson, senare med efternamnet Sandlin, \textborn 06.01.1846 på Gunnar, gifte sig med Maria Mattsdotter Sandberg, \textborn 20.01.1850 på Finskas hemman.
\begin{jhchildren}
  \item \jhperson{\jhbold{\jhname[Jakob]{Jakobsson, Jakob}}}{01.08.1871}{}
  \item \jhperson{\jhname[Anna Lovisa]{Jakobsson, Anna Lovisa}}{08.11.1873}{}, gift Södergård (Böös 44)
  \item \jhperson{\jhname[Hilda Maria]{Jakobsson, Hilda Maria}}{02.10.1875}{}, gift Sandbacka (Mietala nr 113)
  \item \jhperson{\jhname[Sanna]{Jakobsson, Sanna}}{18.05.1879}{}, gift Sandås (Böös 145)
  \item \jhperson{\jhname[Anders Emil]{Jakobsson, Anders Emil}}{02.02.1882}{}, till USA
  \item \jhperson{\jhname[Isak]{Jakobsson, Isak}}{01.05.1884}{11.05.1892}
  \item \jhperson{\jhname[Ida Johanna]{Jakobsson, Ida Johanna}}{23.10.1886}{22.11.1889}
  \item \jhperson{\jhname[Ida Katrina]{Jakobsson, Ida Katrina}}{30.12.1889}{}, gift Fagerholm (Grötas 128)
  \item \jhperson{\jhname[Selma Katrina]{Jakobsson, Selma Katrina}}{13.03.1893}{}, (Grötas 25)
\end{jhchildren}

Anders var född i ett backstuguhem. Så snart han kunde börja arbeta tjänade han bl.a som bonddräng på Keppo och Kaup hemman. Efter giftermålet var Maria och Anders bosatta på Finskas. År 1879 inropade de en 21/256 mantals lägenhet på Grötas, där de sedan hade sitt hem.

Anders \textborn 11.01.1901  ---  Maria \textborn 09.12.1939


\jhoccupant{Andersson}{Anders \& Maria}{1875--\allowbreak 1879}
Anders Andersson Kalljärvi (Kalejärfvi), gift med Maria Jakobsdr. Anders köpte 21/256 dels mantal av Carl Johan Abrahamsson år 1875. Han klarade dock inte av hemmanets skulder och var tvungen att sälja hemmanet på auktion till högstbjudande.\jhvspace{}


\jhoccupant{Abrahamsson}{Carl Johan \& Greta}{1869--\allowbreak 1875}
Carl Johan Abrahamsson, \textborn 17.07.1834 på Grötas, gifte sig 1857 med Greta Lisa Johansdotter Sundlin, \textborn 03.09.1833 på Mannfors i Socklot. Makarna fick 6 barn, se Fors, nr 397.

Före 1869 var Carl Johan skriven som torpare på Grötas. Bland dessa fanns många hantverkare, Carl Johan tillhörde dem. Den 15 mars 1869 blev han bonde. Han köpte då hemmanet av Isak Jakobsson Finskas och hans hustru Kaisa Isaksdotter. Efter 6 år sålde han hemmanet på Grötas och familjen flyttade bort. De blev sedermera (1877) bönder på Fors hemman (karta 6, nr 397).


\jhoccupant{Jakobsson}{Isak \& Caisa}{1868--\allowbreak 1869}
Isak Jakobsson Finskas (senare Sandberg), \textborn 20.11.1805 på Keppo, gift med Caisa Isaksdotter, \textborn 09.10.1814 på Finskas. I.o.m giftermålet med Caisa blev Isak bondmåg på Finskas. Tack vare jordbruket, gästgiverirörelsen samt sparsamhet, samlade paret en allt större förmögenhet, vilket ledde till att de köpte flera hemmansdelar i Jeppo. De lånade också ut pengar, t.o.m utanför hemsocknen. Bl.a i Härmä blev ``Jepuan samperin pankki'' ett utbrett talesätt.

Isak och Caisa hade 2 december 1868 köpt 65/256 dels mantal av bröderna Erik och Matts Eriksson samt deras mor Susanna.


\jhoccupant{Eriksson}{Erik \& Matts}{}
Erik Eriksson var far till Anselm Björkvall (se gård nr 115). En del av Isaks och Caisas ägande på Grötas såldes till Carl Johan Abrahamsson. Denna del var den som Erik Mattsson (Anselm Björkvalls farfar) köpt av Johan Mattsson och Matts Andersson Grötas år 1862.



\jholdhouse{``Björkvall''}{5:}{Grötas}{11}{110}


\jhoccupant{Björkvall}{Anselm \& Irene}{1919--\allowbreak 1934}
Anselm Eriksson Grötas, senare med efternamnet Björkvall, \textborn 06.01.1898 på Grötas. Han gifte sig 1921 med Irene Johansdotter Nyman, \textborn 13.03.1898 i Kantlax.
\begin{jhchildren}
  \item \jhperson{\jhname[Anni]{Björkvall, Anni}}{22.04.1922}{}, gift Elenius (Mietala 18)
  \item \jhperson{\jhname[Vera]{Björkvall, Vera}}{14.09.1924}{}, gift Sandvik i Jussila
  \item \jhperson{\jhname[Märta]{Björkvall, Märta}}{20.03.1927}{}, gift Sarelin
  \item \jhperson{\jhbold{\jhname[Sven]{Björkvall, Sven}}}{08.10.1933}{}, (Grötas 35)
\end{jhchildren}

Anselm hade tänkt följa sina syskons exempel och emigrera till USA. För att förhindra detta skrev föräldrarna över lägenheten till Anselm år 1919. Irene kom som piga till Grötas, tycke uppstod och Anselm och Irene gifte sig år 1921. Lägenheten omfattade 50 ha, varav 18 ha odlad jord. Huset som Irene och Anselm bodde i fanns mellan nuvarande hus nr 27 och 28. Anselm byggde upp ett nytt hus 1922--\allowbreak 1923. Pga att hemman skiftades flyttade 3 familjer, bland dem Anselm och Irene. Huset som Anselm byggt flyttades år 1934 till nuvarande plats, karta 11, nr 135.

Irene \textdied 17.09.1976  ---  Anselm \textdied 06.10.1985


\jhoccupant{Eriksson}{Erik \& Caisa, Erik \& Maria}{1872--\allowbreak 1919}
Erik Eriksson, \textborn 14.10.1845 på Grötas hemman, gift första gången med Caisa Johansdotter, \textborn 08.05.1846 på Skog. År 1872 köpte Erik och Caisa av Isak Sandberg en lägenhet om 21/256 mantal. Detta var en del av  Eriks fars hemman, som man tvingats sälja pga  skulder, bl.a till Isak Sandberg. Caisa avled 11.12.1895 i lungsot. Erik ingick nytt äktenskap med Maria Lovisa Johansdotter Holm, \textdied 02.01.1852 på Pelkos hemman.
\begin{jhchildren}
  \item \jhperson{\jhname[Anna Sanna]{Eriksson, Anna Sanna}}{30.10.1871, till Amerika}{}
  \item \jhperson{\jhname[Erik]{Eriksson, Erik}}{08.03.1874, till Amerika}{}
  \item \jhperson{\jhname[Johan]{Eriksson, Johan}}{15.08.1876, till Amerika}{}
  \item \jhperson{\jhname[Ida Sofia]{Eriksson, Ida Sofia}}{18.05.1879, till Amerika}{}
  \item \jhperson{\jhbold{\jhname[Anselm]{Eriksson, Anselm}}}{06.01.1898}{}
\end{jhchildren}

Erik Grötas, från år 1920 med efternamnet Björkvall, \textdied 22.12.1924  ---  Maria Lovisa \textdied 14.10.1935


\jhoccupant{Sandberg}{Isak \& Caisa}{1868--\allowbreak 1872}
Isak Jakobsson Finskas, senare Sandberg, \textborn 20.11.1805 på Keppo, gift med Caisa Isaksdotter, \textborn 09.10.1814 på Finskas. Isak och Caisa var barnlösa. Tack vare jordbruket på Finskas, gästgiverirörelse och sparsamhet samlade paret åt sig en allt större förmögenhet, vilket ledde till att de kunde köpa flere hemmansdelar i Jeppo. Isak köpte 63/256 dels mantal på Grötas år 1868. En del av detta avyttrade han redan följande år. År 1872 kunde Erik Grötas köpa tillbaka en del av detta, nämligen 21/256 dels mantal av familjens ursprungliga hemman.


\jhoccupant{Mattsson}{Erik \& Sanna}{1862--\allowbreak 1868}
Erik Mattsson, \textborn 17.06.1819 på Grötas, gift med Sanna Eriksdotter, \textborn 10.02.1823 på Gunnar hemman. Makarna fick 11 barn födda 1841--1866. Här nämns sönerna \jhperson{\jhname[Matts]{Mattsson, Matts}}{22.09.1841}{} och \jhperson{\jhbold{\jhname[Erik]{Mattsson, Erik}}}{14.10.1845}{}.

Erik och Sanna fick 21/256 dels mantal i arv av föräldrar samt under äktenskapet förvärvat 21/128 dels mantal av Grötas skattehemman.	Erik Grötas avled 06.12.1866. På grund av  hemmanets stora skulder är hustrun Sanna samt sönerna Matts och Erik tvungna att sälja hemmanet (63/256 dels mantal) till den största gäldenären Isak Jakobsson Finskas (Sandberg). Köpebrevet är daterat 2 dec 1868. -- Sanna \textdied omkring 1878.


\jhoccupant{Eriksson}{Anders \& Anna}{1826--\allowbreak 1870}
Anders Eriksson, \textborn 03.01.1805 på Grötas, gift med Anna Eriksdotter, \textborn 07.05.1827 i Larsmo. Efter brodern Matths' död kom Anders att ha hand om hemmanet. Anders och Anna hade 10 barn (1838--1861), varav dottern \jhperson{\jhbold{\jhname[Maria]{Eriksson, Maria}}}{1840}{} och hennes man Johan Mattsson Måtar deltog i arbetet på hemmanet.

Anders \textdied 06.06.1862  ---  Anna \textdied 1868


\jhoccupant{Eriksson}{Matths \& Sanna}{1816--\allowbreak 1826}
Matths Eriksson Grötas, \textborn 12.12.1794 på Grötas, gift 20.06.1817 med bondedottern Sanna Lisa Henriksdotter Damskata, \textborn 01.08.1797 i Munsala.
\begin{jhchildren}
  \item \jhperson{\jhname[Johan Henrik]{Eriksson, Johan Henrik}}{04.05.1818}{}
  \item \jhperson{\jhbold{\jhname[Erik]{Eriksson, Erik}}}{17.06.1819}{}
  \item \jhperson{\jhname[Matth]{Eriksson, Matth}}{02.10.1820}{04.04.1822}
\end{jhchildren}

Matts \textdied 28.04.1820 genom drunkning  ---  Sanna Lisa \textdied 10.12.1826


\jhoccupant{Mattsson}{Eric \& Anna}{1795--\allowbreak 1819}
Eric Mattsson Grötas, \textborn 03.03.1770 på Grötas, gift med Anna Johansdotter Storsilvast, \textborn 10.02.1768 i Jeppo. Paret fick 13 barn (1791--1809). Här nämns sönerna \jhperson{\jhbold{\jhname[Matths]{Mattsson, Matths}}}{12.12.1794}{} och \jhperson{\jhbold{\jhname[Anders]{Mattsson, Anders}}}{03.01.1805}{}.

Genom avhandling 1816 och fastebrev 1819 överlämnade Eric och Anna sitt 21/128 mantals hemman till sonen Matths, som dock dog redan 1820. Det blev därför brodern Anders, som kom att sköta hemmanet.
Erik Mattsson \textdied 24.11.1818.


\jhoccupant{Ericsson}{Matts \& Maria}{1795--\allowbreak 1801}
Matts Ericsson, \textborn 1735 i Överjeppo, gift med Maria Ericsdr, \textborn 1740 i Lohteå.
\begin{jhchildren}
  \item \jhperson{\jhname[Lisa]{Ericsson, Lisa}}{16.09.1768 på Grötas}{}
  \item \jhperson{\jhbold{\jhname[Eric]{Ericsson, Eric}}}{03.03.1770}{}
  \item \jhperson{\jhname[Anders]{Ericsson, Anders}}{25.04.1771}{}
  \item \jhperson{\jhname[Britha]{Ericsson, Britha}}{09.04.1773}{}
  \item \jhperson{\jhname[Matts]{Ericsson, Matts}}{10.06.1774}{}
  \item \jhperson{\jhname[Anna]{Ericsson, Anna}}{18.06.1775}{}
  \item \jhperson{\jhname[Johan]{Ericsson, Johan}}{29.06.1776 }{}
  \item \jhperson{\jhname[Isaac]{Ericsson, Isaac}}{12.12.1777}{}
  \item \jhperson{\jhname[Catharina]{Ericsson, Catharina}}{18.08.1779}{}
  \item \jhperson{\jhname[Margaretha]{Ericsson, Margaretha}}{03.08.1780}{}
\end{jhchildren}

Familjerna har bott på Grötas och  enligt köpebrev så har lägenheten tillhört samma familj fr.o.m. 1795. Det är dock oklart om de bott just på denna plats. Det är också oklart hur gammalt huset som Anselm föddes i var.



\jhhouse{Fagerholm}{5:140}{Grötas}{11}{28, 28a-b}


\jhhousepic{165-05720.jpg}{Annika Häger och Tobias Widdas}

\jhoccupant{Häger/Widdas}{Annika \& Tobias}{2013--}
Annika Häger, \textborn 11.12.1987 på Stenbacka, sambo med Tobias Widdas, \textborn 31.05.1987 i Nykarleby. Annika är socionom och arbetar på Flyktingförläggningen i Oravais. Tobias är byggnadsingenjör och arbetar på Ingenjörsbyrå Kronqvist i Nykarleby. Paret köpte gården av Annikas mamma Ann-Maj Häger samt Ann-Majs kusin Magnus Sandberg, som ärvt gården av Gunnel Fagerholm år 2008. Annika och Tobias har gjort en grundlig renovering av gården.


\jhoccupant{Fagerholm}{Elna \& Gunnel}{1993--\allowbreak 2012}
Systrarna Elna, \textborn 15.11.1913 på Grötas och Gunnel, \textborn 05.02.1929 på Grötas bodde kvar på hemgården hela sitt liv. Gunnel dog 14.01.2008. Elna vistades de sista åren på Hagaborg. Hon dog 28.07.2012.\jhvspace{}



\jholdhouse{Fagerholm}{5:23}{Grötas}{11}{128}


\jhhousepic{Grotas128-Fagerholm.jpg}{Fagerholms gamla gård}

\jhoccupant{Fagerholm}{Uno}{1955--\allowbreak 1993}
Fagerholm Uno, \textborn 01.01.2921 på Grötas. År 1955 övertog han föräldrarnas lägenhet på Grötas. Denna omfattade 45 ha, varav 25 ha var odlad jord. Jordbruket skötte han tillsammans med systrarna Elna och Gunnel. Då boken ``Svenska Österbottens bebyggelse'' utkom år 1965, fanns på gården 1 häst, 13 kor och 14 ungdjur.

År 1966 uppförde Uno nuvarande gård i tegel. Den gamla gården uppfördes 1894 av Unos farfar Jakob Fagerholm. Den uppfördes i trä under asbestplattor med 5 rum och kök samt bekvämligheter. Ekonomibyggnaden är från 1935, uppförd av Unos far Anders.
Uno \textdied 04.07.1993.


\jhoccupant{Fagerholm}{Anders \& Ida}{1912--\allowbreak 1955}
Anders Fagerholm, \textborn 24.07.1881 på Jungar hemman (Svartbacka). Han gifte sig med bonddottern i granngården, Ida Johanna Andersdotter, \textborn 30.12.1889 på Grötas.
\begin{jhchildren}
  \item \jhperson{\jhname[Sigrid]{Fagerholm, Sigrid}}{16.03.1912}{}, gift Sandberg
  \item \jhperson{\jhname[Elna]{Fagerholm, Elna}}{15.11.1913}{}
  \item \jhperson{\jhname[Valfrid]{Fagerholm, Valfrid}}{15.06.1915}{28.12.1918}
  \item \jhperson{\jhname[Elmer]{Fagerholm, Elmer}}{04.09.1918}{}, (Grötas 30)
  \item \jhperson{\jhbold{\jhname[Uno]{Fagerholm, Uno}}}{01.01.1921}{1993}
  \item \jhperson{\jhname[Judit]{Fagerholm, Judit}}{23.12.1923}{}, gift Sarelin
  \item \jhperson{\jhname[Agda]{Fagerholm, Agda}}{27.12.1926}{}, (Böös 39)
  \item \jhperson{\jhname[Gunnel]{Fagerholm, Gunnel}}{05.02.1929}{}
\end{jhchildren}

Före giftermålet var Anders flera år på arbetsförtjänst i USA. Efter giftermålet övertog Anders och Ida hans föräldrars lägenhet. Anders var bl.a ledamot i direktionen för Jungar folkskola.

Anders \textdied 20.07.1960  --- Ida \textdied 02.12.1967


\jhoccupant{Fagerholm}{Jakob \& Sanna Lisa}{1894--\allowbreak 1912}
Jakob Jakobsson Fagerholm, \textborn 16.11.1845 på Grötas hemman. Jakobs farfar var sockensmeden Matts Fagerholm från Fagernäs i Larsmo. Jakob gifte sig med Sanna Lisa Henriksdotter, \textborn 31.01.1848 på Ojala hemman.
\begin{jhchildren}
  \item \jhperson{\jhname[Johannes]{Fagerholm, Johannes}}{24.06.1876}{}, till USA
  \item \jhperson{\jhname[Jakob]{Fagerholm, Jakob}}{02.03.1879}{}
  \item \jhperson{\jhbold{\jhname[Anders]{Fagerholm, Anders}}}{24.07.1881}{}
  \item \jhperson{\jhname[Ida Maria]{Fagerholm, Ida Maria}}{18.03.1883}{}, gift Björkqvist (Grötas 126)
\end{jhchildren}

Jakob och Sanna Lisa började sitt gemensamma liv som backstugusittare på Svartbackan, men inropade år 1894  bröderna Jakob och Matts Nilsson Böös' 21/256 dels mantal på Grötas och där byggde Jakob sitt nya hem för familjen.

Jakob \textdied 23.02.1910  ---  Sanna Lisa \textdied 19.01.1927


\jhoccupant{Tollikko}{ William \& Sanna}{1890--\allowbreak 1894}
William Jakobsson Tollikko, \textborn 07.12.1861, och hustrun Sanna Johansdotter, \textborn 01.11.1859, köpte av  Johan Johansson Grötas och hans hustru det hemman som de lämnade då de reste till Amerika, men såsom många andra klarade de inte att betala av på sina lån och hemmanet såldes till högstbjudande på auktion. Detta var bröderna Nilsson på Böös, som 1892 tillhandlade sig detta hemman, men redan två år senare sålde det till högstbjudande, nämligen Jakob Fagerholm.


\jhoccupant{Kauppi}{Johan \& Maria}{1871--\allowbreak 1891}
Johan Johansson Kauppi, \textborn 30.01.1851 i Alahärmä, gift med Maria Eliasdotter, \textborn 26.05.1848 i Alahärmä.
\begin{jhchildren}
  \item \jhperson{\jhname[Anna Kaisa]{Kauppi, Anna Kaisa}}{28.09.1873 i Alahärmä}{}
  \item \jhperson{\jhname[Johan]{Kauppi, Johan}}{12.08.1875 i Jeppo}{}
  \item \jhperson{\jhname[Anders Gustaf]{Kauppi, Anders Gustaf}}{13.05.1877 i Jeppo}{}
  \item \jhperson{\jhname[Oscar]{Kauppi, Oscar}}{24.11.1881 i Jeppo}{}
\end{jhchildren}

Johan Kauppi köpte detta 21/256 mantals hemman den 6 november 1871 av Karl Eriksson Grötas och hans hustru Sanna Lisa Isaksdotter. Johan reste till Amerika år 1887. Maria sålde hemmanet 1890 och reste med barnen till sin man i Amerika.


\jhoccupant{Eriksson}{Karl \& Sanna Lisa m.fl.}{1869--\allowbreak 1871}
I november 1871 sålde Karl Eriksson Grötas. \textborn 27.09.1833 och hustrun Sanna Lisa Isaksdotter, \textborn 09.04.1824, hemmanet, som de år 1869 köpt av Isak Isaksson Finskas och hans hustru Kajsa Isaksdotter. Isak och Kajsa hade erhållit hemmanet genom avhandling 1868 av bröderna Matts och Erik Eriksson. Detta var en del av Anselm Björkvalls förfäders hemman (se Grötas nr 110).



\jhhouse{Hemlunden}{5:139}{Grötas}{11}{29}


\jhhousepic{162-05721.jpg}{Agda Nygårds hus}

\jhoccupant{Nygård}{Agda}{1987--}
Agda Nygård, \textborn 27.12.1926 på Grötas, lät bygga denna gård efter att maken Valfrid dött och hon blev ensam i huset på Böös (Böös 39). Tomten till huset befinner sig på Agdas barndomshems marker, på samma åkerjord, som brodern Elmer byggde sitt hus 1949. Huset är ett elementhus från Teri-Hus och uppfördes år 1987.

Agda har varit en värdefull person för gruppen bakom detta alster då det gällt att få fram uppgifter om Grötas och Böös.

Lorem ipsum dolor sit amet, consectetur adipiscing elit, sed do eiusmod tempor incididunt ut labore et dolore magna aliqua. Ut enim ad minim veniam, quis nostrud exercitation ullamco laboris nisi ut aliquip ex ea commodo consequat. Duis aute irure dolor in reprehenderit in voluptate velit esse cillum dolore eu fugiat nulla pariatur. Excepteur sint occaecat cupidatat non proident, sunt in culpa qui officia deserunt mollit anim id est laborum. % FIXME


\jhhouse{Åbrant}{5:63}{Grötas}{11}{30, 30a}


\jhhousepic{161-05723.jpg}{Henrik och Rael Östman}

\jhoccupant{Östman}{Henrik \& Rael}{2016--}
Henrik Östman, \textborn 1977 i Karleby, gift med Rael Leedjärv, \textborn 1977  i Tallinn. Henrik blev teologie magister 2005. Sedan november 2015 kaplan i Nykarleby (Jeppo kapellförsamling). Tidigare bl.a arbetat som församlingspastor i Kronoby och Kvevlax. Hans fritidsintressen är orientering, motionsidrott och litteratur.

Rael BA i svenska språket och litteratur 2004 vid Tallinns universitet. Hon var missionär i Kenya 2004--\allowbreak 2008, biståndskoordinator vid Hoppets stjärna r.f. 2008--\allowbreak 2012.
\begin{jhchildren}
  \item \jhperson{\jhname[Ellinor Rebecka Amanda]{Östman, Ellinor Rebecka Amanda}}{2013}{}
  \item \jhperson{\jhname[Klaus Victor Johannes]{Östman, Klaus Victor Johannes}}{2015}{}
\end{jhchildren}
Familjen köpte huset på Grötas år 2016, flyttade in i mars 2017.


\jhoccupant{Fagerholms}{sterbhus}{2009--\allowbreak 2016}
Gården stod tom åren 2009--\allowbreak 2016.\jhvspace{}


\jhoccupant{Fagerholm}{Elmer \& Dagmar}{1947--\allowbreak 2009}
Elmer Fagerholm, \textborn 04.09.1918 på Grötas, vigd 01.09.1946 med Dagmar Sandberg, \textborn 06.07.1917 på Finskas.
\begin{jhchildren}
  \item \jhperson{\jhname[Magnhild]{Fagerholm, Magnhild}}{11.07.1947}{}
  \item \jhperson{\jhname[Gunvor]{Fagerholm, Gunvor}}{29.12.1949}{}
\end{jhchildren}

Elmer gick utbildningen vid Korsholms lantmannaskola och startade ett snickeri i lillstugan vid hemgården. Kriget kom emellan, men efter krigsslutet fortsatte han verksamheten. År 1947 byggde han en särskild snickeribyggnad och senare, år 1949, på samma tomt en ny bostadsbyggnad. Snickeriet var mycket anlitat. Fönsterbågar, dörrar, köksinredningar och andra tillbehör för husbyggnad och inredning var de vanligaste produkterna i snickeriet. Hans durabla alster finns t.ex. i hus 97 på Fors.

Elmer var musikintresserad, spelade som yngre med i Hornorkestern i Jeppo och var ledare för Jeppo folkdanslag.

Dagmar \textdied 06.07.1987  ---  Elmer \textdied 23.06.2009



\jhhouse{Nymark}{5:11}{Grötas}{11}{112}


\jhhousepic{Grotas112-Nyman.jpg}{Erkki Hietamäki var senaste boende i Anders och Maria Nymans gård}

\jhoccupant{Ahlfors}{Stefan}{1987--}
Stefan Ahlfors (gård nr 20) köpte gården år 1987 av Erkki Hietamäki. Stefan rev gården och införlivade tomten med sina åkrar.\jhvspace{}


\jhoccupant{Hietamäki}{Erkki}{1985--\allowbreak 1987}
Erkki Hietamäki, \textborn 28.12.1953 i Ylihärmä, tidigare gift med Regina Dahlström, \textborn 22.03.1959  på Jungarå (se Fors, nr 88).
\begin{jhchildren}
  \item \jhperson{\jhname[Tanja]{Hietamäki, Tanja}}{14.11.1978}{}
  \item \jhperson{\jhname[Juha-Matti]{Hietamäki, Juha-Matti}}{20.12.1983}{}
\end{jhchildren}

Erkki arbetat som chaufför. Han köpte gården 1985 av Hugo Almbergs dödsbo.


\jhoccupant{Almberg}{Hugo}{1949--\allowbreak 1987}
Hugo Almberg, \textborn 13.01.1904 på Grötas. Hugo var Maria Nymans brorson och han kom att ärva lägenheten efter Marias död. I gården har bott många \jhbold{hyresgäster} under åren:
\begin{enumerate}
  \item Raul och Maija Saviaro (Grötas  21)
  \item Ragni, f. Björkqvist och Ilpo Kujanpää åren 1956 – 1957
  \item Viktor och Lea Jungerstam (Ruotsala 26) på 50-talet
  \item Signe Almberg, \textborn 01.12.1922, g. m. Arthur Stoor, \textborn 11.11.1918  i Pensala bodde i gården på 1940-talet. De hade en rävfarm på lägenheten. Dottern Benita Gunhild Margareta \textborn 16.09.1945.
\end{enumerate}


\jhoccupant{Nyman}{Anders \& Maria}{1916--\allowbreak 1949}
Anders Nyman, \textborn 26.11.1872 i Jeppo, vigd 02.02.1902 med Maria Eriksdotter Grötas, \textborn 26.06.1874 i Jeppo. Anders var skräddare och makarna, som var barnlösa, bodde först som backstugusittare på Jungarå och Gunnar, senare på Grötas. År 1920 löser Anders in det 40 kappland stora backstuguområdet och blir fri från dagsverksskyldigheten. Jordägare var änkan Sofia Grötas, senare  med namnet Ahlfors.

Anders och ``Skrädda-Maj'' flyttar till Grötas 1916, då Marias far dött. Troligen bodde de tillsammans med modern Maja Lovisa.

Anders \textborn 09.09.1928  ---  Maria \textborn 31.10.1949


\jhoccupant{Grötas}{Erik \& Maja Lovisa}{1882--\allowbreak 1916}
Erik Andersson Romar, senare Grötas, \textborn 16.04.1842 på Romar, vigd 16.06.1867 med Maja Lovisa Johansdotter Jungar, \textborn 12.02.1847.
\begin{jhchildren}
  \item \jhperson{\jhname[Anders]{Grötas, Anders}}{13.11.1868}{}, (Grötas 33)
  \item \jhperson{\jhbold{\jhname[Maria]{Grötas, Maria}}}{26.06.1874}{}
  \item \jhperson{\jhname[Hilda Sofia]{Grötas, Hilda Sofia}}{04.04.1881}{}, gift Källman (Silvast 350)
\end{jhchildren}

Då Erik gifte sig var han hemmason på Romar, men med hjälp av sina föräldrar köpte han år 1871 ett 7/64 mantals hemman på Grötas. Det var nuvarande Ahlfors hemman, dvs samma hemman, som han senare blev torpare under. I ett kontrakt från 1882 skriver Henrik Johansson Tyni ``till torparen Erik Andersson till gårdstomt hela min andel i den sk furukangan, till åker får han i åkerskatan 5 tunnland, som börjar från den sk torpåkern uppåt skatan...''

Erik Andersson Grötas \textdied 14.10.1916  ---  Maja Lovisa \textdied 10.10.1943



\jhhouse{Marielund}{5:6}{Grötas}{11}{113}


\jhhousepic{Torp Marielund.jpeg}{Torpet på Marielund}

\jhoccupant{Johansson}{William \& Elin}{1942--\allowbreak 1965}
Torpet och uthusen revs på 1960-talet. Tomten såldes av Erik Johansson 2012. Senast boende i huset var August Willliam Johansson, \textborn 21.07.1896 i Jeppo (nr 107) gift 1929 med Elin Irene Slangar, \textborn 23.04.1896, som med sonen William, \textborn 23.04 1930, övertog huset år 1942. De använde det som semester- och sommarbostad. Deras fasta boplats var Helsingfors, där William var järnvägstjänsteman.

William \textdied 23.12.1956  ---  hustrun Elin \textdied 02.04.1978.
Sonen Erik bor i Jakobstad, gift med Märta Frederika Roos, \textborn 12.09.1930 (nr 104).


\jhoccupant{Grötas}{Maria \& Johan}{1910--\allowbreak 1942}
Senast fast boende i torpet var änkan Maria Danielsdt. Levelä, \textborn 14.01.1854, gift med Johan Henrikson Grötas, \textborn 19.07.1859. Vid försäljningen av jordegendomen (nr 107 ) till sonen Emil och bostaden till sonen Johan Henrik Westin 1910, flyttade de in i torpet med yngsta sonen August William. De hade gjort arrendekontrakt med sonen Anders Emil 1910, som inregistrerades 1921 i legonämndens protokoll.  Barn: (nr 107)

Johan Henrik \textdied 09.04.1911  ---  Maria \textdied 12.07.1942


\jhoccupant{Grötas}{Henrik \& Brita-Lena}{1854--\allowbreak 1906}
Henrik Johansson Grötas, \textborn 25.10.1828 på Lillsilvast, son till Johan Hansson bonde på Forss, \textborn 01.11.1797 och makan Sanna Mattsdt. Bröms, \textborn 16.O7.1806 i Munsala. Hustrun var bonddottern Brita-Lena Olofsdotter, \textborn 07.09.1829 på Grötas (nr 107). De gifte sig 1850, gjorde torpkontrakt med Brita-Lenas far Olof Olofsson år 1854. Enligt avhandling av år 1872 sålde änkan Brita-Lena 61/768 mantal av torpägorna till Anders Johansson och hustrun Susanna Andersdotter. I köpebrevet står att avträdarna har besittningsrätt till en under hemmansdelen underlydande torplägenhet. Troligtvis var det Henrik Johan och Brita-Lena som byggt torpet.
\begin{jhchildren}
  \item \jhperson{\jhname[Johan]{Grötas, Johan}}{25.06.1851}{05.01.1852}
  \item \jhperson{\jhname[Sanna]{Grötas, Sanna}}{10.06.1853}{}, g. m Jakob Isaksson Böös, emig. t. Amerika 1887
  \item \jhperson{\jhname[Maria]{Grötas, Maria}}{17.06.1856}{20.03.1857}
  \item \jhperson{\jhbold{\jhname[Johan Henrik]{Grötas, Johan Henrik}}}{19.07.1859, (nr 107)}{}
  \item \jhperson{\jhname[Erik]{Grötas, Erik}}{10.06.1862, dog som barn}{}
  \item \jhperson{\jhname[Elisabeth]{Grötas, Elisabeth}}{26.07.1864}{}, ej på Grötas 1887
  \item \jhperson{\jhname[Anna-Lovisa]{Grötas, Anna-Lovisa}}{22.07.1869}{}, dog före 1882
  \item \jhperson{\jhname[Brita-Sofia]{Grötas, Brita-Sofia}}{16.07.1871}{}, dog före 1887
\end{jhchildren}
Henrik Johan \textdied före 1887  ---  Brita-Lena \textdied 09.04.1906



\jhhouse{Torp}{5:}{Grötas}{11}{114}


\jhoccupant{Modig}{Johan Petter \& Anna Lovisa}{1894--\allowbreak 1924}
Torparen Johan Petter Simonson Modig, \textborn 10.07.1845 i Korsholm, gift 06.11.1870 med Anna Lovisa Abrahamsdotter,  \textborn 20.01.1848 på Grötas hemman.
\begin{jhchildren}
  \item \jhperson{\jhname[Johan Edvard]{Modig, Johan Edvard}}{05.07.1873}{}, anlänt till USA 1892
  \item \jhperson{\jhname[Johanna Karolina]{Modig, Johanna Karolina}}{15.11.1875}{}, betyg till Amerika 30.01.1896
  \item \jhperson{\jhname[Hilda Katrina]{Modig, Hilda Katrina}}{10.08.1878}{}, pass för Amerika 07.11.1899
  \item \jhperson{\jhname[Anders Alfred]{Modig, Anders Alfred}}{12.12.1880}{1962}, sjöman, till Australien 1905
\end{jhchildren}

Anna Lovisas familj kom från Böös, men var skrivna på Grötas vid Anna Lovisas födelse. På 1860- och 1870-talen var skrivkunnigheten skral också i Jeppo. Då amerikafararnas hustrur ville skriva brev till sina män vände de sig till Anna Lovisa Modig, som var känd för att skriva vackra brev. Trafiken hos Modig i detta syfte var så stor, att man måste vänta på sin tur eller t.o.m. uppskjuta brevskrivandet till en annan dag. Var Anna Lovisa, ``Modiskon'', hade lärt sig skriva och läsa vet man inte. Hon sägs ha börjat breven med ``nu fattar jag pennan i min olärda hand...''

Hon fungerade också som vittne då skriftliga testamenten gjordes. Om Anna Lovisa finns många historier. Hur hårt livet var för torpare och backstugusittare vittnar hennes önskan inför jul att pengarna sku räcka åtminstone till att köpa ett bröd.

Texten i torparkontraktet är exakt lika som Anna Lovisas föräldrars kontrakt. Troligen är det samma torp, som Anna Lovisas föräldrar bodde i. Fram till år 1888 är torpareänkan Maria Mattsdotter, \textborn 21.07.1814 i Larsmo, \textdied 04.04.1888, skriven tillsammans med familjen på Grötas (Anna Lovisas fars andra hustru). År 1894 övertog Johan Petter  hustrun Anna Lovisas föräldrars arrendekontrakt. Förnyandet av kontraktet gjordes med Johan Eriksson Gröthas. Torpet, som familjen sedan bodde i, fanns bakom nuvarande Almbergs gård. Barnen arbetade som drängar och pigor i Jeppo, men emigrerade sedan utomlands.

Johan Petter dog 08.05.1905. År 1907 gifte Anna Lovisa om sig med backstugukarlen, änklingen Jakob Abrahamsson Kronkvist från Långbacka i Munsala. Hon blev änka igen 20.12.1910 och flyttade 21.01.1912 tillbaka till Jeppo. Inga uppgifter finns att hon skulle försökt lösa in torpet.

Anna Lovisa dog 01.11.1924. Torpet revs på 1960-talet.


\jhoccupant{Isaksson}{Abraham \& Maria}{1847--\allowbreak 1894}
Torparen Abraham Isaksson Grötas, \textborn 17.03.1817 på Böös hemman, gift med Carolina Johansdotter, \textborn 05.12.1825, \textdied 08.11.1849. Abraham gifte om sig med Maria Mattsdotter, \textborn 21.07.1814 i Larsmo.
\begin{jhchildren}
  \item \jhperson{\jhname[Katarina]{Isaksson, Katarina}}{12.04.1845}{}
  \item \jhperson{\jhbold{\jhname[Anna Lovisa]{Isaksson, Anna Lovisa}}}{20.01.1848}{}
  \item \jhperson{\jhname[Isak]{Isaksson, Isak}}{14.08.1851}{02.02.1853}
\end{jhchildren}

Torparkontraktet som gjordes med Abraham är daterat 29 okt 1847 och skulle gälla i 47 år, men i.o.m hemmansägarbyten förnyades arrendekontraktet flere gånger. Texten i kontraktet är samma som i Anna Lovisa och Petter Modigs avtal år 1894. Förutom avtal om vilka skiften och åkrar som han får uppodla, vilken skog han får ta virke till stängsel ifrån, så fanns uppgifter om dagsverken. Torparen måste göra 12 dagsverken om året åt ägaren.



\jhhouse{Jokiranta}{5:47}{Grötas}{11}{31}


\jhhousepic{170-05732.jpg}{Urho och Fanny Kennola}

\jhoccupant{Kennola}{Urho \& Fanny}{1954--\allowbreak 2010}
Huset är obebott sen år 2010, ägs av Urho Kennolas dödsbo. Urho Isak, \textborn 08.04.1919 (nr 131) och makan Fanny Josefiina Halmesmäki, \textborn 02.03.1920 i Lappo ,gifte sig 1948, byggde bostadshuset år 1954 och ekonomiebyggnaderna 1956. De flyttade från hus nr 131. Urho var byggnadsarbetare, småbrukare och egen företagare med bl.a. vägsladdning, skogsdikning och gårdsmalning av spannmål. Deltog i båda krigen, därutöver gränsvakt ett år.

Fanny hade gått husmoderskola i Lappo, jobbade som skolköksa i Jeppo(-Pensala) skola och hade några år caférörelse i Silvast på 1960-talet. Fanny \textdied 01.10. 2009  ---  Urho \textdied 01.05.2015.
\begin{jhchildren}
  \item \jhperson{\jhname[Anna-Liisa Kaarina]{Kennola, Anna-Liisa Kaarina}}{21.05.1949}{}, gift Virtakangas , Karleby
  \item \jhperson{\jhname[Marja-Liisa Tellervo]{Kennola, Marja-Liisa Tellervo}}{21.05.1949}{}, gift Erkinheimo, Lappajärvi
  \item \jhperson{\jhname[Anneli Ulla]{Kennola, Anneli Ulla}}{02.07.1952}{}, g. Böling, frånskild, pensionerad sjuksköterska
\end{jhchildren}



\jhhouse{Forsstrand}{5:35}{Grötas}{11}{131}


\jhhousepic{Ilmari Kennola.jpg}{Ilmari Kennola}

\jhoccupant{Kennola}{Ilmari}{1945--\allowbreak 1970}
Bostadshuset jämte uthus har rivits av Urho Kennola 1973. Senast boende var Ilmari Kennola, \textborn 29.08.1915. Han köpte bostadshuset med uthus och tomt 1945 på auktion, anordnad av Nikolai Kennolas dödsbo. Ilmari var ogift diversearbetare, jobbade bl.a. åt Oravais vägmästardistrikt. Han deltog i båda krigen.

Systern Senja (Else) Elisabet bodde i huset tills hon gifte sig med Aarne Krökki från Karelen och flyttade till Rasimäki nära Kajana 1948. År 1947 inlöste brodern Urho Isak hälften av egendomen, gifte sig 1948 med Fanny Josefiina, bodde i  huset tills de byggde nytt på samma lägenhet 1954 (nr 31). I slutet på 1948 rev Ilmari och Urho en del av uthusen och  renoverade bostaden.
Ilmari \textdied 17.01 1970, Urho inlöste hans del i lägenheten.


\jhoccupant{Grötas}{Nikolai \& Liisa}{1900--\allowbreak 1942}
Bostadshuset och uthusen byggdes i slutet på 1890-talet av Matts Nikolai Grötas, senare Kennola, \textborn 18.07.1879 i Alahärmä (nr 103) ,gift 1901 med Liisa Isaksdt., \textborn 26.12 1881 från Ylistaro. Nikolai hade köpt av sin far Juho Gustavsson (nr 103) hälften av 7/128 mantal av Grötas nr. 5 år 1897. Nikolai och Liisa var småbrukare, ägde enligt nyskiftets delningsbok 1931 8 ha skog och 6,7 ha åker. Nikolai arbetade på sågen (Fors, karta 7, nr 406) samt tillverkade likkistor. Paret fick 12 barn, 5 dog vid ung ålder.
\begin{jhchildren}
  \item \jhperson{\jhname[Elis Nikolai]{Grötas, Elis Nikolai}}{23.01.1902, (karta 4, nr 352)}{}
  \item \jhperson{\jhname[Xenia Elisabet]{Grötas, Xenia Elisabet}}{23.01.1905}{21.02.1905}
  \item \jhperson{\jhname[Xenia Elisabet]{Grötas, Xenia Elisabet}}{13.06.1906}{19.06.1914}
  \item \jhperson{\jhname[Milia Johanna]{Grötas, Milia Johanna}}{25.01.1908}{11.02.1908}
  \item \jhperson{\jhname[Helmi Maria]{Grötas, Helmi Maria}}{01.07.1909}{}, (karta 3, nr 25)
  \item \jhperson{\jhname[Vieno Aili Johanna]{Grötas, Vieno Aili Johanna}}{10.05.1911}{}, (nr 113, 32)
  \item \jhperson{\jhname[Lempi Katrina]{Grötas, Lempi Katrina}}{09.03.1913}{24.05.1913}
  \item \jhperson{\jhname[Lempi Katrina]{Grötas, Lempi Katrina}}{03.05.1914}{}, (karta 6, nr 86)
  \item \jhperson{\jhbold{\jhname[Johannes Ilmari]{Grötas, Johannes Ilmari}}}{29.08.1915}{17.01.1970}
  \item \jhperson{\jhname[Isak Emil]{Grötas, Isak Emil}}{16.01.1918}{22.01.1918}
  \item \jhperson{\jhbold{\jhname[Urho Isakki]{Grötas, Urho Isakki}}}{08.04.1919}{}, (nr 31)
  \item \jhperson{\jhname[Senja Elisabet]{Grötas, Senja Elisabet}}{24.08.1923}{2011}
\end{jhchildren}

Liisa \textdied 30.04.1934  ---  Nikolai \textdied 14.09.1942.



\jhhouse{Koski}{5:131}{Grötas}{11}{32}


\jhhousepic{166-05725.jpg}{Kuisma Koski}

\jhoccupant{Koski}{Kuisma \& Tellervo}{1993--}
Gårdens nuvarande ägare Kuisma Mikael Koski, \textborn 06.01.1945, som tillsammans med sin maka Raili Tellervo, \textborn 06.01.1947, \textdied 05.09.2007, köpte gården av Ivar Björkqvists dödsbo den 27.03.1993. Kuisma är pensionär. Han har arbetat på Keppo pälsdjursfarm och senast på KWH-Mirka.


\jhoccupant{Björkqvist}{Ivar \& Vieno}{1954--\allowbreak 1993}
Ivar Björkqvist, \textborn 06.08.1912 (nr 126b) och makan Vieno Aili Johanna, \textborn 10.05.1911 (nr 131), köpte tomten av Ivars mors och arvingars gemensamma lägenhet (nr 126a) den 31.03.1951. Bostadshuset och ekonomiebyggnaderna blev inflyttningsklara hösten 1954.

Ivar arbetade efter hemförlovningen från kriget hösten  1944 vid Statens järnvägars vedplan i Jeppo och senare en tid i Seinäjoki tills han blev pensionär. Vieno arbetade på Jeppo ullspinneri, Silvast meijeri, senare Kiitola pälsberederi och Keppos pälsdjursfarm.

Under åren 1952--\allowbreak 1955 ansvarade de för skötseln av Ivars hemlägenhet och djuren (nr 126a), då brodern Artur var i Canada. Dottern Ragne med sin man Ilpo Kujanpää bodde i det nybyggda huset från hösten 1954 i ett år. Därefter flyttade Ivar och Vieno in med barnen Anita och Paul från hus nr 126b. Barn: Se nr 126b.

Ivar \textdied 27.10.1977  ---  Vieno \textdied 15.03.1993.



\jhhouse{Almberg}{5:20}{Grötas}{11}{33, 33a}


\jhhousepic{167-05728.jpg}{Ivar och Jenny Almberg samt Arne Almberg}

\jhoccupant{Almberg}{Ivars dödsbo}{1975--}
Gårdens nuvarande ägare är Ivar Almbergs dödsbo. Åkermarken är såld. Arne Almberg, \textborn 08.10.1947, var bonde och bodde ensam i gården åren 2008--\allowbreak 2016, före det tillsammans med sin mor Jenny, som dog 2008. År 1979 förstorades gården med en tillbyggnad.

Arne \textdied 03.12.2016.\jhvspace{1}


\jhoccupant{Almberg}{Ivar \& Jenny}{1937--\allowbreak 1975}
Ivar, \textborn 16.04.1901 i Jeppo, gift med Jenny Timonen, \textborn 09.09.1921 i Kortesjärvi.
\begin{jhchildren}
  \item \jhperson{\jhname[Erik]{Almberg, Erik}}{08.08.1946}{}, bor Munsala, arbetat på KWH Mirka
  \item \jhperson{\jhbold{\jhname[Arne]{Almberg, Arne}}}{08.10.1947}{}
  \item \jhperson{\jhname[Allan]{Almberg, Allan}}{05.08.1951}{23.10.1970}
  \item \jhperson{\jhname[Hans]{Almberg, Hans}}{24.01.1953}{}, arbetar som kock i Lycksele
\end{jhchildren}
Ivars far dog i december 1936. I februari 1937 tillföll lägenheten på 21/256 mantal Ivar, sytningskontrakt gällande modern skrevs på samma gång. På gården har bedrivits mjölkproduktion, senare hade man gödsvin. Gården omfattade 25,81 ha odlad jord och 30 ha skog. Ivars körstil med sin ``hög-Zetor'' tilldrog sig en viss uppmärksamhet då han rörde sig i byn. Han växlade gaspådrag med -avdrag i ständig rytm.

Ivar \textdied 13.09.1975  ---  Jenny \textdied 14.08.2008



\jholdhouse{Almberg}{5:}{Grötas}{11}{111}


\jhoccupant{Grötas/Eriksson}{Anders \& Maria Lovisa}{1906--\allowbreak 1937}
Banvakten Anders Eriksson Grötas, \textborn 13.11.1868 på Romar, gift med Maria	Lovisa Andersdotter, \textborn 16.12.1866 på Mietala.
\begin{jhchildren}
  \item \jhperson{\jhname[Anna Lovisa]{Grötas/Eriksson, Anna Lovisa}}{22.12.1890}{}, g. Ahlfors (Grötas 20)
  \item \jhperson{\jhname[Anders William]{Grötas/Eriksson, Anders William}}{03.04.1893}{}
  \item \jhperson{\jhname[Maria Emilia]{Grötas/Eriksson, Maria Emilia}}{19.07.1895}{}, (Mietala 116)
  \item \jhperson{\jhname[Johannes Emil]{Grötas/Eriksson, Johannes Emil}}{26.01.1898}{}, till USA
  \item \jhperson{\jhbold{\jhname[Erik Ivar]{Grötas/Eriksson, Erik Ivar}}}{16.04.1901}{}
  \item \jhperson{\jhname[Hugo Wallfrid]{Grötas/Eriksson, Hugo Wallfrid}}{13.01.1904}{}, bokförare, till Ingå
  \item \jhperson{\jhname[Ester Agneta]{Grötas/Eriksson, Ester Agneta}}{21.01.1907}{}, gift Eriksson
\end{jhchildren}

Anders var banvakt och levde med sin familj bl.a i vaktstugan vid Gunnar. 10 februari 1906 köpte han tillsammans med sin fru 21/256 mantal på Grötas skattehemman av Isak Jakobsson. Anders hade även andra sysslor vid sidan av banvaktsarbetet. Han slutade som banvakt år 1929. På 1930-talet flyttades gårdarna som stått nästan vägg i vägg vid nuvarande Sandlins gård.

Anders \textdied 05.12.1936  ---  Maja Lovisa \textdied 14.05.1943


\jhoccupant{Jakobsson}{Isak \& Sofia}{1903--\allowbreak 1906}
Isak Jakobsson, \textborn 08.08.1862 , gift med Sofia Johansdotter, \textborn 08.09.1860.
\begin{jhchildren}
  \item \jhperson{\jhname[Johannes]{Jakobsson, Johannes}}{26.09.1886}{}
  \item \jhperson{\jhname[Anna Lovisa]{Jakobsson, Anna Lovisa}}{19.10.1888}{}
  \item \jhperson{\jhname[Isak Alfred]{Jakobsson, Isak Alfred}}{27.09.1890}{}
  \item \jhperson{\jhname[Anders Emil]{Jakobsson, Anders Emil}}{22.06.1893}{}
  \item \jhperson{\jhname[Elsa Sofia]{Jakobsson, Elsa Sofia}}{29.07.1896}{25.10.1896}
\end{jhchildren}

Isak och Sofia köpte år 1903 hemmanet på 61/768 dels mantal av Johan Eriksson Grötas (avhandling 29 augusti 1903), men sålde det redan i februari 1906. Familjen bodde kvar som inhyses till 1907, då de flyttade till Löf i Ytterjeppo.


\jhoccupant{Eriksson}{Johan \& Maja}{1878--\allowbreak 1903}
Johan Eriksson Haarus, \textborn 11.02.1830 i Kortesjärvi, gift med Maja Lisa Mattsdotter, \textborn 15.06.1841 i Kortesjärvi.
\begin{jhchildren}
  \item \jhperson{\jhname[Jakob]{Eriksson, Jakob}}{28.12.1870 i Kortesjärvi}{}
  \item \jhperson{\jhname[Herman]{Eriksson, Herman}}{24.12.1873 i Kortesjärvi}{}
  \item \jhperson{\jhname[Sanna Maria]{Eriksson, Sanna Maria}}{07.12.1875 i Kortesjärvi}{}
  \item \jhperson{\jhname[Matts Nikolai]{Eriksson, Matts Nikolai}}{30.03.1878}{08.08.1901}
  \item \jhperson{\jhname[Anders]{Eriksson, Anders}}{22.12.1880 i Jeppo}{}
  \item \jhperson{\jhname[Oskar]{Eriksson, Oskar}}{04.12.1883 i Jeppo}{}
  \item \jhperson{\jhname[Ida Sofia]{Eriksson, Ida Sofia}}{17.06.1886 i Jeppo}{}
\end{jhchildren}

I november 1878 köpte Johan ett 61/768 dels mantals hemman på Grötas i Jeppo. Säljare var Gustaf Mattsson Keskinen och hans hustru. Johan flyttade med familjen till Jeppo. År 1903 sålde de hemmanet och flyttade år 1905 tillbaka till Kortesjärvi.


\jhoccupant{Keskinen}{Gustaf \& Lisa}{1873--\allowbreak 1878}
Gustaf Mattsson Keskinen, \textborn 14.11.1836  i Ylihärmä, gift med Lisa Andersdotter, \textborn 16.04.1848 i Alahärmä.
Barn: Maria, \textborn 18.09.1871.

Gustaf flyttade från Keskinen i Ylihärmä till Jeppo i december 1873. Han hade av Eric Ericson Böös samt hustru Maria Johansdotter den 18 november 1873 köpt detta hemman men sålde det 5 år senare till Johan Eriksson Haarus.


\jhoccupant{Ericsson}{Eric \& Maria}{1869--\allowbreak 1873}
Eric Ericsson Böös, \textborn 1842 i Överjeppo, gift med Maria Johansdotter, \textborn 1844 i Jungarby.
\begin{jhchildren}
  \item \jhperson{\jhname[Johanna]{Ericsson, Johanna}}{1866}{}
  \item \jhperson{\jhname[Maria Lovisa]{Ericsson, Maria Lovisa}}{1867}{}
\end{jhchildren}

Familjen flyttat från Ruotsala till Böös och bodde nu ett par år på Grötas. De hade år 1869 köpt av Finlands Hypoteksförening 1/3 av det förut gemensamt ägda 61/256 mantals stora hemmanet. Bönderna Johan Johansson och Johan Jakobsson Grötas samt bondeänkan Anna Katarina Grötas var tidigare ägare. Den 18.11.1873 köper Eric ett hemman på Ruotsala och flyttar dit med sin familj.



\jhhouse{White House}{5:114}{Grötas}{11}{34, 34a}


\jhhousepic{168-05727.jpg}{Risto och Susanne Yliaho}

\jhoccupant{Yliaho}{Risto \& Susanne}{1989--}
Risto Yliaho, \textborn 09.01.1957 i Canada, gifte sig år 1983 med Susanne Grankulla, \textborn 24.09.1964 i Jakobstad.
\begin{jhchildren}
  \item \jhperson{\jhname[Maria]{Yliaho, Maria}}{17.02.1984}{}, företagare
  \item \jhperson{\jhname[Markku]{Yliaho, Markku}}{09.08.1985}{}, (Fors 99)
\end{jhchildren}

Risto arbetar som chaufför. Susanne arbetar på Mirka, men har tidigare varit företagare. Hon arbetade i många år som butiksbiträde på K-Anita i Silvast och övertog butiken i slutet av 1995. Butiken, under namnet Susannes Närbutik, drev hon till slutet av 2003.

Bilsport har hört till familjens fritidsintressen. Susanne har tävlat i varmansklassen åren 1997 – 2003. Risto har också kört varmans, men övergick så småningom till att bli kartläsare inom rally. Risto och Susanne köpte år 1989 gården av Olof Dahlström. År 2012 byggde Risto ekonomibyggnaderna.


\jhoccupant{Dahlström}{Olof \& Sirkka}{1981--\allowbreak 1989}
Olof Dahlström, \textborn 1940 på Jungarå , gift med Sirkka Sipponen, \textborn 1938 i Alahärmä.
\begin{jhchildren}
  \item \jhperson{\jhname[Sven]{Dahlström, Sven}}{1960}{}
  \item \jhperson{\jhname[Carina]{Dahlström, Carina}}{1962}{}
  \item \jhperson{\jhname[Anne]{Dahlström, Anne}}{1966}{}
\end{jhchildren}

Huset, som Olof byggde år 1981, är byggt för två familjer. Olof bodde med sin familj i ena lägenheten. Sonen Sven bodde med sin familj i den andra lägenheten.

Sven Dahlström, \textborn 23.04.1960, gifte sig 27.12.1980 med Raija Suomalainen, \textborn 03.11.1962 i Alahärmä.
\begin{jhchildren}
  \item \jhperson{\jhname[Susann]{Dahlström, Susann}}{03.05.1981}{}
  \item \jhperson{\jhname[Toni]{Dahlström, Toni}}{26.12.1984}{}
\end{jhchildren}

Sven och Raija flyttade till Alahärmä och Olof och Sirkka sålde huset år 1989, köpte hus i Silvast (nr 50) och flyttade dit.



\jhhouse{Björkvalla}{5:25}{Grötas}{11}{35, 35a}


\jhhousepic{169-05898.jpg}{Björkvalls hus}

\jhoccupant{Björkvall}{Karl-Eriks dödsbo}{2010-}
Efter maken Karl-Eriks död flyttade Jeanette till Jakobstad. Huset är uthyrt och marken är utarrenderad.
Hyresgäster:
\begin{enumerate}
  \item 01.08.2013-   Mikael Jungell,  (Tollikko 10)
  \item 01.06.2011-31.05.2012  Shelley Halldorson och Victor Poza Moreno (från Canada)
\end{enumerate}


\jhoccupant{Björkvall}{Karl-Erik \& Jeanette}{1985--\allowbreak 2010}
Karl-Erik Björkvall, \textborn 05.07.1955 på Grötas hemman, gift med Jeanette Dahl, \textborn 03.11.1957, från Jakobstad.
\begin{jhchildren}
  \item \jhperson{\jhname[Fredrik]{Björkvall, Fredrik}}{22.07.1976}{}, bor i Vasa, ingenjör
  \item \jhperson{\jhname[Linda]{Björkvall, Linda}}{07.01.1980}{}, vestonom, g. Haga, arbetar vid Posti
\end{jhchildren}


\jhoccupant{Sven Björkvalls}{dödsbo}{1975--\allowbreak 1985}
Karl-Erik och Jeanette fortsatte med  jordbruket efter Svens död. De började med pälsdjur år 1978. De födde också upp svin och byggde nytt svinhus. Detta ägs nu av Stefan Ahlfors.

År 1985 övertog de hemmanet. Karl-Erik dog 11.05.2010. Jeanette flyttade till Jakobstad i januari 2011.


\jhoccupant{Björkvall}{Sven \& Hjördis}{1970--\allowbreak 1975}
Sven Björkvall, \textborn 08.10.1933 på Grötas hemman, gift med Hjördis Sjöblom, \textborn 10.10.1933 i Purmo.
\begin{jhchildren}
  \item \jhperson{\jhbold{\jhname[Karl-Erik]{Björkvall, Karl-Erik}}}{05.07.1955}{}
  \item \jhperson{\jhname[Birgitta]{Björkvall, Birgitta}}{25.12.1961}{}, gift Sundholm, företagare
\end{jhchildren}

År 1961 övertog Sven och Hjördis hans föräldrars lägenhet på Grötas. Den omfattade då 50 ha, varav 19 ha var odlad jord. Sven byggde nytt boningshus år 1970. Sven var en tid klubbmästare i Lions, ledamot i vägnämnden samt med i direktionen för ortens medborgarskola. Sven dog genom olyckshändelse 09.06.1975, Hjördis dog 19.12.1992.


\jholdhouse{Björkvalla sytningsstuga}{5:25}{Grötas}{11}{115}


\jhhousepic{Grotas135, 115.jpg}{Gamla gården och sytningsstugan}

\jhoccupant{Björkvall}{Anselm \& Irene}{1954--\allowbreak 1985}
Anselm  Björkvall,  gift med Irene Johansdotter Nyman från Kantlax (se nedan nr 135).
Detta var den sytningsstuga, som Anselm byggde åt sig själv och Irene då sonen Sven gifte sig och bildade egen familj. Några år efter Anselms död rev sonsonen Karl-Erik stugan. Den hade då stått tom i flere år.

Irene \textdied 17.09.1976  ---  Anselm \textdied 06.10.1985



\jholdhouse{Björkvalla Gamla gården}{5:25}{Grötas}{11}{135}


\jhoccupant{Björkvall}{Sven \& Hjördis}{1961--\allowbreak 1973}
Sven Björkvall på Grötas hemman, gift med Hjördis Sjöblom från Purmo. Text finns under Grötas gård nr 35, ovan.\jhvspace{}


\jhoccupant{Björkvall}{Anselm \& Irene}{1934--\allowbreak 1961}
Anselm Eriksson, senare Björkvall, \textborn 06.01.1898 på Grötas hemman, gift med Irene Johansdotter Nyman, \textborn 13.03.1898 i från Kantlax. Se karta 11, nr 110!
\begin{jhchildren}
  \item \jhperson{\jhname[Anni]{Björkvall, Anni}}{22.04.1922 (Mietala 18)}{}
  \item \jhperson{\jhname[Vera]{Björkvall, Vera}}{14.09.1924}{}, gift Sandvik i Jussila, Munsala
  \item \jhperson{\jhname[Märta]{Björkvall, Märta}}{20.03.1927}{}, gift Sarelin
  \item \jhperson{\jhname[Sven]{Björkvall, Sven}}{08.10.1933}{}
\end{jhchildren}
Anselm flyttade med hus och familj år 1934 till Björkvalla lägenhet. Samtidigt byggde han ekonomiebyggnaden i tegel. Enligt sägen har det funnits ett torp på denna plats.



\jholdhouse{Lillstugan, inkl. evakuerade}{5:25}{Grötas}{11}{35a}

\jhoccupant{Björkvall}{Maria Lovisa}{1934--\allowbreak 1935}
Anselm byggde åt sin mor en sytningsstuga i ena ändan av ekonomiebyggnaden. Hon bodde där till sin död 14.10.1935.
Rummet hyrdes senare ut  åt ett par flickor från Kantlax som 	arbetade på Kiitola.

Under krigsåren 1939 – 1944 bodde följande evakuerade i stugan:
\begin{enumerate}
  \item Kosti och Vappu Ryssyläinen, samt mamman Viktoria från Viborg. Mannen arbetade på ammunitionsfabriken. Familjen
	flyttade till Karleby.
  \item Otto, \textborn 1900 och hustrun Fenna, \textborn 1913 samt barnen Taisto, \textborn 1937, Terttu, \textborn 1939 samt Irma, \textborn 1941. Familjen kom från Suojärvi, flyttade till Alahärmä 1942. Taisto sjöng som ung i orkestrarna Salando samt Niemis orkester.
  \item Hösten 1944 kom från Kemijärvi Eino och Elsi Similä med barnen Seppo och Martti, som var nyfödd. De flyttade tillbaka till Kemijärvi.
  \item Paret Suoma Hikipää samt Erkki Vilén från Suomussalmi bodde endast en kort tid hos Björkvalls. De flyttade vidare till Tammerfors.
\end{enumerate}

Man inhyste också kreatur, som evakuerade haft med sig. Två extra kor hade man i fähuset en tid.



\jhhouse{Fogström}{5:92}{Grötas}{11}{36}


\jhhousepic{172-05735.JPG}{Bernhard Fogström}

\jhoccupant{Fogström}{Bernhard \& Thea}{1976--}
Bernhard Fogström, \textborn 24.10.1928 i Munsala, gifte sig 1960 med Thea Backa, \textborn 03.09.1935 i Oravais. Bernhard och Thea köpte tomten år 1974 av Uno Fagerholm. År 1976 var huset färdigbyggt och de kunde flytta in.
\begin{jhchildren}
  \item \jhperson{\jhname[Britt-Mari]{Fogström, Britt-Mari}}{19.03.1961, överläkare vid Jakobstads Hvc}{}
  \item \jhperson{\jhname[Mårten]{Fogström, Mårten}}{10.04.1966, dataprogrammerare vid ABB i Vasa}{}
  \item \jhperson{\jhname[Anders]{Fogström, Anders}}{10.04.1966, arbetar vid Intek i Jakobstad}{}
\end{jhchildren}

Bernhard har åren 1960--\allowbreak 1991 arbetat som verksamhetsledare vid Jeppo Skogsandelslag, Thea arbetade som hälsovårdare i Jeppo fram	till sin pensionering. Bernhard och Thea har innehaft olika förtroendeuppdrag, Thea var bl.a. under många år i föreningen Folkhälsans styrelse.

Thea \textdied 26.07.2010.



\jhhouse{Grötas/torp}{5:}{Grötas}{12 obs!}{116}


\jhoccupant{Jakobsson}{Gustaf \& Maria}{1866--\allowbreak 1872}
Torpare Gustaf Jakobsson, \textborn 12.04.1841 på Böös, gift 1866 med Maria Johansdotter Skog, \textborn 13.08.1841 på Skog.
\begin{jhchildren}
  \item \jhperson{\jhname[Johan Jakob]{Jakobsson, Johan Jakob}}{22.08.1866 (Enkvist)}{}
  \item \jhperson{\jhname[Sanna Sofia]{Jakobsson, Sanna Sofia}}{20.01.1869}{}, gift (Pass) Blomqvist i USA
  \item \jhperson{\jhname[Anders Gustaf]{Jakobsson, Anders Gustaf}}{10.02.1871 (Södergård)}{}
  \item \jhperson{\jhname[Emil]{Jakobsson, Emil}}{03.01.1873 (Sandås)}{}
  \item \jhperson{\jhname[Anna Lovisa]{Jakobsson, Anna Lovisa}}{04.02.1875}{}, gift Bloomquist i USA
  \item \jhperson{\jhname[Axel]{Jakobsson, Axel}}{23.07.1877 (Enqvist)}{}
  \item \jhperson{\jhname[Fredrik]{Jakobsson, Fredrik}}{18.06.1879}{}
  \item \jhperson{\jhname[Maria Johanna]{Jakobsson, Maria Johanna}}{18.12.1885}{}, gift Bergvall i Jakobstad
\end{jhchildren}

I mantalslängder finns Gustaf antecknad som torpare under Anselm Björkvalls farfars lägenhet under åren 1866--\allowbreak 1872. Gustaf övertog torparkontraktet efter sin far Jakob Isaksson. Gustaf köpte år 1872 en 5/48 mantals lägenhet på Böös, flyttade sedan dit med sin familj (Böös nr 44). Om han också flyttade torpet dit är oklart, men inga uppgifter har hittats att någon skulle ha bott på denna plats efter Gustaf.

Gustav \textdied 24.11.1930 på Böös  ---  Maria \textdied 19.04.1886


\jhoccupant{Isaksson/Böös}{Jakob \& Sanna}{1850--\allowbreak 1866}
Torpare Jakob Isaksson, \textdied 27.01.1801 på Böös, gift med Susanna Gustavsdotter Levälä, \textborn 13.02.1807 på Levälä.
\begin{jhchildren}
  \item \jhperson{\jhname[Cajsa]{Isaksson/Böös, Cajsa}}{10.09.1836}{}, gift Sundell i Nykarleby
  \item \jhperson{\jhbold{\jhname[Gustav]{Isaksson/Böös, Gustav}}}{12.04.1841}{}
  \item \jhperson{\jhname[Sanna Lovisa]{Isaksson/Böös, Sanna Lovisa}}{28.06.1844}{21.07.1852}
  \item \jhperson{\jhname[Maja]{Isaksson/Böös, Maja}}{26.10.1848}{}
\end{jhchildren}
Jakob var torpare på Böös. Han dog där 05.05.1866, Susanna dog efter 1870. Platsen för torpet baserar sig på karta, som Boris Sandås gjort enligt uppgifter av Eliel Nygård. Se karta 12!
