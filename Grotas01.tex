\jhchapter{Grötas, hemman Nr 5}

Ohls menar att namnet Grötas har ett fornnordiskt ursprung. Den första delen utgörs av grjöt = sten. Den andra delen är a \= å. Grötas skulle alltså betyda ``stenån''. Också Karsten har samma uppfattning; Grötas kommer från gröt, gröut (stamform graut) \= massa av småsten. Det låter sig sägas, som så ofta när det gäller ortnamn. Frågan inställer sig varför vi inte finner detta namn på flera ställen längs ån där det finns massor av stenansamlingar? Det är frestande att i sammanhanget återge allmogens uppfattningar om ursprunget till namnet, sannolikt tillkomna efter Stora ofreden, berättade med glimten i ögat;
\jhbold{a)} Folket från Grötas var ivriga jägare och var flera dagar borta och jagade. När de kommo hem, plägade de säga: ``Nu skall vi ha oss något lite gröt'' och så fingo de namnet Grötas.
\jhbold{b)} En gång gick gubben på Ojas upp längs åliden och kom till Grötas. Där kokade man just gröt, och därav gav han namnet Grötas åt gården.

Uppfattningen att Grötas är den äldsta bosättningen i Jeppo härstammar från samma sägner. Enligt dessa ``började den första svensken i Jeppo bo på Grötas, det äldsta stället i Jeppo. Han var jägare. På den nummern har därefter alltid bott jägare. Byggningen stod på åbacken. En dag såg karlen en vit tvet (huggspån) komma drivande längs ån. Han gick då uppåt för att söka sin granne och träffade honom vid Heikkilä i Alahärmä åtta km högre upp. Han klandrade ändå den andre för att han hade byggt sig för nära!'' Var och en får avgöra sanningshalten i dessa berättelser, men av någon anledning fick Grötas hemman nummer 1 (ett) när storskiftet genomfördes i slutet av 1700-talet. På initiativ av överdirektören i den kungliga lantmäteristyrelsen, Jacob Faggot, beslöt nämligen riksdagen 1755-56 att ett storskifte skulle genomföras i Finland för att förbättra jordbrukets produktionsmöjligheter. En mätning och kartläggning måste ligga till grund för denna process och den genomfördes redan 1740 av bl.a. Matthias Wörgren (se Skog).

År 1783 ägdes Grötas av Mats 21/64 mtl, Daniel 7/32 mtl och  ..?. 7/32 mtl., men innan dess av bl.a. Sigfrid Persson 1592-1625, Erik Sigfridsson 1625-1653, Sigfrid Eriksson  1657-1673, Erik Sigfridsson med hu. Lisa Tomasdr. 1675-1694, sonen  Anders Eriksson med hu. Karin Andersson 1695-1697, svågern Erik Markusson med hu. Brita Eriksdr. 1698-1713 och som medåbo dennes svåger Mickel Hansson med hu. I Elsa och hu. II Gertrud från 1699.

Vägen genom Grötas och söderut gick tidigare längre österut. Halvvägs mellan ån och dagens järnväg mynnade den ut till dagens vägsträckning strax före Bösas bäck. Längs denna väg växte bebyggelsen fram. Åsen kallades då `Furukangan'', beväxt med ståtlig tallskog därifrån timmer till Nykarleby nya kyrka, invigd 1:a advent 1708, sägs ha avverkats och flottats ner längs älven. Sannolikt fanns ändå den allra första vägen längs åkanten då skogsbrynet inte låg långt borta. Grötas var i medlet av 1800-talet det folkrikaste hemmanet i Jungar by med ca 120 personer. Grötas har också i något sammanhang haft en relation till staden Nykarleby då på dess mark finns ett område kallat ``stadsbohagen''. Kanhända har det sin bakgrund i den landslag av år 1442 där det stadgades, att taverner eller gästgiverier skulle inrättas vid allmänna vägar på två, eller högst två och en halv mils avstånd från varandra. Redan under första delen av 1500-talet var allmogen i Österbotten underkastade detta åtagande och oftast var det herremän som skulle skjutsas. De som reste i konungens och befallningsmans ärenden, skulle få ``mat och öl'' och ``bliva befordrade med skjutshästar och andra skäliga nödtorfter gratis''. Att vägra ledde till åtal. och böter.  I vilket fall anses Grötas vara det hemman som tidigast härbärgerade ett gästgiveri öster om älven, och det omnämns redan 1635. Den troliga platsen var i närheten av där Ahlfors hemman finns idag. Förmodligen fanns en gång i tiden ett område, nu på järnvägens östra sida, där stadens befolkning vid behov kunde rasta sina hästar under resa, d.v.s. Stadsbohagen.

På Grötas stod också soldattorp 139, som upprätthölls av roten; Lawast, Lillsilfvast, Grötas och Rämäcki. Det har genom tider kallats Långängen, Langängen och Lawast (se mera under Soldattorp Nr 139 Grötas).

Insprängt i Grötas geografiska område finns också delar av hemmanen Holm, Skog, Romar, Fors och  Silvast. Efter 1831 fick Grötas nummer 5 och har tydligen alltid varit ett skattehemman och till ingen del ett kronohemman (Kronhemman \= ägt, uthyrt av kronan).

Idag finns också den tätaste bebyggelsen i Jeppo på Grötas hemman. Det är det  markområde som i början av 1970-talet köptes för att i första hand erbjuda ett expanderande Mirka en ny industritomt. Så blev det nu inte och istället planerades området väster om landsvägen till ett kommunalt bostadsområde för såväl egnahem som radhus och detsamma skedde med området öster om landsvägen. Området gränsar i norr till Fors hemman.

På 1950-talet fanns 8 aktiva jordbrukare på Grötas hemman. Idag finns endast en (1).

Grötas hemman omfattas av kartorna \jhbold{nr 10-11}.

KARTA nr 10 hit --->


\jhsubsection{Lägenheter på Grötas}

\jhhouse{Soldattorp Nr 139}{5:147}{Grötas}{10}{}
Detta torp upprätthölls av soldatroten bestående av hemmanen Lawast R:nr 10, Lillsilfwast R:nr 25, Grötas R:nr 1 och Rämäcki R:nr 17. Det utgjorde vad som då stadgades av 4 ¼ rotemantal. Under olika tider har det benämnts Långängen år 1737, Langängen 1752 och Lafwast från 1775.

Själva torpet fanns på Grötas hemmans mark, sannolikt alldeles norr om Bror och Siv Westins hus idag. Torpets åker, ``Kliftens lindor'', fanns ett stycke norrut på Lillsilfwast mark (Fors) och är nu till stor del under asfalt. Dels under vägen mot Oravais, dels under tennisplanen och konditionshallen.

Det är oklart om roten, som skulle uppgå till 4 ¼ rmtl för att kunna försörja en soldatfamilj, vid introduktionen av  indelningsverket alltid byggde upp ett nytt torp eller om gamla stugor kunde tas i användning. Påfallande ofta förefaller det som om torpens skick, vid den syn som hölls 15 år efter starten, var dåligt. Oftast var de nedersta stockvarven ruttna och fönstren blyfallna. Vid syneförrättning 1752 hade också detta torps stuga rivits ner och byggts upp på nytt med färska nedre stockvarv. Försett med ett nytt skurtak ansågs den ``i stånd till alla delar''. Samma vår hade rotesällarna byggt ett nytt fähus med nytt tak, medan ladans takved var rutten och taket läckte. Det lovade roten åtgärda, likaså en ny farstu mellan fähus och lada. Boden fann nåd som varande i försvarligt skick och låsförsedd.

Åkern fanns för hela roten och varande i fullt bruk. På Lafwast var en gärdesgård i dåligt skick. Ängarna som skulle ge foder till husdjuren var delvis ``av inttet wärde'' och kompenserades istället med ett stycke skattejord till dess en större äng hunnit röjas och iståndsättas. Vid nästa antecknade syn år 1791 var husen och uthusen i gott skick. Åkern omfattande 1 tunnland fanns tillgänglig för torpet, likaså ett 2 kappland stort kålland. Ängarna ``wälhäfdade'' och ansågs kunna avkasta stipulerade 12 skrindor hö. En ny brunn skulle grävas under hösten. Tydligen ansågs åvattnet redan då olämpligt att dricka för folk, även om det fanns nära till hands.

Klift var ett gångbart namn på de första knektarna som innehade  torpet:
\begin{enumerate}
  \item Anders Andersson Klift, antagen 1713. Ankom ca 1734. Kvarstod till 1741. Befordrad till korpral N:o 1.
  \item Matts Stephansson Klift, 1742 – 1743. Död hemma på roten efter att ha kommit hem svårt misshandlad under fångenskap i krigets slutskede och där ådragit sig svår diarré.
  \item Matts Eliasson Klift, 1745 – 1758. Död i Pommern.
  \item Hans Larsson Klift, 1758 – 1779. Föravskedad. Rymt?
  \item Johan Persson Klift, 1779 – 1789. Transport till Nr 131 i Ytterjeppo.
  \item Gustaf Mattsson Torn, 1789 – 1791. Byte till Nr 113 i Munsala.
  \item Jakob Johansson Högbom, 1791 – 1804. Död hemma på roten.
  \item Isaac Henricsson Skön,  1805 – 1808 ? Stupade i kriget.
\end{enumerate}
Efter fredsslutet i Fredrikshamn 17.09.1809 höll ryssarna kvar militär i Finland ända till 1881, då landet fick egen militär. Inkvarteringen blev nu istället för knekthållet en börda för bönderna i de socknar, där de ryska soldaterna var förlagda. Husrum skulle upplåtas för officerare och manskap. Stall skulle hänvisas för deras hästar och betesmarker sommartid. Ved skulle fram till manskapet och exercis- och lägerplats för truppernas behov upplåtas. Huruvida dessa pålagor drabbade Jeppobygden är inte dokumenterade. Däremot tvingades allmogen att underhålla de gamla torpen, vilket väckte motstånd. Det sista torpet avvecklades i Jeppo först 1909. Det var torp Nr 137 , ``And'as torp''.

När indelningsverket avvecklades 1810 fick officerare och även underofficerarna bibehålla sina boställen och sin lön. Detta gällde bland annat Kamppinens boställe. Men soldaterna miste sina torp och massor av invalider, soldater, soldatänkor och uttjänta soldater fick t.o.m. ta till tiggarstaven. Hård var deras lott.



\jhhouse{Westin}{5:8}{Grötas}{11}{107}

\jhoccupant{Westin}{Johan \& Anna-Lovisa}{1910-1933}
Senast inneboende var Johan Henrik Johansson Grötas, senare Westin, \textborn 25.12.1881, gifte sig 1903 med Anna Lovisa Andersdt. Eklund, \textborn 05.12.1882 (nr 109). Johan och Anna-Lovisa vistades i Amerika åren 1903-09 och Johan ensam 1910-1911. År 1910 tillhandlade sig Johan hemgården och  brodern Emil jordlägenheten. Enligt köpebrev av den 22.04.1913, köpte makarna 0,0821 mantal av Grötas skattehemman nr 5 (nr 109), säljare är Anna Lovisas föräldrar Anders Eriksson och hustrun Anna Gustavsdt. Ruotsala. Efter storskiftet 1932-1933 flyttade familjen bostadshuset och byggde nya ekonomiebyggnader på ny plats (nr 108). Familjen med hemmavarande barn och kreatur flyttade med.
\begin{jhchildren}
  \item \jhperson{Johannes Edvin}{29.10.1904 i USA}{1983}, fsk.lär. Purmo, Jeppo
  \item \jhperson{Anders Ivar}{21.04.1906 i Amerika (nr 101)}{}
  \item \jhperson{Ester Maria}{30.06.1908 i Amerika}{10.12.1909}
  \item \jhperson{\jhbold{Henrik Valfrid}}{21.05.1910 i Jeppo (nr 8)}{}
  \item \jhperson{\jhbold{Elis Verner}}{29.08.1912 i Jeppo ogift (nr 102)}{}
  \item \jhperson{Agnes Elvira}{11.12.1914 i Jeppo, gift Sandberg}{2006}
  \item \jhperson{Erik Runar}{19.05.1917 i Jeppo, gift i Ingå}{16.11.1968}
  \item \jhperson{Artur Evert}{13.05.1920 i Jeppo, sambo i Sverige}{}
  \item \jhperson{Lennart Wilhelm}{02.11.1922 i Jeppo, g., frånsk., lektor i Borgå}{}
  \item \jhperson{Hjördis Irene}{29.01.1926 i Jeppo, gift Finskas}{15.08.2013}
\end{jhchildren}


\jhoccupant{Grötas}{Johan \& Maria}{1894-1910}
Johan Henriksson Grötas, \textborn 19.07.1859, gift 1879 med Maria Danielsdt. Levelä, \textborn 14.01.1854. Johan och Maria blev bönder på Grötas sedan han övertagit hemmanet av morbrodern Erik efter dennes död 1894. Ett år senare köpte de, änkan Sofia Gustavsdt. Tollikos 61/768 mantal av Grötas 5.  Efter överlåtande av egendomen 1910, flyttade de med yngsta sonen William till torpet (nr 113). Johan var på arbetsförtjänst till Amerika år 1887.
\begin{jhchildren}
  \item \jhperson{Brita Sofia}{29.01.1880}{}, gift Sandlin (nr 127)
  \item \jhperson{\jhbold{Johan Henrik}}{25.12.1881}{}, (nr 108, 102)
  \item \jhperson{\jhbold{Anders Emil}}{28.03.1884}{}, (nr 112,113)
  \item \jhperson{Erik William}{24.08.1886}{1892}
  \item \jhperson{August William}{21.07.1896}{}, (nr 113)
\end{jhchildren}
Johan Henriksson \textdied 09.04.1911  --  Maria \textdied 12.07.1942


\jhoccupant{Grötas}{Johannes \& Sanna Lisa}{1880-1894}
Johannes Eriksson Grötas, \textborn 07.06.1860 på Grötas, son till Erik Olof Olofsson, gift med Sanna Lisa Andersdt. Silvast, \textborn 23.09.1863, köpte år 1880 61/168 mantal av Grötas 5 av fadern Erik Olofsson,  familjen emigrerade till Amerika 1894.
Barn: Anna-Lovisa, \textborn 23.08.1893 i Jeppo - \textdied 1977 i Amerika
Johannes \textdied 1905  --  Sanna-Lisa \textdied 1908, i Amerika.
                          .

\jhoccupant{Grötas}{Erik \& Anna-Maria}{1855-1880}
Erik Olofsson Grötas, \textborn 06.05.1835, son till bonden Olof Olofsson Grötas, gift med Anna-Maria Johansdt. Öhman, \textborn 03.10.1837 på Silvast.

Bröderna Erik och Matts övertog hemmanet år 1855 av fadern Olof. (Samtidigt tryggas systern Brita-Lenas och hennes man torparen Henrik Johanssson vid sitt torparkontrakt av år 1854.) Brodern Matts gifte sig med Maria Andersdotter Skog, \textborn 11.08.1837, de bosatte sig på Skog. Erik bosatte sig på hemmanet och köpte till jord 1870 av Anders Anderson Grötas 61/768 mantal.

Hustrun Anna-Maria \textdied före år 1887  --  Erik \textdied 1894. Alla barn emigrerade till Amerika. Efter Eriks död övertogs hemmanet av systersonen Johan Henriksson Grötas.
\begin{jhchildren}
  \item \jhperson{Anna-Lena}{12.09.1858}{01.10.1859}
  \item \jhperson{\jhbold{Johannes}}{07.06.1860}{1905 i Amerika}
  \item \jhperson{Erik}{25.09.1862}{07.04.1889 i Amerika}
  \item \jhperson{Matts}{06.06.1865}{}, gift, till Amerika 1890
  \item \jhperson{Anna-Lovisa}{23.10.1867}{17.12.1867}
  \item \jhperson{Anders Gustav}{11.05.1871 ?}{}
  \item \jhperson{Jakob Wilhelm}{05.10.1876}{},dog i Amerika,ogift
\end{jhchildren}


\jhoccupant{Grötas}{Olof \& Caisa-Greta, Maria, Magdalena}{1824-1855}
Olof Olofsson Grötas, \textborn 25.05.1802 på Backhemming i Soklot, köpte en lägenhet på 5/32 mantal av Grötas 5, år 1824 av Henrik Danielsson och hustrun Maria Simonsdt. Olof var gift 3 ggr.: 1-gifte med Caisa-Greta Henriksdt. Grötas, \textborn 28.08.1804, \textdied 24.02.1828, 2-gifte med Maria Eriksdt. Flod, \textborn 17.05.1811, \textdied 07.12.1830, 3-gifte med Magdalena Mattsdt. Pesonen, \textborn 06.10.1806, \textdied 20.06.1854.

De var jordbrukare. Av vem och när bostaden med ekonomiebyggnader byggts vet man ej. Olof \textdied 02.01.1867 på Grötas.
\begin{jhchildren}
  \item \jhperson{Erik}{05.09.1826}{07.08.1828}, 1:a giftet
  \item \jhperson{Henrik}{17.12.1827}{07.08.1828}, 1:a giftet
  \item \jhperson{\jhbold{Brita-Lena}}{07.09.1829}{}, 2:a giftet, g. m Henrik Johansson (nr 120)
  \item \jhperson{Maija-Lovisa}{08.12.1832}{dog 01.05.1833}, 3-giftet
  \item \jhperson{Matts}{11.12.1833}{23.03.1867 på Skog i Jeppo}, 3-giftet
  \item \jhperson{\jhbold{Erik}}{06.05.1835}{19.04.1894}, 3-giftet
  \item \jhperson{Anna}{04.11.1837}{06.02.1838}, 3-giftet
  \item \jhperson{Lisa}{02.08.1841, gift Eklund i Lassila}{09.06.1930}, 3-giftet
  \item \jhperson{Sofia}{11.04.1847}{}, emigrerade till Sverige, 3-giftet
\end{jhchildren}



\jhhouse{Westin}{5:27}{Grötas}{11}{101}

\jhoccupant{Westin}{Ivar \& Edit}{1933-1969}
På grund av den nya vägsträckningen över holmen , inlöste Väg- och vattenbyggnadsdistriktet jordområdet med byggnader 1969 av Ivar och  Edit Westin.  Ivar rev husen.

Anders Ivar Westin, \textborn 21.04.1906 i Amerika, gift 22.02.1931 med Edit Maria Jungarå \textborn 29.09.1905  (Jung.       ).  De köpte tomten av  Ivars  föräldrar 1933 och byggde gård och ekonomiebyggnader. Ivar var på arbetsförtjänst till Amerika 1926-1930 och 1934-1937.  Efter hemkomsten 1934 hyrde de ut bostaden i Jeppo och köpte hus i Jakobstad. Ivar arbetade där åt Ab Haldin \& Rose. De återvände  med familjen till Jeppo 1945, sålde huset i Jakobstad år 1946. Ivar arbetade åt Jeppo-Oravais Handelslag tills han blev pensionär. Makarna var också småbrukare. Efter försäljningen av bostaden, köpte de bostad i Silvast (Sil….). Edit dog i trafikolycka 03.08.1977 och Ivar dog 12.09.1991.
\begin{jhchildren}
  \item \jhperson{Doris Regina}{02.12.1931}{}, gift Lundvik (Silvast nr 79)
  \item \jhperson{Dorthy}{19.04.1939}{}, gift Gunnar (Gunnar nr  )
\end{jhchildren}



\jhhouse{Solbacka}{5:18}{Grötas}{11}{102}

\jhoccupant{Nygård}{Melita \& Evald}{1976-1981}
Nykarleby stad köpte tomten med byggnader den 21.08.1981 av Melita Gunvor Johanna Nygård \textborn Wikström 22.02.1928, och resten av lägenheten sålde hon åt köpvilliga jordbrukare. Melita var gift med Karl Evald Nygård, \textborn  12.08.1927 i Pensala, han dog 2013. De har båda arbetat åt Jeppo-Oravais Handelslag och var en lång tid ansvariga för Ytterjeppo filial. Efter pensioneringen köpte de lägenhet i Nykarleby centrum. Melita \textdied 05.02.2017.


\jhoccupant{Wikström}{William \& Sigrid}{1926-1976}
Senast boende var William Wikström, \textborn 11.09.1902 (nr 103), gift 1926 med Sigrid Johanna Ekblad, \textborn 17.10.1906. De köpte lägenheten av Williams svåger Mantere år 1926, med cykelförsäljning och reparationsverkstad som William fortsatte med. Under åren förstorade de ekonomiebyggnaderna och köpte tilläggsjord. De var småbrukare och företagare. William ägde också dåvarande Kilen 3:51 (nuv. Jeppo Krafts kontor) från 1967 fram till sin död (se Fors nr 96).

William dog 24.04.1976, Sigrid flyttade till pensionärshusen, hon dog 19.09.1984.
Barn:\jhbold{Melita} Gunvor Johanna, \textborn 27.02.1928 --  \textdied 2017.

\jhoccupant{Mantere}{Mattias \& Sanna}{1915-1926}                    .
Mattias Eemeli Mantere, \textborn 17.04.1888 i Alahärmä, gifte sig 1911 med Sanna Maija Wikström, \textborn 10.07.1890 (nr 103), som dog 18.08.1928. Mantere gifte om sig med Sannas syster Anna Sofia, \textborn 09.03.1893, som blivit änka efter Leander Lundqvist från Lassila. Mantere hade år 1915 genom  muntligt avtal med sin svärfar, Gustav Wikström (nr 103), erhållit området där han uppförde bostadshus med verkstad och uthus. Mantere började med cykelförsäljning och reparation. Efter försäljningen av verksamheten 1926, flyttade familjen till Nykarleby och fortsatte där med samma verksamhet.
\begin{jhchildren}
  \item \jhperson{August Eemeli}{07.01.1912}{03.02.1925}
  \item \jhperson{Auni Aili Maria}{29.09.1914}{1925}
  \item \jhperson{Aina Juliana}{31.03.1917}{}
  \item \jhperson{Elna Johanna}{23.01.1920}{}
  \item \jhperson{Johannes Wilhelm}{22.06.1922}{01.08.1940}
  \item \jhperson{Elna Alice}{12.09.1924}{29.03.1925}
  \item \jhperson{Göta Helmi Alice}{21.07.1926}{}
  \item \jhperson{Veikko Eemeli}{13.08.1928 i Nykarleby}{26.06.1932}
\end{jhchildren}
Mantere Matttias \textdied 04.01.1962  --  Anna Sofia \textdied 18.06.1980, de är begravda i Jeppo.



\jhhouse{Wikström}{5:163}{Grötas}{11}{103}

\jhoccupant{Köykkä}{Marita \& Heikki}{}
Bostadshuset revs 1998,  ekonomiebyggnaderna på 1980-talet. Lägenheten ägs av Marita Köykkä Wikström, \textborn 02.05.1949, gift med Heikki Köykkä, \textborn 1949. De är pensionärer, bor i Vasa.  Marita är merkonom ,jobbat på Citec i Vasa och Heikki som byggmästare åt Vasa stad.  De har sålt en del av jordlägenheten och arrenderat ut resten.
Barn: Mika, \textborn 1975


\jhoccupant{Wikström}{Edvin \&  Bertta, Senja}{1948-1973}
Urho Edvin Wikström, \textborn 16.05.1916, gift 1947 med Bertta Julia Laita, \textborn 03.09.1914 i Alahärmä, övertog vid arvskiftet 1948 tillsammans med Edvins syster Senja Juliana, \textborn 21.01.1905, bostadshuset och ekonomiebyggnaderna, samt det mesta av jordlägenheten. Inhyses var systern Ida, som gifte sig Wiik 1941, flyttade till Jakobstad 1947, samt Volmar, \textborn 1937,  son till brodern Aarne, som blev änkling 1940. Volmar gick i finska skolan i Silvast, flyttade till sin far i Seinäjoki 1953.  Edvin deltog i båda krigen.

De byggde nytt fähus 1948, brukade lägenheten tills de blev pensionärer 1973 och flyttade till pensionärshus i Jeppo.  Edvin och Bertta flyttade till Vasa 1980. - Senja textdied 15.01.1991.

Edvin \textdied 25.04.1994  --  Bertta \textdied 07.12.2007.
Barn: \jhbold{Doris Marita Helena}, \textborn 02.05.1949, gift Köykkä, bor i Vasa.


\jhoccupant{Grötas/Wikström}{Gustav \& Adolfiina}{1897-1948}
Gustav Johan Grötas, senare Wikström * 31.07.1863 i Alahärmä, gift med Maria Adolfiina *17.06.1869 i Alahärmä, ärvde ½ lägenheten med bostadshus och ekonomiebyggnader 1897 av fadern Johan Gustavssons dödsbo. De brukade lägenheten  tillsammans med barnen och förstorade bostadshuset.

Gustav \textdied 22.03.1943  --  Adolfiina \textdied 06.12.1951
\begin{jhchildren}
  \item \jhperson{Hanna Adolfiina}{06.08.1887}{}, gift Myllykoski (nr 106)
  \item \jhperson{Sanna Maija}{10.07.1890}{}, gift Mantere (nr 102)
  \item \jhperson{Anna Sofia}{09.03.1893}{}, gift 1. Lundqvist, 2. Mantere (nr 102)
  \item \jhperson{Johannes Einar}{12.11.1895}{}, till Amerika 1913
  \item \jhperson{Gustav}{30.12.1897}{}, till Amerika 1923
  \item \jhperson{Nikolai}{24.03.1900}{}, g. m Saimi Laurila \textborn 1896, bodde på Kaupp,  i Kokkola
  \item \jhperson{Viktor}{24.03.1900}{21.06.1900}
  \item \jhperson{William}{11.09.1907}{}, (nr 102)
  \item \jhperson{\jhbold{Senja Juliana}}{21.01.1905}{15.01.1991}, ogift
  \item \jhperson{Ida Katarina}{04.05.1907}{}, gift Wiik, Jakobstad
  \item \jhperson{Aarne Emeli}{30.08.1909}{09.09.1909}
  \item \jhperson{Helmi Aina Karin}{15.08.1910}{25.05.2001}, gift Sandell i Jeppo
  \item \jhperson{Aarne Valtteri}{30.05.1913}{13.07.1976}, gift, änkling 1940
  \item \jhperson{\jhbold{Urho Edvin}}{18.05.1916}{25.04.1994}, gift m Bertta
\end{jhchildren}


\jhoccupant{Paalanen/Grötas}{Johan \& Johanna}{1886-1897}
Johan Gustavsson Paalanen/Grötas, \textborn 19.09.1833 i Alahärmä, och hustrun Johanna Kristiina Johansdt., \textborn 03.01.1837 i Alahärmä, köpte lägenheten 7/64 dels mantal av Grötas skattehemman  nr 5, 1886 av bönderna Karl Johansson Grötas och Johan Johansson Grötas. Enligt brandförsäkringen bestod byggnaderna av ett boningshus, en spannmålsboda, en mindre spannmålsbod, en bod med loft, ett stall jämte lada under samma tak, ett fähus, tvenne foderlador, ett vedlider en badstuga samt en ribyggnad. De idkade jordbruk med kreaturskötsel.

Johan Gustav \textdied 04.01.1897  --  Johanna Kristiina \textdied 18.12.1900

Alla barn var födda i Alahärmä.
\begin{jhchildren}
  \item \jhperson{\jhbold{Johan Gustav}}{31.07.1863}{}
  \item \jhperson{Adolfiina}{05.03.1866}{}, gift Samuel Roos (nr 104)
  \item \jhperson{Johan}{24.06.1869}{}, tog namnet Salo (Silvast       )
  \item \jhperson{Salomon}{31.10.1873}{1912}, gift, till Amerika 1890
  \item \jhperson{Viktor}{15.09.1876}{}
  \item \jhperson{Nikolai}{18.07.1879}{}, tog namnet Kennola (nr 131)
\end{jhchildren}



\jhhouse{ Åbacken}{5:145}{Grötas}{11}{7}

\jhoccupant{Westin}{Bror \& Siv}{1996 -}
Bror Henrik Westin, \textborn 04.04.1953 (nr 102), gifte sig 12.07.1991 med Siv Marianne Sundqvist, \textborn 03.05.1966  i Öja , köpte 1995 tomten med föräldrarnas bostadshus (nr 8). De byggde nytt bostadshus bredvid 1996. Bror har arbetat som arbetsledare på KWH-Mirka och är idag halvpensionär. Siv är församlingsmästare på Jeppo kapellförsamling.
\jhhousepic{128-05666.jpg}{}

\begin{jhchildren}
  \item \jhperson{Ola Henrik}{22.04.1993}{}, bor i Nykarleby, maskinskötare på Mirka
  \item \jhperson{Lucas André}{13.05.1997}{}, arbetar på Mirka
\end{jhchildren}



\jhhouse{Westin}{5:32}{Grötas}{11}{8}

\jhoccupant{Westin}{Bror \& Siv}{1995 -}
Huset obebott och ägs idag av Bror och Siv Westin (nr7).

\jhoccupant{Westin}{Valfrid \& Ingrid}{1955-1995}
Henrik Valfrid Westin, \textborn 21.05.1910 i Jeppo, gift 03.03 1940 med Ingrid Linnea Lillqvist, \textborn 28.04.1911 i Purmo. Valfrid deltog i krigen. Han flyttade till Jakobstad 1935 där han arbetade på Cellulosafabriken.  År 1948 vid arvskiftet övertog Valfrid och brodern Elis ½ lägenheten var av föräldrarna. Valfrid och Ingrid fortsatte med jordbruket, byggde nytt bostadshus 1955, på samma område vid älven.
\jhhousepic{129-05667}{}

I mitten på 1970-talet avslutade de jordbruksverksamheten och sålde ut. Valfrid jobbade också som byggnadsarbetare och Ingrid i pälsningsarbete på Keppo. Valfrid dog 27.12 1988 och Ingrid dog 06.04 2002, de sista fem åren bodde hon i  pensionärshus. Sonen Bror Henrik köpte bostadshuset med tomt 1995.
\begin{jhchildren}
  \item \jhperson{Magnus Valter Johannes}{07.05.1940}{}, gift, bor Sverige
  \item \jhperson{Lars Valdemar}{15.09.1941}{}, gift, bor i Sverige
  \item \jhperson{Sven Yngve}{25.11.1946}{},gift, bor i Jakobstad
  \item \jhperson{Anita Linnea}{05.06.1950}{}, gift Sundström bor i Jakobstad
  \item \jhperson{\jhbold{Bror} Henrik}{04.04.1953}{}, gift (nr7)
  \item \jhperson{Rut Maria}{29.11.1954}{}, sambo, bor i Jakobstad
\end{jhchildren}



\jhhouse{Westin}{5:27}{Grötas}{11}{108}

\jhoccupant{Westin}{Elis}{1948-1981}
Bostadshuset och ekonomiebyggnaderna revs i mitten av 1980-talet. Senast boende var Elis Verner Westin, \textborn 29.08.1912, ogift. Han och brodern Henrik Valfrid med hustrun Ingrid Linnea övertog ½ lägenheten var vid arvskiftet 1948. Valfrid och Ingrid byggde nytt bostadshus åt familjen 1955 (nr 8). Elis arbetade hela sitt liv med jordbruk och kreatur.  Deltog i krigen. Elis \textdied 27.10.1981

\jhoccupant{Westin}{Johan \& Anna-Lov.}{1933-1948}
Johan Henrik och Anna-Lovisa med familj flyttade  bostadshuset (nr 107) till den nya platsen efter storskiftet 1932-33 samt byggde nya ekonomiebyggnader. De var jordbrukare med kreatursskötsel. De överlät hemmanet 1948.
Barn och övriga uppgifter, se (nr 107).
Anna-Lovisa \textdied 31.12 1965  --  Johan Henrik \textdied 20.06 1967



\jhhouse{ Älvliden}{893-50-5003-1 \& 893-50-5001-5}{Grötas}{11}{9}

\jhoccupant{Fastighets Ab}{Älvliden}{1974 -}

Sedan medlet av 1960-talet hade den kommunala investeringtakten i olika fastigheter varit hög. En ny Centrumskola byggdes på Sågtomten och blev klar 1966. Kort därefter startade planeringen av ett nytt hus för den kommunala förvaltningen. I samma veva skulle en ny brandstation byggas med tillhörande bostad och därtill beslöts att också ett pensionärshus med 8 lägenheter skulle uppföras alldeles i närheten av kommunalhuset. Allt detta blev klart i slutet av 1968 och i mars 1969 kunde landshövdingen Martti Viitanen inviga allt det nya.
\jhhousepic{134-05671.jpg}{Älvliden 10, från åsidan}

Men det var inte slut med detta. Keppos expansion inom pälsnäringen samtidigt som Mirka kommit ordentligt igång med sin verksamhet vid Kiitola, gjorde att behovet av bostäder för arbetskraft, som pendlade in till orten växte. Sommaren 1971 inköptes delar av Roos lägenhet på Grötas av kommunen. Det hade två syften; att kunna påvisa tomtmark för Mirka, som hade börjat signalera om ett sådant behov för en ny hall för sin expanderande verksamhet och samtidigt säkerställa ett markområde för egnahemshus och eventuella radhus. Markköpet av Roos lägenhet fördelade sig på östra och västra sidan om vägen mot Voltti. Den östra skulle i första hand reserveras för Mirkas behov och den västra för bostäder.
\jhhousepic{131-05895.jpg}{Älvliden 7}

I nov. 1971 besluter kommunfullmäktige att godkänna planerna på att bygga ett hyreshus vid älven på det inköpta området. I budgeten intas ett anslag om 20.000 mk för mätning och kartläggning under år 1972. I mars 1972 väljs ingenjörsbyrån Jord \& Vatten från Lappo att utföra både generalplan och delplan för området. Det resulterar i en delplan som ger rum för 8 egnahemshus och ett större radhus. Länsarkitekten, som anlitas i sammanhanget, anser att samtidigt som byggplan uppgörs för centrumområdet, kunde också Furubacken flygfotograferas och anslutas. Redan före midsommar fattas beslut om att tomter kan delas ut på det nya området till hösten. I oktober 1972 fattas beslut om att till våren 1973 gräva ner 1300 m vattenledning och 300m avloppsledning på området. Avloppsledningen tänker man senare ansluta till eventuellt eget reningsverk. Fullmäktige meddelas att ansökan om ett bostadslån inlämnats till bostadsstyrelsen.

I november 1973 startar bygget av radhusprojektet, som omfattar 15 lägenheter i 2 huskroppar omfattande totalt 830 m² till en beräknad kostnad om 878.000 mk. Kommunen äger 84 \% av aktierna i det nya bolaget, som får namnet ``Fastighets Ab Älvliden''. Resten av aktierna delas mellan Ab Keppo Oy och Jeppo Församling. Bostadsstyrelsen har beviljat 330.000 mk i bostadslån, men höjer senare summan till 405000 mk.

Den 14 juli 1974 firas taklagsfest på det nya bygget och projektets ordf., bankdir. Johan Stenfors, kan meddela att de 15 lägenheterna borde vara klara för inflyttning i slutet av juli månad.



\jhhouse{Union}{5:95}{Grötas}{11}{11}

\jhoccupant{Cederström}{Håkan \& Lenelis}{2003 -}
Håkan, vvs-montör, och Lenelis, byråsekreterare och närvårdare (Ruotsala nr 20) köpte lägenheten 2003 av Boris Lindén och BSB-Mekan. Redan följande år inrättade Lenelis en present- och blombod 	i kafédelen. Den var verksam 2004-2007. År 2011 renoverade man sagda butik och förvandlade den till	kontor/bostad. Då sonen Jonas och hans sambo Paulina Peltorinne	(Silvast-Fors, karta 5, nr 70) flyttade hem från Sverige, kunde det unga paret ordna sitt boende här under åren 2012-2014. Parets son, Noah Cederström, föddes 19.07.2012.
\jhhousepic{154-05708.jpg}{}
Hallarna används f.n. som garage, verkstad, lagerrum och flisförråd.


\jhbold{Hyresgäster under perioden 1975-2016}

\jhoccupant{Källan}{förening}{2016 -}
Föreningen flyttade in som hyresgäst år 2016 i samband med att dess tidigare utrymme vägg i vägg med Café Funkis saneras för att	ingå i en förstorad kafélokal.


\jhoccupant{Norrback}{Bernt}{2006}
Under 2006 hyrde Bernt in sig med sin spirande verksamhet kring bilreparationer och motorrenoveringar. Se karta 6, nr 87.


\jhoccupant{KGF}{r.f.}{1997}
Sommaren -97 inledde nybildade Kvarnbackens Gymnastikförening r.f. inredningsarbete i Union-stationens serviceutrymmen. Man hade fått lov att utrusta och utnyttja västra hallen som tillfällig träningslokal. Den var råkall och dragig för lättklädda gymnaster, men fick till en början duga som provisorium under den varma tiden på året. Föreningens första	byuppvisning i sju discipliner (parterr, bom, ringar, räck, barr, hopp 	och bygelhäst) för en nyfiken publik skedde med framgång i denna miljö, alltmedan en permanent hall (Arcas) färdigställdes i höladan på Kiitolavägen 15, karta 6, nr 97e.


\jhoccupant{Kesko/}{K-Lantbruk}{}
Företaget med handelsman Tomas Kjellman var hyresgäst en kort period under 1990-talet innan han s.s. drog sig till Esse.


\jhoccupant{Karvonen}{Jari \& Jyrki}{1988-1992}
Bröderna Jari och Jyrki Karvonen drev Union-stationen tillsammans med konceptet bensinförsäljning och bilreparationer. Kafeterian var inte öppen.


\jhoccupant{Broman}{Ralf}{1982-1988}
Ralf köpte verksamheten 1982 och fortsatte med bensinförsäljningen, sålde oljor, däck och utförde däckreparationer m.m. Som ensam företagare med enbart enstaka hjälpande inhoppare blev arbetsbördan dryg.	Den underlättades inte heller p.g.a. den brand som den 3 oktober	1984 höll på att ödelägga verkstaden totalt. Eld hade uppstått i isoleringsmaterial, som BSB Mekan hade lagrat på utsidan vid västra	gaveln, och därefter spridit sig via nocken vidare in i hallen.


\jhoccupant{Grandén}{Tage}{1979-1981}
Tage var en person som verkade vara född med skiftnyckel och skruvmejsel i sina händer. Han var inte rädd för nya utmaningar i all sorts mekanik och behövde en verkstad för att få utlopp för sina talanger. I verkstaden och bränsleförsäljningen jobbade Håkan Back, David 	Nybyggar, Bror Fors och Lars Dahlqvist. Tage etablerade sig i egna 	utrymmen i Gränden år 1982.


\jhoccupant{Hankkija}{Labor}{1975-1979}
Via avtal i december -81 kunde Hankkija, Labor och Pellervo fortsätta använda verkstadsfunktionerna vid Union.
Företaget hyrde in sig med förnödenhetsbutik och fältförsäljning av traktorer, tröskor o.a. Tom Bäckstrand skötte butiken medan Tore	Wallin, Bengt Back och Johan Nylund dejourerade kl. 8-10 på förmiddagarna innan de drog ut i fält som försäljare av maskiner och anläggningar. Verksamheten levde förhållandevis trångt. Efter fyra år flyttade delar av den till finska skolan på Stationsvägen (karta 5,  nr 360).


\jhbold{Ursprunget}

\jhoccupant{Lindén}{Boris \& Gurli}{1969-2003}
Lägenheten på 6008 m2 inköptes år 1969 av Uno Nyberg, R:nr 5:29. Verkstaden med tillhörande kafé och bensinmack, ``Union'', började	byggas senhösten 1969 när det redan fanns ordentligt med snö på marken. Broder Börje deltog också i byggnadsarbetet. Den 375 m2 stora lokaliteten togs i bruk redan följande år då Boris och Gurli	(karta 6, nr 94) inledde verksamheten.

Gurli skötte kaféet och bokföringen. Boris och två anställda, Håkan Back från orten och Matti Riikonen från Oravais, tog hand om verkstaden och bensinstationen. Den 23 oktober 1970 annonserades följande i Österbottniska Posten:

\jhpic{Unionannons.png}{Annons i Österbottniska Posten (ÖP)}

Verksamheten var från början livlig, eftersom den nya omfartsvägen, Riks 19, byggdes då. Under den här perioden rätades  tillika stambanan ut vid Skog, norr om centrum, vilket medförde diverse reparations- och servicejobb på den tunga utrustningen.

Efter 1973 kom Gurlis författarskap mera i fokus och ett kafébiträde blev anställt. 'Farmor' Anni Lindén var även inkopplad i den skötseln. Sönerna Thomas och Andreas jobbade som bensinförsäljare under sina sommarlov. Säsongjobb förslog även åt andra intresserade; t.ex. Nils Forss fick här en början på sin arbetshistoria. Redan i byggnadsskedet anställdes Håkan Cederström från Ytterjeppo och	senare hans blivande fru, Lenelis Lindgren, Jeppo, som skötte kaféet	1973-1974; det var här de träffades och romansen fick sin början.	Också andra personer har skött kaféet i kortare perioder.

Verkstaden gav bröderna Boris, Stig och Börje Lindén möjligheter att testa nya idéer. Det var här man monterade ihop sin första prototyp	av gödsellastare avsedd för pälsfarmer. Det nystartade företaget Ab	BSB-Mekan Oy, vilket infördes i handelsregistret den 27.4.1979 (Yrityshaku), har därefter utvecklats med en alldeles egen historia och med verksamheten förlagd till Nykarleby.



\jhhouse{Ahlfors}{5:173}{Grötas}{11}{18}

\jhoccupant{Ahlfors}{Caroline \& Tommy}{2009 -}
Ägare Stefan Ahlfors (Grötas 20), köpte lägenheten med byggnader av Bruno och Agnes Ahlfors 1997. Boende är Caroline  Maria Ahlfors, \textborn 11.05.1984 (nr 20), sambo med Tommy Andreas Karlsson, \textborn 08.02.1982 i Korsholm, de flyttade in år 2009.

Caroline är utbildad sjuksköterska, Tommy har teknisk utbildning, arbetar på Wärtsilä i Vasa. De har grundreparerat och förstorat gården.
\jhhousepic{153-05709.jpg}{}
\begin{jhchildren}
  \item \jhperson{Sigge Anders}{18.08.2011}{}
  \item \jhperson{Molly Maria}{24.05.2015}{}
\end{jhchildren}


\jhoccupant{Ahlfors}{Bruno \& Agnes}{1954-1997}
Henrik Bruno Ahlfors, \textborn 12.06.1921 (nr 20), gifte sig 1946 med Agnes Seppänen, \textborn 21.12.1920 i Salla.  De köpte bostadstomten av Brunos mors dödsbo 1950 och tilläggsjord av brodern Tor 1955. Makarna byggde gården och ekonomiebyggnaderna 1952-54 samt hönshus 1955. Bruno hade reparationsverkstad i Silvast samt försäljning av motorcyklar, velocipeder och radioapparater.  Han deltog i fortsättningskriget. Familjen emigrerade 1960 till Sverige.  Lägenheten med byggnader har varit uthyrd  åt bl.a. Raul och Maija Saviaro, Olof Laxéns familj, Martti Ahos familj och Pekka och Margit Nikkanen under åren 1961-1997.
\begin{jhchildren}
  \item \jhperson{Ulla Karin}{10.12.1947}{}, gift, bor i Sverige
  \item \jhperson{Krister Henrik}{21.05.1952}{}, gift, bor i Sverige
\end{jhchildren}
Bruno \textdied 14.10.2008  --  Agnes \textdied 09.05.2005, båda i Sverige



\jhhouse{Ahlfors}{5:173}{Grötas}{11}{20}

\jhoccupant{Ahlfors}{Stefan \& Monica}{1975 -}
Ägare och boende Stefan Anders Ahlfors, \textborn 25.05.1952 och sambo Monica Ulla Alice Litens, \textborn 06.08.1953 i Nykarleby. Stefan övertog lägenheten 1975 av sina föräldrar, 35 ha skog och 17 ha odlat. I dag har han ca. 100 ha odlat plus en del arrendejord.  Föräldrarna hade avslutat mjölkproduktionen. Stefan började med svinuppfödning, avslutade den 1984, satsade på sädes- och potatisodling. Han byggde en sädestork 1988 samt förstorade ekonomiebyggnaderna  år 2000, köpte Björkvalls svinhus (nr 20b), byggde om det till potatislager 2015.

\jhhousepic{152-05710.jpg}{}

Stefan rev det gamla bostadshuset och uppförde nytt på samma tomt 1985.
\begin{jhchildren}
  \item \jhperson{Paul Kenneth}{20.01.1977}{}, agrolog, bor i Norge,chaufför
  \item \jhperson{Carolina  Maria}{11.05.1984}{}, sjuksköterska (nr 18)
\end{jhchildren}


\jhoccupant{Ahlfors}{Birger \& Alice}{1953-1975}
Anders Birger, \textborn 04.03.1919, gift 1945 med Maria Alice Lillbacka, \textborn 14.07.1926 i Oravais. Efter giftermålet ombyggdes boningshuset till två lägenheter och hemmanet omhändertogs tillsamman med änkan Anna Lovisa och brödern, tills arvsskiftet 1953. Birger köpte tillbaka bröderna Tors och och Pauls arvslotter och utvidgade ekonomiebyggnaderna.

Birger tjänade landet i båda krigen, han lär ha varit en orädd matkusk. Birger och Alice var mjölkbönder.
\begin{jhchildren}
  \item \jhperson{Solbritt Alice}{30.10.1945}{}, gift Visti, bor i Sverige
  \item \jhperson{\jhbold{Stefan} Anders}{25.05 1952}{}
\end{jhchildren}
Birger \textdied 22.04.1979  --  Alice flyttade till Älvbranten 1983, \textdied 14.08.2012


\jhoccupant{Ahlfors}{Axel  Anna-Lovisa}{1926-1953}
Axel Vilho Henrikkson Grötas, tog namnet Ahlfors, \textborn 26.11.1892 i Amerika, gift 1915 med Anna-Lovisa Andersdt.Grötas, \textborn 22.12.1890. Övertog hemmanet 1926, de var bönder med sädesodling och boskapsskötsel.
\begin{jhchildren}
  \item \jhperson{\jhbold{Anders Birger}}{04.03.1919}{}
  \item \jhperson{Henrik Bruno}{12.06.1921}{}, (nr 18)
  \item \jhperson{Helge Axel}{15.01.1925}{15.03.1998}, ogift till Canada 1951
  \item \jhperson{Paul Magnus}{16.04.1927}{1994}, gift, flyttade till Jakobstad
  \item \jhperson{Tor Erik}{12.11.1931}{2008}, gift, flyttade till Jakobstad
\end{jhchildren}
Axel \textdied 30.05.1942  --  Anna-Lovisa \textdied 07.09.1954


\jhoccupant{Grötas}{Henrik \& Sofia}{1895-1926}
Henrik Johansson Grötas, \textdied 04.05.1860  i Alahärmä, gift med Sofia Johanna Jakobsdt Mietala, \textborn 02.08.1864, ansöker om lagfart på 7/64 del mantal av Grötas skattehemman nr 5  1895, föräldrarna var säljare. Före det hade de varit på arbetsförtjänst i Amerika. De fyra första barnen var födda i Amerika. Makarna var jordbrukare med kreatur och sädesodling.
\begin{jhchildren}
  \item \jhperson{Johan Jakob}{04.06.1885 i Amerika}{}, på nytt till Amerika 1910
  \item \jhperson{Henrik Joel}{26.12.1889 i Amerika}{}, på nytt till Amerika
  \item \jhperson{\jhbold{Axel Wilho}}{26.04.1892 i Amerika}{30.05.1942}
  \item \jhperson{Anna Sofia}{22.10.1893 i Amerika}{18.12.1897}
  \item \jhperson{Hanna Johanna}{04.11.1895 i Jeppo}{08.11.1897}
  \item \jhperson{Otto Henrik}{19.06.1898 i Jeppo}{25.12.1898}
  \item \jhperson{Johannes Artur}{19.06.1898 i Jeppo}{24.12.1898}
  \item \jhperson{Hugo Selim}{11.09.1909}{11.09.1954}, (Jungar     )
\end{jhchildren}
Henrik Johan \textdied 15.12.1909  --  Sofia Johanna \textdied 23.11.1930

\jhoccupant{Grötas/Tyni}{Johan}{- 1895}
Johan Henriksson  Grötas/Tyni f. 22.02. 1833 i Alahärmä, gift med Anna Gustavsdt.f 03. 08.. 1838 i Alahärmä. De har ett fastebrev från 11.09 1886.  Makarna var jordbrukare. Samtliga barn föddes i Alahärmä.
\begin{jhchildren}
  \item \jhperson{\jhbold{Henrik Johan}}{04.05.1860}{}
  \item \jhperson{Vilhelmina}{22.12.1861}{}
  \item \jhperson{Johan}{31.12.1863}{}
  \item \jhperson{Gustav}{07.12.1865}{}
  \item \jhperson{Adolfiina}{21.04.1870}{}
  \item \jhperson{Sanna Maria}{23.12.1872}{}
  \item \jhperson{Oscar}{12.11.1874}{}
  \item \jhperson{Amanda}{06.04.1877}{}
\end{jhchildren}
Johan \textdied 12.01.1901  --  Anna \textdied 16.01.1902



\jhhouse{Nyberg}{5-152}{Grötas}{11}{22}

\jhoccupant{Nyberg}{Sven}{1996 -}
Sven Uno Matias, \textborn 11.09.1970 i Jeppo, köpte bostadshuset med tomt och byggnader 1996 av fadern Uno. Sven rev bostadshuset 2015 och uppförde nytt på samma tomt. Sven är ogift, jobbar som arbetsledare på Finskas Br, Ab.

\jhhousepic{155-05714.jpg}{}

\jhoccupant{Nyberg}{Uno \& Leena}{1966-1996}
Uno Elis, \textborn 25.08.1943 i Jeppo.  Vid arvsskiftet 1966 tog han över lägenheten med byggnader. Han gifte sig 1969 med Leena Järvenpää, \textborn 26.12.1948 i Alahärmä.  Uno, tillsammans med brodern Leo, som gifte sig med Hjördis Häggman från Vörå och flyttade 1965 till Jakobstad, skötte om  dödsboet under tiden 1959-1966. De rev bort den gamla uthuslängan med stall och bod och byggde en ny.  Uno  avslutade boskapsuppfödningen i början på 1970-talet och satsade på svin.  Han byggde om fähuset och renoverade bostadshuset. Problem med hälsovårdsmyndigheterna gjorde att svinuppfödningen avslutades. År 1993 avslutades jordbruket, allt såldes förutom bostaden och ekonomiebyggnaderna med omkringliggande jordområde. Uno var många år i tillfälligt arbete på Keppo pälsdjursfarm samt åt Johan Slangar på svinfarm. Förutom jordbruksarbete var makan Leena postutdelare 32 år i Jeppo och senare i Nykarleby stad, tills hon blev pensionerad. Uno och Leena flyttade till annat hus (nr 24) 1998, som de köpte 2005.
  \begin{jhchildren}
    \item \jhperson{Kaj Jaakko Emil}{13.07.1969, arb. på KWH-Mirka}{}
    \item \jhperson{\jhbold{Sven} Uno Matias}{11.09.1970, arb. på Finskas Br. Ab}{}
    \item \jhperson{Nina Sisko Maria}{04.08.1976, gift, Ek Pensala}{}
  \end{jhchildren}


\jhoccupant{Nyberg}{Emil \& Aino, Leander}{1935-1959}
Emil Nyberg, \textborn 11.01.1892 i Jeppo och brodern Leander, \textborn 03.02.1884 i Jeppo, övertog hemmanet av fadern Carl år 1935. Arealen var 58 ha, varav 18 ha odlad.  Emil gifte sig 1939, med änkan Aino Adolfiina Sipponen, \textborn 26.01.1898 i Alahärmä. Lägenheten var en typisk jordbrukslägenhet med spannmålsodling , boskap, hästar, får och grisar.
\begin{jhchildren}
  \item \jhperson{Anders Leo}{13.12.1940, gift, bor i Jakobstad}{}
  \item \jhperson{\jhbold{Uno} Elis}{25.08.1943, flyttade till nr 24}{}
\end{jhchildren}
Makan Ainos barn från första äktenskapet, Holkkola:
\begin{jhchildren}
  \item \jhperson{Heino Jakob}{27.03.1928}{16.06.2008}, begravd i Veberöd, Skåne
  \item \jhperson{Tauno Esaias}{14.02.1930}{30.10.2007}, '' i Staffanstorp, Skåne
  \item \jhperson{Mauno Johannes}{12.11.1932}{}, i Karlskoga
\end{jhchildren}

Aino dog 26.08.1954. Efter hennes död inlöste maken Emil med sönerna Leo och Uno år 1955 arvslotterna av Ainos barn från  första äktenskapet.
Emil \textdied 26.08.1959, Leander \textdied 19.07.1962, ogift.


\jhoccupant{Nyberg}{Carl \&  Anna-Sofia}{1881-1935}
Carl Mattson Sandberg, \textborn 08.02.1854 på Finskas, tog namnet Nyberg. Efter några års arbetsförtjänst i Amerika , köpte han vid hemkomsten år 1881, tillsammans med systern Caisa och hennes man Anders Abrahamsson Ekoluoma, en lägenhet på Grötas 21/128 mantal av Grötas skattehemman nr. 5 av Jonathan von Essen. Det finns inga uppgifter hur man disponerade bostadshuset och övriga byggnader, troligtvis bodde Anders och Caisa i lägenheten (nr 104) och Carl byggde nytt. Carls syster Caisa dog 1889, följande år sålde maken Anders Ekoluoma deras andel av lägenheten till Gustav Hansson Ryss och flyttade själv med familj till Alahärmä. Samuel Roos köpte år 1895 Gustav Hansson Ryss' del (se nr 104).

År 1883 gifte sig Carl med  Anna-Lovisa Simonsdt. från Pedersöre, hon dog 1888. Carl ingick nytt äktenskap med Anna-Sofia Ryss, \textborn 30.10.1857 i Pedersöre. De var bönder. Enligt hörsägen lär Carl ha varit en bra berättare och lärare för Grötas' ungkarlar. Av familjens barn är det äldsta fött i första giftet, resten i andra giftet.
\begin{jhchildren}
  \item \jhperson{\jhbold{Leander}}{03.02.1884}{19.07.1962}, ogift
  \item \jhperson{\jhbold{Emil}}{11.01.1892}{26.08.1959}
  \item \jhperson{Johannes}{25.12.1894}{1895}
  \item \jhperson{Anna-Lovisa}{21.11.1896}{1983}, ogift, dövstum
\end{jhchildren}
Carl \textdied 13.02.1938  --  Anna-Sofia \textdied 23.08.1936


\jhoccupant{von Essen}{Jonathan}{1879-1881}
Se Grötas nr 104 för mera information.



\jhhouse{Sundholm}{5:9}{Grötas}{11}{24}

\jhoccupant{Nyberg}{Uno \& Leena}{2005 -}
Nuvarande ägare Uno Nyberg och makan Leena köpte lägenheten 2005 av Hans Nybyggar. De hade bott på hyra i gården från år 1998, då de flyttade från hus nr 22.  Efter köpet har Uno reparerart byggnaderna.


\jhoccupant{Nybyggar}{Hans \& Anette}{1987-2005}
Hans Nybygggar, \textborn 08.02.1966 i Jeppo, sambo med Anette Margareta Johansson, \textborn 06.09.1964 i Göteborg,  köpte lägenheten av Risto Yliaho, som köpte nr 34.  Hans har varit truckförare på Jeppo Potatis och lastbilschaufför åt olika transportbolag både i Finland och Sverige. Paret har ett barn, Johan David, \textborn 13.05.1992. De separerade och Hans flyttade till Sverige, där han gifte sig 1994 och fick två barn.

\jhhousepic{157-05715.jpg}{}

Anette  är också långtradarchaufför, hon flyttade till (Jungar    ). Lägenheten på Grötas uthyrdes till Uno Nyberg 1998, och han köpte den år 2005.
Hans \textdied år 2007.


\jhoccupant{Yliaho}{Risto \& Susanne}{1986-1987}
Yliaho Risto med familj köpte lägenheten av Gunnar Söderlund år 1986, bodde i huset två år, tills de köpte Olof Dahlströms lägenhet, nr 34.


\jhoccupant{Söderlund}{Gunnar \& Rauha}{1950-1986}
Gunnar Söderlund, \textborn 17.12.1925 i Esse, köpte lägenheten av Adolfiina (Fiina) Rintalas dödsbo 1950. Han gifte sig med Rauha Maria Perikanta, \textborn 19.03.1934 i Sordavala.  Gunnar var skomakare till yrket. Han arbetade på Keppos pälsdjursfarm och senare på KWH-Mirka. Han byggde hönshus 1957, förstorade bostaden och rev den gamla uthuslängan. Åren 1964-1970 bodde de i Jakobstad, där han arbetade som gårdskarl. De återvände till Grötas 1970, stannade ett år och flyttade 1971 till Vasa, där han fortsatte  att arbeta som  gårdskarl åt ett fastighetsbolag. Nu är han pensionär och bor i Vasa.

Gunnar deltog i fortsättningskriget, blev sårad och är krigsinvalid. Som veteran blev han år 2016 tilldelad Finlands Vita Ros orden medalj. Makan Rauha dog 2001 i Vasa.

Barn: Yngve Elof, \textborn  30.06 1963 i Jeppo, bor i Vasa.

Gunnar hade hyrt ut huset under sin vistelse i Jakobstad till bl.a. Tor och Tulikki Elenius (Fors nr 98), Esko och Kaarin Kalijärvi (Silv.     ), Marja-Liisa och Mikko Erkinheimo, senare 1974-76 till Curt och Monica Elenius och 1977-85 till Ann-Christin och Lars Dahlkvist (Silvast nr 77).


\jhoccupant{Rintala}{Jaakko \& Fiina}{1940-1950}
Skräddaren Jaakko Gustavsson Rintala, \textborn 07.01.1887 i Nurmo, gift med Adolfiina (Fiina) Kaarna, \textborn 01.07.1887 i Lappo, tidigare gift Halme. Jaakko och Fiina  köpte lägenheten 1936 av Viktor Sundholm men flyttade in först 1940. Jaakko var skräddare, dog 13.02.1943.  Fiina hade 3 barn från sitt första äktenskap. Efter försäljningen  av lägenheten 1950 flyttade Fiina till (Silv.     ).
\begin{jhchildren}
  \item \jhperson{Wellamo}{10.06.1912}{}, frisörska  på Silvast, emigrerade till Sverige
  \item \jhperson{Osmo}{23.03.1916}{}, emigrerade till Australien
  \item \jhperson{Oiva Jaakko}{22.09.1918}{}, gift (Silvast      )
\end{jhchildren}


\jhoccupant{Sundholm}{Viktor \& Ellen}{1926-1940}
Viktor Johannes Sundholm, \textborn 07.09.1903 i Nykarleby, gifte sig 1931 med Ellen Anna Alfhild Westerlund, \textborn 11.01.1906 på Silvast. De köpte lägenheten Sundholm 5:9 med fastigheter utgörande 0,0009 mantal av Grötas skattehemman 5, den 01.03.1926 av Viktors mormor, änkan Anna-Lovisa Jakobsson, för 10.000 mark.

Viktor var diverse arbetare. Ellen dog 18.02.1940 och deras son Magnus, \textborn 09.03.1938, dog samma dag. Viktor flyttade till Gamlakarleby, där han dog.


\jhoccupant{Jakobsson}{Erik \& Lovisa}{- 1926}
Backstugusittare Erik Johan Jakobsson, \textborn 01.11.1851 i Jeppo, gift med Anna-Lovisa Andersdotter, \textborn 16.06.1849 i Jeppo. År 1921 löste änkan Lovisa in området av Anders Fagerholm och Anselm Grötas.
Maken Erik \textdied 15.08.1916  --  Anna Lovisa \textdied 08.05.1926.
\begin{jhchildren}
  \item \jhperson{Johannes}{31.12.1877}{}
  \item \jhperson{Maria-Sofia}{01.05.1879}{1965 på Finskas}, gift med J.Andersson
  \item \jhperson{Emil}{12.01.1882}{}
  \item \jhperson{Jakob}{28.04.1886}{}
  \item \jhperson{Anna-Lovisa}{28.05.1891}{}
\end{jhchildren}
Enligt  mantalslängden ägde Gustav Jakobsson Böös torpet 1870-1906 och i mantalslängden 1910 finns banvakten Anders Grötas som ägare av torpet enligt köpebrev 1906. Torparna har dock haft kontrakt med ägarna av bakstuguområdet.



\jhhouse{Farmen}{.893-408-15-2}{Grötas}{11}{26}

\jhoccupant{Aho}{Martti \/ Hilkka}{1979 -}
Ägare och boende Martti Aho, \textborn 1951 i Jeppo, gift 1971 med Hilkka Linnanmäki, \textborn 1952 i Alajärvi. De köpte bostadstomten med två bostadshus och ekonomiebyggnader (nr 126a,126b) av Artur och Siviä Björkqvist år 1979 och byggde nytt bostadshus år 1981. Ekonomiebyggnaderna byggdes om, de sysslade några år med kalkonuppfödning. Martti har jobbat som lastbilschaufför och Hilkka är utbildad kokerska och har arbetat bl.a. som hemhjälpare.

\jhhousepic{160-05716.jpg}{}

\begin{jhchildren}
  \item \jhperson{Veli Pekka}{27.01.1973,  chaufför/elektriker}{}
  \item \jhperson{Johanna Marjaana}{02.04.1974, hälsovårdare}{}
  \item \jhperson{Marika Pauliina}{04.07.1975,  närvårdare}{}
  \item \jhperson{Minna Hannele}{06.10.1976,  sömmerska}{}
  \item \jhperson{Janne Antero}{23.01.1978,  byggnadsarbetare}{}
  \item \jhperson{Elina Kristiina}{28.06.1979,  konditor}{}
  \item \jhperson{Elisa Henna Maaria}{15.03.1981,  laborant}{}
  \item \jhperson{Katri Marjut Inkeri}{29.08.1982,  merkonom}{}
  \item \jhperson{Kaisa Annika}{05.12.1983,  merkonom}{}
  \item \jhperson{Mikko Petteri}{29.09.1985,  byggnadsarbetare}{}
  \item \jhperson{Jukka Tapio Mikael}{11.12.1987,  bilmekaniker}{}
  \item \jhperson{Vesa Jaakko Juhana}{03.05.1989,  byggnadsarbetare}{}
  \item \jhperson{Antti Valtteri}{01.11.1990,  svetsare}{}
  \item \jhperson{Joonas Kristian Samuel}{03.01.1994,  studerande}{}
  \item \jhperson{Jussi Markus Henrikki}{10.10.1996,  studerande}{}
\end{jhchildren}
