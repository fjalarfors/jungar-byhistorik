\jhchapter{Romar, hemman Nr 2}

Romar – varifrån kommer det namnet? Som så ofta tvistar de lärda. Runar Nyholm har i en artikel i Vasabladet 24.6.1976 gett en spännande redogörelse över de lärdas åsikter. Enligt P.E. Ohls är namnet bildat av ordet romar= tala högröstat och kommer från ljudet i ån i närheten. Besläktade ord är åumar=ekar el. en ko som råmar. Jämförande ord lär finnas i Rombaksbotn i Norge och Hundalsälven, där ljudet är likt ett hundskall. Ralf Karsten menar å sin sida att namnet kan komma från personnamnet Hrodmarr. Han menar också att det namn, som i vardagslag länge använts om Romar, nämligen ”Fjössmans”, varit ett okvädningsord för en här boende tassig gubbe. Ortsborna anser dock att varför skulle ett helt hemmansnummer, ett av 1600-talets största, få namn efter en fjollig gubbe. Här bodde istället på 1600-talet en fjärdingsman och från detta har vi böjningen fjässmans som blivit fjössmans.

Romar utgjorde ursprungligen 1 mantal, men utökades på 1640-talet till 1 ½ mtl. Hemmanet ägdes 1551-1553 av Henrik Jönsson,  1556-1560 av Olof Henriksson, 1562-1580 av Henrik Jönsson, 1582-1603 av Simon Olsson, 1603-1622 av Hans Larsson, 1624-1629 av Hans Hansson, 1633-1673 av Hans Mikkelsson som är son till Mikkel Mattsson på Slangar där han 1630 i RL är antagen som fjärdingsman. Han är också antagen till fjärdingsman på Romar, 1675-1681 av Matts Hansson, fjärdingsman, 1683-1705 av mågen  Jakob Knutsson, 1709-1713 av Matts Hansson, måg till Mikkel Mattsson. Ett hemman utan namn uppgår i Romar 1640 omfattande ½ mtl. Med innehavare 1567-1580 av änkan Elin, 1592-1623 av Mårten Hindersson, 1624-1631 av Hans Markusson och 1634-1640 av Markus Jöransson.

Den av Matthias Wörman år 1740 uppgjorda kartläggningen upptar på sin karta endast en gård på Romar hemman, placerad mellan den smala landsvägen och ån, ganska exakt där Ralf och Svea Romars hus nr 10 står i dag.  1783 har hemmanet delats i 4 lika stora delar. I slutet av den svenska tiden i 1807 uppgjord personbok, är Romar det mest folkrika hemmanet i det som senare  skulle bli Jungar by.  Romar hade då 47 personer. Tack vare samhällsutvecklingen har på Romar hemmans mark bibehållits en god bosättning då bl.a. områden som hör till hemmanet längs Nylandsvägen tagits i anspråk för byggande. Antalet brukningsenheter på hemmanet har dock stadigt minskat under efterkrigstiden och idag finns endast 3 aktivt kvar. De tidigare lägenheternas jord är utarrenderad eller såld.

Skog hemman omfattas av kartorna ..... och .....


KARTA KOMMER HIT


\jhsubsection{Lägenheter på Romar}

\jhhouse{Åbacka}{2:91}{Romar}{2}{3, 3a}

\jhoccupant{Romar}{Erik}{1996 -}
Erik, \textborn 14.06.1951, övertog lägenheten år 1996. Han har fungerat som truckförare inom Jeppo Potatis AB nästan ända sedan företagets start år 1976. Innan dess arbetade han på Oy Mirka Ab mellan åren 1972-76. Han lever ogift.

Lägenheten omfattar förutom tomtmarken också 0,7 ha åker och 4,3 ha skog.
