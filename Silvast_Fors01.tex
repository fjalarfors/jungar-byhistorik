\jhchapter{Silvast \& Fors, hemman Nr 4 och 3}

\jhbold{Silvast, hemman Nr 4}  (Storsilvast)

Idag skrivs hemmansnamnet Silvast, men skrevs tidigare Silfvast. Enligt samma källor som tidigare torde ett så pass sällsynt namn som Silvester ligga till grund för hemmansnamnet. Det finns nämligen i det svenska riksarkivet två medeltida handlingar avfattade på latin efter heliga Birgittas död 1373. Urkunderna är av hennes dotter Katarinas hand, skrivna inför moderns kanonisering. Av dessa framgår att kristna människor levde redan då i våra nejder. I den andra av dessa urkunder nämns om ett under om Christina, Michaels hustru boende i ``Lappabi'' i Pedersöre socken. Hon led av fallandesjuka, men hade blivit botad sedan hon anropat den heliga Birgitta om hjälp. Ytterligare nämns i urkunden att ``Siluester nomine, de predicto opido'' d.v.s Silvester från samma by som återfick sin syn sedan han åkallat den heliga Birgitta. Båda urkunderna finns översatta till svenska i Karleby sockens historia.

Christina betyder ursprungligen Den kristna. Michael bär ärkeängeln Mikaels namn och Silvester eller Sylvester är ett helgonnamn efter påven Sylvester I, som avled år 335. De tre här nämnda personerna hade alla kristna namn och då deras föräldrar givit dem dessa namn, borde också de varit kristna. En fornsvensk ombildning av namnet Sylvester blir till Silfvast. Lappabi är detsamma som Lappo by kring nuvarande Nykarleby stad. Lappaby torde ha sträckt sig upp längs älven till Jeppo och Silvast. Ett mycket vägande skäl att ge  hemmanet detta namn förelåg, då Silvester upplevt ett under här. Bosättningen vid Silfvast är bevisligen gammal. Här förgrenar sig Nykarleby älv i två älvfåror, som omsluter en holme och här finns stora bäckmynningar som kunnat tjäna som handelsplatser. Om detta vittnar fynd av ickemagnetiska (kopparmantlat bly) vikter av mycket gammalt slag i närhet av Silvastbäckens mynning. De strida forsarna uppströms bromsade, men gav möjlighet till gott fiske. Att man tidigt förstod att utnyttja vattenkraften framkommer då Madz Mazon i Sylwastholme 1572 omnämns erlägga kvarntull.

Det område som nämnts som Silfvast var ursprungligen uppdelat i ett Storsilvast och ett Lillsilfvast (nuv. Fors) hemman. Före 1831 hade Storsilfvast hemmansnr. 26, men fick därefter nr 4. Hemmanets marker går som en hästsko kring en del av Lillsilfvast marker och någon orsak till detta sakernas tillstånd står inte att finna. Då hemmanens nummer normalt stiger i nummerföljd uppströms räknat från Skog, är det förbryllande att nummer 4 kommer tidigare än nr 3. I vilket fall gränsar Storsilfvast i norr till Romar och gränsen i söder mot Fors går strax söder om gård nr 19, som nu ägs av Pontus Back.

Storsilvast omfattade ursprungligen 1 mantal och innehades 1551-1573 av Erik Mattsson, 1577-1583 av Anders Persson, 1586-1600 av Mickel Mattsson och 1602 av änkan Margareta, 1603-1627 av Olof Mickelsson, 1628-1648 av mågen Matts Mickelsson, 1653-1675 av Olof Mattsson, 1676-1708 av sonen Matts Olofsson (medåbor 1676-1699 var brodern Hans Olofsson med hu Maria Hindersdr. och svågern Mårten Mattsson med hu.Brita Olofsdr.1676), 1709-1719 av Hans Olofssons son Anders Hansson med hu. Anna.,1723 – av bröderna Johan Hansson med hu Brita och Anders Hansson med hu. Anna.  I Nykarleby stads mantalslängd av 1783 framgår att hemmanet delas av Gustaf, Johan	och Anders Mattsson med 17/48 respektive 17/96 + 17/96 mtl. Storsilvast hade år 1807 31 personer antecknade.

På Storsilfvast stod också soldattorpet nr 138 för roten; Måtare nr 28, Storsilfvast nr 26 och Rautzinkåski nr 30. Torpet har gått under namn som Sparfwas, Sparfbacka och slutligen Måtars. Det torde ha stått på bäckliden där f.d. biograf Silva står i dag. Eventuellt har det under tid flyttats något. Vid syneförrättning 1752 konstaterade stugans, fähusets och ladans undergärder vara förruttnade och nedsjunkna i jorden. Samtliga tak var förfallna av ålder och läckte. Fönstren var blyfallna, sädesboden saknade lårar o.s.v. Rotesällarna lovade sätta allt i stånd. Åkern befanns vara i fullt stånd, likaså ängarna. Gärdesgårdar och diken inhöstade dock kritik. Vid ny syneförrättning hade tillståndet på torpet kraftigt förbättrats. Torpet lades, som alla andra av den typen, ner när armén upplöstes efter 1810 och Finland blev storfurstendöme under den ryska tsaren.

Styckningen av hemmanet har fortsatt och i medlet av 1800-talet utgjordes det av 4 enheter om 17/96 mtl., 1890 hade enheterna ökat till 6 st, år 1900 var antalet 11 st och år 1910 13 st. Denna process som fortsatt, har sin grund i järnvägens ankomst, med stationen placerad på Silvast hemman år 1885. Fr.o.m. nu flyttade tyngdpunkten från Stenbacken till det som nu kallas Silvast.

Silvast, bestående av både hemmanen Silvast och Fors, blev efter järnvägens ankomst ett nytt centrum för Jeppo socken. Järnvägen kan också sägas ha tvingat fram en snabbare lösning på socknens tvärförbindelser genom byggandet av nya broar vid Kiitola och Fors åren 1910 och 1911. Till Silvast samlades småningom skolor, butiker, kaféer, bageri, banker, såg och ångsåg, kvarn, mejeri, äggandelslag, skogsandelslag, gästgiverier, ungdomslokal, sportplan, telefoncentral, post, biograf, handelsmagasin vid järnvägen, kommunala funktioner och framför allt bosättning. Som mest har Silvast haft en befolkning på över 600 personer. Hit räknas också den bosättning som växt upp på Lövbacken, huvudsakligen efter krigen 1939-45. Efter att Jeppo år 1975 slogs samman med Nykarleby stad, Munsala och Nyk.by landskommun, har en stor del av den dynamiska verksamheten i  Silvast avtagit. Det är trist att konstatera att inga tåg längre stannar, inga banker finns kvar, inga butiker, inget bageri, ingen post. Kvarn, såg och mejeri står stilla. Café Funkis håller öppet. Kvar på Silvast hemman finns i dag endast två aktiva brukare. Endel kallar det utveckling.


\jhbold{Fors, hemman Nr 3}  (Lillsilfvast)

Fors hemman har en historia, som i långa stycken följer Silvast-namnets historia. Delar av Fors hemman omfamnas geografiskt av Silvast hemman, men huvuddelen av hemmanet finns söder om en linje dragen från Paul Broo's gård (nr 128) rakt över ån i riktning strax söder om ungdomslokalen och därefter snett över järnvägen längs utfallsdiket i riktning mot Göran och Gunborg Stenvalls gård på Lövbacken. Om den norra gränsen finns vid ungdomslokalen, finns den södra gränsen i omedelbar närhet av vägkorsningen till Oravais. Tennisplanen, konditionshallen och numera biblioteket står ännu på Fors hemman, medan själva vägen till Oravais finns på Grötas hemman. Under de nämnda husen och tennisplanen fanns en gång i tiden de s.k. ``Kliftens lindor'', d.v.s. åkern till det soldattorp 139, som upprätthölls av hemmanen med den tidens numrering: nr 10 Lafwast, nr 25 Lillsilfvast, nr 1 Grötas och nr 17 Rämäcki. Namnet på ``lindorna'' hade uppstått av att 5 generationer rotesoldater med detta namn bott på torpet fram till 1789. Själva torpet stod på Grötas hemmans mark, sannolikt där Westins gård nu står på åstranden (se närmare Grötas).

Från 1831 ändrade hemmanet Lillsilfvast namn till \jhbold{Fors} och hemmansnummern ändrade samtidigt från nr 25  till \jhbold{nr 3}. En annan benämning som länge levat och ännu lever kvar kring detta hemman är ``Krymboas'', vilket skall förstås som en dialektal vridning  av ``kronobyboas''. Strax före Stora ofreden hade nämligen en Jakob Jakobsson från Kronoby köpt den ena delen och löst in den andra delen av hemmanet, så att han år 1699 äger hela hemmanet.

Av händelsernas gång har detta hemman i senare tid kommit att bli en plats för, förutom kvarn och såg, de flesta av de kommunala funktioner som hör ett modernt samhälle till. Hit samlades fattiga och sjuka i den s.k. Såggården, hit styrdes den kommunala hälsovården i början av 1920-talet, här byggdes Jeppo kommuns centrumskola, kommunalgård, det första radhuset för pensionärer, brandstation, barnträdgård och konditionshall.

Lillsilfvast (Fors) omfattar 1 mtl. och har 1592-1618 ägts av Matts Nilsson, 1619-1624 av Nils Mattsson, 1625 - ca 1657 av Lars Nilsson, 1664-1667 och av änkan till 1669, 1675-1681 av Tomas Hansson och av änkan Anna Mattsdr. 1683, Jakob Jakobsson 1686-1696 med hu. Margareta Mansdotter. År 1694 är hemmanet kluvet och 3/8 mtl ägs av Johan Tomasson.

Här kommer Jakob Jakobsson från Kronoby in och äger 1699 med hu. Maria hela hemmanet. 1709-1712 av mågen Matts, 1713 av änkan. Medåbo är nu svågern Hans Jakobsson med hu. Susanna Mattsdr. 1719 -  och dennes bror Daniel med hu. Margareta Tomasdr. 1713  -. Hemmanet går sedan in i stora ofreden. År 1807 är 34 personer antecknade.

Att hemmanet fick namnet Fors, som senare blivit till ett efternamn hos dem som bebott detta, är ganska naturligt. Den stora stenansamling som delat upp strömfåran i olika riktningar och bildat små holmar, har gjort att den naturliga forsen tidigt kunnat utnyttjas för att driva vattendrivna kvarnar. När kronan på 1550-talet började driva in ``kvarntull'' och man i Jeppo noterar 6 kvarnar, fanns säkert en av dem här. 1564 antecknas att det i Jeppo fanns 2 stora och 7 små kvarnar. Man kan anta att en av de stora fanns här och lika sannolikt kan man anta att det var en enkel ``skvaltkvarn'' placerad på åns västra sida nedanför Paul Broo's boplats. Där har åtminstone i sentid hittats en kvarnsten och bitar av s.k. lagerstenar. Efter att ännu på 1950-talet hyst 6 brukningsenheter, återstår nu endast en (1).


\jhsubsection{Lägenheter på Silvast och Fors}

Följande	stomlägenheter har 1915 noterats i Jordregistret efter slutförd	lantmäteriförrättning. Ägarna i mantalslängden 1912 stämmer överens 	med Jordregistret 1915.	Detta exempel gäller Silvast hemman nr 4. En del lägenhetsbeskrivningar återges ändå enbart med registernummer, som dagens ägare presenterat.
\begin{center}
  \begin{tabular}{l l l l}
    R:nr & Namn & Mantal & Ägare \\ \hline
    4:6 & Bäckstrand & 0,0495 & Modén Johan \\
    4:7 & Lindström & 0,1771 & Lindström Tomasson Simon \\
    4:8 & Norrgård & 0,0664 & Högbacka Isaksson Isak \\
    4:9 & Nygård & 0,0391 & Stenbacka Isak \\
    4:10 & Silvast & 0,1328 & Gustafsson Isak \\
    4:11 & Fors & 0,0166 & Fors Erik \\
    4:12 & Dahlbo & 0,0332 & Smulter Viktor \\
    4:13 & Orrholm & 0,0166 & Jakobsson Jakob \\
    4:14 & Lindfors & 0,0443 & Lindfors Jakobsson Isak \\
    4:15 & Nyhagen & 0,0221 & Huhtala Jakob \\
    4:16 & Bäck & 0,0222 & Lindfors Jakobsson Isak \\
    4:17 & Strand & 0,0885 & Strand Isaksson Johannes \\ \hline
    \jhbold{4} & \jhbold{Silvast} & 0,7083 & Silvast skattehemman totalt \\
  \end{tabular}
\end{center}

Efter att Jeppo Järnvägsstation (``stomlgh'' 4:5)år 1885 byggdes på 5,21 ha av Silvast hemman, ökade befolkningen snabbt och nya bostäder behövdes. Från ovannämnda tolv stomlägenheter bildades parceller, som efter klyvningsförrättning via styckning eller avsöndring bildade självständiga fastigheter, de flesta	bebyggda på områden mellan järnvägen och älven. Dylikt förfarande gällde också för Fors hemman nr 3, men presenteras inte här.

Silvast och Fors hemman är inflätade i varandra; de presenteras nedan på kartor med raka utsnitt, nr \jhbold{4 - 9}.


KARTBLAD nr 4 placeras hit --------------->>>


\jhhouse{Åbrinken}{4:146}{Silvast}{4}{18}

\jhoccupant{Perälä}{Teuvo \& Maija-Liisa}{1989 -}

Teuvo Markku Benjam, \textborn 1952 i Laihela, gifte sig 1978 med Maija-Liisa Saunala, från Ilmajoki, \textborn 1959. År 1989 köpte de fastigheten av Alf Ludèn. Makarna skiljde sig 1993 och Maija-Liisa flyttade bort tillsammans med barnen. Teuvo har därefter bott ensam på fastigheten. Teuvo har sedan 1989 arbetat på Mirka vid en s.k. Flex-maskin som mjukgör produkten inför användningen. Maija-Liisa arbetade en tid på Mirka, men fortsatte därefter som familjedagvårdare till dess hon flyttade från orten.

\jhhousepic{014-05543.jpg}{}

\begin{jhchildren}
  \item \jhperson{Marko}{1978}{}
  \item \jhperson{Tommi}{1979}{}
  \item \jhperson{Juha}{1983}{}
  \item \jhperson{Minna-Maria}{1989}{}
\end{jhchildren}


\jhoccupant{Ludén}{Alf \& Anja}{1956-1989}
Alfred( Alf) Ingvald Ludén från Sideby, \textborn 01.01.1928 gifte sig med Anja Melita Laxén, \textborn 06.05.1930. Efter att som nygifta bott på flera platser inköptes tomten 1956 och huset timrades upp med hyvlat stående timmer samma år. Alf inköpte en lastbil och började med grustransporter, en verksamhet som fortsatte under familjens tid i Jeppo. I sept. 1989 flyttade Alf och Anja till Nykarleby.
\begin{jhchildren}
  \item \jhperson{Alf Roland}{07.05.1953}{}
  \item \jhperson{Roger Mikael}{09.10.1956}{}
\end{jhchildren}

Alf \textdied 27.01.2000


\jhhouse{Norrback}{4:156}{Silvast}{4}{19, 19a}

\jhoccupant{Back \& Libäck}{Pontus \& Johanna}{2007 -}

Pontus Per-Viking Back, \textborn 06.07.1983 i Jeppo är sambo med Rosanna Libäck, \textborn 02.01.1986 i Maxmo.
\begin{jhchildren}
  \item \jhperson{Fanny}{28.05.2012}{}
  \item \jhperson{Maya}{03.01.2015}{}
\end{jhchildren}

\jhhousepic{015-05540.jpg}{}


Pontus arbetar på Mirka slippappersfabrik i Jeppo sedan 2004. Han köpte fastigheten den 05.04.2007 av Outi Pellikka. Efter köpet har han grundligt renoverat husets alla våningar och förnyat värmesystemet. Vid renoveringen av huset hittades tidningen Hufvudstadsbladet av	den 27 jan 1928 under golvplankorna. Huset torde ändå inte vara	uppfört vid den tidpunkten. Se nedan. I febr. 2017 förnyades vattentaket. Det tidigare uthuset har rivits och ersatts 2012 med ett modernt garage.

Rosanna har efter studenten utbildat sig till bagare/konditor och	har arbete på Sandbergs Bageri i Nykarleby.


\jhoccupant{Pellikka}{Outi}{1991-2007}

Outi Pellikka köpte fastigheten av Vasa Andelsbank den 9.11.1991. Hon har tidvis bott i huset fram till försäljningen 2007.


\jhoccupant{Vasa Andelsbank}{-}{1990-1991}

Under endast ett år stod Vasa Andelsbank som ägare.


\jhoccupant{Saunala}{Seppo \& Wivi-Ann}{1986-1990}

Seppo och Wivi-Ann Saunala köpte fastigheten av Agda Norrbacks sterbhus. De innehade fastigheten bara ett fåtal år.


\jhoccupant{Norrback}{Gunnar \& Agda}{1956-1986}

Gunnar och Agda Norrback köpte huset den 25.07.1956 av Fredrik och Ellen Thors sterbhus. Innan dess hade huset hyrts av bl.a Rafael Kronlöf som anlänt med familj från Vörå 21 juli 1949 och bosatt sig i	nedre våningen samt familjen Runar Holmlund, som anlänt från Vasa 16 sept. samma år i övre våningen. Både Kronlöf och Holmlund var järnvägstjänstemän.

Gunnar Norrback, \textborn 28.19.1914 i Petalax, gifte sig 18.08.1940 med Agda Maria Nordman, \textborn 15.02.1916 i Petalax.
\begin{jhchildren}
  \item \jhperson{Ulla Kristina}{03.04.1941}{}
  \item \jhperson{Karl Oskar Edvin}{13.11.1943}{}
\end{jhchildren}

Gunnar har arbetat som vägmästare för det statsägda Väg och Vatten fram till sin pensionering. I ca 10 års tid arbetade han som arbetsledare vid garveriet på Kiitola, men efter att problemen börjat hopa sig efter ledningens misslyckade köp av dåliga fårskinn från Nya Zeeland, köpte han tillsammans med Wilhelm Åkermark det nyligen nedlagda andelsmejeriet vid Jungar år 1952 och startade bl.a. skinnberedning och senare tillverkning av plastprodukter; ett företag som blev fröet till det företag som 1955, efter mejeriets brand på Jungar, flyttade till Nykarleby	under namnet  Ab Prevex Oy. Andra delägare i det bolaget var Sven	Nyman, Manne Bergman och Wilhelm Åkermark.

Agda har varit lärarinna i flera av Jeppos skolor: Jungar högre folkskola år 1952-56, Måtar lägre folkskola år 1956-66, Centralskolan år 1966-74.

Agda \textdied 24.03.1984. Gunnar flyttade till Nykarleby i september 1985.


\jhoccupant{Thors}{Fredrik \& Ellen}{ca 1920-1956}

På sin ägolägenhet Mellangård Rno 4:99 byggde makarna Thors sannolikt år 1934 den husbyggnad som fortfarande står på tomten. Enligt uppgift skall huset vara timrat av den f.d. Lilla kvarnen vid	Mjölnarsforsen-Finskas. Kvarnen förstördes vid en våldsam islossning 1919. Timran numrerades och plockades ner och uppfördes som	en av tre tjänstebostäder på ``Kukkobacken''. När ullspinneriet vid	Kiitola gick i konkurs 1933 och såldes till kommunen den 5 febr. 1934, köpte Fredrik Thors huset, som plockades ner på nytt och flyttades till Silvast där det byggdes upp som ett pensionärshus på denna tomt.

Redan ett par år tidigare hade en uthuslänga uppförts med gaveln	mot landsvägen. Under krigstiden bodde makarnas dotter Elsa Örndahl, \textborn 14.10.1903, tillsammans med sina barn Gudrun, \textborn 23.02.1933 och Bjarne, \textborn 27.01.1935, i huset. Elsa hade gift sig 24.06.1932, men hennes man hade stupat i vinterkriget den 13.02.1940.

Lärarparet Thors utförde en sällsynt lång lärargärning vid Jungar	folkskola. Fredrik Thors var lärare vid skolan från 1899 till 1940 och hans 	maka Ellen Thors från år 1900 till 1940! De ägde under denna tid flera markområden i Silvast, vilka efterhand avyttrades.

Fredrik Thors, \textborn 18.07.1877 - \textdied	06.12.1949  --  Ellen Thors, \textborn 03.06.1877 - \textdied 13.12.1961



\jhhouse{Åbacken}{3:49}{Fors}{4}{43, 43a}


\jhoccupant{Silfvast}{Ronny}{2009 -}
Ronny Silfvast, \textborn 16.04.1974 i Jeppo. Ronny är utbildad restaurangkock och har vidareutbildat sig till metallarbetare/svetsare och är också Team leader ansvarig på Kaiser Eur Mark i Nykarleby. Han köpte lägenheten 2009 av arvingarna till Lennart och Gemima Gustafsson.

\jhhousepic{016-05884.jpg}{}


\jhoccupant{Gustafsson}{Lennart \& Gemima}{1945-2009}
Lennart Gustafsson, \textborn 24.09.1913, gift 28.08.1932 med Gemima Alfrida Fors, \textborn 08.07.1912.
\begin{jhchildren}
  \item \jhperson{Karin}{08.01.1934}{}
  \item \jhperson{Denice}{29.03.1937}{}
  \item \jhperson{Stefan}{23.09.1938}{}
  \item \jhperson{Jan-Erik}{18.08.1945}{}
  \item \jhperson{Lise-Maj}{28.10.1950}{}
\end{jhchildren}

Vid arvskiftet efter Emil och Ida Gustava Fors den 31.12.1945 erhöll Gemima tomtmarken tillhörande Strand lägenhet 3:15, varefter byggandet av bostadshuset påbörjades. Med och grävde ur källaren var Oskar och Valfrid Blink. Jorden kunde bekvämt tippas i åslänten.

Lennart ägde tillsammans med Lennart Elenius en lastbil, som vid vinterkrigets utbrott rekvirerades av staten. Han fungerade under krigen som chaufför, dels med den egna bilen och senare som chaufför åt officerare och som ambulanschaufför. Efter kriget fick han anställning som chaufför vid arméns vapendepå i Karleby. Under åren fram till 1963 veckopendlade han hem till Jeppo, men fr.o.m. detta år flyttade Lennart och Gemima till Karleby tillsammans med yngsta dottern. Efter pensioneringen 1973 flyttade Lennart och Gemima tillbaka till sitt hem i Jeppo.

Gemima utförde under tiden i Jeppo sy- och vävarbeten förutom att hon skötte om familjen och höll 2 kor i fähuset.

Gemima \textdied 06.06.1999  --  Lennart \textdied 07.01.2010



\jhhouse{Bäckenbacka}{3:50}{Fors}{4}{44, 44a} samt \jhbold{Trekanten} R:nr 3:129


\jhoccupant{Norrback}{Bernt \& Josefin Häggblom}{2007}

Den 08.05.2007 köpte Bernt Norrback fastigheten av Åke Lillas. Bernt, \textborn 1985,  sambor med Josefin Häggblom, \textborn 1988. Genast efter köpet började Bernt renovera huset. Tillsammans med Josefin genomfördes en ytterligare renovering och utbyggnad 2013.


\jhhousepic{018-05548.jpg}{}

Bernt är utbildad fordonsmekaniker från Optima. Han har arbetat vid Mirka åren 2005-2008 och var därefter tjänsteledig 2008-2010. Han hade grundat BN Motor 2005 och hyrde till en början den bilmåleriverkstad i Silvast som Göran Stenvall hade använt sig av (nr ....).

År 2009 byggde han en ny verkstad i ``Ruckus-kroken'' i Silvast och har specialiserat sej på motorrenoveringar utöver andra typer av 	reparationer.

Josefin är restaurangkock samt sjukskötare och arbetar sedan studierna blivit klara på Malmska sjukhuset.
\begin{jhchildren}
  \item \jhperson{Inez}{03.12.2014}{}
  \item \jhperson{Ivar}{08.03.2017}{}
\end{jhchildren}


\jhoccupant{Lillas}{Åke}{2006-2007}

Åke Lillas, \textborn 28.05.1941, övertog fastigheten 2006 sedan mor Elna avlidit. Åren 1984-2006 hade ägandet handhafts av sterbhuset efter	att maken/fadern avlidit 1984.

Åke är född på lägenheten, men har inte bott på denna sedan han	flyttat bort för studierna. Han utbildade sej till lärare och har verkat	på bl.a. Evangeliska folkhögsskolan i Hangö åren 1965-68. Därefter	återkom familjen till Jeppo där Åke fungerade som lärare vid Centralskolan åren 1968-70. Från Jeppo flyttade familjen till Vasa där Åke 	fungerade som lärare vid Evangeliska Folkhögskolan 1970-74, varav	åren 1971-72 som t.f. rektor.

Efter denna period anställdes han som religionslärare vid det för	Vörå-Oravais-Maxmo gemensamma högstadium/gymnasium i Vörå	åren 1974-78. Under denna tid, med början 1974, studerade han	teologi och prästvigdes 18.12.1977. Därefter fungerade han som t.f.	kyrkoherde i Korsnäs åren 1978-79 och i Petalax-Bergö 1979-81 var-	efter han avlade pastoralexamen och kunde söka ordinarie tjänst.

År 1981 sökte han tjänsten som kyrkoherde i Vörå efter pensione-	rade kyrkoherden Håkan Bäck och denna tjänst innehade han till sin 	pensionering 2005. Innan dess hade familjen flyttat till Nykarleby år 2001.

Åke är gift med Ann-Lis, född Sandström, och paret har 2 söner;	Krister och Ove.


\jhoccupant{Lillas}{Valter \ Elna}{1938-2006}

Valter, \textborn 24.10.1911 på Lillas i Jeppo, förlovade sej den 13 augusti 1938 med Elna Sandström, \textborn 07.12.1916. Samma år
köpte Valter bostadstomten invid Silvastbäcken av Emil Fors och ett nytt hus började timras samma år tillsammans med brodern Selvin. Virket hade kommit från Lillas hemman. På årsdagen efter förlovningen gifte sej Valter och Elna, den 13.08.1939, och flyttade nu in i det färdiga huset. Vid denna tid arbetade Valter på Kiitola karosserifabrik och hade där kunnat tillverka fönster och dörrar till det nya hemmet. Ett litet uthus med vedlider och förråd byggdes också gårdsplanen.

År 1941, genast efter sonen Åkes födelse, flyttade familjen till Gamlakarleby i och med att karosseritillverkningen flyttade dit från Kiitola. Valter kom där att syssla med snickeriarbeten och huset hyrdes nu ut i 2 år till familjen Ragnvald Kangas.

1943 återvände familjen till Jeppo och Valter inledde snickeriverksamheten, till en början i uthuset på gården, men senare i det nybyggda snickeriet i svackan invid bäcken. I byggnadens nedre del fick bastu och tvättstuga plats förutom ``spånrummet''.

Ca 1950 revs uthuset på gården och snickeriet förlängdes med en förrådsdel. På 50-talet grävdes också en egen brunn som försåg både bostaden, bastun och snickeriet med vatten, men 1971 kopplades fastigheten till Silvast Vattenandelslag.

Tuberkulosen blev en mörk skugga i Valters liv. I tre olika perioder fick han vistas på sanatorium och 1970 blev han slutligen sjukpensionerad 59 år gammal.

Elna hjälpte ofta till i snickeriet. Under några år var hon köksa i medborgarskolan och städerska på kommunalgården. Efter Valters död levde hon ensam i gården till 2003 då hon flyttade till Florahemmet i Nykarleby.

Barn: Åke, \textborn 28.05.1941

Valter \textdied 11.12.1984  --  Elna \textdied 14.12.2006



\jhhouse{Harisloon}{3:143}{Fors}{4}{45, 45a}


\jhoccupant{Fors}{Christer \& Runa}{2003 -}

Christer Anders Alfred, \textborn 30.09.1947, gift den 31.10.1970 med Runa Gudrun Karolina Kåll, \textborn 27.08.1946 i Sundby, Pedersöre.
\begin{jhchildren}
  \item \jhperson{Patrik Anders}{1972}{}, Se Fors hemman nr 127
  \item \jhperson{Jenny Maria}{1974}{}, Spec.sköt., bor i Chicago
  \item \jhperson{Magdalena Carolina}{1983}{}, fil.mag. bibl.chef, Närpes
  \item \jhperson{Annichen Linnea Josefin}{1987}{}, barnmorska på VCS
\end{jhchildren}


\jhhousepic{017-05906.jpg}{}

Fastigheten har styckats från Strand 3:142, ursprungligen tillhörande Broända gård (se nr 97). Strand 3:142 inköptes via generationsskifte 1979 av Christers föräldrar Gustaf och Hildur Fors och kom därefter att tillhöra Timmerbacken jordbrukslägenhet (se nr 127). Det var under den tiden Christer och Runa hade sina flesta åtaganden för vilka Christer erhållit bl.a. Finlands Svenska Andelsförbunds förtjänstmedaljer i såväl brons, silver och guld. Som ``Årets Jepobåo'' utnämndes han 2010, efter tjärdalsprojektet ``Tjära Mor''.

Vid följande generationsskifte 2003, vid vilket sonen Patrik och dennes hustru Henriette övertog Timmerbacken jordbrukslägenhet, avskiljdes från Strand 3:142 denna fastighet och ombildades till	Harisloon 3:143 som tomt 1 i kvarteret 5036 och 50:e stadsdelen av	Nykarleby stad.

Bostadsbyggnaden, ritad och planerad av arkitekt Tom Skarpaas	från Oslo, Norge (svåger till Christer) började byggas i egen regi hösten 2003 och var inflyttningsklar den 14.07.2005. Garagebyggnaden (planerare som ovan) färdigställdes 2006.

Endast de två yngsta barnen har en kort tid bott på fastigheten.


\jhhouse{Hemåkern}{4:163}{Silvast}{4}{46}

Guy \& Birgitta Kronqvist, se nr 47, är sedan 2003 ägare till fastigheten, som de köpte av Ralf Stenvik.


\jhhousepic{020-05550.jpg}{Huset är f.n. obebott}


\jhoccupant{Stenvik}{Ralf \& Terttu}{1986-2003}
Ralf, \textborn 07.03.1943, gift 1967 med Terttu Pulkkinen, \textborn 30.09.1940 i Veteli.
\begin{jhchildren}
  \item \jhperson{Kristina}{09.11.1967}{}
  \item \jhperson{Pia}{03.11.1972}{}
\end{jhchildren}

Under sin uppväxt i Jeppo var Ralf en av JIF:s snabbaste sprinters och deltog i många klubbkamper med framgång. En av hans främsta konkurrenter var landslagssprintern från Jakobstad, Jorma Ehrström. Ralf torde fortsättningsvis inneha sprintrekordet på kolstybb-bana i Jeppo på 100 m, 11,2 sek.

Ralf har arbetat som bilmekaniker hela sitt yrkesverksamma liv. Terttu har varit posttjänsteman. Familjen har inte bott på denna fastighet, som är Ralf´s barndomshem, utan är bosatta i Karleby.


\jhoccupant{Stenvik}{Margit \& Tauno}{1946-1986}
Tauno, \textborn 27.02.1913 i Virdois, gift 22.06.1941 med  Margareta Lindén, \textborn 16.05.1920 på Silvast.
Barn: Ralf \textborn*07.03.1943.

Margit arbetade som barbiträde vid Jeppo-Oravais Handelslag och fungerade också som 4H-ledare i lantbruksklubben. Margit köpte tomten av sin far Anders Lindén år 1946 varefter bostadsbyggnaden uppfördes.Tauno arbetade till en början som chaufför, men blev sedermera polis i Jeppo. Det var ett uppdrag han utförde fram till pensioneringen. Han var ortens sista bypolis. Han deltog också ivrigt i  Jeppo  Idrottsförenings aktiviteter och var med och byggde sportplanen.

Margit \textdied 04.09.1976  --  Tauno \textdied 08.04.1986


\jhhouse{Älvbranten}{4:110}{Silvast}{4}{47, 47a}

\jhoccupant{Kronqvist}{Guy \& Birgitta}{1999 -}
Guy, \textborn 15.03.1972 på Levälä och Birgitta Knutar, \textborn 12.11.1964 från Terjärv.
\begin{jhchildren}
  \item \jhperson{Linda}{07.09.1999}{}
\end{jhchildren}

\jhhousepic{021-05551.jpg}{}


Birgitta har två barn från sitt tidigare äktenskap.
\begin{jhchildren}
  \item \jhperson{Tony Kalijärvi}{05.04.1990}{}
  \item \jhperson{Nina Kalijärvi}{28.04.1992}{}
\end{jhchildren}

Bostadshuset köptes av Dagny Romars dödsbo år 1999. Guy är privatföretagare med eget snickeri tillsammans med sin bror Johan. Birgitta fungerar som församlingsvärdinna för Jeppo församling.


\jhoccupant{Romar}{Dagny}{1940-1999}
Dagny Sandberg, \textborn 01.09.1910 gift med Evald Romar 22.11.1936. Evald var född 27.12.1908 och stupade i vinterkriget 03.03.1940.
\begin{jhchildren}
  \item \jhperson{Lisa}{30.01.1938}{}
  \item \jhperson{Brita}{19.10.1939}{}
\end{jhchildren}

Dagny köpte tomten med ett påbörjat husbygge den 30.11.1940 av Emil och Lea Sandberg, ett husbygge hon senare fullföljde. Hon arbetade i huvudsak som banktjänsteman på Helsingfors Aktiebank i Jeppo, men i medlet av 1960-talet hyrde Ruben och Ragnborg Nygårds familj bostaden under den tid Dagny vistades i Vasa.

Dagny \textdied 20.07.1999


\jhoccupant{Sandberg}{Emil \& Lea}{1936-1940}
Emil, \textborn 09.06.1911, gift 25.06.1933 med Lea Silfvast, \textborn 19.07.1910.
\begin{jhchildren}
  \item \jhperson{Harry}{14.06.1935}{1988}
  \item \jhperson{Helena}{26.02.1937}{}
  \item \jhperson{Henrik}{19.05.1941}{} på Ojala
\end{jhchildren}

Tomten skiftades åt Lea och Emil från Silvast hemman 11.04.1936. Ett husbygge påbörjades småningom men avbröts av vinterkriget. Tomten och det påbörjade bygget såldes till Dagny Romar 1940 i samband med att Emil och Lea köpte ett hemman på Ojala. De hann bo en tid i det halvfärdiga huset innan det såldes till Dagny.

Emil var husbyggare och ägde dessutom en taxibil. Efter hemmansköpet på Ojala arbetade de med jordbruket, men flyttade 1951 till Krylbo i Sverige, där Emil började arbeta på Kloratfabriken.

Lea \textdied 11.09.1976 i Jakobstad  --  Emil \textdied 02.04.1997 i Krylbo, Sverige.


\jhhouse{Strandbo}{4:109}{Silvast}{4}{48}

\jhoccupant{Lassus}{Paul \& Alfhild}{1959 -}
Paul Lassus, \textborn 04.07.1918 i Oravais, gifte sig med Alfhild Huldén, \textborn 02.08.1924 i Nedervetil. Familjen köpte huset 1959 efter att tidigare bott på Finskasbacken.

\jhhousepic{024-05554.jpg}


Paul hade erhållit tjänst inom Jeppo Skogsvårdsförening som dess lagstadgade skogstekniker, en tjänst han innehade till sin pension. Han hade före tjänstgöringen i Jeppo haft motsvarande tjänst i Nedervetil. Alfhild var utbildad hårfrisörska och har tidvis upprätthållit sin yrkesskicklighet. Småningom sökte hon arbete på Mirka och blev efter en tid huvudförtroendeman på företaget.

Efter Paul:s pensionering flyttade de till Oravais, men har kvarhållit ägandet av fastigheten. Den ägs nu av dödsboet och står obebodd.
\begin{jhchildren}
  \item \jhperson{Bengt}{04.07.1948}{}
  \item \jhperson{Peter}{31.10.1953}{}
  \item \jhperson{Lisbeth}{14.04.1960}{}
  \item \jhperson{Anette}{24.04.1965}{}
\end{jhchildren}


\jhoccupant{Sandberg}{Runar \& Fanny}{1942-1959}
Runar Valfrid Sandberg, \textborn 31.12.1910 på Finskas, gifte sig 1940 med Fanny Hilda Alice Nylund, \textborn 08.10.1910 på Jungar. De hade träffats på Ev. Folkhögskolan 1928/29. Via Västankvarns jordbruksskola blev Runar 1943 utdimitterad från Vasa tekniska skola och dess kommunikationslinje. Därefter har han tjänstgjort hos Vasa vattendistrikts vattenbyrå fram till pensioneringen 1973.

Fanny skaffade sig mejeriutbildning och har haft mejersketjänster i Nedervetil, Kovjoki och Jeppo. I augusti 1954 flyttade familjen till Vasa.
\begin{jhchildren}
  \item \jhperson{Ralf}{20.08.1944}{}
  \item \jhperson{Boris}{18.09.1945}{}
\end{jhchildren}

Familjen Sandberg bibehöll ägandet av fastigheten efter flytten till Vasa 1954 fram till 1959. Under mellantiden har fastigheten varit uthyrd till flera familjer:

1954 -1956   Rafael Kronlöfs familj. (se närmare Stationsområdet)

1956 - 1957  Paul Djupsund, *30.09.1921, gift med Anita Söderström, *24.09.1925, båda från Gamlakarleby. Paul hade under denna tid arbete på kontoret vid Kiitola.
\begin{jhchildren}
  \item \jhperson{Marianne}{31.03.1949}{}
  \item \jhperson{Peter}{03.05.1951}{}
  \item \jhperson{Frank}{1954}{}
\end{jhchildren}

1957 - 1958  ``Zoo Cirkus Sandor''. Senhösten 1957 landade en främmande fågel i Jeppo. Det var nämnda cirkus. De överraskades av en häftig snöstorm med stora snömängder som omöjliggjorde en vidare förflyttning av cirkusens ekipage och utrustning, inklusive djuren. Förtvivlat försökte de finna en plats att stanna på och lösningen blev familjen Sandbergs fastighet. Här installerade de sig så gott det gick. Djuren placerades i uthuset, i huvudsak apor och hundar. Ved för husets uppvärmning var av nöden och prosten Runar Söderholm engagerade sig i cirkusfamiljens överlevnad. För att i någon mån få någon typ av inkomster, turnerade de i kommunens skolor för en ringa avgift.

Under sin tid i Jeppo, närmare bestämt 8 jan. 1958 avled familjen Kleinbarts mor, Elise. Hon var född 07.10.1897 i Altoff, Landau i Tyskland. Hon fick sin sista viloplats på Jeppo bergravningsplats. Hon hade 1924 gift sig med Sandor Kleinbart, som var född i Ungern 1895. Han dog i Finland 1983. Dottern Edit, född 1931 i Budapest, är sedan länge bosatt i Pargas, men brukar så länge krafterna räckte varje sommar besöka sin mors grav.

Familjen tillhörde en traditionsrik cirkusfamilj i Europa som i världskrigets förföljelser och dess efterdyningar slutligen funnit en fristad i Finland. Med sig hade man Ernst Späth, \textborn 1900, som hade huvudansvaret för djuren. Utan pass hade han gömt sig bland djuren under den 10 dagar långa båtresan till Finland 1950. Han dog 1983. Våren 1958 drog Zoo Cirkus Sandor vidare efter en strapatsrik vinter i Jeppo.


\jhoccupant{Sandell}{Evert \& Zaida}{1923-1942}
Evert Sundell, \textborn 12.05.1901 i Lepplax, Pedersöre, gifte sig 13.07.1924 med Zaida Silfvast, \textborn 24.09.1897. Huset byggdes på den köpta tomten som hörde till Zaidas hemgård. Den har genom åren något modifierats.

Familjen Sundell, med föräldrarna Otto och Lina i spetsen, hade kommit till Jeppo 1902, köpt affärsmannen Gustaf Julins fastighet nära stationsområdet och öppnat bageri- och matvaruhandel. Evert arbetade till en början på bageriet som bagare.

På den egna tomten födde han tillsammans med svågern Vilhelm Silfvast upp jakthundar före kriget, stövare som de också sålde. Evert installerade också en äggkläckningsmaskin i ett av uthusen och startade äggproduktion i liten skala. 1937 flyttade familjen till Ekenäs och därefter via Jeppo 14.03.1938 till Uleåborg, där Evert fick tjänst som konditor på ``Arina'' bageri.
\begin{jhchildren}
  \item \jhperson{Teresie}{1927}{}
  \item \jhperson{Enid}{1928}{}
  \item \jhperson{Linda}{1930}{}
\end{jhchildren}

Evert \textdied 1955 i Uleåborg  --  Zaida \textdied 24.09.1963         ”


\jhhouse{Tomtebo}{4:158}{Silvast}{4}{49}

\jhoccupant{Julin}{Alf \& Margit}{1956 -}

Alf, \textborn 12.03.1931, gift 08.08.1954 med Margit Sandberg, \textborn 27.03.1930.

Barn: Christine, \textborn 16.02.1961. Musikmagister från Sibeliusakademin. prisbelönad för sin insats för folkmusiken. Ledare för ``Jepokryddona''. Utnämndes till ``Mästerspelman'' år 2011. Christine tilldelades Svensk-Österbottniska samfundets kulturpris 2017.

Alf har sin utbildning inom el-området. Via en Hermodskurs och en pararellkurs vid Vasa Tekniska läroanstalt och efter godkända tenter fick han examen som övermontör. Under sin värnplikt 1951 utbildades han också till telegrafist på signalavdelningen och skötte Dragsvikgarnisionens in- och utgående divisionsradiotrafik. Samtidigt var han radioamatör inom armén.

\jhhousepic{019-05549.jpg}{}

Han fick sin första anställning inom Jeppo Kraft Andelslag och blev sedan dess verkställande direktör från 1952  fram till sin pensionering 1996. Utöver sitt arbete har han också engagerat sig med många förtroendeuppdrag, bl.a. som styrelseordförande i Jeppo Kommun innan fusionen med Nykarleby Stad, Munsala och Nykarleby landskommun. Som styrelseordförande i kommunen bidrog han mycket aktivt till lösa tomtfrågan för det expanderande företaget Mirka, som tidigare startat sin slippapperproduktion i den lediga fabriksfastigheten vid Kiitola. Genom resolut och skicklig hantering av ärendet, kunde en ny tomt inlösas på Skadaholmen och Mirkas verksamhet på orten säkerställas, en verksamhet som varit av stor betydelse för Jeppo som ort. Han har varit fullmäktigeledamot både i Jeppo och senare Nykarleby stad. Likaså har han varit  medlem i fullmäktige för Vasa regionplaneförbund med start 1963.

År 1953 var Alf ordförande för Jeppo Ungdomsförening. Han har fungerat som ordförande i Nykarleby stads elverksnämnd åren 1975-1993. Bl.a. i början av 1980-talet när kraftverksdammen i Nykarleby höjdes samtidigt som kraftverkets maskineri förnyades. Under byggnadstiden var han ordförande för byggnadskommittén. Inom Jeppo församling har han fungerat som ordförande för ekonomisektionen och under denna tid renoverades prästgården grundligt. Han har också varit medlem i Nykarleby kyrkliga samfällighets kyrkoråd.

Att hans insatser uppskattats visar tilldelningen av förtjänsttecken han erhållit: Finlands Svenska Kommunförbunds förtjänsttecken i silver 1975, Finlands Svenska Andelsförbunds förtjänsttecken i guld 1981 och Finlands vita ros ordens medalj av 1 klass med guldkors 1982. Jeppo Bygdespelmän har i honom och Margit haft långvariga och trogna supporters och aktiva dragare av folkdanslaget, som strävat till att föra traditionen kring ``Jeppomenuetten'' vidare till yngre släktled.

Margit har varit en av Andelsringens mest trogna butiksbiträden och har varit aktiv inom ungdoms – och nykterhetsföreningarna i Jeppo. Alf och Margit köpte tomten 1956 av Tauno Stenvik och uppförde bostadshuset åren 1957-58.

Margit \textdied 30.05.2014


\jhhouse{Hembo}{4:162}{Silvast}{4}{50}

\jhoccupant{Dahlström}{Olof \& Sirkka}{1989 -}

Olof Dahlström, \textborn 06.10.1940, gifte sig den 10.06.1960 med Sirkka Sipponen, \textborn 22.05.1938 i Pelkkala. Olof började i ``stopproikkon'' d.v.s. den arbetsgrupp som förhöjde och förnyade järnvägsspåret i slutet av 1950-talet. Han övertog hemmanet på Prästas 1964 och hade innan dess köpt Haniel Orrholms hus på Fors skattehemman i närheten av stationen 1961, se Fors nr 88. Han hade en tid arbete hos Schauman Oy i Jakobstad. Efteråt har han arbetat på Farm Frys innan han sjukpensionerades.


\jhhousepic{022-05552.jpg}{}


Sirkka har haft arbete på Mirka. Paret köpte denna fastighet av Harry Grahns dödsbo 1989. Tre barn finns i familjen.
\begin{jhchildren}
  \item \jhperson{Sven}{1960}{}
  \item \jhperson{Carina}{1962}{}
  \item \jhperson{Anne}{1966}{}
\end{jhchildren}


\jhoccupant{Grahn}{Harry \& Svea}{1960-1989}
Harry Johan Grahn, \textborn 29.06.1925 på Kampas, gifte sig 03.07.1949 med Svea Ingegerd Roos, \textborn 23.04.1925 på Romar. Harry var utbildad socionom och hade till en början tjänst som kommunalsekreterare i Maxmo innan han fick tjänst i Jeppo som bokförare och kanslist den 6 febr. 1951. Denna tjänst innehade han till 7 aug. 1956 då han utsågs till kommunsekreterare, en tjänst han hade fram till kommunsammanslagningen i Nykarlebynejden 1.1.1975. Därefter var han stadens kamrer fram till sin död 21.08.1987.

Svea arbetade på Helsingfors Aktiebank i Jeppo, till en början som bankfunktionär och senare som föreståndare för kontoret fram tills verksamheten avslutades och Svea flyttade till Jakobstad.
\begin{jhchildren}
  \item \jhperson{Kaj}{01.11.1952}{}
  \item \jhperson{Harriet}{01.11.1952}{06.11.1952}
  \item \jhperson{Monica}{26.10.1956}{}
\end{jhchildren}

Svea \textdied 18.03.2003



\jhhouse{Nordman}{4:108}{Silvast}{4}{51}

\jhoccupant{Nybyggar}{Jari \& Thanyanan}{1997 -}
Jari  Nybyggar, \textborn 01.05.1970, gifte sig 09.05.2010 med Thanyanan, \textborn 01.02.1982 i Thailand. Jari köpte fastigheten av Ragnar Nygårds dödsbo 1997. Han har sedan dess arbetat på Jeppo Potatis och är numera förman på företaget. Thanyanan har gått språkkurser på Kristliga Folkhögskolan i Nykarleby.


\jhhousepic{023-05553.jpg}{}

Jari har i ett tidigare äktenskap med Marika Nordlund, \textborn 1970, från Esse,
\begin{jhchildren}
  \item \jhperson{Jim}{1997}{}
  \item \jhperson{Liza}{1999}{}
  \item \jhperson{Hannah}{2002}{}
  \item \jhperson{Richard}{1996}{}, Marikas son fr. tidigare förhållande
\end{jhchildren}


\jhoccupant{Nygård}{Ragnar \& Ingeborg}{1951-1997}
Ragnar Edvin Elielsson, \textborn 01.04.1916 på Böös, gifte sig 08.02.1948 med Anita Ingeborg Smeds, \textborn 25.06.1925 i Åvist. Ragnar började som maskinist på Jungar andelsmejeri efter kriget. Efter stängningen av mejeriet 1951, flyttade han till Jeppo andelsmejeri i Silvast, där han hade samma arbete till sin pensionering.

Ingeborg hade tidvis dagbarn som hon skötte om. År 1951 köpte Ragnar fastigheten av Runar Nordman och byggde på tomten ett nytt hus åt familjen år 1960.
\begin{jhchildren}
  \item \jhperson{Gunborg}{03.01.1948}{}
  \item \jhperson{Leif-Ole}{14.08.1951}{}
\end{jhchildren}

Ragnar \textdied 25.03.1994  --  Ingeborg \textdied 02.07.2005


---> Silvast hemman nr \jhbold{351}.

\jhoccupant{Nordman}{Runar}{1936-1951}
Runar Nordman, \textborn 07.09.1917 på Silvast, fick fastigheten som gåva vid arvsskiftet efter sin mor Anna Sofia Silfvast som dog 1934. Runar, som fick egendomen 1936, bodde efter övertagandet bara en kort tid i huset. Han blev inkallad till värnplikt 1938 och kom civil bara några veckor innan vinterkriget bröt ut. Han blev på nytt inkallad och deltog i kriget ända till slutet 1944, vilket betyder att han tjänade i finska försvarsmakten nästan 6 år. Syskonen(se nedan) bodde tidvis tillsammans.

\jhhousepic{JNordmans.jpeg}{Huset revs 1950}


Ruth gifte sig 1939 i Oravais med  Paul Savonen, \textborn 09.04.1915 i Tammela, men återkom 1942. År 1944 flyttade Ruth till Gamlakarleby, men födde innan dess en son i detta hus. Han föddes 10.02.1942 och fick namnet Stig och blev storabror till Lars Silfvast.

Också Runar flyttar detta år till Gamlakarleby medan Rakel stannade hemma på Silvast hos sin mormor Anna Lovisa. Efter att Runar flyttat till Gamlakarleby har huset hyrts ut i olika perioder:

-- Familjen Lennart Gustafsson, 1944-1948. Se mer om dem i Fors nr 43.

-- Familjen Hugo Saha. 1948-1951.. Se mera om dem i Romar nr 24.


\jhoccupant{Nordman}{Johannes \& Anna-Sofia}{1918-1936}
Johannes Nordman, \textborn 01.03.1897, gifte sig 18.02.1917 med Anna Sofia Isaksdr. \textborn 03.05.1892. Från hennes hemgård utstyckades en tomt på östra sidan om landsvägen och ett hus för familjen byggdes kort därefter.

Efter barnens födelse reste Johannes till Amerika och återvände inte till hemlandet. Som ensamstående mor försörjde sej Anna-Sofia som piga och städerska på bl.a stationen.
\begin{jhchildren}
  \item \jhperson{Runar}{07.09.1917}{2000}
  \item \jhperson{Ruth}{25.01.1921}{1965}
  \item \jhperson{Rakel}{19.09.1922}{2012}
\end{jhchildren}

Anna-Sofia \textdied 07.09.1934  --  Johannes  \textdied 1977 i USA



\jhhouse{Nyhagen (Onnela)}{4:191}{Silvast}{4}{52}

\jhoccupant{Kalliosaari}{Kalevi \& Hilkka}{1986 -}
Kalevi Kalliosaari, \textborn  27.06.1951 i Alahärmä, gifte sig 06.06.1971 med Hilkka Ekola, \textborn 26.12.1953 i Storkyro. Makarna har under lång tid varit anställda som djurskötare på Johan Slangars svingård. Hilkka har efter det haft anställning på Jeppo Potatis. De köpte huset av Evert och Svea Lindströms dödsbo 1986.
\begin{jhchildren}
  \item \jhperson{Pekka}{1971}{}
  \item \jhperson{Virpi}{1974}{}
  \item \jhperson{Vesa}{1985}{}
\end{jhchildren}

\jhhousepic{072-05611.jpg}{}


\jhoccupant{Lindström}{Evert \& Svea}{1938-1986}

Evert Vilhelm Lindström, \textborn 02.12.1912, gifte sig 11.02.1934 med Svea Strand, \textborn 13.06.1910. Efter giftermålet fungerade de som vaktmästare på Jeppo Ungdomslokal och bodde också där innan de byggde huset bredvid järnvägen något norr om stationsområdet år 1938. Under några månader 1944/45 bodde en flyktingfamilj från Kemijärvi i husets ena ända.

Evert var järnvägsarbetare och, men arbetade också på SJ:s eget snickeri på stationsområdets östra sida mitt emot stationshuset. Efter att det brunnit ner på 1950-talet återupptogs inte verksamheten. Han var också en av dem som hade dejouransvar för pumpfunktionen vid VR:s pumpstation nere vid ån. Han hade därtill ansvaret för pärthyveln och dess drift nere vid ån vid Jeppo Kraft. Här passerade tusentals pärtklabbar genom bettet varje vår när saven stigit och gjorde klabbarna lättarbetade. Speciellt under 1940-talet och början av 50-talet, när många tak skulle repareras efter krigsperiodens eftersatta underhåll, var trycket på pärthyveln stort. Anmärkningsvärt är att Evert aldrig förlorade något finger i det farliga arbetet, vilket annars var vanligt bland hyvelskötarna. Svea skötte om familjens två kor och  arbetade en tid på Kiitola.
\begin{jhchildren}
  \item \jhperson{Gunnel}{06.12.1934}{}
  \item \jhperson{Margit}{04.08.1936}{}
\end{jhchildren}

Svea \textdied 04.08.1978  --  Evert \textdied 28.07.1985



\jhhouse{Lågbacka}{4:221}{Silvast/stoml. Norrgård 4:8}{4}{352}


\jhoccupant{Linder}{R?}{2003 -}

\jhoccupant{Stenvall}{Paul -> Kurt \& Staffan}{1949-2003}
I dag är den lägenheten, där familjen Högbackas bostadshus och ekonomibyggnader fanns, utstyckad odlingsmark. Paul Stenvall köpte fastigheten Norrgård 4:49, 1949 av Elis Isak Högbacka och hans hustru Elli. Byggnaderna revs i början av 1950-talet. Stomlägenheten Norrgård 4:210 (tidigare 4:49 och år 1915 4:8) ägs sedan 1990 av Kurt och Staffan Stenvall.

\jhoccupant{Högbacka}{Elis}{1939-1949}
När Elis Högbacka 1939 köpte tillbaka barndomshemmet av sin syster Helmi Luoma, bodde han själv i Jakobstad. Han         hyrde ut bostaden och de övriga byggnaderna, se nedan. Den 26.04.1941 sålde Elis en del av Norrgård 4:49 till Ensio		Markkula.

\jhbold{Hyresgäster:}

1944-1951
Aaltonen Juho Kustaa, \textborn 1885 och hans sambo Lydia Maria Jokinen, \textborn 1893, hyrde fastigheten 1944-1951. 		Majas dotterbarn från Helsingfors, Soili och Petteri bodde tidvis hos Maija och Jussi. Se Mågas 4:139, karta 5, nr 85. Aaltonen hyrde härefter in sig i Agda och Astrid Bäckstrands hus öster om järnvägen.

1943-1944
Under en kortare tid 1943-1944 c:a ett år hyrde en familj Hiltunen, med 3 eller 4 barn, byggnaderna. Hiltunen arbetade på Kiitola patronfabrik. När det blev fred flyttade de till Vasa.

1941-1942
Arthur och Lempi Pelmas med barnen Sulo och Karin	flyttade på sommaren 1941 till Högbackas hus. Vid krigsutbrottet 1939 hade familjen Pelmas flyttat från Smulters hus vid Stationsvägen till Yxpila i Gamlakarleby. Arthur hade fått arbete på krigsmaterialfabriken. 1942 kom de tillbaka till Jeppo, när han fick arbete som portvakt vid Kiitola. Elsbet föddes på hösten 1942 i Högbackas hus. Familjen Pelmas flyttade 1943 till Klings hus på andra sidan av järnvägen, karta 3, nr 323.

\jhoccupant{Luoma}{Helmi}{1937-1939}
Helmi Luoma, f. Högbacka, säljer 01.08.1938 0,0010 mantal till Edvin Lillström och till Elis och Aina Stenvall 0,0131 mantal, som gått vidare till Kurt och Staffan Stenvall samt Göran Stenvall. Den 24.05.1939 säljer hon barndomshemmet med den odlade jorden intill, till sin bror Elis.

\jhoccupant{Högbacka}{Elis \& Helmi}{1937}
Efter mor Marias död 07.08.1937 köper Helmi bror Elis' andel i sterbhuset. Helmi gifter sig 21.11.1937 med Toivo Emil Luoma, \textborn 10.03.1912 i Alahärmä.

\jhoccupant{Högbacka Isak}{dödsbo}{1918-1937}
Änkan Maria och de minderåriga barnen Elis Isak, Helmi Maria och Aina Linnéa, säljer 0,0032 mantal den 28.07.1934 till Erik Widell.

\jhoccupant{Högbacka}{Isak \& Greta-Lisa}{1903-1937}

År 1899 anlände Isak Högbacka med sin familj till Jeppo från Ylistaro. Han köpte därefter en del mark av Silvast skattehemman och övertog också det hus och de uthus som byggts i omedelbar närhet av järnvägen, på dess västra sida,i höjd med Furubackens sydligaste ända. Fastigheten utgjorde 17/256 dels mtl. Säljare var Karl Eriksson Mågas och dennes hustru Anna.

Isak,  \textborn 07.08.1846 i Ylistaro, var gift med Greta-Lisa Johansdr., \textborn 29.03.1850. Makarna fick endast ett barn, det föddes i Ylistaro.

Barn: Isak Emil, \textborn 06.12.1888.

Isak Emil gifte sig 01.11.1908 med Maria Josefina Samuelsdr. \textborn 10.02.1890 i Jeppo.
\begin{jhchildren}
  \item \jhperson{Elis Isak}{23.03.1910}{}
  \item \jhperson{Aili Olivia}{19.10.1915 i Jeppo}{27.11.1915}
  \item \jhperson{Helmi Maria}{23.07.1917 i Jeppo}{}
  \item \jhperson{Aina Linnea}{14.09.1918 i Jeppo}{19.06-1926}
\end{jhchildren}

Isak fick aldrig se sitt sista barn födas. Han dog 28.08.1918. Änkan Maria fortsatte med barnen på det lilla hemmanet. Småningom blev hon tvungen att avyttra egendomen och flyttade till ett hus, som stod på nästan samma plats som nu nr 53. När hon dog 08.05.1937, gifte sig Elis Isak samma höst 14.11.1937 och flyttade till Vasa. Samma år flyttade också systern Helmi till Vasa och gifte sej.


\jhoccupant{Mågas}{Karl-Erik \& Anna-Sanna}{1898-1903}

Karl-Erik Mågas, \textborn 29.01.1869 i Munsala, gifte sig med Anna-Sanna Greta, \textborn 05.05.1871 i Jeppo. Makarna kom från Munsala 1895 och bosatte sej på Grötas. Väl där förlorade de sitt barn som föddes 26 okt. och dog 12 nov. och hade fått namnet Karl Johan. Fram till 29.01.1898 finns de upptagna bland ``lösa personer'' d.v.s. de äger ingen fastighet. Men denna dag undertecknar de köpebrevet som ger dem mark invid järnvägsspåret norr om Jeppo station.

Som inhyses har de något år senare Isak Högbackas familj. År 1903 säljer Mågas sin egendom och Isak Högbacka köper mark och fastigheter vid järnvägsspåret. Karl-Erik och Anna-Sanna Mågas flyttar i mars 1903 till Munsala.



\jhhouse{Kling}{4:181}{Silvast/stoml. Norrgård 4:8}{3-4}{323}

\jhoccupant{Staten}{SJ}{1967 -}
I samband med att Statens Järnvägar 1967  höjde och gjorde 	järnvägsvallen bredare försvann en stor del av tomten under järnvägen.

\jhoccupant{Kling}{Johannes \& Signe}{1924-1967}
Johannes William Anttison Kling (tidigare släktnamn Ylivainio), \textborn 16.06.1898 i Alahärmä, flyttade till Jeppo 07.05.1924. Han vigdes 01.06.1924 med Signe Susanna Samuelsdotter Roos, \textborn 20.04.1903 i Jeppo. Det nygifta paret byggde ett bostadshus samt ett uthus 	på ett mindre område av Silvast hemman, som blivit kvar på östra sidan 	om järnvägen.
\jhhousepic{Kling2.jpeg}{Kling Johannes och Signe}
Barn: Alice Linnéa, \textborn 01.10.1924 - \textdied 09.12.1960 i Gamlakarleby

Johannes och Signe Kling inköpte jordbruksmark 1937 på västra sidan av	järnvägen, 0,0024 mantal av stomlägenhet Norrgård 4:49. Säljare Signes syster Maria Högbackas barn.	Familjen Kling flyttade 15.05.1941 till Gamlakarleby. Alice gifte sig 31.07.1955 med Elis Ingvald Muukkonen, \textborn 14.02.1927, som flyttat till Nyköping, Sverige. Äktenskapet barnlöst.

\jhbold{Hyresgäster:}

1943-1945	Pelmas
Arthur och Lempi Pelmas med barnen Sulo, Karin och lilla Elsbet hyrde fastigheten ca.3 år 1943-1945. De hade tidigare bott i Högbackas hus på västra sidan av järnvägen. Arthur arbetade som portvakt i vaktstugan vid 	krigsmaterialfabriken vid Kiitola. Marjatta föddes 30.12.1943 i Klings hus. Se mera Svinvallen II 21-0, karta 5, nr 86. År 1946 flyttade familjen till Sundqvist hus, Björkbacka 4:38, nära järnvägsövergången vid Sparvbacken.

1946-1949	Ylivainios
Då familjen Pelmas flyttat till Sundqvists hus flyttade familjen Ylivainio till Klings hus 1946. Familjen bestod av föräldrarna Antti och Ida och 9 barn. Se Dahlbo 4:127, karta 5, figur 377. Antti arbetade under kriget som portvakt i vaktstugan vid infarten till krigsmaterialfabriken vid Kiitola. Våren 1945 flyttades ``Asevarikko'' från Jeppo till Gamlakarleby. Antti reste med tåg varje dag till Gamlakarleby 1946-1949, där han arbetade som hästkarl på krigs- och sjukvårdsmaterialfabriken. De två materialsidorna skildes åt 1949 och tillverkningen av sjukvårdsmaterial överfördes till Ilmajoki. Antti arbetade med transporten mellan Gamlakarleby och Ilmajoki.

Familjen flyttade 1949 till Ilmajoki, förutom äldsta sonen 1) Vilhelm, som 1948 flyttade till Tammerfors och arbetade på Finlayson till pensioneringen. Som ung pojke och till 1948 var Vilhelm jordbruksarbetare hos bönder i Jeppo. 2) Paul blev sjuk i kriget och blev inlagd på Roparnäs mentalsjukhus. Han kom senare till Hagalund i Nykarleby, där han bodde till sin död 2005. Paul är begraven i Ilmajoki. 3) Kaarlo arbetade som reparatör på järnvägen i Jeppo till 1949, då han blev områdesvakt på Försvarsmaktens medicinska depå i Ilmajoki. År 1953-1967 arbetade Kaarlo i Vilppula på Kolho som skogshuggare och senare med impregneringsarbeten. 1967-1980 återtog han sitt tidigare arbete som områdesvakt i Ilmajoki. 4) Leo hann också arbeta som reparatör på järnvägen i Jeppo. Leo blev skogshuggare i Sverige. Sjukpensionerades på grund av svår astma. Även 5) Rainer var järnvägsarbetare i Jeppo. Efter en kort tid i Sverige arbetade han på Försvarsmaktens medicinska depå i Ilmajoki. Han utbildade sig till telefonmontör och flyttade till Tammerfors. 6) Margit arbetade 12 år på fabriken i Ilmajoki och fick därefter tjänst som vaktmästare på Palonkylä skola i Ilmajoki till pensioneringen. 7) Veikko dog i Jeppo i lunginflammation efter en salmonella-/(tyfus)epidemi, endast 12 år gammal. Han är begraven i Jeppo. 8) Mauno har arbetat som lagerhandledare på Försvarsmaktens medicinska depå. 9) Reijo arbetade tre år som telefonmontör. Flyttade 1962 till Vasa och arbetade inom livsmedelsindustrin som köttstyckare på OTKs korvfabrik och Österbottens Kött, där arbetsplatsen senare blev i Nurmo.

Antti Ylivainio \textdied 11.02.1974  --  Ida \textdied 13.02.1979. De är begravna i Ilmajoki.



\jhhouse{Kennola}{4:153}{Silvast}{4}{352x}

\jhoccupant{Kennola}{Elis \& Saima}{1929-1968}

Elis Nikolai Kennola, \textborn 16.09.1902 på Grötas, gifte sig 25.11.1928 med Saima Rajamäki, \textborn 02.04.1903 i Lappo. Elis arbetade nästan hela sitt verksamma liv på järnvägen i olika typer av uppgifter. Han var också cykelmontör och reparerade cyklar. Saima dog 17.03.1932. I samband med ny underfart under järnvägen för Nylandsvägen, kom familjens bostad i vägen, löstes in och avlägsnades 1992.

Barn: Irma Ilona, \textborn 02.05.1929

\jhhousepic{Kennola.jpeg}{}


Elis ingick nytt äktenskap 10.06.1934 med Elsa Irene Lönnqvst, \textborn 24.05.1905 i Purmo. Hon var sömmerska och sydde kläder åt ortens
befolkning i sitt hem.
\begin{jhchildren}
  \item \jhperson{Maj-Britt Johanna}{14.12.1935}{}
  \item \jhperson{Henrik Johan}{01.05.1937}{}
\end{jhchildren}

Elsa \textdied 29.12.1978  --  Elis \textdied 04.03.1980



\jhhouse{Nyhagen}{4:169 - (4:173 el. 4:15)}{Silvast}{4}{356x}

\jhoccupant{Huhtala}{Jakob \& Amanda}{1928-1975}

Jakob Huhtala, \textborn 07.05.1875 i Ylistaro. Han flyttade till Jeppo 12.12.1901 och forslade huset från stationsvägen och den nybildade familjen kunde flytta in. Jakob var gift 3 ggr; hans 3:e hustru var Amanda.

Jakob dog 05.02 -1939 och hon kom att överleva sin man med 36 år. Han arbetade som dikare och nyodlare åt socknens bönder, ett arbete han utförde ända till slutet av sitt liv.

\jhhousepic{Huhtala.jpg}{}


Amanda föddes 25.08.1895 i Alahärmä och gifte sig 02.12.1928 med Jakob. In i äktenskapet hämtade hon tre barn efter sin man Jaakko Filemon Finne, som dog 1920. År 1926 flyttade hon till Alahärmä för att 1928 efter giftermålet med Jakob byta bostadsort till Jeppo.
\begin{jhchildren}
  \item \jhperson{Maria}{1913}{1931}
  \item \jhperson{Aino}{1916}{}
  \item \jhperson{Hilma}{1919}{1934}
\end{jhchildren}

Tillsammans med Jakob föddes en dotter och en son:
  \begin{jhchildren}
    \item \jhperson{Margareta}{08.09.1930}{29.12.1994} i Oravais
    \item \jhperson{Martin}{12.02.1933}{23.08.2014}
\end{jhchildren}


1915 – 1928

Jakob var i sitt 2:a äktenskap gift  med Anna Sofia Andersdr Rajakoski, \textborn 29.10.1877 i Lappo. De gifte sig 13.05.1915. Anna Sofia hämtade en dotter in i äktenskapet, Saima Elisabet, \textborn 02.04.1903. Jakob och Anna Sofia fick inga gemensamma barn. Anna Sofia dog  23.04.1928.

1902 – 1915

Jakob ingick sitt 1:a äktenskap den 10.11.1901 med Maria Sjöblom, \textborn 27.06.1879 i Jeppo och syster till lärarinnan Karin Sjöblom (se nr 55).
\begin{jhchildren}
  \item \jhperson{Ellen}{04.08.1902}{}
  \item \jhperson{Elsa}{01.01.1905}{}
  \item \jhperson{Jakob Vilhelm}{22.02.1908}{09.06.1926}
  \item \jhperson{Anna Linnea}{02.11.1909}{18.02.1925}
  \item \jhperson{Valdine}{07.05.1912}{01.08.1913}
\end{jhchildren}

Det är sannolikt att huset byggdes genast efter giftermålet alldeles nära Silvastbäcken på norra sidan av landsvägen. Det syns skymta på ett fotografi. När det flyttades ca 200 m norrut längs bäcken är inte klart. Fastigheten har under en stor del av sin historia förvaltats som dödsbo.

Lars och Inger Silfvast köpte fastigheten 1976 och byggnaderna revs på 1990-talet.



\jhhouse{Etelämäki}{4:72}{Silvast}{4}{65}

\jhoccupant{Strand}{Sven \& Eira}{1978 -}
Sven Strand, \textborn 31.10.1942 på Silfvast, gift 13.07.1963 med Eira Pelkkala, \textborn 31.12.1941 i Ylihärmä. Makarna köpte tomten av Arvid och Svea Mylläris dödsbo år 1978. Huset restes genast därefter.

\jhhousepic{61-05700.jpg}{}

Sven började i unga år arbeta på Keppo pälsfarm för att år 1964 starta en egen minkfarm på Silvast. År 1985 hyrde han ett område på stadens pälsfarmsområde på Grötas där han födde upp rävar, en verksamhet han fortsatte med till pensioneringen 2001. Eira har en kort tid arbetat på postkontoret som städare innan  hon började som postiljon i Jeppo. Efter att kontoret  stängdes  överfördes hon till Nykarleby.

\begin{jhchildren}
  \item \jhperson{Mona}{15.12.1963}{}
  \item \jhperson{Gunilla}{18.08.1966}{}
\end{jhchildren}



---> Rno 4:72 av Silvast hemman nr \jhbold{365}

\jhoccupant{Mylläri}{Arvi \& Svea}{1942-1978}

Arvi Emanuel Mylläri, \textborn 01.08.1915 i Alahärmä, gifte sig 19.08.1934 med Svea Maria Salo, \textborn 15.10.1909 på Silvast. Arvi var bl.a. telefonmontör men fungerade också som eldare på det tåg som 1 ggr per dag trafikerade från Ylihärmä st. till Kauppi, där det fanns en mindre industrianläggning. En kort tid arbetade han också i Sverige.

Svea arbetade på Kiitola före och efter kriget innan hon kom till Sundells bageri som bagare. Där stannade hon till pensionen.
\begin{jhchildren}
  \item \jhperson{Erik}{01.11.1935}{21.02.2002} i Jakobstad
  \item \jhperson{Alf}{08.12.1943}{}, musiker i Karleby
  \item \jhperson{Dorrit}{29.11.1950}{2014} i Nykarleby
\end{jhchildren}


\jhoccupant{Salo}{Johan \& Sanna}{1902-1942}
Johan (Jukka) Juhanpoika, \textborn 25.06.1869 i Alahärmä, gifte sig 26.03.1893 med Sanna Sofia Silfvast, \textborn 25.04.1871. Johan var både skomakare och fiolspelman och hade därtill 2 kor till familjens försörjning. Den odlade jorden hade utgått från Sanna Sofias hemgård på Silfvast. Tomten arrenderades 1902 och inlöstes 1929.
\begin{jhchildren}
  \item \jhperson{Johan Sigurd}{02.09.1897}{24.05.1918}
  \item \jhperson{Senja Sofia}{07.04.1902}{}
  \item \jhperson{Julius Reinhold}{19.01.1904}{27.07.1952}
  \item \jhperson{Anna Johanna}{14.09.1907}{}
  \item \jhperson{\jhbold{Svea Maria}}{05.10.1909}{05.04.1977}
  \item \jhperson{Aino Elise}{05.07.1912}{05.06.1914}
\end{jhchildren}


\jhhouse{Mäki}{4:34}{Silvast}{4}{66}

\jhoccupant{Strand}{Sven \& Eira}{1969 -}

Sven och Eira Strand inköpte fastigheten 1969. Se närmare nr 65. Här bodde de till dess det nya huset var klart 1978. Efter inköpet 1969 har fastigheten hyrts av fler hyresgäster:

År 1979 – 1993  Alice Sandviks familj. Se Nr 127. Före dem Nina Nyberg och före henne Gunilla Strand och Tomi Kärkkäinen, \textborn 1969 i St. Mickel. Först bland hyrestagarna var Janne Ketola. Huset har stått obebott sedan år 2000.


\jhoccupant{Kuoppala}{Elis \& Ester}{1949-1960}

Elis Alexander Kuoppala, \textborn 05.08.1915, gifte sig 21.12.1939 med Ester Maria Mäkinen, \textborn 07.03.1914 i Alahärmä. Under kriget sårades Elis svårt och fick livslångt handikapp. Trots det var Elis en positiv och arbetsam person. För en äldre generation är hans kioskrörelse på bäckliden invid stationsvägen ett kärt minne. Här dracks mycken lemonad och förtärdes många sockrade grisar. Lemonaden tillverkade han själv, likaså glass. Pilsner hade också god åtgång. Lördagskvällar var stationsvägen full med folk och hit svängde man gärna in. Närheten till bio Silva var strategiskt rätt.

Senare köpte han fastigheten nr 62 där han bedrev kaférörelse och hade samtidigt en fotoaffär, som också att innebar att han färdades runt och fotograferade bl.a. skolklasser och andra objekt enligt önskan.

\jhhousepic{064-05601.jpg}{Huset f.n. obebott.}

Ester hade hand om hushållets göromål och skötte om sin svärmor Johanna, \textborn 13.06.1874, fram till hennes död 1956. Elis' yngsta syster Svea, som var handikappad, fordrade också speciell uppmärksamhet. Johanna och Svea bodde tillsammans i husets övre våning.
\begin{jhchildren}
  \item \jhperson{Erik}{24.04.1940}{31.03.2008}
  \item \jhperson{Eva-Liisa}{10.11.1942}{}
\end{jhchildren}


\jhoccupant{Kuoppala}{Alexander \& Johanna}{1902-1949}

Alexander, \textborn 05.04.1880 i Ylihärmä, gifte sig med efternamnet Hautamäki den 09.03.1902 med Johanna Maria Henrik Gustavsdr. Silfvast, \textborn 13.06.1874. Alexander tog senare efternamnet Kuoppala. Han var som flera andra inflyttade från Härmä skomakare, som var hans och familjens levebröd. Huset byggdes efter att tomten erhållits från Silfvast hemman.
\begin{jhchildren}
  \item \jhperson{Zenia}{31.10.1902}{14.01.1903}
  \item \jhperson{Johannes}{11.03.1904}{08.07.1925}
  \item \jhperson{Ragni}{23.01.1906}{14.10.1907}
  \item \jhperson{Elis Alexander}{29.12.1907}{26.05.1914}
  \item \jhperson{Evert}{17.08.1913}{22.02.1940 i Summa}
  \item \jhperson{Elis Alexander}{05.08.1915}{24.10.1992}
  \item \jhperson{Anna}{28.11.1917}{dog i Sverige}
  \item \jhperson{Svea}{04.02.1922}{02.02.1985}
\end{jhchildren}

Alexander \textdied 21.12.1925  --  Johanna \textdied 14.11.1956



\jhhouse{Mäki}{4:34}{Silvast/stoml. Silvast 10}{4}{67}

\jhoccupant{Leppävuori}{Aki}{2014 -}
Aki Leppävuori, \textborn 04.08.1980 i Karleby, köpte 2014 fastigheten Mäki 4:34 av sina föräldrar. Aki arbetar på Mirka. Aki har även köpt Mejeri-fastigheten, karta 5, nr 58.


\jhoccupant{Leppävuori}{Tapio \& Jaana}{1988-2014}
I juni 1988 köpte Tapio Leppävuori, \textborn 22.04.1953 i Vetil, och	hans hustru Jaana, f. Lieska, \textborn 15.11.1961 i Vetil, fastigheten Mäki 4:34.	Familjen Leppävuori kom till Jeppo i oktober 1983. Jaana arbetar på Mirka och Tapio har arbetet på Mirka och hos Ab Br. Finskas Oy.
\jhhousepic{063-05699.jpg}{Leppävuori Aki}

\begin{jhchildren}
  \item \jhperson{\jhbold{Aki}}{04.08.1980 i Karleby}{}
  \item \jhperson{Annu}{17.09.1982 i Karleby}{}, heter som gift Uusi-Pohdola
\end{jhchildren}


\jhoccupant{Rökman}{Guy \& Maire}{1976-1988}
Guy Rudolf Selim Rökman, \textborn 18.03.1916 i Kronoby, och hans hustru Maire Maria Nurmela, \textborn 24.02.1927 i Salla, köpte fastigheten 1976. De grundrenoverade huset och förstorade bostadsbyggnaden med bastu, tvättstuga och pannrum. Guy pensionerades 1981 från sin mångåriga tjänst som designer vid Oravais Klädesfabrik Ab. Guy har tre söner, Kay, Bo och Kim från sitt första äktenskap.

Guy var en aktiv föreningsmänniska, som ung en ivrig scout, och efter kriget en aktiv reservofficer och veteranmedlem, samt en av dem som 	grundade Lions club i Oravais, brandchef och aktiv inom pensionärsföreningar. Hans stora hobby var filateli och oljemålning.	Maire arbetade också på Oravais Klädesfabrik Ab som sekreterare för ledningen och fr.o.m. 1978 som resesekreterare också för moderbolaget Oy Keppo Ab, som 1978 flyttade sitt huvudkontor från Keppo till Oravais. År 1985 fusionerades dotterbolaget Oy Wik \& Höglund Ab med Keppo och blev KWH-koncernen Ab. Maire gick i pension 1987 och de sålde fastigheten på Silvast och flyttade till Jakobstad. De fortsatte att vara aktiva inom	dagvården, pensionärsföreningar och inom åldringsvården, i all synnerhet Maire. Guy \textdied 26.08.2011, är begraven i Oravais.


\jhoccupant{Finne}{Gunnar \& Ragni}{1955-1976}
Den 22.04.1955 erhöll Gunnar Vilhelm Finne, \textborn 06.05.1916 i Markby, Nykarleby lkm, och hans hustru Ragni Ellen Torhild Elenius, \textborn 03.11.1919 på Jungarå hemman, lagfart på fastigheten Mäki 4:34. Strax efter giftermålet 1950 hade Gunnar och Ragni emigrerat till Sverige. Deras dotter Gun Ellen föddes 12.06.1952 i Karlstad. Ragnis mor, som var änka, Ellen Sofia, född Romar 18.06.1895 på Romar hemman, flyttade till Silvast från Jungarå, och bosatte sig i husets nedre våning. Ellen avled den 18.10.1976.

\jhbold{Hyresgäster} i övre våningen:
\begin{enumerate}
  \item Strand Sven och Eira,	1963-1964
  \item Björkqvist Paul och Carita, 1958-1959
  \item Saha Hugo och Linnéa,	1951-1958
\end{enumerate}


\jhoccupant{Koukkuluoma}{Toivo}{1953-1955}
Den 25.07.1953 erhåller Toivo Koukkuluoma lagfart på Mäki 4:34.	Överlåtare är Juho och Sanna Koukkuluomas arvingar. Toivo fortsätter att bo med sin familj i Jakobstad.


\jhoccupant{Koukkuluoma}{Juho \& Sanna Maija}{1905-1953}
Den 27.10.1920 anhåller Juho Kusta Koukkuluoma, \textborn 15.05.1885 i inkl. Dödsb. Ylistaro, om inregistrering av Mäki kolonisationslägenhet, omfattande 0,0003 mantal av skattelägenhet Silvast 4:10. Den 28.09.1925 erhåller Juho lagfart på Mäki 4:34. Juho och hans hustru Sanna Maija Finni, \textborn 01.02.1884 i Alahärmä, hade köpt huset på tomten, när de gifte sig i början av 1900-talet. Sanna Maija Finni flyttade till Jeppo 09.12.1901.	Bostadshuset brann 1916 då deras son Åke, \textborn 1914, bara 2 år gammal, tände eld och 	gömde sig i vedlåren; han dog i branden. Juho arbetade på SJ:s vedplan och tidvis som bagare hos Otto Sundell. Juho och Sanna byggde	ett nytt bostadshus i två våningar. Även Sanna arbetade ibland på 	vedplanen. Juho och Sanna Koukkuluomas barn är alla födda i Jeppo.
\begin{jhchildren}
  \item \jhperson{Sylvi}{1907}{1909 i lunginflammation i Jeppo}
  \item \jhperson{Kustaa Elis}{1909}{1940 i solsting vid järnvägsarbete}
  \item \jhperson{Onni Johannes}{24.02.1911}{11.07.1991 i Jakobstad}
  \item \jhperson{Toivo Isak}{16.09.1912}{2006 i Jakobstad}
  \item \jhperson{Åke}{1914}{1916 i eldsvåda}
  \item \jhperson{Anni Maria}{21.05.1916}{08.03.2009, Sverige}
  \item \jhperson{Aino Katarina}{11.02.1918}{10.07.2011, Jakobstad}
  \item \jhperson{Vilho Gunnar}{06.05.1921}{16.05.1921, Pargas}
  \item \jhperson{Taimi Göta Linea}{30.01.1924}{25.07.1925, Jeppo i	lunginflammation}
  \item \jhperson{Åke Vilhelm}{20.05.1925}{14.03.2008 i	Australien}
\end{jhchildren}

Förutom spädbarnen Sylvi, Åke och Taimi så är Kustaa och Aino	Katarina, gift Stenvall, begravna i Jeppo. Aino har bott större delen av sitt liv i Jeppo, se mera Öständ 4:197, karta 9, nr 126. Onni har arbetat som 	stationskarl i Jakobstad, hustru Aili Maria, född Rauhala 02.01.1905, dog	05.09.1987. Onni och Aili har två barn; Rauni Maria född 20.01.1934, gift 	Jussila, och 	Risto Juhani född 29.10.1946, flyttade med sin familj 1970	till Sverige. Toivo, gift med Alexandra Irene, född 21.08.1924, har också	arbetat på SJ i Jakobstad. De har tre söner och en dotter. Vilho, gift med	Lotten Linéa, född Vilén 29.08.1921, bor i Pargas och har en dotter	Kerstin född 03.03.1949.  Åke emigrerade 1950 till Brisbane i Australien.

Anni gifte sig med Väinö Jakko Kuivamäki, född 1916 i Nykarleby 	landskommun. Deras son Veikko Armas föddes 18.07.1937, då de bodde i Koukkuluomas hus. En kort tid efter Veikkos födelse flyttade familjen till Kauhajoki och Koskenkorva i Sydösterbotten. De kom tillbaka till Jeppo och bodde i övre våningen 1944-1951. De emigrerade till Sverige 1951. Veikko är gift med Doris Anneli, född Fjellström 1940 i Nykarleby. De har tre barn, Gunnel, Stig och Henrik.

\jhbold{Hyresgäster:}
År 1951, efter familjen Kuivamäki, hyrde frånskilda Linnéa Kosola, född Asplund 14.03.1918 på Fors hemman, och hennes dotter Sirkka Lisa, född 07.09.1942 i Jakobstad, lokalen. Linnéa gifte sig med	Hugo Saha, född 30.10.1919 i Pyhäjärvi. Familjen bodde i lokalen till 1958, då de kunde flytta in i sitt nybyggda bostadshus på Romar hemman, nr 26. Före Annis familj hyrde fru Nordlund från Närpes och hennes två döttrar Virpi och Sonja ca två år under kriget den övre 	våningen. De flyttade till Jakobssons hus, Nybygge 4:28, nr 370.

Juho Kustaa Koukkuluoma \textdied 27.01.1951  --  Sanna Maija \textdied 28.11.1946. De är begravna i Jeppo.

Huset som brann 1916 hade en person ``Måla-Pitter'' byggt. Petter Pettersson Nordman, \textborn 06.06.1858 i Pörtom, kom till Jeppo 12.11.1892. Petter gifte sig med Brita Henriksdotter, textborn 05.11.1869 på Jungar.

Petter \textdied 12.02.1906  --  Brita \textdied 28.09.1911
