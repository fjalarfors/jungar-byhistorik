%%%
% [chapter] Jungar/Svartbacken, hemman Nr 7
%
\jhchapter{Jungar/Svartbacken, hemman Nr 7}

Svartbacken omfattas av vidstående karta nr \jhbold{13}.


KARTA nr 13, utsnitt --->


%%%
% [subsection] Lägenheter på Svartbacken
%
\jhsubsection{Lägenheter på Svartbacken}



Svartbacken känns som ett begrepp, vilket ibland andas något mytomspunnet och speciellt. Många bybor har hört namnet nämnas, men få vet dess historia öster om järnvägen. I dag präglas platsen av att de flesta familjer, som en gång bott här, har flyttat bort. Också husen har försvunnit och spåren efter dem är dolda av efterkommande verksamheter. Samma öde har mött de flesta torpområden i vårt land.

En bebodd lägenhet finns kvar, d.v.s \jhbold{Solbacka} 7:32 med Håkan och Lenelis Cederström, och den presenteras under Jungar-Ruotsala, nr 20.



%%%
% [house] Janne
%
\jhhouse{Janne}{7:3}{Jungar}{13}{S}


%%%
% [occupant] Sandqvist
%
\jhoccupant{Sandqvist}{\jhname[Johan]{Sandqvist, Johan} \& \jhname[Johanna]{Sandqvist, Johanna}}{1899--1931}
Enligt torparlagen 1918 ansöker änkan Johanna Sandqvist om inlösen av området för backstugan på Svartbacka, Jungar. Legogivare Johan och Daniel Jungar. Legonämnden godkänner den 22.10.1920 ansökan och ersättningen. Den 25.09.1925 säljer hon fastigheten med ifrågavarande gård och härtill hörande jord om 14,5 kappland av Jungar hemman N:o 7 i Jungar by, köpare  bondsonen Johannes Strand och hustru Anna Gustafa. Fastigheten Jannes 7:3 införd 1938 i Jordregistret samtidigt med ett stort antal torp- och backstugalägenheter i Jungar by.

Den 04.02.1931 flyttade Johanna till Gamlakarleby till sonen Alfred. Flickorna Edith och Maria hade flyttat 1930 till Gamlakarleby. Huset var det sista som revs på norra sidan av vägen på Svartbacka.

Johanna Sandqvist hade för sin man Johan Sandqvist den 19.04.1892	undertecknat ett köpebrev på en bostadsbyggnad vid Jungarbacken på en arrendetomt om 3 kappland jord på Tollikko	hemman nr 11 i Jungar by. Arrendetid 49 år och arrendegivare Matts	Isakson Gunnar. Familjen bodde ca tre år på Tollikko. Johan hade rest	19.01.1891 till Amerika och återvände i slutet av 1892. Johan reste 1901 en andra gång till Amerika och stannade där ca tre år. Från Tollikko flyttade familjen till Mietala. Orsaken till att de lämnade Tollikko, var antagligen att huset var litet och kallt. Sonen Gustav Alexander, \textborn 28.08.1893  ---  \textdied 03.10.1893 på Tollikko, även följande son fick samma namn Gustav Alexander, \textborn 26.10.1894. Flickorna Maria, Irene och 	Edith föddes på Svartbacka.

Skomakare Johan (Janne) Gustavsson Sandqvist \textborn 07.02.1864 på Jungar bland nio syskon och fyra halvsyskon. Johans föräldrar, Gustav Andersson Jungar Svartbacka (\textborn 25.12.1821  ---  \textdied 15.10.1888), gift 21.04.1843 med Maja Fredriksdotter, \textborn 04.12.1824 på Gunnar  ---  \textdied 1969 på Jungar. Då Gustav blev änkling gifte han sig med Kajsa Jakobsdotter (\textborn 19.11.1847  ---  \textdied 16.03.1922). Av de 10 barnen i det första äktenskapet gifte sig tre pojkar och två flickor och i det andra två pojkar. När tillnamnen ändrades, tog Johan tillnamnet \jhbold{Sandqvist}, bror Erik tillnamnet \jhbold{Blomqvist} och halvbröderna Erik Johan och Emanuel tillnamnet \jhbold{Forslund}.

\jhhousepic{Alexander Sandqvist Svartbacken.jpg}{Alexander Sandqvists 1918-1920: fr.v. Alfred Sandqvist, Edit Björk, Johanna Sandqvist, NN, NN, Joel och Alexander Sandqvist}

Johan Sandqvist gifte sig 22.01.1888 med Johanna Henriksdotter Gädda, \textborn 25.06.1864 i Larsmo. Johannas mor, Johanna Jonedotter Jungar \textborn 07.11.1834 i Jeppo, gifte sig med Henrik Johansson Gädda \textborn 11.11.1839 i Larsmo, och flyttade 1861 till Larsmo. Familjen Gädda flyttade till Jungar i Jeppo 1879.
\begin{jhchildren}
  \item \jhperson{\jhname[Maria]{Sandqvist, Maria}}{11.03.1862}{1863}
  \item \jhperson{\jhname[Johanna]{Sandqvist, Johanna}}{25.06.1864}{}
  \item \jhperson{\jhname[Susanna]{Sandqvist, Susanna}}{06.12.1869}{}
  \item \jhperson{\jhname[Lovisa]{Sandqvist, Lovisa}}{27.10.1872}{}
  \item \jhperson{\jhname[Mathilda]{Sandqvist, Mathilda}}{04.07.1875}{}
\end{jhchildren}
Susanna och Lovisa gifte sig med jeppoborna Backlund respektive Forsman och emigrerade till Amerika vid sekelskiftet. Mathilda flyttade 1894 till Gamlakarleby. Mathilda är mormor till numera pensionerade biskop/professor Gustav Björkstrand.

Johanna och Johan fick en stor barnskara.
\begin{jhchildren}
  \item \jhperson{\jhname[Joel]{Sandqvist, Joel}}{22.10.1888}{01.12.1975, Gamlakby}
  \item \jhperson{\jhname[Johannes Alfred]{Sandqvist, Johannes Alfred}}{28.11.1890}{24.12.1957, Gamlakby}
  \item \jhperson{\jhname[Gustav Alexander]{Sandqvist, Gustav Alexander}}{28.08.1893}{03.10.1893, Jeppo}
  \item \jhperson{\jhname[Gustav Alexander]{Sandqvist, Gustav Alexander}}{26.10.1894}{10.10.1973, Jeppo}
  \item \jhperson{\jhname[Maria Johanna]{Sandqvist, Maria Johanna}}{11.08.1896}{15.09.1896, Jeppo}
  \item \jhperson{\jhname[Edvard]{Sandqvist, Edvard}}{11.06.1898}{06.01.1971, USA}
  \item \jhperson{\jhname[Johanna Maria]{Sandqvist, Johanna Maria}}{30.07.1900}{23.11.1956, Karleby}
  \item \jhperson{\jhname[Anna Irene]{Sandqvist, Anna Irene}}{24.01.1905}{31.05.1925, Jeppo}
  \item \jhperson{\jhname[Edith]{Sandqvist, Edith}}{18.02.1908}{31.07.1989 Gamlakby}
\end{jhchildren}
Familjen bodde på Jungar när Joel och Alfred föddes.

Joel fick ett utmärkt avgångsbetyg 24.05.1901 från högre folkskolan i Jungar by af Jeppo kapell, undertecknat av ordf. Nordgren och skolans	föreståndare Fredrik Thors. Efter konfirmation 1904 köpte Joel tågbiljett	till Gamlakarleby till sin moster på Kallis hemman. På hösten fick han arbete som småskollärare läsåret 1904/1905 på Storby småskola. Hans	bror Alfred fick samma arbete 1905/1906. På sommaren 1905 fick Joel arbete och bostad hos handelsman Alarik Sandström. Joel flyttade officiellt från Jeppo till Gamlakarleby 20.12.1906. Den 03.06.1917 gifter Joel sig med Maria Vierimaa (\textborn 01.08.1899  -–-  \textdied 21.01.1977). Joel och Maria fick 4 flickor och 3 pojkar. Firma Sandström övergick i järnaffär Björklund, där Joel blev direktör 1924. Den 01.09.1935 grundade Joel och ``Maju'' en egen firma, Gamlakarleby Husgerådsaffär. År 1952 blev	han även ägare till Björklunds Järnaffär, vilken han som 71-åring sålde 1960 till OTK eller Renlunds Järnaffär från Helsingfors.

Alfred flyttade officiellt från Jeppo till Gamlakarleby 21.02.1913. Han gifte sig 21.05.1914 med Selma, född Joutsen 11.04.1898 i Evijärvi, dog 23.02.1984 i Gamlakarleby. Alfred och Selma startade en firma med tillverkning av väskor, som fanns i samma hus, som de bodde i på Långbrogatan. De fick tre pojkar och tre flickor. De tre yngsta barnen, en pojke och två flickor, födda 1921-1924, dog som spädbarn av svaghet och kikhosta. Äldsta sonen Sven blev Borgåbo, Ingmar gifte sig med Linnéa i Replot och tillverkade väskor där, Gunhild ``Gunni'' förblev ogift och bodde med föräldrarna. Farmor Johanna kom till familjen 1931.

\jhbold{Alexander} var den enda av barnen, som blev kvar i Jeppo. Han köpte en fastighet på Silvast och öppnade en kolonialvaruaffär 1920. Se mera 	Silvast, Sandqvist 4:21, nr 62 och Romar, Sandqvist 2:23, nr 38.

Maria gifte sig 21.05.1931 med tågkonduktör Walfrid Nykvist (29.06.1906 --- 01.05.1970) från Karleby, där de bosatte sig. Maria och Walfrid fick en son och fyra döttrar. 1939 flyttade familjen till Uleåborg, där de två yngre flickorna föddes. En vinter i slutet av kriget bodde Maria med flickorna en kort tid hos familjen Sandqvist på Nylandsvägen. Den äldsta dottern Mona gick i Jungar skola. I slutet av 1940-talet flyttade familjen från Uleåborg tillbaka till Kyrkbacken i Karleby.

Edith gifte sig 17.06.1930 med Åke Andreas Björk (28.03.1908 --- 14.02.1989) från Gamlakarleby. Åke blev delägare och direktör i Hud- 	och Skinnkompaniet. Familjen bodde i ett tvåvåningshus invid 	vattentornet på Torggatan. Sommartid bodde familjen på en stor stuga ute på udden vid Varvet. Åke var en god seglare med egen större segelbåt. Edith och Åke fick två pojkar och två flickor; Rolf emigrerade med familj till Sverige, Marita gifte sig i Norge, konstnär Kurt flyttade till Vasa och Ulla blev kvar i Gamlakarleby.



%%%
% [house] Hellman
%
\jhhouse{Hellman}{7:5}{Jungar}{13}{H}


%%%
% [occupant] Hellman
%
\jhoccupant{Hellman}{\jhname[Johan]{Hellman, Johan} \& \jhname[Kajsa]{Hellman, Kajsa}}{1894-1921}
Johan Johansson Hellman, \textborn 03.12.1842 i Purmo, gifte sig med Kajsa Johansdotter, \textborn 16.10.1841 på Mietala hemman.
\begin{jhchildren}
  \item \jhperson{\jhname[Johannes]{Hellman, Johannes}}{20.07.1867}{}
  \item \jhperson{\jhname[Anders]{Hellman, Anders}}{11.03.1874}{1892 i Amerika}
  \item \jhperson{\jhname[Gustaf]{Hellman, Gustaf}}{30.11.1878}{}
  \item \jhperson{\jhname[Edla]{Hellman, Edla}}{11.02.1883}{}
\end{jhchildren}
Matts Jungar var ägare till den torplägenhet, som Johan Hellman köpt av Jakob Fagerholm. Huset fanns på samma tomt som Cederströms hus idag (se Jungar-Ruotsala nr 20). Hellman var murare och kallades murar-Hellman eller ``Måg-Jott'', Måg-Jott kanske för att han kom som måg till Jeppo. Han testamenterade torplägenheten åt sin dotter Edla Rystad, som liksom syskonen emigrerat till Amerika. Hon ansökte år 1921 om att få lösa in området, vilket godkändes trots avsaknad av kontrakt. Det kunde nämligen bevisas att dagsverken till jordägaren utförts samt arrende betalats.

Enligt Agda Nygård köpte Elis Kennola gården i början på 1930-talet och flyttade den till Silvast (se Silvast, nr 352x).


%%%
% [occupant] Fagerholm
%
\jhoccupant{Fagerholm}{\jhname[Jakob]{Fagerholm, Jakob} \& \jhname[Sanna Lisa]{Fagerholm, Sanna Lisa}}{1876–1894}
Jakob Jakobsson Fagerholm, \textborn 16.11.1845 på Grötas hemman gifte sig med Sanna Lisa Henriksdotter, \textborn 31.01.1848 på	Ojala hemman.
\begin{jhchildren}
  \item \jhperson{\jhname[Johannes]{Fagerholm, Johannes}}{24.06.1876, till USA}{}
  \item \jhperson{\jhname[Jakob]{Fagerholm, Jakob}}{02.03.1879}{}
  \item \jhperson{\jhbold{\jhname[Anders]{Fagerholm, Anders}}}{24.07.1881}{}
  \item \jhperson{\jhname[Ida Maria]{Fagerholm, Ida Maria}}{18.03.1883, gift Björkqvist (Grötas 126)}{}
\end{jhchildren}
Jakobs farfar var sockensmeden Matts Fagerholm från Fagernäs i Larsmo. Han var bror till soldaterna Johan Knut och Anders Bly.

Drängen Matts Fagerholm flyttade till Nykarleby 1804 och bosatte sig på Grötas. Var han och hans familj bott är okänt. I kyrkböcker benämns han smed och sockensmed. Sonsonen Jakob är född på Grötas. Jakob och Sanna Lisa började sitt gemensamma liv som backstugusittare på Svartbackan, men inropade år 1894 ett hemman på Grötas och där byggde Jakob sitt nya hem åt familjen.

Jakob \textdied 23.02.1910  ---  Sanna Lisa \textdied 19.01.1927




%%%
% [house] Backlund
%
\jhhouse{Backlund}{7:9}{Jungar}{13}{B}


%%%
% [occupant] Backlund
%
\jhoccupant{Backlund}{\jhname[Anders]{Backlund, Anders} \& \jhname[Maria]{Backlund, Maria}}{1929 – 1943}
Backstugusittare, timmermannen Anders Backlund, \textborn 11.12.1860 på Jungar hemman, gift med Maria Lovisa Johansdotter, \textborn 18.02.1863. Se Gunnar, gård nr 115.

Anders dog 28.02.1905. År 1929 flyttade änkan Maria Lovisa samt barnen Erik Villiam och Anders Valfrid till Jungar, Svartbacken. År 1921 behandlade legonämnden Marias ansökan om att få lösa in det 4 kappland stora backstuguområdet. Jordägare var Maria Strengell (4 kpl) samt Henrik Gunell (1/3 kpl på Ruotsala). Ansökan godkändes trots att skriftligt kontrakt saknades.

Familjen hade tidigare bott på Svartbacken och eventuellt enbart hyrt ut sin stuga. Familjen flyttade senare till gård nr 13 på Ruotsala. Maria Lovisa dog 22.09.1943.


%%%
% [occupant] Backlund
%
\jhoccupant{Backlund}{\jhname[Johan]{Backlund, Johan} \& \jhname[Maria]{Backlund, Maria}}{1896-1902}
Johan Eriksson Backlund, \textborn 11.10.1865 på Jungar hemman, gift med Maria Henriksdotter, \textborn 02.12.1866.
\begin{jhchildren}
  \item \jhperson{\jhname[Johan Edvard]{Backlund, Johan Edvard}}{1890}{1890}
  \item \jhperson{\jhname[Johannes Emil]{Backlund, Johannes Emil}}{1891}{1895}
  \item \jhperson{\jhname[Vendla Maria]{Backlund, Vendla Maria}}{1894}{}
  \item \jhperson{\jhname[Katrina Irene]{Backlund, Katrina Irene}}{1896}{}
  \item \jhperson{\jhname[Edith Johanna]{Backlund, Edith Johanna}}{1899}{}
  \item \jhperson{\jhname[Erik Joel]{Backlund, Erik Joel}}{1901}{1901}
\end{jhchildren}
Familjen bodde på Jungarå några år, flyttade till Svartbacken 1896. År 1901 tar Johan ut betyg till Amerika, år 1902 följde hustru och barn efter.


%%%
% [occupant] Kauhajärvi
%
\jhoccupant{Kauhajärvi}{\jhname[Erik]{Kauhajärvi, Erik} \& \jhname[Kajsa]{Kauhajärvi, Kajsa}}{1849 -	1900}
Erik Jonasson Kauhajärvi, \textborn 10.01.1827 i Purmo, vigd 01.07.1849 med Kajsa Andersdotter, \textborn 17.09.1826 på Jungar hemman.
\begin{jhchildren}
  \item \jhperson{\jhname[Greta]{Kauhajärvi, Greta}}{1850}{}
  \item \jhperson{\jhname[Anders]{Kauhajärvi, Anders}}{1851}{}
  \item \jhperson{\jhname[Catharina]{Kauhajärvi, Catharina}}{1854}{}
  \item \jhperson{\jhname[Anna]{Kauhajärvi, Anna}}{1856}{}
  \item \jhperson{\jhname[Erik]{Kauhajärvi, Erik}}{1858}{}
  \item \jhperson{\jhname[Anders]{Kauhajärvi, Anders}}{1860}{}
  \item \jhperson{\jhname[Simon]{Kauhajärvi, Simon}}{1863}{}
  \item \jhperson{\jhbold{\jhname[Johannes]{Kauhajärvi, Johannes}}}{1865}{}
  \item \jhperson{\jhname[Simon]{Kauhajärvi, Simon}}{1866}{}
  \item \jhperson{\jhname[Gustaf]{Kauhajärvi, Gustaf}}{1869}{}
\end{jhchildren}
(6 av barnen dog som små)

Av barnen är Johannes (Johan) med sin familj skriven på Jungar, troligen övertog han föräldrarnas torpkontrakt. Sonen Anders flyttade först till Tollikko, sedan till Gunnar. Både Erik och Kajsa dog år 1900.



%%%
% [house] Jungar
%
\jhhouse{Jungar}{7:10}{Jungar}{13}{J}


%%%
% [occupant] Jungar
%
\jhoccupant{Jungar}{\jhname[Emil]{Jungar, Emil} \& \jhname[Sanna]{Jungar, Sanna}}{}
Emil Jungar, \textborn 1879,  gifte sig med Sanna Slangar, \textborn 1880. De var bosatta på Svartbackan som backstugusittare. Emils far var bonden Anders Jungar, \textborn 1838 på Jungar. Av honom och hustrun Sanna Lisas 3 söner Gustaf, \textborn 1871, Anders, \textborn 1873 och Emil, \textborn 1879, var det sonen Emil som kom att bosätta sig på Svartbacken. Här föddes Emils och Sannas barn.
\begin{jhchildren}
  \item \jhperson{\jhname[Emil Severin]{Jungar, Emil Severin}}{23.07.1900}{}, till Amerika 1923
  \item \jhperson{\jhname[Johannes Mauritz]{Jungar, Johannes Mauritz}}{18.12.1901}{}, till Amerika 1922
  \item \jhperson{Paul \jhbold{Valdemar}}{06.09.1906}{}, till Amerika 1924, återkom 1935
\end{jhchildren}

Som så många andra från Finland reste Emil ner till Syd-Afrika för att hitta arbetsförtjänst i gruvorna tillsammans med sina bröder. Under tiden i Syd-Afrika bytte bröderna efternamn till Strengell och i ett av breven hem till familjen har han undertecknat det med såväl Emil Jungar som Emil Strengell. Senare reste han till USA där han dog i  minsjuka 12.05.1922. Han efterlämnade änkan Sanna och barnen. Hon gifte sig senare med Anders Åstrand.

Huset flyttades 1935 av sonen Valdemar Strengell till Nyåkern nära ån (se lägenhet Ådahl, Jungar, nr 25). Platsen där huset ursprungligen stått på Svartbackan kunde ännu för några decennier sedan urskiljas. Det är inte känt om det var Emils föräldrar Anders Jungar (1838-1909) och hustrun Sanna \textborn Mietala (1835-1908) som uppfört huset.
