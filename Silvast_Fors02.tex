
<--- se KARTA nr 5 --->

%%%
% [house] Lindén
%
\jhhouse{Lindén}{893-408-18-0}{Silvast}{5}{H1}

\jhhousepic{Holmen 1.jpg}{Stefan och Bodil Nyvall}
%%%
% [occupant] Nyvall
%
\jhoccupant{Nyvall}{\jhname[Stefan]{Nyvall, Stefan} \& \jhname[Bodil]{Nyvall, Bodil}}{2008--}
Stefan Nyvall, \textborn 09.05.1984, gift 2011 med Bodil Kronqvist, \textborn 12.07.1988 i Nykarleby. Huset inköptes år 2008.

Stefan är skogsmaskinförare och arbetar åt Holma T \& T, vars verksamhetsområde är maskinentreprenad inom primärproduktion och jordbyggnad. Bodil är f.n. hemmafru och tar hand om den unga barnaskaran.
\begin{jhchildren}
  \item \jhperson{\jhname[Thias]{Nyvall, Thias}}{22.05.2008}{}
  \item \jhperson{\jhname[Siri]{Nyvall, Siri}}{23.05.2011}{}
  \item \jhperson{\jhname[Sanni]{Nyvall, Sanni}}{11.08.2014}{}
\end{jhchildren}


\jhoccupant{Lindén - Malinen}{Staffan - Jonna}{--2008}
\jhname[Staffan Lindén]{Lindén, Staffan}, \textborn 1983, och sambon \jhname[Jonna Malinen]{Malinen, Jonna}, \textborn 1986, bodde i huset sedan Staffans föräldrar flyttat till Nykarleby. Staffan är elmontör på BSB i Nykarleby.
\begin{jhchildren}
  \item \jhperson{\jhname[Vilma]{Lindén, Vilma}}{2003}{}
  \item \jhperson{\jhname[Felicia]{Lindén, Felicia}}{2007}{}
\end{jhchildren}


\jhoccupant{Lindén}{\jhname[Thomas]{Lindén, Thomas} \& \jhname[Lilian]{Lindén, Lilian}}{1985--}
Thomas Lindén, \textborn 1961 (se Fors, nr 94), gift med Lilian Ojalammi, \textborn 1963, uppförde huset 1985. Paret flyttade till Nykarleby, varefter sonen Staffan med familj senare övertog gården.
\begin{jhchildren}
  \item \jhperson{\jhname[Christoffer]{Lindén, Christoffer}}{1980}{}, styrman
  \item \jhperson{\jhbold{\jhname[Staffan]{Lindén, Staffan}}}{1983}{}, elmontör
  \item \jhperson{\jhname[Erika]{Lindén, Erika}}{1988}{}
  \item \jhperson{\jhname[Robin]{Lindén, Robin}}{1992}{}
\end{jhchildren}



%%%
% [house] Näs
%
\jhhouse{Näs}{893-408-4-188}{Silvast}{5}{H2}

\jhhousepic{Holmen 2.jpg}{Juha och Merja Hietala}
%%%
% [occupant] Hietala
%
\jhoccupant{Hietala}{\jhname[Juha]{Hietala, Juha} \& \jhname[Merja]{Hietala,Merja}}{2013--}
Juha Jorma Aatos Hietala, \textborn 28.04.1988 i Jeppo (se Fors, nr 88), gift med Merja Hannele Latvala, \textborn 13.03.1990 i Jakobstad. Makarna fann sin boplats här och köpte huset av Göran och Annikki Näs år 2013.

Juha arbetar som processkötare på KWH-Mirka. Hans fritidsintresse är jakt. Merja är laborant på Jeppo Potatis.
\begin{jhchildren}
  \item \jhperson{\jhname[Sofia Amanda]{Hietala, Sofia Amanda}}{16.09.2012}{}
  \item \jhperson{\jhname[Jere Jasper]{Hietala, Jere Jasper}}{02.07.2015}{}
\end{jhchildren}


\jhoccupant{Näs}{\jhname[Göran]{Näs, Göran} \& \jhname[Annikki]{Näs, Annikki}}{1986--2013}
Göran Näs, \textborn 1944, gift med Annikki Mattila, \textborn 1946 uppförde gården år 1986. Makarna har arbetat på KWH Mirka, Göran som logistikkoordinator och Hannele som löneräknare. Göran hade dessförinnan, år 1980--1984, fungerat som köpman på butiken K-Göran i Silvast, se nr 92.



%%%
% [house] Häggblom
%
\jhhouse{Häggblom}{893-408-19-1}{Silvast}{5}{H3}

\jhhousepic{Holmen 3.jpeg}{Ralf och Yvonne Häggblom}
%%%
% [occupant] Häggblom
%
\jhoccupant{Häggblom}{\jhname[Ralf]{Häggblom, Ralf} \& \jhname[Yvonne]{Häggblom, Yvonne}}{1989--}
I gården bor Ralf Erik Häggblom, \textborn 07.12 1953, och hustrun Yvonne Merja Kamilla Häggblom (Leino) 01.08 1955. Huset byggdes 1989. Ralf är diplomekonom, pensionerad vd, och Yvonne pensionerad lektor. Ralf har bl.a. varit styrelseordförande i Silvast Vattenandelslag (se Romar, nr 40), Jeppo Ungdomsorkester, IF Minken och långvarig medlem i KPO:s förvaltningsråd. Han har också suttit med i styrelsen för Nykarleby Affärsverk och Nykarleby Bostäder.

Ralf är mångårig medlem i LC Jeppo och har hunnit inneha flera befattningar i organisationen. Yvonne var en av de två första kvinnliga medlemmarna i klubben år 2008 och blev dess första kvinnliga president 2014--2015.

\begin{jhchildren}
  \item \jhperson{\jhname[Joakim]{Häggblom, Joakim}}{29.03.1981}{}, ekon.mag., försäljn.chef Oy Primo Finland Ab, bor i Korsholm
  \item \jhperson{\jhname[Jonina]{Häggblom, Jonina}}{27.05.1985}{}, förskollärare i Jeppo - Pensala skola, bor i Kimo
  \item \jhperson{\jhname[Teresa]{Häggblom, Teresa}}{07.09.1991}{}, ped.kand., förest. på Jacob’s Kindergarten i J:stad, bor i J:stad
\end{jhchildren}



%%%
% [house] Rönnlandet
%
\jhhouse{Rönnlandet}{4:168}{Silvast}{5}{53}

Styckat av stomlägenhet Silvast 4:10

\jhhousepic{025-05556.jpg}{Kim och Virpi Elenius vid korsningen Östra Jeppovägen -- Nylandsvägen}

%%%
% [occupant] Elenius
%
\jhoccupant{Elenius}{\jhname[Kim]{Elenius, Kim} \& \jhname[Virpi]{Elenius, Virpi}}{1993--}
Kim Elenius, \textborn 1968, gifte sig 1998 med Virpi Kalliosaari, \textborn 1974 i Jeppo. Kim har arbetat med fastighetsservice och är nu anställd på KWH Mirka. Han utför också bilreparationer via egen firma.

Virpi har i 10 års tid varit kock på ABC-servicestation i Oravais, men är sedan något år tillbaka anställd på Jeppo daghem.
\begin{jhchildren}
  \item \jhperson{\jhname[Henri]{Elenius, Henri}}{1997}{}
  \item \jhperson{\jhname[Rasmus]{Elenius, Rasmus}}{2007}{}
\end{jhchildren}


%%%
% [occupant] Nygård
%
\jhoccupant{Nygård}{\jhname[Hilding]{Nygård, Hilding} \& \jhname[Ellen]{Nygård, Ellen}}{1962--\allowbreak 1993}
Paul Hilding Nygård, \textborn 24.07.1929 på Böös, gifte sig 1956 med Ellen Elisabet Stenbacka, \textborn 18.10.1933 i Oravais. Hilding utbildade sig till järnvägsbokhållare. Han har varit anställd på flera av regionens järnvägsstationer, så också i Jeppo. Han har bl.a. skött ankommande och avgående trafik utöver andra  funktioner på stationerna.

Hilding har hela livet varit mycket intresserad av idrott och varit styrelseordförande i Jeppo Idrottsförening. Han har deltagit i flera Vasalopp och deltog också i JIF:s klubbkamper på 1950-talet.

Ellen har varit lärare och startade sin gärning i Jeppo vid Jungar folkskola år 1956 fram tills dess skolorna i Jeppo centraliserades 1966. Därefter tjänstgjorde hon en tid, 1970-71, i den nya centrumskolan. Hon har under tiden i Jeppo varit aktiv i både nykterhets- och marthaföreningarna samt suttit i församlingens kyrkoråd. Småningom flyttade makarna till Oravais och huset som byggts 1962 såldes 1993.

Hilding \textdied 15.12.2014



%%%
% [house] Strand
%
\jhhouse{Strand}{4:217}{Silvast}{5}{54}

Styckad av stomlägenhet Strand 4:17

\jhhousepic{026-05557.jpg}{Einar och Agneta Strand}

%%%
% [occupant] Strand (Dödsbo)
%
\jhoccupant{Strand (Dödsbo)}{\jhname[Einar]{Strand (Dödsbo), Einar} \& \jhname[Agneta]{Strand (Dödsbo), Agneta}}{1935--}
Einar Isak Strand, \textborn 22.02.1912 på Silvast, gifte sig 17.02.1935 med Maria Agneta Stolt, \textborn 19.03.1913 på Hilli. Tomten utstyckades från Einars hemgård 1935 och huset byggdes.

Familjen försörjde sej på det lilla jordbruket och med diversearbete. Einar blev som de flesta andra inkallad till kriget där han utmärkte sej med att visa avsaknad av rädsla. Jordbruket avvecklades småningom efter kriget och med expansionen av pälsdjursfarmningen på Keppo fick bägge arbete där. Han hjälpte också tidvis till vid Varma-butiken.

Einar sjukpensionerades och Agneta fick arbete på Mirka sedan dess verksamhet startat på Kiitola. Hon arbetade senare som städerska på Jeppo Sparbank.
\begin{jhchildren}
  \item \jhperson{\jhname[Maj-Britt]{Strand (Dödsbo), Maj-Britt}}{30.06.1935}{2004 i Oravais}
  \item \jhperson{\jhname[Volmar]{Strand (Dödsbo), Volmar}}{25.01.1940}{}
  \item \jhperson{\jhname[Sven]{Strand (Dödsbo), Sven}}{31.10.1942}{}
  \item \jhperson{\jhname[Nils]{Strand (Dödsbo), Nils}}{13.07.1948}{}
\end{jhchildren}

Einar \textdied 10.02.1968  ---  Agneta \textdied 26.03.2008

Dödsboet äger fortsättningsvis fastigheten.



%%%
% [house] Lugnbo
%
\jhhouse{Lugnbo}{4:45}{Silvast}{5}{55}

Styckad av stomlägenhet Lindfors 4:14

\jhhousepic{KarinS.jpg}{Senaste ägare Denice och Erik Norrgård, gården riven}

%%%
% [occupant] Norrgård
%
\jhoccupant{Norrgård}{\jhname[Erik]{Norrgård, Erik} \& \jhname[Denice]{Norrgård, Denice}}{1969--}
Erik Norrgård, \textborn 27.05.1936, gifte sig med Denice Back, \textborn 29.11.1945. De hyrde fastigheten 1969 av Karin Sjöbloms arvingar för att slutligen 1972 köpa den. Familjen bodde på fastigheten fram till 01.02.1995. Huset revs 22-25 febr. 2015.

Erik arbetade redan i unga år som grävmaskinsförare och övergick småningom i Andelsringens tjänst. Han arbetade tillsammans med Edvin Jungell på magasinet vid järnvägsstationen, ett arbete som krävde mycket lyftande.  Senare fick han anställning vid Prevex i Nykarleby. Han var en av dem som startade hjälporganisationen ``Polen källan'' som, efter legaliseringen av ``solidaritetrörelsen'' och Sovjetunionens sammanbrott, påbörjade hjälpsändningar till i huvudsak nordöstra Polen. Den intensivaste verksamheten skedde under den första delen av 1990-talet. Erik var i detta sammanhang en drivande kraft och fick som ett tack för sitt engagemang en utmärkelse av Polens ambassadör i Finland. Ambassadören kom med ambassadens tjänstebil till Denices och Eriks sommarstuga i Pensala för att överlämna ``prenikan''. För allmänheten i regionen var Erik i radion känd som ``Jepo-Erik'' i flera av de publika sändningarna.

Denice kom som kontrollassistent till Jeppo. Efter giftermålet bodde familjen på Ruotsala, därifrån de flyttade 1969. Hon har haft arbete vid Sundells bageri och under lång tid varit butiks- och kassabiträde inom Andelsringen och via olika affärsnamnbyten varit S-kedjan trogen.

Barn: \jhname[Katarina]{Norrgård, Katarina}, \textborn 09.09.1966

Erik \textdied 28.02.2013


%%%
% [occupant] Sjöblom
%
\jhoccupant{Sjöblom}{\jhname[Karin]{Sjöblom, Karin}}{(1929)1952--\allowbreak 1969}
Karin (Katarina) Sjöblom, \textborn 12.08.1885 i Jeppo, men hon brukade säga: ``Jag är född i Vasa, men bosatt i Jeppo''. Därför blev hon ibland kallad för ``Vasa-Karin''. Hon var småskollärarinna på Jungar skola åren 1910--\allowbreak 1952 där hon också bodde. Hon har vid sidan av lärarparet Thors utfört den längsta lärargärningen i Jeppo.

Från 1952 bodde hon stadigvarande på fastigheten, som under tiden efter föräldrarnas död varit uthyrd till bl.a. Gunnar Romar. Också Georg och Vera Romar bodde här efter giftermålet 1941 fram till dess det egna huset stod klart 1950 vid Romarbäcken. Efter att Karin flyttat in hade hon också underhyresgäster på övre våningen. Bl.a. bodde Anita Ström (gift med Brian Lavast) där 1958--\allowbreak 1961.

Karin var ogift och dog 25.07.1969.


%%%
% [occupant] Sjöblom
%
\jhoccupant{Sjöblom}{\jhname[Jakob]{Sjöblom, Jakob} \& \jhname[Anna-Maria]{Sjöblom, Anna-Maria}}{1870--\allowbreak 1929}
Jakob Sjöblom, \textborn 16.09.1849, gifte sig med Anna Maria Gustavsdr., \textborn 12.12.1844 i Nykarleby. Under deras tid byggdes huset som senare ombyggdes med s.k. ``brutet tak''.
\begin{jhchildren}
  \item \jhperson{\jhname[Anna Sanna]{Sjöblom, Anna Sanna}}{12.04.1874}{}
  \item \jhperson{\jhname[Johan Jakob]{Sjöblom, Johan Jakob}}{21.09.1876}{}
  \item \jhperson{\jhname[Edla Maria]{Sjöblom, Edla Maria}}{27.06.1879}{}
  \item \jhperson{\jhname[Ida]{Sjöblom, Ida}}{11.10.1882}{05.04.1969 i USA}
  \item \jhperson{\jhbold{\jhname[Karin (Katarina)]{Sjöblom, Karin}}}{12.08.1885}{25.07.1969 i Jeppo}
  \item \jhperson{\jhname[Selma]{Sjöblom, Selma}}{09.01.1888}{27.09.1978 i Jeppo}
  \item \jhperson{\jhname[Matts Vilhelm]{Sjöblom, Matts Vilhelm}}{30.12.1889}{}
\end{jhchildren}

Ida och Selma reste till Amerika. Ida stannade, Selma återvände.



%%%
% [house] Silfvast
%
\jhhouse{Silfvast}{4:169}{Silvast}{5}{56, 56a-b}

Stomlägenhet Silvast 4:10

\jhhousepic{027-05559.jpg}{Lars och Inger Silfvast}

%%%
% [occupant] Silfvast
%
\jhoccupant{Silfvast}{\jhname[Lars]{Silfvast, Lars} \& \jhname[Inger]{Silfvast, Inger}}{1974--}
Lars, \textborn 27.08.1945 i Gamlakarleby, gift 24.06.1972 med Inger Söderström, \textborn 26.03.1951 i Karleby, Storby.
\begin{jhchildren}
  \item \jhperson{\jhname[Ronny]{Silfvast, Ronny}}{16.04.1974}{}
  \item \jhperson{\jhname[Eva-Lotta]{Silfvast, Eva-Lotta}}{18.04.1978}{}
  \item \jhperson{\jhname[Robert]{Silfvast, Robert}}{29.12.1986}{}
\end{jhchildren}
Lars har gått i Evangeliska Folkhögskolan i Vasa och har fått lantbruksutbildning vid Lannäslund i Jakobstad, där Inger samtidigt gick på husmorslinjen. Lars och Inger övertog hemmanet 1974 efter att Vilhelm Silfvast plötsligt avlidit. En kort tid fortsatte jordbruket med djurhållning innan man definitivt sadlade om till specialiserad potatisodling, som  fortsatte under många år. Sedan år 2013  är den odlade jorden utarrenderad.

Utöver arbetet på den egna lägenheten har Lars under många år kört lastbil oftast i utrikestrafik. Inger har också utfört kompletterande arbete vid sidan av jordbruket och har haft fast tjänst som närvårdare inom Nykarleby stad. Inger gick i pension 2014.


%%%
% [occupant] Silfvast 4:113
%
\jhoccupant{Silfvast 4:113}{\jhname[Vilhelm]{Silfvast 4:113, Vilhelm} \& \jhname[Ellen]{Silfvast 4:113, Ellen}}{1936--\allowbreak 1974}
Vilhelm, \textborn 16.09.1913, gift 19.03.1939 med Ellen Gustafsson, \textborn 06.02.1912 med efternamnet Rundt. Ellen var tidigare gift med Axel Silfvast, \textborn 13.12.1908. Han dog 14 maj 1935 av hjärnhinneinflammation. Paret var barnlöst. Ellen gifte om sig 1939 med hans yngre bror, Vilhelm (se ovan). Axels och Vilhelms äldsta syster Anna Sofia, som 1917 gift sig med Johannes Nordman, hade dött 1934. Hennes dotter Rakel flyttade då hit till sin mormor Anna Lovisas hem och bodde här till 1945, då hon gifte sig med Hugo Nybyggar.

Vilhelm ägde nu tillsammans med sin mor Anna Lovisa och Ellen hemmanet, som fram till dess var ett dödsbo. Anna Lovisa sålde år 1936 sin andel av hemmanet till Vilhelm. I och med att Vilhelm och Ellen gifte sig 1939 kom de att tillsammans äga hela hemmanet. Under kriget tjänstgjorde Vilhelm på hemmafronten och var stationerad bl.a. i Långåminne och Ekenäs. Under ``Lapplandskriget'' bodde evakuerade från Kemijärvi en kort tid i huset.

År 1951 byggdes ny ladugård och ekonomiebyggnad. En ny bostadsbyggnad på fastigheten började byggas 1955, varefter den gamla bostaden, som stod närmare älven men vägg i vägg med den nya, revs.

Barn: \jhbold{Lars}, adoptivson, \textborn 27.08.1945. Lars är son till Vilhelms systerdotter Ruth Savonen och adopterad av Vilhelm och Ellen år 1946.

Vilhelm \textdied 03.04.1974  ---  Ellen \textdied 01.08.1975



%%%
% [oldhouse] Gamla huset
%
\jholdhouse{Gamla huset}{4:169}{Silfvast}{5}{356}

Styckad av stomlägenhet Silvast 4:10

\jhhousepic{SilfvastL.jpeg}{Fr.v. Johannes Strand(Silfvast), Johanna Löv, Anna-Sofia, Zaida, Vilhelm d.ä., Anna-Lovisa, Isak.}

%%%
% [occupant] Silfvast 4:73
%
\jhoccupant{Silfvast 4:73}{\jhname[Isak]{Silfvast 4:73, Isak} \& \jhname[Anna-Lovisa]{Silfvast 4:73, Anna-Lovisa}}{1902--\allowbreak 1936}
Isak, \textborn 11.01.1867, gift 18.01.1887 med Anna Lovisa Eliasdotter Forss, \textborn 07.06.1867.
\begin{jhchildren}
  \item \jhperson{\jhname[Elias Vilhelm]{Silfvast, Elias Vilhelm}}{1888}{}
  \item \jhperson{\jhname[Johannes]{Silfvast, Johannes}}{1890}{1930}, antog namnet Strand
  \item \jhperson{\jhname[Anna Sofia]{Silfvast, Anna Sofia}}{1892}{1934}
  \item \jhperson{\jhname[Zaida]{Silfvast, Zaida}}{1897}{1963}
  \item \jhperson{\jhname[Isak Vilhelm]{Silfvast, Isak Vilhelm}}{1903}{1903}
  \item \jhperson{\jhname[Isak Vilhelm]{Silfvast, Isak Vilhelm}}{1906}{1911}
  \item \jhperson{\jhname[Axel Joel]{Silfvast, Axel Joel}}{1908}{1935}
  \item \jhperson{\jhname[Lea]{Silfvast, Lea}}{1910}{1976}
  \item \jhperson{\jhbold{\jhname[Isak Vilhelm]{Silfvast, Isak Vilhelm}}}{1913}{1974}
\end{jhchildren}

Som så många andra reste Isak till Sydafrika för arbete i gruvorna och som så många andra ådrog han sig där ``minarsjukan''. Plågad av sin sjukdom dog han 1917. Han efterlämnade änkan Anna Lovisa, som tillsammans med hemmavarande barn skötte jordbruket med sina 7 kor.

Isak \textdied 15.12. 1917  ---  Anna Lovisa \textdied 09.03.1941


%%%
% [occupant] Silfvast 4:35
%
\jhoccupant{Silfvast 4:35}{\jhname[Henrik]{Silfvast 4:35, Henrik} \& \jhname[Anna]{Silfvast 4:35, Anna}}{1865--\allowbreak 1902}
Henrik Gustaf Johansson, \textborn 20.12.1834, från Pedersöre, Lepplax, gift 29.01.1859 med Anna Isaksdotter Silfvast, \textborn 14.09.1839.
\begin{jhchildren}
  \item \jhperson{\jhname[Johan]{Silfvast/Johansson, Johan}}{24.03.1859}{08.06.1910}
  \item \jhperson{\jhbold{\jhname[Isak]{Silfvast/Johansson, Isak}}}{11.01.1867}{15.12.1917}
  \item \jhperson{\jhname[Sanna]{Silfvast/Johansson, Sanna}}{25.04.1871}{30.01.1942}
  \item \jhperson{\jhname[Johanna]{Silfvast/Johansson, Johanna}}{13.06.1874}{14.11.1956}
\end{jhchildren}

Anna \textdied 02.06.1913  ---  Henrik \textdied 12.04.1909

Silvast hemman nr 356, 356a-b:
356; Här stod den gamla bostadsbyggnaden uppförd i medlet av 1800-talet. Den revs av Vilhelm Silfvast 1956.

356a; Här stod den gamla ladugården. Den revs 1952 för att bereda plats för den nya ladugården.

356b; Häststallet utgjordes av separat byggnad, som revs 1952.



%%%
% [house] Strand
%
\jhhouse{Strand}{4:214, tidigare 4:17}{Silvast}{5}{354}

Styckad av stomlägenhet Strand 4:17

\jhhousepic{Strand 4-214.jpg}{Fr.v. Lars o. Ellen Silvast, Terese Kiimalainen m. dotter Anna-Maija, Vilhelm S. med det aktuella huset i bakgrunden}

%%%
% [occupant] Strengell
%
\jhoccupant{Strengell}{\jhname[Oskar]{Strengell, Oskar} \& \jhname[Hellin]{Strengell, Hellin}}{1948--\allowbreak 1983}
Oskar Evald Strengell, \textborn 21.11.1910 på Jungar, gifte sig 03.08.1941 med Hellin Maria Strand, \textborn 05.03.1922 på Silvast. Makarna byggde huset i slutet av 1940-talet alldeles på gränsen till Handelslagets tomt. Innan byggnaden restes revs det gamla huset som var Hellins barndomshem. Efter kriget fick Oskar tjänst på Haldin \& Rose som chaufför och också som montör på verkstaden. Familjen bodde en tid i Jakobstad innan den flyttade till Jeppo och byggde huset. Rutten som kördes var Jeppo-Voltti-Jeppo. Bussen stod parkerad på gårdsplanen mellan turerna. Under tiden i Jeppo hade familjen två kor som Hellin skötte om. Oskar hade planer på att öppna en verkstad i anslutning till fastigheten, men nekades tillstånd. Familjen flyttade tillbaka till Jakobstad 1954 och huset hyrdes ut.
\begin{jhchildren}
  \item \jhperson{\jhname[Boris]{Strengell, Boris}}{25.12.1942}{}
  \item \jhperson{\jhname[Berit]{Strengell, Berit}}{18.07.1945}{}
\end{jhchildren}

Huset hyrdes av bl.a.:
\begin{enumerate}
  \item Paul och Gretel Nylind  1954--\allowbreak 1957
  \item Ingmar och Margit Sandvik 1957--\allowbreak 1960
  \item Ellen Orrholm 1960--\allowbreak 1963
  \item Jorma och Karin Kalijärvi 1963
  \item Adolfina och Juhani Rintala
  \item Juho Kustaa Aaltonen och Lydia Maria Jokinen
  \item Reijo Krooks   …..1983
\end{enumerate}
Andelsringen köpte fastigheten år 1983 och huset revs av Ernfrid och Sven Kronqvist.



%%%
% [oldhouse] Strand
%
\jholdhouse{Strand}{4:17}{Silvast}{5}{355}

Styckad av stomlägenhet Strand 4:17

%%%
% [occupant] Strand
%
\jhoccupant{Strand}{\jhname[Johannes]{Strand, Johannes} \& \jhname[Ida]{Strand, Ida}}{1908--\allowbreak 1947}
Johannes Silfvast, från 1922 med efternamnet Strand, \textborn 17.03.1890, gifte sig 12.04.1908 med Ida Maria Isaksdr. Liljeqvist, \textborn 20.10.1885. Johannes övertog denna del av hemmanets byggnader och en del jord i samband med giftermålet. Efter att barnen fötts reste han till USA och återvände inte. Han dog där 01.03.1930. Ida dog här hemma den 22.02.1947. Huset revs i slutet av 1940-talet.
\begin{jhchildren}
  \item \jhperson{\jhname[Sven]{Strand, Sven}}{08.10.1908}{04.05.1961 i Canada}
  \item \jhperson{\jhname[Svea]{Strand, Svea}}{13.06.1910}{}
  \item \jhperson{\jhname[Einar]{Strand, Einar}}{22.02.1912}{}
  \item \jhperson{\jhname[Elna]{Strand, Elna}}{20.11.1913}{1926}
  \item \jhperson{\jhname[Henry]{Strand, Henry}}{07.07.1920}{23.06.1944}, stupade
  \item \jhperson{\jhbold{\jhname[Hellin]{Strand, Hellin}}}{05.03.1922}{06.07.2008 i Jakobstad}
\end{jhchildren}


%%%
% [house] Handelslag
%
\jhhouse{Handelslag}{4:215}{Silvast}{5}{57}

Styckad av stomlägenhet Lindström 4:7

%%%
% [occupant] J \& S
%
\jhoccupant{J \& S}{Fastigheter}{2014--}
Den 31 mars 2014 köpte bröderna Jonas och Sebastian Cederström (se Ruotsala nr 20) fastigheten av Oy Hankkija, som ägs av Danish Agro Holding A/S. Deras firma är specialiserad på VVS-installationer och den tidigare butiksfastigheten har inretts och utnyttjas för företagets behov. De bostäder som finns på övre våningen hyrs ut. Likaså fortsätter Café Funkis sin verksamhet i den tidigare lokalen, nu med Jasmin Aalto som hyresgäst sedan 2016.


\jhhousepic{030-05561.jpg}{Länge centrum för handel och annan verksamhet i Jeppo}

%%%
% [occupant] Hankkija
%
\jhoccupant{Hankkija}{m.fl.}{1985--\allowbreak 2014}
Efter en sällsynt turbulent tid på ägarfronten såldes fastigheten som ovan nämnts till J \& S Fastigheter 2014. Den egentliga affärsverksamheten hade året dessförinnan upphört i praktiken i Agrimarkets namn, ägt av Oy Hankkija. Agrimarket hade uppstått efter att centrallaget Hankkija gått i konkurs 1992 och ett nybildat Oy Hankkija bildats som under marknadsnamnet Agrimarket fortsatte lantbrukshandeln i 73 butiker av olika storlek, varav butiken i Jeppo var en. Butiken i Jeppo var tidvis bland de bästa vad lönsamhet beträffar, men andra överväganden styrde besluten.

År 1997 uppstod problem med bensinförsäljningen då cisternerna blev granskade. Den 25 sept. utförde Pieksämäen Pensseli granskningen och cisternerna blev utdömda. Ernfrid Kronqvist och Ove Strand beställde på eget bevåg en ny 16 m³ cistern med mellanvägg, grävde ner den och fyllde de gamla med sand efter rengöring. Trots att man var medveten om att investeringen inte skulle bli lönsam, räddades bränslefrågan för tillfället. Arbetet var klart 28 okt.

År 2013 övertar Danish Agro Holding A/S 60\% av aktierna i Oy Hankkija  och nedmonteringen börjar. Protester och telefonsamtal från Jeppo till Danmark har ingen verkan. Besluten fattades på annat håll och effekten har blivit att mycket få lantbrukare fortsatt sin handel med de nya ägarna, som flytt orten.


%%%
% [occupant] Botnica-Waasko
%
\jhoccupant{Botnica-Waasko}{KPO – Sale}{1985--\allowbreak 2006}
Tiden före 1992 hade också varit omvälvande. 1987 hade Waasko handelslag sträckt ut sina tentakler och övertagit Handelslaget Botnica som uppstått genom en fusion mellan Andelsringen, Pedersöre Hlg, Esse Hlg och Terjärv Hlg år 1985 i ett försök att konsolidera den finlandssvenska andelsrörelsens ställning på marknaden. Tveksamhet inför processen präglade Andelsringens beslutande organ och det krävdes 4 beslut innan samgångsplanerna godkändes och Botnica bildades fr.o.m. 1985. Då hade lagerreserven redan utnyttjats. Det lyckades inte trots samgången och KPO övertar småningom dagligvaruhandeln som i namn av Sale-kedjan fortsätter verksamheten i fastigheten i Silvast till år 2006, varefter Salebutiken flyttar till korsningen riksväg 19/Pensalavägen.

\jhpic[pic:centrum-60]{Jeppo centrum 1960.jpg}{Museiverket, foto M. Poutavaara}

Under Waaskos tid var Håkan Smeds Vd för handelslaget och flera saneringsåtgärder vidtogs för att stärka verksamheten. Handeln i Jeppo uppges gå bra 1987/88 efter sammanslagningen av lantbrukshandeln mellan Hankkija och SOK. Omsättningen stiger men lönsamheten försämras på nytt. Textilsidan läggs ner och de friställda utrymmena hyrs 1988 ut till Föreningsbanken i Finland och 1989 såldes bl.a. spannmålsmagasinet till nystartade Jeppo Food.

Ny hyresgäst sedan både Posten och Föreningsbanken flyttat bort, blir Gunilla Nyman, som år 1990 här öppnar en textil/ blomsteraffär, vilken övertas av Regina Dahlström, gift Kujala. Den fortsätter i hennes regi tills den läggs ner efter några år.


%%%
% [occupant] Andelsringen
%
\jhoccupant{Andelsringen}{Alg}{1960--\allowbreak 1985}
Den 3 juni 1959 fattade stämman för Nykarleby Handelslag  beslutet om att fusionera in i Jeppo-Oravais Handelslag. Fusionen trädde i kraft den 1 januari 1960 och efter en namntävling där ca 500 namnförslag lämnades in, stannade man för namnet \jhbold{Andelsringen}.

Under 1962 startade en omfattande renovering av huvudaffären i Silvast och den 10 maj 1963 öppnades dörrarna till nyrenoverade utrymmen. Denna period specialiserades handeln allt mer och indelades i ansvarsområden. Bo-Göran Backman var ansvarig för maskinhandeln och Ingvar Karf för handeln med gödsel,kalk,utsäde,byggnadsmaterial m.m. Vattentäta skott fanns naturligtvis inte. Ernfrid Kronqvist kom småningom in i bilden och blev en trotjänare inom företaget. Paul Granholm hade kommit 1952 och ansvarade för livsmedelssidan och speciellt köttdisken. Han efterträddes av Jan-Erik Nybyggar.

Den 1 dec. 1981 flyttar \jhbold{Posten} in i Andelsringens lokaler för en tid.


%%%
% [occupant] Jeppo-Oravais
%
\jhoccupant{Jeppo-Oravais}{Handelslag}{1927--\allowbreak 1960}
Centrallaget för Handelslagen i Finland hade under en tid med oro följt Oravais Handelslags ekonomiska utveckling och 1926 framställde Centrallaget ett förslag om att Oravais Handelslag och Jeppo dito skulle gå samman. Vid stämmorna vid årsslutet beslöts att Oravais Handelslag skulle uppgå i Jeppo Handelslag och fr.o.m 1 januari 1927 blev det gemensamma Jeppo-Oravais Handelslag ett av de större handelslagen i svenska Österbotten. Handelslaget var mycket expansivt inom sitt område och flera filialer öppnades eller inköptes: Jungar, Karvat, Komossa, Pensala,Ytterjeppo, Ruths, och Kimo.

\jhpic{Jeppo-O Handelslag.jpg}{Gamla Jeppo-Oravais Handelslag och närliggande byggnader}

1930-talets globala depression var också tydligt märkbar här på lokalnivå och gjorde verksamheten problematisk för allt handlande, men i slutet av -30 talet började ett kraftigt uppsving som gjorde att styrelsen 1939 började fundera på att bygga en ny huvudaffär. Den 24 mars inköptes en tomt väster om korsningen till Stationsvägen av bonden Anders Lindén för ett pris om 70 000 mk. Ritningar skaffades från Centrallagets byggnadsavdelning, men då vinterkriget bröt ut den 30.11 stoppade processen. Först våren 1941, under mellankrigstiden, startade nu byggandet under byggmästare Georg Romars ledning. Fortsättningskriget som startade 25 juni 1941 innebar ett nytt hinder och arbetet gick i stå då männen ryckte in i armén. På något sätt lyckades ändå Georg Romar få igång byggandet igen och trots stora svårigheter kunde bygget färdigställas under våren 1943 och vara klart för inflyttning före midsommar. Nu invigdes också Café Funkis, som fått sitt namn efter den stil som introducerats av Alvar Aalto i och med byggandet av stadsbiblioteket i Viborg.

\jhhousepic{Andelsringens butiksbil 1964}{Maj-Len Sunngren, Gretel Andersson g. Kronqvist och Stig Byman vid Andelsringens butiksbil 1964}
År 1947 byggdes ett spannmålslagerhus vid Jeppo Station för att 1956 utvidgas med en magasinbyggnad i två våningar. I slutet av samma år avlider handelslagets styrelseordförande sedan 22 år, Leander Finskas. 1959 anskaffas handelslagets första butiksbil. Samma år byggdes ett nytt garage för den utökade lastbilsparken och här fick också butiksbilen rum. Garagets övre våning inrymde 4 bostadslokaler för personalen. Läs mera om denna byggnad under \jhbold{nr 59}.

Under denna tid har följande personer fungerat som föreståndare:

\begin{center}
\begin{tabular}{l l}
  Hugo Nyman       & 1923--\allowbreak 1930 \\
  Vilhelm Renvall  & 1931--\allowbreak 1934 \\
  Konrad Perus     & 1935--\allowbreak 1938 \\
  Edvin Wistbacka  & 1938--\allowbreak 1942 \\
  Elis Fagerholm   & 1943--\allowbreak 1947 \\
  Sven Häll        & 1947--\allowbreak 1971 (Andelsringen från 1960) \\
  John Helsing     & 1971--\allowbreak 1978 \\
  Bo-Göran Backman & 1978--\allowbreak 1979 \\
  Håkan Smeds      & 1979--\allowbreak 1981 \\
  Bengt Sandvik    & 1981--\allowbreak 1984 \\
\end{tabular}
\end{center}

Styrelseordförande/förvaltningsrådsordf. från 1956 var: Runar Nyholm 1956--59 och Jarl Romar 1960--85.

\jhpic{Handelslagspersonal 1940-tal.jpg}{Personalfoto sent 1940-tal utanför huvudingången: 1. Valdemar Johansson chaufför, 2. Edvin Jungell, 3. Elise Romar(Bergman), 4. Ester Källman, 5. Gun-Britt Johansson(Sandström), 6. Gustav Johansson, 7. Lars Lindberg, 8. Ingvar Karf, 9. Valdemar Back, 10. Ivar Westin, 11. Ernst Pensar, 12. Paul Sandström, 13. Lars Finne, 14. Svea Grahn?, 15. Vera Nyholm, Kantlax, 16. Melita Wikström}

På huvudaffärens 2:a våning finns bostäder som uthyrts till personalen under åren lopp. Här finns lokalen för föreståndaren som inhyst Elis Fagerholms familj 1943--\allowbreak 1947, Sven Hälls familj 1947--\allowbreak 1971, John Helsings familj 1971--\allowbreak 1978, Håkan Smeds familj 1979--\allowbreak 1981 och Bo-Göran Backmans familj 1978--\allowbreak 1979 och 1981--\allowbreak 1995. Sedan 1995 har Erik och Denice Norrgård bott i lägenheten fram till 2016.

I de andra lokalerna har bott expediter, chaufförer och annan personal genom åren, bl.a. Anita Lindholm (g. Häggstrand), Oili Broo, Gretel Andersson (gift Kronqvist), Birgitta Nylund (gift Back), Magda Simons, Bo-Göran Backman, Valdemar Johanssons familj. Paul Granholm, Volmar och Benita Strand, Magda Lindström från Oravais, m.fl., m.fl.

\jhperson{Häll, Sven Andreas}{15.11.1911}{07.04.1971}, kom från Esse den 25.04.1934 som packmästare till Jeppo Hlg. År 1947 blev han VD för Jeppo-Oravais Hlg och kvarstod till sin förtidiga död 1971. Under hans tid som VD genomgick Handelslaget stora förändringar och utvidgades bl.a. via fusioner med Nykarleby Hlg 1960, varvid nya namnet Andelsringen antogs. Kontoret i Jeppo förstorades med en ny flygel. \jhname[Sven]{Häll, Sven} var engagerad i föreningslivet och trakterade fiol.

I maj 1948 gifte han sig med fil.kand. \jhperson{Stenholm f. Lönnqvist, Werna Carita Maria}{04.04.1912}{05.01.1992}, i Jeppo. Hon var änka efter lektor Torsten Viking Stenholm, \textborn 21.08.1907, som stupade i vinterkriget 19.02.1940. Deras son Stig Stenholm, \textborn 26.02.1939 fick framgångsrik forskarkarriär som fysiker, vilket belyses med att han 1978 valdes in i Finska vetenskapssocieteten, 1992--97 var han särskild forskningsprofessor vid Finlands Akademi, 1997 prof. i laser- och kvantfysik och 1999 utländsk ledamot i svenska Vetenskapsakademin.

\jhname[Verna]{Häll, Verna} hade tjänst som lektor vid Nykarleby Samskola. Hon var också en av initiativtagarna till uppstarten av Jeppo Hembygdsförening. Tillsammans med Sven fick hon tre barn.
\begin{jhchildren}
  \item \jhperson{\jhname[Lars-Erik Andreas]{Häll, Lars-Erik}}{10.09.1949}{}, dipl.ing. 1974, engagerad inom kärnkraftsindustrin, mera än 30 år hos TVO och involverad i Olkiluoto 1, 2 och 3, pensionär, Helsingfors
  \item \jhperson{\jhname[Gunilla Solveig Maria]{Häll, Gunilla}}{1953}{}, pensionär, Holmsund
  \item \jhperson{\jhname[Birgitta Margareta]{Häll, Birgitta}}{1959}{}, specialtandläkare, Vanda
\end{jhchildren}



%%%
% [oldhouse] Det gamla handelsslagshuset i vägkorsningen till Jeppo station
%
\jholdhouse{Det gamla handelsslagshuset i vägkorsningen till Jeppo station}{4:57}{Silvast}{5}{357}

Styckad av stomlägenhet Lindfors 4:14

\jhhousepic{Jeppo Hlg mbt.jpg}{Butik med bränslemack och hästbom.}

%%%
% [occupant] Jeppo Handelslag
%
\jhoccupant{Jeppo Handelslag}{\jhname[m.b.t.]{Jeppo Handelslag, m.b.t.}}{1880--\allowbreak 1960}
Huset byggdes på 1880-talet av handlanden Gustav Julin. Han hade köpt timran från ett hus på Jungar och byggde nu upp det i korsningen. År 1890 förstorades tomten genom att på 50 år arrendera 4 kappland (600 m²) av Jakob Sjöblom och Anders Björklund, men redan 1891 hade han upphört med verksamheten. Den 26 november 1891 hölls en konstituerande stämma på Jungar folkskola i akt och mening att bilda ett handelsaktiebolag. Detta blev också fallet och man beslöt att köpa Gustav Julins gård för 3000 mk. Den skulle bli bolagets butik och huvudkontor. I direktionen satt John Svedberg som ordf. och medlemmar var bönderna Vilhelm Sarelin, Johan Romar, Johan Rundt och bondsonen Johan Jungar. Den 1 januari 1892 öppnades affären, men lönsamheten var inte den önskade. Julin som kvarblivit i tjänst i det nya bolaget sade nu upp sin tjänst och redan ett år efter starten såg insatskapitalet ut att vara förbrukat och den 29.12.1893 diskuterade man om bolaget skulle upplösas. Trots en tillfällig optimism beslöt man den 22 jan. 1894 att butiken var till salu och den 30 april hade handlanden Isak Liljeqvists anbud om 3025 mk godkänts.

I denna byggnad bedrev Liljeqvist handel till 28 mars 1913 då han sålde den till Emil Ström för 8500 mk. Liljeqvist period blev 19 år.

Under tiden hade prof. Hannes Gebhards idé om andelslag börjat mogna också i Jeppo och 19 jan. 1914 hade Jeppo Handelslag bildats och startat sin verksamhet i den gamla gillesboden med J.T.Backlund som förste föreståndare. Butiken var inte varmare än en vanlig lada och den var för liten, fastän man 1915 förstorat den något med en enkel tillbyggnad av bräder.

Plötsligt infann sej en öppning i utrymmesfrågan då Emil Ström den 17 mars 1916 erbjöd sej att sälja sin handelslokal inklusive alla inventarier för 12000 mk. Erbjudandet godkändes och det hus som Gustav Julin köpt från Jungar och timrat upp på 1880-talet fick nu sin 5:e ägare. År 1921 förstorades huset västerut med en affärslokal och en skylt med texten ``Jeppo Handelslag m.b.t.'' sattes upp. Denna affärslokal skulle tjäna Jeppoborna tills den nya huvudbutiken blev klar 1943, dock med en förändring av namnet till Jeppo – Oravais Handelslag m.b.t. från och med  1927.

\jhpic{Handelslagspersonal 1925.jpg}{Interiör från Jeppo Handelslag 1925--1926 med anställda: \jhname[Hellin Jungarå (Sandberg)]{Jungarå, Hellin}, \jhname[Hugo Nyman]{Nyman, Hugo} förest., \jhname[Vilhelm Renvall]{Renvall, Vilhelm}, \jhname[Reinhold Lithén]{Lithén, Reinhold}, \jhname[x Karlsson]{Karlsson, x} och \jhname[Lauri Enlund]{Enlund, Lauri}.}

Efter 1943 användes byggnaden främst som lagerlokal, men också som bostad åt bl.a. Ingvar Karfs familj, Bo-Göran Backmans familj och en tid åt Paul Granholm och Lars Lindberg.

I samband med att nytt garage skulle byggas med början 1959 avlutades denna byggnads historia, men på fotografier från år 1960 kan man ännu se den karakteristiska byggnaden. Den revs 1965.

Affärsföreståndare medan handelslaget verkade i denna fastighet har varit:
\begin{center}
  \begin{tabular}{l l l l}
    J.T. Backlund & 08.03.1914--05.12.1916 & Gustav Liljeqvist & 01.01.1919--31.07.1923 \\
    Martin Ahlnäs & 06.12.1916--31.12.1916 & Hugo Nyman        & 01.08.1923--31.12.1930 \\
    Frithjof Varg & 07.01.1917--05.02.1917 & Vilhelm Renvall   & 01.01.1931--31.12.1934 \\
    A.E. Sandvik  & 07.03.1917--15.05.1917 & Konrad Perus      & 01.01.1935--31.05.1938 \\
    Emil Ekström  & 12.06.1917--30.11.1918 & Edvin Wistbacka   & 01.06.1938--31.10.1942 \\
  \end{tabular}
\end{center}

Vi finner de täta bytena på föreståndarposten anmärkningsvärda, speciellt i starten. Kanhända är de ett tecken på de svårigheter en ny handelsform stod inför, men är med säkerhet också ett bevis på det kärva klimat ett samhälle utan stora resurser levde under.

Styrelseordföranden under denna tid var:

\begin{center}
\begin{tabular}{l l}
  \jhname[Johan Jungar]{Jungar, Johan}       & 1914--21 \\
  \jhname[Johan Romar]{Romar, Johan}         & 1921--31 \\
  \jhname[Viktor Smulter]{Smulter, Viktor}   & 1932 \\
  \jhname[Gunnar Wadström]{Wadström, Gunnar} & 1933 \\
  \jhname[Leander Finskas]{Finskas, Leander} & 1934--56 \\
\end{tabular}
\end{center}



%%%
% [house] Lindfors
%
\jhhouse{Lindfors}{4:224}{Silvast}{5}{350}

Styckad av stomlägenhet Lindfors 4:14

%%%
% [occupant] Källman
%
\jhoccupant{Källman}{\jhname[Matts]{Källman, Matts} \& \jhname[Hilda]{Källman, Hilda}}{1905-ca 1973}
Matts Källman, \textborn 10.04.1879, gift 07.12.1903 med Hilda Almberg, \textborn 04.04.1881. Huset byggdes efter giftermålet och det berättas att Hilda, efter att första stockvarvet placerats ut på stenfoten, ställde sej mitt i den påbörjade byggnaden och utbrast: ``Ja tycker at he kenns mytji varmari rej!'' Hilda var på många sätt en speciell och stridbar person. Paret hade två kor som Hilda sommartid föste iväg till hagen som fanns där Konditionshallen står idag. Ännu länge efter att Matts dött, fortsatte hon med sin djurhållning ända tills krafterna svek.

När mejeriet, som startat vid Böös 1902, skulle flytta sin verksamhet till Silvast, byggdes nya utrymmen på samma område där redan Silvast kvarn och såg verkade nere vid forsen. Den 21 nov. 1913 stod mejeriet klart att tas i användning och här kom Matts Källman att fungera som maskinist fram tills det nya andelsmejeriet, som ännu står kvar, togs i bruk i februari 1933.

Hilda och Matts var barnlösa. Huset revs i samband med att vägen genom Silvast grundförbättrades i början av 1970-talet.

Matts \textdied 15.08.1937  ---  Hilda \textdied 17.10.1968



%%%
% [house] Lindfors
%
\jhhouse{Lindfors}{4:122}{Silvast}{5}{359}

Styckad av stomlägenhet Lindfors 4:14

%%%
% [occupant] Andelsringen
%
\jhoccupant{Andelsringen}{\jhname[Hlg]{Andelsringen, Hlg}}{1965--\allowbreak 1980}
Andelsringen köpte fastigheten i medlet av 1960-talet och hyrdes ut till Helge Nygård (fr. Baggas) 1968--69. Efter det har huset mest använts som lager för olika produkter innan det revs på 1980-talet.


%%%
% [occupant] Norrgård
%
\jhoccupant{Norrgård}{\jhname[Einar]{Norrgård, Einar} \& \jhname[Märta]{Norrgård, Märta}}{1945--\allowbreak 1954}
Einar Norrgård, \textborn 27.03.1915, gifte sig med Märta Liljeqvist, \textborn 07.05.1922. Han var sysselsatt inom handelslaget som butiksbiträde. Einar kom i slutet av 1939 från Munsala och bosatte sej på Back och fungerade som kontorist på Keppo innan han flyttade till Silvast. År 1942 gifte han sej med Märta och 6 dec 1945 flyttade familjen till Oravais. Han köpte fastigheten 4 aug. 1945 av Isak Lindfors sondotter Agnes Westerlund. 1953 återkom familjen denna gång från Helsinge och flyttade 1954 till Korsholm.
\begin{jhchildren}
  \item \jhperson{\jhname[Gundel Alice]{Norrgård, Gundel Alice}}{1942}{}
  \item \jhperson{\jhname[Gun Kerstin Elisabeth]{Norrgård, Gun Kerstin Elisabeth}}{1947}{}
  \item \jhperson{\jhname[Bjarne Gustav Johannes]{Norrgård, Bjarne Gustav Johannes}}{1953}{}
\end{jhchildren}

Fastigheten såldes till Andelsringen i medlet av 1960-talet, men innan dess var den uthyrd i olika perioder från att familjen Norrgård flyttat bort 1954. Bl.a: Paul och Gretel Nylinds fam. 1957-60, Ruben och Ragnborg Nygårds fam. 1961-63, Lars Finne, \textborn 14.11.1926 i Jakobstad, med modern Laina Lilius, \textborn 01.09.1908 i Jakobstad, och systern Lisbet Marianne, \textborn 16.05.1928 i Markby. Lars var butiksföreståndare och Laina var kaféinnehavare.

Ernst Pensar, \textborn 02.01.1917 i Munsala, kom 21.01.1949 från Oravais. Han var gift med Edith Josefin Nordberg, \textborn 17.02.1922. Ernst var vice Vd på handelslaget, familjen flyttade 27.12.1949 till Munsala. Köttmästaren Paul Granholms familj bodde i huset 1952--\allowbreak 1956  (se handelslaget).


%%%
% [occupant] Lindfors
%
\jhoccupant{Lindfors}{\jhname[Isak]{Lindfors, Isak} \& \jhname[Sofia]{Lindfors, Sofia}}{-1953}
Isak Jakobsson Lindfors, \textborn 08.08.1862, kom 1910 från Ytterjeppo och gifte sig med Sofia Johansdr., \textborn 08.05.1860. Isak blev senare känd som ``Billnäs-Iikka''.

På Handelslagets vårstämma 1920 godkändes inköp av Lindfors hemman och den 6 maj såldes mindre lotter av detta för 12050mk.

Isak arbetade sannolikt på stationens vedplan. Om han också varit dräng eller timmerman är oklart. Huset som rymde familjen byggdes på östra sidan om landsvägen men på västra ändan av den tomt där senare Andelsringens garagebyggnad placerades. Husgaveln syns i Museiverkets bild \ref{pic:centrum-60} från 1960 och även t.v. på bilden av gamla Jeppo Handelslag m.b.t.,ovan.
\begin{jhchildren}
  \item \jhperson{\jhname[Anna Lovisa]{Lindfors, Anna Lovisa}}{19.10.1888}{}
  \item \jhperson{\jhname[Isak Alfred]{Lindfors, Isak Alfred}}{27.09.1890}{}
  \item \jhperson{\jhname[Anders Emil]{Lindfors, Anders Emil}}{22.06.1892}{}
\end{jhchildren}

Isak Alfred Lindfors, \textborn 27.09.1890,  växte upp som det mellersta barnet i Isak och Sofias familj (se ovan). Han gifte sej med Sanna Emelia Isaksdr. Liljeqvist, \textborn 19.05.1888. Paret fick en dotter, Agnes Emelia, \textborn 21.01.1914. När Agnes var drygt ett år dog hennes mor Sanna, den 02.05.1915. Fadern Isak Alfred lämnade sin dotter i sina föräldrars vård och reste 28.10.1915 till USA. Hans kontakt med hemlandet skulle bli liten.

Agnes växte upp med sina farföräldrar; Billnäs-Iikka och Sofia. Hon gick i ``Sparvback'' skola (Jungar skola) och visade tidigt intresse för skådespel och teater. Hon deltog i den teatergrupp som upprätthölls av Ungdomsföreningen. Vid 20 års ålder arbetade hon som expedit på Ströms butik på Stenbacken och den här kvällen 1934 skulle teatergruppen ha en föreställning. Hon hade varit med och övat och hon var inte nervös. En enda replik skulle hon ha: ``-Min far kommer hem ikväll''.

Precis innan stängningsdags kommer en kund in i butiken och berättar att det kommit en kappsäck till stationen med namnet Isak Lindfors. Den hade sänts från Amerika. ``-Du måste fara hem genast, Agnes. Kanske din pappa kommer hem!''.  Men Agnes hade ingen brådska. Ingen visste ju om kappsäcken och dess ägare följdes åt. Hon förberedde sej som vanligt inför föreställningen och deltog i den, men när hennes uppgift var över föstes hon iväg till stationen av sina vänner: ``-Snälltåget kommer ju snart!'' Hon gick iväg till stationen och stod och gömde sej mellan träden som växte i den lilla parken mellan stationshuset och godsterminalen.

Småningom anlände tåget och bromsade in. Ut vällde människor; jeppobor, munsalabor, oravaisbor och nykarlebybor, men vem var hennes pappa? Folk började försvinna från perrongen när hon såg en ensam man, som stigit ur en av de bakre tågvagnarna, komma gående längs träplattformen som hörde till godsterminalen. Han gick förbi henne där hon stod bakom ett träd och gömde sej och han såg henne inte. Litet på avstånd följde hon efter i skydd av träden och hon såg hur mannen gick fram till stinsen Hilding Hästbacka och började prata med denne. Nu vågade hon sej ut ur skuggorna och när stinsen fick syn på henne pekade han på henne och sa åt mannen: ``Titta Isak, där har du din dotter Agnes!''

Kanske hade Isak tänkt stanna hemma, ingen vet, men efter en tid märktes hans oro. Han började prata om att åka tillbaka till Amerika och en dag lät han Agnes förstå, att om hon ville kunde hon följa med honom. Det blev för henne en vånda; skulle hon stanna hemma, där hon var van att leva, eller skulle hon följa med sin pappa till okända öden. Hon frågade sina farföräldrar om råd, de som tagit hand om henne sedan hon var alldeles liten. Men de ville inte ge något annat råd än att hon skulle göra det som hon  ansåg vara bäst för henne själv. Efter en tid av eftertanke ger hon sin pappa besked: -``Pappa, jag kommer inte med dej tillbaka till Amerika. Jag kommer aldrig att fara till Amerika!''

Agnes tar en kort paus i berättandet där hon 1998 sitter på sin terass på Merloy Ave. Corvallis, Oregon. Hon slår ut med händerna:-``Och här har jag nu varit nästan halva livet! Tänk vad litet man vet om framtiden''. Hon skrattar. Det ville sej nämligen så, att Agnes 1938 gifte sej med Ernst Erik Westerlund, \textborn 15.11.1913 på Fors skattehemman.

Efter kriget beslöt de att flytta ut från Silvast och bosätta sej öster om järnvägen (se nr 127). Där byggdes deras nya bostad och ladugård och livet stabiliserades. Men någon gång efter medlet av 1950-talet började locktoner från USA göra sej påminda och 1958 arrenderar de ut hemmanet med fastigheter och reser till USA för att inte komma tillbaka för annat än besök och för att sälja sin egendom. Agnes återser sin far innan han dör 15.02.1969.

Deras 3 döttrar: Greta, \textborn 1942, Gunnevi, \textborn 1947, och Gun-Britt, \textborn 1949, är alla gifta med amerikaner och har skapat sin framtid i detta stora land. Sin begåvning för skådespeleri fick Agnes också utöva i radion. Läraren Olli Thors, son till Ellen och Fredrik Thors, hade lovat att som hörspel i radion framföra en skröna om Ytterjeppo föreningstjur. Men när det var så dags, blev han inkallad till kriget och den enda han tyckte var kompetent nog att framföra det i hans ställe var Agnes, och det gjorde hon.

Ernst \textdied 13.12.1992  ---  Agnes \textdied 19.11.2001 i USA. Deras stoft vilar på begravningsplatsen i Jeppo.



%%%
% [house] Handelslag, inkl. boende
%
\jhhouse{Handelslag, inkl. boende}{4:57}{Silvast}{5}{59}

Styckad av stomlägenhet Lindfors 4:14. Lägenheten omfattar även Lindfors 4:122 och Källman 4:149.


\jhhousepic{028-05560.jpg}{Garagefastigheten}

%%%
% [occupant] Norrgård Kim
%
\jhoccupant{Norrgård Kim}{\jhname[Nygård Bo]{Norrgård Kim, Nygård Bo}}{2009--}
Den 01.04.2009 köpte Bo-Erik Nygård och Kim Norrgård garagefastigheten Handelslag 4:57 med tilläggstomten Lindfors 4:122. Bo-Erik Johannes Nygård, \textborn 18.08.1945 i Lassila by, har arbetat som chaufför och transportledare. Kim Nikolai Johannes Norrgård, \textborn 18.08.1978 i Vasa, är son till Bo-Erik och hans f.d. sambo Gunnevi Norrgård, \textborn 08.02.1950. Huset är byggt 1959--\allowbreak 1960, nedre våningen är helt i betong, med sju garage, värmecentral m.m. på 240 m2 samt den övre våningen består av fyra 60 kvadratmeters bostäder. Nedre våningen byggdes delvis om till bil- och däckverkstad, med kontor, VC och dusch. Övre våningen är av trästomme och hela huset har fasadtegel. Kim driver firman K.N. Bil och Däck, som utför bilservice och reparationer samt försäljning och montering av däck med däckhotell.

Hösten 2010 flyttade Kim in i lokal 2. Lokal 1 renoverades och Kim flyttade 2011 till den.

\jhsubsection{Hyresgäster i bostäder 2009--}

\begin{center}
\begin{tabular}{ll}
  \jhbold{Lokal 2}  & \\
  Ståhlberg Heidi   & 01.07.2015-- \\
  Jessika Ab        & våren 2015 för tillfälliga estländare \\
  Aalto Tiina       & 01.11.2012-01.02.2014 \\
  Häggman Johan     & hösten 2011-01.10.2012 \\
  \jhbold{Lokal 3}  & \\
  Thachenko Nikolai & 15.08.2011--; från Ukraina	med fru och dotter \\
  Gennadiev Pavlo   & 01.01.2011-15.08.2011; från Ukraina	med fru \\
  Silfvast Ronny    & 01.04.2009-01.05.2010 (01.11.2003, föreg.ägare) \\
  \jhbold{Lokal 4}  & \\
  Gennadiev Pavlo   & 15.08.2011-15.05.2013; till eget hus i Härmä med familj \\
  Sandvik Alice     & 01.04.2009-30.03.2011, (01.07.1993 föreg.ägare) \\
\end{tabular}
\end{center}

2011 köpte Kim Norrgård granntomten från vägen till bäcken, Källman 4:149, då med register nummer 4:224. Säljare  B-G. Finskas. Se mera nr 350.


%%%
% [occupant] Sandvik Bengt
%
\jhoccupant{Sandvik Bengt}{\jhname[Rajakangas Ulla]{Sandvik Bengt, Rajakangas Ulla}}{1991--\allowbreak 2009}
Bengt Holger Sandvik, \textborn 19.10.1956 i Purmo, och hans 	sambo	Ulla Maija Rajakangas, \textborn 05.05.1953 i Alahärmä, köpte 15.11.1991, fastigheterna 4:57 och 4:122. De flyttade in i lokal 2.	Bengt och Ulla arbetade på Westwood snickeri i Nykarleby. Efter ca 30 år i snickeribranschen övergick Bengt 2002 till jord- och skogsarbeten på Lövbacka Ab, karta 9, nr 123. Ulla arbetar fortfarande (2017) på snickeriet.

Bengt och Ulla vigdes 1999. Sommaren 2003 förenade de lokal 1 och 2 för eget bruk. Ulla är intresserad av trädgårdsarbeten och tillsammans med sin svärmor Alice Sandvik planterade de häckar, träd och buskar samt blomrabatter och åstadkom en vacker gårdsplan. Hösten 2010 flyttade Bengt och Ulla till Ullas barndomshem i Forsby, Nykarleby.

\jhsubsubsection{Hyresgäster}

Vid fastighetsköpet 1991 var garagen uthyrda till ACW-Mekaniska KB, Christer Wikblom, som hösten 1992 flyttade till Nykarleby. Laattakoski KY, J.Jaskar hyrde ett av garagen 01.02.1992 till hösten samma år. Den 15.11.1991 var 1, 3 och 4 uthyrda.

\begin{longtable}{p{0.14\textwidth} p{0.8\textwidth}}
  \jhbold{Lokal 1} & \\
  1997--\allowbreak 2003 & Långtradarchaufför Karl-Erik Haglund, \textborn 30.05.1951 på Skog hemman, hyrde lokal 1 den 01.10,1997. Han växte upp på Villa 2:38 på Romar hemman nr 30. I slutet av juni 2003 köpte han Pelmas fastighet Svinvallen II 21-0, nr 86. \\
  1990--\allowbreak 1997 & Ellen Katarina Lillas, \textborn 22.8.1915, född Thomsson på Lussi, flyttade till Silvast 1990, då sonen John och sonsonen Magnus Lillas övertog Sundells bageriverksamhet. Trapporna blev betungande och 31.03.1997	flyttade hon till Nykarleby, där hon dog 12.12.1997. \\
  \jhbold{Lokal 3} & \\
  2003--\allowbreak 2010 & Ronny Lars Vilhelm Silvast, \textborn 16.04.1974 på Silvast hemman, hyrde	01.11.2003 lokal 3. Ronny är metallarbetare och arbetade på Eurmark i Nykarleby. Han flyttade till eget hus 01.05.2010, Åbacken 3:49, se karta 4, nr 43. \\
  2000--\allowbreak 2002 & Den 01.10.2000 hyrde Ralf Kangas lokal 3. Han flyttade ut 2002. \\
  1998--\allowbreak 1999 & Den 01.09.1998 hyrdes lokal 3 av Christoffer Lindén, \textborn 1980 på	Holmen 4:18 av Silvast hemman. Christoffer flyttade ut 01.08.1999. \\
  1995--\allowbreak 1998 & Mats Göran Lindborg, född 22.05.1969 på Nybyggar hemman och hans	sambo, hälsovårdare Maria Beatrice Lillbroända, \textborn 29.08.1971 i Kronoby, hyrde lokal 3 den 07.04.1995. Mats var utbildad jordbrukstekniker från Lannäslund. Senare har han studerat till maskinritare. Maria är barnmorska och hälsovårdare och arbetar i dag, 	2017, som företagshälsovårdare i Nykarleby. Mats och Maria vigdes 05.04.1997. De flyttade 30.04.1998 till Nykarleby Bostäder, Jeppo-stugan på Åkervägen, där de bodde när deras första barn Erik föddes 06.06.1998. De flyttade till sitt nya hus på Blåmans, Ojala, 30.04.2000. \\
  1991--\allowbreak 1994 & Kenneth Norrgård, \textborn 19.06.1966 på Keppo, och hans sambo Elisa	Korkiakangas, \textborn 15.06.1965 i Toholampi, hyrde i januari 1991 lokal 3. Kenneth och Elisa vigdes 1992. De två äldsta barnen föddes när de bodde i lokal 3. Ann-Sofie Maria	\textborn 01.04.1991, merkonom,	Dan-Otto Håkan	\textborn 29.01.1993, idrottsledare. Kenneth arbetade som elementmontör på Myresjö Hus i Oravais. Elisa	vikarierade Ann-Christine Dahlqvist som damfrisör från oktober 1991--\allowbreak 1992. Familjen flyttade 30.09.1994 till sitt nybyggda hem, Villa Camilla, på Mietala. \\
  1 maj 1986--\allowbreak 1 maj 1989 & Leif-Ole Romar och Kristina Blomberg bodde här med sin son Tobias, \textborn 1988. Se även Romar nr 14. \\
  \jhbold{Lokal 4} & \\
  1993--\allowbreak 2011 & Alice Sandvik, \textborn 11.06.1930 i Purmo, hyrde lokal 4 den 01.07.1993 efter att sonen Sten med sambo Kerstin flyttat till Jakobstad. Alice sjöng under många år i kyrkokören. Hennes stora intresse var att plantera och sköta om trädgården. Alices och familjens liv beskrivs i nr 127 på Fors hemman. Alice flyttade till Nykarleby bostäder (Pastellen) på	Åkervägen den 30.03.2011. \\
  1988--\allowbreak 1993 & Sten Johan Sandvik, \textborn 09.07.1958 på Fors hemman, nr 127, och hans sambo Kerstin Elisabet Ljung, \textborn 13.06.1964 på Skog hemman, nr 1, hyrde lokal 4 1988. Sten och Kerstin gifte sig 12.11.1994. Sten arbetade 1974--\allowbreak 1991 på Sjöholms farm i Ytterjeppo och från 02.04.1992 på Oy KWH-Mirka Ab. Kerstin var vikarerande klasslärare i Nykarleby 1988--\allowbreak 1991 och i Larsmo 1992--\allowbreak 1993, speciallärare i Karleby 1993--\allowbreak 1996 och i Larsmo 01.08.1996 och tillsvidare. De flyttade 1993 till Jakobstad och i juni 1996 till Kerstins barndomshem. Se Skog, nr 1. \\
\end{longtable}


%%%
% [occupant] KPO och
%
\jhoccupant{KPO och}{\jhname[Andelsringen]{KPO och, Andelsringen}}{1959--\allowbreak 1991}
Vid försäljningen 1991 var ägaren av fastigheterna 4:57 och 4:122 KPO med huvudkontor i Karleby. Vid årsskiftet 1959/1960 fusionerades Nykarleby handelslag med Jeppo-Oravais Handelslag och firmanamnet ändrades till Andelsringen med huvudkontor i Jeppo. Jeppo-Oravais Handelslag byggde 1959	garagebyggnaden samt 1960 bostäder i våningen ovanpå. År 1959 anskaffades också en butiksbil. Andelsringen fusionerades 1985 med Pedersöre Handelslag med nya namnet Botnica. Botnica fusionerades och uppgick 1987 i Waasko, som uppgick i KPO. Garagerna användes för egen verksamhet och bostäderna hyrdes av anställda.

\begin{longtable}{p{0.14\textwidth} p{0.8\textwidth}}
  \jhbold{Lokal 1}       & \\
  1980--\allowbreak 1985 & Fridlund Rainer John Valter, \textborn 10.10.1954 i Pensala, hyrde lokal 1 1980. Rainer fick tjänst som butiksbiträde på Andelsringen 1975. Sambo med Benita Christina Julin, född 15.10.1959 på Mjölnars. De flyttade från Bostads Ab Älvliden. 1982 flyttade Benita ut. I mars 1984 blev Rainer sambo med Birgitta Nyman, \textborn 16.03.1965 i Oravais. Vigda 1986. De flyttade i januari 1985 till Nykarleby då hans arbetsplats blev Botnica filialen i Socklot. Den 10.10.1985 startade de egen butik i f.d. handelslagets byggnad i Pensala, som de hade i elva år. Rainer började arbeta på Mirka 1996. \\
  1960--\allowbreak 1979 & Familjen Ingvald Karf flyttade i december 1960 från den gamla butiksbyggnaden på samma tomt till lokal 1. Se mera om familjen vid butiksokalen. Ingvald Karf var aktiv inom församlingen bland annat ekonomisektionens ordförande i kyrkorådet. Ingvar gick i pension 1979. \\
  \jhbold{Lokal 2} & \\
  1985--\allowbreak 1986 & Benita född Stoor i Pensala och hennes man Veikko Perälä från Oravais hyrde lokal 2. De hade under samma tid verksamhet i Café Funkis. Se Nybygge 4:28, nr 370. \\
  1981--\allowbreak 1984 & Chaufför Alf Rune Andersson, \textborn 21.06.1954 i Munsala och Märtha Benita Nyberg, \textborn 29.11.1955 i Larsmo, hyrde lokal 2 år 1981. Deras son Kaj Stefan föddes 21.03.1981. Alf var chaufför och försäljare på Adelsringens butiksbil 1976--\allowbreak 1987. Benita började 1979 som butiksbiträde på Andelsringen. År 1984 köpte de fastigheten Sandqvist 2:98 på Romar, nr 38. \\
  1960--\allowbreak 1976 och 1979--\allowbreak 1981 & Bo-Göran Backman, \textborn 16.08.1933 i Kvevlax, kom till Jeppo-Oravais Handelslag 1954 som lagerarbetare. Hans hustru Inga Linnéa Hurr, \textborn 16.11.1933 i Kvevlax, kom 15.07.1957 med sonen Lars-Erik, \textborn 23.04.1957 i Kvevlax. Inga, merkonom 1953, fick tjänst som bokförare på Handelslaget. De flyttade till lokal 2 på garaget från norra lokalen på Handelslagets övre våning. År 1976, då Granholms flyttade till Nykarleby, övertog familjen Backman deras lokal. Bo-Göran tog hand om gårdskarlsarbeten utöver sitt ordinarie arbete som lagerkarl. Då dir. Smeds 1979--\allowbreak 1981 kom till Jeppo måste familjen Backman flytta från direktörsbostaden. 1991 gick Bo-Göran i pension och 1995 flyttade de till Kvevlax. Bo-Göran dog 09.03. 2000. \\
  \jhbold{Lokal 3} & \\
  1982--\allowbreak 1986 & Snickare Ari Sorjonen, \textborn 1960 på Kaup, och Riitta Marjanen, \textborn 1963 i Kortesjärvi, hyrde lokal 3 år 1982. Ari arbetade på Viking Backs snickeri och Riitta på Sorjonens Trädgård på Kaup. De flyttade på sommaren 1986 till Elvi Strengells gård på Åkervägen. Deras första barn Jonna föddes 01.12.1986. 1989 flyttade de till eget hus på Måtar. \\
  1978--\allowbreak 1981 & Elmontör Kjell Karl-Gustav Norrgård, \textborn 1954 på Romar hemman och sömmerska Ann-Britt (Gitta) Carola Strandell, \textborn 1955 i Purmo, hyrde lokal 3 på sommaren 1978. Kjell och Gitta gifte sig 1978. Kjell hade fått arbete på Jeppo Kraft Andelslag 1974 och Gitta var arbetsledare på Scantop i Bennäs. Deras dotter Malin föddes 23.02.1980 när de bodde på garaget. Se Romar nr 37 och 38. \\
  1960--\allowbreak 1977 & Chaufför Stig Johan Byman, \textborn 20.06.1930 i Pensala och hans hustru Ines Linneá Häggblom, \textborn 14.10.1938 i Pensala, flyttade in i lokal 3 när bostäderna var inflyttningsklara i december 1960. Stig fick tjänst som chaufför på Jeppo-Oravais Handelslags butiksbil 10.04.1959 och Ines som affärsbiträde och senare som kontorist. Barn: Gun Carina Linnéa, \textborn 01.09.1964, lärare och Johan Henrik Gabriel, \textborn 18.11.1975, dipl. ing. I maj 1977 flyttade familjen till Stigs barndomshem i Pensala. \\
  \jhbold{Lokal 4} & \\
  1981--\allowbreak 1988 & Affärsbiträde Lars Sundberg, från Öja och cafébiträde Anita Linnéa Rönnblom, \textborn 1947 i Nedervetil, hyrde 1981 lokal 4. 1983 skilsmässa. Anita gifte sig 1984 med Kurt Hongisto. Lars blev senare sambo med Helena Hautala. \\
  1969--\allowbreak 1980 & Stig Jan-Erik Nybyggar, \textborn 16.03.1945 på Heikfolk, och hans hustru Anna Margaretha Nord, \textborn 04.01.1945 på Mietala, hyrde i januari 1969 lokal 4. Jan-Erik fick 1961 arbete som lagerman på Andelsringen. Margaretha hade studerat till merkant vid Gamlakarleby Handelsläroverk och fick tjänst 1965 på Andelsringens kontor i Jeppo.	1985--\allowbreak 2008 arbetade hon på Sparbanken/Andelsbanken. Jan-Eriks stora intresse är jakt och jakthundar. Han har under många år varit ordförande för Nykarlebynejdens Jaktförening och 50 år ordförande för	Jeppo Jaktförening. \\
  1968                   & Under några månader hösten 1968 hyrde Helge Nygård lokal 4. \\
  1960--\allowbreak 1968 & Paul Evald Granholm, \textborn 05.09.1925 i Oravais, och hans hustru Marianne Berg, \textborn 14.04.1930 i Munsala, kom i januari 1952 till Jeppo-Oravais Handelslag. Paul blev ansvarig för kött- och livsmedelsvaror. När garagebostäderna blev inflyttningsklara i december 1960	flyttade de in i lokal 4. Sommaren 1968 flyttade de till handelslaget, efter att familjen Johansson flyttat till Oravais. År 1976 flyttade familjen Granholm till sitt nybyggda hus i Nykarleby. Barn: Dan Kenneth, \textborn 10.12.1954 och Nancy Ann-Louis, \textborn 21.04.1961. \\
\end{longtable}


\jhhousepic{Jeppo Handelslags butik ca 1918.JPG}{Handelslaget och Landtmannabanken 1918}

%%%
% [occupant] Jeppo Hlg
%
\jhoccupant{Jeppo Hlg}{\jhname[Jeppo-Oravais Hlg]{Jeppo Hlg, Jeppo-Oravais Hlg}}{1916--\allowbreak 1959}
Handelslaget, som grundades 1914 och verkade ca två år i Lantmannagillets butik, köpte 09.04.1916 fastigheten 4:57 av Emil Ström.	Samtidigt ändrades namnet till Jeppo Handelslag m.b.t. På tomten fanns en gård med butik och bostad samt en uthusbyggnad. Telefonanslutning anskaffades. Lantmannabanken flyttade från Ida Lindströms	hus till handelslagets kontor den 01.01.1920 och fanns i samma rum fram	till 30.11.1923, då banken flyttade till Otto Jakobssons gård. År 1921 	tillbyggdes butikslokalen. 1926 övertogs Oravais handelslag och	namnet ändrades till Jeppo-Oravais Handelslag. År 1943 flyttade affärsrörelsen till den nya affärsbyggnaden på tomten 4:117, nr 57.	1946 köpte handelslaget granntomten mot väster med bostadshus och uthus, Lindfors 4:122, säljare Tor Ahlstrand.

\jhsubsubsection{Hyresgäster}
Efter att affärsrörelsen flyttat 1943 användes byggnaden som bostadslokaler. Inköpare Mats Edvin Ingvar Karf, född i Pensala, Munsala, vigd 08.10.1941 med Anna Viktoria Hilli, \textborn 27.06.1921 i Ytterjeppo. Deras dotter Ulla Helena föddes 11.08.1948 när de bodde i 	gamla handelslagsbyggnaden. 1960 flyttade familjen till lokal 1 på 	garagebyggnaden.


%%%
% [occupant] Ahlstrand
%
\jhoccupant{Ahlstrand}{\jhname[Tor]{Ahlstrand, Tor}}{1945--\allowbreak 1946}
Tor Ahlstrand köper 17.06.1946 av \jhname[Einar Norrgård]{Norrgård, Einar} fastigheten Lindfors 4:58, som Norrgård förvärvat 04.08.1945 av Agnes Westerlund.	Vid styckning av tomt Alstrand 4:121 (biograftomten) kvarstår av skattelägenhet Lindfors 4:14 endast Lindfors 4:122 (tidigare 4:58), se figur 359. Byggnaderna revs på 1960-talet.


%%%
% [occupant] Ström
%
\jhoccupant{Ström}{\jhname[Emil]{Ström, Emil}}{1913--\allowbreak 1914}
Den 28.03.1913 köper Emil Ström fastigheten, som senare får	registernummer 4:57. På tomten finns en gård med butik och uthus.\jhvspace{}


%%%
% [occupant] Liljeqvist
%
\jhoccupant{Liljeqvist}{\jhname[Isak]{Liljeqvist, Isak}}{1894--\allowbreak 1913}
Den 30.04.1894 köper handelsman, bonden Isak Liljeqvist, fastigheten och affärsrörelsen av ett handelsbolag vars föreståndare var Gustav Julin.\jhvspace{}


%%%
% [occupant] Julin
%
\jhoccupant{Julin}{\jhname[Gustav]{Julin, Gustav}}{1874--\allowbreak 1894}
Gustav Julin, \textborn 20.11.1838 på Lilljungar 6, var den första som	1871 erhöll rättighet att idka lanthandel i Jeppo. Julin hade 1870 öppnat en lanthandel på Mietala. Gustav gifte sig 1861 med Lovisa Eriksdotter Draka, \textborn 19.09.1842 i Ytterjeppo. De fick 12 barn. I medlet på 1870-talet flyttade familjen till Silvast och byggde en gård med butik. 01.10.1890 förstorades tomten med 4 kappland arrendejord på 50 år av bönderna Jakob Sjöblom och Anders Björklund. År 1891 sålde han gården till ett nystartat handelsbolag. Under tiden på Silvast flyttade familjen till andra orter 1877, 1881 och 1889. Gustav Julin var också resepredikant och kolportör. 1894 köpte han en tomt nära järnvägsstationen av blivande skattelägenhet Bäck 4:16, ägare Isak Lindfors, och byggde en gård med bageri och butik, nr 372.



%%%
% [house] Alstrand
%
\jhhouse{Alstrand}{4:121}{Silvast}{5}{60}

Styckad av stomlägenhet Lindfors 4:14

\jhhousepic{071-05696.jpg}{Bostadsdelen av Bio Silva, f.d.huvudingången och trappan avlägsnad.}

%%%
% [occupant] Nyman
%
\jhoccupant{Nyman}{\jhname[Dödsbo]{Nyman, Dödsbo}}{2006--}
Bjarne Nymans dödsbo sedan 2006. Huset påminner fortfarande den äldre generationen om tider då ett biobesök i Jeppo hörde till emotsedda nöjen.\jhvspace{}


%%%
% [occupant] Nyman
%
\jhoccupant{Nyman}{\jhname[Bjarne]{Nyman, Bjarne} \& \jhname[Gunhild]{Nyman, Gunhild}}{1957--\allowbreak 2006}
Den 27.09.1957 erhöll Bjarne Valdemar Nyman, \textborn 22.01.1929 på Silvast hemman, och Gunhild Irene Björkvik, \textborn 20.02.1932 på Lassila hemman, lagfart på fastigheten Alstrand 4:121, som de köpt 1957. Bjarne och Gunhild vigdes i Jeppo kyrka 20.07.1958.

Bjarne	hade emigrerat 1949 och Gunhild 1953 till Sverige. De	bosatte sig 1956 i Västerås. Före utresan utbildade sig Gunhild år 1949 till yrkessömmerska samt gick i Kronoby folkhögskola 1951/1952. Hon har varit verksam ett femtontal år inom sömnadsbranschen. Därefter utbildade hon sig till	undersköterska på Vårdskolan i Västerås. Gunhild har arbetat som undersköterska i 25 år på Centrallasarettet i	Västerås. Bjarne har arbetat i 35 år inom byggnadsbranschen	och blev en skicklig snickare. Han var med om flera stora	byggnadsprojekt i Västerås stad; stadshuset, Domus, Sigma	och stora Parkhuset med sina gågator inomhus. I Västerås har de ett större egnahemshus.

Bostaden i det gamla biografhuset Silva i Jeppo använder familjen som semesterbostad. Efter att paret Nyman övertog	biografen var den verksam i drygt 2 år både som uthyrd och i egen regi. År 1998 revs biografsalen medan bostadsdelen lämnades kvar. Under sommarvistelserna har Bjarne och Gunhild renoverat huset och planterat träd och buskar på tomten.
\begin{jhchildren}
  \item \jhperson{\jhname[Kjell Johan Bjarne]{Nyman, Kjell Johan Bjarne}}{27.03.1960}{}
  \item \jhperson{\jhname[Johnny Frank Michael]{Nyman, Johnny Frank Michael}}{13.01.1962}{}
\end{jhchildren}
Efter studentexamen har Kjell studerat till ingenjör och Johnny till civilekonom. Johnny har två söner, Philip f. 1991 och André f. 1998.

Bjarne dog i cancer 05.12.2006. Han har sin grav i Jeppo. Även efter Bjarnes död tillbringar Gunhild och den övriga familjen flera veckor varje sommar i huset.


%%%
% [occupant] Sandberg
%
\jhoccupant{Sandberg}{\jhname[Joel]{Sandberg, Joel} \& \jhname[Jenny]{Sandberg, Jenny}}{1949--\allowbreak 1957}
Joel Sigfrid Sandberg, \textborn 24.01.1914 på Ojala hemman, och hans hustru Jenny Sofia Åstrand, \textborn 26.07.1912 på	Ruotsala hemman, köpte fastigheten Alstrand 4:121 av Tor Ahlstrand. Lagfart 26.02.1949. De byggde en biografsalong med bostad i övre våningen ovanför vestibulen och ett rum	i nedre våningen, där Jenny hade sin damfrisering. Tillsammans med Tor Ahlstrand bedrev de biograf Silva i byggnaden fram till 1954.
\begin{jhchildren}
  \item \jhperson{\jhname[Margot Inga Sofia]{Sandberg, Margot Inga Sofia}}{19.11.1935}{}
  \item \jhperson{\jhname[Siv Alfhild]{Sandberg, Siv Alfhild}}{27.08.1940}{}
\end{jhchildren}
Margot är född i Pedersöre och Siv på Böös hemman. År 1955 emigrerade familjen till Seattle i USA. Margot gifte sig	med sin finländska fästman Tage Rolf Gustav Westerlund, \textborn 05.02 1934 på Fors hemman. Rolf emigrerade ca två år efter Margot. De har två barn, Kennet Rolf, \textborn 09.08.1959 och Gina Anette, \textborn 20.04.1971. Rolf dog 1996.

Siv gifte sig vid 18 år med Donald Charles Mc Glothlen. De har två barn, Gregory Charles \textborn 02.11.1967 och Christina Marie, \textborn 07.05.1970.

Jenny och Joels äktenskap slutade i skilsmässa.	Joel \textdied 08.03.1983  ---  Jenny \textdied 11.10.1978 i Olympia, Washington.


%%%
% [occupant] Ahlstrand
%
\jhoccupant{Ahlstrand}{\jhname[Tor]{Ahlstrand, Tor}}{1946--\allowbreak 1949}
Den 17.06.1946 köper Tor Ahlstrand, \textborn 16.04.1927 på Ojala hemman, fastigheten Lindfors 4:58, tidigare 4:14 av Silvast hemman. Tomten Alstrand 4:121 styckas ut. Tor byggde inte på tomten, men efter att han sålde området till sin morbror Joel, gick han 1949--\allowbreak 1951 i Svenska teaterns elevskola i Helsingfors. Tor bodde aldrig på lägenheten. Han hyrde biografen i Jeppo och var involverad i branschen också i Oravais och Vörå till år 1954, då han övertog en bilservicestation i Oravais, som han bedrev i 11 år.

Tor gifte sig 1951 med \jhname[Karin Ingeborg Siffris]{Siffris, Karin Ingeborg}, \textborn 28.05.1927 i Kimo, Oravais. De har tre barn: Göran, Susanne och Ann-Christin.	År 1960 blev Tor frilansare på Finlands Rundradio och fick fast anställning 1968. År 1979 blev han TV-redaktör och producent med svenska Österbotten som verksamhetsfält.


%%%
% [occupant] Norrgård
%
\jhoccupant{Norrgård}{\jhname[Einar]{Norrgård, Einar}}{1945--\allowbreak 1946}
Ovannämnda område innehades 04.08.1945-17.06.1946	av Einar Norrgård. Säljare Agnes Westerlund.\jhvspace{}


%%%
% [occupant] Lindfors
%
\jhoccupant{Lindfors}{\jhname[Isak]{Lindfors, Isak}}{1915--\allowbreak 1945}
Klyvning och lantmäteriförrättning av de fyra skattelägenheterna 1, 2, 3, och 4 av Silvast hemman nr 4 påbörjades i medlet av 1880-talet och slutfördes 1915. I Jordregistret 1915 infördes Isak Lindfors som ägare till stomlägenhet Lindfors 4:14. Arvinge till Isak Lindfors var	sondottern Agnes Westerlund, född Lindfors. Se mera om familjen Lindfors och byggnaderna på Lindfors 4:122, (4:14) nr 359.


%%%
% [occupant] Soldattorp
%
\jhoccupant{Soldattorp}{138}{}
På området, som idag heter Alstrand 4:121 fanns redan på 1700-talets första hälft ett soldattorp. I 1807 års mantalslängd	ägde Hans Jönsson 17/96 mantal av Storsilfvast hemman om 17/24 mantal. På lägenheten fanns torp nr 138, som	låg på västra sidan om Silvast bäcken på den tomt där sedermera biografen i Silva byggdes. Se nedan!



%%%
% [house] Soldattorp Nr 138
%
\jhhouse{Soldattorp Nr 138}{4}{Storsilfvast}{5}{360}

\jhnooccupant{}

Objekt: Soldattorp 138 på Storsilfvast hemman nr 26 i dåvarande Överjeppo by.

Soldattorpen grundlades efter det fasta knekthållets införande 1733. Detta hade blivit aktuellt på grundval av de erfarenheter Karl XI fått av bl.a. slaget vid Lund 1676. Insikten var att landets fortbestånd var beroende av en egen fast armé baserad på andra principer än utskrivning och legosoldater. Nu startade indelningsverket inom Sverige fr.o.m. 1682 och genomfördes på de flesta håll, men i Österbotten stretade man emot, som ett av de sista områdena inom riket. Under kung Fredrik I genomfördes slutligen det fasta knekthållet också här, efter 50 års fördröjning och då  fick det ske med list. Det nya systemet ställde nämligen i utsikt att arbetskraften kunde få en förstärkning på landsbygden då soldaterna inte skulle kommenderas bort under fredstid. I själva verket blev systemet billigare för kronan då allmogen nu skulle få ta hand om soldaternas uppehälle.

Rote 138 dyker upp 1737 i Öfwerjeppo by och för att fylla det nödiga antalet rotemantal om 4 ¾ rmtl krävdes i detta sammanhang tre hemmans insatser, nämligen Måtare nr 28, Storsilfwast nr 26 och ännu Rautzinkåski nr 30. Torpet kallas då Sparfwas och senare 1752 för Sparfbacka och 1775 Måtars.

Exakt var torpstugans  fanns är något osäkert, men enighet råder att den stod nära Silvastbäcken i dagens centrum och kanske på Börje Lundviks tomt. Själv har han gett som ett alternativ det som kallades ``skräddar Rintalas stuga'' stående på hans tomt och som bortrevs i medlet av 1950-talet. Men han har också berättat hur han vid en av de allra första gångerna, när han som 15-åring fick backa en lastbil till en gammal stuga, som stod 50 m längre norrut och måste bort, när trappan upp till den nybyggda biografens projektorrum skulle gjutas. Då sade ordinarie chauffören Johannes Eklöv: ``Nu försvinner det gamla soldattorpet för gott''. Detta skulle ha skett 1950.

På torpet har varit stationerad:
\begin{center}
  \begin{tabular}{l l l}
    Johan Andersson Wendelin    & 1734--\allowbreak 1747 & Avsked. \\
    Gustaf Wendelin             & 1747--\allowbreak 1748 & Död i Helsingfors \\
    Abraham Gustafsson Wendelin & 1749--\allowbreak 1770 & Föravskedad \\
    Johan Jonasson Malm         & 1770--\allowbreak 1773 & Död på Sveaborg \\
    Erich Malm                  & 1774--\allowbreak 1779 & Byte till N:o 150 i Härmä \\
    Matts Erhard från N:o 150   & 1779--\allowbreak 1799 & Kasserad ? \\
    Lars Larsson Bastion        & 1779--\allowbreak 1790 & Död på krigssjukhus \\
    Elias Eliasson Bastion      & 1790--\allowbreak 1808 & Hu. Carin, 3 barn. Föravskedad \\
    Carl Fredric Bast           & 1808--\allowbreak 1810 & Avsked vid arméns upplösning \\
  \end{tabular}
\end{center}
Allt tyder på att den första soldaten fick flytta in i gamla byggnader. Hur skall man annars förstå att vid syneförrättningen 15 år efter ibruktagandet var både stugans, fähusets och ladans undergärder ruttna och nersjunkna i marken. Taken var förfallna av ålder och läckte. Fönstren var blyfallna, sädesboden saknade lårar och var endast utrustad med ett odugligt lås. Då bristerna konstaterades ``vara av ålder'', fick rotesällarna tillsägelse att ställa allt i fullt stånd. Även om  åkrarna var i skick gällde tillsägelsen också gärdesgårdar och diken.

Vid  syneförrättningen 1791 befanns stugan och uthus vid gott stånd. Åkern var på hela roten ``wäl häfdad'' omfattande 1 tunnland. Kålland om 2 kappland (300 m² ) fanns och ängar goda för 12 skrindor hö. Också brunn fanns.



%%%
% [house] Bäckliden
%
\jhhouse{Bäckliden}{4:91}{Silvast}{5}{61}

Styckad av stomlägenhet Dahlbo 4:12

\jhhousepic{070-05609.jpg}{Pärssinen Kimmo och Jaana}

%%%
% [occupant] Pärssinen
%
\jhoccupant{Pärssinen}{\jhname[Kimmo]{Pärssinen, Kimmo} \& \jhname[Jaana]{Pärssinen, Jaana}}{2013--}
Kimmo Kalevi Pärssinen, \textborn 26.08.1968 i Vanda lk och hans hustru Jaana Elisabeth, \textborn 30.07.1968 i Jakobstad, övertog fastigheten Bäckliden 4:91. Kimmo arbetar på KWH-Mirka i Jeppo inom 	produktionen på bandsidan. Jaana är operativ ledare på städfirman SOL-palvelut Oy inom Kauhava-området. Kimmo och Jaana blev sambo 1997 när de bodde i Jakobstad. År 2001 flyttade familjen till Jeppo och Kimmos mor Terttu flyttade till Nykarleby. Jaana och Kimmo vigdes 12.06.2004 vid ett besök hos Kimmos far och hans sambo i Estland.
\begin{jhchildren}
  \item \jhperson{\jhname[Riko Kalevi]{Pärssinen, Riko Kalevi}}{15.07.1999 i Jakobstad}{}
  \item \jhperson{\jhname[Katja Kaisa]{Pärssinen, Katja Kaisa}}{17.02.2002 i Jeppo}{}
  \item \jhperson{\jhname[Tanja Margarita]{Pärssinen, Tanja Margarita}}{27.04.2003 i Jeppo}{}
  \item \jhperson{\jhname[Niko Valdemar]{Pärssinen, Niko Valdemar}}{02.02.2006 i Jeppo}{}
\end{jhchildren}
Jaana har två flickor från ett tidigare äktenskap, som under de första åren bodde varannan vecka hos Kimmo och Jaana. Jonna Elisabeth Englund	\textborn 31.12.1989 i Jakobstad	och	Janina Sofia Englund	\textborn 11.01.1992 i Jakobstad.	Jonnas och Janinas far dog hastigt på hö-ängen i Bennäs 2008. Jonna och Janina bodde därefter helt med Kimmo och Jaana.	Under de två första åren i Jeppo bodde även Kimmos dotter tidvis här. Kimmo och Jaana har renoverat huset, även källaren med bastu och	tvättstuga. Trädgården med lusthus och växthus har Jaana som fritidsintresse.


%%%
% [occupant] Pärssinen
%
\jhoccupant{Pärssinen}{\jhname[Teuvo]{Pärssinen, Teuvo} \& \jhname[Terttu]{Pärssinen, Terttu}}{1986--\allowbreak 2013}
År 1986 köpte Teuvo Kalevi Pärssinen, \textborn 30.01.1934 i Viborg, och hans hustru Terttu, född Korhonen, \textborn 05.05.1948 i Sonkajärvi fastigheten av Erik och Jenny Gleisners dödsbo. Teuvo arbetade då som	försäljningschef på KWH-Mirka. Han blev egen företagare i medlet av 1990-talet och flyttade till Estland. Skilsmässa 1996. Terttu och dotter Pia hade verksamhet i café Funkis 1995--\allowbreak 1998.
\begin{jhchildren}
  \item \jhperson{\jhname[Kimmo Kalevi]{Pärssinen, Kimmo Kalevi}}{26.08.1968}{}
  \item \jhperson{\jhname[Pia]{Pärssinen, Pia}}{05.11.1971}{}
  \item \jhperson{\jhname[Anne-Marie]{Pärssinen, Anne-Marie}}{08.07.1973}{}
\end{jhchildren}
Terttu och barnen bodde kvar i huset. Terttu flyttade till Nykarleby 2001. Efter pensionering återvände Teuvo 2010 till Jeppo och bodde 1,5 år med Kimmos familj. Han köpte en lokal i Jakobstad och flyttade dit, men blev sjuk och dog 23.11.2013. Teuvo jordfästes och blev begraven i Jeppo.


%%%
% [occupant] Gleisner Erik
%
\jhoccupant{Gleisner Erik}{\jhname[dödsbo]{Gleisner Erik, dödsbo}}{1971--\allowbreak 1986}
Erik Gleisner avled hastigt mitt i sitt slakteriarbete den 28.11.1971. Änkan Jenny bodde kvar i huset till 1986. Sonen Per hade 1962 flyttat till Nykarleby. Erik hade hjälpt honom med husbygget samma år som han dog, men hann aldrig se det inflyttningsklart.


%%%
% [occupant] Gleisner
%
\jhoccupant{Gleisner}{\jhname[Erik]{Gleisner, Erik} \& \jhname[Jenny]{Gleisner, Jenny}}{1932--\allowbreak 1971}
Slaktaren Erik Otto Gleisner, \textborn 18.12.1910 i Socklot, Nykarleby lkm, köpte i början av 1930-talet ett område av stomlägenhet Dahlbo 4:12. Erik gifte sig 1940 med Jenny Emilia Vikström, \textborn 31.05.1920	på Tapelbacka vaktstuga. I början av 1940-talet byggde de en större bostadsbyggnad i två våningar samt med källare under huset. Redan innan bostaden var färdig flyttade Erik sin köttaffär från Jakobssons hus till källaren. Efter ett par år inreddes rummet i sydost som köttaffär med dörr och trappa mot vägen. Under kriget och åren strax efter var försäljningen av kött på licens.

På tomten fanns en ria, som revs och Erik uppförde en skild uthusbyggnad med slakteri. Den drabbades av brand i tidigt skede en höstkväll, varvid familjens ståtliga schäfer omkom i lågorna.
\begin{jhchildren}
  \item \jhperson{\jhname[Per Erik]{Gleisner, Per Erik}}{26.05.1942}{}
  \item \jhperson{\jhname[Gun-Lis Margareta]{Gleisner, Gun-Lis Margareta}}{08.03.1945}{23.06.1971}
\end{jhchildren}
Per blev tryckeriarbetare med ansvar för postningsavdelningen på 	Jakobstads Tidning. Per flyttade till Nykarleby 1962. Gun-Lis avled före sin far Erik på Roparnäs sjukhus. Jenny avled i Nykarleby den 02.06.1997. De är begravna i Jeppo.


\jhsubsubsection{Hyresgäster}
I början på 1950-talet färdigställdes övre våningen till en skild bostad.	Jennys bror Lennart Vilhelm Vikström, \textborn 02.12.1924 på Gunnar vaktstuga, söder om Jeppo station, och hans hustru Elsa, född Englund, \textborn	25.05.1927 i Bennäs, flyttade in som hyresgäster.
\begin{jhchildren}
  \item \jhperson{\jhname[Carita Viveca]{Gleisner, Carita Viveca}}{02.07.1951}{}
  \item \jhperson{\jhname[Yvonne Kristina]{Gleisner, Yvonne Kristina}}{26.12.1957}{}
\end{jhchildren}
Som 21-åring ingick Carita äktenskap med	jordbrukare Karl Ingmar Brännäs \textborn 24.02.1950 i Pedersöre. Barn: Sara Johanna född 1981 och Sofia Katarina född 1985. Yvonne föddes efter att familjen 1956 flyttat till det röda husets norra lokal på stationsområdet. Yvonne blev student 1977 och utbildade sig till tandskötare. Hon är gift och bor i	Jakobstad.

Lennart Vikström tjänstgjorde som stationskarl i Jeppo. De hade en ko	när de bodde på stationsområdet. År 1965 fick han motsvarande	befattning på järnvägsstationen i Jakobstad. Lennart var intresserad av	jakt och deltog många år efter pensioneringen i älgjakten i Jeppo. Lennart och Elsa är begravna i Jakobstad.

Som pensionärer 1956 flyttade Jennys föräldrar, banvakten Jakob	Vilhelm Vikström, \textborn 26.05.1891 på Lassila hemman, och	hans hustru Ester Sofia Stenbacka, \textborn 12.09.1886 på Tapelbacka	vaktstuga, till bostaden i övre våningen. De flyttade från vaktstugan på Gunnar. Jenny hade två äldre systrar, som också var födda på Tapelbacka; Else Viola, \textborn 02.04.1913, gift 1934 med Elroy Berg, \textborn 02.11.1912 på Kojonen i Lassila by, samt Ester Anna Alice, \textborn 10.06.1917. Jennys mor Ester avled den 31.01.1962 och far Vilhelm den 14.7.1969.


%%%
% [occupant] Huhtala
%
\jhoccupant{Huhtala}{\jhname[Jakob]{Huhtala, Jakob} \& \jhname[Edla Maria]{Huhtala, Edla Maria}}{1901--\allowbreak 1928}
Jaakko Huhtala, \textborn 07.05.1875 i Ylistaro, gifte sig 10.11.1901 med Edla Maria Sjöblom, \textborn 27.06.1879 på Lungbo 4:45 av skattelägenhet Lindfors 4:14. Genom ett byte av jordområden i slutet av 1800-talet hade Edla Marias far Jakob Sjöblom erhållit ett område vid Stationsvägen. På tomten byggdes ett bostadshus med vindsvåning (karta 5, nr 361a).
\begin{jhchildren}
  \item \jhperson{\jhname[Ellen]{Huhtala, Ellen}}{04.08.1902}{}
  \item \jhperson{\jhname[Elsa]{Huhtala, Elsa}}{01.01.1905}{}
  \item \jhperson{\jhname[Vilhelm]{Huhtala, Vilhelm}}{22.02.1908}{09.06.1926}
  \item \jhperson{\jhname[Anna]{Huhtala, Anna}}{02.11.1909}{18.02.1925}
  \item \jhperson{\jhname[Valdine]{Huhtala, Valdine}}{07.05.1912}{01.08.1913}
\end{jhchildren}
Edla Maria dog 1915. Samma år gifte sig Jakob andra gången med Anna Sofia Andersdotter Rajakoski, \textborn 29.10.1877 i Lappo. Anna Sofia hade en dotter Saima Elisabet \textborn 02.04.1903. Anna Sofia dog 23.04.1928.


%%%
% [oldhouse] Gamla gården
%
\jholdhouse{Gamla gården}{4:}{Silvast}{5}{361a}

%%%
% [occupant] Berg
%
\jhoccupant{Berg}{\jhname[Anna-Sanna]{Berg, Anna-Sanna}}{1897--\allowbreak 1928}
Bostadshuset byggdes troligen redan 1897 eller 1898, då Edla Marias äldre syster Anna-Sanna, \textborn 12.04.1874 på Lugnbo 4:45, nr 55,	gifte sig 13.12.1897 med Nikolai Andersson Ryssberg från Alahärmä. Paret ändrade 1898 tillnamnet till Berg. I början av 1900-talet vistades de en kortare tid i Amerika.
\begin{jhchildren}
  \item \jhperson{\jhname[Anna Elvira]{Berg, Anna Elvira}}{09.12.1898 i Jeppo}{1968}, 1916 till USA
  \item \jhperson{\jhname[Johan Jakob]{Berg, Johan Jakob}}{29.01.1902 i USA}{26.12.1907 i Jeppo}
  \item \jhperson{\jhname[Ellen Elisabet]{Berg, Ellen Elisabet}}{12.05.1903 i Jeppo}{}
  \item \jhperson{\jhname[Emil Nikolai]{Berg, Emil Nikolai}}{22.02.1905 i Jeppo}{01.02.1906 i Jeppo}
\end{jhchildren}
I ``Historik över Jeppo'' berättas att kommunalstämman hölls 1908--\allowbreak 1909 i ett rum, som hyrdes av Anna-Sanna Berg på Silvast. Anna-Sanna dog 15.03.1915. I bostadshuset fanns således 1915 två familjers moderslösa barn.

I samband med lantmäteriförrättning av Silvast hemman, som infördes i Jordregistret 1915, finns inte bonden Jakob Sjöblom mera som ägare, utan svärsonen Jakob Huhtala är ägare till stomlägenheten Nyhagen 4:15. Vid Jakobs tredje äktenskap 1928 med Amanda flyttades huset till Nyhagen 4:15, karta 4, nr (356x).



%%%
% [house] Hemåker
%
\jhhouse{Hemåker}{4:89}{Silvast}{5}{361}

Styckad av stomlägenhet Dahlbo 4:12

\jhhousepic{Finskaskolan3.jpg}{Finska skolan med Jungerstams uthus t.h.}

%%%
% [occupant] Nkby stad
%
\jhoccupant{Nkby stad}{\jhname[Jeppo kommun]{Nkby stad, Jeppo kommun}}{1949--}
Jeppo kommun övertog den 11.06.1949 den finska skolans fastighet med lösöre och skolverksamhet. Överlåtare var Finska skolans garantiförening. Vid kommunsammanslagningen 1975 blev Nykarleby stad ägare till fastigheten. Finska skolan fortsatte sin verksamhet till våren 1977, då den nedlades på grund av brist på elever. Redan 1961 hade elevinternatet nedlagts, då Jeppo kommun började sköta om skolskjutsarna. Från hösten 1977 skjutsades de finskspråkiga eleverna till Alahärmä enligt önskan av de finskspråkiga föräldrarna. I dag går en del av eleverna även i den finska skolan i Nykarleby.

Då skolan stängdes hyrde Lantbruksaffären Hankkija fastigheten för försäljning av lantbruksprodukter och –maskiner. \jhname[Bengt Back]{Back, Bengt} var ansvarig försäljare. Hankkija köpte KPO fastigheten, nr 57, och flyttade verksamheten dit. Finska skolans byggnader stod tomma en tid. På 1990-talet revs byggnaderna av Karl-Gustav och Kaj Sandberg, som av stockarna byggde sommarstugor.


%%%
% [occupant] Finska skolans
%
\jhoccupant{Finska skolans}{\jhname[Garantiförening]{Finska skolans, Garantiförening}}{1938--\allowbreak 1949}
Redan på 1920-talet arbetades det för ett finskt skoldisrikt i Jeppo.	Den 13.09.1923 beslöt kommunfullmäktige att inte bilda något finskt skoldistrikt på grund av att de finsktalande barnen var få till antalet och de flesta av dem var bosatta nära gränsen till Alahärmä. Det ansågs att barnens bekvämlighet bäst tillgodosågs om barnen på kommunens 	bekostnad fullgjorde sin läroplikt vid närmaste skola i Alahärmä kommun. Arbetet och stridigheterna fortsatte. Den 11.02.1932 fastställde kommunfullmäktige två finskspråkiga skoldistrikt, Södra finska distriktet, som omfattade Krouvi, Pauhu, Pelkkala, Kihakoski och Motal	hemman. Norra distriktet omfattade resten av kommunen.

Förutsättningen för en finsk skola tillkom 21.06.1934, då skolstyrelsen fastställde ett finskspråkigt skoldistrikt i Jeppo. Förberedelserna tog ännu ett år innan man bland den finska befolkningen den 04.09.1935	bildade en garantiförening för skolan ``Jepuan kirkonkylän yksityisen	kansakoulun kannatusyhdistys'', som registrerades 10.09.1935.	Redan på första mötet beslöt man att starta skolverksamheten samma höst genom att av Erik Nordström hyra den så kallade Såggården på	Fors hemman. Den 07.10.1935 startade skolan sin verksamhet.

Tanken att bygga en egen skola väcktes sommaren 1937. Våren 1937 hade Riksdagen godkänt en lagändring, som gav staten möjlighet att 1938 ge understöd vid byggande av skolor. Direktionen för finska skolan köpte	05.03.1938 en tomt på 4000 m2 av Hemåkern, som utgjorde en del av skattelägenhet Dahlbo 4:12. Säljare var kartograf Wiktor Smulter. Den	24.03.1938 godkände direktionen skolritningarna, uppgjorda av arkitekt Toivo Salervo, på en skolbyggnad i två våningar med internat och bostäder i övre våningen, samt en skild ekonomibyggnad. På basen av	erhållna anbud den 18.05.1938 gav man byggnadsarbetet till Eero Kalliokoski och Vilho Tyni. Skolverksamheten, läraren och de första åtta internateleverna kunde flytta in i den nya skolan i oktober 1938. Denna höst fick skolan också samma rättigheter som andra kommunala skolor.

Under vinterkriget 1939--\allowbreak 1940 fanns på skolan en ammunitionsfabrik och	på hösten 1941 en militäranläggning. Skolbygget var ett stort projekt för en fattig garantiförening och en stor belastning togs från garantiföreningen då kommunen 11.06.1949 köpte	fastigheten och lösöret.

Finska skolans lärare:
\begin{enumerate}
  \item 1935--\allowbreak 1943  \jhname{Hakala, Kaino}
  \item 1943--\allowbreak 1946	\jhname{Saari, Annikki}
  \item 1946--\allowbreak 1949	\jhname{Ojakangas, Inkeri}
  \item 1949--\allowbreak 1950	\jhname{Alamäki, Anna}
  \item 1950--\allowbreak 1973	\jhname{Rautio, Aili}
  \item 1973--\allowbreak 1977	\jhname{Mäkitorkko, Eero}
\end{enumerate}

Skolköksa/städerska:
1938--\allowbreak 1961;	Frk. Ylinen från Ekola och därefter \jhname[Regina Alaranta]{Alaranta, Regina} (gift Karkaus). De var bosatta på skolan på grund av elevinternatet.
1962--\allowbreak 1977;	\jhname{Pelmas, Lempi}, se nr 86.



%%%
% [house] Dahlbo \& Höglund
%
\jhhouse{Dahlbo \& Höglund}{4:127 resp. 4:106}{Silvast}{5}{77, 77a och 377}
Styckad av stomlägenhet Dahlbo 4:12

\jhhousepic{068-05565.jpg}{Dahlqvists bostad och s.k. Tikkas röda hus t.v. Vy från söder.}

%%%
% [occupant] Dahlqvist
%
\jhoccupant{Dahlqvist}{\jhname[Lars]{Dahlqvist, Lars} \& \jhname[Ann-Christin]{Dahlqvist, Ann-Christin}}{1988--}
Den 22.02.1988 köpte Lars Dahlkvist, \textborn 07.03.1953 i Purmo och	hans hustru Ann-Christin, född Levlin, \textborn 12.03.1959 i Ytterjeppo, de båda ovanstående fastigheterna. De flyttade med barnen Christian och Pamela från en bostad på Gunnar 01.04.1988 till sitt nya hem, på Höglund	4:106. En brand i en tvättmaskin utbröt i övre våningen den 01.10 1989 	och hela våningen brandskadades och en stor del revs och 	återuppbyggdes, även nedre våningen rök- och vattenskadades. 	Renoveringen gick snabbt, under tre månader, så familjen kunde flytta tillbaka från HAB huset, nr 70, på	julafton 1989.

Ann-Christin är utbildad frisör och öppnade frisörsalong i affärslokalen och fortsatte med medicinskåp och hårvårdsprodukter. I april 2008 flyttade Ann-Christin sin verksamhet till norra lokalen i Sparbanken 4:194 och hyrde ut	salongslokalen i hemmet till Bösas Massage, Ulla-Stina Lassander. Den 28.02 2013 avslutade Ann-Christin sin frisörsalong och studerade till närvårdare. Hon arbetar numera på serviceboendet Hagalund.

Lars är busschaufför på Haldin \& Rose, som numera ägs av Ingves \& Svanbäcks bussbolag. Sommaren 2016 började Lars pensionärstid.
\begin{jhchildren}
  \item \jhperson{\jhname[Christian Lars Emil]{Dahlqvist, Christian Lars Emil}}{05.10.1982}{}, ekon.mag., gift, Helsingfors
  \item \jhperson{\jhname[Pamela Anna Irene]{Dahlqvist, Pamela Anna Irene}}{19.12.1985}{}, socionom, sambo, Åbo
  \item \jhperson{\jhname[Julia Selma Christin]{Dahlqvist, Julia Selma Christin}}{27.01.1992}{}, stud. YH Novia
\end{jhchildren}
Familjen Dahlkvist använder byggnaden på Dahlbo 4:127 som	uthus och förrådsutrymmen.


%%%
% [occupant] Haga
%
\jhoccupant{Haga}{\jhname[Erik]{Haga, Erik} \& \jhname[Hjördis]{Haga, Hjördis}}{1957--\allowbreak 1988}
Den 16.09.1957 köpte Erik Alfred Haga, \textborn 14.05.1923 på Skog	hemman, och hans hustru Hjördis Margareta Backlund, \textborn 13.11.1922 i Kimo, fastigheterna Höglund 4:106 och Dahlbo 4:127 av stomlägenhet Dahlbo 12. De flyttade från sitt egnahemshus på Holmen på	Fors hemman, som blev	deras hem när de gifte sig (se karta 7, nr 129).

Hjördis hade gått i	Handelsskola och arbetat på Mellersta Österbottens Andelsbank. Hon övertog och fortsatte med kemikalieaffären samt medicinskåpet med	förmedling av mediciner från apoteket i Nykarleby. Erik hade en mindre lastbil, men övergick helt till taxiverksamhet. Erik hade deltagit vid	fronten i fortsättningskriget och var aktiv i Nykarlebynejdens krigsveteraner. Hjördis var intresserad av poesi, heminredning och	vackra saker. Hon var en mild och omtyckt affärskvinna.
\begin{jhchildren}
  \item \jhperson{\jhname[Inga-Lisa]{Haga, Inga-Lisa}}{1944}{}, psykoterapeut, Sthlm
  \item \jhperson{\jhname[Monica]{Haga, Monica}}{1948}{}, speciallärare, Sthlm
  \item \jhperson{\jhname[Ingmar]{Haga, Ingmar}}{1951}{}, Europadir vid  gruvbolaget Agnico-Eagle, Hfors
\end{jhchildren}
Erik och Hjördis byggde i Nykarleby ett bostadshus för pensionärstiden,	som de fick uppleva i 33 år i sitt nya hem. Erik dog hastigt den 04.12.2013 och samtidigt fick Hjördis en stroke. Hjördis dog 26.08.2015 på Östanlid sjukhus.


%%%
% [occupant] Åkermark
%
\jhoccupant{Åkermark}{\jhname[Vilhelm]{Åkermark, Vilhelm} \& \jhname[Birgit]{Åkermark, Birgit}}{1952--\allowbreak 1957}
Den 22.09.1953 erhöll Vilhelm Magnus Åkermark och hans hustru	Birgit Vilhelmina lagfart på Höglund 4:106. Aktuella satsningar drog till Jeppo; Gunnar Norrback, Manne Bergman och Vilhelm Åkermark hade 1950 startat ett pälsberederi i f.d. Jungar mejeri. Tillverkningen ändrades snart till plastprodukter och Sven Nyman kom med som ägare. Se mera historien om Jungar mejeri på Böös.

Vilhelm Åkermark, \textborn 18.11.1912 i Purmo, föll som barn från ett träd och skadade ett ben svårt så att han fick leva med ett träben. Vilhelm blev klasslärare 1934 och studerade senare till kemisk diplomingenjör 1949. Lärare Birgit, född Liljeqvist i Larsmo, är dotter till lärare Elna Liljeqvist från Jeppo. Familjen Åkermark kom till Jeppo 1949 från Åbo, då Vilhelm	fick arbete på pälsberederiet på Kiitola. Familjen bodde först i Ingrid Hägers gård på Holmen, och flyttade därifrån till övre våningen på fabriken vid Jungar och senare till Silvast, Jakobssons hus, Nybygge 4:28, nr 370, där Birgit öppnade en kemikalieaffär. Vilhelm och Birgit byggde på Höglund 4:106 ett större bostadshus i två våningar, med en 	affärslokal. I juni 1953 fick Birgit Åkermark tillstånd att i sin kemikalieaffär hålla ett medicinskåp och förmedla mediciner från apoteket i Nykarleby.
\begin{jhchildren}
  \item \jhperson{\jhname[Kerstin Birgitta]{Åkermark, Kerstin Birgitta}}{14.12.1943}{}
  \item \jhperson{\jhname[Henrik Vilhelm]{Åkermark, Henrik Vilhelm}}{11.10.1946}{}
  \item \jhperson{\jhname[Gustav Anders Sigfrid]{Åkermark, Gustav Anders Sigfrid}}{20.10.1952}{}
\end{jhchildren}
Mormor Elna Liljeqvist bodde även med familjen i det nya huset. Huset är byggt i stående timmer från en ria på Tallbacka i Purmo. Hösten 1956 fick Vilhelm arbete som lärare i Härnösand.

Plastfabriken Prevex på Jungar brann 28.12.1956 och verksamheten flyttade 1957 till Nykarleby. Familjen Åkermark flyttade därefter på våren 1957 till Härnösand. Den 16.09.1957 sålde Åkermark fastigheten till Erik och Hjördis Haga. Vilhelm dog 13.07.1986 och Birgit 21.06.2003. De är begravna i Purmo. Kerstin bor i Sommen/Småland, Henrik i Lidingö och Gustav utanför Göteborg.


%%%
% [occupant] Törnqvist
%
\jhoccupant{Törnqvist}{\jhname[Runar]{Törnqvist, Runar}}{1932--\allowbreak 1952}
Paul Runar Törnqvist, \textborn 1910 på Back hemman, köpte 1932 fastigheten Höglund 4:106. På tomten fanns en större vitmålad bostadsbyggnad. Runar och hans syster Hilma sålde 1949 sitt hemman på Back och byggde ett bostadshus på Lavast. Runar dog 1992.

\jhbold{Hyresgäster}
Telegrafisten Karl-Johan Erik Kuni, \textborn 31.03.1900 i Gamlakarleby och hans hustru Sigrid Gunhild Ottosdotter Strandberg, \textborn 28.11.1906 i Närpes hyrde 08.02.1933 bostadshuset. De hade gift sig 23.06.1928 i Närpes.
\begin{jhchildren}
  \item \jhperson{\jhname[Rolf Eric]{Törnqvist, Rolf Eric}}{19.10.1928 i Närpes}{}
  \item \jhperson{\jhname[Greta Helena]{Törnqvist, Greta Helena}}{16.08.1933 i Jeppo}{}
\end{jhchildren}
Efter kriget flyttade familjen till Vörå. Huset kom under en längre tid	att stå tomt och revs i början på 1950-talet. Viktor Smulter byggde	bostadshuset 1909, då han gifte sig.


%%%
% [occupant] Tikka
%
\jhoccupant{Tikka}{\jhname[Paul]{Tikka, Paul}}{1942--\allowbreak 1957}
Som flyktingar från Karelen, Ruskeala, kom änkefru Hilda Tikka, \textborn 10.06.1902, och hennes son Paul, \textborn 22.01.1928 och dotter Mirja, \textborn 22.11.1932, till Jeppo. Hildas man Herman hade dött 15.04.1938 och 2 döttrar var gifta och bosatta på annan ort. Tikka köpte fastigheten Dahlbo 4:127. Paul renoverade bostaden, som gick uppåt från bäcken, eller från väster mot öster, samt förstorade bostaden med ett rum i vedlidret. Hilda kom att ta hand om änklingen Lennart Hägglunds son Leo, som bodde hos henne fram till år 1957.

Mirja gifte sig med Tarmo Kalliosaari, \textborn 27.06.1934 i Alahärmä, se f.d. Fagerholms rivna hus, karta 3, nr 328 på Löte av Romar hemman.

Paul \textdied tragiskt 26.10.1974 --- Hilda \textdied 02.03.1983


%%%
% [occupant] Smulter
%
\jhoccupant{Smulter}{\jhname[Viktor]{Smulter, Viktor}}{1909--\allowbreak 1932}
I Jordregistret efter slutförd lantmäteriförrättning i början av 1900-talet	och vid klyvningsreglering 1915 av Silvast hemman har i registret införts	Smulter Viktor som ägare till stomlägenhet Dahlbo 4:12 om 0,0332 mantal av Silvast hemman. Kartograf Viktor Wilhelm Smulter, \textborn	12.04.1872 i Malax, gifte sig 17.01.1909 med Edla Kristina Dahlbo, \textborn 16.11.1877 i 	Pörtom. Smulter byggde bostadshuset (nr 377) på Höglund 4:106 samt huset med två lokaler och uthus vid bäcken på Dahlbo 4:127.

Fosterbarn: Kaino Rakel (senare gift Hakala) \textborn 24.07.1911 i Vasa. I vuxen ålder lärare till yrket.

Edla dog 04.03.1927. Viktor Smulter gifte sig andra gången 01.05.1928 med Aino Aili Ines Hairisto, \textborn 19.10.1879 i Uleåborg. De flyttade 30.06.1932 till Helsinge i Sverige.


\jhsubsubsection{Hyresgäster}
Byggnaden på Dahlbo 4:127 hade två bostäder i vinkel med vedlider och	utedass emellan.

Bostaden från väster till öster:
1933--\allowbreak 1936 hyrde en finsk familj, eventuellt med tillnamnet Ojala. De hade 4 barn.

Ifrågavarande bostad hyrde det nygifta paret Arthur och Lempi	Pelmas 1937--\allowbreak 1939. Arthur arbetade med dikesgrävning. Under den	här tiden föddes barnen Sulo Artturi 08.07.1937 och Karin	Margareta	08.10.1938. Vid krigsutbrottet 1939 flyttade familjen till	krigsmaterialfabriken i Yxpila, Gamlakarleby. År 1941 återkom familjen till Jeppo. Se mera karta 4, nr 323, 352 och karta 5, nr 86.

I bostaden, som sträckte sig från söder mot norr längsmed bäcken:
Här bodde	familjen Ylivainio 1939--\allowbreak 1946. Antti Andersson Ylivainio föddes 08.05.1895 i Alahärmä. Hans far var Anders Gustaf Kojonen och mor Anna Kaisa Andersdotter Tollikko från Tollikko i Jungar by. Anttis hustru var Ida Emilia Jakobsdotter född Försti 19.11.1897 i Kortesjärvi. Antti och Ida kom till Svartbacken på Jungar 09.04.1923 med sin 3 dagar gamla son Vilhelm. Från Svartbacken flyttade familjen med 4	barn till Tasko på Keppo, där 2 barn föddes. På grund av en eldsvåda 1934, som förstörde huset fick familjen hyra Villiam Stenbackas hus på	Back, där Veikko och Maunu är födda. Stenbacka sålde 1938 huset till Ensio Kula och familjen Ylivainio flyttade till kommunens hus på Stenbacken, där de bodde en kort tid. Antti arbetade hos de större 	bönderna bl.a. Leander Elenius, som han enligt dottern Margit tyckte om att arbeta med. Han fick arbete som hästkarl på Keppo gård till vilket hörde bl.a. att som kusk föra mjölk till järnvägsstationen för vidare transport till Malmska sjukhuset i Jakobstad. Antti Ylivainio var en av Jägarna, som utbildades i Tyskland.
\begin{jhchildren}
  \item \jhperson{\jhname[Anders Vilhelm]{Smulter, Anders Vilhelm}}{06.04.1923 i Alahärmä}{07.12.1996, Tammerfors}
  \item \jhperson{\jhname[Pauli Kalervo]{Smulter, Pauli Kalervo}}{09.12.1924 i Jeppo}{02.02.2005,Nykarleby}
  \item \jhperson{\jhname[Kaarlo Olavi]{Smulter, Kaarlo Olavi}}{24.07.1926 i Jeppo}{04.07.1981, Ilmajoki}
  \item \jhperson{\jhname[Leo Emil]{Smulter, Leo Emil}}{02.08.1928 i Jeppo}{20.05.1964, Ilmajoki}
  \item \jhperson{\jhname[Arvid Rainer]{Smulter, Arvid Rainer}}{15.09.1930 i Jeppo}{17.06.2006, Tammerfors}
  \item \jhperson{\jhname[Margit Anna-Lisa]{Smulter, Margit Anna-Lisa}}{06.11.1932 i Jeppo}{}
  \item \jhperson{\jhname[Veikko Johannes]{Smulter, Veikko Johannes}}{05.01.1935 i Jeppo}{08.06.1947, Jeppo}
  \item \jhperson{\jhname[Maunu Viljam]{Smulter, Maunu Viljam}}{18.10.1936 i Jeppo}{}
  \item \jhperson{\jhname[Reijo Ilmari]{Smulter, Reijo Ilmari}}{01.02.1940 i Jeppo}{}
\end{jhchildren}
De fyra äldsta barnen gick i svensk folkskola och de fem yngre gick i den nya finska skolan på Silvast. Se mera om barnen under Klings rivna hus,	karta 3-4, nr 323 vid Furubacken.

År 1939 flyttade familjen till Smulters hus på Silvast. Försvarsministeriet köpte 27.03.1941 Kiitola fabriker för laddning av	granater. Antti fick anställning som portvakt i vaktstugan vid infarten till Kiitola. Våren 1945 sändes de sista granaterna från Jeppo och Asevarikko flyttades till Gamlakarleby. Antti reste med tåg till arbetet på Varikko i Gamlakarleby.

Våren 1943 blev det på natten översvämning i Silvastbäcken vid Smulters. Vattnet kom in i bostaden upp till sängbottnen, och då dörren öppnades strömmade skor, nattkärl, mattor och Reijos gunghäst ut med vattnet. Familjen fick sitt hem 1946--\allowbreak 1949 i Klings hus nedanför Furubacken på östra sidan om 	järnvägsspåret. Se mera 4:181, figur 323. Efter kriget revs den översvämmade delen av byggnaden på Dahlbo 4:127. Margit Rintamäki, 83 år, har	skrivit några minnen på finska från tiden vid Smulters under krigstiden. Här en episod: ``Luoman vastarannalla oli ruotsinkielisten pommisuoja. Se oli metsikön suojaan kevyesti rakennettu ja toimi enemmänkin näkösuojana. Sodan loputtua pommisuojamme jäi tarpeettomana lahoamaan paikoilleen. Katon päällä oli maakerros, joka vuosien päästä romahti kun hevonen käveli katon päälle. Hevonen putosi kuoppaan selälleen ja kuoli.''  Maunu är musikalisk och har på äldre dagar gett ut CD-skivor.

1937--\allowbreak 1938 Inte känt vem som bodde i lokalen.

1931 eller 1932 flyttade Ellen och Alexander Sandqvist till Smulters	hus i lokalen längs med bäcken, med barnen Ragnar, Holger, Hjördis	och Marita, som är födda på Sandqvist 4:21, nr 62. Se även Sandqvist	2:98 på Romar hemman, kartblad 3, nr 38. Lea Elisabet föddes 28.06.1934 och Eva Anita 05.06.1936. De flyttade 1936 till Östermans hus på Grötas hemman och	tillbaka 1939 till Silvast, Lilland 4:36, nr 79, på andra sidan bäcken mitt emot Smulters hus. Lea kommer ihåg då hon såg lekkamraterna på andra sidan bäcken drabbas av översvämningen våren 1943. I december 1943 flyttade familjen Sandqvist till sitt nybyggda hus på Nylandsvägen.

Hyresgäster på 1920-talet och tidigare var antagligen anställda tjänstemän och arbetare på Statens Järnvägar i Jeppo.



%%%
% [house] Sandqvist \& Ström
%
\jhhouse{Sandqvist \& Ström}{4:21 resp. 4:90}{Silvast}{5}{62 och 362}

Styckad av stomlägenhet Fors 4:11

\jhhousepic{065-05603.jpg}{Huset Vovk sett från Centralvägen}

%%%
% [occupant] Vovk
%
\jhoccupant{Vovk}{\jhname[Serhii]{Vovk, Serhii} \& \jhname[Larysa]{Vovk, Larysa}}{2012--}
Svetsare Serhii Vovk, \textborn 08.10.1970 i Ukraina och hans hustru Larysa, \textborn 21.04.1974 i Ukraina, köpte ovannämnda fastigheter 2012. De flyttade från Elenius hus, nr 98 på Fors hemman med barnen.	De tre förstnämnda barnen är födda i Ukraina.
\begin{jhchildren}
  \item \jhperson{\jhname[Sergei]{Vovk, Sergei}}{18.09.1999}{}
  \item \jhperson{Dimitrij (Dima)}{06.05.2003}{}
  \item \jhperson{\jhname[Angelina]{Vovk, Angelina}}{12.09.2009}{}
  \item \jhperson{\jhname[Jan]{Vovk, Jan}}{29.09.2015	i Jeppo}{}
\end{jhchildren}

Familjen Vovk kom  till Jeppo år 2009. När Larysas far i Ukraina dog 	2012 kom i januari 2013 deras sjuka dotter Lena,  \textborn 20.06.1993 i Ukraina, till familjen i Jeppo. Serhii fick arbete 2013 på Jesika Oy i Jeppo, tidigare arbetade han på Lillbackas svinfarm i Kimo. Larysa har arbete på Oy Snellman Ab i	Jakobstad.

2015 grundrenoverade de huset, som försågs med betonggolv och tilläggsisolering. 2016 renoverade och inredde de ett kallrum på övre våningen.


%%%
% [occupant] Leppävuori
%
\jhoccupant{Leppävuori}{\jhname[Aki]{Leppävuori, Aki}}{2009--\allowbreak 2012}
På hösten 2009 köpte Aki Leppävuori fastigheterna Sandqvist 4:21 och Ström 4:90 på Stationsvägen 20. Aki är född 04.08.1980 i Karleby. Han arbetar på KWH-Mirka Ab. Aki bodde ensam i det stora huset. Efter	ett par år bjöd han ut huset till försäljning.


%%%
% [occupant] Teikari
%
\jhoccupant{Teikari}{\jhname[Pertti]{Teikari, Pertti} \& \jhname[Helena]{Teikari, Helena}}{2000--\allowbreak 2009}
Fastigheterna Sandqvist 4:21 och Ström 4:90 köptes våren 2000 av	Pertti Teikari, \textborn 1944 i Isojoki, och hans hustru Helena, \textborn 1941 i Simo. De bodde i Kemi, deras vuxna barn med familjer i Sverige. De renoverade huset själva under flera år. I maj 2004 flyttade de in i huset och samma sommar arbetade de med gårdsplanen och	trädgården och fick det att växa och blomma. På hösten 2009 sålde de plötsligt fastigheten och flyttade till Helenas födelseort Simo.


%%%
% [occupant] Kuoppala
%
\jhoccupant{Kuoppala}{\jhname[dödsbo]{Kuoppala, dödsbo}}{1992--\allowbreak 2000}
Den 24.10.1992 dog Elis Alexander Kuoppala och hans hustru Ester fortsatte att bo i nedre våningen och sonen Erik i bostaden på övre våningen. Ester Maria dog den 15.01.1995.


%%%
% [occupant] Kuoppala
%
\jhoccupant{Kuoppala}{\jhname[Elis]{Kuoppala, Elis} \& \jhname[Ester]{Kuoppala, Ester}}{1966--\allowbreak 1992}
År 1966 köptes fastigheterna av krigsinvalid Elis Alexander Kuoppala,	\textborn 05.08.1915 på Silvast, och hans hustru Ester Maria, född Mäkinen, \textborn 07.03.1914 i	Alahärmä. De flyttade från sitt egnahemshus på Centralvägen, Mellantomt 4:112, nr 66. Efter en mindre renovering öppnade Elis kafé/bar och en fotoaffär i den gamla butikslokalen, där han fotograferade människor och sålde filmer, kameror, fotoramar m.m. På 1980-talet avslutade Elis affärsverksamheten och gjorde en större renovering av huset och butikslokalen blev vardagsrum. Ester planterade buskar, träd och blommor på tomten.

Frånskilda sonen \jhname[Erik Alexander]{Kuoppala, Erik}, \textborn 24.04.1940 på Silvast/Mellantomt 4:112, återvände i början av 1970-talet från Sverige. 1974 gifte han sig med frånskilda Gunfrid Maria Evertsdotter Enlund, \textborn 20.01.1935 på Kaup hemman. De bosatte sig i övre våningen. Gunfrids yngre dotter, Rita Maria Janström, \textborn 30.10.1958 i Vexala, Munsala, bodde tidvis med \jhname[Gunfrid]{Kuoppala, Gunfrid} och ``Erkki'' och tidvis med sin mormor och morfar. Erkki arbetade då inom byggnadsbranschen och Gunfrid på KWH-Mirka. År 1985 köpte Gunfrid en aktielokal i Bostads Ab Lövgränd på Åkervägen och familjen flyttade dit. Skilsmässa 1989. Erkki dog i en bilolycka i Nykarleby. Han är begraven i Jeppo.
\begin{jhchildren}
  \item \jhperson{\jhname[Pia Minna Johanna]{Kuoppala, Pia Minna Johanna}}{16.09.1975}{08.06.2006}
\end{jhchildren}
Pia gifte sig i Nykarleby med Jari Häkki från Ylistaro och flyttade till Ylistario. Hon dog 2006, är begraven i Jeppo.


%%%
% [occupant] Granlund
%
\jhoccupant{Granlund}{\jhname[Cecilia]{Granlund, Cecilia} \& \jhname[Arthur]{Granlund, Arthur}}{1953--\allowbreak 1966}
Den 16.07.1953 erhöll Arthur och Cecilia Granlund, Åbo, lagfart	på Sandqvist 4:21 och Ström 4:90. Säljare var Cecilias styvmor Sofia Ström.\jhvspace{}


%%%
% [occupant] Ström
%
\jhoccupant{Ström}{\jhname[Sofia]{Ström, Sofia}}{1932--\allowbreak 1953}
Vilhelm Ströms andra hustru Edla Sofia, född Stor, \textborn 18.09.1878 i Kronoby, startade en filial 1930/1931 och övertog Alexander Sandqvists butiksverksamhet och fastighet. Sofia Ström erhöll lagfart på fastigheten 15.09.1932. Hon köpte ett tilläggsområde och förstorade tomten med Ström 4:90 samt gjorde en tillbyggnad till bostadshuset. Tillbyggnaden blev lokal för butiken, som flyttades från den röda magasinsbyggnaden. Sofia flyttade in i bostaden 18.10.1933 med styvdottern Hilma Edit Cecilia Ström, \textborn 20.11.1912 på Stenbacka. Cecilia gifte sig med Arthur Granlund från Åbo, dit paret flyttade. De fick en son Alf, \textborn 10.03.1940. Alf dog 05.09.1975 och Cecilia 03.12.2002 i Åbo.
\jhhousepic{Stationsv 1950.jpg}{Stationsvägen i mitten av 1950-talet}

\jhbold{1940--\allowbreak 1949}	År 1940 sålde Sofia affärsrörelsen och hyrde ut byggnaderna till Dagny Maria Wikman, \textborn 18.02.1916. Sofia flyttade till Kronoby, där hon dog 28.10.1963. Dagny Wikman var förlovad med Elis Valdemar Stenvall, \textborn 05.05.1914 på skattelägenhet Nygård 4:9 av Silvast hemman. Valdemar hade studerat i kooperativa handelsskolan i	Helsingfors. Före kriget arbetade han på Jeppo-Oravais Handelslag. Valdemar stupade i fortsättningskriget 20.07.1943. Dagny var en	omtyckt butiksinnehavare. I slutet av 1949 avslutade Dagny affärsverksamheten och flyttade till Vasa, där hon dog 03.10.2009.


\jhsubsubsection{Hyresgäster i nedre våningen}

\jhbold{1960--\allowbreak 1966}:  Efter att Posten flyttat till Björkhagen 4:160, nr 74, öppnade år 1960 Sulo Pelmas ett café med pub i nedre våningen. Efter	en kortare tid övertog Fanny Kennola verksamheten och Sulo emigrerade till Sverige. 1963 övertog Erkki och Marketta Kuoppala	caféverksamheten och bosatte sig i huset. De har en son Janne och en dotter Jaana. År 1966 blev det skilsmässa och Marketta flyttade med barnen till Jakobstad och Erkki emigrerade till Sverige. 1966 köpte Elis och Ester Kuoppala fastigheten.

\jhbold{1950--\allowbreak 1959}:  Den 01.09.1950 blev Jeppo Postexpedition självständig post- och telegrafexpedition och flyttade från Järnvägsstationen till Ströms hus.	Butikslokalen och hörnrummet användes för Postens behov fram till	1956 och därefter hela nedre våningen till slutet av 1959, då Jeppo Post- och telegrafexpedition flyttade till specialbyggda utrymmen på	andra sidan Stationsvägen i Erland och Maire Sundells nybyggda hus,	Björkhagen 4:160, nr 74. Föreståndare Einar Nybäck hyrde 1950--\allowbreak 1953 köket och sovrummet.

Under tiden 1950--\allowbreak 1959 som Jeppo Postexpedition fanns i Ströms hus arbetade där följande befattningar och personer.

Post-och telegrafexpeditionsföreståndare:

\begin{tabular}{l l}
  \jhname{Nybäck, Einar}    & 1950--\allowbreak 1953 \\
  \jhname{Sundell, Maire}   & 1953--\allowbreak 1978 \\
  \jhname{Lawast, Anita}    & 1978--\allowbreak 1990 \\
\end{tabular}

Posttjänstemän:

\begin{tabular}{l l}
  \jhname{Sandler, Elna Sofia}  & 1948--\allowbreak 1953 \\
  \jhname{Roos, Anne-Maj}	      & 1953--\allowbreak 1958 \\
  \jhname{Nyman, Birgit}        & 1954--\allowbreak 1956 \\
  \jhname{Snickars, Marith}	    & 1958--\allowbreak 1962 \\
  \jhname{Lawast, Anita}        & 1958--\allowbreak 1978 \\
\end{tabular}

Hela anställningstiden för de anställda anges, även efter att Posten flyttade från Ströms hus.

Postiljoner/Lantbrevbärare:

\begin{tabular}{l l}
  \jhname{Sandqvist, Ellen}     & 1941--\allowbreak 1961 \\
  \jhname{Sandqvist, Alexander} & 1951--\allowbreak 1956 \\
  \jhname{Strengell, Bengt}     & 1957--\allowbreak 1962 \\
\end{tabular}

Förutom paket, tidningar och brev till Jeppo kom även motsvarande till Oravais, Munsala, Nykarleby stad samt delar av Nykarleby landskommun via Jeppo. Till postiljonen hörde förutom utdelningen runtom i Jeppo även att ta emot posten vid morgon-, kvälls- och nattposttågen. Postsäckarna till de andra orterna sändes med Haldin \& Roses bussar och senare med Postbussen.

Maire Malinen, \textborn 30.06.1927 i Jaakkima i Karelen, hade arbetat som postfröken några år bl.a. i Oravais och Kronoby. Maire gifte sig 1953	med Erland Sundell, \textborn 25.11.1928 på Silvast. Samma år fick Maire tjänst som post-och telegrafexpeditionsföreståndare i Jeppo. Maire och Erland fick sitt första hem i köket och kammaren i nedre våningen. Deras son Christer föddes 27.03.1954. År 1956 flyttade familjen till Jungbo	4:96, nr 63, där de bodde när deras dotter Britt-Marie föddes 28.06.56.	Efter att fam. Sundell flyttat, använde Posten hela nedre våningen.


\jhsubsubsection{Hyresgäster i övre våningen}

\jhbold{1973--\allowbreak 1985}:  Erkki och Gunfrid Kuoppala, se ovan under ägare.
\jhbold{1970--\allowbreak 1972}:  Lastbilschaufför Terho Ekola och hans hustru Gun-Lis med barnen Rita och Margita. Se mera Bostads Ab Gökbrinken, nr 76 och Silva 4:43 nr 70.
\jhbold{1963--\allowbreak 1966}:  Cafèägare Erkki och Marketta Kuoppala. Se ovan.
\jhbold{1954--\allowbreak 1958}:  Posttjänsteman Anne-Maj Roos från Sydösterbotten. Hon dog i cancer 1958.
\jhbold{1950--\allowbreak 1954}:  Telefonist Else-Maj Johansson, \textborn 13.10.1927 i Oravais. Else-Maj var förlovad med Erik Stenbacka, född på Stenbacka hemman. Erik hade vistats i Canada några år, men återvänt till Jeppo. Erik återvände till Amerika och förlovningen bröts. Else-Maj flyttade till Sverige, där hon dog 16.2.2010.
\jhbold{1949--\allowbreak 1950}:  I juni 1949 flyttade Runar Holmlund, \textborn 09.12.1923 i Seinäjoki och hans hustru Blondine, \textborn 08.02.1925 i Solf, till övre våningen i Ströms hus. Deras son Johan Erik föddes 18.10.1949, då de bodde i huset. Runar arbetade på kontoret vid SJ i Jeppo. På sommaren 1950 flyttade de till Thors fastighet 4:143 och därifrån 1957 till Sand-Ek 4:134. Se mera karta 2, nr 19 och karta 5, nr 84.


%%%
% [occupant] Sandqvist
%
\jhoccupant{Sandqvist}{\jhname[Alexander]{Sandqvist, Alexander}}{1920--\allowbreak 1931}
Efter att Jeppo Handelslag 09.04.1916 köpt fastighet 4:57, nr	357, och flyttat affärsverksamheten dit, fortsatte Alexander Sandqvist	1920 med kolonialvaruhandel i den av Jeppo Lantmannagille 1904 byggda gillesbutiken, nr 362.

\jhhousepic{Stationsv 1930.jpg}{Stationsvägen 1930 med Sandqvists bostadshus t.v. vänster, HAB-banken och Jakobssons hus.}

Gustav Alexander Sandqvist, \textborn 26.10.1894 på Svartbacken på Jungar, köpte området med byggnaderna och efter styckning	erhöll han lagfart år 1921 på fastighet Sandqvist 4:21. Den 04.11.1923 vigdes han med Ellen Johanna, född Bro, \textborn 08.07.1903 i Markby, Nykarleby lk. Alexander och Ellen byggde genast ett bostadshus i stock från Markby på tomten vid vägkorsningen. Alexander hade kolonial- och matvaruaffär fram till de dåliga åren i början av 1930-talet, då Sofia Ström övertog affärsverksamheten och fastigheten.

Alexander och Ellen fick 4 barn i sitt bostadshus.
\begin{jhchildren}
  \item \jhperson{\jhname[Ragnar Gustav]{Sandqvist, Ragnar Gustav}}{24.10.1924}{}
  \item \jhperson{\jhname[Holger Alfred]{Sandqvist, Holger Alfred}}{13.09.1926}{}
  \item \jhperson{\jhname[Hjördis Johanna]{Sandqvist, Hjördis Johanna}}{12.02.1928}{}
  \item \jhperson{\jhname[Svea Marita]{Sandqvist, Svea Marita}}{12. 11.1929}{}
\end{jhchildren}
Lea och Eva föddes när de bodde på Dahlbo 4:27, nr 377. Se mera Sandqvist 2:98 på Romar hemman, nr 38.

När Silvast Marthakrets före kriget skaffade mangel, placerades den i det gamla butiksmagasinet och användes flitigt. Av Jeppo kommun köpte Silvast Marthakrets den 28.08.1951 Björkbacka 4:38, som tidigare ägts av familjen Sundqvist. Mangeln flyttades till det inköpta huset och den röda magasinslängan på 4:21 revs.


%%%
% [occupant] Jeppo
%
\jhoccupant{Jeppo}{\jhname[Handelslag]{Jeppo, Handelslag}}{1914--\allowbreak 1916}
År 1914 bildades Jeppo Handelslag med verksamhet i gilleshuset. Den 09.04.1916 köpte Jeppo Handelslag fastigheten 4:57, nr 357, vid vägkorsningen till Stationsvägen, och verksamheten flyttades dit.


%%%
% [occupant] Lantmannagille
%
\jhoccupant{Lantmannagille}{\jhname[Jeppo]{Lantmannagille, Jeppo}}{1904--\allowbreak 1913}
Den 21.04.1901 grundades Jeppo Lantmannagille för att förbättra jordbrukarnas inköp och försäljning av jordbruksprodukter och anskaffning av maskiner för uthyrning. Vid sekelskiftet övergick man från självhushållning till en försiktig marknadsinriktad ekonomi i en strävan att förbättra tillvaron. På hösten 1902 hyrdes ett hus nära stationen av Isak Viklund	(troligen på tomten där Bostads Ab Gökbrinken nu finns). På hösten 1904 byggde Lantmannagillet ett gilleshus/butik vid Stationsvägen, nr 362. Lanthandeln gjorde att gillets medlemsantal steg hastigt, år 1910 fanns	248 medlemmar av jordbrukare, arbetare och tjänstemän, även från 	Pensala och Ytterjeppo. Lanthandeln upphörde 1913 och år 1919 hade 	Lantmannagillet endast 19 medlemmar. Lantmannagillet hade dock en livlig verksamhet med bland annat föreläsningar, kurser och 	utställningar. År 1924 hade Lantmannagillet 72 medlemmar. Jeppo	Lantmannagillet har under 1900-talet tagit ett stort ansvar för bygdens utveckling.



%%%
% [house] Härbärge
%
\jhhouse{Härbärge}{4:161}{Silvast}{5}{69}

Styckad av stomlägenhet Bäck 4:16

\jhhousepic{058-05591.jpg}{Sandvik Bernt och Lina}

%%%
% [occupant] Sandvik
%
\jhoccupant{Sandvik}{\jhname[Bernt]{Sandvik, Bernt} \& \jhname[Lina]{Sandvik, Lina}}{2008--}
Sommaren 2008 köpte Bernt Erik Andreas Sandvik, \textborn 19.05.1977 på Keppo, och hans hustru Lina Gun-Britt Carita, född Johansson, \textborn 30.05.1982 i Jakobstad, fastigheten Härbärge 4:161. Efter studentexamen 1996 studerade Bernt vid Svenska yrkeshögskolan och blev Musiker 2002, samt Musikpedagog 2011 från Yrkeshögskolan Novia i Jakobstad. Han arbetade 2002--\allowbreak 2012 som musiker i Österbottens Militärmusikkår. År 2012 fick han tjänst på KWH-Mirka Ab, och genom 	läroavtalsutbildning Mirka-Optima fick han examen 2015 som processkötare inom pappers- och kemisk industri. Lina blev student 2001 och fick Juris kandidat-examen 2006 vid Umeå Universitet, samt	blev 2010 Vicehäradshövding, Österbottens Tingsrätt. Arbetsplatser: Servicefacket PAM 2005--\allowbreak 2007, Flyktingrådgivningen rf 2010--\allowbreak 2011, OK Indrivning 2013--\allowbreak 2015, Skatteförvaltningen 2015-.
\begin{jhchildren}
  \item \jhperson{\jhname[Ivar Eliel]{Sandvik, Ivar Eliel}}{17.10.2007 i Vasa}{}
  \item \jhperson{\jhname[Arvid Wilhelm]{Sandvik, Arvid Wilhelm}}{05.01.2009 i Jeppo}{}
  \item \jhperson{\jhname[Vidar Karl]{Sandvik, Vidar Karl}}{02.10.2010 i Jeppo}{}
  \item \jhperson{\jhname[Enid Linnéa]{Sandvik, Enid Linnéa}}{04.05.2012 i Jeppo}{}
\end{jhchildren}


%%%
% [occupant] Norrgård
%
\jhoccupant{Norrgård}{\jhname[Ronny]{Norrgård, Ronny}}{2004--\allowbreak 2008}
Ronny Jan Olav Norrgård, \textborn 27.06.1977 på Keppo, köpte	fastigheten 02.09.2004. Ronny arbetar som processkötare och huvudförtroendeman på Oy KWH-Mirka Ab i Jeppo. År 2010 gifte sig Ronny med Katja, född Viiliäinen, \textborn 13.10.1976 i Joensuu.
\begin{jhchildren}
  \item \jhperson{\jhname[Sofia]{Norrgård, Sofia}}{2007}{}
  \item \jhperson{\jhname[Zara]{Norrgård, Zara}}{2009}{}
  \item \jhperson{\jhname[Maya]{Norrgård, Maya}}{2011}{}
\end{jhchildren}
Ronny och Katja byggde nytt bostadshus på Holmen och flyttade dit sommaren 2008.


%%%
% [occupant] Lindström
%
\jhoccupant{Lindström}{\jhname[Tomas]{Lindström, Tomas}}{1989--\allowbreak 2004}
Den 08.09.1989 övertog Tomas Leif Erik Lindström, \textborn 04.09.1960, av sina föräldrar barndomshemmet Härbärge 4:161 av Bäck 4:16 på Silvast. Tomas arbetade då som maskinmästare på lastbåtar. Åren 1994--\allowbreak 1995 byggde han ett dubbelgarage på 90 m2 och höjd på 5 m. Under samma tid träffade Tomas Satu Johanna Svartsjö, \textborn 10.04.1969 i Karleby, och 	hon blev sambo med Tomas. Tomas och Satu vigdes i Karleby kyrka 23.08.1997.
\begin{jhchildren}
  \item \jhperson{\jhname[Robin Willhelm]{Lindström, Robin Willhelm}}{04.03.1997}{23.04.1997}, föddes med hjärtfel
  \item \jhperson{\jhname[Robert Eric]{Lindström, Robert Eric}}{29.07.1998}{}
  \item \jhperson{\jhname[Jessica Iris Aurora]{Lindström, Jessica Iris Aurora}}{17.01.2001}{}
  \item \jhperson{\jhname[Venla Helinä Elanor]{Lindström, Venla Helinä Elanor}}{04.06.2008}{}
\end{jhchildren}
År 1999 totalrenoverades bostadshuset och blev tillbyggt med 90 m2.	Tomas fick tjänst i Vasa och familjen flyttade till ett hyreshus i Petsmo 14.11.2004. De byggde ett bostadshus i Karperö, som blev inflyttningsklart i februari 2006. Tomas arbetar sedan 2004 på Trafiksäkerhetsverket som sjöfartsöverinspektör med att besikta och kontrollera fartyg. Satu arbetar på HVC i Smedsby som instrumentvårdare.


%%%
% [occupant] Lindström
%
\jhoccupant{Lindström}{\jhname[Evert]{Lindström, Evert} \& \jhname[Iris]{Lindström, Iris}}{1959--\allowbreak 1989}
När Ida Lindström och barnen Evert och Anni 1959 sålde tomten, där deras gamla bostadshus och uthus fanns av lägenheten Härbärge 4:24, fick resten av lägenheten ny registernummer, 4:161. Isak Evert Lindström, \textborn 24.02.1913 på stomlägenhet 4:7 av Silvast hemman, gifte sig 1959 med Iris Runhild, född Lingonblad, \textborn 26.04.1923 i Kimo, 	Oravais. Evert och Iris byggde 1959 ett nytt bostadshus, 90 m2, på åkern vid Centralvägen med rum även för mor Ida.

Evert arbetade som 	stationskarl på SJ i Jeppo till sin pensionering. Iris blev mejerist 1948 och kom från Nederpurmo 1951 till Jeppo Andelsmejeri, där hon arbetade till 1979, då verksamheten flyttades till Milka i Kaitsor, Vörå, och Iris följde med. Under veckans arbetsdagar bodde Iris och Evert i ett radhus i Kaitsor. Efter 1981 arbetade hon en kort tid på Milka i Vasa. Hon kom tillbaka till Jeppo-mejeriet, där hon sålde foder fram tills hon gick i pension 29.04.1988.

Barn: 	\jhbold{Tomas} Leif Erik,	\textborn 04.09.1960

Efter examen från yrkesskolan i Jakobstad blev Tomas sjöman. Han gick i Ålands Tekniska skola 1981--\allowbreak 1984 och blev utbildad övermotormaskinmästare och arbetade nästan 20 år på sjön. Evert och Iris köpte en höghuslokal i Nykarleby, då Tomas övertog fastigheten. Everts tid i Nykarleby blev kort, han dog 11.01.1991. Iris dog 16.11.2011. De är begravna i Jeppo.


%%%
% [occupant] Lindström
%
\jhoccupant{Lindström}{\jhname[Ida]{Lindström, Ida}}{1920--\allowbreak 1959}
Efter att Ida Lindström 1916 sålt skattelägenhet Lindström 4:7 till sin	kusin Anders Lindén, köpte hon 0,0072 mantal av skattelägenhet Bäck 4:16. Säljare bonden Isak Lindfors. På området vid nuvarande Centralvägen, där bostadshuset står, fanns då en ria (nr 369, se bild under nr 63) och en lada, som troligen byggdes i slutet av 1800-talet av Lindfors. Byggnaderna revs efter vinter-/fortsättningskriget. Ida Maria Lindström dog 09.05.1965 (karta 5, nr 369).



%%%
% [house] Silva
%
\jhhouse{Silva}{4:43}{Silvast}{5}{70}

Styckad av stomlägenhet Bäck 4:16

\jhhousepic{056-05697.jpg}{Jonas Cederström och Paulina Peltorinne}

%%%
% [occupant] Cederström \& Peltorinne
%
\jhoccupant{Cederström \& Peltorinne}{\jhname[Jonas]{Cederström \& Peltorinne, Jonas} \& \jhname[Paulina]{Cederström \& Peltorinne, Paulina}}{2012--}
Den 08.11.2012 köpte VVS-montör Jonas Cederström fastigheterna Silva R:nr 4:53 och Nybygge R:nr 4:28. Jonas är född 23.07.1980 på Svartbacken i Jungar by. Han blev student från Topeliusgymnasiet 1999 och IT-tradenom från Vasa Yrkeshögskola 2006. Jonas' sambo Paulina Peltorinne, \textborn 13.11.1979 i Björneborg, blev student från Björneborgs Vuxengymnasium år 2000. Paulina är bilskattespecialist på BMW och är egen företagare. De har bott och arbetat i Stockholm, därifrån de återvände 01.07.2011 till Jeppo och bodde tillfälligt i Cederströms byggnad på Grötas, karta 10, nr 11.
\begin{jhchildren}
  \item \jhperson{\jhname[Noah]{Cederström, Noah}}{19.07.2012}{}
  \item \jhperson{\jhname[Selma]{Cederström, Selma}}{11.08.2015}{}
\end{jhchildren}


%%%
% [occupant] Höglund \& Blomqvist
%
\jhoccupant{Höglund \& Blomqvist}{\jhname[Carl-Henrik]{Höglund \& Blomqvist, Carl-Henrik} \& \jhname[Mariann]{Höglund \& Blomqvist, Mariann}}{2008--\allowbreak 2012}
I maj 2008 köpte Carl-Henrik Höglund, \textborn 31.10.1983 på Keppo, fastigheterna Silva 4:53 och Nybygge 4:28. Sambo Mariann Blomqvist, \textborn 16.07.1984 i Forsby, Pedersöre. Calle blev student från Topelius Gymnasium och ingenjör från Yrkeshögskolan i Vasa. Han har arbetat på KWH-Mirka, men är numera egen företagare. Mariann är utbildad barnträdgårdslärare.
\begin{jhchildren}
  \item \jhperson{\jhname[Alexander Höglund]{Höglund, Alexander Höglund}}{19.02.2009}{}
  \item \jhperson{\jhname[Marielle Höglund]{Höglund, Marielle Höglund}}{24.10.2011}{}
  \item \jhperson{\jhname[Emil Höglund]{Höglund, Emil Höglund}}{12.05.2015}{}, har inte bott i huset
\end{jhchildren}
Carl-Henrik och Mariann grundrenoverade och inredde övre våningen i bostadshuset samt planerade och planterade den förstorade tomten. År 2011--\allowbreak 2012 byggde de ett nytt större bostadshus på Keppo och flyttade dit på hösten 2012.


%%%
% [occupant] Gething
%
\jhoccupant{Gething}{\jhname[David]{Gething, David} \& \jhname[Ulla]{Gething, Ulla}}{2004--\allowbreak 2008}
År 2004 köpte David och Ulla Gething Silva 4:43. Ulla är född 06.03.1946 i Vasa, med rötter i Socklot. David är född i Wales, England. De har arbetat och bott i olika länder och världsdelar under många år. Ulla har två döttrar och David en dotter från tidigare äktenskap. De bor i England. År 2006 köpte Ulla och David granntomten Nybygge 4:28. Den 15 maj 2008 flyttade Ulla och David till Bergö, där de hade köpt ett hus, som de renoverade.


%%%
% [occupant] Häggman
%
\jhoccupant{Häggman}{\jhname[Anders]{Häggman, Anders} \& \jhname[Christine]{Häggman, Christine}}{1990--\allowbreak 2004}
HAB-bankhuset, Silva 4:43, köptes 1990 av Gerd Christine-Louise Julin-Häggman och hennes man Leif Anders Häggman. Christine är född på Silvast (nr 49), \textborn 16.02.1961 och Anders i Nedervetil, \textborn 16.03.1961. De vigdes i Jeppo 15.06.1985.

Christine blev musikmagister 1987 och fick pianodiplomexamen 1990 vid Sibelius-Akademin. Hon arbetar vid Musikhuset i Jakobstad och har sedan 1990 pianolärartjänst vid Jakobstads musikinstitut, samt timlärare vid YA och Novia. Christine har spelat dragspel och fiol från barndomen och lärt sig Jeppo folkmusik på gehör av de gamla spelmännen. År 1994 startade hon folkmusikgruppen Jepokryddona, som gett ut CD skivor samt spelat och dansat menuetter och polskor i bl.a. Nordamerika, Australien, Ryssland och Europa. Christine och Jepokryddona har erhållit flera pris och utmärkelser, senast i februari 2017. År 2000 utsågs Christine till årets Jeppobo. Vid Kaustby folkmusikfestival 2011 erhöll Christine titeln Mästerspelman.

Anders blev skogsbruksingenjör från Ekenäs Forstinstitut 1986, och fick tjänst som områdesförman för Enso-Gutzeit på Kimitoön. 1990 blev han disponent för Jeppo skogsandelslag samt därtill verksamhetsledare för Jeppo skogsvårdsförening 1997. År 2002 blev han Vd vid Kvevlax Handelslag fram till 2006. Från 2007 arbetar Anders som ombud för Pohjola Försäkring.
\begin{jhchildren}
  \item \jhperson{\jhname[Johan Alexander]{Häggman, Johan Alexander}}{25.01.1991}{}
  \item \jhperson{\jhname[Doris Carina Anneline]{Häggman, Doris Carina Anneline}}{16.04.1996}{}
\end{jhchildren}
Johan blev student 2009 vid Topeliusgymnasiet i Nykarleby. Han arbetar på Jeppo Potatis Ab. Doris blev student från samma skola 2015. Anders och Christine byggde ett större stockhus på Bärs. Familjen flyttade 2004 och sålde sitt hus på Silvast.


%%%
% [occupant] FNB
%
\jhoccupant{FNB}{\jhname[HAB]{FNB, HAB}}{1928--\allowbreak 1990}
Den 01.12.1923-15.09.1928 verkade Ab Unionbanken i Finland i Otto Jakobssons gård, Nybygge 4:28. Den 15.09.1928 flyttade banken till sitt nybyggda hus på granntomten Silva 4:43. Ab Unionbanken i Finland var tidigare Lantmannabanken, som 14.10.1919 öppnade ett kontor i Ida Lindströms gård. Redan 31.12.1919 flyttade Lantmannabanken till Jeppo Handelslags kontor, där de verkade 01.01.1920-30.11.1923. Lantmannabanken fusionerades 01.01.1921 med Wasa Aktiebank, som kom till Jeppo 05.11.1919. Namnet ändrades 01.04.1924 till Ab Unionbanken i Finland, som 01.07.1931 fusionerades med Helsingfors Aktiebank.

Föreståndare/bankdirektörer:

\begin{tabular}{l l}
  1919--\allowbreak 1921  & \jhname[Johan Jungar]{Jungar, Johan},	jordbrukare \\
  1921--\allowbreak 1922  & \jhname[Fredrik Thors]{Thors, Fredrik}, lärare \\
  1922--\allowbreak 1926  & \jhname[Gustav Liljeqvist]{Liljeqvist, Gustav}, hlg-förest. \\
  1926--\allowbreak 1952  & \jhname[Gunnar Wadström]{Wadström, Gunnar}, fil.mag., direktör \\
  1952--\allowbreak 1961  & \jhname[Bengt Blom]{Blom, Bengt}, samskola, direktör \\
  1961--\allowbreak 1965  & \jhname[Tor Hagberg]{Hagberg, Tor}, merkant, direktör \\
  1965--\allowbreak 1967  & \jhname[Ole Fagernäs]{Fagernäs, Ole}, direktör \\
  1967--\allowbreak 1985  & \jhname[Allan Blom]{Blom, Allan},	merkonom, direktör \\
\end{tabular}

Banktjänsteman/kassör:

\begin{tabular}{ll}
  1928--\allowbreak 1939  & \jhname{Modén, Aina} \\
  1940--\allowbreak 1946  & \jhname{Hästbacka, Ruth} \\
  1946--\allowbreak 1961  & \jhname{Romar, Dagny} \\
  1962--\allowbreak 1985  & \jhname{Grahn, Svea} \\
  1985--\allowbreak 1989  & \jhname{Nymark, Gunhild} \\
  1985--\allowbreak 1988  & \jhname{Gunnar, Yvonne} \\
  1990-                   & \jhname{Carlstedt, Raul} \\
\end{tabular}

Dagny Romar önskade prova något nytt när hennes döttrar var klara med sin utbildning och flyttade till HAB i Vasa 1962. Dagny kom tillbaka till Jeppo 1964 och fortsatte sitt arbete på HAB:s kontor i Nykarleby till sin pensionering 1970. År 1985 köpte Nordiska Föreningsbanken upp Helsingfors Aktiebank. 1986 fusionerades HAB med NFB och HAB:s kontor tömdes. De anställda fortsatte i NFBs kontor i Andelsringens byggnad. Fr.o.m. 1956 och till 1986 hyrdes bostaden av utomstående familjer.

\jhbold{Bostaden i Silvia} 4:43

1988:
Lars och Ann-Christin Dahlkvists nyinköpta hus, Höglund 4:106 skadades i en brand 01.10.1988. Under renoveringstiden i 3 månader bodde familjen i bostadslokalen i nedre våningen.


1985--\allowbreak 1986:
\jhname[Bjarne Kurt Eric Hongisto]{Hongisto, Kurt}, \textborn 14.01.1947 i Lassila och hans hustru Siv Anita Linnéa, född Rönnholm, \textborn 16.07.1947 i Nedervetil, hyrde ca 2 år bostaden. Kurt arbetade som lastbilschaufför och de hade en mink- och rävfarm i Gränden i Lassila by. Barn; Mikaela Linnéa Henriette, \textborn 28.12.1984 i Gränden. Familjen flyttade 1986 till stadens hyresbostäder i f.d. Pastellen på Åkervägen och 1989 till Vasa Andelsbanks direktörsbostad. Se mera Andelsbanken 4:199, nr 92.


1975--\allowbreak 1982:
Familjen Ekola:	Lastbilschaufför \jhname[Terho Ekola]{Ekola, Terho}, \textborn 26.08.1943 i Ekola, Voltti och hans hustru Gun-Lis, född Jakobsson, \textborn 19.04.1947 i Tålamods, Vörå, samt deras döttrar Rita Anne Helena, \textborn 16.01.1967 i Vörå och Margita Pia Louise, \textborn	17.01.1970 i Jeppo, flyttade 1975 till Silvia från Bostads Ab	Gökbrinken. Gun-Lis var sömmerska och öppnade en ateljé. Senare höll Gun-Lis Arbetarinstitutkurser i olje- och porslinsmålning. Terho arbetade då som mjölkbilschaufför för trafikant Paul Sandström. År 1982 köpte de Linnea och Karl Sunngrens hus på Böös, dit de flyttade på nyårsafton 1982. Deras tredje dotter, Lisa, föddes på Böös, \textborn 18.03.1985.


1956--\allowbreak 1975:
Familjen Kronlöf:	Efter att bankdirektör Bengt Blom med familj flyttat till	Oravais, hyrde telegrafist \jhname[Rafael Kronlöf]{Kronlöf, Rafael}, \textborn 08.11.1924 i Oravais, och hans hustru Mary, född Isakas, \textborn 18.03.1929 i Vörå, bostaden. De flyttade 1956 från Sandbergs hus, Strandbo 4:73, nr 48, med barnen.
\begin{jhchildren}
  \item \jhperson{\jhname[Lisbeth]{Kronlöf, Lisbeth}}{05.07.1951}{}
  \item \jhperson{\jhname[Stig]{Kronlöf, Stig}}{04.04.1954}{}
  \item \jhperson{\jhname[Tom J]{Kronlöf, Tom}}{19.02.1957}{}
\end{jhchildren}
Lisbeth och Stig föddes när familjen bodde i Thors fastighet, R:nr 4:143, karta 2, nr 19, och Tom när de flyttat till HAB-bostaden. Både Stig och Tom blev diplomingenjörer och	arbetar inom företagsvärlden, Stig i Lappfjärd, Kristinestad och Tom i Göteborg. Stig är gift med Eivor, född Nybäck i Pedersöre. De har tre barn, Markus, Maria och Linda. Tom är gift med Greta, född Häggblom i Vörå, och de har en son Jakob. Lisbeth blev kemilaborant och emigrerade till Rättvik	Sverige. Lisbeth är gift med Caj Raunio, och de har två söner, Johan och Anders.

Till familjen hörde även Rafaels far, folkskollärare Johan Kronlöf, \textborn 25.11.1875 i Vörå. Han kom till Jeppo 15.12.1954 och dog 22.02.1959. Han är begraven i Vörå.

Efter att Rafael Kronlöf blivit civil från kriget 1945, fick han	tjänst som telegrafist på Jeppo station 1949. Han hade gått i	Vasa pojklyceum och Statsjärnvägarnas tjänstemannautbildning.

Rafael och Mary gifte sig 17.04.1949. De flyttade till Jeppo 20.07.1949. Sitt första hem fick de i nedre våningen i Thors bostadshus, nr 19.  Rafael var aktiv i veteranföreningen från att den startade 1968 och ordförande för Nykarlebynejdens krigsveteraner 1968--\allowbreak 1984. År 1975 utnämndes Rafael till stationsinspektör och de flyttade till s.k. ``Hästbacka huset'' på stationsområdet. Barnen var utflugna och Rafael och Mary flyttade 12.06.1984	till eget hus i barndomens hemby, Koskeby i Vörå.	Rafael pendlade till Jeppo 1984--\allowbreak 1988, då han gick i pension sista maj. Mary arbetade på KWH-Mirka på 1970-talet fram till pensioneringen, då hennes arbetsplats var i Oravais.  Hennes intressen var hemmet och handarbeten.

Rafael \textdied 29.07.2005  ---  Mary \textdied 16.11.2013. De är begravna i Vörå.


Bankdirektör åren 1961--\allowbreak 1965, \jhname[Tor Hagberg]{Hagberg, Tor}, \textborn 31.07.1929 i Kaitsor, Vörå, bodde inte i bankens bostad utan med sin syster, mejeriassistent Gerda Hagberg i Leo och Margit Sandbergs hus på Finskas hemman. Tor hade studerat vid Evangeliska folkhögskolan och vid Handelsskolan i Vasa. Under sin tid i Jeppo var han aktiv inom Evangeliska Unga i	Jeppo som ungdoms- och juniorledare. Han flyttade från Jeppo 1965 till Stadsfogdekontoret i Vasa därifrån han avgick med pension 1992. Tor dog hastigt 10.01.2009 och hans hustru Benita 15.08.2010.


1952--\allowbreak 1956:
\jhname[Bengt Erik Johan Blom]{Blom, Bengt} blev bankdirektör på Helsingfors Aktiebank i Jeppo 11.11.1952--\allowbreak 1961, samt även i Oravais 1956--\allowbreak 1962. Familjen flyttade till Oravais i slutet av 1956. Bengt är född i Vörå den 26.04.1927 och gick i Vörå Samskola samt bokförings- m.fl. kurser. Tiden innan han kom till Jeppo arbetade han som prokurist 1950--\allowbreak 1952 på HAB i Vörå och som posttjänsteman 1945--\allowbreak 1950. År 1948 gifte sig Bengt med dagklubbsledare Gunnel Maria, född Karlsson 28.07.1927 i Vörå. Bengts intressen är kristen verksamhet, sång, musik, humanitärt hjälparbete och på äldre dagar även släktforskning. Han var aktivt med när det första Missionslägret startade 1956 i Jeppo.
\begin{jhchildren}
  \item \jhperson{\jhname[Benita Gunborg]{Blom, Benita Gunborg}}{05.02.1951 i Vörå}{}
  \item \jhperson{\jhname[Dorrit Anita]{Blom, Dorrit Anita}}{27.08.1953 i Jeppo}{}
  \item \jhperson{\jhname[Bertil Michael]{Blom, Bertil Michael}}{06.09.1958 i Oravais}{}
\end{jhchildren}
Merkonom Benita Salonen har arbetat som bankrådgivare vid Nordea i Vasa. I slutet av 1962 blev Bengt direktör vid Livförsäkringsbolaget Suomi i Vasa, och familjen flyttade till Korsholm. Den 01.06.1967 återvände han till HAB som Expeditionschef i Vasa, därifrån i pension  31.10.1986. Gunnel avled den 25.02.2016 i Korsholm.


1926--\allowbreak 1952:
Filosofiemagister \jhname[Gunnar Wadström]{Wadström, Gunnar}, \textborn 30.05.1895 i Lappfjärd/Dagsmark, blev 1926 direktör i Jeppo för 	Unionbanken i Finland då banken ännu verkade i Otto Jakobssons hus. Då det nya bankhuset 1928 blev färdigt flyttade Gunnar in i direktörsbostaden. Gunnar hade deltagit i frihetskriget i Kristinestad och Björneborg och efter kriget studerade han till magister i Nordisk filologi vid Åbo Akademi. I slutet av 1920-talet kom även Ester Sofia Munsin till Jeppo som postföreståndare för Posten, som då hade sin verksamhet i stationsbyggnaden. Ester var född i Munsala 24.08.1900 och efter avlagd mellanskolexamen i Jakobstad fick hon anställning på posten på olika orter; Simo, Pargas, Sastmola och Jeppo. Gunnar och Ester vigdes på prästgården den 29.06.1930.

Barn:	\jhname[Johan Gunnar]{Wadström, Johan}, \textborn 07.01.1934

Gunnar Wadström var socialt intresserad och deltog flitigt i olika verksamheter i Jeppo. Han var medlem och	ordförande 	i kommunalnämnden, aktiv medlem i jaktföreningen och idrottsföreningen, arbetade som skyddskårschef och under kriget som regionchef för skyddskårerna från Vörå i söder till Esse i norr. Gunnar fungerade även som evakueringschef dvs som förbindelseman mellan de evakuerade och lokalbefolkningen. Kemijärviborna hade med sig samskolan och lärarna, som var inkvarterade hos familjen Wadström på banken. Detta berättar Johan i sin berättelse om barn- och ungdomsminnen från Silvast på 30-50-talen.

Gunnar Wadström sjukpensionerades 1952 och flyttade till Sjös hus på Böös hemman, huset numera rivet. Gunnar dog 05.06.1953. Han blev jordfäst i Jeppo kyrka men är gravlagd i Lappfjärd. Hans hustru Ester dog tidigare, den 13.08.1948, efter flera år på sjukhus.

Efter folkskolan på Sparvbacken i Silvast blev Johan student 1953 från Kristinestads Samlyceum och ingenjör från Tekniska Läroverket i Helsingfors 1959. Han har arbetat inom industrin i olika uppdrag på Tampella och under en lång tid på Wärtsilä. Johan är intresserad av astronomi,	resor och friluftsliv. Till familjen hör fru och två döttrar samt fyra barnbarn.



%%%
% [house] Nybygge
%
\jhhouse{Nybygge}{4:28}{Silvast}{5}{370}

Styckad av stomlägenhet Bäck 4:16

%%%
% [occupant] Cederström
%
\jhoccupant{Cederström}{\jhname[Jonas]{Cederström, Jonas}}{2012--}
Den 08.11.2012 köpte VVS-montör Jonas Cederström fastigheterna Silva R:nr 4:53 och Nybygge R:nr 4:28. Nybygge utgör numera en förstoring av tomten till bostadsbyggnaden på 4:53. Se mera nr 70.


%%%
% [occupant] Höglund
%
\jhoccupant{Höglund}{\jhname[Carl-Henrik]{Höglund, Carl-Henrik}}{2008--\allowbreak 2012}
Våren 2008 köpte Carl-Henrik Höglund fastigheterna Silva R:nr 4:53 och Nybygge R:nr 4:28, båda av stomlägenhet Bäck, år 1915 med R:nr 4:16. Nybygge utgör numera en förstoring av tomten till bostadsbyggnaden på 4:53, f.d. HAB bank. Tomterna planerades och planterades. Se mera nr 70.


%%%
% [occupant] Gettings
%
\jhoccupant{Gettings}{\jhname[Ulla]{Gettings, Ulla} \& \jhname[David]{Gettings, David}}{2006--\allowbreak 2008}
Den 17.01.2006 köpte Ulla och David Gettings fastigheten Nybygge R:nr 4:28 för att förstora tomten till sin		bostadsbyggnad på Silva 4:53. nr 70. Säljare W. \& E. Almbergs arvingar. Ulla är född i Vasa och David i Wales, England. Den 15 maj 2008 flyttade Ulla och David Gettings till Bergö.


%%%
% [occupant] Almberg
%
\jhoccupant{Almberg}{\jhname[dödsbo]{Almberg, dödsbo}}{1969--\allowbreak 2006}
\jhvspace{}


%%%
% [occupant] Almberg
%
\jhoccupant{Almberg}{\jhname[William]{Almberg, William} \& \jhname[Edit]{Almberg, Edit}}{1953--\allowbreak 1969}
F.d. jordbrukaren på Nybyggar hemman i Lassila by, Anders William Almberg, \textborn 03.04.1893 på Mietala, och hans hustru Edit Katarina Israelsdotter, född Lundkvist, \textborn 29.08.1898 i Pensala, köpte 1953 fastigheten Nybygge 4:28 på Stationsvägen i Silvast. Lagfart erhöll de den 15.07.1953. Det fanns 3 lokaler i nedre våningen och 2 lokaler i övre våningen.

I den mellersta lokalen på nedre våningen bosatte sig William och Edit. Deras 4 vuxna barn, födda på Grötas och uppvuxna på Nybyggar hemman, flyttade med sina familjer in i huset. William dog den 08.03,1969 och Edit den 13.07.1979. Deras dotter Signe Katarina Stoor, född Almberg, \textborn 01.12.1922 på Grötas och uppvuxen på Nybyggar hemman i Lassila by, och hennes man Artur Matias Stor, \textborn 01.11.1918 i Pensala, samt deras dotter Benita Gunhild Margareta, \textborn 16.09.1945 i Pensala, hyrde redan före köpet västra lokalen i nedre våningen, före detta café, där Signe bedrev en dam- och herrfrisersalong.

\jhhousepic{Jakobsson-Almberg.jpg}{Jakobssonska huset, nr 370, på 1960-talet. Vy från stationen.}

Artur var mjölnare och delägare i en kvarn i Pensala och arbetade tidigare som mjölnare på Kiitola. Efter skilsmässa flyttade Signe till Korsholm och senare till Vasa, där Signe gifte sig med Sven West, \textborn 06.11.1930 i Pörtom. West dog den 08.10.1974.  Benita hade tidigare flyttat till Vasa och 1966 gift sig med Tage Viitala från Ömossa.  Skilsmässa 1972. Benitas andra man, Veikko Perälä från Oravais, dog hastigt som ung. År 1985 öppnade Benita och Veikko en matservering i café Funkis i Jeppo. Som pensionär flyttade Signe West och Benita Viitala till Kristinestad, där de dog och är begravna. Benita dog några år före sin mor och Signe dog den 16.10.2011.

Elvi Maria Lindgren, född Almberg, \textborn 13.03.1921 på Grötas, och hennes man Ragnar Alexandr Lindgren, \textborn 21.05.1924 i Tenala, bosatte sig i den västra övre lokalen. Ragnar startade med sin svåger Gunnar en minkfarm på östra sidan av järnvägen. Ragnar hade hand om och skötte farmen. De flyttade ner när familjen Stoor lämnade lokalen och Elvi övertog Signes damfrisering. År 1973--74 byggde de ett egnahemshus på stadens planerade områden på Grötas. Ragnar dog hastigt den 27.04.1986 och Elvi likaså hastigt den 01.06.1988

Gunnar Almberg, \textborn 04.04.1924 på Grötas och uppvuxen på Nybyggar hemman i Lassila by, vigdes 11.08.1952 med Marianne Westerlund, \textborn 17.02.1930 i Markby, Nykarleby landskomun. Efteråt studerade Gunnar två år i Handelsskolan i Gamlakarleby. Marianne bodde hemma i Markby under hans studietid. 1954 fick Gunnar tjänst på Varma i Oravais. 1955 flyttade de in i den nordöstra lokalen i nedre våningen mot järnvägen, då Gunnar och Marianne fick tjänst på Andelslaget Varma i Jeppo, där de arbetade 1955--\allowbreak 1967. Andelslaget Varma avslutade sin verksamhet 1967. Gunnar fick då tjänst på slippappersfabriken Mirka på Kiitola. Marianne fick arbete som butiksbiträde på Jeppo-Oravais Handelslag. Gunnar avancerade till driftschef och båda gick i pension 1984. År 1973 köpte de lokal nr 6A i Bostads Ab Gökbrinken-Asunto Oy och flyttade dit.

Manne Almberg, \textborn 04.01.1926 på Nybyggar hemman i Lassila by, och hans hustru Kerttu, född Ylinen i Alahärmä, \textborn 21.01.1922, bosatte sig i övrevåningens nordöstra lokal med sonen Fjalar, \textborn 25.12.1950 på Nybyggar hemman. Dottern Mona föddes på Silvast \textborn 03.05.1958. Manne arbetade som långtradarchaufför med foder till pälsdjurfarmen Oy Keppo Ab. Sedan barnen blivit större bedrev Kerttu kosthållet för de anställda på Oy Keppo Ab:s farm i Jeppo.  Fjalar studerade till maskintekniker och fick tjänst i Karleby, dit han flyttade. Fjalar gifte sig 1978 med Seija Puulata \textborn 13.12.1954 i Gamlakarleby. De fick sonen John Christian, \textborn 17.09.1979. Mona är dipl.ekon. och arbetar som banktjänsteman i Jakobstad, sambo med Stefan Granberg. Fjalar dog hastigt i Karleby 1980. År 1985 flyttade även Manne och Kerttu till Jakobstad, där Manne dog 02.11.1999 och Kerttu dog den 06.11.2012.

År 1985 revs uthuset, där en granat hittades, som polisen hämtade. Den visade sig vara tom. En skrivelse från stadens myndigheter gjorde att det ståtliga bostadshuset revs 1986. Tomten såldes 2006 till ägarna av granntomten; Ulla och David Gettings, som med den förstorade gårdstomten till Silva 4:43, nr 70.


%%%
% [occupant] Jakobsson
%
\jhoccupant{Jakobsson}{\jhname[dödsbo]{Jakobsson, dödsbo}}{1925--\allowbreak 1953}

\jhsubsubsection{Hyresgäster}

Den 01.12.1923-15.09 1928 verkade \jhbold{Ab Unionbanken i Finland} i O. Jakobssons gård. Den 15.09.1928 flyttade banken till sitt nybyggda hus på granntomten.

År 1924 hade handlare \jhbold{E. Ström} en kortare tid butik i huset och efteråt hade handlare \jhbold{Kåll} från Jakobstad både butik och café i nedre våningen, som i början av 1930-talet övertogs av \jhbold{K-J. Nylund}. Före kriget fortsatte {Laina Finne} med caféverksamheten, som hon hade till slutet av 1940-talet. Under några år arbetade \jhbold{Lea Andersén}, senare Lassén, som cafébiträde och hade där en liten kemikaliehandel. Med Laina bodde även barnen Lasse född 1926 och Lisa född 1928. Lasse fick arbete som butiksbiträde på Jeppo-Oravais Handelslag, senare blev han butiksföreståndare på ett handelslag i Kuni, Kvevlax.  Lisa gick i Nykarleby Samskola. Hon gifte sig med Holmberg från Lappfjärd och paret emigrerade till Florida, USA.

\jhpic{Stationsv1931}{Stationsvägen 1931, sedd från stationsområdet.}

I nedre våningens nordöstra lokal fanns i slutet av 1930-talet och i början av 1940-talet slaktare \jhbold{Erik Gleisners} köttaffär och bostad. Erik är född 1910 i Socklot och hans hustru Jenny, född Vikström 1920 på Tapelbacken. De vigdes 1940. Deras son Per-Erik föddes 26.05.1942, då de hyrde bostad och affärslokal i Jakobssons hus. Ungefär samtidigt fanns i övre våningens västra lokal \jhbold{Wellamo Halmes} barberarsalong. Wellamo Halme, \textborn 10.06.1912 i Nykarleby, var dotter till Josefina Rintala i hennes första äktenskap. Wellamo, gift Åkerlund, emigrerade till Stockholm 1948. \jhbold{Edvin Ojala} övertog Wellamos herr och damfrisering. Edvin och Vieno Ojala samt barnen Paul och Sally bodde i den övre våningen.

Den 14.02.1941 kom \jhbold{John Lassén} från Esse till Jeppo. Vigd 14.06.1943 med Else Ingegärd Nybacka från Esse. Lassén flyttade tillbaka till Esse 21.04.1945. Då Kemijärvi kyrkby evakuerades till Jeppo 1944--\allowbreak 1945 följde också en restaurang \jhbold{Ravintola Tammi} med, som placerades i huset.

Kransbinderi med konstblommor hade Kemijärvibon \jhbold{Kerttu} i östra lokalen. Efter kriget gifte Kerttu sig med John Frilund från Pensala. Efter kriget bodde en tid fru Nordlund från Norrvik i Närpes, med sina 2 flickor Virpi och Sonja i övre våningen. De flyttade från Koukkuluomas hus. Mannen arbetade på SJ i Helsingfors. Familjen flyttade 1953 till Ingrid Hägers hus på Holmen och därifrån 1954 till Helsingfors.

Ellen Nyman och dottern Britta bodde en kort tid i övre våningen mot station. Bröderna Sirén Ab från Vörå med skräddarna Alf Lillkung och Leo Wiik startade hösten 1947 firman \jhbold{Sirén \& Co}. Företaget hyrde nedre västra lokalen som skrädderi och butik. Tre sömmerskor hade även arbete de första åren.

Ytterligare hyrdes två lokaler som familjebostäder för \jhbold{Lillkung och Wiik}. Familjen Lillkung bestod av; Alf, \textborn 30.01.1915 i Bennäs, Pedersöre,	Alma (Siren), \textborn 09.03.1911 i Kimo, Oravais,	Carey, \textborn 25.09.1936 i Oravais, Rainer,	\textborn 28.08.1942 i Vörå, Nils,	\textborn 15.10.1944 i Vörå	och	Solveig,	\textborn 04.09.1949 i Jeppo.

Fam. Wiik	Leo, \textborn 02.06.1917 i Oravais, Helmi,	\textborn 05.05.1918 i Oravais,	Ritva, \textborn 12.06.1947 i Oravais och Kaija, \textborn 30.01.1949 i Jeppo. År 1950 drog sig Leo Wiik ur företaget, som då flyttades till William och Olga Jungerstams hus. Firman övertog Olga Jungerstams varulager. Endast en sömmerska hade jobbet kvar. Familjen Lillkung flyttade även till Jungerstams hus, se mera nr 63, Jungbo 4:96.

Familjen Vilhelm och Birgit \jhbold{Åkermark} med tre barn flyttade från övre våningen på pälsberederifabriken i f.d. Jungar mejeri till västra lokalen. Efter en kort tid flyttade de till östra lokalen närmast station. Birgit öppnade en kemikalieaffär där. 1952 flyttade de till sitt nybyggda hus på Höglund 4:06, se nr 77.

Karl Runar \jhbold{Källström}, \textborn 17.09.1915 i Munsala, kom till Jeppo 16.03.1949. Runar gifte sig 03.02.1951 med Helga Emilia, \textborn 16.09.1916 i Korsholm. Runar och Helga öppnade skomakeri och skobutik i västra nedre lokalen. De bodde även i huset. År 1954 köpte Runar Källström Lindéns gård, nr 78, R:nr 4:152, och flyttade dit med sin verksamhet.
Uppgifter om alla hyresgäster är inte fullständiga.


%%%
% [occupant] Jakobsson
%
\jhoccupant{Jakobsson}{\jhname[Otto]{Jakobsson, Otto}}{1919--\allowbreak 1925}
Otto Jakobsson Präst, \textborn 31.10.1882 i Pensala, och hans hustru Katarina Lovisa, född Andersdotter Pesonen, köpte 1919 en tomt invid järnvägsstationen i Jeppo. Tomten utgjorde en del av Ida Lindströms skattelägenhet Härbärge 4:24 av stomlägenhet Bäck 4:16. Vid lantmäteriförrättning 1925 fick fastigheten namnet Nybygge R:nr 4:28.

Otto var frimurare och bonde i Pensala. Han reste första gången till Amerika år 1905 och vistades där ca 7 år i olika delstater. Andra gången var 1915, medan hustrun Katarina och barnen var kvar i Pensala. År 1919 såldes deras hemman i Pensala och familjen var redo att emigrera till Amerika, där Otto fanns. Ödet ville annorlunda och kort tid efter auktionen fick Katarina ett brev, att Otto kommer hem på grund av ``minarsjukan''. Otto hade tjänat sig en rejäl slant däröver och de byggde ``miljonärsvillan'' i Jeppo, senare kallad ``Jakobsonska huset''.  Gårdsbyggnaden i två våningar  bestod av 3 lägenheter i nedre våningen och 2 lägenheter i övre våningen. Den 05.05.1923 fick Otto Jakobsson caférättigheter av 3:de klass. Otto och Kaisa-Lovisa (Katarina) Jakobsson köpte även flera jordbruksområden i Silfvast, Jeppo.

Otto och Katarina fick 4 barn; Helmi, Fanny, Teodor och Eugen, som föddes på Präst hemman i Pensala, men som växte upp i Jeppo. Eugen blev dipl.ing. och Helmi folkskollärare, gift med Georg Lygdman från Oravais. Fanny Lovisa, \textborn 12.05.1904, dog ung i Jeppo, \textdied 04.07.1924, och Teodor, \textborn 13.03.1909, stupade i vinterkriget, \textdied 11.02.1940 vid Summa. Otto Jakobsson dog 23.08.1925 i blodstörtning på farstutrappan till deras lägenhet i huset.



%%%
% [house] Bageri
%
\jhhouse{Bageri}{4:23}{Silvast}{5}{72-73}

Styckad av stomlägenhet Bäck 4:16. Lägenheten inbegriper Orrholm 4:141, 4:137/stoml. Orrholm 4:13, och nr 373.

\jhhousepic{053-05585.jpg}{Mykola och Natalia Dudendo}

%%%
% [occupant] Dudendo
%
\jhoccupant{Dudendo}{\jhname[Mykola]{Dudendo, Mykola} \& \jhname[Natalia]{Dudendo, Natalia}}{2015--}
Sommaren 2015 köpte Dudendo Mykola, \textborn 09.08.1981 i Ukraina och hans husttru Natalia, \textborn 02.10.1984 i Ukraina ``Sundells Bageri'', eller de tre ovan nämnda fastigheterna. De kunde genast flytta in med sin son Timofey, \textborn 06.05.2011 på BB i Jakobstad. Mykola och Natalia kom 2009 till Jeppo, Nykarleby Bostäder på Älvvägen (f.d. Älvliden). Mykola är utbildad byggare och arbetar på Karlsborgs svinfarm i Lassila och där arbetade också Natalia i 2 år och därefter studerade hon 2 år på Credo i Nykarleby, i dag arbetar hon på Nykarleby Fastighetscenter Ab.


%%%
% [occupant] Nyvall \& Furu
%
\jhoccupant{Nyvall \& Furu}{\jhname[Andreas]{Nyvall \& Furu, Andreas} \& \jhname[Jessica]{Nyvall \& Furu, Jessica}}{2002--\allowbreak 2015}
Orrholm  4:141 och 4:137 utgör en förstoring av tomten Bageri 4:23 som tillhör f.d. Sundells Bageri och ingick i fastighetsköpet när Nyvall Andreas och Furu Jessica köpte lägenheterna och affärsverksamheten i februari 2002 av Erland Sundells dödsbo.

Andreas Nyvall, \textborn 01.02.1975 i Vexala by, är utbildad kock. Jessica Furu, \textborn 28.03.1982 i Nykarleby, är utbildad sömmerska. De flyttade in i bostaden med sin dotter Jasmine, \textborn 10.09.2000 i Nykarleby, och tog sig an bagerirörelsen med friskt mod. De grundrenoverade bostaden och gjorde flera förbättringar och ändringar i bageriet och de övriga utrymmena.

\jhhousepic{055-05588.jpg}{``Sundells Bageri''}

År 2008 utsågs Andreas Nyvall och Jessica Furu till årets företagare i Nykarleby. Jessica drabbades år 2010 av mjölallergi och Andreas beslöt att bli delägare i ett nytt bageri i Sandsund tillsammans med sin kusin Luukkonen, som hade fortsatt sin fars och farfars bageriverksamhet. I februari 2012 upphörde Sundells 110-åriga bageriverksamhet i Jeppo. Nyvall/Luukkonen övertog en del av Sundells recept, men det nya bageriet fick namnet Luukkonen.
\begin{jhchildren}
  \item \jhperson{\jhname[Jasmine Nyvall]{Nyvall, Jasmine Nyvall}}{10.09.2000 i Nykarleby}{}
  \item \jhperson{\jhname[Casper Nyvall]{Nyvall, Casper Nyvall}}{01.11.2004 i Jeppo}{}
  \item \jhperson{\jhname[Oliver Nyvall]{Nyvall, Oliver Nyvall}}{21.03.2008 i Jeppo}{}
\end{jhchildren}
Jessica arbetar numera på Ostro-Center i Nykarleby och Andreas kör på morgonnatten till Sandsund. Fastigheterna såldes sommaren 2015 och familjen flyttade till Nykarleby.


%%%
% [occupant] Sundell
%
\jhoccupant{Sundell}{\jhname[dödsbo]{Sundell, dödsbo}}{1998--\allowbreak 2002}
\jhvspace{}

%%%
% [occupant] Sundell
%
\jhoccupant{Sundell}{\jhname[Erland]{Sundell, Erland}}{1975--\allowbreak 1998}
På grund av Erland Sundells sjukpensionering 1988 var han tvungen att upphöra med bagerirörelsen och stänga butiken.

\jhsubsubsection{Hyresgäster}

\jhbold{1990--\allowbreak 1998}

Bagare Glenn Magnus Lillas, född 20.07.1971 på Lillas hemman, med hjälp av föräldrarna John Kristian, född 27.06.1942 på Lillas,	och Greta född Eriksson 23.10.1945 i Vörå, drev de Sundells Bageri med framgång under 8 år. Magnus drabbades av mjölallergi och var tvungen att upphöra med bageriverksamheten 1998. Efter Erlands död bjöds fastigheten ut till salu och stod tom till 2002.

\jhbold{1989--\allowbreak 1990}

Under ca 1 år 1989--\allowbreak 1990 bedrev Henrik Överfors bagerirörelsen. Dagmar Norrgård handhade kontorsarbetet och varudistributionen.

\jhbold{1975--\allowbreak 1998}

År 1975 tog Erland Sundell över verksamheten i Sundells Bageri efter sin far Einar, som blev pensionär. Redan som 13-åring började Erland hjälpa till i bageriet och under krigsåren när pappa Einar var inkallad hade Erland ansvar för bageriet. Åren 1950-51 gick Erland i konditorfackskola i Karlstad, Sverige. Erlands arbetsdag började vid 3-tiden på morgonnatten. Det var snarare regel än undantag att han med ljudlig röst stämde i med sång under arbetet, precis som Einar före honom. Som yngre deltog Erland i kyrkokören och manskören samt i folkdanslaget och i Lions. Som yngling hörde han i många år till Sven Jungars berömda ``pyramidgrupp''.  Erland tillhörde en av förespråkarna när Silvast vattenandelslag bildades och var också under några år bolagets ordförande. Orken räckte inte till så han var tvungen att avstå från flera av sina fritidsintressen.

År 1959 köpte Erland och hans fru Maire, Ida Lindströms fastighet, som låg granne med bageriet. De byggde ett bostadshus med lokaler även för posten och posthanteringen, se Björkhagen R:nr 4:60,nr 74.  På grund av skilsmässa hyrde Erland en lokal i Bostads Ab Jeppo Lövgränd på Åkervägen. Efter sin mor Helfrids död 1993 bodde Erland sina sista år i hemgården invid bageriet.

Efter Erlands sjukpensionering 1988 stängdes butiken för gott, medan bagerirörelsen hade ett års uppehåll 1988--\allowbreak 1989. Erland hade då varit bagare i över 40 år och rörelsen då drivits i tre generationer i över 86 år. Från 1989 uthyrdes bageriet jämte de gamla recepten och var igång med korta eller längre uppehåll tills Erlands dödsbo sålde fastigheten 2002 till Andreas Nyvall.

\jhsubsubsection{Anställda}

Då Erland 1975 övertog verksamheten efter sin far Einar, fortsatte systrarna Dagmar och Ragni sitt arbete i företaget, likaså Anneli Laxen, Alice Sandvik, Eivor Sandvik och Anita Westin. Ungdomarna Gunvor Romar och Karin Liljeqvist var anställda under några år. Under sommarloven arbetade Björn Stenbacka och likaså Erlands och Maires barn Christer, Britt-Marie och Fredrik i olika arbetsuppgifter. Anneli och Alice arbetade med Erland fram till pensioneringen.

Erland Sundell dog den 17.03.1998 efter en längre och svår sjukdomstid. Om Erland Sundells familj berättas mera under nr 74, Björkhagen R:nr 4:160.


%%%
% [occupant] Sundell
%
\jhoccupant{Sundell}{\jhname[Einar]{Sundell, Einar} \& \jhname[Helfrid]{Sundell, Helfrid}}{1936--\allowbreak 1975}
År 1936 övertog Einar Sundell, \textborn 19.08.1905 på Silvast i Jeppo och hans hustru Helfrid, född Johannesdotter Jungar, \textborn 14.02.1905 på Jungar hemman i Jeppo, fastigheten Bageri R:nr 4:23 jämte verksamheten Sundells Bageri. Överlåtare var Otto Sundells dödsbo.

Före övertagandet av bageriet reparerade och sålde Einar radioapparater och Monarkcyklar i huset de byggt på den inköpta tomten, ett visst antal mantal av stomlägenhet Nygård 4:9, numera Sparbanken R:nr 4:194, nr 81. År 1934 erbjöds han arbete som radiotekniker i Helsingfors, men övertalades av bröderna att övertaga bageriet.

Sundells Bageri har präglats av ett familjeföretag. Alla Einars barn har under olika perioder hjälpt till med varierande göromål och/eller varit anställda i firman.
\begin{jhchildren}
  \item \jhperson{\jhname[Ragni Linnea]{Sundell, Ragni Linnea}}{12.09.1926}{}
  \item \jhperson{Charles \jhbold{Erland}}{25.11.1928}{}
  \item \jhperson{\jhname[Gunfrid Katarina]{Sundell, Gunfrid Katarina}}{20.08.1930}{}
  \item \jhperson{\jhname[Dagmar Ingegärd]{Sundell, Dagmar Ingegärd}}{02.05.1935}{}
  \item \jhperson{\jhname[Gerda Gun-Lis]{Sundell, Gerda Gun-Lis}}{14.03.1944}{}
\end{jhchildren}
Endast Gun-Lis är född på Bageri 4:23, de äldre barnen föddes på fastigheten, som numera heter Sparbanken 4:194, förutom Ragni, som föddes på Jungar hemman. Äldsta dottern Ragni arbetade i företaget från tidig ungdom med allehanda göromål; degblandare, gräddare, brödpackare och affärsbiträde.  År 1946 gifte sig Ragni med grannsonen Valter Willman, se mera om familjen under nr 372, Orrholm R:nr 4:141.

Den 27.11.1964 flyttade familjen till Jakobstad, men Ragni reste dagligen till Jeppo och arbetade i bageriet till 1988, då brodern Erland sjukpensionerades. Valter dog 2013 och Ragni vistades sina sista år på Östanlid sjukhem. Ragni dog 04.02.2015. De är begravna i Jeppo.

\jhbold{Erland} började redan som 13-åring i bagariet. Han blev ägare till företaget 1975, sjukpensionerades 1988 och dog 1998. Läs mera om Erlands familj under nr 74, Björkhagen R:nr 4:160.

\jhbold{Gunfrid} arbetade redan som skolflicka som servitris på deras cafè. Hon hade ansvar för butiken 1946--\allowbreak 1958, gifte sig 1958 med jordbrukare Sven Löfqvist och flyttade till Strandbyn i Oravais. Gunfrid gick ett år i Korsholms husmodersskola, före giftermålet. Sven dog hastigt 28.03.2016.

\jhbold{Dagmar} gick efter Samskolan i Nykarleby i den Merkantila Aftonskolan i Jakobstad 1953--54. Efter samskolan tog Dagmar hand om bokföringen i företaget. Efter Gunfrid f.o.m. år 1958 hade hon även ansvar för butiken och fortsatte i familjeföretaget tills brodern Erland 1988 blev sjukpensionär, då affärsverksamheten upphörde. Därefter skötte hon bokföringen samt var brödpackare och bagerichaufför åt Henrik Överfors. År 1990 fick Dagmar arbete som bokförare på fiskboden i Jakobstad. Dagmar gifte sig med Olof Norrgård och flyttade till Mietala i Jeppo. De är i dag båda borta, ``Ole'' dog den 30.04.2014 och Dagmar 31.01.2016.

\jhbold{Gun-Lis} uppgift var att efter skoldagen med kärra föra in flera lass ved till den vedeldade ugnen. På loven under studietiden i Ekenäs Seminarium arbetade Gun-Lis som affärsbiträde, chaufför och i hushållet. Gun-Lis blev lärare 1966 och flyttade 1967 till Sverige, gifte sig  med byggmästare Torsten Knuts från Kimito. År 1984 blev Gun-Lis speciallärare från Lärarhögskolan i Stockholm. Som pensionärer återvände de till Nykarleby 2009. Gun-Lis har gett och skrivit uppgifter om familjen.

Sundells Bageri blev något av en institution. Företaget utvecklades och utbyggdes kraftigt under Einars tid. Kundkretsen sträckte sig mellan Maxmo och Vörå i söder till Jakobstad i norr. Före kriget förekom en livlig torghandel i ``Sundells bullabod'' på Nykarleby torg. Då tågen stannade vid Jeppo station brukade beväringar, på väg norr- eller  söderut, kliva av tåget för att köpa Sundells grisar. I väntan på mötande tåg fanns det gott om tid att handla. Inga andra kunder hade då rum i butiken. Ett besök i hemtrakten av utflugna Jeppobor och andra besökare beställde grisar och olika sorters bröd vid hemfärden.

Största ruljansen på brödfronten hade Sundells Bageri under krigsåren. Då fungerade bageriet som militärbageri. Den elektriska degblandaren, som inköptes 1928, och var i användning ännu 2012, gick nu utan uppehåll, då 2500 bröd per dag levererades till trupper, förlagda i närheten, främst i Jeppo och Nykarleby. Det bakades även till evakuerade från Kemijärvi, som åren 1944-45 uppsamlades i Jungar skola, innan de förflyttades till olika hem i trakten. Största delen av krigsåren skötte Einars hustru Helfrid och den unga sonen Erland bageriet med hjälp av fjorton anställda, varav en del var utkommenderade soldater, däribland två bagare från Helsingfors. Helfrid skulle även förse de anställda med mat, vilket betydde 12-14 extra matgäster varje dag.

Einar var under fortsättningskriget under flera år utkommenderad till bl.a. fronten vid Svir för att förse soldaterna med bröd. Innan fredslutet blev Einar hemskickad för att baka bröd till militären. Vid den här tiden var han och Evert Jungar åtminstone två gånger med hjälpsändningar till soldaterna vid Svir. De hade med sig bl.a. bröd, smör, surmjölk och andra matvaror. Efter kriget avtackades de med varsin tavla av ``pojkarna vid Svir''. Einar avtackades även med en tavla, troligen av ``Jeppo-männen vid fronten''. Förutom att förse soldaterna med bröd hjälpte Einar till som radiotekniker. Han lyssnade bl.a. på engelska radionyheter. Skulle det inte blivit fred hade han ställts inför krigsrätt för detta.

Handlandet med radioapparater och cyklar låg nere under krigen. År 1944 öppnade Einar på nytt en radio- och elaffär i samma hus som caféet. Han var då på sidan om bageriverksamheten även distriktsrepresentant för flera olika radiomärken, som såldes i stora delar av svenska Österbotten. Efter några år blev arbetsbördan för tung och han blev tvungen att göra ett val och bestämde sig att satsa på bageriet. Under några år hyrde brodern Torsten caféet och verksamheten där.

År 1962 byggdes ett nytt bageri- och livsmedelsaffär. Den gamla byggnaden med bageri och butik samt cafébyggnaden revs (nr 372). Tegelkällaren blev kvar i den nya byggnaden.  År 1964 inköptes grannfastigheten, Orrholm R:nr 4:141. Byggnaderna på tomten revs och gårdsplanen för verksamheten förstorades. I och med att den gamla vedeldade ugnen revs i samband med nybygget, försvann också traditionen att en del Jeppobor kom med sina julskinkor kvällen mot julafton. Skinkorna bakades in i deg gjord på rågmjöl för att sakta gräddas och avhämtas tidigt på julaftons morgon. En nyhet för trakten var hemkörning av matvaror. Från 1960-talet tills affären upphörde 1988 var detta en betydande del av affärsverksamheten. Det var en självklarhet att alltid sända bröd som blivit över för dagen till Barnhemmet i Nykarleby.

Recepten på grisar, limpor, franskbröd och rågbröd härstammar från början av 1900-talet, då Otto Sundell startade bageriet. Fortfarande används de gamla recepten på grisar, limpor och blandbröd sedan 2012 av bagarna Nyvall och Luukkonen i Sandsund.

Einar och Erland använde inte några konstgjorda färg- eller smakämnen. De innehade ett speciellt ``smörpass'', som tillät dem att baka med smör och köpa ett visst antal kg till nedsatt pris.

Anställda under Einars tid; Runar Nordman, Ragnar Sandqvist, Saga Sandberg, Frank Nyman, Stefan Gustafsson, Svea Mylläri, Viking Orrholm, Rainer och Nils Lillkung, Marina Saviaro, Ruth Kula, Inga Ljung, Regina Dahlström, Eivor Sandvik, Anita Westin samt Anneli Laxen och Alice Sandvik.  Ungdomarna stannade vanligen endast några år.

Einar hade en stark gudstro. Han levde efter devisen: Se alltid framåt med mod och förtröstan. Se åt sidorna med kärlek. Se alltid bakåt med tacksamhet och alltid uppåt med tro och hopp! Det var viktigt att dela med sig av denna övertygelse. Det gjorde han bl.a. genom att långa tider beställa böcker från Bibeltrogna Vänner i Sverige. Böckerna delades ut gratis i Jeppo och i närkommunerna. Einar deltog även i offentliga uppdrag, bl.a. satt han flera perioder i kyrkofullmäktige och i olika kommunala nämnder och som principal i Jeppo Sparbank och senare i HAB.

Einar avled 15.04.1980 efter en längre tids sjukdom. Hans hustru Helfrid avled 10.09.1993.


\jhhousepic[pic:Ida]{IdaL.png}{Stationsvägen 1920. Sundells bageri och Ida Lindströms hus t.v. Till höger en ria på blivande HAB-tomt}

%%%
% [occupant] Sundell
%
\jhoccupant{Sundell}{\jhname[Otto]{Sundell, Otto}}{1902--\allowbreak 1933}
Den 14.03.1902  flyttade Evelina och Otto Sundell från Lepplax by i Pedersöre till Silvast i Jeppo. Otto Leander Sundell \textborn 07.10.1881 i Pedersöre, gifte sig den 07.07.1901 med Anna Evelina Karl-Johansdotter Skutnabba, \textborn 06.09.1874. Otto som började sin karriär inom bageribranschen hos Pålssons i Jakobstad, köpte nu lanthandlare Gustav Julins affärsfastighet nära järnvägsstationen i Jeppo. Vid en lantmäteriförrättning 1921 infördes i Jordregistret fastigheten Bageri R:nr 4:23 av stomlägenhet Bäck 4:16 av Silvast skattehemman. På tomten fanns ett bageri med butik och bostad samt ett litet café, nr \jhbold{372}. Under de närmaste åren byggde de ytterligare ett bostadshus (nr 73). Otto och Evelina fick 9 barn, av vilka 3 blev bagare och 3 tillverkade glass och läskedrycker.
\begin{jhchildren}
  \item \jhperson{\jhname[Otto Evert]{Sundell, Otto Evert}}{12.05.1901 i Pedersöre}{16.5.1955 i Uleåborg}
  \item \jhperson{\jhname[Uno Leander]{Sundell, Uno Leander}}{13.11.1902 i Jeppo}{1954 i Myllykoski}
  \item \jhperson{\jhname[Oscar Johannes]{Sundell, Oscar Johannes}}{04.04.1904 i Jeppo}{07.04.1904}
  \item \jhperson{\jhbold{\jhname[Einar]{Sundell, Einar}} Johannes}{19.08.1905 i Jeppo}{15.04.1980 i Jeppo}
  \item \jhperson{\jhname[Oscar Evald]{Sundell, Oscar Evald}}{08.03.1907 i Jeppo}{}, gr Jstads Glassfabrik, Sun-Ice
  \item \jhperson{\jhname[Ragnar Vilhelm]{Sundell, Ragnar Vilhelm}}{01.08.1909 i Jeppo}{12.06.1972 i Jakobstad}
  \item \jhperson{\jhname[Torsten Valdemar]{Sundell, Torsten Valdemar}}{18.06.1913 i Jeppo}{13.05.1974 i Jeppo}, jfr nr 90
  \item \jhperson{\jhname[Bror Fredrik]{Sundell, Bror Fredrik}}{20.04.1915 i Jeppo}{13.01.1961 i Kristinestad}
  \item \jhperson{\jhname[Inga Oinila]{Sundell, Inga Oinila}}{14.01.1918 i Jeppo}{07.05.1945 i Jeppo}, g. Lassus
\end{jhchildren}

Caféanställda under Ottos tid var Katarina Gustavsson och Ester Källman.

\jhbold{1933--\allowbreak 1936}
Otto Sundell dog ung, redan den 02.11.1933 och hans hustru Evelina levde till 01.01.1959. Verksamheten handhades av dödsboet fram till 1936, då Einar och	Helfrid övertog bageri-, butiks- och caféverksamheten. De sålde sitt hus, nr \jhbold{381}, Sparbanken R:nr 4:194, och familjen flyttade till Sundells Bageri på Stationsvägen.


%%%
% [occupant] Julin
%
\jhoccupant{Julin}{\jhname[Gustav]{Julin, Gustav}}{1895--\allowbreak 1902}
I medlet på 1890-talet anskaffade affärsmannen Gustav Julin, \textborn 20.11.1838 på Lilljungar R:nr 7, i Jungar by, en tomt nära järnvägsstationen. Säljare var Isak Lindfors, som ägde stomlägenheten Bäck R:nr 4:16. Gustav var son till sexman, bonden Simon Andersson Jungar. Gustav gifte sig 1861 med Lovisa Eriksdotter Draka, \textborn 19.09.1842 i Ytterjeppo. De fick tolv barn, varav åtta är födda i Jeppo, ett i Helsingfors, två i Kortesjärvi och ett i Vörå.

Gustav och brodern Simon tog tillnamnet Julin. År 1865 flyttade Gustav som byggnadsarbetare med familjen till Helsingfors, men återvände 1869 och öppnade 1870 en lanthandel på Mietala. Gustav var den som först fick affärstillstånd i Jeppo den 05.09.1871. Gustav flyttade till Silvast, där han byggde en gård med butik.

Familjen flyttade ofta; år 1877 till Kortesjärvi, 1881 till Vasa och vidare till Vörå och Korsnäs samt åter till Vasa 1889. Han ägnade den mesta tiden och intresset åt att vara kringresande predikant	och kolportör. År 1890 återvände han och förstorade tomten, idag 4:57,	med ett arrendekontrakt på 50 år, som han 1891 sålde vidare till ett nybildat handelsbolag med Julin som föreståndare. År 1894 såldes affären till Isak Liljeqvist. Julin köpte då tomten nära järnvägsstationen och byggde ett bageri med butik och ett större bostadsrum, samt i en skild byggnad ett café med kök och förråd.

Gustav Julin dog i Vasa 09.02.1902 och hans hustru Lovisa 15.04.1917 i Munsala.



%%%
% [house] Björkhagen
%
\jhhouse{Björkhagen}{4:160}{Silvast}{5}{74, 374}

Styckad av stomlägenhet Bäck 4:16

\jhhousepic{057-05590.jpg}{F.d. postkontor. I bostaden bor Jorma Kalijärvi.}

%%%
% [occupant] Kalijärvi
%
\jhoccupant{Kalijärvi}{\jhname[Jorma]{Kalijärvi, Jorma}}{2006--}
Långtradarchaufför Jorma Tapio Kalijärvi, \textborn 05.01. 1968 på Grötas hemman, köpte 06.09. 2006 fastigheten Björkhagen 4:160, lagfart 20.09.2006. Han flyttade med sin familj, hustrun Polina, född Galtseva, \textborn 13.04.1975 och deras son Joonas Mathias, \textborn 26.07.1996, samt bonussonen Ernest, född Galtseva, \textborn 02.07.1992, idag med tillnamnet Kalijärvi. Dessutom flyttade även Polinas mor, Valentina Galtseva  född 11.12.1938, med familjen till Jeppo. Polina, Ernest och Valentina är födda i Nalchik, Gabardina baltchik i Sovjet. Valentina dog 06.12.2015 och jordfästes i Ortodoxa kyrkan i Vasa och blev kremerad.

Jorma växte upp på fastighet Björken på Romar hemman, se nr 29. Jorma arbetar som långtradarchaufför i transportfirman NTC, Nykarleby. Polina har en egen städfirma. Joonas blev student våren 2015 från Alahärmän Yliopisto. Ernest har studerat i Vaasan Amattikoulu på billinjen.

I sitt första äktenskap har Jorma med Birgitta, nu gift Kronqvist, två barn; Tony Mathias Kalijärvi,	\textborn 05.04.1990 och Marie Helen Kalijärvi, \textborn 28.04.1992, se Älvbrant 4:110, nr 47. Jorma och Polina är i dag frånskilda och Polina bor i Karleby.


%%%
% [occupant] Sundell
%
\jhoccupant{Sundell}{\jhname[Erland	\& Maire]{Sundell, Erland	\& Maire}}{1959--\allowbreak 2005}
Bagare Charles Erland Sundell, \textborn 25.11.1928 på Nygård lägenhet av Silvast hemman, och hans hustru 		postföreståndare Maire Vellamo, född Malinen, \textborn 30.06.1927 i Jaakkima i Karelen, köpte 25.04 1959 den rösade tomten om 0,0025 mantal, som 05.09.1929 styckats av lägenhet Härbärge 4:44. Fastigheten benämndes Björkhagen 4:160. Bostadshuset (nr 374) och uthuset, som Ida Lindström byggde 1916--\allowbreak 1920, revs.

Erland och Maire byggde ett nytt bostadshus med garage, förråd och bastu samt en större lokal för uthyrning till Jeppo 	Post- och telegrafexpedition. Posten fanns då i huset på andra sidan av Stationsvägen, nr 62.	Maire och Erland var båda aktiva i olika föreningar och	verksamheter, bland annat SFP, Röda Korset, Lions och	Kyrkokören.
\begin{jhchildren}
  \item \jhperson{\jhname[Christer Johannes]{Sundell, Christer Johannes}}{27.03.1954}{}, företagare i Duluth, Alabama
  \item \jhperson{\jhname[Britt-Marie Harriet]{Sundell, Britt-Marie Harriet}}{28.06.1956}{}, barntr.lärare, Esbo, g. Mickelson
  \item \jhperson{\jhname[Fredrik Staffan]{Sundell, Fredrik Staffan}}{24.09.1964}{}, rektor i Gerby, JUO-dirigent/ledare
\end{jhchildren}
Fredrik är gift med Lotta född 1965 i Åbo. De har två barn födda i Vasa Ida och Anton. Fredrik har varit med över 40 år i Jeppo Ungdomsorkester från starten, först som elev, senare som vicedirigent och från 1985 som dirigent och dragare. Varje lördag under nästan hela året samlas orkestern kl. 10 för övning på Jeppo-Pensala	skola. Familjen bor i Gerby och deltar i flera musikgrupper.

År 1975 övertog Erland firman Sundells Bageri, där han arbetat sen han var i tretton års ålder. Se mera Bageri 4:23, nr 72. Maire fick 1953, samma år som de gifte sig, tjänst som föreståndare på Jeppo Post- och Telegrafexpedition. År 1978 blev Maire postkontorschef i Kannus, dit hon reste varje vardag med tåg. Skilsmässa 1986. Maire övertog fastigheten, och Erland hyrde en lokal i Bostads Ab Lövgränd, vid Åkervägen på Grötas hemman. Jeppo Postexpedition flyttade i december 1981 till Andelsringens fastighet med en yta på 102 m2. År 1993, när Erlands mor Helfrid dog, flyttade han till barndomshemmet invid Bageriet. Erland dog 17.03.1998. Maire sålde fastigheten 2006 och flyttade till Vasa.


\jhsubsubsection{Hyresgäster}

Under tiden som Jeppo Post- och telegrafexpedition, 1960--\allowbreak 1981, hyrde de specialbyggda lokalerna i Erlands och Maires hus arbetade följande personer där.

Föreståndare:

\begin{tabular}{ll}
  Sundell Maire & 1960--\allowbreak 1978 \\
  Lawast Anita  & 1978--\allowbreak 1981 \\
\end{tabular}

Posttjänstemän:

\begin{tabular}{ll}
  Lawast Anita      & 1960--\allowbreak 1978 \\
  Snickars Marith   & 1960--\allowbreak 1962 \\
  Hinkkanen Siv     & 1961--\allowbreak 1981 \\
  Lindborg Gun-Viol & 1968--\allowbreak 1970 \\
\end{tabular}

Postfröknar eller kontorister:

\begin{tabular}{l}
  Stenbacka Maija
  Hägglund Marianne
\end{tabular}

Postutdelare: 1960 fick Silvast området posten hemburen  till dörren.

\begin{tabular}{ll}
  Sandqvist Ellen     & 1960--\allowbreak 1961 \\
  Strengell Bengt     & 1960--\allowbreak 1962 \\
  Kalliosaari Mirja	  & 1961 \\
  Liljeqvist Holger	  & 1961--\allowbreak 1964 \\
  Kalijärvi Karin	    & 1962--\allowbreak 1981 \\
  Strand Eira	        & 1964--\allowbreak 1981 \\
  Nyberg Leena        & 1974--\allowbreak 1981 \\
\end{tabular}

Totala anställningstiden för ovanstående finns angivna under Ströms hus 4:90, nr 62. Under kortare perioder fanns det även andra postfröknar och 15-16 års ungdomar som veckosluts- och sommarpostutdelare. De som var i tjänst 1981 fortsatte sitt arbete på den nya platsen i KPO:s hus, se nr 57.



%%%
% [occupant] Lindström
%
\jhoccupant{Lindström}{\jhname[Ida]{Lindström, Ida}}{1916--\allowbreak 1959}
När Ida Maria Lindströms man Isak dog 1916, sålde Ida och hennes mor Maria stomlägenhet Lindström 4:7 till Marias brorson Anders Kaukus, se Källström, nr 78, reg.4:152. Ida köpte ett jordområde om 0,0072 mantal av stomlägenhet Bäck 4:16, som 1921 infördes i jordregistret som Härbärge 4:24. Vid Stationsvägen mot järnvägsstationen byggde hon ett nytt bostadshus (nr 374, se bild \ref{pic:Ida}) med vindsvåning, samt ett uthus. Av Idas och Isaks 4 barn levde Anni Maria, \textborn 12.01.1910 och Isak Evert, \textborn	24.02.1913.

Anni gifte sig med Torsten Forss på Fors hemman, se nr 108. Evert arbetade som stationskarl på SJ i Jeppo och bodde efter 1930 med sin mor i nedre våningen. Övre våningen uthyrdes till unga flickor, som arbetade bl.a. påposten, mejeriet och som kontrollassistenter. Nedan följer en förteckning över hyresgästerna.

Som 46-åring gifte sig Evert med mejerskan Iris Runhild Lingonblad \textborn 26.04.1923 i Kimo, Oravais. De byggde en egen gård vid Centralvägen, se Härbärge 4:161, nr 69. När fastigheten såldes 1959 flyttade även mor Ida till det nya huset. Ida Maria dog 09.05.1963.

\jhbold{Hyresgäster:}
Den 14.10.1919 öppnade Lantmannabanken ett kontor i Ida Lindströms gård. Redan 31.12.1919 flyttade banken till Jeppo Handelslags kontor.

År 1920 bildade Nykarleby, Jeppo och Munsala handelslag en gemensam inköpsaffär för hö, halm, säd, potatis och lingon. Affären hyrde Ida Lindströms gård och fick firmanamnet Mellersta Österbottens H:lag m.b.t. Reinholt Husberg blev föreståndare och Lennart Jungar biträde. År 1926 upphörde affären. Lennart Jungar och J.A. Jungarå fortsatte 1926--\allowbreak 1930.

Hyresgäster i övre våningen, där rummen i östra och västra ändan var försedda med köksspis, var under 1930--\allowbreak 1950 talen:
Toini Puro, kontrollassistenterna Anna Elisabeth Majors, \textborn 07.08.1932 i Solf, 18.11.1953-29.12.1956, Barbro Finne från Kronoby, Gerda Valdine Hagberg, \textborn 18.07.1922 i Kaitsor, 1957--\allowbreak 1960, Siv Haga från Kronoby, postfröken Birgit Nyman 1954--\allowbreak 1956, butiksbiträde Ingmar Saarela och evakuerade Karelare, m.fl.



%%%
% [house] Mejeriet, inkl. boende
%
\jhhouse{Mejeriet, inkl. boende}{4:184}{Silvast}{5}{58}

Styckad av stomlägenhet Lindström 4:7

\jhhousepic{029-05893.jpg}{Mjölkkannorna skramlar inte längre i arla morgonstund}

%%%
% [occupant] Leppävuori
%
\jhoccupant{Leppävuori}{\jhname[Aki]{Leppävuori, Aki}}{2014--}
Aki Leppävuori, \textborn 04.08.1980 i Karleby, köpte fastigheten år 2014. Han arbetar på Mirka.\jhvspace{}


%%%
% [occupant] Botnica
%
\jhoccupant{Botnica}{\jhname[m.fl.]{Botnica, m.fl.}}{1989--\allowbreak 2014}
Handelslaget Botnica, som grundats 1978 och till vilket Andelsringen fusionerats, hade i sin tur fusionerats med Waasko 1981. Innan Waaskos verksamhet avslutades 1990 hade handelslaget köpt mejerifastigheten och börjat använda den som lagerbyggnad för olika förnödenheter. Efter en närmast obeskrivlig röra av företagsbyten mellan Waasko – Hankkija – KPO - Agri-Market - Hankkija, såldes såväl handelslagets som mejerifastigheten år 2013 till Danish Agro Holding A/S som år 2014 i sin tur lade fastigheterna till försäljning på den öppna marknaden, varvid Aki Leppävuori köpte mejeriet.

\jhbold{Basfunktionen}

Den 24 oktober 1931 besluter styrelsen för Jeppo Andelsmejeri att bygga ett nytt mejeri. Innan dess hade nästan två årtionden med spänningar mellan Jungar Andelsmejeri och Jeppo Andelsmejeri fortgått. Pensalabönder hade sedan 1922 levererat till Jeppo Andelsmejeri, men i slutet av år 1927 utträtt på nytt när rykten om nybygge kommit i dagen. Och det hade sin giltighet därför att i oktober 1929 hade beslut fattats om att söka efter en ny tomt. Det utkristalliserades 7 alternativ, alla i eller nära Silvast centrum, men till slut kunde man komma till ett avgörande där ett område ägt av Anders Lindén inköptes och mejeriet uppfördes i huvudsak under 1932. Till grund för hur mejeriet skulle utformas hade styrelsen rest runt i regionen och fastnat för att utgå från Purmo Andelsmejeri, som nyss blivit klart.

För att spara kostnader fastställdes antalet dagsverken varje medlem skulle bidra med utgående från antalet kor. I februari 1933 stod mejeriet klart att tas ibruk. Enligt tillgängliga uppgifter blev kostnaderna för mejeriet 669000 mk. Tomten utgjorde 16000mk, tegel 49000mk, kakeltaket 15000mk, parkettgolvet i mejerisalen 9000mk, skorsten 10000mk, maskiner 240000mk. Gratisdagsverken av medlemmarna värderades till 155000mk och baserade sig på 1 ko = 1 andel = 6,25 dagsverken, 1 stock, 20 st takstickor(?), 1 tunnskorg mossa. De här nämnda kostnaderna utgör 494000mk. Det är oklart vad resten består av. Mejeriet ansågs modernt och attraktivt och bönder från Åvist var intresserade att ansluta sej, men jeppoborna tackade nej.

\jhpic{Jeppo Andelsmejeri 1932.JPG}{Byggnadsskede år 1932}

Mejeriet var ett utpräglat produktmejeri och den dominerande produkten var smör och genom åren fick mejeriet stort erkännande för sin höga kvalitet, mycket tack vare mejerskan Lydia Jungerstams insatser (se Jungar hemman nr 10d). Under de drygt 50 år mejeriet tjänat jeppobönderna och bygden har mycket hänt och kan bäst tillgodogöras i både ``Historik över Jeppo'' och ``Jeppo i kris och krigstid''.

Det nya mejeriets första disponent blev jepposonen \jhname[Selim Romar]{Romar, Selim}, som satt 10 år på sin post fram till 20.03.1943 (se Romar hemman, karta 3, nr 20). Då valdes också jeppobördige \jhname[Robert Liljeqvist]{Liljeqvist, Robert} till ny disponent. Hans tid blev tragiskt kort då han den 30.09.1944 dog efter svåra brännskador orsakade av en exploderande blåslampa. Till ny disponent kallades nu Erik Alfred Kullman, \textborn 20.06.1912 i Oravais. Han hade  den 24.08.1941 gift sej med jeppoflickan Lea Maria Backlund, som var född 30.10.1914. De anlände med kort varsel från Vasa och familjen installerade sej i mejeriets disponentbostad den 13 november 1944. \jhname[Erik Kullman]{Kullman, Erik} verkade som disponent fram till 4 oktober 1951 varefter familjen sökte sej tillbaka till Vasa.
I Jeppo föddes barnen Susanne Ulrika, \textborn 25.12.1945 och Frej Erik, \textborn 11.10.1947

Ny disponent blev nu Matts Gunnar Träisk, \textborn 03.01.1914 i Kronoby, gift 14.02.1943 med Elsa Maria Nyqvist, \textborn 08.09.1921 i Nykarleby. Familjen anlände till Jeppo 5 mars 1952. Äldsta dotter \jhname[Ulla-Britt]{Träisk, Ulla-Britt} hade fötts 13.01.1947, men den yngre dottern \jhname[Christina Marlene]{Träisk, Christina} föddes under tiden i Jeppo, den 09.07.1957.

Under \jhname[Gunnar Träisk]{Träisk, Gunnar} tid i Jeppo skedde omvälvande förändringar både inom mjölkproduktionen och mjölkhanteringen. Större men färre besättningar krävde tekniska förändringar i både producentledet och mejeriet, krav som i sin förlängning innebar att mejeriets fortsatta verksamhet i praktiken upphörde den 31 oktober 1977, då den sista mjölken vägdes in. Gunnar Träisk gick i pension, men hustrun \jhname[Elsa]{Träisk, Elsa} fortsatte en tid som expedit på Andelsringen. De flyttade till Jakobstad 18 september 1979.

\jhpic{Mejeriet1970.JPG}{Mjölkkusken Helge Bergman levererar en last i början på 1970-talet.}

Efter att mjölkinvägningen upphört 1977 fortsatte ännu verksamheten på mejeriet, men nu med leverantörsservice, foderförmedling och övriga produktionstillbehör fram till 29 april 1988, då mejeriets sista mejerska, Iris Lindström f. Lingonblad, som haft hand om verksamheten, gick i pension. Mejerifastigheten utbjöds nu till salu.

Många människor har under årens lopp  bott på mejeriet och alla har vi inte kunnat fånga upp, men här presenteras ändå några, förutom de disponenter vi nämnt:

\begin{longtable}{lp{0.7\textwidth}}
  Kontrollass.  & Vieno Lyyli Salminen, \textborn 1902 i Multiala, kom 12.09.1931 \\
       \ditto        & Gerda Maria Backman, \textborn 1916 i Nykarleby, kom 19.02.1940, flyttade till Nkby 1942 \\
       \ditto        & Gretel Gertud Stenfors, \textborn 1924 i Övermark, kom 1940,flyttade 1948 \\
       \ditto        & Anna Elisabeth Majors, \textborn 1932 i Solf, kom 1953, flyttade 1956 \\
       \ditto        & Barbro Viola Finne, \textborn 1940 i Kronoby, kom 1961, flyttade 1965 \\
       \ditto        & Gerda Hagberg, \textborn 1922 i Vörå, kom 1957, flyttade 1976 \\
   Mejerskan     & Brita Gunhild Sundvik, \textborn 1917 i Munsala, kom 1944, gift 1947 m. Henrik Jungarå \\
       \ditto        & Fanny Nylund g. Sandberg, \textborn 1910 i Jeppo, 1940--\allowbreak 1942 \\
       \ditto        & Iris Lingonblad, \textborn 1923 i Oravais, kom 1953, gift m. Evert Lindström \\
       \ditto        & Verna Strand, \textborn 1925  i Jeppo, flyttade till Vasa 1985 \\
       \ditto        & Tora Biskop, \textborn 1938  i  Kronoby, kom 1962, flyttade 1963 \\
   Skogstekniker & Guy Djupsjöbacka, \textborn 1958 i Terjärv. kom 1980, hade här kontor f. skogsv.fören. Flyttat till Oravais 1982 \\
       \ditto        & Bengt Dahlskog hade sitt kontor här en kort tid \\
   Hårfrisörskan & Anne-Kristine Häggblom hade här sin frisörssalong åren 1982-85 \\
                 & Thomas Lindström \\
   Lantbr.avbyt. & Merja Tepponen (gift Dahlström) \\
       \ditto        & Terese Witting \\
\end{longtable}



%%%
% [house] Källström
%
\jhhouse{Källström}{4:152}{Silvast}{5}{78}

Styckad av stomlägenhet Lindström 4:7

%%%
% [occupant] Fors
%
\jhoccupant{Fors}{\jhname[Fjalar]{Fors, Fjalar} \& \jhname[Mayvor]{Fors, Mayvor}}{2017--}
Den 29 juni 2017 besöker Juhani Soidinmäki fastigheten. Efter inspektion på lägenheten är han villig att avyttra inte bara huset utan också tomten, eftersom den är svår att sköta på det långa avståndet från Seinäjoki och kan komma till bättre gagn då bybor får rå om den. Man skakar hand på snabb försäljning. Paret undertecknar köpebrevet den 25 september 2017 och blir fastighetens nya ägare. Uppstädning på tomten och rivning av huset inleds s.g.s. omedelbart.


\jhhousepic{031-05562.jpg}{I skomakarens hus pliggas inga fler sulor, sys inga fler stövelskaft, formas inga nya lästar.}

%%%
% [occupant] VTK
%
\jhoccupant{VTK}{\jhname[Oy]{VTK, Oy}}{2010--\allowbreak 2017}
VTK Oy har inte löst ut sin lagfart, men står som innehavare. Nykarleby stad har återkommande fordrat förklaringar av Hawkline till att fastigheten fortsättningsvis förfular miljön. Den har i maj 2017 tagit till grövre artilleri och skickat krav på vite á 500 euro. Vitesumman höjs till 2000 euro i oktober om ingenting utförs. I slutet av juni 2017 är ägaren sålunda beredd att låta riva huset och städa upp på tomten, sedan Fjalar Fors tagit kontakt och erbjudit sig att mot skild överenskommelse få jobbet gjort.


%%%
% [occupant] Hawkline
%
\jhoccupant{Hawkline}{\jhname[Oy]{Hawkline, Oy}}{2007--\allowbreak 2010}
Lampakka sålde fastigheten år 2007 till \jhbold{Hawkline Oy}, hemort Helsingfors, men med besöksadress i Seinäjoki, där det företräds av Juhani Soidinmäki. Fastigheten överförs år 2010 på \jhbold{Valkealan Tili- ja Kiinteistöpalvelu Oy}, ``VTK Oy'', hemort Kouvola, med fortsatt säte i Seinäjoki och med samma folk i bakgrunden.


%%%
% [occupant] Lampakka
%
\jhoccupant{Lampakka}{\jhname[Teuvo]{Lampakka, Teuvo} \& \jhname[Maarit]{Lampakka, Maarit}}{1990--\allowbreak 2007}
Fastigheten Källström 4:152 har mist sin glans och är numera ett öde förfallet hus, som bör rivas. Stadens myndigheter har inte kommit överens med nuvarande ägaren, som vägrar riva huset. Lampakka köpte huset 1990 och gjorde mindre renoveringar, plåt på vattentaket m.m. Teuvo Markus Lampakka och hustrun Maarit kom från Seinäjoki och bodde i huset 1990--\allowbreak 1994. Under tiden föddes en flicka 1991 och en pojke 1993. År 1994 återvände familjen till Seinäjoki.


%%%
% [occupant] Kirsilä
%
\jhoccupant{Kirsilä}{\jhname[skomakare]{Kirsilä, skomakare}}{1975--\allowbreak 1990}
Skomakare Kirsilä från Jakobstad fortsatte en tid med skoaffär och skomakeriverksamheten	efter Källström. Bl.a. Ingeborg Forss var under en kort period anställd som skomålare. Invalidpensionär, konstnär \jhname{Visti, Pentti Juhani}, \textborn 1944, bodde i huset några år. Han hittades mördad 1996 i en sandgrop i Socklot.


%%%
% [occupant] Källström
%
\jhoccupant{Källström}{\jhname[Runar]{Källström, Runar} \& \jhname[Helga]{Källström, Helga}}{1954--\allowbreak 1966}
Skomakare Karl Runar Källström, \textborn 17.09.1915 i Munsala och hans hustru Helga Evelina, \textborn 16.09.1916 i Korsholm, köpte 1954 av Johannes och Marianne Lindén bostadsbyggnaden samt ett bestämt tomtområde av lägenheten R:nr 4:116  av stomlägenheten Lindström 4:7. Den 04.05.1954 vid lagfart fick fastigheten benämning Källström 4:152.

I storstugan, där Jeppo Sparbank haft sitt första kontor, gjorde Källström ett större fönster och	öppnade en skoaffär samt hade skomakeri i en kammare. Runar kom till Jeppo 16.03.1949 och arbetade som skomakare i Nybygge 4:28, nr 370, nära stationen. Runar och Helga gifte sig 03.02.1951. År 1975 sålde Runar och Helga fastigheten till skomakare Kirsilä och emigrerade till Sverige.


%%%
% [occupant] Lindén
%
\jhoccupant{Lindén}{\jhname[Johannes]{Lindén, Johannes} \& \jhname[Marianne]{Lindén, Marianne}}{1946--\allowbreak 1954}
I samband med generationsskifte 1946 erhöll Johannes Lindén en del av stomlägenheten Lindström 4:7. På de erhållna områdena, Hemtomt 4:116, fanns familjens bostadsbyggnad med Johannes, Torhild och Margareta, gift med Tauno Stenvik, samt deras far Anders Lindén, som boende i mangårdsbyggnaden. Anders dog den 13.12.1950. Johannes gifte sig 1947 med Marianne Antbacka, \textborn 08.01.1927 i Kronoby. De flyttade in i den gamla röda stugan, vars tomt år 1925 hade delats från stomlägenheten och tilldelats namnet Lilland R:nr 4:36. De byggde ett nytt bostadshus bredvid den gamla stugan, nr 79 på karta 5.

Margareta och Tauno Stenvik bodde i huset när deras son Ralf Erik föddes den 07.03.1943. I slutet av 40-talet byggde de ett eget bostadshus på sin lägenhet Hemåker 4:163 (4:117) av stomlägenhet 4:7, nr 46 på karta 4.

Torhild gifte sig 1946 med Yngve Finskas och övertog 1946 stomlägenheten Lindström 4:7 (som då fick R:nr 4:118). De flyttade ladugården till Lövbacken, där de byggde sitt bostadshus, nr 123 på kartblad 9.

Den 09.01.1947 hyrde den nybildade banken, Jeppo Sparbank, storstugan i södra ändan av huset, som blev banklokal. Rakel Romar (gift Törnqvist) blev kamrer och Aidi Högdahl (gift Jungar) kassörska. Lokalen var kall och redan 1 oktober 1948 flyttade banken till ett av caférummen i Jeppo-Oravais Handelslags byggnad på andra sidan landsvägen.


%%%
% [occupant] Lindén
%
\jhoccupant{Lindén}{\jhname[Anders]{Lindén, Anders} \& \jhname[Hilda]{Lindén, Hilda}}{1916--\allowbreak 1946}
Anders Lindén (f.d.Kaukus), \textborn 15.06.1886 på Kaukus i Lassila, gift 1914	med Hilda Katrina Jungarå, \textborn 09.02.1888, dotter till bonden Samuel Jungarå. Före giftermålet hade Anders vistats några år i Amerika. Hans faster Marias man, Simon Thomasson Lindström, dog 1912 och deras son Vilhelm 1915 samt när kusin Ida Maria Lindströms man Isak avled år 1916, köpte Anders Lindén hemmanet, Lindström 4:7, på Silvast, där hans faster Maria varit husmor i 40 år. På 1930-talet kunde Anders förvärva även släkthemmanet Kaukus i Lassila. Anders och Hilda byggde ett större fähus med förråd närmare Silvast bäcken. Endast en smal väg gick mellan uthuset och bäcken. Anders och Hilda fick 6 barn.
\begin{jhchildren}
  \item \jhperson{\jhname[Anders Runar]{Lindén, Anders Runar}}{23.11.1915}{05.03.2008}
  \item \jhperson{\jhname[Johannes Edvin]{Lindén, Johannes Edvin}}{21.01.1917}{18.08.1973}
  \item \jhperson{\jhname[Paul Jakob]{Lindén, Paul Jakob}}{16.08.1918}{26.07.1920}
  \item \jhperson{\jhname[Heldine Margareta]{Lindén, Heldine Margareta}}{16.05.1920}{04.09.1976}
  \item \jhperson{\jhname[Erik Robert]{Lindén, Erik Robert}}{01.07.1921}{25.07.1945}
  \item \jhperson{\jhname[Torhild Elisabet]{Lindén, Torhild Elisabet}}{23.12.1922}{19.02.2015}
\end{jhchildren}
Runar gifte sig som tjugoåring med kontrollassistent Annie Träsk, \textborn 1915 i Komossa. Deras äldsta son Boris föddes 07.06.1940 på hemgården. Runar blev bonde på Kaukus hemman i Lassila, Johannes blev stationskarl, se nr 79, Paul dog 26.07.1920 i mässling, Margareta gifte sig med poliskonstapel Tauno Stenvik, se nr 46, Erik avled som ogift på sjukhus 25.07.1945 av granatskärvor han fått i magen under kriget. Som ovan nämnts blev Torhild tillsammans med sin man Yngve Finskas jordbrukare på Lövbacken.

Anders var aktiv inom kommunalfullmäktige och flera nämnder samt i styrelsen för kvarnen, andelsmejeriet, handelslaget och lantmannagillet i Jeppo. Han blev tidigt änkling, då hans hustru Hilda dog den 16.07.1929. Under några år, före sin död 16.12.1944, var Anders' kusin Hilda Maria Jungarå, gift Vikström, hushållerska hos familjen. Anders dog 13.12.1950.


%%%
% [occupant] Lindström
%
\jhoccupant{Lindström}{\jhname[Isak]{Lindström, Isak} \& \jhname[Ida]{Lindström, Ida}}{1907--\allowbreak 1916}
Ida Maria Simonsdotter Lindström, \textborn 18.01.1879 på Silvast hemman, var dotter till Maria och Simon Thomasson. År 1907 gifte hon sig med torparsonen på Böös hemman i Jeppo, Isak Löfgren, \textborn 22.11.1872. Makarna antog Idas tillnamn, Lindström, och arbetade på Idas hemgård på Silvast. Efter far Simons död 1912 och bror Vilhelms död 1915 blev Ida tillsammans med mor Maria ägare till lägenheten Lindström 4:7. Isak och Ida fick fyra barn.
\begin{jhchildren}
  \item \jhperson{\jhname[Anni Linnea]{Lindström, Anni Linnea}}{19.08.1908}{25.04.1909}
  \item \jhperson{\jhname[Anni Maria]{Lindström, Anni Maria}}{12.01.1910}{}
  \item \jhperson{\jhname[Isak Evert]{Lindström, Isak Evert}}{24.02.1913}{}
  \item \jhperson{\jhname[Ida Elisabet]{Lindström, Ida Elisabet}}{02.07.1915}{23.01.1916}
\end{jhchildren}
Anni Linnea och Ida Elisabet dog som spädbarn. Anni Maria gifte sig med Torsten Forss. De bosatte sig på Forss hemman och fick 4 barn. Se karta 7, nr 108. Evert bodde med sin mor Ida tills han som 46-åring gifte sig med mejerskan Iris Runhild Lingonblad, \textborn 26.04.1923 i Oravais. Se mera nr 374, Björkhagen 4:160 och nr 69, Härbärge 4:161.

Idas man Isak dog redan 1916 och då såldes hemmanet, Lindström 4:7, då med R:nr 4:20, omfattande 0,1521 mantal av Silvast 4, till Idas kusin Anders Kaukus, som senare antog tillnamnet Lindén. Ida flyttade då med barnen Anni och Evert till det gamla röda bostadshuset. Till familjen hörde även mor Maria och bror Vilhelms änka, Ida Susanna, som 1920 flyttade tillbaka till Munsala. Mor Maria dog 1922. Det gamla bostadshusets tomt delades från stomlägenheten och fick 1925 namnet Lilland 4:36, enligt notering i jordregistret, med ägare Lindström Maria. Ida byggde också genast ett nytt bostadshus närmare järnvägsstationen. Det nya huset uthyrdes 1920--\allowbreak 1930 för affärsverksamhet.


%%%
% [occupant] Thomasson
%
\jhoccupant{Thomasson}{\jhname[Simon]{Thomasson, Simon} \& \jhname[Maria]{Thomasson, Maria}}{1870--\allowbreak 1912}
Simon Thomasson, \textborn 04.08.1848 på Fors hemman, gifte sig 1870 med	Maria Böös, \textborn 01.02.1847. De blev ägare till ¼ om 0,1771 mantal av Silvast skattehemman 4. Säljare Jakob Larsson. På 1890-talet antog de tillnamnet Lindström. I samband med slutförd lantmäteriförrättning år 1915 av Silvast skattehemman 4, fick lägenheten namnet Lindström R:nr 4:7. Under 1915 noterades i jordregistret att Jakobsson Ida blivit ägare till Holmen R:nr 4:18 och Orrholmen 4:19, varefter stomlägenheten Lindström fick R:nr 4:20, om 0,1552 mantal. Simon och Maria byggde på 1870-talet ett nytt större bostadshus och fähus m.m. nära landsvägen på ovannämnda fastighet, nr 79, numera R:nr 4:152, som 1954 utstyckades till en skild lägenhet. Simon och Maria fick 4 barn.
\begin{jhchildren}
  \item \jhperson{\jhname[Anna Lovisa]{Thomasson, Anna Lovisa}}{01.07.1872}{1962 i Amerika}
  \item \jhperson{\jhname[Ida Maria]{Thomasson, Ida Maria}}{18.01.1878}{09.05.1965}
  \item \jhperson{\jhname[Simon Vilhelm]{Thomasson, Simon Vilhelm}}{25.04.1882}{01.01.1915}
  \item \jhperson{\jhname[Johannes]{Thomasson, Johannes}}{20.03.1885}{13.12.1906 i Amerika}
\end{jhchildren}
Fadern Simon avled 09.04.1912 och modern Maria 13.02.1922. Marias far, Jakob Böös, \textborn 1821, fick ha sitt hem hos Maria och Simon ett trettiotal år före sin död 1907. Sonen Vilhelm gifte sig i 20-års åldern med Ida Susanna Fogel, från Munsala. Paret blev barnlöst. De bodde och arbetade på hemmanet, likaså systern Ida och hennes man Isak. Efter Vilhelms död 1915 flyttade änkan Ida Susanna år 1920 tillbaka till Munsala, där hon är begraven. Mor Maria med dotter Ida blev ägare till hemmanet. År 1916 dog Idas man Isak och Lindström 4:20 såldes till Anders Kaukus, senare Lindén.


\jhbold{Magasin på 4:152, nr 357b}

På tomtområdet vid korsningen av Östra Jeppovägen och Stationsvägen fanns ännu i slutet av 1940-talet en magasinbyggnad. Området ligger nu under vägkorsningen. Fotograf Mikko Peltonen från Kemijärvi hade fotoateljé i byggnaden 1945 och några år därefter. Peltonen bodde hos familjen Selim Romar på Furubacken. Byggnaden troligen uppförd av Lindström i början av 1900-talet och överlåten till Handelslaget. Den 29.02.1934 finns ett köpebrev på mantal under styckning, säljare Ida Lindström, köpare Jeppo-Oravais Handelslag. Nedan ett fotografi på byggnaden. Se även foto under Handelslag, karta 5, nr 57.

\jhpic{Fotoatelje Mikko Peltonen 1945 vid Hlg.jpg}{Fotograf Mikko Peltonen hade ateljé vid Handelslaget 1945}



%%%
% [house] Lilland-Hemtomt
%
\jhhouse{Lilland-Hemtomt}{4:36 resp. 4:154}{Silvast}{5}{79, 379}

Styckad av stomlägenhet Lindström 4:7

\jhhousepic{033-05563.jpg}{Börje och Doris Lundvik}

%%%
% [occupant] Lundvik
%
\jhoccupant{Lundvik}{\jhname[Börje]{Lundvik, Börje} \& \jhname[Doris]{Lundvik, Doris}}{1955--}
Den 23.10.1955 köpte Börje Lundvik, \textborn 04.05.1935 på Kaup hemman, fastigheterna Lilland 4:36 och Hemtomt 4:154 med bostadsbyggnad. Tomterna utgjorde en del av stomlägenhet Lindström 4:7. Lagfart erhölls 25.01.1956. Den 26.08.1956 vigdes Börje med Doris Westin, \textborn 02.12.1931 på Grötas hemman (se karta 10, nr 101.

Börje har från unga år varit intresserad av bilar och arbetat som chaufför, bl.a. som långtradarchaufför inom Norden på Oy Keppo Ab:s pälsdjursfarm. Han har haft egen taxiverksamhet under många år och hade hand om skolskjutsar inom Jeppo. Barnen tyckte mycket om och beundrade sin chaufför. Vid pensionering år 2000 överlät han taxirättigheten till Jens Björklund.

Doris lärde sig frisörsyrket av Jenny Sandberg, och öppnade damfrisering i ett rum i bostaden. Damfriseringen arbetade hon med till sin pensionering vid 65 års ålder.  Börje och Doris fick två barn.
\begin{jhchildren}
  \item \jhperson{\jhname[Hans]{Lundvik, Hans}}{19.12.1958}{03.09.1983}
  \item \jhperson{\jhname[Regina]{Lundvik, Regina}}{09.06.1959}{}
\end{jhchildren}
Hans var också intresserad av bilar och blev liksom sin far långtradarchaufför på utlandet. Hans var förlovad, då han 3 september 1983 dog i en trafikolycka i arbetet utanför Vasa.  Regina blev merkonom från Jakobstads Handelsläroverk. Hon arbetade på Oy Keppo Ab:s kontor innan hon flyttade till Stockholm, där hon numera är redovisningschef på en större firma. Regina är gift med en man från Norge och de har tre flickor. Börje och Doris har byggt ett uthus med flera garage, pannrum och förråd på Hemtomt 4:154.

Börje \textdied 23.06.2017


%%%
% [occupant] Lindén
%
\jhoccupant{Lindén}{\jhname[Johannes]{Lindén, Johannes} \& \jhname[Marianne]{Lindén, Marianne}}{1946--\allowbreak 1955}
Johannes Edvin Lindén , \textborn 21.01. 1917 på stomlägenhet Lindström 4:7 av Silvast hemman, erhöll vid generationsskifte 18.03.1946, Hemtomt 4:116 och Lilland 4:36, som utgjorde en del av stomlägenheten.  Johannes gifte sig 1947 med Inga Marianne Antbacka, \textborn 08.01.1927 i Kronoby. De bosatte sig i den gamla röda stugan på Lilland 4:36, där de byggde ett nytt bostadshus. Johannes arbetade som stationskarl i Jeppo och från 1950 i Kovjoki. År 1965 blev han tågkarl i Jakobstad. Huset i Jeppo sålde de 1955 och familjen flyttade till Jakobstad. Johannes dog 18.08.1973 i Jakobstad och Marianne flyttade under några år till Sverige, men återkom till Jakobstad där hon dog 2004.

I samband med att Johannes sålde hemgården 1954 till skomakare Källström, utstyckades tomt för den med R:nr 4:152 och Hemtomt fick R:nr 4:154. På området Hemtomt fanns tidigare familjen Lindéns ekonomibyggnad, som 1946 flyttades av Yngve och Torhild Finskas till Lövbacken. Den gamla röda stugan monterades ner 1955 av Hilding och Ellen Nygård och blev sommarstuga på Kalvholmen i Oravais.

Johannes och Marianne fick två flickor.
\begin{jhchildren}
  \item \jhperson{\jhname[Gerd Marianne]{Lindén, Gerd Marianne}}{26.12.1947}{}
  \item \jhperson{\jhname[Gunn Margareta]{Lindén, Gunn Margareta}}{01.05.1949}{}
\end{jhchildren}
Gerd blev merkonom från Jakobstads Handelsläroverk. 1972 gifte hon sig med Justine Cullar Garcio-Plaza, \textborn 1950 i Spanien. Gerd är språklärare i Madrid. Gunn blev också merkonom och är sambo med konsult Bertil Dahllund i Esse.


%%%
% [occupant] Lindström
%
\jhoccupant{Lindström}{\jhname[Ida]{Lindström, Ida} \& \jhname[Maria]{Lindström, Maria}}{1916--\allowbreak 1930}
Ida Maria Simonsdotter Lindström och hennes mor Maria blev ägare till stomlägenhet Lindström 4:7 efter att Idas far Simon Thomasson Lindström och hennes bror Vilhelm dött. När Idas man Isak, \textborn 1872 på Böös, dog 1916, såldes hemmanet till Idas kusin Anders Kaukus senare Lindén. Ida flyttade då med barnen Anni och Evert samt sin mor Maria till den gamla stugan. Till familjen hörde även bror Vilhelms änka Ida Susanna, som 1920 flyttade tillbaka till Munsala. Mor Maria dog 1922. I jordregistret 1925 noteras att Lindström 4:7 (då med R:nr 4:20) delats i Lilland 4:36, ägare Lindström Maria och Lindström 4:37, ägare Kaukus Anders. Ida byggde ett nytt bostadshus, närmare järnvägsstationen. Nedre våningen uthyrdes 1920--\allowbreak 1930 för affärsverksamhet. Se mera nr 74, Björkhagen 4:160.


\jhbold{Hyresgäster}

Många familjer har under 1930--\allowbreak 1950 talen bott som hyresgäster i huset. 1931 flyttade skräddare Rintala med familj till Jeppo och hyrde huset. Till familjen hörde Jaakko Gustavsson Rintala född 24.05.1886 i Lappo och hans hustru Sanna Adolfiina, född Kaara 07.01.1887 i Nurmo, tidigare gift Halme (dotter Vellamo Halme) samt sönerna Rintala, Osmo född 23.03.1916 i Lappo och Oiva född 22.09.1918 i Lappo. Familjen Rintala flyttade till Grötas 1938 och 1961 till Orrholms hus på Fors hemman.

Under år 1939--\allowbreak 1943 var hyresgästerna Alexander och Ellen Sandqvist med barnen Ragnar, Holger, Hjördis och Marita, alla födda på fastighet Sandqvist R:nr 4:21, nr 62 på karta 5, samt Lea och Eva, båda födda på fastighet Dahlbo R:nr 4:127, nr 77, karta 5.	Till julen 1943 flyttade familjen till sitt nybyggda hus på Nylandsvägen, Sandqvist 2:98, karta 3, nr 38.

Kajsa-Lovisa Jakobsson bodde också en tid i bostaden efter försäljningen av Nybygge 4:28.

Det lilla rödmålade huset, omgivet av stora vackra björkar byggdes antingen av Petter Gustavsson i medlet av 1800-talet eller av Jakob Larsson, ägare till den del av Silvast hemman om 0,1771 mantal ¼, som Simon Thomasson köpte på slutet av 1870-talet, och som 1915 infördes i jordregistret som skattelägenhet Lindström 4:7. I mantalslängden 1875 ägde Jakob Larsson, Silvast 4:1 och Silvast 4:2, således halva hemmanet.	Jakob Larsson \textborn 02.02.1832 och hans hustru Kajsa Eriksdotter \textborn 30.05.1823 fick sonen Jakob Jakobsson \textborn 22.04.1857, se Orrholm 	4:141 nr 372. Larssons fick också en dotter Kajsa Sofia \textborn 01.07.1864.



%%%
% [house] Sand-Ek
%
\jhhouse{Sand-Ek}{4:134}{Silvast}{5}{84}

Styckad av stomlägenhet Orrholm 4:13

\jhhousepic{052-05584.jpg}{Jorma och Hilkka Teikari}

%%%
% [occupant] Teikari
%
\jhoccupant{Teikari}{\jhname[Jorma]{Teikari, Jorma} \& \jhname[Hilkka]{Teikari, Hilkka}}{1982--}
Jorma Teikari, \textborn 1949 i Lauhosaari, Kemi landskommun, och hans 	hustru Hilkka, \textborn 1950 i Kemi, köpte 1982 lägenheten Sand-Ek R:nr 4:134, som utgör en del av skattelägenhet Orrholm R:nr 4:13. De kom till Jeppo 1979.
\begin{jhchildren}
  \item \jhperson{\jhname[Janne]{Teikari, Janne}}{1971 i Kemi}{}
  \item \jhperson{\jhname[Hanna]{Teikari, Hanna}}{1977 i Kemi}{}
\end{jhchildren}
Familjen bodde i en av skolans lärarbostäder 1979--\allowbreak 1982.	Janne bor i Alahärmä, Kauhava, och arbetar på Jepo-Voltti potatisfabrik. Hanna är gift Aalto och har 3 barn, bor i Tavastehus. Jorma har arbetat 1979--\allowbreak 2012 på KWH Mirkas maskinverkstad. Hans fritid fylldes också av metallarbeten. Hilkka, som är utbildad merkonom, har arbetat på Mirkas Logistikavdelning som strukturkoordinator	1979--\allowbreak 2009, då hon erhöll sjukpension. Hilkka är intresserad av hundar. Bostadshuset har de renoverat i flera repriser både in- och utvändigt.


%%%
% [occupant] Romar
%
\jhoccupant{Romar}{\jhname[Edvin]{Romar, Edvin} \& \jhname[Alexandra]{Romar, Alexandra}}{1960--\allowbreak 1982}
Byggmästare Erik Edvin Andersson Romar, \textborn 10.07.1896 på Romar hemman i Jeppo, och hans hustru Alexandra, född Andersdotter Åvist, \textborn 22.10.1895 i Purmo, köpte 17 april 1960 lägenheten Sand-Ek 4:134. De flyttade från banmästarbostaden på stationsområdet i Jeppo. Edvin började arbeta som ung på järnvägen i Kovjoki och blev banvakt vid Byttmåsa banvaktstuga nära Sorvist i Kovjoki, där barnen är födda. Han studerade 1927--\allowbreak 1930 inom bro, väg och vatten i Industriskolan i Vasa. Han fick tjänst 1930 som förman på järnvägen i Gamlakarleby och familjen flyttade dit. År 1938 fick han banmästartjänsten vid Puukari i Valtimo, som han innehade till 1948, då han med familjen flyttade till Jeppo som banmästare mellan Bennäs och Voltti. Barnen är födda i Kovjoki.
\begin{jhchildren}
  \item \jhperson{\jhname[Margit Alexandra]{Romar, Margit Alexandra}}{1918}{1937}
  \item \jhperson{\jhname[Eva Maria]{Romar, Eva Maria}}{01.09.1920}{2014}
  \item \jhperson{\jhname[Åke Edvin]{Romar, Åke Edvin}}{02.05.1927}{}
  \item \jhperson{\jhname[Sven Erik]{Romar, Sven Erik}}{22.07.1929}{}
\end{jhchildren}
Eva var lotta inom luftvärnet under kriget. Hon återvände från Jeppo till Valtimo som posttjänsteman och gifte sig där med Kilpiläinen. De har en dotter och en son. Åke var en god friidrottare i Jeppo Idrottsförening. Åke är gift med Kerstin och bor i Karleby och har två flickor. Åke har arbetat som distriktschef inom Esso-bolagen. Sven är forstmästare och bor i Rovaniemi och har en son och dotter.

Edvin Romar gick i pension 1960 och dog i Jeppo den 28.06.1974. Alexandra levde till den 10.07.1981, då hon dog i hemmet. De är begravna i Nykarleby i familjegraven. 1982 såldes Sand-Ek fastigheten.


%%%
% [occupant] Silén
%
\jhoccupant{Silén}{\jhname[Petter]{Silén, Petter} \& \jhname[Gunhild]{Silén, Gunhild}}{1945--\allowbreak 1960}
Styrman/dykare Petrus (Petter) Mikael Silén, \textborn 21.04.1913 i Lojo, och hans hustru Edit Gunhild, född Finskas, \textborn 06.09.1913 på Finskas hemman i Jeppo, köpte lägenheten Sand-Ek 1945. Petter och Gunhild träffades när de gick i Evangeliska folkhögskolan på Keppo gård 1931--\allowbreak 1932. De vigdes i Jeppo kyrka den 04.09.1938 och bosatte sig i Helsingfors. Petter hade efter militärtiden utbildat sig till dykare i krigsflottan och fortsatte 4 år inom Marinen. År 1937 fick han arbete som dykare/styrman på bärgningsfartyget Nepthuni. Under kriget bodde familjen tidvis i Jeppo. Den 19.01.1945 flyttade familjen till sin nyinköpta bostad på Silvast.
\begin{jhchildren}
  \item \jhperson{\jhname[Bengt Rangvald Mikael]{Silén, Bengt Rangvald Mikael}}{11.05.1940}{05.09.2014}
  \item \jhperson{\jhname[Bernhard Gustav]{Silén, Bernhard Gustav}}{29.09.1944}{05.11.1990}
  \item \jhperson{\jhname[Rurik Matts Gerald]{Silén, Rurik Matts Gerald}}{13.04.1946}{22.04.2013}
\end{jhchildren}
Bengt och Bernhard föddes när de bodde i Helsingfors och Rurik i Jeppo. Petter och Gunhild var aktiva som frivilliga medarbetare inom Evangeliska Unga och församlingen. Familjen flyttade den 18.12.1952 till Vasa. Bengt blev lärare, Bernhard polis och Rurik merkonom och adb-informationschef.

Gunhild \textdied 27.05.1987 i Vasa  ---  Petter \textdied 26.04.1993.


\jhbold{Hyresgäster:}
Som pensionär hyrde Carl Didrik (Charles) von Essen, \textborn 1875 i Helsingfors, och hustru Martha, född Bredelow, \textborn 1890 i Magdeburg, Tyskland, fastigheten under 1953--\allowbreak 1956. Charles var son till Carl Jonathan von Essen på Kiitola. Charles och Martha hade inga barn. \jhname[Charles von Essen]{von Essen, Charles \& Martha} hade studerat i Vaasan Lyseo och arbetat på ullspinneriet och sågen som chef på olika poster och som disponent för Kiitola gård.

Charles von Essen \textdied 1954  ---  Martha \textdied 1965. De är begravna i familjegraven på Jeppo begravningsgård.

Postfröken Doris Elenius, född på Fors hemman, hyrde övre våningen en kortare tid i början av 1950-talet.

Bokhållare Runar Holmlund, \textborn 09.12.1923, i Seinäjoki, och hustru Blondine, \textborn 08.02.1925 i Solf, hyrde fastigheten 1957--\allowbreak 1960.
\begin{jhchildren}
  \item \jhperson{\jhname[Johan Erik]{Silén, Johan Erik}}{18.10.1949 i Jeppo}{}
  \item \jhperson{\jhname[Ann-Christine]{Silén, Ann-Christine}}{14.07.1952 i Jeppo}{}
\end{jhchildren}
Familjen flyttade från Thors fastighet R:nr 4:143, nr 19, där de bodde i övre våningen från 1950 och senare i nedre våningen efter att familjen Kronlöf flyttat till Sandbergs hus, Strandbo 4:73. \jhname[Runar]{Holmlund, Runar \& Blondine} är krigsveteran efter ca.1,5 år vid fronten. Efter hemförlovningen började han arbeta på SJ. Runar och Blondine kom till Jeppo i juni 1949 och bodde först ca 1 år i övre våningen av Ströms hus, Sandqvist 4:21, nr 62. Familjen Holmlund flyttade efter 10,5 år i Jeppo till Kaskö 1960 och därifrån 1964 till Vasa. Blondine \textdied 26.04.2014.

Johan föddes när de bodde på nr 62. Han är byggnadsingenjör och har några år arbetat i Nykarleby, men en längre tid i Vasa.


%%%
% [occupant] Sandberg
%
\jhoccupant{Sandberg}{\jhname[Leander]{Sandberg, Leander} \& \jhname[Ellen]{Sandberg, Ellen}}{1938--\allowbreak 1945}
Leander Sandberg, \textborn 18.05.1906 på Ojala i Jeppo och h. h. Ellen, född Jungell, \textborn 05.01.1908 på Skog hemman, köpte 1938 fastigheten Sand-Ek 4:134, som utgjorde en del av stomlägenhet Orrholm 4:13. De byggde bostadshuset och ett mindre uthus på tomten. På området fanns från tidigare en mindre byggnad. Ellen hade blivit moderlös vid 9 månader och växte upp hos sin farmor, som dog när Ellen var 18 år. Hon flyttade då till sin farbror Anders Jungell och h. h. Olivia på Gästgiveriet i Silvast. Ellen och Leander gifte sig 15.05.1932 och bodde till 1938 på Gästgiveriet, nr 380 på karta 5. De hade ett mindre jordbruk, som förstorades genom köp och byten. Leander var en skicklig snickare och arbetade även inom byggnadsbranschen. Under vinterkriget var han inkallad. Under fortsättningskriget fungerade han som polis på hemmafronten samt arbetade med uppbyggnad och underhåll av telefonlinjer.
\begin{jhchildren}
  \item \jhperson{\jhname[Magda Katarina]{Sandberg, Magda Katarina}}{17.11.1933}{09.05.1999}
  \item \jhperson{\jhname[Ingmar Matias]{Sandberg, Ingmar Matias}}{04.11.1937}{}
\end{jhchildren}
Magda gifte sig 14.02.1954 med jordbrukare Ingmar Johannes Björkvik, \textborn 30.12.1927 på Lassila hemman. De var aktiva jordbrukare med mjölkproduktion fram till 1997. Magda och Ingmar fick tre döttrar och en son; Stina, Britt, Lena och Jan. Jan drabbades av en svår muskelsjukdom och dog ung 20.07.2007.

Ingmar blev merkonom 1957 och utbildade sig senare till lärare 1964. Ingmar gifte sig 19.06,1961 med Solveig Kristina, född Nyberg, \textborn 24.07.1940 i Korsholm. De har en son Jens Fredrik, \textborn 24.05.1965 i Nykarleby och en dotter Isa Kristina, \textborn 05.02.1970 i Vasa. Ingmar hade sin första lärartjänst i Nykarleby läsåret 1964--\allowbreak 1965. Under seminarietiden bodde Solveig och Ingmar med Ingmars far Leander på Fors hemman, nr 407. Ingmar arbetade som lärare i Vasa 1965--\allowbreak 1973 och inom skoladministrationen 1974--\allowbreak 1986 och därefter som rektor i Korsholms Högstadieskola 1986--\allowbreak 1997, då han gick i pension, men fortsatte som skoldirektör i Korsholm till 1999.

År 1944 gjorde Leander och Ellen samt Ellens fostermor Olivia Jungell ett köp/byte med Katarina Sikström. Leander, Ellen, Olivia samt barnen Magda och Ingmar flyttade till Sikströms stora hus, Karlsborg och Katarina och Karl-Johan till Jungells hus, båda på Fors hemman. Se nr 99 och 407 (107) på karta 6.

Ellen dog 1961 endast 53 år gammal och Leander dog 1975. Som änkling byggde Leander 1973 ett nytt bostadshus och rev det stora gamla huset på Karlsborg.



%%%
% [house] Mågas
%
\jhhouse{Mågas}{4:139}{Silvast}{5}{85}

Styckad av stomlägenhet Orrholm 4:13

%%%
% [occupant] Stenvall
%
\jhoccupant{Stenvall}{\jhname[Göran]{Stenvall, Göran} \& \jhname[Gunborg]{Stenvall, Gunborg}}{2011--}
Efter att föräldrarna Ragnar Stenvall dog 27.06.2004 och Aino	dog 19.07.2011, blev Göran Stenvall ägare till fastigheten. Huset används som förvaringsrum. Under några år har på sommaren anordnats loppis där.


\jhhousepic{050-05583.jpg}{Mågas}

%%%
% [occupant] Stenvall
%
\jhoccupant{Stenvall}{\jhname[Ragnar]{Stenvall, Ragnar} \& \jhname[Aino]{Stenvall, Aino}}{1973--\allowbreak 2011}
År 1973 köpte stationskarl Ragnar Stenvall, \textborn 06.08.1919 på Nygård skattelägenhet 4:9/Silvast och hans hustru Aino född	Koukkuluoma, \textborn 11.02.1918 på Mäki 4:34/Silvast, fastigheten Mågas som sommarbostad. De hade 1968 flyttat till Jakobstad på grund av förändringar inom SJ. Varje sommar tillbringade de i Jeppo, då de reparerade bostadshuset och arbetade i trädgården. Se mera Öständ 4:197, karta 9, nr 126.


%%%
% [occupant] Källman
%
\jhoccupant{Källman}{\jhname[Ester]{Källman, Ester}}{1959--\allowbreak 1973}
Ester Sofia Källman, \textborn 01.11.1904 i USA, ärvde 1959 fastigheten Mågas av sin faster Anna-Sanna Mågas. Vad beträffar Ester se mera lägenhet Bäckström 3:96 på Fors hemman, nr 90 och 390. Ester hyrde ut fastigheten.


\jhbold{Hyresgäster}:

På 1950-talet hyrdes fastigheten av Aaltonen Juho Kustaa, \textborn 1885 och hans sambo Jokinen Lydia Maria, \textborn 1893. De kallades i dagligt tal för Aallon-Jussi och Maija. De fick ofta besök av Maijas dotter och en dotterdotter Soili från Helsingfors. På sin ålderdom bodde de i en pensionärsbostad på Fors. Juho dog 1976 och Maria 1975. De är begravna i Jeppo.

Asko Esaias Linjamäki, \textborn 15.07.1939 i Rejpelt, Vörå, och hans mor Siri Marja, född Lehmäjoki, \textborn 17.10.1910 i Storkyro, hyrde fastigheten 1959--\allowbreak 1963. De köpte därefter Ingmar och Sally Dahlströms hus på Jungarå år 1963.	Asko gifte sig med Marja-Liisa, född Mäki, \textborn 1948 i Alahärmä. De har en dotter Merja Marianne, \textborn 1967 på Jungarå. Merja är merkant, gift Tikkamäki i Alahärmä. Siri dog 21.01.1980 och är begraven i Jeppo, likaså Askos far, som inte bodde med dem, Esais Linjamäki, född 06.06.1903 och dog 20.08.1973.


%%%
% [occupant] Mågas
%
\jhoccupant{Mågas}{\jhname[Karl]{Mågas, Karl} \& \jhname[Anna-Sanna]{Mågas, Anna-Sanna}}{1930--\allowbreak 1959}
Anna-Sanna Källman, \textborn 05.05.1871 på Jungarå, gifte sig 1895 med Karl Eriksson Mågas, \textborn 1869 i Hirvlax, Munsala. Anna-Sanna och Edvard Källman var syskon. Anna-Sanna och Karl fick en son 26.10.1895 då de bodde på Grötas och som dog 12.11 samma år. Den 29.01.1898 köpte Karl och Anna-Sanna Mågas Norrgård 4:8 om 0,664 mantal av Silvast hemman.

Karl led av svår astma och behövde luftombyte. Karl reste 7 gånger och Anna-Sanna 3 gånger till Amerika, där Karl kunde arbeta. Mellan resorna bodde de i Köurus från 1903 till 1916, då de flyttade till Vesterbacka/Pensala och 1928 köpte de en gård i Pensala. Karl dog 1929 i Pensala. Anna-Sanna köpte 1930 en tomt, byggde ett bostadshus i 1½ plan med två rum och tambur på den då utbrutna	tomten Mågas 4:139 av skattelägenhet Orrholm 4:13 av Silvast hemman. Hon blev tvungen att flytta till ålderdomshemmet Östervall i Nykarleby i början av 1950-talet. Hon dog 01.11.1959.



%%%
% [house] Orrholm
%
\jhhouse{Orrholm}{4:141}{Silvast}{5}{373}

Styckad av stomlägenhet Orrholm 4:13

%%%
% [occupant] Dudenko
%
\jhoccupant{Dudenko}{\jhname[Mykola]{Dudenko, Mykola} \& \jhname[Natalia]{Dudenko, Natalia}}{2015--}
På sommaren 2015 köpte Dudenko Mykola och hans hustru Natalia, från Ukraina, tomten Orrholm 4:141, som är en del av f.d. Sundells Bageri-tomt. Se mera om familjen under nr 72, Bageri 4:23.


%%%
% [occupant] Nyvall-Furu
%
\jhoccupant{Nyvall-Furu}{\jhname[A]{Nyvall-Furu, A} \& \jhname[J]{Nyvall-Furu, J}}{2000--\allowbreak 2015}
Orrholm R:nr 1:141 utgör numera en del av tomten som tillhör  f.d. Sundells Bageri och ingick i köpet när  Nyvall Andreas och Furu Jessica köpte bageriet. Mera finns omnämnt under fastighet, Bageri R:nr 4:23, nr 72 på karta 5. Erland Sundell köpte fastigheten av Valter och Ragni Willman 1963 och rev byggnaderna.


%%%
% [occupant] Willman
%
\jhoccupant{Willman}{\jhname[Valter]{Willman, Valter} \& \jhname[Ragni]{Willman, Ragni}}{1946--\allowbreak 1963}
Willman Valter Alexander, \textborn 11.06.1925  i Purmo, och h. h. Ragni Linnea, född Sundell, \textborn 12.09.1926 på Jungar och uppvuxna som grannar på Silvast, bosatte sig i byggnaden efter vigseln 31.08.1947. Valter hade övertagit fastigheten 1946 och bodde där med sin mamma Tatiana och syster Ruth Elisabeth. Även mormor Sanna bodde i huset. Valter arbetade som stationskarl i Jeppo och Ragni på Sundells butik.
\begin{jhchildren}
  \item \jhperson{\jhname[Inga Marianne]{Willman, Inga Marianne}}{16.10.1947}{}
  \item \jhperson{\jhname[Kenneth Valter Johannes]{Willman, Kenneth Valter Johannes}}{28.12.1956}{}
\end{jhchildren}
År 1963 flyttade familjen till Jakobstad på grund av förändringar av arbetsförhållanden inom Statens järnvägar. Både Marianne och Kenneth har studerat vid Jakobstads Handelsläroverk. De emigrerade till Sverige. Marianne bor i Visby/Vikingsta och har en dotter, Maria Susanne f. 06.02.1972 och Kenneth i Filipstad, som också har en dotter, Linnéa Maria f. 14.01.1989.

Maria Tatjana, \textborn 16.01.1898, Jakobsdotter Orrholm på Silvast, gifte sig 17.04.1921 med Werner Finne, (1899--\allowbreak 1923) från Markby i Nykarleby lkm. De nygifta bosatte sig i Purmo. De fick en dotter som dog. Tatjana gifte om sig 1924 med Herman Willman, \textborn 11.07.1878 i Purmo. Herman emigrerade till USA 1939 och dog i Tascona, WA 1945. Tatjana och Herman fick barnen, Valter Alexander född i Purmo och Ruth Elisabeth, \textborn 22.05.1932 i Purmo. När Tatjana blev änka, flyttade hon år 1946 med Valter och Ruth tillbaka till barndomshemmet vid Stationsvägen i Silvast. Ruth och Tatjana Orrholm emigrerade till Sverige år 1951. Tatjana dog i Västerås 06.02.1975. Ruth gift med Anders Gustafsson i Västerås, skilsmässa. Ruth bodde ett tiotal år i Stockholm, där hon dog 03.06.2017.


%%%
% [occupant] Orrholm
%
\jhoccupant{Orrholm}{\jhname[Jakob]{Orrholm, Jakob} \& \jhname[Sanna]{Orrholm, Sanna}}{1893--\allowbreak 1946}
Jakob Jakobsson Orrholm, \textborn 22.04.1857 på Silvast, och h. h. Sanna Andersdotter Stam, \textborn 19.06.1870 på Keppo och uppvuxen på Svartbacken, byggde ett större bostadshus (gästgiveri) 1893 vid Stationsvägen på skattelägenhet Orrholm 4:13, närmast stationsområdet, numera Orrholm 4:141. Jakob och Sanna gifte sig 24.11.1892.

Jakob var son till Jakob Larsson och hans hustru Kajsa, född Eriksson, som 1875 ägde 0,3541 mantal av Silvast hemman med åbo Simon Tomasson och Gustav Gustavsson. I 1880 års mantalslängd kvarstår Jakob Larsson som ägare till 0,1771 mantal med åbo Simon Tomasson, som på 1970-talet köpt 0,1771 mantal av Silvast hemman. Efter sammanslagningar och klyvning av Silvast erhöll Jakob 0,0166 mantal och Larssons dotter, Ida Jakobsson, 0,0249 mantal på Holmen.

Den 9 november 1907 erhöll bonden Jakob Silfvast, sedan Orrholm och hans hustru Susanna Silfvast, fastebrev på 17/1024 mantal av Silfvast benämnda jordebokshemman nummer 4 i Jungar by av Jeppo socken, vilken fastighet de tillhandlat sig.
Den 15.02.1915 infördes i jordregistret de av skattelägenheterna 4:1, 4:3 och 4:4, den 23 februari 1898, bildade skattelägenheterna 4:6 – 4:13.

Vid klyvning år 1810 av Silfvast skattehemman R:nr 4 i Jungar by bildades skattelägenheterna 4:1 – 4:4. Den 09.09.1952 har skattelägenheten Orrholm R:nr 4:13 styckats i 4:134 – 4:141.
\begin{jhchildren}
  \item \jhperson{\jhname[Katrina Alina]{Orrholm, Katrina Alina}}{04.04.1895}{18.09.1981}
  \item \jhperson{\jhname[Jakob Haniel]{Orrholm, Jakob Haniel}}{14.09.1896}{27.05.1957}
  \item \jhperson{\jhname[Maria Tatiana]{Orrholm, Maria Tatiana}}{16.01.1898}{06.02.1975}
  \item \jhperson{\jhname[Ellen Sofia]{Orrholm, Ellen Sofia}}{30.08.1901}{05.03.1905}
  \item \jhperson{\jhname[John Villiam]{Orrholm, John Villiam}}{23.10.1903}{17.09.1904}
  \item \jhperson{\jhname[John Villiam]{Orrholm, John Villiam}}{01.07.1905}{01.06.1914}
  \item \jhperson{\jhname[Ellen Susanna]{Orrholm, Ellen Susanna}}{12.01.1907}{14.12.1969}
  \item \jhperson{\jhname[Anders Gunnar]{Orrholm, Anders Gunnar}}{16.08.1909}{23.01.1947}
  \item \jhperson{\jhname[Erik Edvin]{Orrholm, Erik Edvin}}{11.01.1913}{05.06.1914}
\end{jhchildren}
Jakob Orrholm dog 05.04.1921 och Sanna dog 22.03.1953, nära 83 år gammal. Genom ett domstolsbeslut hade Sanna efter Jakobs död ensam bestämmanderätt över fastigheten.

Dottern Alina gifte sig 1914 med Isak Wilhelm Thomson, \textborn 10.03.1887 på Lussi i Jeppo. Alinas dotter Signe övertog hemmanet vid Lussi efter föräldrarnas död.

Haniel Orrholm, vigd i Oravais 27.2.1929 med Anna Sofia Nygren, \textborn 12.08.1905 i Oravais. Efter giftermålet bodde Haniel och Anna i Oravais där deras fyra barn är födda. När familjen kom till Jeppo bodde de några år i Haniels hemgård. Haniel och Anna byggde en gård åt sig och familjen nära järnvägens vattentorn på lägenhet 3:34 av Fors skattehemman, karta 6, nr 88, som ger mera information om familjen. Anna dog 10.11.1959, två år efter sin man Haniel, som dog 27.05.1957. I slutet av 1950-talet hade alla barnen emigrerat till Sverige.

Jakob och Sannas dotter Ellen Susanna förblev ogift och dog 14.12,1969 i Jeppo. Ellen hade arbetat som modist i Gamlakarleby och kom tillbaka till Jeppo 09.08 1952.

Gunnar Orrholm gifte sig 1939 med Heldine Jungerstam från Jungar. År 1935--\allowbreak 1940 fick han arbete som postbärare i Jeppo. Gunnar hade svaga lungor och dog ung år 1947.
\begin{jhchildren}
  \item \jhperson{\jhname[Betty Marie]{Orrholm, Betty Marie}}{15.06.1941}{}
  \item \jhperson{\jhname[Erik Edvin]{Orrholm, Erik Edvin}}{11.01.1943}{05.06.1944}
\end{jhchildren}
Heldine arbetade som skolköksa på Jungar skola. Betty flyttade med sin mamma Heldine till Sverige.


%%%
% [subsection] Intervju med Sanna Orrholm
%
\jhsubsection{Intervju med Sanna Orrholm}

År 1948 intervjuade grannsonen Erik Stenwall änkan Sanna Orrholm, då 78 år, kanske ett arbete för lärarstudierna. Nedan återges Sannas berättelse ordagrant. Hon har med egen namnteckning godkänt den.

\jhsubsubsection{Sanna Orrholm, Silfvast, 78 år  ---   Bott 55 år i Silfvast.}

Gården, som hon nu bebor, byggdes för 56 år sedan (alltså 1892). Inga träd var då planterade vid stationen. Sen de flyttat hit köpte de jord på Källmossan och gräfta där upp 3 tegar och tog hem en björk varje kväll. Björkarna äro nu på gårdsplanen här.

Stationen som nu, fast inspektorsbostaden är iskarvad, liksom telegrafistgården, som då ej fanns. 2 passagerar- och 2 godståg gick den tiden.

Jakobssons gård fanns ej, en äng fanns där, som Ida Lindström sålde åt Jakobsson för 5000:-kapplandi. Banken kom senare liksom Vikmans butiken. Sundqvists gård fanns då de flyttade hit. Måla – Pitters gård stod vid Koukkuluomas.

Skolan fanns. Backlund hade en liten gård nära den nuvarande. Alla andra gårdar ha tillkommit senare. En gård fanns där Samuel Roos gård finns, där bodde Anders Romars föräldrar. Julin byggde ``Sundells'', där han handla. Gustav Julin är morbror åt Sanna Orrholm. Byggde först bageriet plus en liten butik.

Lindström byggde för ung. 25 år sedan. Kajsas gård fanns här då (för 55 år tillbaka). Gustavssons Jeppas Edla och Jaska bodde där ung. Gleisner nu bor. Jaska, Jaakko Huhtala, senare gift med Amanda. Gården möjligen den som Huhtala nu bor i.

Lindens gård fanns, Simas gård, Ida Lindströms pappa. 4 – 5 år efter det Sanna gift sig byggdes handelslaget. Där Leander Sandberg nu bor fanns en liten gård, som revs och byggdes större.

Där Emil Fors nu bor sytning, den gården fanns, likaså fanns Göstas gården liksom där Erik Fors bor, ej där Elenius bor.
Broar fanns ej vid Fors eller Kiitola. Rodde med båt över eller gick vid dammen. Kvarnen fanns.

Riar som försvunnit: Vid Amanda Vesterlund; 2 riar,  ``Krymbo-Janne'' = Johan Frilund har ej släktingar i Jeppo, barnen i Amerika. Han hade Vesterlunds hemman. Klockans heiman - söder om Leanders fanns en ria, som Krymbo-Janne ägde. Krymbo Janne bodde där Varma nu är - stor bonde, hem från Vexala, hustrun härifrån, Maja Lisa, hon hade hemmanet. Ida Andersson enda barnet i Pensala.

Interjuvarens svärfar bodde i en gård som var vid ``Ida- Marias''. Gården flyttades till Grötas, där Fagerholms nu bor.

Butiker: Lönnqvist där Brunell är. Backlunds Ida var på den butiken, Janne fria då. ``Bolagshandeln'' kallades nuvarande J-O-H, senare andelshandel. Gustav Julin vid nuvarande Sundells.  ``Gille'', Lantmannagille mitt för Kajsa Fors, där mangeln nu är.

För 5-6 mk klara man en vecka; 0,50:-/kg socker, 1,50:-/kg kaffe, 0,40:-/kg risgryn, 50p eller 70p bunten tändstickor. Karameller fanns nog. Vetebröd såldes av Sanna-Kajsa från Stenbacka, 5p/örfil, 10p/grisar. Hon gick varje dag hit och sålde. Med färja for man över till kyrkan.

Då hon var 13 år byggdes den nuvarande stationsbyggnaden, 65 år sedan. Tidningar hade man ej denna tid, Amerikabrev den enda post som kom.

Kallt var det förr i världen! Termometer fanns ej allmänt. Hon förundrar sig över att vintrarna nu äro så milda.

Brännvin köpte man från Nykarleby; vanligt att slåss på bröllop. De hurtigaste karlarna voro föreståndare vid bröllop. Kampas Janne blev ihjälslagen vid ett bröllop vid Lavast. En ``kampare'' slog honom. ``Fello Ant''  bodde vid Fellon, en elak slagsbulte. Skomakar-Janne var elak, då han drack, körde ut både hustru och barn om nätterna.

Sanna Orrholm



%%%
% [house] Hemåker
%
\jhhouse{Hemåker}{4:32}{Silvast}{5}{75-76/375-376}

Styckad av stomlägenhet Bäckstrand 4:6

\jhhousepic{066-05607.jpg}{Gökbrinken 2, hus nr 76}

Lägenheten omfattar även \jhbold{Brunell} 4:144 av Silvast/stoml. Fors 4:11 samt \jhbold{Modén} 6:22 av Måtar/Samfällighet


%%%
% [occupant] Bostads Ab
%
\jhoccupant{Bostads Ab}{Gökbrinken}{}
Den 5 juni 1968 köpte byggmästare Tor-Erik Ågren av Eliel och Ines Brunell ovannämnda fastigheter om c:a 5500 kvadratmeter för ett bostadsaktiebolag under bildning. De gamla byggnaderna på tomterna revs.

Bostadsbolaget Bostads Ab Gökbrinken-Asunto Oy infördes i Patent- och handelsregistret 23.10.1970. Slutsynegranskning på radhusens nio bostadslägenheter, pann- och förrådsrum samt bastu/tvättrum utfördes den 28.04.1970. Några biltak och cykeltak byggdes sommaren 1970. Biltaken och cykelrummet byggdes om och förstorades 1992. År 2004 byggdes sadeltak med brandväggar mellan lokalerna, och 2007 revs den gamla brännoljepannan ut och värmepump för bergsvärme installerades och tre bergvärmehål borrades. År 2012 grundrenoverades gemensamma bastu- och tvättrum samt förrådsutrymmen efter ett vattenläckage i duschrummet.

\jhhousepic{067-05566.jpg}{Gökbrinken 1, hus nr 75}

Den 24.01.2014 konstaterades i lokal 2 ett vattenläckage, som hade spridit sig till lokal 1 och 3 samt delvis till lokalerna 4-5. Fullständig grundrenovering med nytt VVS-system och golvvärme samt ny värmepump och elledningar installerades. Ävenså borrades ytterligare två 200 meters bergvärmehål. Lokalerna 6-9 i det andra huset grundrenoverades på samma sätt som hus I.


LOKAL 1 A

\jhbold{Silfvast Robert} 2013-- :
Den 13.04 2013 köpte fordonsmekaniker Robert Bror Johannes Silfvast, \textborn 29.12 1986 på Silvast hemman, lokal 1A. I medlet av juni flyttade Robert och hans sambo Marina Terese Björklund, \textborn 30.07.1991 på Stenbränn i Purmo. Robert arbetar som bergsborrare åt K.S. Geoenergi, Kronoby. Marina är utbildad kock och barnskötare och arbetar som barnskötare åt Nykarleby stad i Jeppo. Som hobby arbetar Robert på fritiden med bilar och maskiner. Marina älskar att baka och är intresserad av heminredning. Våren 2016 åtog sig Robert gårdskarlsarbeten för bostadsbolaget. Den 28 september 2016 fick Robert och Marina dottern Nelly Terese.

\jhbold{Lindén Gurli}	2006--\allowbreak 2013:
Författarna Gurli Lindén, \textborn 18.01.1940 i Öja och Eva-Stina Byggmästar, \textborn 07.06.1967 i Nykarleby, köpte lokal 1 av Berit Lönnvik år 2006. Efter en grundlig renovering av lokalen flyttade de från en hyreslägenhet i Bostads Ab Lövgränd på Åkervägen, som de hyrde år 2004 vid återkomsten från Sverige. Gurli Lindén har utgivit 19 skönlitterära verk och 2 böcker om Hilma af Klint. Eva-Stina Byggmästar har utgivit 14 böcker och varit utsedd som kandidat för bl.a. August-priset och Nordiska rådets litteraturpris och har erhållit flera andra pris, särskilt i Sverige. Sommaren 2013 sålde Gurli Lindén lokal 1 jämte biltak till Robert Silvast. Gurli och Eva-Stina flyttade till  hyreslokaler i Nykarleby centrum. Se mera på fastighet R:nr 3:126, karta 6, nr 94.

\jhbold{Lönnvik Berit} 1989--\allowbreak 2006:
Efter att år 1989 köpt lokal 1 flyttade hjälpsjukskötare Berit Lönnvik, född Finskas, \textborn 06.10.1952 på Lövbacken, till Gökbrinken med sin dotter Belinda, \textborn 1982 på Kaup hemman. Berit arbetade tidvis inom  åldringsvården i Vasa och 1989--\allowbreak 1991 som t.f. diakonissa i Jeppo församling. Berit och Belinda flyttade år 1992 tillbaka till hemmet på Kaup.

\jhbold{Hyresgäster 1992--\allowbreak 2006}:

2004--\allowbreak 2006: Anders Sjö, född 1977 i Esse, och Linda Hortáns, född 1979 från Sundom, flyttade i september 2004 in i lokal 1 med sin son Jasper född 20.06.04. Anders Sjö arbetade på Oy Ora-Pack Ab vid Oravaisfabrik. Anders och Linda	gifte sig 16.07.2005. På grund av ett vattenläckage i lägenheten, förorsakad av en diskmaskin, var de tvungna	att flytta ut i september 2006. Familjen Sjö hyrde Bo	Strengells hus på Fors, nr 99 på kartan. Sonen Viggo föddes 17.04.2007. Skilsmässa år 2010.

2003--\allowbreak 2004: Helge och Lillemor Sandström från Nykarleby hyrde lokal 1 ca 1 år. De flyttade till centrum i Nykarleby.

1996--\allowbreak 2003: Lokal 1 hyrde pensionärerna Erik Nybyggar, född 1914 på Nybyggar hemman i Lassila by, och hans hustru Anna, född 1923 i Forsby, Nykarleby. De flyttade från sitt hus på Heikfolk i Lassila by, där de bedrivit sitt jordbruk m.m.  Erik dog den 05.12.2003. Anna flyttade i januari 2004 till den mindre lokalen 5 E på Gökbrinken.

1994--\allowbreak 1996: Camilla Lavast, född Back 24.9.1970, i Överjeppo by, och dottern Ida, född 15.03.1993 på Lavast hemman hyrde lokal 1 under ca 3 år. Hunden Molgan ingick också i familjen. Camilla och Ida flyttade 1996 till Bostads Ab Älvliden på Grötas,

1992--\allowbreak 1994: Lokal 1 hyrdes en kortare tid av en familj, som hade en pälsdjursfarm i Härmä.


\jhbold{KWH-koncernen Ab} 1973--\allowbreak 1989:
Under tiden KWH-koncernen Ab (Oy Keppo Ab) ägde lokal 1, bodde under olika perioder familjer här, som arbetade på Mirka slippappersfabrik.

1984--\allowbreak 1989: I november 1984 hyrde diplomingenjör Ronny Klemets, född 01.10.1957, i Ytteresse, och hans hustru Anna-Lisa, född Roos 30.09.1958 i Pensala. Ronny arbetade som produktutvecklare på Mirka och Anna-Lisa var textillärare på hantverkarlinjen vid Kronoby folkhögskola. I mars 1989 köpte de Kiitola herrgård och flyttade dit. Under tiden på Gökbrinken föddes dottern Alexandra 09.09.1988, student från Topeliusgymnasiet och har studerat sång vid Musikinstitutet och blev barnträdgårdslärare 2017 i Jakobstad. Familjen utökades med sonen Daniel, född 18.06.1990 och dottern Rebecka, född  07.03.1997.

1981--\allowbreak 1984: I maj 1981 flyttade Holger och Birgit Nyman med sina 2 döttrar till lokal 1 på Gökbrinken från Oravaisfabrik. Holger Nyman, född 1928 i Helsingfors, var arbetsledare på bandsidan på Mirka i Jeppo och hans fru Birgit, född 1931 på Kimito, hade tjänst på Oravais Klädesfabriks tekniska kontor. Deras flickor Barbro född 1962, och Birgitta 1965 i Oravais, blev student från Vörå 1981 respektive Nykarleby 1984. Barbro, gift Kaivola bor i Vasa och Birgitta, gift Fridlund, bor i Pensala, Nykarleby. Holger hade tidigare arbetat som vävmästare 1961--\allowbreak 1980 på KWH:s klädesfabrik i Oravais. På hösten 1984 flyttade Holger och Birgit till centrum i Nykarleby.

1978--\allowbreak 1981: Hösten 1978 till vårvintern 1981 bodde två inflyttade finskatalande familjer efter varandra under kortare tider i lokal 1.

1974--\allowbreak 1978: Merkonom Kurt Betlehem, född 1945 i Gamlakarleby och uppvuxen på Kiitola i Jeppo och hustrun, fysioterapeut Anna-Maria, född 1949 i Kvevlax, jämte sonen Ulf Andreas flyttade hösten 1974 till Gökbrinken från en lärarbostad på Jeppo skola. Kurt var då försäljningsdirektör på Mirka och Anna-Maria arbetade som fysioterapeut på HVC Nykarleby. Hon planerade och öppnade en tillfällig terapiavdelning i Hvc:s regi i Jeppo kommunalgård 1976. Avdelningen flyttade 1978 till Hvc:s förstorade utrymmen i Nykarleby. Andra sonen Jan Matias föddes 1977 under tiden de bodde på Gökbrinken.  Familjen flyttade 01.05.1978 till eget hus i Nykarleby. Kurt blev VD för KWH-Mirka Ab och chef för Mirka-gruppen och medlem i KWH-koncernledningen. I dag är Kurt pensionär och har varit medlem under flera år i styrelsen för KWH-koncernen Ab. Sonen Ulf, dipl.ing., elektronik, arbetar som utvecklingsingenjör på Teleste Oy, Åbo. Sonen Jan, ekonomie stud., arbetar på KWH-Mirka och bor i Jakobstad.

1973--\allowbreak 1974: Exportchef på Mirka Ole Cajander med fru Harriet och 2 barn, Jan och Maria Christina bodde ca 2 år i lokal 1 före familjen Betlehem. Familjen Cajander kom från Esbo, dit de återvände.


LOKAL 2 B

\jhbold{Finskas Birgitta o. Berit} 1980-- :
Vid generationsskifte erhöll genom köp systrarna Birgitta och Berit Finskas lokal 2.  Föräldrarna Yngve Finskas, född 04.02.1920 på Finskas hemman, och Torhild, född Lindén 23.12.1922 på Silvast, flyttade år 1980 som pensionärer in i lokal 2. De hade 1946 byggt sitt hem på Lövbacken, när de övertog skattelägenhet Lindström 4:118 av Silvast hemman, som Torhilds far Anders Kaukos Lindén inköpt av sin kusin. Yngve Finskas dog plötsligt i augusti 2000 och Torhild levde ensam kvar i lokalen till de två sista åren, då hon en stor del av tiden bodde hos sin dotter Berit på Kaup. Vattenläckage inträffade 24.01.2014 i sängkammaren i lokal 2. Torhild dog 11.02.2015. Mera uppgifter om Yngve och Torhild finns under nr 123, karta 9.

\jhbold{Hyresgäster}:

2016-: Den 01.09.20160 hyrde Robin Rönnqvist fr. Larsmo lokal 2.

2015--\allowbreak 2016: Under ca ett år hyrde Linda Kulla, född Nyman från Pensala, och hennes två yngre barn, lokalen från juli 2015 till maj 2016.

\jhbold{Nyman John o. Gunilla} 1970--\allowbreak 1980:
När Gökbrinkens två bostadsbyggnader var inflyttningsklara 1970 köpte John och Gunilla Nyman lokal 2. John Nyman är född 25.11.1945 i Esse och Gunilla, född Thel, 30.05.1947 i Jakobstad. John arbetade i affären FYRENS Snabbköp fr.o.m. starten 1967 i Andelsbankens nybyggda hus. År 1971 övertog eller köpte han affären Fyren och utvidgade affärrörelsen. John och Gunilla bodde i en etta i samma hus i flera år och flyttade till bankdirektörsbostaden 1969. De fick 2 flickor, Ulrica, född 14.02.1969 i Nykarleby, där de tillfälligt bodde några månader, och Camilla, född 26.05.1972, i Jeppo.  Ulrica har studerat språk, engelska och franska, samt arbetar nu som Verksamhetsledare i Förbundet för Arbets- och Medborgarinstituten i svensk Finland.  Camilla är modermålslärare i Carleborgsskolan i Nykarleby. Familjen flyttade 1980 till sitt närbelägna nybyggda hus på Centralvägen, nr 64 på karta 4.


LOKAL 3 C

\jhbold{Sandström Paul och Carita} 1986--\allowbreak 2009:
År 1986 köpte trafikant Paul Sandström, född 24.06.1926 på Fors, och hans hustru Carita, född Sandberg 23.03.1933 på Skog hemman, lokal 3. De hyrde ut lokalen 1986--\allowbreak 1994.

Paul och Carita flyttade från sitt egnahemshus, Svinvallen I, på Fors hemman/samfällighet, till Gökbrinken hösten 1994, då sonen John bildade familj. Paul var under många år styrelseordförande i bostadsbolaget. Som pensionärer deltog de flitigt i pensionärsföreningens olika verksamheter och i andra sociala och kyrkliga verksamheter. Paul dog 19.09.2007 och Carita dog 13.01.2009. Deras verksamma liv skildras mera under fastighet Svinvallen I, R:nr 20, karta 5-6, nr 82.

\jhbold{Sandströms dödsbo} 2009-

\jhbold{Hyresgäster}:

När grundrenoveringen efter vattenskadan 24.01.2014 var klar i hus 1, flyttade 01.10.2014, enligt överenskommelse, Lea Stenvall från lokal 9 i hus 2. Den 31.03.2015 var renoveringen av hus 2 klar och Lea flyttade tillbaka till sin lokal.

2016--: Den 01.04.2016 hyrde Aune Mirjam Kaarina Malinen, född 05.09.1932 i Sordavala, lokal 3. Hon och hennes sambo Rainer Vesterinen och en son kom 1995 till Jeppo, Måtar, Keppovägen 240. De hade köpt fastigheten av Törnqvists dödsbo. Aune Malinen och Rainer Vesterinen har fyra fullvuxna pojkar och en flicka. Rainer Vesterinen dog 2001.

2011--\allowbreak 2014: I mars 2011 flyttade Paul och Carita Sandströms dotterdotter, barnträdgårdslärare Susanna Hägg, född 1986 i Komossa, Oravais, in i lokal 3. I augusti 2011 blev hon sambo med pojkvännen Janne Käld från Jakobstad. Den 25.05 2012 föddes deras dotter Magda på Centralsjukhuset i Vasa. Janne har arbetat på Wärtsilä i Vasa och är utbildad maskiningenjör. I juli 2013 fick Janne anställning på Jeppo Biogas Ab. Susanna och Janne vigdes den 03.08.2013 av prosten Åke Lillas på sommarstugan i Larsmo. Janne har under flera år renoverat sin farfars hus på Skatan i Jakobstad och på grund av vattenskadan på Gökbrinken var de tvungna att hastigt 01.02.2014 flytta till Jakobstad.

2009--\allowbreak 2010:	Ca 1 år hyrde lastbilschaufför Björn Lundkvist, född 1964 på ``Skatabackan'', Back, tillfälligt lokal 3. Även hans barn Fred, född 18.01.1996 och Lisette, född 27.01.2000, på Back hemman, bodde varannan vecka hos pappan på Gökbrinken.

\jhbold{1994--\allowbreak 2009}:	Paul och Carita Sandström bodde som pensionärer i sin lokal 3, se ovan.

1992--\allowbreak 1994: Änkefru Greta Eklöv, född Finnilä 1920 i Ytterjeppo, hade 1992 sålt sitt egnahemshus på Nylandsvägen och hyrde då lokal 3 på Gökbrinken. Greta flyttade hösten 1994 till Älvliden, lokal 6 på Älvvägen 10 på Grötas hemman. Läs mera om familjen Eklöv under fastighet, Metsäpirtti R:nr 2:35, karta 3, nr 34.

1989--\allowbreak 1992: Jorma Karkaus född i Ekola, Härmä, och hans hustru Agneta från Munsala, hyrde lokalen i ca 3 år.

1986--\allowbreak 1989: I över 3 år, augusti 1986 – oktober 1989 hyrde Barbro Eklund, född 1963 på Stenbacka, lokal 3. Barbro arbetade som bokförare på Jeppo Kraft Andelslag och Nykarleby Bokföringsbyrå. Hon flyttade till Mjölnars och gifte sig med arbetsledare på Jeppo Potatis Ab, Karl-Gustav Julin.

Hemvårdare Bodil Nyström har också en kortare tid bott i lokalen.

\jhbold{Elenius Uno} 1975--\allowbreak 1986:
Vid generationsskifte 1975 flyttade familjen Bengt E. till Gunnar och Uno med hustru Aune, född Sipponen 1916 i Pelkkala, flyttade då till Gökbrinken. Uno övertog arbetet som gårdskarl och Aune arbetade på Mirka. År 1986 sålde Uno Elenius lokal 3 och flyttade till före detta ``Midinette'', på Fors, hus nr 98 på karta 6.

\jhbold{Elenius Bengt} 1970--\allowbreak 1975:
År 1970 köpte Bengt Elenius lokal 3 och flyttade med familjen från det s.k. ``Hästbacka-huset'' i stationsparken.  Bengt Elenius är född 1948 på Gunnar hemman och hans dåvarande hustru Sirkka-Liisa, född Ievanen 24.07.1947 i Kajana, samt den äldre sonen Kim född 09.05.1968 i Jakobstad. Den yngre sonen Nicklas föddes 26.4.1973 då de bodde på Gökbrinken. Bengt arbetade på Fyrens snabbköp samt som gårdskarl för bostadsbolaget.


LOKAL 4

\jhbold{Perälä Marjatta} 2013-- :
Den 28 augusti 2013 köpte Marjatta Anneli Perälä, född 13.07.1953 i Kauhajoki, lokal 4.	 Bostaden renoverade hennes man Sauli Perälä, som lade parkettgolv, nytt innertak, ny köksinredning, garderober och tapeter. En vecka efter att renoveringen var klar konstaterades vattenläckaget den 24.01.2015 och i slutet av februari var Marjatta tvungen att tömma lägenheten och flytta till Oravais. Lokal 4 och 5 blev färdig renoverade 01.06.2016, då Marjatta kunde flytta tillbaka. Marjatta har en dotter, Noora Luoma-Stenvall. F.o.m.1999 har Marjatta drivit Funkis Pub och Café-verksamhet med matservering. Den 01.07.2016 gick Marjatta i pension och sålde verksamheten till Jasmin Aalto.

\jhbold{Eklöv Ragnar o. Siv} 2002--\allowbreak 2013:
Pensionärerna Ragnar Eklöv, född 10.07.1927 på Gunnar hemman, och hustrun Siv, född Julin 09.10.1937 i Nykarleby, Forsby, köpte år 2002 lokal 4 och flyttade från sitt jordbrukarhem på Gunnar. Ragnar dog hastigt 31.12.2003 och Siv dog på Vasa Centralsjukhus den 13.03.2013. Se mera om deras liv på Gunnar hemman, gård 25, 125.

\jhbold{Jungar Doris, Ninni och Thea} 1970--\allowbreak 2002:
Som pensionärer flyttade Lennart Jungar, född 26.07.1900 på Jungar hemman, och hustrun Ester, född Forss 10.08.1907 på Fors hemman, till Gökbrinken. Vid generationsskifte erhöll genom köp döttrarna Doris Eriksson, Ninni Back och Thea Jungar lokal 4. Lennart dog den 11.05.1983 och Ester dog den 15.03.2001. Mera om deras liv finns på skattelägenhet Markus av Jungar hemman.


LOKAL 5 E

\jhbold{Sandströms dödsbo} 2009-- :
\jhbold{Sandström Paul o. Carita} 2002--\allowbreak 2009
Efter Ellen Hedmans sjukdomstid och död stod lokal 5 tom under många år. Paul och Carita Sandström köpte lokalen år 2002 och hyrde ut den till Anna Nybyggar, som flyttade från lokal 1 till lokal 5 i januari 2004.  I april 2012 var Anna p.g.a. sämre hälsa tvungen att flytta till De gamlas hem, Hagalund, i Nykarleby centrum. Anna Nybyggar somnade in i sin säng på natten den 14.04.2013.


Övriga \jhbold{hyresgäster} 2013-- :
I november 2015 hyrde Henrik Kaj Johan Löfroth, född 14.12.1982 i Kronoby, lokal 5. Löfroth arbetar på Oy KWH Mirka Ab.

Den 01.10.2013 hyrde sjukpensionär Anders Nylund, född 1968 i Kronoby lokalen. Den 28.02.2014 var han tvungen att flytta på grund av vattenskadan.

Oy KWH Mirka Ab hyrde lokal 5 under mars-juli 2013 för anställda från utländska dotterbolag.

\jhbold{Dödsbo} 1999--\allowbreak 2002:

\jhbold{Hedman Ellen} 1970--\allowbreak 1999:
Änkefru Ellen Hedman, född Backlund 28.06.1905 på Sparvbacken på Romar hemman, flyttade som änka från Pensala till lokal 5 på Gökbrinken som första invånare. Hon hade arbetat som jordbrukare med sin man Alfred på Köurus i Pensala. Hon skrev dikter och hade en förmåga att kunna citera dem, sånger ur Sionsharpan och psalmer utantill. Ellen flyttade till Hagalund 1987 och låg ca 10 år på Nykarleby sjukhus. Ellen dog den 20.08.1999 och är begraven i Munsala. Ellen Hedmans dödsbo sålde lokalen 2002.


LOKAL 6 A

\jhbold{Almberg Gunnar} 1973-- :
År 1973 köpte driftschef Gunnar Almberg och hans hustru butiksbiträde Marianne, lokal 6 i hus 2. Gunnar Almberg är född 04.04.1924 på Ojala i Lassila by och Marianne, född Westerlund 17.02.1930 i Markby.  År 1973 arbetade Gunnar på Mirka och Marianne på Jeppo-Oravais Handelslag. Under 1955--\allowbreak 1967 var de anställda av Andelslaget Varma i Oravais och Jeppo. Gunnar hade ett stort intresse för jakt och fiske, samt hundrasen stövare.
Deras tidigare bostad fanns även på Stationsvägen i det s.k. ``Jakobsonska huset''.  Se mera under fastighet 4:28, nr 370 på karta 5.

Gunnar Almberg \textdied 29.05.1915


LOKAL 7 B

\jhbold{Julin F. dödsbo} 2016-- :

\jhbold{Hyresgäster}:
Den 01.08.2016 hyrde Robin Eric Nygård, född 28.07.1995 på Gränden i Lassila by, och hans sambo	Nina Maria Mikkola, född 19.11.1996 i Österhankmo, Kvevlax, lokal 7. Robin är utbildad Skogsmaskinförare från Yrkesakademin i Vasa och arbetar på KS Geoenergi.	Nina har också studerat på Yrkesakademin och har examen i Inredning och tapetsering. I dag 2017 arbetar Nina som båtinredare på Essma i Esse.

\jhbold{Julin Felix} 1980--\allowbreak 2014:
Jordbrukare Felix Julin, född 08.09.1922 på Mjölnars, köpte lokal 7, år 1980 på Edwin och Selma Wickmans dödsbos auktion. År 1988 flyttade pensionärerna Felix och hans hustru Margit, född Hägglund 18.05.1929 på Tånabacken i Överjeppo, till Gökbrinken. Felix och Margit hade varit idoga jordbrukare, och med de första, som började med potatisodling i större skala. De har 5 söner med familjer. Margit älskade att väva och att pyssla i trädgården, där hon fick det att blomma. Felix tyckte om att arbeta i skogen och ännu under de sista åren röjde han sly och hjälpte till vid potatisupptagning.

Margit \textdied 01.08.2011  ---  Felix \textdied 11.09.2014, 92 år gammal.


1987--\allowbreak 1988:	Under tiden som Felix och Margits son Nils Julin och hans hustru Sonja renoverade sitt hus på Mjölnars bodde de i Nils föräldrars lokal 7 på Gökbrinken.

1980--\allowbreak 1986:	Bertel Julin, född 1944 på Holmen, och hans hustru Marja-Liisa ,född 1944, samt sonen Peter, född 28.11.1974 i Jakobstad, hyrde lokal 7 av Felix Julin. Bertel arbetade då som teknisk ledare på Ab Jakobe Oy i Jakobstad och blev senare VD för firman. De hade en pälsfarm i Jeppo, som Marja-Liisa skötte om. Deras dotter Petrina föddes 05.06.1980 när de hade flyttat till Gökbrinken. År 1986 flyttade familjen till sitt nybyggda hus på Nylandsvägen, nr 41 på karta 3.

\jhbold{Wickman Edvin} 1970--\allowbreak 1980:
År 1970 köpte Edvin Wickman, född 07.10.1887 i Munsala, och han hustru Selma, född Sjöblom 09.01.1888 på Silvast. De hade emigrerat som unga till USA och återvände som pensionärer till fädernebygden. Selma Wickman dog den 27.09.1978 och Edvin Wickman 21.05.1980.


LOKAL 8 C

\jhbold{Sandströms Db} 2007-- :
\jhbold{Sandström Paul o. Carita} 2004--\allowbreak 2007
År 2004 köpte Paul och Carita Sandström lokal 8.

\jhbold{Hyresgäster}	2015-- :
Den 01.10.2015 hyrde Amanda Ida Helena Cederström, född 28.05.1997 i Bennäs, Pedersöre och hennes sambo Tony Carl-Erik Björklund, född 11.12.1994 i Kronoby, lokal 8. Amanda och Tony är förlovade. Tony blev landsbygdsföretagare från Optima, Lannäslund 2014, därefter i militären som civiltjänstgörare. Amanda fick båtbyggarexamen från Optima den 03.06.2016. Tony arbetar helst med grävmaskiner. Båda tycker om djur och att vistas ute i naturen. I dag 2017 arbetar Tony på Oy Jakobe Ab och Amanda som personlig assistent i Oravais.

Mathias Asplund från Nykarleby hyrde lokalen 2004--\allowbreak 2014. Mathias föräldrar bodde i samma lokal när Mathias föddes den 08.03.1977. Mathias har studerat på Vasa tekniska högskola och arbetar nu på Oy KWH-Mirka Ab, Jakobstad. På grund av den förestående grundrenoveringen av lokalerna köpte Mathias ett hus i Jakobstad och flyttade dit 31.08.2014.

\jhbold{Sandlin Stefan} 1970-- :
När lokalerna blev inflyttningsklara 1970 köpte Stefan Sandlin, född på Böös hemman, lokal 8.

1970--\allowbreak 2004:	Stefan Sandlins föräldrar Emil Sandlin, född 05.05.1914 på Grötas hemman och hans hustru Hjördis, född  Källman 1915 på Ruotsala, flyttade 1990 från sin jordbrukslägenhet på Böös till sonens lokal 8 på Gökbrinken.
Hjördis \textdied 2001  ---  Emil \textdied 07.02.2004.

\jhbold{Hyresgäster} 1970--\allowbreak 1990:

1983--\allowbreak 1990: År 1983 flyttade Maija Stenbackas mamma, bokförare Aino Savolainen, född 06.08.1915 i Uleåborg, från Oulainen till Jeppo och hyrde lokal 8 på Gökbrinken. Aino Savolainen dog 1990.

1978--\allowbreak 1982: Pensionär Berta Lundkvist, född Kunnari 1912, hyrde under ca 5 år lokal 8. Berta arbetade inom industrin och före sin pensionering på Mirka slippappersfabrik. Hon flyttade till Bostads Ab Älvliden 1982. Berta dog 1993.

1976--\allowbreak 1977: Elmontör Bengt Asplund, född 16.03.1954 på Holmen i Jeppo, och hans hustru Lilian, född Björklund 16.02.1956 i Kronoby, hyrde lokal 8 1976. Deras son Mathias föddes 08.03 1977 under tiden de bodde i lokalen. De köpte en lokal i ett radhus i Nykarleby och flyttade dit på hösten 1977.

1972--\allowbreak 1975: Lastbilschaufför Terho Ekola, född 26.08.1943 i Ekola, Voltti, och hans hustru Gun-Lis, född Jakobsson 19.04.1947 i Tålamods, Vörå, samt deras döttrar Rita Anne Helena, född 16.01.1967 i Vörå och Margita Pia Louise, född 17.01.1970 i Jeppo, flyttade 1972 till lokal 8 från ``Kuoppala-huset'' på andra sidan vägen. År 1975 flyttade de till HAB-bankhuset, Rnr 4:43, nr 70, karta 5.

1970--\allowbreak 1972: Mikko Erkinheimo och hans hustru Marja-Liisa, född Kennola i Jeppo, hyrde lokal 8, när den blev inflyttningsklar 1970. De har numera en större järnaffär i Härmä och Lappo.


LOKAL 9 D

\jhbold{Stenvall Lea} 2000-- :
I augusti 2000 köpte Lea Elisabet Stenvall, född Sandqvist 28.06.1934 på Silvast, lokal 9. Lokalen renoverades och i slutet på oktober flyttade hon från sitt hem på granntomten, där hon bott i 45 år. Se Nygård Rnr 4:208, nr 83, karta 5, där mera uppgifter finns. Lea har varit i bostadsbolagets styrelse från 2001 och är i dag styrelseordförande.
		.
\jhbold{Westerlund Ester} 1970--\allowbreak 1976-2000 db:
Pensionär Ester Westerlund från Markby i Nykarleby lkm, född 1899 i Purmo, köpte lokal 9 år 1970, när den blev inflyttningsklar. Ester Westerlund hade tillsammans med sin man jordbruk och snickeri i Markby. Ester är Marianne Almbergs mor.

Ester Westerlund dog 11.12.1976. Sonen Birger och hans hustru Ingrid, bosatta i Stockholm, använde efter moderns död lokalen som semesterbostad. Birger Westerlund dog 1995 i Stockholm.


\jhhousepic{Brunell.jpg}{Brunells hus på 1960-talet, just före rivning. Nr 375}

\jhbold{BRUNELL} 1933--\allowbreak 1969; nr 375

Den 27.12.1932 köpte Eliel och Ines Brunell av Aina Modén, Hemåker R:nr 4:32 av Bäcksrand skattelägenhet R:nr 4:6 på Silvast hemman samt Modén R:nr 6:22 av Måtar hemmans samfällighet (nr 375). Lagfart erhölls 30.09.1933. Den 24.04.1951 köpte Eliel Brunell tomten mot Stationsvägen (nr 376) av Toini Maria Leino. Lagfart erhölls 23.08.1954 på Brunell R:nr 4:144.

Eliel Brunell, född 11.11.1904 i Nykarleby lkm., och hans hustru Ines, född Granskog 19.12.1909 i Åvist, bodde och hade butiksverksamhet i husbyggnaden och i magasiner på tomten till 1953, då Eliel blev föreståndare på Mellersta Österbottens Andelskassa. Se figur 96 på Fors hemman. Ines och Eliel Brunell fick 3 söner.

\jhpic{Brunell-etikett.png}{Marknadsföring i tiden}{0.5}

\begin{jhchildren}
  \item \jhperson{\jhname[Göran]{Brunell, Göran}}{04.02.1936}{}
  \item \jhperson{\jhname[Rolf-Erik]{Brunell, Rolf-Erik}}{07.03.1940}{}
  \item \jhperson{\jhname[Viking]{Brunell, Viking}}{20.12.1941}{}
\end{jhchildren}

Göran studerade till vägbyggmästare. Han gifte sig 21.06.1958 med Lisa Birgitta, född Antfolk 07.10.1935 i Esse. De flyttade till Jeppo 30.11.1960 och flyttade 12.06.1962 till Nykarleby. Viking har undervisat och forskat vid Jyväskylä Universitet. År 1967 flyttade Eliel och Ines till direktörsbostaden i Andelsbankens nya bankhus samt vid pensioneringen 1969 till Nykarleby.

Eliel \textdied 12.03.1973  --- Ines \textdied 08.05.2010 och blev således över 100 år.

\jhbold{Hyresgäst}

Lönnqvist Maria Sofia, född Bertula 21.01.1871 i Munsala, handlade 1931--\allowbreak 1932 i butiken före Brunells köpte fastigheten av Modéns. Butiksbiträde var Ester Backlund. Maria Sofia Lönnqvist var gift med företagaren Elias Lönnqvist på Back. Familjen bodde och fick en dotter Verna, född 1912 på Back. Magister Vernas första man, Stenholm, stupade i kriget. Verna gifte sig senare med dir. Sven Häll på Jeppo-Oravais Handelslag. Mera om familjen under R:nr 4:87, nr 57, karta 5. Maria Sofia Lönnqvist dog 19.01.1961.


\jhhousepic{Brunell vy.jpg}{Brunells hus och uthus samt i förgrunden Ida Lindströms uthus. Ungdomslokalen skymtar i bakgrunden. Vy från Ester Wadströms fönster.}

\jhbold{MODÈN Johan} 1895--\allowbreak 1932

Lanthandlare \jhname[Johan Modén]{Modén, Johan \& Sofia}, \textborn 15.03.1861 i Jeppo, och hans hustru Sofia, född Jakobsdotter, \textborn 12.08.1854 i Munsala, köpte 05.10.1895 skattelägenhet Bäckstrand R:nr 4:6 om 0,0699 mantal av Silvast skattehemman R:nr 4, lagfart 15.03.1900. Säljare Isak och Lovisa Wiklund. En tomt utstyckades från stomlägenheten och infördes i jordregistret år 1915 som Hemåker 4:32. På tomten fanns en byggnad med butik och bostad samt en uthusbyggnad.

Den 13.02.1920 sålde Modén skattelägenhet Bäckstrand R:nr 4:33 om 0,0495 mantal av Silvast hemman. Han hade sålt tomter och undantagit några parceller. Hustrun Sofia dog 04.12.1923. Den 26.07.1924 gifte sig Johan med frånskilda Edla Maria, född Henriksson, \textborn 19.09.1878 i Pernå.
Barn med Sofia:
Selma Emilia, \textborn 14.09.1892, fick en son Johannes Alarik, \textborn 26.07.1922, flyttade till Jakobstad 1933.
Aina Alice, \textborn 16.02.1895, var bankkassör. Aina flyttade till Helsingfors 23.11.1939.

Den 12.11.1925 köpte Anders Modén av Måtar R:nr 6 samfällighet i Överjeppo ett område i Silvast, numera Modén R:nr 6:22. På området uppfördes en större magasinbyggnad, som tidvis tjänade som affärsutrymmen för bl.a. möbelförsäljning och som äggcentral. Handlare Modén var aktiv inom kommunalförvaltningen och i början av 1900-talet hölls kommunalnämnden hemma hos Modéns. Modén flyttade till Gamlakarleby den 20.01.1931.


\jhbold{FORS Erik} 1885--\allowbreak 1951; nr 376

Bonden Karl Johan Abrahamsson Fors hade 1885 förvärvat 0,0166 mantal av Silvast skattehemman (se nr 376).
Sonen \jhname[Erik Fors (Stöipas-Erik)]{Fors, ``Stöipas-Erik''}, född 19.02.1863 på Fors hemman, köpte 13.02.1899 av sin far 0,0166 mantal av Silvast hemman. Eriks syster, Sanna-Lisa, gift Rijf, hade fått överta hemmanet på Fors (se karta 6, nr 97/397). Vid lantmäteriförrättning 1915 registrerades stomlägenhet Fors 4:11. Erik Fors och hans hustru Kajsa, född Andersdotter, \textborn 07.08.1857, bodde med barnen i stugan, som troligen byggdes av den tidigare ägaren på 1880-talet på området, som idag utgör Brunell 4:144.
\begin{jhchildren}
  \item \jhperson{\jhname[Johannes]{Fors, Johannes}}{27.02.1891}{01.09.1891}
  \item \jhperson{\jhname[Ida Katarina]{Fors, Ida Katarina}}{21.03.1892}{28.01.1937}, till P.söre 10.09.16, gift Westermark
  \item \jhperson{\jhname[Anna Irene]{Fors, Anna Irene}}{24.11.1894}{14.01.1921}, till J.stad 10.12.11, g. i Jeppo 01.05.13
  \item \jhperson{\jhname[Toini Maria]{Fors, Toini Maria}}{13.05.1898}{19.01.1995}, g. Andersson/Leino, till Seinäjoki
\end{jhchildren}
Irene gifte sig 29.07.1917 med Ture Westerlund på Fors hemman, se nr 394. Jfr även Fors, karta 6, nr 397 \& 397a.

Erik \textdied 24.03.1933  ---  Kajsa \textdied 02.07.1947

Dottern Toini, gift Leino och bosatt i Seinäjoki, sålde tomten 1951 till Eliel Brunell varefter gården och uthuset revs.


\jhbold{Viklund} Abrahamsson Isak, \textborn 12.10.1852, och hans hustru Lovisa Johansdotter, \textborn 02.08.1850, hade på 1880-talet förvärvat 0.0166 mantal av Silvast hemman, som blev stomlägenhet Bäckstrand 4:6 och Fors 4:11. Troligt är att Isak och Lovisa Viklund byggde båda bostadshusen och uthusen
\begin{jhchildren}
  \item \jhperson{\jhname[Anna-Lovisa]{Abrahamsson, Anna-Lovisa}}{24.02.1885}{}
  \item \jhperson{\jhname[Ida]{Abrahamsson, Ida}}{24.07.1889}{}
  \item \jhperson{\jhname[Sanna Sofia]{Abrahamsson, Sanna Sofia}}{21.05.1894}{18.04.1896}
\end{jhchildren}

Isak Viklund \textdied 14.09.1909  ---  Lovisa \textdied 07.012.1916



%%%
% [house] Bäckliden
%
\jhhouse{Bäckliden}{4:195}{Silvast}{5}{80}

Styckad av stomlägenhet Bäckstrand 4:6

\jhhousepic{034-05568.jpg}{Uf sommaren 2016}

%%%
% [occupant] Jeppo
%
\jhoccupant{Jeppo}{\jhname[Ungdomsförening]{Jeppo, Ungdomsförening}}{1912--}

\jhbold{Uppstarten}
Den 08.01.1894 beslöts att grunda Jeppo Ungdomsförening, som anslöts till den första föreningen, Malax Ungdomsförening. Den 10.04.1894 fastställdes föreningens stadgar. Möten hölls turvis i bondstugor och turvis i Jungar skola på Sparvbacken. Genom att tigga stockar, pengar och dagsverken kunde man 1909 bygga en egen ungdomslokal på Furubacken. Byggnaden invigdes 07.11.1909.

\jhpic{UF-urklipp.png}{Den högtidliga invigningen år 1909 på Furubacken noterades}{0.5}

Knappt två år efter invigningen, tidigt på morgonen den 21.10.1911 stod lokalen i eldslågor och förstördes helt. Ungdomslokalen var försäkrad till 12000 mark.

\jhbold{Fortsättningen}
Stockar, pengar och dagsverken samlades åter in och man beslöt att placera ungdomslokalen i centrum på Silvast. Kommunen deltog vid byggandet av det nya huset. Handelsman Modén, aktiv kommunalman, upplät ett område mellan Silvast-bäcken och landsvägen av skattelägenhet Bäckstrand 4:6. Invigningen av ungdomslokalen, vars byggnadskostnader noterades till 20000.- mark, kunde ske redan i slutet av 1912. Lagfart erhölls 19.02.1935 på 0,0013 mantal av Bäckstrand 4:33.

Förutom en livlig ungdomsverksamhet med danser, teater, sångkör, hornkapell, folkdanslag, gymnastikgrupp ordnades större skördefester, informations- och diskussionstillfällen om nykterhet mm. Lokalen användes ofta som biograf vid visning av filmer.

I norra ändan av huset fanns det s.k. kommunalrummet, som användes av kommunen fram till 1934. Ungdomsföreningen hade ett litet bibliotek, som 1922 överläts till kommunen. Utlåning skedde fram till 1938 i ungdomslokalens dåvarande rum för kläder mot söder. Ungdomsföreningen hade även hand om idrott och 1933 köpte föreningen mark till idrottsplan. År 1936 bildades Jeppo Idrottsförening, som då tog hand om idrotten och gymnastiken.

År 1921--\allowbreak 1922 i 46 veckor hyrde den ambulerande folkhögskolan Breidablick ungdomslokalen. Under vinterkriget användes lokalen som luftbevakningsstation och under fortsättningskriget var lokalen även under en tid militäranläggning. I krigets slutskede evakuerades Kemijärvi kyrkobys befolkning till Jeppo och deras kommunalkansli placerades i ungdomslokalens buffé- och vaktmästarbostad. I södra ändan av huset fanns vaktmästarbostaden. Bostadslokalen var fram till slutet av 1940-talet uthyrd till olika personer och familjer.

\jhpic{UF 1940.png}{Soldater och lottor vid Jeppo ungdomslokal i början av 1940-talet.}

Den 06.08.1953 beslöts att köpa hälften (1/2) av det ca 800 m$^2$ stora jordområdet, som låg mellan ungdomslokalens och Jeppo Sparbanks tomter. Vid den slutförda lantmäteriförrättningen 1988 sammanslogs tomterna och blev Bäckliden 4:195. I slutet av 1990-talet inköptes granntomten, Gästgiveri 4:200.

\jhbold{Renoveringar}
År 1939 utbyttes det utslitna furugolvet i festsalen till björkparkettgolv. Den 02.09.1939 invigdes det nya golvet med stor karneval och mycket folk. År 1955 gjordes en genomgripande renovering, festsalen, serveringsrum och kapprum förstorades. Artisten Tauno Manninen hade gjort väggdekorationerna, lamporna och takbeläggningen i buffén. Skådespelare Bjarne Commont hade utformat scenöppningen. Återinvigningen i november skedde med en stor fest med rikligt och fint program.

\jhpic{Lottor vid Uf.jpg}{Lottorna i Jeppo 1940}

Under 1960 och 70-talen växte behovet av större utrymmen för olika aktiviteter och 1974--76 blev det att på nytt ta sig an en ombyggnad av ungdomslokalen. Huset försågs med bottenvåning där biljardrum och hobbyrum inreddes, nytt kök, serveringsrum och toaletter. Salen förstorades, scenen förbättrades och nytt uppvärmningssystem installerades. Kostnaderna blev 700.000 mark. Trots insamlingar och bidrag från kommunen blev föreningen nu skuldsatt med räntor och amorteringar, som belastade verksamheten under tiotals år. I slutet av 1990-talet ändrades färgen från vit till röd.

Under 2000-talet har en duktig amatörgrupp årligen spelat en revy, som dragit folk till 8--10 föreställningar, och således bidragit ekonomiskt till föreningens verksamhet.

\jhbold{Hyresgäster i vaktmästarbostaden}
Evert och Svea Lindström bodde i medlet av 1930-talet flera år i vaktmästarbostaden. Gunnel föddes när de bodde i Ungdomslokalen. Se nr 52, Onnela av Strand skattelägenhet 4:17.

Affärsbiträde Svea Lahtinen, född 1924 på Stenbacken, bodde en tid i vaktmästarbostaden, likaså Märtha Linjamäki i slutet av 1940- och början av 1950-talet. Efter Georg Ströms skilsmässa 1952 flyttade Märtha och Georg till Sverige.


%%%
% [house] Gästgiveri, inkl. hyresgäster i båda husen
%
\jhhouse{Gästgiveri, inkl. hyresgäster i båda husen}{4:200}{Silvast}{5}{380-380a}

Styckad av stomlägenhet Bäckstrand 4:6

\jhbold{Hus 380a}

%%%
% [occupant] Jeppo
%
\jhoccupant{Jeppo}{\jhname[UF]{Jeppo, UF}}{1996--}
För att förstora gårdsplanen vid ungdomslokalen köpte Jeppo	Ungdomsförening i medlet av 1990-talet området mot norr,	Gästgiveri 4:200. Säljare var en fastighetsfirma i Karleby, som köpt området av Kenneth Johansson, Jakobstad. Ungdomsföreningen	rev bostadshuset i slutet av 1990-talet. Tidigare hade uthusraden	flyttats i ett stycke till Purmo av Stig-Olof Lillqvist, som använder det som garage och redskapsbod.


%%%
% [occupant] Johansson
%
\jhoccupant{Johansson}{\jhname[Kenneth]{Johansson, Kenneth}}{1969 o.1972--\allowbreak 1996}
År 1988 slutfördes en lantmäteriförrättning, som påbörjats 1958	av stomlägenhet Bäckstrand 4:6. Kenneth Allan William Johansson,	\textborn 13.09.1956 i Ludvika, Sverige, fick då lagfart på Gästgiveri 4:200.	År 1980 dimitterades han från ellinjen vid Tekniska läroanstalten i Vasa.	Kenneth bor i Jakobstad och arbetar tidvis i Norge. Nuvarande Gästgiveritomt var tidigare två skilda tomter med	bostadsbyggnader och uthus med olika ägare.

Kenneth Johansson erhöll genom testamente 13.11.1969 av sin farmor Elin Johansson, och godkänt 18.05.1978 av sonen Erik, en fastighet på ett visst område enligt styckningsöverenskommelse 06.11.1958, med tillägg 26.10.1987. År 1972 köpte Kenneth och erhöll lagfart 08.11.1972 på fastigheten mellan hans och ungdomslokalens tomter. Säljare Karl och Sigrid Svanbäck, Johannes Sandlin och Selma Sandlins arvingar.


%%%
% [occupant] Johansson
%
\jhoccupant{Johansson}{\jhname[William]{Johansson, William} \& \jhname[Elin]{Johansson, Elin}}{1939--\allowbreak 1969}
Den 19.10.1939 fick August William Johansson, \textborn 21.07.1896 på	Grötas hemman, och hans hustru Elin Irene, född Slangar, \textborn 23.04.1896	på Slangar hemman, lagfart på 0,0011 mantal av Bäckstrand 4:6. På tomten fanns en bostadsbyggnad i två våningar samt ett uthus. Säljare 05.04.1939 var Anna Raivio. William hade 1922 fått tjänst på 	statsjärnvägarnas godsmagasin i Helsingfors. År 1929 gifte sig William med Elin och hon flyttade till Helsingfors.

Barn: Erik Villiam, \textborn	23.04.1930 i Helsingfors. Erik är eltekniker. Han gifte sig 1955 med 	Märtha Fredrika, född Roos, \textborn 12.09.1930 på Romar hemman. Erik och Märtha bodde en kort tid i huset innan de flyttade till Sverige. William fick inte uppleva pensionstiden, han dog i cancer 23.12.1956. Elin flyttade 03.11.1961 till deras hus på Silvast. Efter 10 år flyttade hon till Jakobstad, där sonen med familj bodde. Elin dog i hjärtslag 02.04.1978 vid ett besök i sitt barndomshem på Slangar.


%%%
% [occupant] Nyman Flora
%
\jhoccupant{Nyman Flora}{\jhname[Anna Raivio]{Nyman Flora, Anna Raivio}}{1929--\allowbreak 1939}
Flora Nyman köpte 1929 0,0011 mantal av Bäckstrand 4:33. Anna Raivio  köpte fastigheten i början av 1930-talet. Johansson ansökte om fastebrev i efterskott, vilket noterats 10.10.1929 i samband med ansökan	om fastebrev 1939. Den 05.04.1939 säljer Anna Raivio fastigheten till William Johansson.


%%%
% [occupant] Ståhl
%
\jhoccupant{Ståhl}{\jhname[Paul]{Ståhl, Paul}}{1925--\allowbreak 1929}
Stationskarl Paul August Ståhl, \textborn 03.12.1900 I Nykarleby Forsby, blev änkling 05.11.1924 efter 2 års äktenskap med Hilda Sofia Jungarå, \textborn 12.05.1898 på Jungarå. Paret hade 15.05.1924	köpt en tomt den s.k. Trekanten av lägenhet Bäck 3:11 av Fors hemman. Deras son Paul Olof, \textborn 08.03.1923 i Pedersöre, dog 14.10.1925.	Paul Ståhl gifte sig 1926 med Agnes Maria, född Sandberg, \textborn 04.11.1905 på Fors hemman.

Från Tollikko flyttades von Essens bostadshus i två våningar till	Silvast till en tomt av lägenhet Bäckstrand 4:33 vid Silvastbäcken. Agnes och Paul fick 4 barn, som föddes på Silvast.
\begin{jhchildren}
  \item \jhperson{\jhname[Magnhild Birgitta]{Ståhl, Magnhild Birgitta}}{30.12.1926}{}, g. Lehtola -47, g. Stoor -60, Jstad
  \item \jhperson{\jhname[Maj-Britt Regina]{Ståhl, Maj-Britt Regina}}{21.09.1928}{}, g. Granroth -58, Jstad
  \item \jhperson{\jhname[Erna Veneta]{Ståhl, Erna Veneta}}{23.02.1932}{}, g. Kurtén -58, Karleby
  \item \jhperson{\jhname[Stig Fjalar]{Ståhl, Stig Fjalar}}{21.10.1937}{}, g. -60, Jstad
\end{jhchildren}
I slutet av 1920-talet såldes huset och familjen flyttade till norra bostadslokalen av det röda huset på stationsområdet. År 1937 flyttade 	familjen till Jakobstad och 1938 därifrån till Gamlakarleby, där Paul fick tjänst som stationskarl.  Paul \textdied 09.11.1962  ---  Agnes \textdied 06.09.1971.


\jhbold{Hyresgäster.}  (Alla hyresgäster är inte kända.)

\jhbold{Arne Valtteri Wikström}, \textborn 30.05.1913 på Grötas hemman bodde 1976 som pensionär ca ½ år i huset. I Seinäjoki hade han som företagare haft en cykelverkstad. Se mera Grötas R:nr 103. Arne Wikström dog 13.07.1976.

\jhbold{Helmi Koski}, \textborn 29.05.1913 och sonen Kuisma Koski, \textborn 06.01.1945 i Lehtimäki,	bodde i nedre våningen några år innan de flyttade till Bostads Ab Älvliden på Grötas. Kuisma arbetade på minkfarmen på Keppo och senare på KWH-Mirka. Kuisma Koski köpte 	1993 Ivar och Vieno Björkqvists bostadshus på Grötas. Helmi Koski dog den 26.03.1989.

På 1970- och 80-talet bodde i huset flera personer, som arbetade på KWH-Mirka kortare eller längre perioder. De kom från norra och östra	Finland.

Massör, fröken \jhbold{Aina Juliana Berg}, \textborn 26.08.1880 i Sysmä kom från Helsingfors 28.03.1936 till Romar. Hon fick sin bostad 1936 i övre våningen i ``kemikalia''-huset, nr 381, då Sarelin köpte huset av Sundell. Aina Berg bodde några år hos familjen Söderholm på Prästgården. I slutet av 1940-talet flyttade hon till övre våningen i Johanssons hus. Hon var aktiv inom Frälsningsarmén. Aina Berg dog den 22.09.1966.

Barnsköterskan \jhbold{Birgitta Nyqvist} bodde i huset 1972--\allowbreak 1973. En privat förening med småbarnsföräldrar och Gurli Lindén som ordförande startade i början av 1970 en barnträdgård med talkokrafter vid gamla Jungar skolas lärarbostäder. Birgitta Nyqvist anställdes som ledare för barnträdgården. Kommunen övertog verksamheten före sammanslagningen av Jeppo, Munsala och Nykarleby 1975.

Byggnadsingenjör \jhbold{Ralf Rainer Lärka}, \textborn 15.05.1939 i Närpes gifte sig 30.09.1961 med Tora Venny Magdalena, född Biskop, \textborn 30.10.1938 i Kronoby. Tora fick tjänst som mejerska på Jeppo Andelsmejeri 01.05.1960 och bodde på mejeriet. De hyrde övre våningen i Johanssons hus 01.08.1961-31.03.1963.

På hösten 1964--\allowbreak 1968 hyrde \jhbold{Per Mattias Lillström}, \textborn 25.02.1940 på Silvast och hans hustru Gundel Ingegerd, född Juslin, \textborn 24.11.1942 på Måtar hemman, den övre våningen. Per och Gundel gifte sig 1964. Våren 1961 dimitterades Per från bilavdelningen vid centralyrkesskolan i Vasa, och fick arbete på Haldins verkstad i Jakobstad. I slutet av 1968 flyttade de till Vasa Andelsbanks hus på andra sidan av landsvägen. Gundel arbetade 10 år som butiksbiträde på andelslaget Varma och därefter 4 år på Fyren i Andelsbankens hus. Deras dotter Pia är född 1969.  År 1973 flyttade familjen till Jakobstad. Under tiden som Lillström Per och Gundel bodde i Johanssons hus hyrde en familj Hänhinmäki nedre våningen. Fru Tuula var sömmerska. De hade en son, Pasi.

På sommaren 1964 hyrde \jhbold{Stig Lindén}, \textborn 08.08.1943 på Kaukos hemman, och hans hustru Eva, född Rögård, \textborn 14.09.1944 i Pensala, nedre våningen. De hade då sonen Bo Krister Johan, \textborn 26.10.1963. Stig hade sommararbete på verkstaden hos sin bror Boris Lindén. Stig studerade till ingenjör på Vasa tekniska läroanstalt.

Socionom \jhbold{Harry Johan Grahn}, \textborn 29.06.1925 på Kampas kronoboställe, och hans hustru Svea Ingegerd, född Roos, \textborn 23.04.1925 på Romar hemman, hyrde nedre våningen 1957--\allowbreak 1960. Harry arbetade då som kommunsekreterare i Jeppo och Svea på HAB:s bankkontor.	Barn:	Kaj Johan, \textborn 01.11.1952, 	Gun Monika,	\textborn 26.10.1956.	Barnen föddes när de bodde i Östermans hus på Fors hemman. De byggde ett egnahemshus, som blev inflyttningsklart 1960 (se Silvast nr 50).

Damfrisör \jhbold{Aune Hellsten} från Ytterjeppo, g.m. Harry Saviaro (karta 10, nr 106), hade under en kortare tid sin salong i huset.

\jhbold{John Erik Helsing}, \textborn 01.12.1915 i Munsala, gift 18.07.1937 med Adele Alexandra Andersson, \textborn 29.05.1918 i Munsala, hyrde 05.03.1941  övre våningen. Barn: Ulla Johanna,	\textborn 28.09.1937 i Munsala, Kita Mari-Ann, \textborn 09.01.1939 i Munsala och 	Alf Erik Bernhard, \textborn 18.03.1941 i Jeppo. Samtidigt hyrde familjen \jhbold{Sven Fors} nedre våningen. Huset kallades då ``Helsingfors''. Familjerna flyttade 1943 till Jeppo-Oravais Handelslags nya byggnad. Se även nr 17 på Romar hemman, som har uppgifter om familjen John Helsing.

Agronom \jhbold{Johannes Lennart Jungerstam}, \textborn 28.08.1901 på Jungar hemman, bodde några år efter kriget på Silvast. Han flyttade 21.12.1951 till Vasa.

Under kriget 1939--\allowbreak 1940 togs huset i användning av staben för militären, som fanns på ungdomslokalen och senare på finska folkskolan.


\jhbold{Gästgiverihuset. Hus 380}


%%%
% [occupant] Jeppo
%
\jhoccupant{Jeppo}{\jhname[UF]{Jeppo, UF}}{ca 1997--}
Jeppo Ungdomsförening äger idag tomten där f.d. Gästgiverihuset	fanns vid landsvägen med uthusbyggnad i vinkel längs bäcken och upp 	mot vägen. Säljare i medlet av 1990-talet var en Fastighetsbyrå i Karleby. Jeppo Ungdomsförening rev byggnaderna, även Johanssons bostadsbyggnad. Uthusbyggnaden på Johanssons hus hade tidigare, 1996, forslats till Purmo av Stig-Olof Lillqvist.

\jhpic{Jeppoboden mm.JPG}{Johanssons hus t.v.(nr 380a), affärs- och bostadshus i f.d. Jungells Gästgiveri (nr 380) och t.h. Jeppo Uf (nr 80).}


%%%
% [occupant] Johansson
%
\jhoccupant{Johansson}{\jhname[Kenneth]{Johansson, Kenneth}}{1972--\allowbreak 1996}
Kenneth Johansson köpte fastigheten av Karl och Sigrid Svanbäck	och Johannes Sandlin samt Selma Sandlins arvingar. Johansson erhöll lagfart 08.11.1972.


%%%
% [occupant] Svanbäck
%
\jhoccupant{Svanbäck}{\jhname[Karl]{Svanbäck, Karl} \& \jhname[Sigrid]{Svanbäck, Sigrid}}{1946--\allowbreak 1972}
År 1946 överlät Johannes Sandlin och Selma Sandlin hälften av sin	fastighet på 0,14 ha eller 0,0010 mantal av Bäckstrand 4:33. Mottagare	var Johannes syster Sigrid Juliana Svanbäck, \textborn Sandlin	07.10.1905 på Grötas, gift 1941 med Karl William Svanbäck, \textborn	05.05.1906 i Nykarleby. Lagfart 23.03.1946 på 0,0005 mantal. Karl arbetade som taxichaufför och som mjölnare på Jeppo Kraft Andelslag, Silvast Kvarn.

Barn:	Karl-Johan,	\textborn 19.05.1943.	Karl-Johan blev konditor i Göteborg och gifte sig med Solbritt, född	Åkesson, \textborn 09.05.1944.

I slutet på 1960-talet köpte Karl och Sigrid Svanbäck en lokal i Nykarleby och flyttade dit.
Karl \textdied 21.07.1985  ---  Sigrid \textdied 25.08.1998. De är	begravna i Nykarleby.


%%%
% [occupant] Sandlin
%
\jhoccupant{Sandlin}{\jhname[Johannes och Selma]{Sandlin, Johannes och Selma}}{1938--\allowbreak 1972}
Den 23.03.1938 erhöll Johannes Sandlin, \textborn 03.05.1903, och hans faster Selma Katarina Sandlin, \textborn 13.03.1893, båda på Grötas hemman, lagfart på 0,0010 mantal av stomlägenhet Bäckstrand 4:33. Selma var ogift och invalid. Hon försörjde sig som stickerska i en	liten stuga vid hemgården på Grötas. Selma sålde samtidigt 1938	till säljarna av Gästgiveriet, Anders och Olivia Jungell, sitt och broder Emils hus på Fors hemman. Emil, \textborn 02.02.1882, hade återvänt från USA som ogift. Han dog i lungsot 09.01.1937. Johannes och hans hustru Hilda och syster Sigrid fortsatte med gästgiveriverksamheten. År 1941, då Sigrid och Karl Svanbäck gifte sig, bosatte de sig i övre våningen.
Selma \textdied 11.05.1965 i hjärnblödning  ---  Johannes \textdied 11.10.1973.


%%%
% [occupant] Jungell
%
\jhoccupant{Jungell}{\jhname[Anders	\& Olivia]{Jungell, Anders	\& Olivia}}{1928--\allowbreak 1938}
Några år efter att Anders Jungell, \textborn 26.07.1885 på Skog hemman,	och hans hustru Olivia, född Pettersson, \textborn 19.08.1887 i Övermark, återvänt från USA, sålde de 1928 Hemfrid lägenhet på Skog hemman till Joel Sandberg från Finskas. De byggde ett 2-vånings bostadshus samt ett	större uthus mellan Silvastbäcken och landsvägen på en tomt av Stomlägenhet Bäckstrand 4:6. I oktober 1928 beslöt kommunfullmäktige i Jeppo att till gästgivare och skjutskarl anställa Anders Jungell.

På 1930-talet avvecklades gästgiverirörelsen i vårt land såsom överflödig	i bilismens och bussarnas tidevarv. Jungells gästgiveri indrogs 1938 och	Anders och Olivia beslöt att sälja fastigheten och byta till ett mindre	bostadhus, som ägdes av Selma Sandlin på en tomt av Bäck stomlägenhet på Fors hemman. Anders och Olivia var barnlösa. Anders bror ``Jannes'' 	dotter Ellen, \textborn 1908, blev moderlös vid 9 månader och bodde med sin farmor tills hon dog 1926. Vid 18 år flyttade Ellen till Anders och	Olivia och blev deras fosterdotter. När Ellen gifte sig 1932 med Leander Sandberg bodde de på Gästgiveriet till 1938.

Barn:	Magda Katarina,	\textborn 17.11.1933 och Ingmar Matias,	\textborn 14.11.1937.	Se mera Sand-Ek 4:134, karta 5, nr 84.

År 1945 gjorde Olivia ett byte av fastigheter med Katarina Sikström. Se	mera nr 99 och 107 på Fors hemman.
Anders \textdied 07.02.1942  ---  Olivia \textdied 24.12.1961


\jhbold{Hyresgäster}:

År 1972 renoverade Kenneth och Erik Johansson övre våningens bostäder. \jhbold{Johan Slangar} hyrde 1973 lokalerna för sina anställda. I nedre våningen öppnade \jhbold{Gunilla Nyman} 1982 en Blomster- och	Beklädningsaffär. I slutet av 1990-talet flyttade hon affären till lokalen,	som hyrts av FBF som bankkontor i KPO-huset. Anställda butiksbiträden	var Gunborg Julin och Birgitta Elenius m.fl., bl.a. Maija Stenbacka en kortare tid.

År 1976--\allowbreak 1982 hyrde \jhbold{John Nyman Kb} nedre våningen, som järn- och	lantbruksaffär. Matvaruaffären fanns på andra sidan landsvägen i	Andelsbankens fastighet. År 1975 köpte Nyman av Andelslaget Varma magasinbyggnaden på stationsområdet. År 1982 renoverades byggnaden till affärslokal och förstorades med en större lagerbyggnad. 	Lantbrukscentralen flyttade till stationsområdet. Den 07.12.1978 	övertogs Nymans matvaruaffären av Rolf Mörk.

Chaufför \jhbold{Veikko Kontio}, \textborn 1941 och hans hustru Gunnel, född Bergström \textborn 1946, samt deras dotter Katarina \textborn 1971, kom från Nykarleby 1973. Veikko och Gunnel hade gift sig 1970. De hyrde nedre 	våningen 1973--\allowbreak 1975, då de flyttade till sitt nybyggda hus på Åkervägen. Gunnel dog plötsligt i Andelsringens butik 1989.

Under tiden Sandlin och Svanbäck ägde fastigheten hyrde \jhbold{John och Lea Lassén} nedre våningen 1953--\allowbreak 1968, där de drev diversehandeln	\jhbold{``Jeppoboden''}. Verksamheten avslutades på grund av dålig lönsamhet. Se mera om familjen Lassén under nr 95, karta 6, på Fors hemman.

1947--\allowbreak 1949 hyrde \jhbold{Elmer och Dagmar Fagerholm} en lokal på Gästgiveriet. Deras dotter Magnhild föddes 11.07.1947 då de bodde där.	På ett område på Grötas av hemgården Fagerholms mark vid älven byggde Elmer ett snickeri och en bostadsbyggnad, som familjen kunde flytta in i sensommaren 1949.



%%%
% [house] Andelsbanken, inkl. hyresgäster
%
\jhhouse{Andelsbanken, inkl. hyresgäster}{4:199}{Silvast}{5}{92}

Styckad av stomlägenhet Bäckstrand 4:6

%%%
% [occupant] Fab Jeppo
%
\jhoccupant{Fab Jeppo}{\jhname[Fastighetscenter]{Fab Jeppo, Fastighetscenter}}{2007--}
\jhbold{Ab Jeppo Lantgris Oy} köper år 2007 aktierna i Fab Jeppo Fastighetscenter, som äger fastigheten Andelsbanken 4:199.	Säljare Vasa Andelsbank, som den 14.03.1994 hyrt banklokalen i fastigheten Sparbanken 4:194, nr 81. Fab Jeppo Fastighetscenter renoverar och förändrar butikslokalen till fyra bostadslokaler med egna ingångar. Tillsammans med de tre gamla lokalerna finns nu sju bostadslägenheter i byggnaden, som i huvudsak används för anställda i Ab Jeppo Lantgris.


\jhhousepic{032-05567.jpg}{Platsen mest känd som ``bokföringsbyrån'' i dag}

\jhbold{Nykarleby Bokföringsbyrå Ab} öppnade Jeppo-kontoret den 01.10.1981, nr 81 på Fors hemman. Den 15.08.1996 flyttade byrån till Andelsbankens f.d. banklokal. Bokföringsbyrån övertar 2006 Barbro Stenfors kunder; ÖSP-skattetjänst med jord- och skogsbrukares bokföring och redovisning samt en del andra företagskunder.

Den 31.12.2012 fusionerades Nykarleby Bokföringsbyrå med Ab Norlic Oy.

Anställda 2017;
\begin{enumerate}
  \item \jhname{Forsgård, Stig}     \textborn 1956   05.06.1979--
  \item \jhname{Ludén, Ronnie}      \textborn 1979	 24.09.2007--
  \item \jhname{Söderlund, Maria}   \textborn 1985	 16.08.2010--
  \item \jhname{Sandberg, Gunilla}  \textborn 1961	 29.03.2016--
\end{enumerate}

Tidigare anställda;
\begin{enumerate}
  \item \jhname{Hanstén, Anna-Greta}    28.07.2004--31.07.2016
  \item \jhname{Berg, Sofie}            02.01.2008--31.07.2011
  \item \jhname{Björkqvist, Marit}      14.12.2009--12.05.2010
  \item \jhname{Wiklund, Catharina}     02.01.2006--31.08.2007
  \item \jhname{Stoor, Anita}           01.07.1992--30.04.1996
  \item \jhname{Julin, Barbro}          16.10.1986--30.06.1992, 01.05.1996--31.07.2004
\end{enumerate}
Stig Forsgård är ansvarig byråchef i Jeppo.


\jhbold{Hyresgäster i gamla butikens bostäder 2017}:
\begin{enumerate}
  \item Artur Baiev med fru Janina
  \item Oleksandr Shevchuk med fru Yugeniia
  \item Vladyslav och Anna Tverdous med sonen Ivan
  \item Oleksandr Skakun
  \item Hanna Lakhman
\end{enumerate}
Artur Baiev, Vladyslav och Anna Tverdous samt Oleksandr Shevchuk arbetar på Jeppo Lantgris Ab. Oleksander Skakun arbetar på Nassab Ab, Pensala. Hanna Lakhman bor i lokal 1 och arbetar hos Olav Jungarå.


\jhbold{Hyresgäster i gamla butikens lokaler 2013--\allowbreak 2016}, som flyttat bort nyligen eller för några år sedan:

\begin{itemize}
  \item Juliia Bashynska \textborn 04.06.1985 i Ukraina, kom 18.06.2012, flyttade till lokal 3 år 2013 och till Jakobstad 2017, arbetar på Jeppo Lantgris.
  \item Anton Tsyhanov från Ukraina, flyttat till Jakobstad, arbetar på Jeppo Lantgris.
  \item Eve Juurma och Janis Mitts, från  Estland, bor kvar i Jeppo på radhusområdet.
  \item Nataliia Lefimenko från Ukraina, bor i Jakobstad, arbetar på Jeppo Lantgris.
  \item Kristian Jacobsen med fru Ann-Mari och sonen Asker, danskar som nu bor i Danmark.
\end{itemize}

\jhbold{Hyresgäster 2007--\allowbreak 2013}:

\begin{tabular}{ll}
  Gennadiev Pavlo,  \textborn 01.01.1977,   & 26.04.2010-31.01.2011 \\
  Löhmus Eha,       \textborn 16.01.1965,   & 06.09.2011-17.12.2011 \\
  Naida Oleksandr,  \textborn 11.10.1979,   & 09.10.2008-31.01.2010 \\
  Lysak Kostyantyn, \textborn 28.03.1979,   & 29.04.2008-25.04.2010 \\
  Lysak Olga,       \textborn 15.01.1986,   & 07.01.2009-25.04.2010 \\
  Lysak Arlen,      \textborn 03.02.2010,   & 03.02.2010-25.04.2010 \\
  Vovk Serhii,      \textborn 08.10.1970,   & 24.04.2009-07.09.2009 \\
\end{tabular}


%%%
% [occupant] Vasa
%
\jhoccupant{Vasa}{Andelsbank}{1967--\allowbreak 2006}

\jhsubsubsection{Mellersta Österbottens Andelskassa}

Mellersta Österbottens Andelskassa, med huvudkontor i Kimo, köper 28.08.1963 ett område av Bäckstrand 4:33 på Silvast. Filialkontoret i Jeppo bygger 1966--\allowbreak 1967 ett nytt bankhus. Andelskassan hade startat sin verksamhet i Jeppo 1953 i fastigheten Kilen, nr 96, på Fors hemman.	Förutom banklokalen på tre rum mot norr, byggdes en större butikslokal mot söder och förråd, pannrum, två mindre bostadslokaler samt en större med tre rum och kök mot väster, mot älven. 1986 gjordes en bytesaffär av tomtområdet med Kurt och Mona Stenvall, samt	försäljning 2003 av en del av tomten till dem. År 1970 blev andelskassorna andelsbanker. 1971 fusionerades Mellersta Österbottens Andelsbank med Vasa Andelsbank.

\begin{table}[ht]
  \centering
  \begin{tabular}{l l l}
    \jhname{Brunell, Eliel}    & 1953--\allowbreak 1969 ett kort avbrott. & \textborn 1904 \\
    \jhname{Haga, Hjördis}     & 1956--\allowbreak 1957		                & \textborn 1922 \\
    \jhname{Hermans, Bo-Erik}  & 1970--\allowbreak 1978		                & \textborn 1947 \\
    \jhname{Nordström, Ulla}   & 1975--\allowbreak 1998		                & \textborn 1938 \\
    \jhname{Slangar, Märtha}   & 1978--\allowbreak 1998		                & \textborn 1937 \\
  \end{tabular}
  \caption{Föreståndare och anställda i banken}
\end{table}

Den 14.03.1994 tömdes banklokalen då verksamheten flyttades till Sparbanken 1:194, figur 81. Vasa Andelsbank ombildade fastigheten till ett aktiebolag, Fab Jeppo	Fastighetscenter.


\jhbold{Hyresgäster i butikslokalen 1967--\allowbreak 2004}:
1967 öppnades matvaruaffären \jhbold{Fyrens Snabbköp} av fyra handelsmän från Ytteresse, Smeds, Ena, Nyman och Östman. \jhbold{John Nyman} och hans hustru Gunilla flyttade till Jeppo och John blev den som drev affären med anställda butiksbiträden. 1971 blev Nyman ensam ägare till Fyrens matvaruaffär. Han utökade med järn- och lantbruksprodukter.

1976 hyrde John Gästgiverihuset på andra sidan vägen och 1982 flyttades Lantbrukscentralen till Järnvägsområdet. Se mera Fors 4:171, nr 64.	Anställda i Fyrens Snabbköp:	Lillström Gundel,	1968--\allowbreak 1973	och	Kula Ruth, 1975--\allowbreak 1978.

I december 1978 övertar \jhbold{Rolf Mörk} matvaruaffären och 07.12. öppnar han \jhbold{Jeppo Livsmedel}. Mörk blir K-köpman i Nykarleby 1980. \jhbold{Göran Näs} blir köpman i Jeppo och öppnar 12.03.1980 \jhbold{K-Göran}. Näs blir lagerchef på Mirka 1984. \jhbold{Anita Kula} blir köpman på 	\jhbold{K-Anita} 1984--\allowbreak 1996. På grund av nedsatt syn blir Anita tvungen att sluta	som handelsman. \jhbold{Susanne Yli-Aho} anställd som butiksbiträde övertog	butiksrörelsen 1996 under namnet \jhbold{K-Susanne}. På grund av dålig lönsamhet upphörde Susanne med butiken 2004.

Anställda under 1978--\allowbreak 2004: Ludén Anja, Elenius Birgitta, Erikslund Nina, Dahlsten Karin och Lindvall Leila.


\jhbold{Hyresgäster i bostadslokaler 3-1, 1967--\allowbreak 2017}:

\jhbold{Lokal 3}

\begin{table}[ht]
  \centering
  \begin{tabular}{l p{0.6\textwidth} l}
    År & Boende & Anm. \\
    2013--\allowbreak 2017 & Bashynska Juliia, från Ukraina, flyttade till Jakobstad &   \\
    2012--\allowbreak 2013 & Jakobsson Mats o Glenn för 5 studerande från Nepal &   \\
    2009--\allowbreak 2011 & Halldorson Shelley, f.08.06.1965 i Canada och hennes man Victor Poza, f. 23.02.1977 i Spanien, flyttade 31.05.2011 till Jungarvägen 136. De bor nu i Spanien. &   \\
    1989--\allowbreak 2009 & Hongisto Kurt, Anita och Mikaela & Se lokal 3 \\
    1988--\allowbreak 1989 & Dahlström Alf och Päivi & Se nr 35 \\
    1986--\allowbreak 1988 & Stenvall Kurt och Mona & Se nr 93 \\
    1980--\allowbreak 1986 & Näs Göran och Annikki & Se lokal 3 \\
    1978--\allowbreak 1980 & Mörk Rolf & Se lokal 3 \\
    1970--\allowbreak 1978 & Hermans Bo-Erik och Carita & Se lokal 3 \\
    1969--\allowbreak 1970 & Nyman John och Gunilla & Se nr 64 o.75 \\
    1967--\allowbreak 1969 & Brunell Eliel, Ines och Rolf & Se nr 75 \\
  \end{tabular}
  \caption{Lokal 3}
\end{table}

1989--\allowbreak 2009:
Familjen Hongisto flyttade 1989 till lokal 3 från Bostadsab Pastellen på Åkervägen. Bjarne Kurt Eric, \textborn 14.01.1947 i Lassila by, arbetade som lastbilschaufför och hade en tid en mink- och rävfarm i Lassila. Hustrun	Anita Linnéa, född Rönnblom, \textborn 16.07.1947 i Nedervetil. Deras dotter Mikaela Linnéa Henriette föddes 28.12.1984, då de bodde i Kurts barndomshem i Lassila. Anita har arbetat som affärs- och cafébiträde samt under 2000-talet som hemhjälpare. Kurt drunknade 25.06.1995 vid Sexsjö i Purmo. Mikaela blev student från Jakobstads Gymnasium och studerade till styrman vid Sjöfartsskolan i Mariehamn. Mikaela är sambo med växthusodlare Ulf Strandholm i Närpes och arbetar i växthus	och kör långtradare. Anita flyttade 2009 till Larsmo.

1980--\allowbreak 1986:
Göran Näs, \textborn 20.10.1944 på Stenbacken och hans hustru Annikki, född Mattila, \textborn 29.09.1946 i Alahärmä, flyttade 1980 till lokal 3 när Göran blev köpman i butikslokalen. Göran och Annikki flyttade från Grankulla. 1973--\allowbreak 1980 hade de bott och arbetat i Esbo och Grankulla. Som ung fick Göran arbete på Handelslaget i Jeppo 1960--\allowbreak 1966. År 1966 gifte sig Göran	och Annikki och Göran blev föreståndare på Andelsringens filial i Kimo 1966--\allowbreak 1972 och därefter vid Pedersöre Handelslags filial i Hirvlax. Göran studerade 2 år vid Kooperativa Handelsinstitutet i Helsingfors. När de 	kom tillbaka till Jeppo fick Annikki arbete som löneräknare på Mirka. 1984 fick Göran tjänst som logistikkoordinator på KWH-Mirka. Göran och Annikki köpte 1986 en tomt, Näs 4:118, en del av Holmen 4:18	av Silvast hemman. De byggde ett bostadshus och kunde flytta in i sitt nya hem samma år. De gick i pension 2007 och 2013 sålde de egnahemshuset till Juha Hietala och flyttade till Nykarleby.

1970--\allowbreak 1979:
När merkonom Bo-Erik Hermans, \textborn 1947 i Kimo, fick tjänst 1970	som föreståndare på Andelsbanken, flyttade han med sin hustru,	sjukskötare Carita, född Finskas, \textborn 1949 på Finskas, till direktörsbostaden. År 1979 kunde de flytta in i sitt nybyggda hus på Finskas hemman. Bo-Erik blev 1978 direktör för Vasa Andelsbanks kontor i Oravais.
\begin{jhchildren}
  \item \jhperson{\jhname[Bo Samuel]{Hermans, Bo Samuel}}{1971}{}, automationsingenjör
  \item \jhperson{\jhname[Anna Maria Irene]{Hermans, Anna Maria Irene}}{1973}{}, personalvetare
  \item \jhperson{\jhname[Ida Sofia]{Hermans, Ida Sofia}}{1980}{}, socionom
  \item \jhperson{\jhname[Tobias Erik Joel]{Hermans, Tobias Erik Joel}}{1986}{}, ingenjör
\end{jhchildren}
Samuel bor och arbetar i Trollhättan, Sverige och Anna bor med sin 	familj i Varberg, Sverige. Ida gift med Kimi Holmström bor med sin familj, man och fyra barn på Gunnar hemman, se Gunnar. Tobias är gift	med Caroline från Dominikanska Republiken och bor i sitt bostadshus vid Jutas. Tobias arbetar på Wärtsilä i Vasa. Bo-Erik gick i pension 2004 och Carita 2014 från sin sjukskötartjänst vid Nykarleby Sjukhem.


\jhbold{Lokal 2}

\begin{tabular}{lp{0.7\textwidth}}
  2013--                  & Omelchenko Sviatoslav,  \textborn 27.03.1980 i Ukraina \\
  2012--\allowbreak 2013  & Shkaruba Olexiy,        \textborn 15.09.1984 i Ukraina \\
  2012--\allowbreak 2012  & Bashynska Juliia,       \textborn 04.06.1985     ''  \\
  2012--\allowbreak 2012  & Gasiak Valerii,         \textborn 05.04.1987     ''  \\
  2010--\allowbreak 2011  & Tamavskyy Oleh,         \textborn 18.07.1969     ''  \\
  2004--\allowbreak 2009  & Den 01.01.2004 flyttade målaren Maunu Tapani Hahto, \textborn 1961, in i lokal 2. Han kom från Sparbanken, nr 81, där han bodde tidigare med sin familj fram till skilsmässan 1997. 31.10.2009 flyttade Hahto ut.  \\
  2000--\allowbreak 2003  & Inga hyresgäster 01.05.2000-31.12.2003  \\
\end{tabular}

1982--\allowbreak 2000:
Signe Näs, \textborn Broman 29.09.1916 i Ytterjeppo, flyttade 1982 från sitt hus på Stenbacken till lokal 2. Signe Näs var änka till församlingsmästare Gunnar Näs, som dog 1981. Hon hjälpte sin man med städningen av församlingshemmet.  Signe är mor till Göran Näs. Signe flyttade till Jeppo-stugan på Åkervägen i maj 2000. Hon dog midsommarafton 2015 på Hagalund.

1973--\allowbreak 1982:
Ellen Linnéa Romar, född Elenius, \textborn 09.05.1905 på Gunnar, hyrde lokal 2 åren 1973--\allowbreak 1982. Ellen var änka till Selim Romar, \textborn 11.01.1900, \textdied 07.07.1959. Ellen \textdied 07.10.1982. Se mera nr 20 på Romar hemman.

1969--\allowbreak 1973:
Per Lillström, \textborn 1940, och hans hustru Gundel, \textborn 1942, hyrde 1969--\allowbreak 1973 lokal 2. Deras dotter Pia föddes 1969, när de bodde på Andelsbanken. År 1973 flyttade familjen till Jakobstad och senare till Bennäs, Lövö. Se mera Gästgiveri 4:200, nr 379-380.

1967--\allowbreak 1968:
Socionom Kaj Keskinen och hustrun merkonom Pirkko, född Hanhinmäki från Alahärmä arbetade på Mirka och hyrde lokal 2 1967--\allowbreak 1968. De flyttade till Kälviå.


\jhbold{Lokal 1}

\begin{tabular}{ll}
  2016-                   & Lakhman Hanna \\
  2016--\allowbreak 2016  & Tverdous Vladyslav, 10 månader \\
  2015                    & Ingen hyresgäst \\
  2014                    & Dovgopolyi Jurii, 4 mån. \\
  2012—2014               & Balazs Ödön-Csaba, f. 19.01.1981. \\
  2012	                  & Ellart Kristi, f. 19.06.1958,  5 mån., 23.4-30.9. \\
  2012                    & Hayday Serhiy, f. 21.04.1978, 30.12.11-14.2.12, till Nykarleby. \\
  2011                    & Rand Reena, f. 16.02.1968, 1 mån. \\
  2011--\allowbreak 2012  & Kuzmenko Iryna, f. 28.03.1985 och Kuzmenko Russian, f.09.12.1977, till Nykarleby. \\
  2010--\allowbreak 2011  & Filonenco Olga, f.08.05.1983 i Ukraina, flyttade till Brandstationen 31.01.2011 och därifrån 31.03.2013 till eget hus i Hirvlax. \\
  2009--\allowbreak 2010  & Inga hyresgäster 01.10.2009-31.08.2010 \\
\end{tabular}

2007--\allowbreak 2009 och 2006:
Miljöingenjör Anni Paadar, född i Vasa, blev anställd 2006 av Jeppo Kraft för att inventera älven för vattenkraft. Hon hyrde lokal 1 den 03.07.2006-31.10.2006, då hon flyttade till Eva Malinen på Keppo.	01.11.2006-03.06.2007 hyrde Jeppo Lantgris lokalen. Den 04.06.2007 flyttade Anni Paadar tillbaka. Anni är kvalitets- och miljöansvarig på Oy Jeppo Biogas Ab. Anni flyttade 30.09.2009 till Lillkyrö med sin sambo. De har en dotter. Hon pendlar till Jeppo.

2005--\allowbreak 2006:
Ingen hyresgäst 01.10.2005-02.07.2006

2002--\allowbreak 2004:
Dahlström Ralf. Den 01.12.2002 hyrde Ralf	Erik, \textborn 08.05.1972 på Jungarå. Han flyttade till sin hemgård den 31.12.2004.

1998--\allowbreak 2002:
Ingen hyresgäst 01.10.1998-30.11.2002.

1991--\allowbreak 1998:
Dahlström Stig Ingmar, \textborn 21.01.1958 på Jungarå. 30.09.1998 flyttade Stig till Munsala. Se nr 35 på Nylandsvägen.

1973--\allowbreak 1991:
Ester Sofia Källman, \textborn 01.11.1904, i USA hyrde lokal 1 år 1973. Ester flyttade från sitt hem på Bäckström 3:96 på Fors hemman, se nr 90 och 390. Ester dog 09.07.1991.



%%%
% [house] Nygård
%
\jhhouse{Nygård}{4:208}{Silvast}{5}{83}

Stomlägenhet Nygård 4:9

\jhhousepic{048-05564.jpg}{Staffan Stenvall och Noora Luoma-Stenvall}

%%%
% [occupant] Stenvall, Luoma-
%
\jhoccupant{Stenvall, Luoma-}{\jhname[Staffan]{Stenvall, Luoma-, Staffan} \& \jhname[Noora]{Stenvall, Luoma-, Noora}}{1990--}
Stenvall Staffan Paul Alexander, \textborn 12.09.1973 på stomlägenhet Nygård 4:9 av Silvast hemman, övertog gården och jordbrukslägenheterna tillsammans med sin bror Kurt. Staffan var endast 16 år då deras far Paul dog. Staffan blev student 1992 och arbetade samtidigt på jordbruket. Efter militärtjänstgöring vid Dragsvik 1993 studerade han 3 år vid Barnträdgårdsinstitutet i Jakobstad, men slutförde inte studierna. Staffan är även intresserad av teknik och vårvintern 1997 fick han anställning på KWH-Mirka. Han har haft möjlighet att via läroavtal med Optima studera till processkötare, arbetsledare, kvalitetstekniker och är i dag Quality Improvement Specialist på Oy KWH-Mirka Ab. Jordbruket har minskat och består av 5 ha odlad jord och 60 ha skog.

Den 11.09.1999 förlovade sig Staffan och Noora Luoma från Oravais,	\textborn 11.09.1979 i Vasa. Som förlovad sambo utbildade sig Noora till sjukskötare vid Vasa Yrkeshögskola. Hon pendlade från Jeppo till Vasa. Den 30.11.2003 vigdes Staffan och Noora i Jeppo kyrka. Noora tog tillnamnet Luoma-Stenvall. Noora har tjänst på Jakobstads Social- och Hälsovårdsverk. Hon har arbetat på Nykarleby sjukhus och i dag som sjukskötare inom den öppna hemsjukvården i Munsala.
\begin{jhchildren}
  \item \jhperson{\jhname[Oscar Paul Alexander]{Stenvall, Luoma-, Oscar Paul Alexander}}{25.09.2003}{}
  \item \jhperson{\jhname[Ida Emilia Elisabet]{Stenvall, Luoma-, Ida Emilia Elisabet}}{04.08.2005}{}
  \item \jhperson{\jhname[Emil Martti Elias]{Stenvall, Luoma-, Emil Martti Elias}}{25.02.2013}{}
\end{jhchildren}
Makarna har varit styrelsemedlemmar i ca 10 år i Jeppo Uf och Noora styrelsemedlem/-ordförande i FRK, Jeppo sv. avd. Staffan har varit styrelsemedlem i SFP, Jeppo avd., Hem- och skolaföreningen och f.n. styrelsemedlem i Jeppo ungdomsorkester.

Staffan och Noora har under de senaste åren renoverat bostadshuset. Ekonomibyggnaden har också genomgått renovering och vattentaket försetts med plåt 2013.


%%%
% [occupant] Stenvall
%
\jhoccupant{Stenvall}{\jhname[Paul]{Stenvall, Paul} \& \jhname[Lea]{Stenvall, Lea}}{1951--\allowbreak 1990}
Stenvall Paul Einar, \textborn 21.12.1915 i Jeppo på Silvast/Nygård R:nr 4:9 som andra barnet bland 5 syskon. Paul arbetade från unga år på jordbruket tillsammans med sina föräldrar Elis och Aina. Militärtjänsten gjorde Paul på Sandhamn i Helsingfors 1937/1938 och deltog i vinter- och fortsättningskrigen 1939--\allowbreak 1944. Paul sade att han gick i militärkläder i över 6 år. Han sårades i vinterkriget och låg på militärsjukhuset i Viborg. Paul hemförlovades 16.11.1944 och befordrades till översergeant.

Före kriget arbetade han även med lastnings- och lossningsarbeten åt Jeppo-Oravais Handelslag. År 1951 gjordes generationsväxling och Paul övertog hemgården. På julaftonen 1954 förlovade sig Paul och merkonom Lea Elisabet Sandqvist, född på Silvast/Dahlbo 4:127, \textborn 28.06.1934. Paul och Lea vigdes midsommaraftonen 1955. De renoverade bostadshusets nedre våning samma år och 1962 installerades centralvärme med pannrum och bastu i ekonomibyggnaden, vars vattentak 1961 hade försetts med av Paul gjutna taktegel. År 1971 renoverades och inreddes den övre våningen.
\begin{jhchildren}
  \item \jhperson{\jhname[Gerd Elisabet]{Stenvall, Gerd Elisabet}}{16.01.1956}{},	avd.sjukskötare 1982--2017 på Ålands centralsjukhus
  \item \jhperson{\jhbold{\jhname[Kurt]{Stenvall, Kurt}} Peter Einar}{11.09.1960}{}, elingenjör, VD, se nr 93
  \item \jhperson{\jhname[Patrik Mikael Jakob]{Stenvall, Patrik Mikael Jakob}}{03.03.1966}{}, nyhets- och projektchef på HSS-Media
  \item \jhperson{\jhbold{\jhname[Staffan]{Stenvall, Staffan}} Paul Alexander}{12.09.1973}{}, kvalitetstekniker
\end{jhchildren}
Lantmäteriförrättning av stomlägenheterna Nygård R:nr 4:9,  Bäckstrand R:nr 4:6 och Norrgård R:nr 4:8 hade påbörjats 1958 av lantmäteriingenjör Lang och slutfördes 1988 av lantmäteriingenjör G. Karlsson. P.g.a. att det fanns fel i de mantal, som använts vid köp och försäljning av bostadstomter under 1900--\allowbreak 1950-talen, måste flera överenskommelser uppgöras för att klargöra vilka områden det gällde. Stomlägenheterna fick nya registernummer; 4:208, 4:209 och 4:210.

Under de första åren hade Paul och Lea 2 hästar innan de köpte en traktor, Ferguson en s.k. guldkalv. I fähuset fanns 12 mjölkkor, kalvar, får och grisar. Paul byggde också 1956 ett mindre svinhus, men verksamheten upphörde efter några år, då grisarna fick rödsot. År 1968 avslutades mjölkproduktionen och Paul övergick till spannmålsodling. Paul dog i cancer 31.08.1990 i hemmet omgiven av familjen.

Lea hade avlagt merkonomexamen 31.05.1953 vid Vasa Handelsinstitut och fick genast arbete som kontorist på Jouper-Produkt Ab i Nykarleby, där hon stannade till 13.01.1955. Den 21.12.1954 bad Kommerserådet Emil Höglund och Vd K.J.Tidström Lea komma till Keppo gård. Hon anställdes som bokförare för Oy Keppo Ab och Petsmo Minkgård Ab. Pälsfarmerna växte snabbt, nya företag köptes inom olika branscher och en del verksamheter såldes eller avslutades, samt fusioner och difusioner av företag förekom ofta. För att klara av alla utmaningar, först som kontorschef och redovisningschef, läste hon facklitteratur och studerade på kvällar och helger; redovisning, bokförings-, skatte- och personallagar, psykologi m.m.

Lea har haft en rad förtroendeuppdrag; över 30 år medl. i kyrkorådet, mångårig sekr. i kyrkofmge, i kyrkorådet för Nkby-nejdens kyrkl. samf., senare revisor, fmge-medl. i Jeppo kommun, revisor i Nkby stad, senare i revisionsnämnden. Hon har förlänats FRK-förtjänsttecken 1978, dito medalj i brons 1988 av republikens president, församlingens hedersdiplom för 30 år som söndagsskollärare.

En stor del av arbetet i hemmet och med barnen föll således på Paul, som utförde dem med beröm. Lea utnämndes 1985 till ekonomidirektör för KWH-koncernen med ansvar för ett gemensamt ekonomi- och koncernredovisningssystem för alla bolag inom koncernen. Efter nära 45 år i KWH:s tjänst gick Lea i pension som 65-åring den 30.06.1999. Den 20.10.2000 flyttade hon till en lägenhet i Bostads Ab Gökbrinken, som ligger granne till gamla hemmet.

\jhhousepic{Ruckusv 20 1930-t.jpg}{Ruckusvägen 20 i 1930-talsversion}

%%%
% [occupant] Stenvall
%
\jhoccupant{Stenvall}{\jhname[Elis]{Stenvall, Elis} \& \jhname[Aina]{Stenvall, Aina}}{1913--\allowbreak 1951}
Stenvall (f.d. Stenbacka) Jakob Elis, \textborn 18.04.1889 på Tapelbacken, järnvägens vaktstuga några kilometer norr om Jeppo station. Elis gifte sig 12.10.1913 med Aina Susanna Jonasdotter Jungell, \textborn 17.02.1891 på Skog hemman i Jeppo. Aina flyttade in i den nya bostadsbyggnaden, där Elis bodde med syskon.  Elis' far, banvakten Isak Stenbacka, byggde huset jämte ekonomibyggnad på Nygård 4:9, som han köpt 01.01.1908.
\begin{jhchildren}
  \item \jhperson{\jhname[Elis Waldemar]{Stenvall, Elis Waldemar}}{05.05.1914}{20.07.1943 vid Svir}
  \item \jhperson{\jhbold{\jhname[Paul]{Stenvall, Paul}} Einar}{21.12.1915}{31.08.1990 i Jeppo}
  \item \jhperson{\jhname[Anna Adele]{Stenvall, Anna Adele}}{02.05.1917}{12.05.1990 i Helsingfors}
  \item \jhperson{\jhname[Ragnar Edvin]{Stenvall, Ragnar Edvin}}{06.08.1919}{27.06.2004 i Jstad, begraven i Jeppo}
  \item \jhperson{\jhname[Åke Wilhelm]{Stenvall, Åke Wilhelm}}{25.05.1924}{i april 2004 i Örebro, Sverige}
  \item \jhperson{\jhname[Erik Rafael]{Stenvall, Erik Rafael}}{16.02.1928}{10.02.2012 i Bennäs (sjukhus i Jstad)}
\end{jhchildren}
Elis och Ainas omyndiga barn erhöll medels gåvobrev den 30 september 1928 av farfar Isak och farmor Anna Stenbacka, Nygård benämnda lägenhet R:nr 4:9 om 0,0390 mantal av Silvast skattehemman R:nr 4 i Jungar by, Jeppo. I gåvobrevet fanns många undantag och villkor, som berörde Isak och Annas övriga barnbarn. Elis och Aina skulle bruka och förvalta lägenheten. Elis och Aina Stenvall (Stenbacka) hade tidigare den 15.01.1920 köpt Bäckstrand benämnda skattelägenhet R:nr 4:6 om 0,0495 mantal av Silvast skattehemman R:nr 4, säljare affärsmannen Karl Henrik Hoks, Lappfjärd, som undantog vissa parceller av 4:6, som före försäljningen utgjorde 0,0699 mantal. Försäljaren Hoks hade rätt att avverka skogen inom 2 år så långt lagen medgav. På vissa områden och tomter, som ingick i R:nr 4:6, gjordes 28.09.1923 köpebrev där Elis och Aina Stenvall stod som säljare. Affärsmannen Hoks hade den 13.01.1920 köpt Bäckstrand 4:6 om 0,0495 mantal av säljarna Johan och Sofia Modèn. Den 05.10.1895 hade Modèn köpt 19/384 mantal av Silvast skattehemman 4 av Isak och Lovisa Wiklund.

Den 01.08.1938 köpte Elis och Aina av lägenhet Norrgård R:nr 4:49  0,0131 mantal. Paul Stenvall köpte 18.06.1949 stomfastigheten om 0,056 mantal, samt ett tidigare sålt område om 0,024 mantal. Den 15.09.1931 köpte Elis och Aina Jinjärvi II skattelägenhet R:nr 12:17 i Lassila by, Jeppo om 8,49 ha.

År 1936 byggde familjen en ny ladugård. Elis och Aina hade 3 hästar och 10 mjölkkor samt kalvar och får och dessutom arbetade Elis utanför jordbruket. Som ung pojke började han som bromsare på järnvägståg med sand. Sanden togs från Gunnar i Jeppo till bansträckan mellan Seinäjoki-Gamlakarleby. På grund av nedsatt syn kunde han inte bli stationskarl som sina bröder. Senare under några år hade han hand om posttransporten från Jeppo till Oravais med häst. En julafton var posttåget försenat och då han återvände hem med Pålle kl. 3 på juldagsmorgonen visade termometern minus 36 grader.  Efter posttransportåren ägnade sig Elis åt lossnings- och lastningsarbeten av bland annat konstgödsel och hö vid Jeppo station. Det var ett tungt arbete och ibland fick han hjälp av sin hustru Aina och sönerna när de blev äldre.
Förutom Elis med familj bodde i huset bror Edvin med familj 1914--\allowbreak 1923 samt från 1922 Elis mor Anna och far Isak, se följande sidor.

Elis och Aina Stenvall flyttade i medlet av juni 1955 till deras år 1946 nybyggda bostad på Mellanåkern vid älvstranden, som var en del av Nygård 4:9. Aina \textdied 24.05.1961  ---  Elis \textdied 19.04.1968.


%%%
% [occupant] Stenbacka
%
\jhoccupant{Stenbacka}{\jhname[Isak]{Stenbacka, Isak} \& \jhname[Anna]{Stenbacka, Anna}}{1908--\allowbreak 1928}
Banvakt Isak Jakobsson Stenbacka och hans hustru Anna Jakobsdotter Stenbacka köpte 09.12.1907 med tillträde 01.01.1908 skattelägenhet Nygård 4:9. Säljare Fredrik och Ellen Thors. De erhöll lagfart den 20.04.1909 på 5/128 mantal = 0,0390 mantal av Silvast skattehemman nr 4 i Jungar by. Vid lantmäteri- och klyvningsförrättning i slutet av 1800- och i början av 1900-talet av Silvast hemman nr 4 bildades av skattelägenheterna 4:1, 4:3, och 4:4  åtta nya skattelägenheter med R:nr 4:6 – 4:13. De av Statens Järnvägar år 1897 exproprierade 5,21 ha, med R:nr 4:5 har inte påverkat mantalen, som införts i jordregistret 15.02.1915. Stenbackas 0,0390 mantal blev skattelägenhet Nygård R:nr 4:9 i Jungar by.

Isak Stenbacka föddes 18.01.1859 som femte barn på Stenbacka hemman i Överjeppo by. Hans mor Anna Sofia Elenius var sondotter till kaplan Thomas Elenius, anfader till släkten, Elenius – Jungarå, skildrad i en släktbok av prosten Johannes Kronlund. Isak Stenbacka var endast 2 år gammal när hans far, bonden på Stenbacka hemman, Jakob Simonsson Stenbacka dog 1860 endast 36 år gammal. Isaks mor Anna Sofia ingick 2 år senare äktenskap med Mats Sarelin (född Göransson i Kauhava). Isak Stenbacka hade 4 äldre syskon och fick 4 yngre halvsyskon. Hans mor Anna Sofia blev mor till 9 barn.

Isak hade ingen lätt barndom. Det var dåligt med kläder och mat. Tidvis bodde han hos äldsta brodern Janne. När Isak blev 18 år fick han följa med sin äldre bror Jakob i järnvägsbyggnadsarbeten. År 1886 gifte han sig med butiksbiträdet på Stenbacka, Anna Peth, \textborn 25.08.1850 i Nykarleby. Isak Stenbacka fick tjänst som banvakt på Tapelbacken mellan Jeppo och Kovjoki stationer. I en arbetsolycka 1922 fick han benen i kläm och blev förlamad. Isak och Anna Stenbacka flyttade då från vaktstugan på Tapelbacken till en del av huset, som de byggt på den av dem 1908 inköpta lägenheten Nygård 4:9 av Silvast skattehemman. Stugan var troligen timrad på Ruotsala för försäljning. Deras son Elis bodde med familj i den södra delen av huset och skötte jordbruket. Sonen Edvin, som bodde med familj i den norra lokalen flyttade 1923 till Kronoby.

Den 30.09.1928 upprättade Isak och Anna Stenbacka ett i detalj i 15 punkter uppgjort gåvobrev, där de överlät all sin egendom till alla sina barnbarn.  Bl.a. delades bostadshuset i tre delar. Elis' 6 barn erhöll Nygård lägenhet med undantag av vissa områden, lagfart 21.09.1931. Sonen Elis utsågs att bruka och förvalta lägenheten. Sonsonen Paul har senare av kusinerna inlöst byggnaderna och vissa parceller och 1951 köpte han syskonens andel av Nygård R:nr 4:9.
Isak och Anna Stenbacka fick 4 barn, som föddes i vaktstugan på Tapelbacken.
\begin{jhchildren}
  \item \jhperson{\jhname[Ester Sofia]{Stenbacka, Ester Sofia}}{12.09.1886}{31.01.1962 i Jeppo}
  \item \jhperson{\jhname[Isak Einar]{Stenbacka, Isak Einar}}{18.04.1889}{23.04.1969 i H:fors}
  \item \jhperson{\jhbold{\jhname[Elis]{Stenbacka, Elis}} Jakob}{18.04.1889}{19.04.1968 i Jeppo}
  \item \jhperson{\jhname[Edvin Johannes]{Stenbacka, Edvin Johannes}}{07.06.1891}{23.05.1950 i Jeppo}
\end{jhchildren}
Ester blev småskollärarinna. Hon gifte sig med Vilhelm Vikström, banvakt på Tapelbacken och senare vid Gunnar. På Tapelbacken föddes tre flickor; Else, Anna och Jenny. Sonen Lennart föddes vid Gunnar vaktstuga.
Einar och Elis var tvillingar, som synes. Einar flyttade med sin familj, fru och 3 barn från Kronoby till Helsingfors med Åkerblom-rörelsen. Den yngsta sonen föddes i Helsingfors. Edvin blev stationskarl och arbetade i Jeppo och Kronoby. Edvin bodde med sin familj i norra köket och kammaren då han arbetade i Jeppo 1910--\allowbreak 1923 och senare som pensionär 1945--\allowbreak 1950. Edvin gifte sig 1914 med Maria Angelina, född Pott 29.10.1891 i Pensala by.
\begin{jhchildren}
  \item \jhperson{\jhname[Gerda Maria]{Stenbacka, Gerda Maria}}{13.02.1915}{21.04.1969, Karleby}
  \item \jhperson{\jhname[Anna Teresia]{Stenbacka, Anna Teresia}}{14.12.1916}{12.04.1999, Kronoby}
  \item \jhperson{\jhname[Erik Edvin]{Stenbacka, Erik Edvin}}{27.07.1918}{03.03.1919, Jeppo}
  \item \jhperson{\jhname[Erik Edvin]{Stenbacka, Erik Edvin}}{17.09.1921}{07.06.2002, Karis}
\end{jhchildren}
Familjen flyttade 1923 till Kronoby.
Gerda gifte sig med byggmästare Bror Snellman född 23.02.1913 i Kållby. Han har tjänstgjort som stadsbyggmästare i Gamlakarleby. Anna gifte sig med Valdemar Aspfors född 16.07.1916 i Terjärv. Han har arbetat inom butiksrörelsen i Nedervetil, Kronoby och Esse. Erik utbildade sig till konduktör. Han har arbetat i Gamlakarleby, Bennäs och flyttade 1963 till Karis. Efter att Edvin med familj flyttat till Kronoby blev norra lokalen Isak och Annas hem. Elis och Aina tog hand om Isak, som var förlamad.

Anna Stenbacka \textdied 18.01.1929  ---  Isak Stenbacka \textdied 04.04.1936 i sitt hem på Silvast.



%%%
% [house] Svinvallen II
%
\jhhouse{Svinvallen II}{21-0}{Silvast}{5}{86}

Styckad av Fors/Silvast samfällighet

\jhhousepic{049-05581.jpg}{Karl-erik Haglund}

%%%
% [occupant] Haglund
%
\jhoccupant{Haglund}{\jhname[Karl-erik]{Haglund, Karl-erik}}{2003--}
Trafikant Karl-erik Haglund, \textborn 30.05 1951 på Skog hemman och uppvuxen på fastigheten Villa 2:38 på Romar hemman, köpte 30.06.2003 Svinvallen II 21-0 på Silvast samfällighet. Säljare Artur och Lempi Pelmas dödsbo. Karl-erik är singel och har under många år kört långtradare på utrikesrutter i Europa. Karl-erik sjukpensionerades 2012. Hans barndomstid och familj skildras under fastighet Villa 2:38, Nylandsvägen 51, nr 30.


%%%
% [occupant] Pelmas
%
\jhoccupant{Pelmas}{\jhname[Arthur]{Pelmas, Arthur} \& \jhname[Lempi]{Pelmas, Lempi}}{1950--\allowbreak 2003}
År 1950 köpte Artur och Lempi Pelmas av Johan Blinks dödsbo ovannämnda fastighet med en liten bostadsbyggnad och ett mindre uthus. Artur Herman Pelmas, \textborn 02.02.1919 i Kuoppala by i Alahärmä och hans hustru Lempi Katarina, född Kennola, \textborn 03.05.1914 på Grötas hemman. Artur arbetade efter kriget på SJ:s vedplan och Lempi på skolköket och internatet i Finska folkskolan efter 2 år på Kiitola 	Pälsberederi.
\begin{jhchildren}
  \item \jhperson{\jhname[Sulo Artturi]{Pelmas, Sulo Artturi}}{08.07.1937}{}
  \item \jhperson{\jhname[Karin Margareta]{Pelmas, Karin Margareta}}{08.10.1938}{}
  \item \jhperson{\jhname[Elsbet]{Pelmas, Elsbet}}{15.09.1942}{}
  \item \jhperson{\jhname[Sulve Marjatta]{Pelmas, Sulve Marjatta}}{30.12.1943}{}
  \item \jhperson{\jhname[Ritva Karita]{Pelmas, Ritva Karita}}{02.03.1953}{}
\end{jhchildren}
Familjen Pelmas hade tidigare hyrt bostäder, Smulters på Dahlbo 4:127, nr 77A, där Sulo och Karin är födda och Arthur grävde då diken, Lågbacka 4:221, nr 352, där Elsbet föddes och Klings hus 4:181, nr 323, där Marjatta föddes, och 1946 till Björkbacka 4:38, nr 367. De tre sista husen är rivna. Endast Ritva är född i deras eget bostadshus. När kriget 1939 bröt ut fick Arthur arbete på krigsmaterialfabriken i Yxpila och familjen bodde då där till 1941, då han fick arbete som portvakt vid patronfabriken på Kiitola.

Sulo var i tonåren en god idrottare. Han emigrerade till Sverige 1962. Karin (Greta) gift Kalijärvi, se nr 29, fick sitt första arbete på Kiitola pälsfabrik och därefter arbetade hon 1957--\allowbreak 1961 på Keppo minkfarm och som postutdelare 1962--\allowbreak 1981 på Silvast området och 1982--\allowbreak 1991 i Nykarleby centrum, 1991 gick hon i pension. Elsbet, gift Pellikka, reste till Sverige, där hon arbetat som förman på en konfektionsfirma i Borås. Marjatta gift Högbjörk på Finskas, har arbetat på Jeppo Potatis Ab. Ritva är utbildad sjukskötare, gift Peltola i Voltti. Hon arbetar på Seinäjoki Centralsjukhus. I dag frånskild och bor i Lappo.

Arthur Pelmas var en händig mångsysslare och år 1967 förstorades och grundrenoverades bostadshuset, bl.a. med stockar från Klings hus. Även uthuset renoverades och fick en tillbyggnad. Efter lantmäteriförrättning på 1960-talet av tomterna på Silvast/Fors samfälligheter erhölls lagfart på fastigheten Svinvallen II 20-0.

Arthur Pelmas \textdied 13.05.1996  ---  Lempi Pelmas \textdied 01.10.2002


%%%
% [occupant] Blink
%
\jhoccupant{Blink}{\jhname[Johan]{Blink, Johan}}{1921--\allowbreak 1949}
Johan Eskil Andersson Blink, \textborn 01.05.1874 i Jeppo, och h. h. Sanna Kajsa Eriksdotter, \textborn 11.04.1869 i Nykarleby, bosatte sig i huset när de gifte sig. Även Johans mor bodde troligen med dem. Enligt torparlagen 1918 godkände Legonämnden i Jeppo den 11.11.1921 Johan Blinks torparkontrakt för inregistrering i Häradsrätten.
\begin{jhchildren}
  \item \jhperson{\jhname[Karl Viktor]{Blink, Karl Viktor}}{23.12.1899}{}
  \item \jhperson{\jhname[Anders Valfrid]{Blink, Anders Valfrid}}{26.07.1904}{}
  \item \jhperson{\jhname[Ida Linnéa]{Blink, Ida Linnéa}}{26.11.1906}{}
  \item \jhperson{\jhname[Fanny Maria]{Blink, Fanny Maria}}{08.02.1909}{}
  \item \jhperson{\jhname[Oscar Evert]{Blink, Oscar Evert}}{26.01.1911}{}
\end{jhchildren}
Oscar blev målare och gifte sig med Maria, född Filman, \textborn 13.02.1911 i Alahärmä. I början av 1940-talet köpte de ett hus på Holm hemman. När de äldre flicorna föddes, Siri 1939 och Eira 1940, bodde familjen en tid med Sanna. Johan föddes 1944 och Eva 1949 på Holmen. Oscar dog 25.01.1974 och brodern Valfrid 1999. De är begravna i Jeppo.

Johan Blink \textdied 18.09.1938  ---  Sanna \textdied 15.06.1949.


%%%
% [occupant] Blink
%
\jhoccupant{Blink}{\jhname[Anders]{Blink, Anders} \& \jhname[Greta]{Blink, Greta}}{1890--\allowbreak 1921}
År 1890 flyttade Anders Simonsson Blink, \textborn 12.03.1835 på Jungarå, och hans hustru Gerda Johansdotter, \textborn 20.05.1844 på Romar, till Fors hemman.
\begin{jhchildren}
  \item \jhperson{\jhname[Johan Eskil]{Blink, Johan Eskil}}{01.05.1874 i Jeppo}{}
  \item \jhperson{\jhname[Anna Sanna]{Blink, Anna Sanna}}{24.08.1880 i Vasa}{}
  \item \jhperson{\jhname[Lisa Adolfina]{Blink, Lisa Adolfina}}{31.12.1882 i Jeppo}{}
  \item \jhperson{\jhname[Ida Maria]{Blink, Ida Maria}}{25.04.1884 i Jeppo}{}
\end{jhchildren}
Bostadshuset och ett indre uthus byggdes troligen på 1890-talet av Anders och Greta Blink på Silvast/Fors Samfällighet, numera Svinvallen	II 21-0.

Anders \textdied 25.02.1894  ---  Greta \textdied 31.01.1921.


%%%
% [occupant] Samfälligheten
%
\jhoccupant{Samfälligheten}{\jhname[F/S]{Samfälligheten, F/S}}{1934--\allowbreak 1955}
På Fors/Silvast hemmans samfälligheter, benämnd Svinvallen, fanns ett hus på vägkanten som 1934 inreddes för \jhbold{Lantmannagillets klövernötare}. På grund av olägenheter med agnar, som klövernötaren spred omkring sig, flyttades den 1955 till sågtomten på Fors hemman, se nr 110.
