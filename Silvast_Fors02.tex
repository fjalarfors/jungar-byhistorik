

KARTBLAD nr 5 placeras hit --------------->>>


\jhhouse{Rönnlandet}{4:168}{Silvast}{5}{53}

\jhoccupant{Elenius}{Kim \& Virpi}{1993 -}

Kim Elenius, \textborn 1968, gifte sig 1998 med Virpi Kalliosaari, \textborn 1974 i Jeppo. Kim har arbetat med fastighetsservice och är nu anställd på KWH Mirka.  Han utför också bilreparationer via egen firma . Virpi har i 10 års tid varit kock på ABC-servicestation i Oravais, men är sedan något år tillbaka anställd på Jeppo daghem.

\jhhousepic{025-05556.jpg}{}{}

\begin{jhchildren}
  \item \jhperson{Henri}{1997}{}
  \item \jhperson{Rasmus}{2007}{}
\end{jhchildren}


\jhoccupant{Nygård}{Hilding \& Ellen}{1962-1993}

Paul Hilding Nygård, \textborn 24.07.1929 på Böös, gifte sig 1956 med Ellen Elisabet Stenbacka, \textborn 18.10.1933 i Oravais. Hilding utbildade sig till järnvägsbokhållare. Han har varit anställd på flera av regionens järnvägsstationer, så också i Jeppo. Han har bl.a. skött ankommande och avgående trafik utöver andra  funktioner på stationerna.

Hilding har hela livet varit mycket intresserad av idrott och varit styrelseordförande i Jeppo Idrottsförening. Han har deltagit i flera Vasalopp och deltog också i JIF:s klubbkamper på 1950-talet.

Ellen har varit lärare och startade sin gärning i Jeppo vid Jungar folkskola år 1956 fram tills dess skolorna i Jeppo centraliserades 1966. Därefter tjänstgjorde hon en tid, 1970-71, i den nya centrumskolan. Hon har under tiden i Jeppo varit aktiv i både nykterhets- och marthaföreningarna samt suttit i församlingens kyrkoråd. Småningom flyttade makarna till Oravais och huset som byggts 1962 såldes 1993.

Hilding \textdied 15.12.2014


\jhhouse{Strand}{4:217}{Silvast}{5}{54}

\jhoccupant{Strand (Dödsbo)}{Einar \& Agneta}{1935 -}

Einar Isak Strand, \textborn 22.02.1912 på Silvast, gifte sig 17.02.1935 med Maria Agneta Stolt, \textborn 19.03.1913 på Hilli. Tomten utstyckades från Einars hemgård 1935 och huset byggdes.

\jhhousepic{026-05557.jpg}{}

Familjen försörjde sej på det lilla jordbruket och med diversearbete. Einar blev som de flesta andra inkallad till kriget där han utmärkte sej med att visa avsaknad av rädsla. Jordbruket avvecklades småningom efter kriget och med expansionen av pälsdjursfarmningen på Keppo fick bägge arbete där. Han hjälpte också tidvis till vid Varma-butiken.

Einar sjukpensionerades och Agneta fick arbete på Mirka sedan dess verksamhet startat på Kiitola. Hon arbetade senare som städerska på Jeppo Sparbank.
\begin{jhchildren}
  \item \jhperson{Maj-Britt}{30.06.1935}{}
  \item \jhperson{Volmar}{25.01.1940}{}
  \item \jhperson{Sven}{31.10.1942}{}
  \item \jhperson{Nils}{13.07.1948}{}
\end{jhchildren}

Einar \textdied 10.02.1968  --  Agneta \textdied 26.03.2008

Dödsboet äger fortsättningsvis fastigheten.


\jhhouse{Lugnbo}{4:45}{Silvast}{5}{55}

\jhoccupant{Norrgård}{Erik \& Denice}{1969 -}

Erik Norrgård, \textborn 27.05.1936, gifte sig med Denice Back, \textborn 29.11.1945. De hyrde fastigheten 1969 av Karin Sjöbloms arvingar för att slutligen 1972 köpa den. Familjen bodde på fastigheten fram till 01.02.1995. Huset revs 22-25 febr. 2015.

Erik arbetade redan i unga år som grävmaskinsförare och övergick småningom i Andelsringens tjänst. Han arbetade tillsammans med Edvin Jungell på magasinet vid järnvägsstationen, ett arbete som krävde mycket lyftande.  Senare fick han anställning vid Prevex i Nykarleby. Han var en av dem som startade hjälporganisationen ``Polen källan'' som, efter legaliseringen av ``solidaritetrörelsen'' och Sovjetunionens sammanbrott, påbörjade hjälpsändningar till i huvudsak nordöstra Polen. Den intensivaste verksamheten skedde under den första delen av 1990-talet. Erik var i detta sammanhang en drivande kraft och fick som ett tack för sitt engagemang en utmärkelse av Polens ambassadör i Finland. Ambassadören kom med ambassadens tjänstebil till Denices och Eriks sommarstuga i Pensala för att överlämna ``prenikan''. För allmänheten i regionen var Erik i radion känd som ``Jepo-Erik'' i flera av de publika sändningarna.

\jhhousepic{KarinS.jpg}{}

Denice kom som kontrollassistent till Jeppo. Efter giftermålet bodde familjen på Ruotsala, därifrån de flyttade 1969. Hon har haft arbete vid Sundells bageri och under lång tid varit butiks- och kassabiträde inom Andelsringen och via olika affärsnamnbyten varit S-kedjan trogen.
Barn: Katarina, \textborn 09.09.1966

Erik \textdied 28.02.2013


\jhoccupant{Sjöblom}{Karin}{(1929)1952-1969}

Karin (eg. Katarina) Sjöblom, \textborn 12.08.1885 i Jeppo, men hon brukade säga: ``Jag är född i Vasa, men bosatt i Jeppo''.  Därför blev hon ibland kallad för ``Vasa-Karin''. Hon var småskollärarinna på Jungar skola åren 1910-1952 där hon också bodde. Hon har vid sidan av lärarparet Thors utfört den längsta lärargärningen i Jeppo.

Från 1952 bodde hon stadigvarande på fastigheten, som under tiden efter föräldrarnas död varit uthyrd till bl.a. Gunnar Romar. Också Georg och Vera Romar bodde här efter giftermålet 1941 fram till dess det egna huset stod klart 1950 vid Romarbäcken. Efter att Karin flyttat in hade hon också underhyresgäster på övre våningen. Bl.a. bodde Anita Ström (gift med Brian Lavast) där 1958-1961.

Karin var ogift och dog 25.07.1969.


\jhoccupant{Sjöblom}{Jakob \& Anna-Maria}{1870-1929}

Jakob Sjöblom, \textborn 16.09.1849, gifte sig med Anna Maria Gustavsdr., \textborn 12.12.1844 i Nykarleby. Under deras tid byggdes huset som senare ombyggdes med s.k. ``brutet tak''.
\begin{jhchildren}
  \item \jhperson{Anna Sanna}{12.04.1874}{}
  \item \jhperson{Johan Jakob}{21.09.1876}{}
  \item \jhperson{Edla Maria}{27.06.1879}{}
  \item \jhperson{Ida}{11.10.1882}{05.04.1969 i USA}
  \item \jhperson{\jhbold{Karin(Katarina)}}{12.08.1885}{25.07.1969 i Jeppo}
  \item \jhperson{Selma}{09.01.1888}{27.09.1978 i Jeppo}
  \item \jhperson{Matts Vilhelm}{30.12.1889}{}
\end{jhchildren}

Ida och Selma reste till Amerika. Ida stannade, Selma återvände.


\jhhouse{Silfvast}{4:169}{Silvast}{5}{56, 56a-b}

\jhoccupant{Silfvast}{Lars \& Inger}{1974 -}

Lars, \textborn 27.08.1945 i Gamlakarleby, gift 24.06.1972 med Inger Söderström, \textborn 26.03.1951 i Karleby, Storby.
\begin{jhchildren}
  \item \jhperson{Ronny}{16.04.1974}{}
  \item \jhperson{Eva-Lotta}{18.04.1978}{}
  \item \jhperson{Robert}{29.12.1986}{}
\end{jhchildren}

Lars har besökt Evangeliska Folkhögskolan i Vasa och har hämtat lantbruksutbildning vid Lannäslund i Jakobstad, där Inger samtidigt gick på husmorslinjen. Lars och Inger övertog hemmanet 1974 efter att Vilhelm Silfvast plötsligt avlidit. En kort tid fortsatte jordbruket med djurhållning innan man definitivt sadlade om till specialiserad potatisodling, som  fortsatte under många år. Sedan år 2013  är den odlade jorden utarrenderad.

\jhhousepic{027-05559.jpg}{}

Utöver arbetet på den egna lägenheten har Lars under många år kört lastbil oftast i utrikestrafik. Inger har också utfört kompletterande arbete vid sidan av jordbruket och har haft fast tjänst som närvårdare inom Nykarleby stad. Inger gick i pension 2014.


\jhoccupant{Silfvast 4:113}{Vilhelm \& Ellen}{1936-1974}

Vilhelm, \textborn 16.09.1913, gift 19.03.1939 med Ellen Gustafsson, \textborn 06.02.1912 med efternamnet Rundt. Ellen var tidigare gift med Axel Silfvast, \textborn 13.12.1908. Han dog 14 maj 1935 av hjärnhinneinflammation. Paret var barnlöst. Ellen gifte om sig 1939 med hans yngre bror, Vilhelm (se ovan). Axels och Vilhelms äldsta syster Anna Sofia, som 1917 gift sig med Johannes Nordman, hade dött 1934. Hennes dotter Rakel flyttade då hit till sin mormor Anna Lovisas hem och bodde här till 1945, då hon gifte sig med Hugo Nybyggar.

Vilhelm ägde nu tillsammans med sin mor Anna Lovisa och Ellen hemmanet, som fram till dess var ett dödsbo. Anna Lovisa sålde år 1936 sin andel av hemmanet till Vilhelm. I och med att Vilhelm och Ellen gifte sig 1939 kom de att tillsammans äga hela hemmanet. Under kriget tjänstgjorde Vilhelm på hemmafronten och var stationerad bl.a. i Långåminne och Ekenäs. Under ``Lapplandskriget'' bodde evakuerade från Kemijärvi en kort tid  i huset.

År 1951 byggdes ny ladugård och ekonomiebyggnad. En ny bostadsbyggnad på fastigheten började byggas 1955, varefter den gamla bostaden, som stod närmare älven men vägg i vägg med den nya, revs.
Barn: \jhbold{Lars} , adoptivson, \textborn 27.08.1945. Lars är son till Vilhelms systerdotter Ruth Savonen och adopterad av Vilhelm och Ellen år 1946.

Vilhelm \textdied 03.04.1974  --  Ellen \textdied 01.08.1975

\jhhousepic{SilfvastL.jpeg}{Fr.v. Johannes Strand(Silfvast), Johanna Löv, Anna-Sofia, Zaida, Vilhelm d.ä., Anna-Lovisa, Isak.}


\jhoccupant{Silfvast 4:73}{Isak \& Anna-Lovisa}{1902-1936}

Isak, *11.01.1867, gift 18.01.1887 med Anna Lovisa Eliasdotter Forss *07.06.1867.
\begin{jhchildren}
  \item \jhperson{Elias Vilhelm}{1888}{}
  \item \jhperson{Johannes}{1890}{1930}, antog namnet Strand
  \item \jhperson{Anna Sofia}{1892}{1934}
  \item \jhperson{Zaida}{1897}{1963}
  \item \jhperson{Isak Vilhelm}{1903}{1903}
  \item \jhperson{Isak Vilhelm}{1906}{1911}
  \item \jhperson{Axel Joel}{1908}{1935}
  \item \jhperson{Lea}{1910}{1976}
  \item \jhperson{\jhbold{Isak Vilhelm}}{1913}{1974}
\end{jhchildren}

Som så många andra reste Isak till Sydafrika för arbete i gruvorna och som så många andra ådrog han sig där ``minarsjukan''. Plågad av sin sjukdom dog han 1917. Han efterlämnade änkan Anna Lovisa, som tillsammans med hemmavarande barn skötte jordbruket med sina 7 kor.

Isak \textdied 15.12. 1917  --  Anna Lovisa \textdied 09.03.1941


\jhoccupant{Silfvast 4:35}{Henrik \& Anna}{1865-1902}

Henrik Gustaf Johansson, \textborn 20.12.1834, från Pedersöre, Lepplax, gift 29.01.1859 med Anna Isaksdotter Silfvast, \textborn 14.09.1839.
\begin{jhchildren}
  \item \jhperson{Johan}{24.03.1859}{08.06.1910}
  \item \jhperson{Isak}{11.01.1867}{15.12.1917}
  \item \jhperson{Sanna}{25.04.1871}{30.01.1942}
  \item \jhperson{Johanna}{13.06.1874}{14.11.1956}
\end{jhchildren}

Anna \textdied 02.06.1913  --  Henrik \textdied 12.04.1909

Silvast hemman nr 356, 356a, 356b, 356c, och 356d.
356:   Här stod den gamla bostadsbyggnaden uppförd i medlet av 1800- talet. Den revs av Vilhelm Silfvast 1956.
356a. Här stod den gamla ladugården. Den revs 1952 för att bereda plats för den nya ladugården.
356b. Häststallet utgjordes av denna separata byggnad. Revs 1952.
356c. Här stod Johannes och Ida Maria Strands bostadshus. Johannes var född Silfvast, men ändrade efternamnet till Strand.
356d. Här stod Oskar och Hellin Strengells bostadshus, byggt i slutet på 1940-talet. Revs i slutet av 1970-talet.



\jhhouse{Strand}{4:214, tidigare 4:17}{Silvast}{5}{354a}

\jhoccupant{Strengell}{Oskar \& Hellin}{1948-1983}

Oskar Evald Strengell, \textborn 21.11.1910 på Jungar, gifte sig 03.08.1941 med Hellin Maria Strand, \textborn 05.03.1922 på Silvast. Makarna byggde huset i slutet av 1940-talet alldeles på gränsen till Handelslagets tomt. Innan byggnaden restes revs det gamla huset som var Hellins barndomshem. Efter kriget fick Oskar tjänst på Haldin \& Rose som chaufför och också som montör på verkstaden. Familjen bodde en tid i Jakobstad innan den flyttade till Jeppo och byggde huset. Rutten som kördes var Jeppo-Voltti-Jeppo. Bussen stod parkerad på gårdsplanen mellan turerna. Under tiden i Jeppo hade familjen två kor som Hellin skötte om. Oskar hade planer på att öppna en verkstad i anslutning till fastigheten, men nekades tillstånd. Familjen flyttade tillbaka till Jakobstad 1954 och huset hyrdes ut.
\begin{jhchildren}
  \item \jhperson{Boris}{25.12.1942}{}
  \item \jhperson{Berit}{18.07.1945}{}
\end{jhchildren}

\jhhousepic{Strand 4-214.jpg}{Fr.v. Lars o. Ellen Silvast, Terese Tiimalainen m. dotter Anna-Maija, Vilhelm S. med det aktuella huset i bakgrunden}

Huset hyrdes av bl.a.:
\begin{enumerate}
  \item Paul och Gretel Nylind  1954-1957
  \item Ingmar och Margit Sandvik 1957-1960
  \item Ellen Orrholm 1960-1963
  \item Jorma och Karin Kalijärvi 1963
  \item Adolfina och Juhani Rintala
  \item Juho Kustaa Aaltonen och Lydia Maria Jokinen
  \item Reijo Krooks   …..1983
\end{enumerate}

Andelsringen köpte fastigheten år 1983 och huset revs av Ernfrid och Sven Kronqvist.


---> R:no 4:17 av Silvast hemman, nr \jhbold{354b}


\jhoccupant{Strand}{Johannes \& Ida}{1908-1947}

Johannes Silfvast, från 1922 med efternamnet Strand, \textborn 17.03.1890, gifte sig 12.04.1908 med Ida Maria Isaksdr. Liljeqvist, \textborn 20.10.1885. Johannes övertog denna del av hemmanets byggnader och en del jord i samband med giftermålet. Efter att barnen fötts reste han till USA och återvände inte. Han dog där 01.03.1930. Ida dog här hemma den 22.02.1947.
\begin{jhchildren}
  \item \jhperson{Sven}{08.10.1908}{04.05.1961 i Canada}
  \item \jhperson{Svea}{13.06.1910}{}
  \item \jhperson{Einar}{22.02.1912}{}
  \item \jhperson{Elna}{20.11.1913}{1926}
  \item \jhperson{Henry}{07.07.1920}{23.06.1944}, stupade
  \item \jhperson{\jhbold{Hellin}}{05.03.1922}{06.07.2008 i Jakobstad}
\end{jhchildren}


\jhhouse{Lindström}{4:215 (4:7)}{Silvast}{5}{57}

\jhoccupant{J \& S}{Fastigheter}{2014 -}

Den 31 mars 2014 köpte bröderna Jonas och Sebastian Cederström (se Ruotsala nr 20) fastigheten av Oy Hankkija, som ägs av Danish Agro Holding A/S. Deras firma är specialiserad på VVS-installationer och den tidigare butiksfastigheten har inretts och utnyttjas för företagets behov. De bostäder som finns på övre våningen hyrs ut. Likaså fortsätter Café Funkis sin verksamhet i den tidigare lokalen, nu med Jasmin Aalto som hyresgäst sedan 2016.

\jhoccupant{Hankkija}{m.fl.}{1985-2014}

Efter en sällsynt turbulent tid på ägarfronten såldes fastigheten som ovan nämnts till J \& S Fastigheter 2014. Den egentliga affärsverksamheten hade året dessförinnan upphört i praktiken i Agrimarkets namn, ägt av Oy Hankkija. Agrimarket hade uppstått efter att centrallaget Hankkija gått i konkurs 1992 och ett nybildat Oy Hankkija bildats som under marknadsnamnet Agrimarket fortsatte lantbrukshandeln i 73 butiker av olika storlek, varav butiken i Jeppo var en. Butiken i Jeppo var tidvis bland de bästa vad lönsamhet beträffar, men andra överväganden styrde besluten.

\jhhousepic{030-05561.jpg}{}

År 1997 uppstod problem med bensinförsäljningen då cisternerna blev granskade. Den 25 sept. utförde Pieksämäen Pensseli granskningen och cisternerna blev utdömda. Ernfrid Kronqvist och Ove Strand beställde på eget bevåg en ny 16 m³ cistern med mellanvägg, grävde ner den och fyllde de gamla med sand efter rengöring. Trots att man var medveten om att investeringen inte skulle bli lönsam, räddades bränslefrågan för tillfället. Arbetet var klart 28 okt.

År 2013 övertar Danish Agro Holding A/S 60\% av aktierna i Oy Hankkija  och nedmonteringen börjar. Protester och telefonsamtal från Jeppo till Danmark har ingen verkan. Besluten fattades på annat håll och effekten har blivit att mycket få lantbrukare fortsatt sin handel med de nya ägarna, som flytt orten.

\jhoccupant{Botnica-Waasko}{KPO – Sale}{1985-2006}

Tiden före 1992 hade också varit omvälvande. 1987 hade Waasko handelslag sträckt ut sina tentakler och övertagit Handelslaget Botnica som uppstått genom en fusion mellan Andelsringen, Pedersöre Hlg, Esse Hlg och Terjärv Hlg år 1985 i ett försök att konsolidera den finlandssvenska  andelsrörelsens ställning på marknaden. Tveksamhet inför processen präglade Andelsringens beslutande organ och det krävdes 4 beslut innan samgångsplanerna godkändes och Botnica bildades fr.o.m. 1985. Då hade lagerreserven redan utnyttjats. Det lyckades inte trots samgången och KPO övertar småningom dagligvaruhandeln som i namn av Sale-kedjan fortsätter verksamheten i fastigheten i Jeppo till år 2006, varefter Salebutiken flyttar till korsningen riksväg 19/Pensalavägen.

\jhhousepic{Jeppo centrum 1960.jpg}{Museiverket, foto M. Poutavaara}

Under Waaskos tid var Håkan Smeds Vd för handelslaget och flera saneringsåtgärder vidtogs för att stärka verksamheten. Handeln i Jeppo uppges gå bra 1987/88 efter sammanslagningen av lantbrukshandeln mellan Hankkija och SOK. Omsättningen stiger men lönsamheten försämras på nytt. Textilsidan läggs ner och de friställda utrymmena hyrs 1988 ut till Föreningsbanken i Finland och 1989 såldes bl.a. spannmålsmagasinet till nystartade Jeppo Food.

Ny hyresgäst sedan både Posten och Föreningsbanken flyttat bort blir Gunilla Nyman som år 1990 här öppnar en textil/ blomsteraffär. som år ….......övertas av Regina Dahlström, gift Kujala.Den fortsätter till år............,när den läggs ner.

\jhoccupant{Andelsringen}{x}{1960-1985}

Den 3 juni 1959 fattade stämman för Nykarleby Handelslag  beslutet om att fusionera in i Jeppo-Oravais Handelslag. Fusionen trädde ikraft den 1 januari 1960 och efter en namntävling där ca 500 namnförslag lämnades in, stannade man för namnet \jhbold{Andelsringen}.

Under 1962 startade en omfattande renovering av huvudaffären i Silvast och den 10 maj 1963 öppnades dörrarna till nyrenoverade utrymmen. Denna period specialiserades handeln allt mer och indelades i ansvarsområden. Bo-Göran Backman var ansvarig för maskinhandeln och Ingvar Karf för handeln med gödsel,kalk,utsäde,byggnadsmaterial m.m. Vattentäta skott fanns naturligtvis inte. Ernfrid Kronqvist kom småningom in i bilden och blev en trotjänare inom företaget. Paul Granholm hade kommit 1952 och ansvarade för livsmedelssidan och speciellt köttdisken. Han efterträddes av Jan-Erik Nybyggar.

Den 1 dec. 1981 flyttar \jhbold{Posten} in i Andelsringens lokaler för en tid.

\jhoccupant{Jeppo-Oravais}{Handelslag}{1927-1960}

Centrallaget för Handelslagen i Finland hade under en tid med oro följt Oravais Handelslags ekonomiska utveckling och 1926 framställde Centrallaget ett förslag om att Oravais Handelslag och Jeppo dito skulle gå samman. Vid stämmorna vid årsslutet beslöts att Oravais Handelslag skulle uppgå i Jeppo Handelslag och fr.o.m 1 januari 1927 blev det gemensamma Jeppo-Oravais Handelslag ett av de större handelslagen i svenska Österbotten. Handelslaget var mycket expansivt inom sitt område och flera filialer öppnades eller inköptes: Jungar, Karvat, Komossa, Pensala,Ytterjeppo, Ruths, och Kimo.

1930-talets globala depression var också tydligt märkbar här på lokalnivå och gjorde verksamheten problematisk för allt handlande, men i slutet av -30 talet började ett kraftigt uppsving som gjorde att styrelsen 1939 började fundera på att bygga en ny huvudaffär. Den 24 mars inköptes en tomt väster om korsningen till Stationsvägen av bonden Anders Lindén för ett pris om 70 000 mk. Ritningar skaffades från Centrallagets byggnadsavdelning, men då vinterkriget bröt ut den 30.11 stoppade processen. Först våren 1941, under mellankrigstiden, startade nu byggandet under byggmästare Georg Romars ledning. Fortsättningskriget som startade 25 juni 1941 innebar ett nytt hinder och arbetet gick i stå då männen ryckte in i armén. På något sätt lyckades ändå Georg Romar få igång byggandet igen och trots stora svårigheter  kunde bygget färdigställas under våren 1943 och vara klart för inflyttning före midsommar. Nu invigdes också Café Funkis, som fått sitt namn efter den stil som introducerats av Alvar Aalto i och med byggandet av stadsbiblioteket i Viborg.

År 1947 byggdes ett spannmålslagerhus vid Jeppo Station för att 1956 utvidgas med en magasinbyggnad i två våningar. I slutet av samma år avlider handelslagets styrelseordförande sedan 22 år, Leander Finskas. 1959 anskaffas handelslagets första butiksbil. Samma år byggdes ett nytt garage för den utökade lastbilsparken och här fick också butiksbilen rum. Garagets övre våning inrymde 4 bostadslokaler för personalen. Under denna tid har fungerat som föreståndare:
\begin{enumerate}
  \item Hugo Nyman        1923-1930
  \item Vilhelm Renvall   1931-1934
  \item Konrad Perus      1935-1938
  \item Edvin Wistbacka   1938-1942
  \item Elis Fagerholm    1943-1947
  \item Sven Häll         1947-1971 (Andelsringen från 1960)
  \item John Helsing      1971-1978
  \item Bo-Göran Backman  1978-1979
  \item Håkan Smeds       1979-1981
  \item Bengt Sandvik     1981-1984
\end{enumerate}

Styrelseordförande/förvaltningsrådsordf. från 1956 var: Runar Nyholm 1956-59 och Jarl Romar 1960-85   ---> BILD AV PERSONALEN <-----

På huvudaffärens 2:a våning finns bostäder som uthyrts till personalen under åren lopp. Här finns lokalen för föreståndaren som inhyst Elis Fagerholms familj 1943-1947, Sven Hälls familj 1947-1971, John Helsings familj 1971-1978, Håkan Smeds familj 1979-1981 och Bo-Göran Backmans familj 1978-1979 och 1981-1995. Sedan 1995 har Erik och Denice Norrgård bott i lägenheten fram till 2016.

I de andra lokalerna har bott expediter, chaufförer och annan personal genom åren, bl.a. Anita Lindholm (gift Häggstrand), Oili Broo, Gretel Andersson (gift Kronqvist), Birgitta Nylund (gift Back), Magda Simons, Bo-Göran Backman , Valdemar Johanssons familj. Paul Granholm, Volmar och Benita Strand, Magda Lindström från Oravais m.fl. m.fl......

---> \jhbold{Det gamla handelsslagshuset i vägkorsningen till Jeppo station}, nr \jhbold{357}

\jhoccupant{Jeppo Handelslag}{m.b.t.}{1880-1960}

Huset byggdes på 1880-talet av handlanden Gustav Julin. Han hade köpt timran från ett hus på Jungar och byggde nu upp det i korsningen. År 1890 förstorades tomten genom att på 50 år arrendera 4 kappland ( 600 m²) av Jakob Sjöblom och Anders Björklund, men redan 1891 hade han upphört med verksamheten. Den 26 november 1891 hölls en konstituerande stämma på Jungar folkskola i akt och mening att bilda ett handelsaktiebolag. Detta blev också fallet och man beslöt att köpa Gustav Julins gård för 3000 mk. Den skulle bli bolagets  butik  och huvudkontor. I direktionen satt John Svedberg som ordf. och medlemmar var bönderna Vilhelm Sarelin, Johan Romar, Johan Rundt och bondsonen Johan Jungar. Den 1 januari 1892 öppnades affären, men lönsamheten var inte den önskade. Julin som kvarblivit i tjänst i det nya bolaget sade nu upp sin tjänst och redan ett år efter starten såg insatskapitalet ut att vara förbrukat och den 29.12.1893 diskuterade man om bolaget skulle upplösas. Trots en tillfällig optimism beslöt man den 22 jan. 1894 att butiken var till salu och den 30 april hade handlanden Isak Liljeqvists anbud om 3025 mk godkänts.

\jhhousepic{Jeppo Hlg mbt.jpg}{}

I denna byggnad bedrev Liljeqvist handel till 28 mars 1913 då han sålde den åt Emil Ström för 8500 mk. Liljeqvist period blev 19 år.

Under tiden hade prof. Hannes Gebhards idé om andelslag börjat mogna också i Jeppo och 19 jan. 1914 hade Jeppo Handelslag bildats och startat sin verksamhet i den gamla gillesboden med J.T.Backlund som förste föreståndare. Butiken var inte varmare än en vanlig lada och den var för liten, fastän man 1915 förstorat den något med en enkel tillbyggnad av bräder.

Plötsligt infann sej en öppning i utrymmesfrågan då Emil Ström den 17 mars 1916 erbjöd sej att sälja sin handelslokal inklusive alla inventarier för 12000 mk. Erbjudandet godkändes och det hus som Gustav Julin köpt från Jungar och timrat upp på 1880-talet fick nu sin 5:e ägare. År 1921 förstorades huset västerut med en affärslokal och en skylt med texten ``Jeppo Handelslag m.b.t.'' sattes upp. Denna affärslokal skulle tjäna Jeppoborna tills den nya huvudbutiken blev klar 1943, dock med en förändring av namnet till Jeppo – Oravais Handelslag m.b.t. från och med  1927.

Efter 1943 användes byggnaden främst som lagerlokal, men också  som bostad åt bl.a. Ingvar Karfs familj , Bo-Göran Backmans familj och en tid åt Paul Granholm och Lars Lindberg.

I samband med att nytt garage skulle byggas med början 1959 avlutades denna byggnads historia, men på fotografier från år 1960 kan man ännu se den karakteristiska byggnaden. Den revs 1965.

Affärsföreståndare medan handelslaget verkade i denna fastighet har varit:
\begin{center}
  \begin{tabular}{l l l l}
    \hline
    J.T. Backlund & 08.03.1914 - 05.12.1916 & Gustav Liljeqvist & 01.01.1919 - 31.07.1923 \\
    Martin Ahlnäs & 06.12.1916 - 31.12.1916 & Hugo Nyman & 01.08.1923 - 31.12.1930 \\
    Frithjof Varg & 07.01.1917 - 05.02.1917 & Vilhelm Renvall & 01.01.1931 - 31.12.1934 \\
    A.E. Sandvik & 07.03.1917 - 15.05.1917 & Konrad Perus & 01.01.1935 - 31.05.1938 \\
    Emil Ekström & 12.06.1917 - 30.11.1918 & Edvin Wistbacka & 01.06.1938 - 31.10.1942 \\ \hline
  \end{tabular}
\end{center}

Vi finner de täta bytena på föreståndarposten anmärkningsvärda, speciellt i starten. Kanhända är de ett tecken på de svårigheter en ny handelsform stod inför, men är med säkerhet också ett bevis på det kärva klimat ett samhälle utan stora resurser levde under.

Styrelseordföranden under denna tid var:
\begin{enumerate}
  \item Johan Jungar     1914-21
  \item Johan Romar      1921-31
  \item Viktor Smulter   1932
  \item Gunnar Vadström  1933
  \item Leander Finskas  1934-56
\end{enumerate}



\jhhouse{Lindfors}{4:112}{Silvast}{5}{359}

\jhoccupant{Andelsringen}{Hlg}{1965-1980}

Andelsringen köpte fastigheten i medlet av 1960-talet och hyrdes ut till Helge Nygård (fr. Baggas) 1968-69. Efter det har huset mest använts som lager för olika produkter innan det revs på 1980-talet.

\jhoccupant{Norrgård}{Einar \& Märta}{1945-1954}

Einar Norrgård, \textborn 27.03.1915, gifte sig med Märta Liljeqvist, \textborn 07.05.1922. Han var sysselsatt inom handelslaget som butiksbiträde. Einar kom i slutet av 1939 från Munsala och bosatte sej på Back och fungerade som kontorist på Keppo innan han flyttade till Silvast. År 1942 gifte han sej med Märta och 6 dec 1945 flyttade familjen till Oravais. Han köpte fastigheten 4 aug. 1945 av Isak Lindfors sondotter Agnes Westerlund. 1953 återkom familjen denna gång från Helsinge och flyttade 1954 till Korsholm.
\begin{jhchildren}
  \item \jhperson{Gundel Alice}{1942}{}
  \item \jhperson{Gun Kerstin Elisabeth}{1947}{}
  \item \jhperson{Bjarne Gustav Johannes}{1953}{}
\end{jhchildren}

Fastigheten såldes till Andelsringen i medlet av 1960-talet, men innan dess var den uthyrd i olika perioder från att familjen Norrgård flyttat bort 1954. Bl.a: Paul och Gretel Nylinds fam. 1957-60, Ruben och Ragnborg Nygårds fam. 1961-63, Lars Finne, \textborn 14.11.1926 i Jakobstad, med modern Laina Lilius, \textborn 01.09.1908 i Jakobstad, och systern Lisbet Marianne, \textborn 16.05.1928 i Markby. Lars var butiksföreståndare och Laina var kaféinnehavare.

Ernst Pensar, \textborn 02.01.1917 i Munsala, kom 21.01.1949 från Oravais. Han var gift med Edith Josefin Nordberg, \textborn 17.02.1922. Ernst var vice Vd på handelslaget, familjen flyttade 27.12.1949 till Munsala. Köttmästaren Paul Granholms familj bodde i huset 1952-1956  (se handelslaget).

\jhoccupant{Lindfors}{Isak \& Sofia}{-1953}

Isak Jakobsson Lindfors, \textborn 08.08.1862, kom 1910 från Ytterjeppo och gifte sig med Sofia Johansdr., \textborn 08.05.1860. Isak blev senare känd som ``Billnäs-Iikka''.

På Handelslagets vårstämma 1920 godkändes inköp av Lindfors hemman och den 6 maj såldes mindre lotter av detta för 12050mk.

Isak arbetade sannolikt på stationens vedplan. Om han också varit dräng eller timmerman är oklart. Huset som rymde familjen byggdes på östra sidan om landsvägen men på västra ändan av den tomt där senare Andelsringens garagebyggnad placerades. Husgaveln syns i Museiverkets bild från 1960 och även t.v. på bilden av gamla Jeppo Handelslag m.b.t.,ovan.
\begin{jhchildren}
  \item \jhperson{Anna Lovisa}{19.10.1888}{}
  \item \jhperson{Isak Alfred}{27.09.1890}{}
  \item \jhperson{Anders Emil}{22.06.1892}{}
\end{jhchildren}

Isak Alfred Lindfors, \textborn 27.09.1890,  växte upp som det mellersta barnet i Isak och Sofias familj (se ovan). Han gifte sej med Sanna Emelia Isaksdr. Liljeqvist, \textborn 19.05.1888. Paret fick en dotter, Agnes Emelia, \textborn 21.01.1914. När Agnes var drygt ett år dog hennes mor Sanna, den 02.05.1915. Fadern Isak Alfred lämnade sin dotter i sina föräldrars vård och reste 28.10.1915 till USA. Hans kontakt med hemlandet skulle bli liten.

Agnes växte upp med sina farföräldrar; Billnäs-Iikka och Sofia. Hon gick i ``Sparvback'' skola (Jungar skola eller Sparrback skola) och visade tidigt intresse för skådespel och teater. Hon deltog i den teatergrupp som upprätthölls av Ungdomsföreningen. Vid 20 års ålder arbetade hon som expedit på Ströms butik på Stenbacken och den här kvällen 1934 skulle teatergruppen ha en föreställning. Hon hade varit med och övat och hon var inte nervös. En enda replik skulle hon ha: ``-Min far kommer hem ikväll''.

Precis innan stängningsdags kommer en kund in i butiken och berättar att det kommit en kappsäck till stationen med namnet Isak Lindfors. Den hade sänts från Amerika. ``-Du måste fara hem genast, Agnes. Kanske din pappa kommer hem!''.  Men Agnes hade ingen brådska. Ingen visste ju om kappsäcken och dess ägare följdes åt. Hon förberedde sej som vanligt inför föreställningen och deltog i den, men när hennes uppgift var över föstes hon iväg till stationen av sina vänner: ``-Snälltåget kommer ju snart!'' Hon gick iväg till stationen och stod och gömde sej mellan träden som växte i den lilla parken mellan stationshuset och godsterminalen.

Småningom anlände tåget och bromsade in. Ut vällde människor; jeppobor, munsalabor, oravaisbor och nykarlebybor, men vem var hennes pappa? Folk började försvinna från perrongen när hon såg en ensam man, som stigit ur en av de bakre tågvagnarna, komma gående längs träplattformen som hörde till godsterminalen. Han gick förbi henne där hon stod bakom ett träd och gömde sej och han såg henne inte. Litet på avstånd följde hon efter i skydd av träden och hon såg hur mannen gick fram till stinsen Hilding Hästbacka och började prata med denne. Nu vågade hon sej ut ur skuggorna och när stinsen fick syn på henne pekade han på henne och sa åt mannen: ``Titta Isak, där har du din dotter Agnes!''

Kanske hade Isak tänkt stanna hemma, ingen vet, men efter en tid märktes hans oro. Han började prata om att åka tillbaka till Amerika och en dag lät han Agnes förstå, att om hon ville kunde hon följa med honom. Det blev för henne en vånda; skulle hon stanna hemma, där hon var van att leva, eller skulle hon följa med sin pappa till okända öden. Hon frågade sina farföräldrar om råd, de som tagit hand om henne sedan hon var alldeles liten. Men de ville inte ge något annat råd än att hon skulle göra det som hon  ansåg vara bäst för henne själv. Efter en tid av eftertanke ger hon sin pappa besked: -``Pappa, jag kommer inte med dej tillbaka till Amerika. Jag kommer aldrig att fara till Amerika!''

Agnes tar en kort paus i berättandet där hon 1998 sitter på sin terass på Merloy Ave. Corvallis, Oregon. Hon slår ut med händerna:-``Och här har jag nu varit nästan halva livet! Tänk vad litet man vet om framtiden''. Hon skrattar. Det ville sej nämligen så, att Agnes 1938 gifte sej med Ernst Erik Westerlund, \textborn 15.11.1913 på Fors skattehemman.

Efter kriget beslöt de att flytta ut från Silvast och bosätta sej öster om järnvägen (se nr 127). Där byggdes deras nya bostad och ladugård och livet stabiliserades. Men någon gång efter medlet av 1950-talet började locktoner från USA göra sej påminda och 1958 arrenderar de ut hemmanet med fastigheter och reser till USA för att inte komma tillbaka för annat än besök och för att sälja sin egendom. Agnes återser sin far innan han dör 15.02.1969.

Deras 3 döttrar: Greta, \textborn 1942, Gunnevi, \textborn 1947, och Gun-Britt, \textborn 1949, är alla gifta med amerikaner och har skapat sin framtid i detta stora land. Sin begåvning för skådespeleri fick Agnes också utöva i radion. Läraren Olli Thors, son till Ellen och Fredrik Thors, hade lovat att som hörspel i radion framföra en skröna om Ytterjeppo föreningstjur. Men när det var så dags, blev han inkallad till kriget och den enda han tyckte var kompetent nog att framföra det i hans ställe var Agnes, och det gjorde hon.

Ernst \textdied 13.12.1992 och Agnes \textdied 19.11.2001 i USA, men deras stoft vilar på begravningsplatsen i Jeppo.


\jhhouse{Mejeriet}{4:184}{Silvast}{5}{58}

\jhoccupant{Leppävuori}{Aki}{2014 -}

Aki Leppävuori, \textborn 04.08.1980 i Karleby, köpte fastigheten år 2014. Han arbetar på Mirka.

\jhoccupant{Botnica}{m.fl.}{1989-2014}

Handelslaget Botnica som grundats 1978 och till vilket Andelsringen fusionerats, hade i sin tur fusionerats med Waasko 1981. Innan Waaskos verksamhet avslutades 1990 hade handelslaget köpt mejerifastigheten och börjat använda den som lagerbyggnad för olika förnödenheter. Efter en närmast obeskrivlig röra av företagsbyten mellan Waasko – Hankkija – KPO - Agri-Market - Hankkija, såldes såväl handelslagets som mejerifastigheten år 2013 till Danish Agro Holding A/S som år 2014 i sin tur lade fastigheterna till försäljning på den öppna marknaden, varvid Aki Leppävuori köpte mejeriet.

\jhbold{Basfunktionen}

Den 24 oktober 1931 beluter styrelsen för Jeppo Andelsmejeri att bygga ett nytt mejeri. Innan dess hade nästan två årtionden med spänningar mellan Jungar Andelsmejeri och Jeppo Andelsmejeri fortgått. Pensalabönder hade sedan 1922 levererat till Jeppo Andelsmejeri, men i slutet av år 1927 utträtt på nytt när rykten om nybygge kommit i dagen. Och det hade sin giltighet därför att i oktober 1929 hade beslut fattats om att söka efter en ny tomt. Det utkristalliserades 7 alternativ, alla i eller nära Silvast centrum, men till slut kunde man komma till ett avgörande där ett område ägt av Anders Lindén inköptes och mejeriet uppfördes i huvudsak under 1932. Till grund för hur mejeriet skulle utformas hade styrelsen rest runt i regionen och fastnat för att utgå från Purmo Andelsmejeri, som nyss blivit klart.

\jhhousepic{029-05893.jpg}{}

För att spara kostnader fastställdes antalet dagsverken varje medlem skulle bidra med utgående från antalet kor. I februari 1933 stod mejeriet klart att tas ibruk. Enligt tillgängliga uppgifter blev kostnaderna för mejeriet 669000 mk. Tomten utgjorde 16000mk, tegel 49000mk, kakeltaket 15000mk, parkettgolvet i mejerisalen 9000mk, skorsten 10000mk, maskiner 240000mk. Gratisdagsverken av medlemmarna värderades till 155000mk och baserade sig på 1 ko = 1 andel = 6,25 dagsverken, 1 stock, 20 st takstickor(?), 1 tunnskorg mossa. De här nämnda kostnaderna utgör 494000mk. Det är oklart vad resten består av. Mejeriet ansågs modernt och attraktivt och bönder från Åvist var intresserade att ansluta sej, men jeppoborna tackade nej.

Mejeriet var ett utpräglat produktmejeri och den dominerande produkten var smör och genom åren fick mejeriet stort erkännande för sin höga kvalitet, mycket tack vare mejerskan Lydia Jungerstams insatser (se Jungar hemman nr 10d). Under de drygt 50 år mejeriet tjänat jeppobönderna och bygden har mycket hänt och kan bäst tillgodogöras i både ”Historik över Jeppo” och ”Jeppo i kris och krigstid”.

Det nya mejeriets första disponent blev jepposonen Selim Romar, som satt 10 år på sin post fram till 20.03.1943 (se Romar hemman, karta 3, nr 20). Då valdes också jeppobördige Robert Liljeqvist till ny disponent. Hans tid blev tragiskt kort då han den 30.09.1944 dog efter svåra brännskador orsakade av en exploderande blåslampa. Till ny disponent kallades nu Erik Alfred Kullman, \textborn 20.06.1912 i Oravais. Han hade  den 24.08.1941 gift sej med jeppoflickan Lea Maria Backlund, som var född 30.10.1914. De anlände med kort varsel från Vasa och familjen installerade sej i mejeriets disponentbostad den 13 november 1944. Erik Kullman verkade som disponent fram till 4 oktober 1951 varefter familjen sökte sej tillbaka till Vasa.
I Jeppo föddes barnen Susanne Ulrika, \textborn 25.12.1945 och Frej Erik, \textborn 11.10.1947

Ny disponent blev nu Matts Gunnar Träisk, \textborn 03.01.1914 i Kronoby, gift 14.02.1943 med Elsa Maria Nyqvist, \textborn 08.09.1921 i Nykarleby. Familjen anlände till Jeppo 5 mars 1952. Äldsta dotter Ulla-Britt hade fötts 13.01.1947, men den yngre dottern Christina Marlene föddes under tiden i Jeppo, den 09.07.1957.

Under Gunnar Träisk tid i Jeppo skedde omvälvande förändringar både inom mjölkproduktionen och mjölkhanteringen. Större men färre besättningar krävde tekniska förändringar i både producentledet och mejeriet, krav som i sin förlängning innebar att mejeriets fortsatta verksamhet i praktiken upphörde den 31 oktober 1977, då den sista mjölken vägdes in. Gunnar Träisk gick i pension, men hustrun Elsa fortsatte en tid som expedit på Andelsringen.  De flyttade till Jakobstad 18 september 1979.

Efter att mjölkinvägningen upphört 1977 fortsatte ännu verksamheten på mejeriet, men nu med leverantörsservice, foderförmedling och övriga produktionstillbehör fram till 29 april 1988, då mejeriets sista mejerska, Iris Lindström f. Lingonblad, som haft hand om verksamheten, gick i pension. Mejerifastigheten utbjöds nu till salu.

Många människor har under årens lopp  bott på mejeriet och alla har vi inte kunnat fånga upp, men här presenteras ändå några, förutom de disponenter vi nämnt:
\begin{enumerate}
  \item Kontrollass.  Vieno Lyyli Salminen, \textborn 1902 i Multiala, kom 12.09.1931
  \item     ''        Gerda Maria Backman, \textborn 1916 i Nykarleby, kom 19.02.1940, flyttade till Nykarleby 1942
  \item     ''        Gretel Gertud Stenfors, \textborn 1924 i Övermark, kom 1940,flyttade 1948
  \item     ''        Anna Elisabeth Majors, \textborn 1932 i Solf, kom 1953, flyttade 1956
  \item     ''        Barbro Viola Finne, \textborn 1940 i Kronoby, kom 1961, flyttade 1965
  \item     ''        Gerda Hagberg, \textborn 1922 i Vörå, kom 1957, flyttade 1976
  \item Mejerskan     Brita Gunhild Sundvik, \textborn 1917 i Munsala, kom 1944, gift 1947 m. Henrik Jungarå
  \item     ''        Fanny Nylund g. Sandberg, \textborn 1910 i Jeppo, 1940-1942
  \item     ''        Iris Lingonblad, \textborn 1923 i Oravais, kom 1953, gift m. Evert Lindström
  \item     ''        Verna Strand, \textborn 1925  i Jeppo, flyttade till Vasa 1985
  \item     ''        Tora Biskop, \textborn 1938  i  Kronoby, kom 1962, flyttade 1963
  \item Skogstekniker Guy Djupsjöbacka, \textborn 1958 i Terjärv. kom 1980, hade här kontor f. skogsv.fören. Flyttat till Oravais 1982
  \item     ''        Bengt Dahlskog hade sitt kontor här en kort tid
  \item Hårfrisörskan Anne-Kristine Häggblom hade här sin frisörssalong åren 1982-85
  \item Thomas Lindström
  \item Lantbr.avbyt. Merja Tepponen (gift Dahlström)
  \item     ''        Terese Witting
\end{enumerate}

Kompletteras---
