\jhchapter{Jungar, hemman Nr 7}

Vi har redan i början av denna historik behandlat Jungar-namnet och dess osäkra ursprung.
I likhet med hemmanen Silvast och Fors går också hemmanen Jungar och Ruotsala om varandra i geografiskt hänseende. Jungar börjar med en något hackig gräns strax efter  Bösas bäck och når fram till vägen som leder ner mot ån efter Tom Jungerstams driftcentrum. Där tar Ruotsala hemman vid och fortsätter förbi Benny Gunells gård, varefter Jungar hemman på nytt kommer in i bilden. Dan Backs bostad öster om landsvägen är den sista på Jungar hemman, men Kjell Liljeqvists gård strax före det, är en  insprängd del av Ruotsala. På västra sidan av vägen börjar Mietala vid Kjell Forsgårds. Inte lätt! De bifogade kartutsnitten följer därför inte alltid hemmansgränserna.

Jungar var alltså vid början av den nya tiden ett vedertaget namn på ett hemman om 1 mtl i Jäpu. Det är rimligt att anta att det funnits redan vid den skattläggning, som Gustav Vasa föranstaltade efter sitt maktövertagande 1521. Var fanns då Jungar hemman beläget? Idag associeras namnet Jungar i första hand till den gårdsgrupp som fortsättningsvis bär det namnet strax söder om Bösas bäck. Det finns ändå skäl att tro att bostället , d.v.s. platsen med bostad, fähus, lider, bodar och visthus funnits ett stycke längre söderut. Detta på basen av gamla kartor.

Hemmanet klövs, sannolikt 1644, mellan bröderna Jakob och Mårten Mattsson. Den nya hemmansdelen omfattande ½ mtl övergick till Mårten. Denna del fick nu ett annat namn, nämligen Rautzinkåski (Ruotsinkoski), senare Ruotsala. Det är troligt att den geografiska fördelningen har sin ursprung i denna klyvning.

Enligt ``Släkt och bygd'' och Pär-Erik Levlin har hemmanet ägts sedan 1582 av:
Hans Larsson  1582-1607 (hade 6 kor), Markus Hansson 1608-1627, Matts Hansson 1627-1644 varvid hemmanet klövs, Jakob Mattsson (son) 1644-1651, Olof Jakobsson, hu. Lisa Eriksdr. 1654-1673, Hans Jakobsson (bror), hu. Lisa Johansdr. 1675-1681, Matts Hansson, hu. Lisa Johansdr. Medåbo brorson Simon Olofsson, hu. Brita Henriksdr. 1683-1697, svägerskan Brita 1698, Gustav Matsson, hu. Brita 1699-1710, Jakob Matsson, hu. Maria 1711-1719, änkan Maria 1723-1725, Gustav Jakobsson (son) 1725-1760, Daniel Gustafsson 1760-1790, Daniel Danielsson 1790-1828, Johan Danielsson 1828-1874, Isak Johansson 1874-1896. Hemmanet har efter tid styckats i flera delar, vilket framgår ur gårdsbeskrivningen.

Bosättningen på Jungar hemman har länge expanderat i jämn takt och var år 1900 störst i Jeppo med 141 pers. Beaktar man att samtidigt på grannhemmanen Ruotsala och Mietala fanns 119 resp 113 personer, var koncentrationen av människor på en sträcka om ca 1,5 km ansenlig. Detta år fanns det bönder 59 st, torpare 48, tjänstefolk 20 och hantverkare 14 på Jungar. Jungar hemman deltog i rote tillsammans med Gunnar, Krögarn, Eckolla (Ekola) och Kojonen hemman och hade ansvar för soldattorp 141. Det benämndes över tid med namn som Snarubacka,Nårrbacka och Jungar. Under gård nr 22 (Senny Nylund) nämns om att den platsen skulle ha hyst ett litet soldattorp, men uppgiften är osäker (se Gunnar hemman).

Jungar-namnet är också kopplat till ett gästgiveri/krog. Sedan 1677 låg Jungar krog på älvens östra sida vid Holmens södra spets. När  drygt 80 år senare handlanden Johan Blad från Vasa i samband med ett planerat upprättande av ett pappersbruk i Överjeppo, ville upprätta en krog mitt emot Jungar krog,  på älvens västra sida, var stridigheter att vänta. Mera därom under Jungarå hemman.

Från Jungar har utgått många dugliga människor genom tiderna. Ättling till en av dessa backstugusittare, som på 1700/1800-talet flyttat bort, är Mikko Juva och var Finlands ärkebiskop 1978-1982. Han dog 2004. På Jungar har i övrigt verkat många duktiga personer som satt sin prägel på Jeppo-bygden. Idag finns kvar 3 aktiva brukningsenheter på detta hemman, som fortfarande har kvar sin agrara karaktär.


Jungar hemman omfattas av vidstående karta nr \jhbold{J}.


KARTA hit --->


\jhsubsection{Lägenheter på Jungar}


\jhhouse{}{}{}{}{}
