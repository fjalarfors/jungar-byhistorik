\jhchapter{Jungar \& Ruotsala, hemman Nr 7 \& 8}

\jhbold{Jungar, hemman  Nr 7}

Vi har redan i början av denna historik behandlat Jungar-namnet och dess osäkra ursprung.
I likhet med hemmanen Silvast och Fors går också hemmanen Jungar och Ruotsala om varandra i geografiskt hänseende. Jungar börjar med en något hackig gräns strax efter  Bösas bäck och når fram till vägen som leder ner mot ån efter Tom Jungerstams driftcentrum. Där tar Ruotsala hemman vid och fortsätter förbi Benny Gunells gård, varefter Jungar hemman på nytt kommer in i bilden. Dan Backs bostad öster om landsvägen är den sista på Jungar hemman, men Kjell Liljeqvists gård strax före det, är en  insprängd del av Ruotsala. På västra sidan av vägen börjar Mietala vid Kjell Forsgårds. Inte lätt! De bifogade kartutsnitten följer därför inte alltid hemmansgränserna.

Jungar var alltså vid början av den nya tiden ett vedertaget namn på ett hemman om 1 mtl i Jäpu. Det är rimligt att anta att det funnits redan vid den skattläggning, som Gustav Vasa föranstaltade efter sitt maktövertagande 1521. Var fanns då Jungar hemman beläget? Idag associeras namnet Jungar i första hand till den gårdsgrupp som fortsättningsvis bär det namnet strax söder om Bösas bäck. Det finns ändå skäl att tro att bostället, d.v.s. platsen med bostad, fähus, lider, bodar och visthus funnits ett stycke längre söderut. Detta på basen av gamla kartor.

Hemmanet klövs, sannolikt 1644, mellan bröderna Jakob och Mårten Mattsson. Den nya hemmansdelen omfattande ½ mtl övergick till Mårten. Denna del fick nu ett annat namn, nämligen Rautzinkåski (Ruotsinkoski), senare Ruotsala. Det är troligt att den geografiska fördelningen har sin ursprung i denna klyvning.

Enligt ``Släkt och bygd'' och Pär-Erik Levlin har hemmanet ägts sedan 1582 av:
Hans Larsson  1582--\allowbreak 1607 (hade 6 kor), Markus Hansson 1608--\allowbreak 1627, Matts Hansson 1627--\allowbreak 1644 varvid hemmanet klövs, Jakob Mattsson (son) 1644--\allowbreak 1651, Olof Jakobsson, hu. Lisa Eriksdr. 1654--\allowbreak 1673, Hans Jakobsson (bror), hu. Lisa Johansdr. 1675--\allowbreak 1681, Matts Hansson, hu. Lisa Johansdr. Medåbo brorson Simon Olofsson, hu. Brita Henriksdr. 1683--\allowbreak 1697, svägerskan Brita 1698, Gustav Matsson, hu. Brita 1699--\allowbreak 1710, Jakob Matsson, hu. Maria 1711--1719, änkan Maria 1723--\allowbreak 1725, Gustav Jakobsson (son) 1725--\allowbreak 1760, Daniel Gustafsson 1760--\allowbreak 1790, Daniel Danielsson 1790--\allowbreak 1828, Johan Danielsson 1828--\allowbreak 1874, Isak Johansson 1874--\allowbreak 1896. Hemmanet har efter tid styckats i flera delar, vilket framgår ur gårdsbeskrivningen.

Bosättningen på Jungar hemman har länge expanderat i jämn takt och var år 1900 störst i Jeppo med 141 pers. Beaktar man att samtidigt på grannhemmanen Ruotsala och Mietala fanns 119 resp 113 personer, var koncentrationen av människor på en sträcka om ca 1,5 km ansenlig. Detta år fanns det bönder 59 st, torpare 48, tjänstefolk 20 och hantverkare 14 på Jungar. Jungar hemman deltog i rote tillsammans med Gunnar, Krögarn, Eckolla (Ekola) och Kojonen hemman och hade ansvar för soldattorp 141. Det benämndes över tid med namn som Snarubacka,Nårrbacka och Jungar. Under gård nr 22 (Senny Nylund) nämns om att den platsen skulle ha hyst ett litet soldattorp, men uppgiften är osäker.

Jungar-namnet är också kopplat till ett gästgiveri/krog. Sedan 1677 låg Jungar krog på älvens östra sida vid Holmens södra spets. När  drygt 80 år senare handlanden Johan Blad från Vasa i samband med ett planerat upprättande av ett pappersbruk i Överjeppo, ville upprätta en krog mitt emot Jungar krog,  på älvens västra sida, var stridigheter att vänta. Mera därom under Jungarå hemman.

Från Jungar har utgått många dugliga människor genom tiderna. Ättling till en av dessa backstugusittare, som på 1700/1800-talet flyttat bort, är Mikko Juva och var Finlands ärkebiskop 1978--\allowbreak 1982. Han dog 2004. På Jungar har i övrigt verkat många duktiga personer som satt sin prägel på Jeppo-bygden. Idag finns kvar 3 aktiva brukningsenheter på detta hemman, som fortfarande har kvar sin agrara karaktär.

En alldeles speciell plats på Jungar hemmans marker är ``Svartbackan'', omgiven av Ruotsala. Nu finns endast en bostad på denna plats, men under 1800-talet och också in på 1900-talet fanns här en riklig bosättning för torpare och backstugusittare, d.v.s. personer utan eget jordinnehav. Det sägs att här har mer än 20 bostäder fått plats med rika barnfamiljer. Vilka alla som här har bott kan vi inte redogöra för, men en del av dem dyker upp i gårdsbeskrivningarna. Många har i tiden emigrerat och husen lämnats öde för att antingen flyttas närmare ån eller rivas. I mantalsförteckningarna över ``lösa personer'' från slutet av 1800-talet finns antecknat att också Ryssland var en populär plats att vistas på och där tjäna sitt uppehälle, precis som för andra byars människor.


\jhbold{Ruotsala, hemman  Nr 8}

I samband med storskiftet i slutet av 1700-talet noteras hemmanet under nummer 30 och har då benämningen Ruotsinkoski. På lantmäterikartor från 1870-talet finns hemmanet nämnt som både 30 I \= Ruotsinkoski och som 30 U \= Ruotsala. Deras gemensamma nya nummer är ändå Nr 8, Ruotsala. Namnet Ruotsala har antagits betyda ``stället där svenskar bor'' och kommit i användning efter Stora ofreden i början av 1700-talet, när inflyttande finnar övertog öde hemman. Men numera anses det vara äldre än så. Det går tillbaka åtminstone till 1600-talet.

Hemmanet har omfattat ½ mtl och är därmed ett av de mindre hemmanen. År 1644 delades nämligen Jungar hemman, som tidigare utgjort 1 helt mantal, och därigenom uppstod Ruotsala hemman. Sedan 1644 har bl.a. följande ägt detta hemman:
Mårten Mattsson 1646--\allowbreak 1678, sonen Markus Mårtensson 1681--\allowbreak 1697 och hans änka Malin Hansdr. 1698--\allowbreak 1699, Mårten Mårtensson  1700, Mårten Markusson med hu. Margareta 1701--. Medåbo är då brodern Matts Markusson med hu. Margareta Karlsdr. År 1783 ägdes hemmanet av Johan Jakobsson 5/24 mtl och Erik och Johan Eriksson 5/24 mtl vardera. År 1850 bodde här 58 personer.

Som framgår av Jungar hemmans beskrivning går hemmanen om varandra och har säkert sin bakgrund i det faktum att hemmanet delades mellan bröderna Jakob och Mårten Mattsson 1644. Sannolikt försökte de finna en överenskommelse om en någorlunda rättvis jordfördelning vad gällde bördighet och skick och därför blev resultatet detta. Vid all jordrevning d.v.s uppmätning och kartläggning av mark, var rättviseaspekten viktig när det gällde skatteläggningen.

Ruotsala deltog i rote tillsammans med hemmanen Storsilfwast och Måtare. Även om själva torpet fanns på Storsilfwast, fanns torpets ängar med säkerhet utspridda på de andra hemmanens marker och de uppges 1791 vara i gott stånd och avkasta 12 skrindor hö.


Jungar och Ruotsala hemman omfattas av vidstående karta nr \jhbold{13}.


<--- se KARTA nr 13 --->




\jhsubsection{Lägenheter på Jungar}


\jhhouse{Torrlandet}{7:69}{Jungar}{13}{3}


\jhoccupant{Strengell}{\jhname[Markus]{Strengell, Markus}}{2009--}
Markus Erik Gustav Strengell, \textborn 05.09.1982, övertog sina föräldrars bostad 2009 med en tomt på 6000 m2. Han lever ogift och är anställd som lantbruksarbetare på sin bror Glenns lägenhet på Jungar (se nr 5).\jhvspace{}


\jhhousepic{190-05908.jpg}{Strengell Erik, Ann-Mari och Markus}

\jhoccupant{Strengell}{\jhname[Erik]{Strengell, Erik} \& \jhname[Ann-Mari]{Strengell, Ann-Mari}}{1978--}
Erik Ingmar Strengell, \textborn 12.10.1935, gifte sig 22.12.1963 med Ann-Mari Ekman, \textborn 22.05.1943 i Oravais. Bostaden uppfördes 1978 en liten bit från landsvägen och driftcentrum. Den gamla mangårdsbyggnaden lämnades kvar bredvid landsvägen och grundrenoverades till bostad för följande generation.\jhvspace{}



\jhhouse{Norråker}{7:60}{Jungar}{13}{5}


\jhhousepic{193-05763.jpg}{Strengell mangårdsbyggnad}

\jhoccupant{Strengell}{\jhname[Glenn]{Strengell, Glenn} \& \jhname[Christina]{Strengell, Christina}}{1993--}
Glenn Mikael Strengell, \textborn 17.12.1964 på Jungar gifte sig 23.06.1990 med Christina West, \textborn 28.03.1967 i  Oravais. Glenn är \jhbold{nr 21 i obrutet släktled} från 1593 som innehaft gården!

År 1993 övertog de Glenns hemgård, som genom åren avsevärt förstorats och idag utgörs av ca 190 ha skogsmark, 96 ha egen odlad mark och 22,5 ha arrende. Gården har specialiserat sig på potatis och sköter utöver odling också sortering, förpackning och marknadsföring av produkten. Spannmålsodlingen är också omfattande. Gården har därtill utvecklat inkvarteringstjänster i 2 separata hus. Familjen bor i hemmanets gamla mangårdsbyggnad uppförd 1896 och numera pietetsfullt renoverad. I detta hus har släkten bott i mera än 120 år och härifrån flyttade Glenns föräldrar 1978 till sitt nybyggda hus (se nr 3).

Christina är utbildad sjukskötare, Glenn lantbruksutbildad vid Korsholms skolor. Makarna är engagerade i ett flertal föreningar och andra uppdrag. Glenn är bl.a. mångårig ordförande i Jeppo lokalavdelning av ÖSP. Han har suttit i Nykarleby stadsfullmäktige och i Jeppo Foods styrelse. Han är också ordf. i Jeppo Skogsandelslag och Jeppo SFP-avd.

Christina har medverkat i många Jepporevyer och var med vid bildandet av ``Jeposmettona'', en förening för yngre kvinnor, liksom styrelsemedlem vid nystarten av Jeppo Byaråd 2003--\allowbreak 2007, nu Jeppo Byaförening. Hon har fotvårdsmottagning, sysslar med turism och hyr ut bl.a. ``Lovisastugan'' (se Böös, nr 55) för övernattningar i Jeppo.
\begin{jhchildren}
  \item \jhperson{\jhname[Erika Maria]{Strengell, Erika Maria}}{23.03.1991}{}
  \item \jhperson{\jhname[Jakob Mikael]{Strengell, Jakob Mikael}}{04.07.1993}{}
  \item \jhperson{\jhname[William Carl Gustaf]{Strengell, William Carl Gustaf}}{18.03.1997}{}
  \item \jhperson{\jhname[Viktor Glenn Christian]{Strengell, Viktor Glenn Christian}}{27.09.2008}{}
\end{jhchildren}


\jhoccupant{Strengell}{\jhname[Erik]{Strengell, Erik} \& \jhname[Ann-Mari]{Strengell, Ann-Mari}}{1960--\allowbreak 1993}
Erik Ingmar Strengell, \textborn 12.10.1935 på Jungar, gifte sig 22.12.1963 med Ann-Mari Ekman, \textborn 22.05.1943 i Oravais. Makarna övertog 1960 Eriks hemgård av hans föräldrar. Odlingen var till en början traditionell med mjölkkor och ungnöt. Småningom i takt med att ny mekanisering visade nya möjligheter, ökade också intresset för potatisodling och efter ingången av 1970-talet fick korna stryka på foten och gården blev kreaturslös. Satsningen på potatis blev nu huvudfåran och odlingen utökades successivt.

Erik har varit engagerad i samhället. Han har bl.a. suttit med i Jeppo Kommunalfullmäktige, ÖSP:s styrelse, Ungdomsföreningen och älgjaktlaget. Som änkling bor han nu på Florahemmet i Nyk:by.
\begin{jhchildren}
  \item \jhperson{\jhbold{\jhname[Glenn]{Strengell, Glenn}} Mikael}{17.12.1964}{}
  \item \jhperson{\jhname[Sören Anders]{Strengell, Sören Anders}}{01.11.1966}{06.11.1991}
  \item \jhperson{\jhname[Camilla Ann-Christine]{Strengell, Camilla Ann-Christine}}{11.12.1971}{}
  \item \jhperson{\jhname[Markus Erik Gustav]{Strengell, Markus Erik Gustav}}{05.02.1982}{}
\end{jhchildren}

Ann-Mari \textdied 30.04.2015


\jhoccupant{Strengell}{\jhname[Johannes]{Strengell, Johannes} \& \jhname[Elna]{Strengell, Elna}}{1928--\allowbreak 1960}
Johannes Sigfrid Strengell, \textborn 28.09.1905, gifte sig 17.06.1928 med Elna Johanna Sandell, \textborn 29.11.1906 på Måtar. Efter att bröderna, Alfred, Joel, och Simon avstått från att ta hand om fädernegården var Johannes den sista av familjens söner kvar. Han upplevde det som ett kall och även om ekonomin var körd i botten övertog han gården med allt det slit som avsaknaden av motordrivna maskiner innebar.

Genast efter bröllopet flyttade Elna  in i ``den på kvinnfolk tomma gården'' och tog itu med arbetet, som naturligtvis aldrig tog slut. Johannes byggde en smedja på Torrlandet bredvid Böösas bäcken där han ofta tillbringade långa stunder med mekaniska göromål. Han blev en skicklig smed som inte hade någon tilltro till  den moderna svetsningen. Den var inte hans likör. Förvällning var det som gällde, d.v.s. de järnstycken som skulle sammanfogas hettades i fogområdet upp till omedelbar närhet av smältpunkten och smiddes snabbt ihop med hammare på städ. ``Nu håller det'', sa Johannes. Smedjan revs i medlet av 1970-talet. Under Johannes och Elnas tid förkovrades hemmanet, vilket fortsatt.
\begin{jhchildren}
  \item \jhperson{\jhname[Ragni Alice]{Strengell, Ragni Alice}}{13.11.1928}{}
  \item \jhperson{\jhname[Wolmar Gustav]{Strengell, Wolmar Gustav}}{05.12.1929}{}
  \item \jhperson{\jhbold{\jhname[Erik]{Strengell, Erik}} Ingmar}{12.10.1935}{}
  \item \jhperson{\jhname[Gunda Gertrud Monika]{Strengell, Gunda Gertrud Monika}}{22.09.1937}{}
\end{jhchildren}

Elna \textdied 04.08.1978  ---  Johannes \textdied 02.10 1989


\jhbold{1907--\allowbreak 1928}
Gustav Strengells arvingar utgör under denna period det 18:e släktledet och efter att hans änka Maria gift om sig och 1914 flyttat till Rundt (se nedan) sköttes hemmanet av dottern Helmi och sonen Johannes, som inte flyttade med. Styvfar Johannes Granlund hjälpte också till, men hemmanet förföll till en del under denna tid.


\jhoccupant{Strengell}{\jhname[Gustav]{Strengell, Gustav} \& \jhname[Maria]{Strengell, Maria}}{1899--\allowbreak 1907}
Gustav Andersson Jungar, \textborn 08.01.1871 gifte sig med Maria Bro, \textborn 28.02.1872 i Nykarleby. Giftermålet skedde 14.08.1892 och Maria flyttade nu till Jeppo. De bosatte sig som nygifta i den 1896 nybyggda fädernegårdens ena ända. 1899 övertog Gustav ½ av hemmanet. Brodern Anders övertog den andra delen. Under denna tid ändrades efternamnet från Jungar till \jhbold{Strengell}. Äktenskapet blev kort p.g.a. Gustavs för tidiga död 23.07.1907 i minsjuka, men 6 barn hann födas.
\begin{jhchildren}
  \item \jhperson{\jhname[Johan August Eliel]{Strengell, Johan August Eliel}}{1893}{1902 i hjärtfel}
  \item \jhperson{\jhname[Gustav Alfred]{Strengell, Gustav Alfred}}{06.07.1895}{}
  \item \jhperson{\jhname[Anders Joel]{Strengell, Anders Joel}}{03.02.1898}{09.07.1919 (i spanska  sjukan)}
  \item \jhperson{\jhname[Helmi Elisabet]{Strengell, Helmi Elisabet}}{28.05.1900}{}
  \item \jhperson{\jhname[Simon Severin]{Strengell, Simon Severin}}{27.06.1902}{}
  \item \jhperson{\jhbold{\jhname[Johannes Sigfrid]{Strengell, Johannes Sigfrid}}}{28.09.1905}{}
\end{jhchildren}

Maria gifte om sig 11.07.1909 med Johannes Rundt Granlund, \textborn 24.09.1862 på Rundt. 1914 flyttade familjen till Rundt, men på Jungar föddes ytterligare 2 barn innan det 3:e föddes på Rundt.
\begin{jhchildren}
  \item \jhperson{\jhname[Helga]{Strengell, Helga}}{12.11.1910}{}
  \item \jhperson{\jhname[Lea]{Strengell, Lea}}{26.12.1911}{}
  \item \jhperson{\jhname[Edit]{Strengell, Edit}}{30.11.1916}{}
\end{jhchildren}

De 13 första släktleden är gemensamma för flera av de nuvarande hemmanen på Jungar. 1790 delas hemmanet mellan bröderna Simon Danielsson och Daniel Danielsson. Simon Danielssons hemmansdel delas på nytt 1899 mellan bröderna Anders och Gustav Andersson och det är Gustavs hemmansinnehav vi följt ovan. Hans föregångare ses i förteckningen nedan.

\begin{center}
  \begin{tabular}{l l l l}
    \hline
    Markus Olofsson & 1593--\allowbreak 1604 & Jakob Mattsson(bror) & 1708--\allowbreak 1722 \\
    Hans Markusson & 1604--\allowbreak 1621 & Gustav Jakobsson & 1722--\allowbreak 1760 \\
    Markus Hansson & 1621--\allowbreak 1627 & Daniel Gustavsson & 1760--\allowbreak 1790 \\
    Matts Hansson(bror) & 1627--\allowbreak 1635 & Simon Danielsson & 1790--\allowbreak 1826 \\
    Jakob Mattsson & 1635--\allowbreak 1660 & Anders Simonsson & 1826--\allowbreak 1845 \\
    Olof Jakobsson & 1660--\allowbreak 1668 & Simon Andersson & 1845--\allowbreak 1866 \\
    Hans Jakobsson(bror) & 1668--\allowbreak 1681 & Anders Simonsson & 1866--\allowbreak 1899 \\
    Matts Hansson & 1681--\allowbreak 1697 & \jhbold{Gustav Andersson} & 1899--\allowbreak 1907 \\
    Gustav Mattsson & 1697--\allowbreak 1708 &  &  \\
    \hline
  \end{tabular}
\end{center}

Hemmanets öden har kunnat spåras ännu längre tillbaka i tiden, ända till 1382, men uppgifter om ägarna saknas.



\jhhouse{Lilja}{7:19}{Jungar}{13}{6}


\jhoccupant{Strengell}{\jhname[Glenn]{Strengell, Glenn} \& \jhname[Christina]{Strengell, Christina}}{1974--}
Glenn Strengell (se Ruotsala nr 5) köpte år 1974 fastigheten av Manne Liljeqvist, son till de tidigare ägarna och som emigrerat till USA. Fastigheten har stått obebodd sedan Emmy Liljeqvists död 1971.\jhvspace{}


\jhhousepic{194-05761.jpg}{Leonard och Emmy Liljeqvist; nu Strengell}

\jhoccupant{Liljeqvist}{\jhname[Leonard]{Liljeqvist, Leonard} \& \jhname[Emmy]{Liljeqvist, Emmy}}{1926--\allowbreak 1971}
Gustav Leonard Liljeqvist, \textborn 08.06.1894 gifte sig 02.07.1916 med Emmy Maria Mattsdr. Nylund, \textborn 20.12.1897 på Böös. De bedrev sitt jordbruk på Jungar fastän en stor del av jorden befann sig på Böös hemman. Fastigheten byggdes av Isak Liljeqvist, far till Leonard.

Under krigen och också en tid efteråt var Leonard ansvarig för folkförsörjningskommissionen i Jeppo. Kansliet inrättades i den del av huset som tidigare inhyst en liten butik. Hit hade hela kommunens befolkning på sätt eller annat ärende under denna tid för erhållande av licenser och kuponger av alla de slag. Senny Nylund var kanslist (se nr 22).
\begin{jhchildren}
  \item \jhperson{\jhname[Magnus Robert]{Liljeqvist, Magnus Robert}}{25.01.1917}{1944}, dog i olyckshändelse på mejeriet i Silvast
  \item \jhperson{\jhname[Margit Maria]{Liljeqvist, Margit Maria}}{28.03.1918}{}
  \item \jhperson{\jhname[Max Gustav]{Liljeqvist, Max Gustav}}{25.08.1920}{}
  \item \jhperson{\jhname[Manne]{Liljeqvist, Manne}}{05.07.1924}{}
  \item \jhperson{\jhname[Eva Marita]{Liljeqvist, Eva Marita}}{20.08.1937}{}
\end{jhchildren}

Leonard \textdied 26.07.1959  ---  Emmy \textdied 26.12.1971


\jhoccupant{Liljeqvist}{\jhname[Isak]{Liljeqvist, Isak} \& \jhname[Maria]{Liljeqvist, Maria}}{1885--\allowbreak 1926}
Isak Liljeqvist, \textborn 01.02.1855 (11.09.1855), gifte sig med Maria Andersdr., \textborn 09.11.1861. Isak hade  en lanthandel i Jungar och 30.04.1894 köpte han det återstående varulagret av ett handelsbolag i Silvast vilket befann sig i upplösning. Han hade affärsrörelse i den fastigheten i 19 år och därefter sålde han gården 28.03.1913 till Emil Ström,  som fortsatte verksamheten tills dess att Jeppo Handelslag övertog huset 09.04.1916 och fram till 1943 hade det som sin huvudaffär (se Silvast karta 5, nr 357).

År 1913 sålde han sitt barndomshem på Böös för 1800 Fmk till Lutherska Evangelieföreningens Ungdomsfilial i Jeppo som hade behov av en s.k. ``kolportörsgård''. Barndomshemmet hade stått längs den gamla vägen till Böös, kallad ``Skatagatån''. Det plockades ner och transporterades över den nybyggda bron i Silvast till Holmen, där det timrades upp på nytt och blev kolportören K.V. Näs bostad. Bostaden byggdes senare till med det som idag är bönehuset och fungerar nu som dess kök med kammare.

På Jungar hemman nr 7 hade Isak på 1880-talet byggt det hus som fortsättningsvis står kvar bredvid landsvägen med många fönster och rik snickarglädje. Tomten, som hörde till Strengells hemman, hade Maria och Isak fått som bröllopsgåva i samband med giftermålet.
\begin{jhchildren}
  \item \jhperson{\jhname[Hilma Lovisa]{Liljeqvist, Hilma Lovisa}}{15.12.1883}{}
  \item \jhperson{\jhname[Ida Maria]{Liljeqvist, Ida Maria}}{20.10.1885}{}
  \item \jhperson{\jhname[Sanna Emilia]{Liljeqvist, Sanna Emilia}}{19.05.1888}{}
  \item \jhperson{\jhname[Isak Joel]{Liljeqvist, Isak Joel}}{09.05.1890}{}
  \item \jhperson{\jhname[Anders Vilhelm]{Liljeqvist, Anders Vilhelm}}{01.07.1892}{}
  \item \jhperson{\jhbold{\jhname[Gustaf Leonard]{Liljeqvist, Gustaf Leonard}}}{08.06.1894}{}
  \item \jhperson{\jhname[Edvin Johannes]{Liljeqvist, Edvin Johannes}}{29.10.1896}{}
  \item \jhperson{\jhname[Otto Eliel]{Liljeqvist, Otto Eliel}}{14.01.1899}{}
  \item \jhperson{\jhname[Anna Aurora]{Liljeqvist, Anna Aurora}}{02.03.1901}{02.07.1901}
  \item \jhperson{\jhname[Karl Evald]{Liljeqvist, Karl Evald}}{02.05.1902}{}
  \item \jhperson{\jhname[Valborg Alfrida]{Liljeqvist, Valborg Alfrida}}{31.03.1906}{}
\end{jhchildren}

Maria \textdied 18.07.1916  ---  Isak \textdied 04.01.1927



\jhhouse{Jungar}{7:52}{Jungar}{13}{7}


\jhhousepic{195-05765.jpg}{Ruben Strengell}

\jhoccupant{Strengell}{\jhname[Ruben]{Strengell, Ruben}}{1975--}
Anders Ruben Strengell, \textborn 04.10.1942 övertog hemmanet 1975 som den 19:e odlaren i släktledet. Under sin aktiva tid fortsatte Ruben tillsammans med sin syster Elvi den mjölkproduktion som hade funnits på gården. Den avvecklades år 2007 och odlingsjorden utarrenderades till systersonen Lasse Mäenpää, som är jordbrukare och pälsfarmare i Kauhava.

Ruben är ogift medan Elvis tidigare äktenskap är upplöst. Denna del av Jungar hemman, som har samma ägohistoria som flera andra hemman i byn, utgörs av 18,25 ha åker och 32,5 ha skog. Den nuvarande bostadsbyggnaden uppfördes 1977 av Ruben. Den gamla karaktärsbyggnaden uppförd i medlet av 1800-talet revs år 1980.


\jholdhouse{Gamla gården}{7:52}{Jungar}{13}{107}


\jhhousepic{Strengells 2 grdar.jpg}{Strengells gårdar, Georg och Dagny Strengell i förgrunden}

\jhoccupant{Strengell}{\jhname[Georg]{Strengell, Georg} \& \jhname[Dagny]{Strengell, Dagny}}{1932--\allowbreak 1975}
Georg Valdemar Strengell, \textborn 28.08.1908 på Jungar, gifte sig 25.06.1936 med Dagny Johanna Berg, \textborn 01.12.1910 på Kojonen i Lassila. Redan som 8 åring överfördes hemmanet på hans axlar p.g.a. faderns ohälsa. 1929 reste han till Canada och återvände 1932 varefter han övertog hemmanets skötsel och gifte sig. Ladugården av kilad sten har renoverats och ekonomiebyggnaden uppfördes 1942 mitt under kriget. Mjölkproduktionen blev familjens huvudsakliga inkomstkälla. Georg har också suttit med i hälsovårdsnämnden.
\begin{jhchildren}
  \item \jhperson{\jhname[Olof Georg]{Strengell, Olof Georg}}{21.01.1938}{}
  \item \jhperson{\jhname[Elvi Dagny Anita]{Strengell, Elvi Dagny Anita}}{25.08.1939}{}
  \item \jhperson{Anders \jhbold {Ruben}}{04.10.1942}{}
  \item \jhperson{\jhname[Doris Carita]{Strengell, Doris Carita}}{28.04.1950}{}
\end{jhchildren}

Georg \textdied 07.04.1987  ---  Dagny \textdied 17.03.1993


\jhoccupant{Jungar}{\jhname[Anders]{Jungar, Anders} \& \jhname[Maria]{Jungar, Maria}}{1899--\allowbreak 1916}
Anders Andersson Jungar, senare Strengell, \textborn 19.09.1873, gifte sig 20.10.1907 med Maria Sofia Pettersdr. Pensar \textborn 14.06.1877 i Pensala. Anders löste den 14.02.1902 biljett från Hangö till Kapstaden och reste därmed ner till gruvorna i Sydafrika. Han återvände efter 2 år, men reste på nytt 03.12.1904 och drabbades där som så många andra av `minarsjukan''. Han hade övertagit ½  hemmanet 1899. 1907 gifte han sig och redan 1917 dog han av sviterna från arbetet i gruvorna och hemmanet kvarstod som sterbhus tills sonen Georg år 1932 övertog skötseln.
\begin{jhchildren}
  \item \jhperson{\jhbold{\jhname[Georg]{Jungar, Georg}} Valdemar}{28.08.1908}{}
  \item \jhperson{\jhname[Oskar Evald]{Jungar, Oskar Evald}}{21.11.1910}{}
  \item \jhperson{\jhname[Simon Edvin]{Jungar, Simon Edvin}}{27.03.1913}{}
  \item \jhperson{\jhname[Gustaf Runar]{Jungar, Gustaf Runar}}{04.07.1915}{}
  \item \jhperson{\jhname[Astrid Gustava]{Jungar, Astrid Gustava}}{10.09.1920}{}
\end{jhchildren}

De 13 första släktleden är gemensamma för flera av de nuvarande hemmanen på Jungar. År 1790 delas hemmanet mellan bröderna Simon Danielsson och Daniel Danielsson. Simon Danielssons hemmansdel delas på nytt 1899 mellan bröderna Anders och Gustav Andersson och det är Anders hemmansinnehav vi följt ovan.

\begin{center}
  \begin{tabular}{l l l l}
    \hline
    Markus Olofsson & 1593--\allowbreak 1604 & Jakob Mattsson(bror) & 1708--\allowbreak 1722 \\
    Hans Markusson & 1604--\allowbreak 1621 & Gustav Jakobsson & 1722--\allowbreak 1760 \\
    Markus Hansson & 1621--\allowbreak 1627 & Daniel Gustavsson & 1760--\allowbreak 1790 \\
    Matts Hansson(bror) & 1627--\allowbreak 1635 & Simon Danielsson & 1790--\allowbreak 1826 \\
    Jakob Mattsson & 1635--\allowbreak 1660 & Anders Simonsson & 1826--\allowbreak 1845 \\
    Olof Jakobsson & 1660--\allowbreak 1668 & Simon Andersson*) & 1845--\allowbreak 1866 \\
    Hans Jakobsson(bror) & 1668--\allowbreak 1681 & Anders Simonsson**) & 1866--\allowbreak 1899 \\
    Matts Hansson & 1681--\allowbreak 1697 & \jhbold{Anders Andersson} & 1899--\allowbreak 1917 \\
    Gustav Mattsson & 1697--\allowbreak 1708 &  &  \\
    \hline
  \end{tabular}
\end{center}

*) \textborn 19.11.1812--- \textdied 16.08.1868; g m Cajsa Anderdr. Bärs, \textborn 26.08.1814--- \textdied 1870-talet

**) \textborn 27.07.1836--- \textdied 28.01.1909; g m Sanna-Lisa Johansdr Mietala, \textborn 10.04.1835--- \textdied 23.02.1906

Hemmanets öden har kunnat spåras ännu längre tillbaka i tiden, ända till 1382, men då med andra släkten som ägare.



\jhhouse{Älvdal}{7:42}{Jungar}{13}{8}


\jhhousepic{197-05764.jpg}{Jungar Bengt och Monica}

\jhoccupant{Jungar}{\jhname[Bengt]{Jungar, Bengt} \& \jhname[Monica]{Jungar, Monica}}{1970-talet}
Bengt Gustav Jungar, \textborn 16.04.1953, gifte sig 2013 med Monica Blomqvist, \textborn 06.07.1961 i Nykarleby. Redan som 8 åring blev Bengt faderlös sedan fadern plötsligt avlidit i hjärtattack under utearbetet 1961. Tillsammans med sin mor Linnea skötte han i unga år om jordbruket tills han övertog hemmanet på 1970-talet

Bengt har varit yrkeschaufför på tunga fordon med start som chaufför vid Forss transportfirma i Munsala, därefter hos Holm´s transportfirma vid Jutas och från början av 1980-talet hos Nyko Frys.

Sedan Monica flyttade till Jungar i medlet av 1980-talet har hon arbetat inom åldringsvården både på Östervall och Hagalund åldringshem i Nykarleby. Därefter har hon haft anställning på Mirka för att sedan i 7 års tid arbeta hos Nykarleby Lastbilscentral. För tillfället arbetar hon på distans för speditionsfirman Mixell Logistics i Närpes som transportsekreterare.
\begin{jhchildren}
  \item \jhperson{\jhname[Louise Blomqvist]{Jungar, Louise Blomqvist}}{02.12.1979}{}, g Holmäng, jobb i P:öre församling
  \item \jhperson{\jhname[Johnny Jungar]{Jungar, Johnny Jungar}}{04.07.1987}{}, arbetar på Mirka
\end{jhchildren}


\jhoccupant{Jungar}{\jhname[Runar]{Jungar, Runar} \& \jhname[Linnea]{Jungar, Linnea}}{1939--\allowbreak 1961}
Gustav Runar Jungar, \textborn 10.09.1912 på Jungar gifte sig 22.10.1950 med Anna Linnea Häggstrand, \textborn 07.01.1912 på Mietala. Hemmanet utgjorde ½ av föräldrarna Daniel och Anna-Sofias hemman som han delat med brodern Sigurd den 23.02.1939, omfattade 30 ha skog och 13 ha jord. Jordbruket sköttes traditionellt men Linnea hade redan tidigare fungerat som kontrollassistent och fortsatt med detta efter makens död.

Allt ställdes på ända när Runar avled den 3 november 1961. Linnea stod nu ensam med 2 barn. Hemmanet hölls dock kvar i familjens ägo, men utarrenderades. Efter att Bengt övertagit lägenheten flyttade Linnea 1987 till Nykarleby.
\begin{jhchildren}
  \item \jhperson{\jhbold{\jhname[Bengt]{Jungar, Bengt}} Gustav}{16.04.1953}{}
  \item \jhperson{\jhname[Benita Ann-Katrin]{Jungar, Benita Ann-Katrin}}{22.10.1954}{}
\end{jhchildren}

Runar \textdied 03.11.1961  ---  Linnea \textdied 12.01.1996


\jhhousepic{Daniel Jungar.jpg}{Daniel och Anna-Sofia Jungar}

\jhoccupant{Jungar}{\jhname[Daniel]{Jungar, Daniel} \& \jhname[Anna-Sofia]{Jungar, Anna-Sofia}}{1896--\allowbreak 1939}
Daniel Jungar, \textborn 13.02.1875 på Jungar, gifte sig 1896 med Anna-Sofia Johansdr., \textborn 30.11.1876 i Markby. Han besökte Kronoby folkhögskola vintern 1893/94 vilket skulle påverka hans framtid. Två år senare, den 10.10.1896 delades föräldrarnas hemman i 2 delar varvid Daniel fick ½ och brodern Johan, \textborn 1867 fick den andra delen. Föräldrarna Isak Johansson, \textborn 1837 och Kajsa, \textborn 1833, stannade som inhyseshjon hos Daniel och Anna-Sofia. Till Amerika reste han 19.10.1901 sedan de 3 första barnen fötts. Hemkommen byggde han den mangårdsbyggnad som fortsättningsvis står kvar och bebos av hans sonson Bengt med familj.

År 1916 blev han polis samtidigt som han var engagerad i motståndet mot ryssarna, vars trupper var stationerade i landet. Tillsammans med bl.a. Carl Jonathan von Essen på Kiitola var han aktiv med att smuggla landsmän över till Sverige via Monäs för vidare transport till militärutbildning i Tyskland. Det var von Essen och kyrkoherden Westergren som föreslog honom till polis därför att det var viktigt att man som polis hade en person man kunde lita på. Ofta stod han vid stationen vid denna tid och kom med hemliga lösenord i kontakt med dem som skulle vidare. Det betydde att Daniel om dagarna tjänade ryssarna och om natten hjälpte jägarna. Att någon efter en tid läckte obekväma uppgifter till ryssarna väckte deras misstänksamhet och situationen blev farlig, men när väl frihetskriget startade den 28 jan. 1918 var kurragömmaleken slut. Han deltog såväl i Gamlakarleby som Uleåborg vid avväpnadet av de ryska garnisonerna.

Daniel utvecklade senare ett facistiskt tankemönster och hade länge sympatier för nazi-Tyskland och han var också ``Lappo-man''. Hans åsikter delades inte nödvändigtvis av hans nära släktingar. Han var också en renlevnadsman och var 1905 grundande medlem av Jeppo nykterhetsförening, vars första ordförande han blev. Likaså deltog han aktivt vid genomförandet av telefonin i Jeppo och var direktör för Jeppo Telefon Ab 1921-34.

Skyddskåren och dess verksamhet stod honom varmt om hjärtat och han ledde ett av distriktets övningar. Naturligt nog var han intresserad av vapen och jakt. Han var sträng, men inte elak, sa en av grannarna. Otvetydigt var han en respektfull gestalt och många ungdomar var rädda för honom fastän han flera gånger yttrade: ``An ska it vaar träätand mi hundan å smoåpåjkan''.

Han deltog gärna i diskussioner och var envis i sina åsikter utan att fördenskull dominera. Han ansågs vara en sympatisk sällskapsmänniska och ännu i hög ålder med en positiv syn på livet. Han läste dagstidningarna mycket noggrant och inom litteraturen speciellt sovjetfientlig litteratur. Det hör till saken att på hans dödsbädd fanns en del sovjetfientlig litteratur halvläst.

Daniel \textdied 31.10.1958  ---  Anna -Sofia \textdied 29.08.1959. Makarna ligger begravna tillsammans med 6 av sina barn som aldrig nådde vuxen ålder.
\begin{center}
  \begin{tabular}{l l l l l l}
    Med kort liv & \textborn & \textdied & Med kort liv & \textborn & \textdied \\ \hline
    Fanny & 18.02.1897 & 08.05.1907 & Agda & 10.09.1912 & 18.10.1914 \\
    Sigrid & 08.09.1901 & 13.12.1905 & Torsten & 28.02.1918 & 27.03.1919 \\
    August & 08.07.1905 & 28.12.1905 & Linnea & 28.02.1918 & 22.03.1919 \\
    Vuxen & \textborn &  & Vuxen & \textborn &  \\ \hline
    Isak Evert & 10.09.1899 &  & Sigurd Johannes & 24.10.1909 &  \\
    Aino Elisa & 08.07.1905 &  & Gustav \jhbold{Runar} & 10.09.1912 &  \\
    Sigrid Sofia & 08.11.1907 &  &  &  &  \\
  \end{tabular}
\end{center}
I hela syskonskaran finns 3 tvillingpar.


\jhoccupant{Jungar}{\jhname[Isak]{Jungar, Isak} \& \jhname[Kajsa]{Jungar, Kajsa}}{1875--\allowbreak 1896}
Isak Johansson, \textborn 11.12.1837, gift med Kajsa Johansdr., \textborn 10.08.1833, hade 1875 övertagit hemmanet på Jungar om 5/32 mtl av Johan Danielsson och hans hustru Sanna. Dessa kvarblev på sytning. Isak Johansson benämns ``sexman'' d.v.s. en av 6 förtroendemän i socknen som skulle se till att sockenstämmans beslut verkställdes. Isak kallades också ``Luthergubben'' på grund av sin religiositet. Kajsa dog 24.01.1902. Med henne fick han barnen,
\begin{jhchildren}
  \item \jhperson{\jhname[Johannes]{Jungar, Johannes}}{19.03.1867}{}
  \item \jhperson{\jhbold{\jhname[Daniel]{Jungar, Daniel}}}{13.02.1875}{}
  \item \jhperson{\jhname[Isak]{Jungar, Isak}}{19.07.1879}{}
\end{jhchildren}

Isak gifte om sig med Anna Sanna Joh.dr Lassila, \textborn 25.08.1841. De vigdes den 14.09.1902 och redan 16.10 dog hon. Han gifte sig för 3:e gången med Greta Andersdotter Granqvist från Munsala, \textborn 12.03.1848. De vigdes 20.03.1904. Isak avled den 12.06.1926 och Greta flyttade därefter den 08.08.1926 tillbaka till Munsala.



\jhhouse{Markus}{7:64}{Jungar}{13}{9}


\jhhousepic{196-05766.jpg}{Stig-Johan och Ann-Louise Jungar}

\jhoccupant{Jungar}{\jhname[Stig-Johan]{Jungar, Stig-Johan} \& \jhname[Ann-Louise]{Jungar, Ann-Louise}}{1971--}
Stig-Johan Jungar, \textborn 31.03.1952, gifte sig 03.07.1976 med Maria Ann-Louise Riska, \textborn 05.01.1956 i Jakobstad. Stig-Johan har gått i Kronoby Folkhögskola 1968/69 och Ann-Louise i Kristliga Folkhögskolan i Nykarleby 1973/74. Hon har vid Vasa sjukvårdsläroanstalt avlagt hjälpskötarexamen.

Stig-Johans övertog hemgården år 1971 omfattande 21 ha odlad mark och ca 36 ha skog. Lägenheten har genom åren förstorats och omfattar nu ca 45 ha odlad jord och ca 70 ha skog. De har odlat spannmål och specialiserat sig på potatisodling och svinproduktion.  Potatisodling fick sin start 1976 medan svinproduktionen, som övertogs från fadern Ruben, avslutades 2008.

År 1976 ombyggdes ett servicehus till bostad. Denna renoverades och byggdes ordentligt ut 1983 till nuvarande utseende. Med början som ordf. i Jeppo Ungdomsförening har Stig-Johan engagerat sig i ett flertal förtroendeuppdrag bl.a. som ledamot i ungdomsnämnden, principal i Jeppo Sparbank, ledamot i styrelse och fullmäktige i Nykarleby stad, samt ledamot i styrelse och senare ordf. för Jeppo Potatis.


\jholdhouse{Gamla gården på Markus lägenhet}{7:64}{Jungar}{13}{109}


\jhhousepic{Jungar Ruben.jpg}{Ruben och Ragni Jungar. Huset stod på vägkanten, byggdes 1887 och revs 1990.}

\jhoccupant{Jungar}{\jhname[Ruben]{Jungar, Ruben} \& \jhname[Ragni]{Jungar, Ragni}}{1957--\allowbreak 1971}
Ruben Lennart Jungar, \textborn 10.05.1928, gifte sig 23.06.1951 med Ragni Viola Löv, \textborn 25.02.1931 på Bärs. År 1950 gick han i Kronoby Lantmannaskola (numera Optima Lannäslund) och 1957 övertogs ½ av Rubens hemgård, som då utgjordes av 21 ha odlad jord och 36 ha skog. Svinuppfödningen utvecklades ytterligare och blev den helt dominerande produktionsformen. 1968 revs det gamla fähuset, som ersatts av ett nytt svinstall. 1969 byggde han samtidigt med Brage Finskas i Sorvist i Nykarleby landskommun en för den tiden exeptionell stor svingård om ca 1100 svinplatser. Satsningen var ett nytänk inom näringen.

Ruben har varit ledamot i socialnämnden, byggnadsnämnden, Jeppo Kraftandelslags styrelse och lantmannagillet. Därtill har han varit principal i Jeppo Sparbank och ordf. för Jungar Vattenandelslag. Hans hustru Ragni dog 10.11.1967. Hemmanet fungerade då som sterbhus och fick sin slutliga lösning 2008.

Barn: \jhbold{Stig-Johan}, \textborn 31.03.1952

Den 05.03.1983 gifte sig Ruben på nytt med Tuula Marjatta Keltanen, \textborn 14.01.1946 i Kortesjärvi. Hon och hennes 2 barn från ett tidigare äktenskap, Anne och Sari, hann en kort tid bo i det gamla huset innan de 1970 flyttade till Nykarleby och där byggde ett nytt hus 1974.


\jhoccupant{Jungar}{\jhname[Lennart]{Jungar, Lennart} \& \jhname[Ester]{Jungar, Ester}}{1921--\allowbreak 1957}
Isak Lennart Jungar, \textborn 26.07.1900 på Jungar, gifte sig 10.07.1927 med Ester Sofia Forss, \textborn 10.08.1907 på Fors. År 1922/23 fick Lennart utbildning vid Korsholms Lantmannaskola och redan 1921 hade han övertagit halva föräldragården. Det gamla fähuset hade rum för 10 kor, 3-4 suggor och stall för 2 hästar, bastu och sädesbod. Efter en tid var makarna bland de första att utöka svinproduktionen i större  skala.

Lennart var ledamot i taxeringsnämnden, vårdnämnden och i ägodelningsrätten. Han invaldes i kommunalfullmäktige och i ett
flertal andra nämnder. Tidigt var han medlem i Ungdomsföreningen. Likaså var han Sparbanksprincipal. Som hästintresserad satt han länge med i styrelsen för Nykarleby-Jeppo Hästförsäkringsförening.
\begin{jhchildren}
  \item \jhperson{\jhbold{\jhname[Ruben]{Jungar, Ruben}} Lennart}{10.05.1928}{}
  \item \jhperson{\jhname[Ruth Louise]{Jungar, Ruth Louise}}{14.10.1929}{}
  \item \jhperson{\jhname[Ninni Ragnhild Katarina]{Jungar, Ninni Ragnhild Katarina}}{04.06.1936}{}
  \item \jhperson{\jhname[Thea Christina]{Jungar, Thea Christina}}{21.02.1944}{}
\end{jhchildren}

Lennart \textdied 17.08.1983  ---  Ester \textdied 15.03.2001


\jhoccupant{Jungar}{\jhname[Katarina]{Jungar, Katarina} \& \jhname[Johannes]{Jungar, Johannes}}{1896--\allowbreak 1921}
Johannes Jungar, \textborn 19.03.1867 på Jungar, gifte sig 12.08.1887 med Katarina Jungarå, \textborn 08.04.1868 på Jungarå. Genom gåvobrev av den 19.10.1896 blev de ägare till hälften av faderns hemman om 5/64. En lägenhet ägd av Emil Rasmus, också den på 5/64 mtl, köptes på auktion 1918 och gav utrymme för ett aktivt brukande. Läsåret 1892/93 besökte han Kronoby Folkhögskola.

Jungar ``Janni'' var aktiv i samhället. Han var ordf. i kommunalstämman, var med i bildandet av Jeppo ungdomsförening, ordf. i Jeppo lantmannagille, likaså i dikesnämnden. Han var ledamot i taxeringsnämnden, hälsovårdsnämnden, prövningsnämnden, valnämnden, Jungar folkskoledirektion och Nykarleby – Jeppo brandstodsförening. Han var direktionsmedlem i det 1892 nybildade Handelsaktiebolaget i Silvast, vars tid dock blev kort.
\begin{jhchildren}
  \item \jhperson{\jhname[Hilda Katarina]{Jungar, Hilda Katarina}}{18.05.1888}{}
  \item \jhperson{\jhname[Anna Lovisa]{Jungar, Anna Lovisa}}{22.02.1890}{}
  \item \jhperson{\jhname[Ester Emilia]{Jungar, Ester Emilia}}{17.05.1891}{}
  \item \jhperson{\jhname[Isak Evert]{Jungar, Isak Evert}}{19.01.1893}{10.10.1897}
  \item \jhperson{\jhname[Johan Arthur]{Jungar, Johan Arthur}}{10.11.1894}{09.10.1897}
  \item \jhperson{\jhname[Edit Maria]{Jungar, Edit Maria}}{04.11.1896}{}
  \item \jhperson{Johan \jhbold{Arthur}}{22.10.1898}{}, (se Vivan Back)
  \item \jhperson{Isak \jhbold{Lennart}}{26.07.1900}{}
  \item \jhperson{\jhname[Elsa Viola]{Jungar, Elsa Viola}}{13.10.1903}{}
  \item \jhperson{\jhname[Helfrid Linnea]{Jungar, Helfrid Linnea}}{04.02.1905}{}
  \item \jhperson{\jhname[Sven Mikael]{Jungar, Sven Mikael}}{29.09.1907}{}
\end{jhchildren}

Tidigare generationer av samma släkte har brukat jorden på detta hemman och de övriga hemman som utgått från samma stomlägenhet ända sedan 1593. De är för denna hemmansdel följande personer, antecknade mellan vilka år ägandet ägt rum bakåt i tiden:

\begin{center}
  \begin{tabular}{l l l l}
    \hline
    Isak Johansson & 1874--\allowbreak 1896 & Hans Jakobsson & 1668--\allowbreak 1681 \\
    Johan Danielsson & 1828--\allowbreak 1874 & Olof Jakobsson & 1635--\allowbreak 1660 \\
    Daniel Danielsson & 1790--\allowbreak 1828 & Jakob Mattsson & 1635--\allowbreak 1660 \\
    Daniel Gustafsson & 1760--\allowbreak 1790 & Matts Hansson & 1627--\allowbreak 1635 \\
    Gustaf Jakobsson & 1722--\allowbreak 1760 & Markus Hansson & 1621--\allowbreak 1627 \\
    Jakob Mattsson & 1708--\allowbreak 1722 & Hans Markusson & 1604--\allowbreak 1621 \\
    Gustaf Mattsson & 1697--\allowbreak 1708 & Markus Olofsson & 1593--\allowbreak 1604 \\
    Matts Hansson & 1681--\allowbreak 1697 &  &  \\
    \hline
  \end{tabular}
\end{center}

Stig-Johan är således den 19:e odlaren i samma släkte som nu brukat denna hemmansdel! År 2016 avslutades epoken, då åkermarken såldes och brukandet som självständig lägenhet upphörde.



\jhhouse{Aho}{7:12}{Jungar}{13}{10}


\jhoccupant{Jungar}{\jhname[Stig-Johan]{Jungar, Stig-Johan} \& \jhname[Ann-Louise]{Jungar, Ann-Louise}}{1999--}
Stig-Johan Jungar, \textborn 31.03.1952 på Jungar,  gift med Ann-Louise Riska, \textborn 05.01.1956 i Kronoby köpte 18.06.1999 fastigheten av Tor Nylind, \textborn 20.03.1932 på Böös. Tor hade i sin tur erhållit fastigheten som gåva av sin mor Signe Nylind 1981.

Som enda kvarlevande arvinge erhöll systern Signe Nylind detta hus, som Lydia Jungerstam lämnade efter sig vid sin död 1968. Huset har endast använts sommartid efter Lydias död, tidvis av Åke Nylind och Tor Nylind med familjer.


\jhhousepic{198-05767.jpg}{Lydia Jungerstam; nu Jungar}

\jhoccupant{Jungerstam}{\jhname[Lydia]{Jungerstam, Lydia}}{1953--\allowbreak 1960}
Lydia Maria Andersdr., \textborn 14.07.1900 på Jungar. Hon levde ogift. År 1907 dog hennes mor Sanna Sofia och därefter sålde fadern Anders den lägenhetsdel han övertagit 1904 till sin bror Mats. Lydia och hennes syskon Johannes och Signe växer upp hos sina farföräldrar Erik Hilli och Maria Jungar, (senare med efternamnet Jungerstam, se mera under nr 11).

Lydia avlägger examen från Teoretiska Mejeriskolan i Gamlakarleby år 1921. Hon får sin första tjänst på Mejeriet i Malax och kommer sedan till Jeppo Andelsmejeri. Hennes kompetens att utföra kvalitativt arbete resulterade i att mejeriet flera gånger fick pris av Statens Smörkontrollanstalt. Personligen fick hon extra pris från Centrallaget Enigheten och från Finlands Svenska Mejeriförbund. 1939 anställdes hon av detsamma som mjölkinstruktör, som innebar ett kringresande liv i hela Svenska Österbotten med provtagningar, information och skolning kring allt som rörde mjölkens kvalitét. Denna tjänst hade hon fram till sin pension 1960.

Därefter återvände hon till sin hemby och flyttade in i det hus som hon köpt av Sanna Ahos arvingar i början av 1950-talet. Efter köpet engagerade hon de bästa timmermännen i Jeppo vid den tiden, Vilhelm och Eliel Sandvik, Tyyvi-pojkarna kallade. De kastade ner huset, justerade stenfoten och timrade upp huset på nytt, något förstorat. Samtidigt byggdes en innetoalett i huset, något som var ovanligt ännu i början av 1950-talet. En tid bodde hennes bror Johannes tillsammans med henne. De sista åren av sitt liv bodde hon tillsammans med systern Signe.

Lydia \textdied 02.05.1968


\jhoccupant{Aho}{\jhname[Sanna]{Aho, Sanna} \& \jhname[Erkki]{Aho, Erkki}}{1890--\allowbreak 1953}
Anna Sanna Joh.dr Stam, \textborn 14.08.1865 på Böös, gifte sig 24.02.1889 med Erkki Erkinpoika Aho, \textborn 11.03.1856 i Kortesjärvi. Han kom år 1889 från Nykarleby. Som änka anhåller hon den 23 oktober 1920 om att få inlösa sitt backstuguområde om 12 kappland (ca 1500 m²) vilket beviljas.
\begin{jhchildren}
  \item \jhperson{\jhname[Johan Emil]{Aho, Johan Emil}}{30.04.1889}{}
  \item \jhperson{\jhname[Hilda Maria]{Aho, Hilda Maria}}{15.08.1890}{}
  \item \jhperson{\jhname[Anna Lovisa]{Aho, Anna Lovisa}}{20.10.1892}{}
  \item \jhperson{\jhname[Erik William]{Aho, Erik William}}{22.03.1901}{}
\end{jhchildren}

Inhysesänkan Anna Brita Andersdr. Stam, \textborn 02.07.1838 bodde en tid hos sin dotter Anna Sanna. Huset byggdes i slutet av 1800-talet.



\jhhouse{Laxén}{8:16}{Ruotsala}{13}{120}


\jhoccupant{L \& A}{arvingar}{1990-talet}
Lennart och Adolfina Laxéns arvingar; Alf Elenius rev huset på 1990-talet.\jhvspace{}


\jhhousepic{Imanuel Forslund.jpg}{Oskar Lennart och Adolfina Laxén}

\jhoccupant{Laxén}{\jhname[Lennart]{Laxén, Lennart} \& \jhname[Adolfina]{Laxén, Adolfina}}{1965--\allowbreak 1978}
Oskar Lennart Laxèn, \textborn 20.01.1909 gifte sig 27.11.1927 med Anna Adolfina Andersdr. Kangasniemi, \textborn 08.05.1909. De köpte huset av Emanuel och Maria Forslunds dödsbo 1965. Till dess hade de bott på nr 119, ngt längre söderut.

Största delen av sitt verksamma liv arbetade Lennart som diversearbetare och efter kriget med att såga ved på stationens vedplan till bränsle åt ångloken. Länge skedde detta för hand innan el-cirkelsågen kom in i bilden. Av barnen är det endast de två yngsta döttrarna som bott i detta hus. De äldre växte upp i hus nr 119 (Ruotsala).
\begin{jhchildren}
  \item \jhperson{\jhname[Lennart Gunhard]{Laxén, Lennart Gunhard}}{17.05.1928}{}
  \item \jhperson{\jhname[Anja Melita]{Laxén, Anja Melita}}{06.05.1930}{}
  \item \jhperson{\jhname[Sven Olof]{Laxén, Sven Olof}}{26.06.1933}{}
  \item \jhperson{\jhname[Johannes Ronald]{Laxén, Johannes Ronald}}{11.12.1935}{}
  \item \jhperson{\jhname[Eva Anna Elisabet]{Laxén, Eva Anna Elisabet}}{05.04.1940}{}
  \item \jhperson{\jhname[Benita Viola]{Laxén, Benita Viola}}{16.09.1945}{}
  \item \jhperson{\jhname[Saga Diana]{Laxén, Saga Diana}}{17.10.1947}{}
\end{jhchildren}

Efter att Älvliden hyresbostäder byggts 1974 flyttade makarna dit.

Lennart \textdied 23.06.1978  ---  Adolfina \textdied 24.12.1978


\jhoccupant{Forslund}{\jhname[Emanuel]{Forslund, Emanuel} \& \jhname[Maria]{Forslund, Maria}}{1920--\allowbreak 1965}
Jakob Emanuel Gustafsson (Forslund), \textborn 18.09.1884, gifte sig 01.11.1908 med Maria Lovisa Joh dr. Backlund,  \textborn 05.10.1886 på Måtar. Jakob var anställd av VR på Jeppo station som bränsleförrådsarbetare d.v.s. uppgiften var att se till att färdig kapad ved och torv fanns tillgänglig när ångloken så behövde och det var stora mängder. Familjens 5 söner fick alla på sätt eller annat anknytning till järnvägen.

I aug. 1920 inlöste Emmanuel backstuguområdet där de bodde, det s.k. ``Torpgärdet'' tillhörande Anders Åstrands lägenhet och 3 kappland stort (1 kappland = 154 m²).
\begin{jhchildren}
  \item \jhperson{\jhname[Winkler]{Forslund, Winkler}}{09.11.1909}{}, eldare
  \item \jhperson{\jhname[Elna Maria]{Forslund, Elna Maria}}{14.06.1911}{22.11.1916}
  \item \jhperson{\jhname[Else]{Forslund, Else}}{03.01.1914}{}, servitris
  \item \jhperson{\jhname[Birger]{Forslund, Birger}}{12.03.1917}{}, stationskarl
  \item \jhperson{\jhname[Runar]{Forslund, Runar}}{21.08.1919}{}, lagerkarl
  \item \jhperson{\jhname[Signe]{Forslund, Signe}}{01.03.1922}{}, hemmafru
  \item \jhperson{\jhname[Bertel]{Forslund, Bertel}}{24.08.1924}{}, eldare
  \item \jhperson{\jhname[Ines]{Forslund, Ines}}{11.07.1927}{}, postutdelare
  \item \jhperson{\jhname[Helge]{Forslund, Helge}}{31.01.1930}{}, förman på VR:bilverkstad
\end{jhchildren}

Som framgår var familjen stor, huset litet och rummen små. Helge berättar att det var så trångt att till natten lyftes dörrarna från gångjärnen och placerades på stolar för en del av barnen att  sova på.

Emanuel \textdied 21.12.1959  ---  Maria \textdied 04.06.1965


\jhoccupant{Forslund (Gustafsson)}{\jhname[Anders Johan]{Forslund (Gustafsson), Anders Johan} \& \jhname[Kajsa]{Forslund (Gustafsson), Kajsa}}{- 1920}
Anders Johan Eriksson Brännsved, \textborn 05.11.1836 i Kvevlax, gifte sig med Kajsa Jakobsdr., \textborn 19.11.1847. Från sitt första gifte hade hon sonen Erik Johan Gustafsson, \textborn 24.02.1879. Han flyttade med efternamnet  Forslund till Lassila.

Anders och Kajsa fick barnet Jakob Emanuel, \textborn 18.09.1884.

Anders Johan \textdied 03.10.1908  ---  Kajsa \textdied 16.03.1922



\jholdhouse{Gamla huset Laxén på Håldhagan}{7:68}{Jungar}{13}{119}


\jhoccupant{Laxén}{\jhname[Lennart]{Laxén, Lennart} \& \jhname[Adolfina]{Laxén, Adolfina}}{1927--\allowbreak 1965}
Oskar Lennart Laxén, \textborn 20.01.1909, gifte sig 27.11.1927 med Anna Adolfina Kangasniemi, \textborn 08.05.1909. Lennart och Adolfina byggde ett hus på samma tomt som Lennarts föräldrar. Här föddes barnen: Lennart Gunhard, Anja, Sven-Olof, Johannes Ronald, Eva Anna Elisabet, Benita Viola och Saga Diana (se närmare nr 120).

Lennart var största delen av sitt yrkesliv diversearbetare, men utförde efter krigen sågningsarbete med arbetsamma skiften på vedplanen vid stationen, där ett dygns förbrukning ofta uppgick till 100 m³. Med tiden började huset upplevas obekvämt och 1965 köpte fam. Laxén det hus som tillfallit Maria och Emanuel Forslunds dödsbo. Detta hus revs i slutet av 1960-talet.


\jhoccupant{Laxén}{\jhname[Oskar]{Laxén, Oskar} \& \jhname[Anna-Sanna]{Laxén, Anna-Sanna}}{1914--}
På samma tomt som Lennart Laxén byggde sitt hus, hade hans far, backstugukarlen Simon Oskar Simonsson Laxén byggt sej ett hus för sej och sin familj när de 1914 flyttade från Mietala till Ruotsala. Platsen kallades sedan gammalt för ``Håldhagan'' och syftar på ett område där resande kunde hålla sina hästar för vila och bete under ett uppehåll. Om denna rätt var kopplad till något gästgiveri eller ``Krouvi'' är oklart.

Oskar, \textborn 21.07.1872, gifte sej 07.11.1890 med Anna-Sanna Fredriksdr, \textborn 16.09.1871. Simon Oskar var snickare och tillverkade stolar, bord och skåp enligt önskemål, utöver övrigt diversearbete. I äktenskapet föddes fem barn.
\begin{jhchildren}
  \item \jhperson{\jhname[Elin Elisabeth]{Laxén, Elin Elisabeth}}{11.09.1891}{15.09.1891}
  \item \jhperson{\jhname[Anna Ottilia]{Laxén, Anna Ottilia}}{05.03.1893}{}
  \item \jhperson{\jhname[Johannes Joel]{Laxén, Johannes Joel}}{27.08.1903}{03.08.1912}
  \item \jhperson{\jhname[EllenMaria]{Laxén, EllenMaria}}{06.11.1900}{28.12.1904}
  \item \jhperson{Oskar \jhbold{Lennart}}{20.01.1909}{}
\end{jhchildren}

Simon Oskar \textdied 14.02.1939  ---  Anna-Sanna \textdied 15.03.1953

Hans far, som hade samma namn, Simon Oskar, hade på 1840-talet kommit från Gamlakarleby till Keppo och där fått arbete på sågen. Han var född 14.02.1824 och gift sej med Lisa Davidsdr., född 02.02.1828 från Alahärmä. De fick två söner.
\begin{jhchildren}
  \item \jhperson{\jhname[Gustaf]{Laxén, Gustaf}}{19.05.1865}{}, dog i Amerika
  \item \jhperson{\jhbold{\jhname[Simon Oskar]{Laxén, Simon Oskar}}}{21.07.1872}{}
\end{jhchildren}

Gustafs  5:e barn fick namnet Oskar Lennart Laxén, \textborn 25.11.1902 kallad  d.ä. (far till Paul Laxén). Simon Oskars 3:e barn fick namnet Oskar Lennart Laxén, \textborn 20.01.1909, kallad d.y. (se ovan). Postförsändelserna kom ofta till fel Lennart Laxén!



\jhhouse{Petterssons}{7:12}{Jungar}{13}{110}


\jhoccupant{Pettersson}{\jhname[Johan]{Pettersson, Johan} \& \jhname[Sanna]{Pettersson, Sanna}}{}
På denna plats har stått ett hus tillhörande en torpare/backstugukarl med namnet Johan Jakob Pettersson, \textborn 27.11.1848, gift med Sanna Johansdr., \textborn 24.10.1846.

Makarna fick ett barn: Anna Lovisa, \textborn 12.09.1873.

Vid 23 års ålder flyttade hon hemifrån till Nykarleby. Huruvida hennes familj haft en dotter med namnet Vendla har vi inte kunnat klargöra, men en sådan person brukar stiga av bussen från Jakobstad sommartid, iförd hatt, och under vackra sommarkvällar sitta på trappan till släktens hus och njuta av tillvaron. På 1970-talet revs huset av Elis Källman.



\jhhouse{Kvarnåker}{7:68}{Jungar}{13}{11}


\jhhousepic{202-05776.jpg}{Tom Jungerstam}

\jhoccupant{Jungerstam}{\jhname[Tom]{Jungerstam, Tom}}{1992--}
Tom Mats Åke Jungerstam, \textborn 24.01.1965, övertog hemgården den 1 januari 1992 av sina föräldrar Åke och Gunnel Jungerstam. Gården omfattade då knappt 90 ha varav 26 ha odlad mark. Han övergick från gårdens tidigare mjölkproduktion till specialiserad köttproduktion i nybyggda utrymmen. Via arrenden odlar han nu ca 80 ha inalles.

Som ung producent valdes Tom till ordförande för Unga jordbrukare i Nykarleby och i både ÖSP och SLC:s  ungdomsutskott. Samtidigt har han varit medlem i Jeppo Lokalavdelning av ÖSP, tidvis som sekreterare och likaså medlem i ÖSP:s förbundsstyrelse i 18 år. Utgående från detta har han varit ordf. i ÖSP:s köttutskott och SLC:s slaktdjursutskott. I MTK:s slaktdjursutskott har han varit SLC:s representant och också i det sakkunnigutskott som bildades för att främja djurhälsan. Han har också under lång tid varit styrelsemedlem i Vasa Andelsbank, 15 år. Jeppo Idrottsförening har även haft nytta av hans insatser.


\jhoccupant{Jungerstam}{\jhname[Åke]{Jungerstam, Åke} \& \jhname[Gunnel]{Jungerstam, Gunnel}}{1956--\allowbreak 1992}
Åke Eliel Ferdinand, \textborn 27.07.1926, gifte sig 15.08.1954 med Gunnel Anna Maria Hongisto, \textborn 25.12.1932 på Hilli i Gränden. År 1950 genomgick Åke lantbruksutbildning i Kronoby Lantmannaskola, numera Optima Lannäslund i Jakobstad. Läsåret 1951--52 besökte Gunnel Kronoby folkhögskola.

Hemmanet delades 1956 i två delar varav Åkes bror Walter övertog den ena halvan och flyttade några hundra meter söderut och där byggde ett nytt hus (se nr 27 och Ruotsala nr 20), medan Åke och Gunnel övertog hemgården. De övertog ca 12 ha odlad jord och 25 ha skog vardera. Gunnel och Åke övertog också Gunnels hemgård på Hilli omfattande ca 38 ha varav ca 14 ha odlat.

Gårdens produktionsinriktning har varit mjölkproduktion och genast efter övertagandet startade byggandet av en ny ladugård som togs i bruk julafton 1956. Gunnel hade vid flytten hemifrån fått 2 kor i hemgift och de fick nu för andra gången flytta in i en annan ladugård. Utöver skötseln av hemmanet med allt vad det innebar, ägde Åke också en traktordriven luftkompressor med stenborrningsaggregat, som var i flitig användning i bygden. Han utförde också borrningsarbete åt brodern Arne i Vasa och åt armén i Gamlakarleby.

I medlet av 1960-talet beslöts att ett nytt bostadshus skulle byggas istället för renovering av det gamla och till lillajul 1965 flyttade familjen in i det nya huset. Timret från det gamla huset såldes till Jakobstad där det hyvlades upp och blev till ny bostad för en annan familj med 7 barn.

Åke blev invald i kommunalstyrelsen i början av 1960-talet och satt med några perioder. Han blev invald i kyrkofullmäktige i slutet av 50-talet, fungerade en tid som dess viceordf. innan han på 80-talet valdes till ordförande och fungerade som sådan fram till 1997. Också i Jeppo församlings kyrkoråd har Åke suttit, likaså i Nykarleby samfällighets kyrkofullmäktige till 1997. Sin längsta gärning har han ändå utfört i Jeppo Församlings kyrkokör där han i basstämman sjungit 40 år, 1960--\allowbreak 2000.
\begin{jhchildren}
  \item \jhperson{\jhname[Helén Barbro Marie]{Jungerstam, Helén Barbro Marie}}{12.01.1957}{}
  \item \jhperson{\jhbold{\jhname[Tom]{Jungerstam, Tom}} Mats Åke}{24.01.1965}{}
  \item \jhperson{\jhname[Tore Ulf Henrik]{Jungerstam, Tore Ulf Henrik}}{13.08.1970}{}
\end{jhchildren}

I det nybyggda huset har också både Åkes mor Maria och Gunnels mor Gustava framlevt sina sista år.


\jholdhouse{Gamla huset på Kvarnåker}{7:68}{Jungar}{13}{111}


\jhhousepic{Matts,William,Maria Lovisa Jungerstam m farfar Erik Hilli Jungar.jpg}{Mats och Maria Jungerstam m William samt farfar Erik Hilli Jungar}

\jhoccupant{Jungerstam}{\jhname[Mats]{Jungerstam, Mats} \& \jhname[Maria]{Jungerstam, Maria}}{1909--\allowbreak 1956}
Mats gifte om sig den 17.10.1909 med Maria Lovisa Mietala, \textborn 01.12.1885. Den nya familjen bosatte sig i det hus som upptimrats åt brodern Anders kring sekelskiftet 1900. Huset var stort och som ofta utrustat med två skorstenar. 3 kakelugnar, 2 kökshällar, öppen spis och bakugn höll huset varmt vintertid, men drog naturligtvis mycket ved.

Mats var en betrodd man. Han var många år ordförande och kassör i Nykarleby-Jeppo brandstodsförening, medlem i handelsslaget, folkskoldirektionen, ordf. i dikesnämnden, taxeringsnämnden och prövningsnämnden. Han var begåvad med en utsökt handstil som ännu kan beundras i gamla protokollböcker och han skrev ofta kåserier i bland annat Österbottniska Posten och andra tidningar.
\begin{jhchildren}
  \item \jhperson{\jhname[Gerda Valdine]{Jungerstam, Gerda Valdine}}{07.03.1911}{}
  \item \jhperson{\jhname[Arne Magnus Johannes]{Jungerstam, Arne Magnus Johannes}}{02.05.1913}{}
  \item \jhperson{\jhname[Maria Heldine]{Jungerstam, Maria Heldine}}{24.03.1915}{}
  \item \jhperson{\jhname[Walter Edvin]{Jungerstam, Walter Edvin}}{21.10.1916}{}
  \item \jhperson{\jhname[Ragnar Lennart]{Jungerstam, Ragnar Lennart}}{03.10.1919}{}
  \item \jhperson{\jhname[Ines Ingegerd]{Jungerstam, Ines Ingegerd}}{10.11.1921}{}
  \item \jhperson{\jhname[Ethel Ragnhild Viola]{Jungerstam, Ethel Ragnhild Viola}}{02.04.1924}{}
  \item \jhperson{\jhbold{\jhname[Åke]{Jungerstam, Åke}}}{Eliel Ferdinand}{27.07.1926}{}
  \item \jhperson{\jhname[Edna Elfrida]{Jungerstam, Edna Elfrida}}{29.07.1928}{04.07.1932 av brännskada}
\end{jhchildren}

Mats \textdied 25.01.1967  ---  Maria \textdied 26.06.1980


\jhoccupant{Jungerstam }{\jhname[Mats]{Jungerstam , Mats} \& \jhname[Lena Sofia]{Jungerstam , Lena Sofia}}{1908}
Mats Eliel, \textborn 19.05.1880, gifte sig den 13.04.1903 med Lena Sofia Sandell, \textborn 20.11.1883 på Måtar. I äktenskapet föddes två barn:
\begin{jhchildren}
  \item \jhperson{\jhname[Erik William]{Jungerstam , Erik William}}{14.11.1903}{}, gift med Olga Sandberg
  \item \jhperson{\jhname[Helga Irene]{Jungerstam , Helga Irene}}{27.08.1908}{30.10.1908}
\end{jhchildren}

Mats hade emigrerat till USA 1899, men återkom i årskiftet 1902/03 och gifte sig i april 1903. Efter varningar om att hämtas av den ryska polisen för inställelse till 3 års militärtjänstgöring, rymmer han hals över huvud tillsammans med 10 andra Jeppopojkar, den 17 maj i en dramatisk färd över Kvarken i roddbåt till Sverige. Därifrån reste han vidare på nytt till USA och återkom 4 år senare hösten 1907. Han möttes av den nu 4-årige Willam, som han inte sett och  hustrun hade insjuknat i tuberkulos och började snart vänta sitt andra barn. Helga Irene föddes 27 aug. 1908 och tre veckor senare dog modern Lena. Också Helga Irene dog snart, 30 oktober och lades i sin mors grav.

Detta år, 1908, köpte Mats den halva hemmansdel som hans bror Anders övertagit av sina föräldrar tillsammans med brodern Erik. Anders hade blivit änkling 1907 och han emigrerar till USA.


\jhoccupant{Jungerstam}{\jhname[Anders]{Jungerstam, Anders} \& \jhname[Sanna Sofia]{Jungerstam, Sanna Sofia}}{1904--\allowbreak 1907}
Anders Johan Eriksson Jungerstam, \textborn 21.07.1874, gifte sig 11.11 1899 med Sanna Sofia Isakdr. Sandberg, \textborn 01.05.1881 på Mietala. Ett nytt hus byggdes åt det nygifta paret (se ovan). De övertog hälften av föräldrarnas hemman 1904. I familjen föddes 5 barn.
\begin{jhchildren}
  \item \jhperson{\jhname[Lydia Maria]{Jungerstam, Lydia Maria}}{14.07.1900}{}
  \item \jhperson{\jhname[Johannes Lennart]{Jungerstam, Johannes Lennart}}{28.08.1901}{}
  \item \jhperson{\jhname[Signe Sofia]{Jungerstam, Signe Sofia}}{15.05.1903}{}
  \item \jhperson{\jhname[Anders William]{Jungerstam, Anders William}}{03.06.1905}{03.09.1905}
  \item \jhperson{\jhname[Hilda]{Jungerstam, Hilda}}{09.02.1907}{09.02.1907}
\end{jhchildren}
En månad efter sista barnets födelse dog Sanna Sofia den 12.03.1907. Anders var nu änkling. Han beslutar sig för att sälja sin hemmansdel till brodern Mats och emigrera till USA. Han överlämnar de 3 barnen till sina föräldrar och reser till Ironwood i USA. År 1939 återkommer han och tillbringar sina 28 sista år tillsammans med dottern Signe, som gift sig med Jürgen Nylind.



\jhhouse{Perus}{7:56}{Jungar}{13}{211}


\jhoccupant{Hongisto Mörk}{\jhname[Jeanette]{Hongisto Mörk, Jeanette}}{2015--}
Jeanette Susanne Helén Hongisto, \textborn 05.06.1968, gift med Kenneth Mörk, är fr.o.m. 2015 ägare till huset och tomten. På grund av husets dåliga skick påbörjades rivningen av detsamma år 2015. Jeanette med familj bor i Jakobstad och hon har arbete på stadens bibliotek.


\jhoccupant{Hongisto}{\jhname[Leif]{Hongisto, Leif}}{1996--\allowbreak 2015}
Leif Richard Gustav Hongisto, \textborn 30.01.1939 på Kaukos i Jeppo, gift 29.07.1963 med Lisen Helena Eklund, \textborn 23.06.1938 i Töjby, Korsnäs. Leif har arbetat som tekniker på Jaro i Jakobstad och Lisen som familjedagvårdare i samma stad. Leif blev ägare till  huset inklusive tomt 1996. Husets tidigare ägare Gerda Perus hade flyttat bort från huset 1992 varefter det stått tomt.

William och Olga Jungerstam bodde en tid i det gamla huset innan det övertogs av Gerda. Willam, \textborn 14.11.1903, Olga, \textborn 07.03.1904.


\jhhousepic{Gerda Perus.jpg}{Gerda Perus, senare Leif Hongisto. Nr 211}

\jhoccupant{Perus}{\jhname[Gerda]{Perus, Gerda}}{1950-t-1996}
Gerda Ragnhild Jungerstam, \textborn 25.01.1914 i Jeppo, gift med Oskar Konrad Perus, \textborn 20.01.1897 i Närpes. Paret vigdes i Jakobstad 01.09.1940. De förblev barnlösa.

Gerda erhöll ¼ av föräldrarnas hemman och hemgården i samband med delningen av hemmanet. Hon hade två-årig utbildning i handelsskola och hade anställning som bokförare och kassörska för bl.a. Jeppo Oravais Handelslag, privathandlare Nyby i Vörå, Närpes grönsaker AB, folkpensionsanstalten och resebyrån i Närpes.


\jhoccupant{Jungar}{\jhname[Erik]{Jungar, Erik} \& \jhname[Katarina]{Jungar, Katarina}}{1904--\allowbreak 1936}
Erik Eriksson Jungar, \textborn 15.01.1877 på Jungar, gift 16.10.1898 med Johanna Katarina Mietala, \textborn 28.09.1878 på Mietala. De övertog halva hemmanet och brodern Anders den andra halvan (se ovan). Jordbrukandet fortsatte på den gamla hemgården där barnaskaran blev stor.
\begin{jhchildren}
  \item \jhperson{\jhname[Signe Maria]{Jungar, Signe Maria}}{29.05.1899}{23.09.1899}
  \item \jhperson{\jhname[Erik Sigurd]{Jungar, Erik Sigurd}}{10.07.1900}{}
  \item \jhperson{\jhname[Signe Katarina]{Jungar, Signe Katarina}}{13.07.1902}{}
  \item \jhperson{\jhname[Anders Ivar]{Jungar, Anders Ivar}}{10.06.1904}{}
  \item \jhperson{\jhname[Johannes Alfred]{Jungar, Johannes Alfred}}{18.09.1906}{25.07.1920}
  \item \jhperson{\jhname[Viktor]{Jungar, Viktor}}{02.02.1909}{}
  \item \jhperson{\jhname[Agnes Julia]{Jungar, Agnes Julia}}{15.04.1911}{}
  \item \jhperson{\jhbold{\jhname[Gerda]{Jungar, Gerda}} Ragnhild}{25.01.1914}{}
  \item \jhperson{\jhname[Hellin Hjördis]{Jungar, Hellin Hjördis}}{14.06.1916}{}
  \item \jhperson{\jhname[Sven Olof]{Jungar, Sven Olof}}{27.10.1919}{}
\end{jhchildren}


\jhoccupant{Jungar}{\jhname[Erik]{Jungar, Erik} \& \jhname[Maria]{Jungar, Maria}}{1870--\allowbreak 1904}
Erik Andersson Hilli tog i samband med giftermålet sin hustrus efternamn Jungar när han som måg flyttade till Jungar hemman. Erik, \textborn 01.04.1847 på Hilli, och Maria, \textborn 17.09.1849 på Jungar, vigdes i Nykarleby 30.10.1870. Huset som familjen bodde i stod på ``åbackan'' och utgjorde Marias hemgård. Den var sannolikt upptimrad i början av 1800-talet. Erik blev 1889 invald i den nya folkskoldirektionen och satt också som ledamot i uppbådsnämnden, taxeringsnämnden och prövningsnämnden.

Den 15.03.1898 erhåller Erik tillstånd för byggandet av en kvarn och ett mindre sågverk på hemmanets mark i Jungar fors. Varken kvarn eller såg torde ha varit i drift särskilt länge, men ``juoåpan'' och andra lämningar visar var platsen fanns.
\begin{jhchildren}
  \item \jhperson{\jhname[Anna Sanna]{Jungar, Anna Sanna}}{02.02.1871}{10.12.1889}
  \item \jhperson{\jhbold{\jhname[Anders} Johan]{Jungar, Anders Johan}}{21.07.1874}{}
  \item \jhperson{\jhbold{\jhname[Erik]{Jungar, Erik}}}{15.01.1877}{}, (far till Sven Olof)
  \item \jhperson{\jhbold{\jhname[Matts} Eliel]{Jungar, Matts Eliel}}{19.05.1880}{}, (se 7:68, karta 13, nr 11)
  \item \jhperson{\jhname[Gustav Alfred]{Jungar, Gustav Alfred}}{18.06.1883}{}
  \item \jhperson{\jhname[Maria Lovisa]{Jungar, Maria Lovisa}}{10.12.1886}{}
\end{jhchildren}

Erik \textdied 14.09.1936  ---  Maria \textdied 10.03.1918

Fr.o.m 1850--\allowbreak 1875 står Marias far Anders Jakobsson med hustru som ägare till 5/32 mantal av Jungar hemman i mantalslängden.

Fr.o.m. 1835--\allowbreak 1850 står fadern till Anders, d.v.s. Jakob Eriksson med hustru Kajsa som ägare till samma mantal. Övriga mantalsägare på Jungar hemmansnummer vid denna tid är:
\begin{enumerate}
  \item Jonas Andersson 5/32
  \item Johan Danielsson 5/32
  \item Anders Simonsson 5/32
\end{enumerate}

Var bosättningen funnits har inte med säkerhet kunnat fastställas.



\jhhouse{Åbacka}{8:90}{Ruotsala}{13}{12}


\jhhousepic{201-05774.jpg}{Sebastian Jungell}

\jhoccupant{Jungell}{\jhname[Sebastian]{Jungell, Sebastian}}{2015--}
Daniel Sebastian Manuel Jungell, 10.01.1990 köpte år 2015 fastigheten av Birgitta Kronqvist. Sebastian har utbildat sig på plåt- och svetslinjen vid Optima. Till en början arbetade han på Kovjoki Mekan för att sedan flytta över till LKI -Källmans mekaniska verkstadsföretag i Pedersöre.


\jhoccupant{Kronqvist}{\jhname[Birgitta]{Kronqvist, Birgitta}}{2005--\allowbreak 2015}
Majris Birgitta Kronqvist, \textborn 1948 i Överpurmo Kauhajärvi, köpte fastigheten år 2000. Hon har arbetat på Mirka-enheten i Oravais och tidigare varit bosatt i Nykarleby. Hon sålde fastigheten 2015 och flyttade tillbaka till Nykarleby.\jhvspace{}


\jhoccupant{Cederström}{\jhname[Lenelis]{Cederström, Lenelis} \& \jhname[Håkan]{Cederström, Håkan}}{1999--\allowbreak 2005}
Lenelis och Håkan Cederström var ägare till fastigheten åren 1999--\allowbreak 2005 utan att ha varit bosatta i den. Bostaden har varit Lenelis barndomshem.\jhvspace{}


\jhoccupant{Lindgren}{\jhname[Stig]{Lindgren, Stig} \& \jhname[Inger]{Lindgren, Inger}}{1975--\allowbreak 1999}
Stig Lindgren, \textborn 07.11.1949, gifte sig 16.10.1981 med Inger Helsing, \textborn 04.10.1953 i Vexala. De övertog fastigheten 1975 av Stigs föräldrar. De livnärde sig många år på pälsdjursuppfödning men 1999 sålde de fastigheten  till Stigs syster och svåger, Lenelis och Håkan Cederström, och flyttade för en tid om 8 år till Norge.


\jhoccupant{Lindgren}{\jhname[Karl-Johan]{Lindgren, Karl-Johan} \& \jhname[Etel]{Lindgren, Etel}}{1948--\allowbreak 1975}
Karl-Johan Lindgren, \textborn 22.03.1915 i Karleby,  gift med Elin Johanna Furu, \textborn 17.01.1913 i Terjärv köpte fastigheten i maj 1948 av Joel Sandberg och hans hustru Jenny. Elin dog redan 15.10.1948 av svår sjukdom och Karl-Johan gifte sig nu med hennes yngre syster Etel Matilda Furu, \textborn 14.12.1914. Systrarnas mor, Anna Matilda, \textborn 19.07.1879, flyttade också till Jeppo och bodde i huset bredvid, fram till sin död.

Karl-Johan har varit timmerman och snickare och bl.a. arbetat på Keppo utöver de uppdrag han fått av många privatpersoner. Efter att ha överfört fastigheten på Stig och Inger, bodde Karl-Johan och Etel en tid tillsammans med dem innan de flyttade över i grannhuset.
\begin{jhchildren}
  \item \jhperson{\jhname[Mats Fjalar]{Lindgren, Mats Fjalar}}{07.11.1949}{}
  \item \jhperson{\jhbold{\jhname[Stig]{Lindgren, Stig}} Johan}{07.11.1949}{}
  \item \jhperson{\jhname[Guy]{Lindgren, Guy}}{18.03.1951}{}
  \item \jhperson{\jhname[Karl Andrew]{Lindgren, Karl Andrew}}{25.09.1952}{}
  \item \jhperson{\jhname[Anna Lenelis]{Lindgren, Anna Lenelis}}{21.02.1955}{}
\end{jhchildren}

Karl-Johan \textdied 13.02.1987  ---  Etel \textdied 15.07.1997


\jhoccupant{Sandberg}{\jhname[Joel]{Sandberg, Joel} \& \jhname[Jenny]{Sandberg, Jenny}}{1939--\allowbreak 1948}
Joel Sigrid Sandberg, \textborn 24.12.1914 på Ojala, gifte sig 1934 med Jenny Sofia Åstrand, \textborn 26.07.1912 på Ruotsala. De övertog hemmanet av Jennys föräldrar 13.05.1939. Joel var sysselsatt med flera projekt och han byggde bl.a. den enda biograf som uppförts i Jeppo (Silvast, nr 60). Folkhumorn började därefter kalla honom ``bio-Joel'' och alla visste vem det var fråga om. Makarna fick två döttrar.
\begin{jhchildren}
  \item \jhperson{\jhname[Margot Inga Sofia]{Sandberg, Margot Inga Sofia}}{19.11.1935}{}
  \item \jhperson{\jhname[Siv Alfhild]{Sandberg, Siv Alfhild}}{27.08.1940}{}
\end{jhchildren}

Familjen emigrerade i febr. 1955 till USA, där Joel arbetade inom byggnadsbranchen. Se Silvast hemman, karta 5, nr 60 (biografen).


\jhoccupant{Åstrand}{\jhname[Anders]{Åstrand, Anders} \& \jhname[Anna Sofia, Sanna]{Åstrand, Anna Sofia, Sanna}}{1910--\allowbreak 1939}
Anders Matts Vilhelm Åstrand, \textborn 01.11.1877, gifte sig 08.12.1901 med Anna Sofia Andersdr. Mietala, \textborn 25.10.1874. Anna Sofia dog 17.08.1921. I äktenskapet föddes tre barn.
\begin{jhchildren}
  \item \jhperson{\jhname[Anders Vilhelm]{Åstrand, Anders Vilhelm}}{12.03.1903}{}
  \item \jhperson{\jhname[Johannes Emil]{Åstrand, Johannes Emil}}{27.07.1906}{}
  \item \jhperson{\jhbold{\jhname[Jenny Sofia]{Åstrand, Jenny Sofia}}}{29.07.1912}{}, g Sandberg
\end{jhchildren}
Under denna tid uppges huset vara flyttat från Kasackbacken i Åvist och upptimrat på nytt där det nu står, av Anders Åstrand. Årtalet okänt.

Anders gifte 23.09.1923 sig på nytt med änkan Sanna Katarina Joh.dr. Strengell, \textborn 29.06.1880. Sanna var änka efter Johan Emil Andersson Strengell, \textborn 26.07.1879 på Svartbackan. Han sökte sig till Amerika på förtjänstarbete och dog där av ``minarsjukan'' 12.05.1922.

I Sannas första äktenskap föddes tre barn.
\begin{jhchildren}
  \item \jhperson{\jhname[Erik Severin]{Åstrand, Erik Severin}}{23.07.1900}{}
  \item \jhperson{\jhname[Johannes Maurits]{Åstrand, Johannes Maurits}}{18.12.1901}{}
  \item \jhperson{\jhname[Paul Valdemar]{Åstrand, Paul Valdemar}}{06.09.1906}{}
\end{jhchildren}



\jhhouse{Åbacka}{8:19U}{Ruotsala}{13}{13}


\jhhousepic{200-05771.jpg}{Stig och Inger Lindgren}

\jhoccupant{Lindgren}{\jhname[Stig]{Lindgren, Stig} \& \jhname[Inger]{Lindgren, Inger}}{2005--}
Stig Johan Lindgren, \textborn 07.11.1949, gifte sig 16.10.1981 med Inger Berit Helena Helsing, \textborn 04.10.1953 i Vexala. Stig övertog hemgården 1977 och byggde upp sin pälsfarmning på Svartbacken. Efter att fadern 1971 köpt detta hus av Göran Liljeqvist, som i sin tur ärvt det av Villiam och Irene Backlund, användes huset som pälsningshus, men i samband med giftermålet renoverades det på nytt till bostadshus och 1982 flyttade föräldrarna hit.

Efter att föräldrarna avlidit, Karl-Johan 1987 och Etel 1997, överfördes ägandet på dottern Leneslis, innan Stig och Inger köpte det 2005. Pälskrisen på 1990-talet tvingade Stig att lägga ner farmningen och makarna flyttade till Norge 1999 för att återkomma 2008.


\jhoccupant{Cederström}{\jhname[Lenelis]{Cederström, Lenelis}}{1997--\allowbreak 2005}
Lenilis köpte huset 1997 av modern Etel, men bodde inte där innan det köptes av brodern Stig 2005.\jhvspace{}


\jhoccupant{Lindgren}{\jhname[Karl-Johan]{Lindgren, Karl-Johan} \& \jhname[Etel]{Lindgren, Etel}}{1971--\allowbreak 1997}
Karl-Johan och Etel köpte huset av Göran Liljeqvist 1971. De flyttade in efter renovering först 1982.\jhvspace{}


\jhoccupant{Liljeqvist}{\jhname[Göran]{Liljeqvist, Göran}}{1965--\allowbreak 1971}
Göran Liljeqvist ärvde huset 1965, men bebodde det inte. Istället hyrdes huset ut till Erik Norrgård, \textborn 27.05.1936, gift med Denice Back, \textborn 29.11.1945 i Pensala, under åren 1966--69. Under denna tid föddes dottern Katarina, \textborn 09.09.1966.


\jhoccupant{Backlund}{\jhname[William]{Backlund, William} \& \jhname[Irene]{Backlund, Irene}}{1936--\allowbreak 1965}
Erik William Backlund, \textborn 12.02.1900, gifte sig med Irene Maria Forsman, \textborn 14.05.1906 på Ruotsala år 1946. Nu flyttade Irene dit efter att fram till dess bott hos sina föräldrar Johan och Mina Forsman. Hon var moster till ovan nämnda Göran Liljeqvist. Hon fortsatte att hjälpa till på hemgården.

William hade två smeknamn: ``Präntta-William'' efter sin mor ``Präntta-Maj'' (Maria Lovisa Backlund, \textborn 18.02.1863 --- \textdied 22.09.1943) som flyttade med William. Maria var tydligen skrivkunnig och levde på Svartbacken. Hemgården på Svartbacken brann på 1940-talet. William kallades också för ``Måsapomo'' p.g.a. att han handlade med vitmossa som plockades i trälådor, torkades och exporterades till Tyskland av åbofirman Bergelin.

William var anställd på ammunitionsfabriken vid Kiitola och fick senare anställning på Keppo där han till en början arbetade på snickeriet med svarvningsarbeten. Han var timmerman och deltog i byggnadsarbetet på den nystartade farmen. Vintertid brukade han vandra tvärs över ån till arbetet med en stormlykta i handen. Han var ordentligt döv och måste tilltalas i hård ton.

William \textdied 27.11.1965  ---  Irene \textdied 16.12.1965


\jhoccupant{Ekström}{\jhname[Emil]{Ekström, Emil} \& \jhname[Hilda]{Ekström, Hilda}}{1930--\allowbreak 1936}
Emil Ekström, \textborn 16.06.1882 gifte sig 19.12.1909 med Hilda Karolina, \textborn 29.01.1888. Efter att ha köpt stället av Johannes Åstrand bodde familjen här under tiden 22.03.1933--04.05.1936 innan flytten till Fors, karta 6, nr 397a. Han var sågdisponent innan han övertog Vd posten på Jeppo Kraft. Familjen hade 10 barn (se Mietala nr 116). Han säljer fastigheten inkl. bostad och uthus till William Backlund den 04.05.1936.



\jhhouse{Gammelgård}{8:60 (8:36)}{Ruotsala}{13}{15}


\jhhousepic{199-05768.jpg}{Siv Källman och Gustav Källman}

\jhoccupant{Källman}{\jhname[Siv]{Källman, Siv}}{1987--}
Siv Källman, \textborn 27.03.1949, gift med Hannu Hautanen, \textborn 09.03.1949. Siv övertog den del av hemmanet som hennes far Elis ärvt tillsammans med sin bror Alfred. Hemmansdelen har varit oskiftad fram till 1987 och utarrenderad under lång tid. Siv bor inte i hemgården utan har ett eget hus vid Blåmans, Nybyggar. Hon har arbetat på Mirka och lever frånskild.

Barn: Kristian Hautanen, \textborn 06.03.1974


\jhoccupant{Källman}{\jhname[Elis]{Källman, Elis} \& \jhname[Hedvig]{Källman, Hedvig}}{1947--\allowbreak 1987}
Elis Walfrid Källman, \textborn 03.12.1913, gifte sig 05.10.1947 med Hedvig Vikblom, \textborn 11.04.1925. Den 29.12.1947 övertogs hemmanet av bröderna Alfred och Elis Källman och för utbrytning till systrarna Ellen, Ester och Hjördis. Ägandet fortsatte som sämjobyte under Elis och brodern Alfreds tid (se Ruotsala nr 16) Elis har vid sidan av jordbruket också haft diverse förtjänstarbete, bl.a. på sågen i Silvast och också som timmerman.

Barn: \jhbold{Siv}, \textborn 27.03.1949

Elis \textdied 30.06.1987  ---  Hedvig \textborn 07.04.1985


\jhoccupant{Källman}{\jhname[Anders]{Källman, Anders} \& \jhname[Maria]{Källman, Maria}}{1905--\allowbreak 1947}
Anders Gustaf Gustafsson Källman, \textborn 09.12.1876, gifte sig 21.06.1903 med Maria Lovisa Ruotsala, \textborn 05.10.1884. Anders kom som måg och tillsammans med Maria övertog de hemmanet av Marias föräldrar. När har inte kunnat fastställas, men sannolikt ca 1905.
\begin{jhchildren}
  \item \jhperson{\jhname[Ellen Maria]{Källman, Ellen Maria}}{09.12.1904}{}
  \item \jhperson{\jhname[Ester Irene]{Källman, Ester Irene}}{23.03.1907}{}
  \item \jhperson{Anders \jhbold{Alfred}}{02.08.1909}{}
  \item \jhperson{\jhname[Johannes Edvin]{Källman, Johannes Edvin}}{13.01.1912}{}
  \item \jhperson{\jhbold{\jhname[Elis]{Källman, Elis}} Walfrid}{03.12.1913}{}
  \item \jhperson{\jhname[Hjördis Emilia]{Källman, Hjördis Emilia}}{08.04.1915}{}
  \item \jhperson{\jhname[Johannes Edvin]{Källman, Johannes Edvin}}{06.12.1919}{}
\end{jhchildren}

Anders \textdied 18.09.1950  ---  Maria \textdied 24.01.1969


\jhoccupant{Ruotsala (Mietala)}{\jhname[Anders]{Ruotsala (Mietala), Anders} \& \jhname[Maria]{Ruotsala (Mietala), Maria}}{ca 1875--\allowbreak 1905}
Anders Gustaf, \textborn 18.05.1845 på Ruotsala, gifte sig med Maria Johansdr. Romar, \textborn 28.08.1845. De hade Gammelgård hemman på Ruotsala. Exakta tidpunkter inte fastställda.
\begin{jhchildren}
  \item \jhperson{\jhname[Johan]{Ruotsala (Mietala), Johan}}{20.01.1878}{}, (till USA, dog där 10.12.1902)
  \item \jhperson{\jhbold{\jhname[Maria]{Ruotsala (Mietala), Maria}} Lovisa}{05.10.1884}{}
\end{jhchildren}


\jhoccupant{Mietala}{\jhname[Johan]{Mietala, Johan} \& \jhname[Sanna-Brita]{Mietala, Sanna-Brita}}{ca 1835--1875}
Johan Johansson Mietala, \textborn 03.09.1811 på Jungar, gifte sig 15.10.1837 med Sanna Brita Andersdr. Jungar, \textborn 13.02.1817 på Jungar. År 1835 hade Johan köpt 5/48 mtl på Ruotsala. År 1850 finns antecknat ett mantalsinnehav om 5/24 d.v.s en fördubbling.
\begin{jhchildren}
  \item \jhperson{\jhname[Caisa]{Mietala, Caisa}}{16.10.1841}{}
  \item \jhperson{\jhname[Maria]{Mietala, Maria}}{07.03.1844}{}
  \item \jhperson{\jhbold{\jhname[Anders Gustaf]{Mietala, Anders Gustaf}}}{18.05.1845}{}
  \item \jhperson{\jhname[Anna Sanna]{Mietala, Anna Sanna}}{24.10.1846}{}
  \item \jhperson{\jhname[Johan]{Mietala, Johan}}{02.02.1849}{}
  \item \jhperson{\jhname[Sofia]{Mietala, Sofia}}{14.12.1852}{}
  \item \jhperson{\jhname[Jakob]{Mietala, Jakob}}{20.07.1859}{31.08.1859}
  \item \jhperson{\jhname[Johanna]{Mietala, Johanna}}{22.06.1862}{17.07.1862}
\end{jhchildren}



\jhhouse{Sonora}{8:61}{Ruotsala}{13}{16}

Se husbild på föregående Gammelgård, nr 15.

\jhoccupant{Källman}{\jhname[Gustav]{Källman, Gustav} \& \jhname[Seija Niemi]{Källman, Seija Niemi}}{1975--}
Bo Gustav Källman, \textborn 1951, gift 1995 med Seija Helena Niemi, \textborn 1960 i Jakobstad. Den 2 juli 1975 övertog Gustav äganderätten till hemmansdelen. År 1987 skiftades hemmanet efter fadern Alfred och hans bror Elis som fram till dess varit sämjoskiftat. Gustav erhöll den del som idag utgörs av Sonora 8:61. Hemmansdelen har 15 ha jord och 34 ha skog. Odlade jorden har länge varit utarrenderad och det gamla huset står tomt.

Gustav har arbetat på UPM i Jakobstad fram till pensionen i jan 2016, medan Seija har arbetat på Walki Visa i Jakobstad och numera för Pedersöre församling på begravningsplatsen. De bor i Jakobstad.
\begin{jhchildren}
  \item \jhperson{\jhname[Tony]{Källman, Tony}}{1997}{}
  \item \jhperson{\jhname[Suvi]{Källman, Suvi}}{1997}{}
\end{jhchildren}


\jhoccupant{Källman}{\jhname[Alfred]{Källman, Alfred} \& \jhname[Signe]{Källman, Signe}}{1947--\allowbreak 1974}
Anders Alfred Källman, \textborn 02.08.1909, gifte sig 22.06.1946 med Signe Elise Örn, \textborn 03.06.1920.

Om hemmanet ingicks sämjobyte med brodern Elis 29.12.1947. Ett nytt hus uppfördes före krigen på västra sidan av landsvägen och var tänkt för Alfred (se gård nr 115), men en eldsvåda 4 aug. 1948, där ekonomibyggnaderna eldhärjades, blev en katastrof. Hur elden börjat vet ingen och den spred sig till granngården. Föräldrarna och äldsta systern Ellen flyttade nu in i huset väster om vägen och istället delades hemmanets långa mangårdsbyggnad på hälften mellan bröderna. Den eldhärjade gårdens ekonomiebyggnad och ladugård måste snabbt byggas upp och blev klar året därpå.

Anders och Signe hade mjölkproduktion som sin huvudsakliga inkomstkälla.
\begin{jhchildren}
  \item \jhperson{\jhbold{\jhname[Bo Gustav]{Källman, Bo Gustav}}}{18.10.1951}{}
  \item \jhperson{\jhname[Britt-Mari Margareta]{Källman, Britt-Mari Margareta}}{17.09.1961}{}
\end{jhchildren}

Alfred \textdied 12.03.1987  ---  Signe \textdied 30.01.2008


\jhoccupant{Källman}{\jhname[Anders]{Källman, Anders} \& \jhname[Maria]{Källman, Maria}}{1905--\allowbreak 1947}
Anders Gustaf Gustafsson Källman, \textborn 09.12.1876, gifte sig 21.06.1903 med Maria Lovisa Ruotsala, \textborn 05.10.1884 på Ruotsala. Som måg kom Anders till gården och tillsammans med Maria övertog de hemmanet av Marias föräldrar. När har inte kunnat fastställas, men sannolikt ca 1905. Den stora mangårdsbyggnaden är byggd 1901.
\begin{jhchildren}
  \item \jhperson{\jhname[Ellen Maria]{Källman, Ellen Maria}}{09.12.1904}{}
  \item \jhperson{\jhname[Ester Irene]{Källman, Ester Irene}}{23.03.1907}{}
  \item \jhperson{\jhbold{\jhname[Anders Alfred]{Källman, Anders Alfred}}}{02.08.1909}{}
  \item \jhperson{\jhname[Johannes Edvin]{Källman, Johannes Edvin}}{13.01.1912}{}
  \item \jhperson{\jhbold{\jhname[Elis Walfrid]{Källman, Elis Walfrid}}}{03.12.1913}{}
  \item \jhperson{\jhname[Hjördis Emilia]{Källman, Hjördis Emilia}}{08.04.1915}{}
  \item \jhperson{\jhname[Johannes Edvin]{Källman, Johannes Edvin}}{06.12.1919}{}
\end{jhchildren}

Anders \textdied 18.09.1950  ---  Maria \textdied 24.01.1969

De tidigare ägarna kan utläsas från gård nr 15, Gammelgård Rno 8:60, där bakgrunden är densamma.


\jholdhouse{Gamla gården}{8:25}{Ruotsala}{13}{115}

\jhhousepic{Ellen Kallmans hus.jpg}{Ellen Källmans hus}

Före kriget byggdes huset för Alfred Källman, men blev istället bostad för hans föräldrar Anders och Maria och systern Ellen (se ovan). Huset stod tomt efter Ellens död 31.12.1989 och revs 2016.\jhvspace{}


\jholdhouse{Gamla gården}{8:25}{Ruotsala}{13}{116}


\jhoccupant{Strengell}{\jhname[Bengt]{Strengell, Bengt}}{1979--}
Bengt Strengell, \textborn 26.05.1941, ärvde huset av sin farmor, änkan Sanna Åstrand. Huset, som uppförts 1936 av nytt timmer, revs 1979 och forslades till Terjärv, där det 1986 återuppfördes på en tomt i Grannabba, som tillhörde hans hustru Eivor Grannabba, \textborn 22.02.1943. Till en början fungerade byggnaden som sommarstuga, men har senare förstorats och ombyggts och fungerar nu som makarnas bostad.

Bengt brukade som liten besöka sin farmor och bära in ved. Som tack lovades han huset vid hennes frånfälle.


\jhhousepic{BengtS hus.jpg}{Curt Strengell och Sanna på trappan till det nya huset. Källmans hus i bakgrunden.}

\jhoccupant{Åstrand}{\jhname[Anders]{Åstrand, Anders} \& \jhname[Sanna]{Åstrand, Sanna}}{1936--\allowbreak 1955}
Sanna Katarina Åstrand (Strengell) född Slangar, \textborn 29.06.1880 --- \textdied 20.01.1958, gift med Johannes Emil Strengell (född Jungar), \textborn 26.07.1879 --- \textdied 12.05.1922  i USA.
\begin{jhchildren}
  \item \jhperson{\jhname[Emil Severus]{Åstrand, Emil Severus}}{23.07.1900}{01.03.1937 i USA}
  \item \jhperson{\jhname[Johannes Maurits]{Åstrand, Johannes Maurits}}{18.12.1901}{06.02.1936 i USA}
  \item \jhperson{\jhname[Paul Valdemar]{Åstrand, Paul Valdemar}}{06.09.1906}{24.03.1993 i Jeppo}, Valde for till Canada 1925 och återvände 1934-35.
\end{jhchildren}
Änkan Sanna Strengell gifte om sig 23.09.1923 med Anders Åstrand, \textborn 01.11.1877. Anders påbörjade husbygget efter att ha sålt sitt hus närmare ån till Joel och Jenny (Anders' dotter) Sandberg år 1935, men hann inte slutföra det före sin död. Uno Bärs och Viktor Löw färdigställde huset så Sanna kunde flytta in, hon bodde där till mitten av -50 talet. Sina sista år tillbringade hon på Kuddnäs.

Anders \textdied 26.03.1939  ---  Sanna \textdied 20.01.1958

Efter Sannas död hyrdes huset ut till syskonen Irene (Reni) Backlund och Arvid Liljeqvist (Forsman) samt systersonen Tor Liljeqvist. Irene	\textdied 1965  ---  Arvid \textdied 1966  ---  Tor	\textdied 1968.

Huset stod efter 1968 tomt till 1979, då Bengt Strengell flyttade huset till Terjärv och förstorade det lite. Bengt och Eivor Strengell använder det nu som bostadshus.



\jhhouse{Mellanskog}{8:96}{Ruotsala}{13}{17}


\jhhousepic{204-05780.jpg}{Alf Elenius}

\jhoccupant{Elenius}{\jhname[Alf]{Elenius, Alf}}{1952--}
Alf Elenius, \textborn 31.07.1937 på Ruotsala, övertog hemgården år 1952. Den utgörs av 23 ha odlad jord och 42 ha skog. Mjölkproduktion har framgångsrikt bedrivits på lägenheten fram till 1995. Den ersattes då av biffdjursuppfödning fram till 2001, varefter spannmål blev den huvudsakliga grödan. Idag är åkern utarrenderad och delvis såld och som pensionär  sköter han idag mest om skogen.

År 1973 utvidgades ladugården och moderniserades och gav plats för ca 20 kor plus ungdjur. Ny bostadsbyggnad började byggas 1979 och blev klar året därpå. Den gamla gården revs 2001, 100 år efter att den byggts. Han bor tillsammans med sina systrar Gunnel och Karin på lägenheten. Gunnel har mycket aktivt deltagit i skötseln av hemmanet och Karin har i flera årtionden varit kanslist på kommungården.

Evangeliska folkhögskolan besökte Alf läsåret 1955/56. Alf har haft ett flertal förtroendeuppdrag; styrelsen för Jeppo lokalavd. av ÖSP, Jeppo Skogsandelslag, Jeppo Skogsvårdsförening. Han har också varit ledamot i Jeppo kommuns hälsovårdsnämnd. Alf är ogift.



\jholdhouse{Gamla bostadsbyggnaden på Mellanskog}{8:96}{Ruotsala}{13}{117}


\jhhousepic{Alf Elenius gamla.jpg}{Gamla huset på Mellanskog}

\jhoccupant{Elenius}{\jhname[Alfred]{Elenius, Alfred} \& \jhname[Ester]{Elenius, Ester}}{1920--\allowbreak 1952}
Alfred Isaksson Ruotsala Elenius, \textborn 26.10.1896 på Ruotsala, gifte sig 08.11.1925 med Ester Emilia Jungarå, \textborn 08.02.1903 på Jungarå. År 1917 gick han i Korsholms Lantmannaskola, vilket inte var så vanligt den tiden. Han övertog hälften av sin hemgård 1920, redan innan han gifte sig. Huset byggdes 1900/1901.

Utöver jordbruket hade han många förtroendeuppgifter. En av de främsta är uppdraget som nämndeman, vilket 1959 förlänade honom titeln häradsdomare. Otaliga är de köpebrev han upprättat i denna egenskap av offentligt köpvittne.

Han var sekreterare i ungdomsföreningen, satt som medlem i skoldirektionen för Gunnar folkskola, viceordf. för Jungar Andelsmejeri och lantmannagillet, styrelseledamot i Jeppo Såg \& Kvarn Andelslag plus uppdrag i flertalet nämnder och kommittéer. Han var också uppställd som kandidat i 1936 års riksdagsval. Den 4 augusti 1948 drabbades gården av en katastrof när elden spred sig från grannen Källmans och uthusen och ladugården brann ner. Branden upptäcktes klockan 2 på natten och hur den uppstått är oklart. Med hjälp av grannar och släktingar kunde korna stallas in i det nya fähuset efter 3 månader!
\begin{jhchildren}
  \item \jhperson{\jhname[Berit Maria]{Elenius, Berit Maria}}{02.09.1926}{}
  \item \jhperson{\jhname[Gunnel Helena]{Elenius, Gunnel Helena}}{08.01.1928}{}
  \item \jhperson{\jhname[Gretel Emilia]{Elenius, Gretel Emilia}}{19.12.1929}{}
  \item \jhperson{\jhname[Gunvor Sofia]{Elenius, Gunvor Sofia}}{24.10.1932}{}
  \item \jhperson{\jhname[Karin Viola]{Elenius, Karin Viola}}{09.08.1935}{}
  \item \jhperson{\jhbold{\jhname[Alf]{Elenius, Alf}} Gustav Johannes}{31.07.1937}{}
  \item \jhperson{\jhname[Britta Linnea]{Elenius, Britta Linnea}}{06.11.1941}{}
\end{jhchildren}

Alfred \textdied 09.02.1960  ---  Ester \textdied 26.09.1980


\jhoccupant{Elenius}{\jhname[Isak]{Elenius, Isak} \& \jhname[Lovisa]{Elenius, Lovisa}}{1889--\allowbreak 1920}
Isak Johansson Jungarå, senare Ruotsala, \textborn 16.01.1862 på Jungarå, gifte sig 1883 med Lovisa Mattsdotter Sandberg, \textborn 08.04.1857 på Finskas. Efter att till en början bott på Jungarå inropades på auktion år 1889 ett hemman om 5/48 mtl av bondeänkan Brita Jonasdotter, \textborn 1834, ett hemman belagt med en grundränta om 3 rubel och 35 kopek. Här bosatte sig nu familjen med 2 söner. Senare, år 1906, köpte han tillsammans med sin yngsta bror Henrik en lägenhet på Tollikko omfattande 164 ha, av vilket 53 ha odlat. Efter förvärvet av hemmanet på Ruotsala började bostaden byggas år 1900. Nya uthus byggdes några år senare.

Isak var invald i kommunalnämnden och förmyndarnämnden.
\begin{jhchildren}
  \item \jhperson{\jhname[Leander]{Ruotsala, Leander}}{06.02.1886}{}
  \item \jhperson{\jhbold{\jhname[Johannes]{Ruotsala, Johannes}}}{24.12.1887}{}
  \item \jhperson{\jhname[Emil]{Ruotsala, Emil}}{03.04.1890}{}
  \item \jhperson{\jhname[Anna Lovisa]{Ruotsala, Anna Lovisa}}{29.07.1892}{}
  \item \jhperson{\jhname[Emil]{Ruotsala, Emil}}{24.07.1894}{}
  \item \jhperson{\jhname[Alfred]{Ruotsala, Alfred}}{26.10.1896}{}
  \item \jhperson{\jhname[Hilda Maria]{Ruotsala, Hilda Maria}}{13.03.1899}{}
\end{jhchildren}



\jhhouse{Mellangård}{8:11b}{Ruotsala}{13}{18}


\jhhousepic{Greta Back.jpg}{Greta och Tor-Erik Back}

\jhoccupant{Back}{\jhname[Dan]{Back, Dan} \& \jhname[Ann Kristin]{Back, Ann Kristin}}{1990--}
Dan Erik Johan, \textborn 10.02.1966, gift år 2000 med Ann-Kristin Lindgren, \textborn 28.07.1962 från Jakobstad. Dan övertog hemgården 1990 av sina  föräldrar Tor-Erik och Greta Back. De bor kvar på gården.

På gården bedrivs intensiv mjölkproduktion i en nybyggd ladugård som succesivt utvidgats, 1996 och senast 2009. Plats finns nu för ca 80 mjölkkor och robotmjölkning är införd. Besättningen är nu den största i Jeppo. Den odlade arealen har utökats genom köp och nyodling och uppgår idag till ca 90 ha. Driften sköts med gårdens folk och en (1) anställd person.

Dan genomgick lantbruksutbildning vid Korsholms skolor 1983-85. Han var som ung aktiv inom Jeppo Ungdomsorkester, Unga Producenter i Nykarlebynejden, bl.a som ordförande, och samtidigt aktiv som skidlöpare inom Minken och har deltagit 11 gånger i Vasaloppet. ``Tina'' har fått utbildning i Lannäslunds husmodersskola och var anställd på Mejeriandelslaget Milka innan hon flyttade till Jeppo.

Ann-Kristin förde två döttrar in i familjen.
\begin{jhchildren}
  \item \jhperson{\jhname[Nina Särkiniemi]{Back, Nina Särkiniemi}}{15.05.1989}{}, frisörska
  \item \jhperson{\jhname[Hanna Särkiniemi]{Back, Hanna Särkiniemi}}{09.07.1991}{}, examen fr Lannäslund
\end{jhchildren}


\jhoccupant{Back}{\jhname[Greta]{Back, Greta} \& \jhname[Tor-Erik]{Back, Tor-Erik}}{1970--\allowbreak 1990}
Tor-Erik Back, \textborn 05.02.1932 på Back i Jeppo, gifte sig 11.05.1963 med Greta Elisabet Johannesdotter Elenius, \textborn 23.06.1933 på Ruotsala i Jeppo.

Greta har arbetat på hemgården förutom 1952--53 då hon besökte Kronoby folkhögskola och 1954--55 då hon sökte sig till  Svenska Folkakademin i Borgå. Hon har varit aktiv  inom Jeppo Marthaförening fr.o.m. 1965. I kyrkokören har hon sjungit 1969--\allowbreak 2013. I Nykarleby stad har hon suttit med i kulturnämnden, hälsovårdsnämnden och äldrerådet. Tor-Erik gick i slöjdskola på Keppo 1953--54 och arbetade på Kiitola och Keppo fram till giftermålet 1963.

Makarna övertog gården 1970 av Gretas far Johannes Elenius. Under åren mekaniserades gården och en ny ladugård byggdes 1977 för 16 mjölkkor och ungdjur, inklusive rörmjölkning och mekanisk utgödsling.

Barn: \jhbold{Dan} Erik Johan, \textborn 10.02.1966.


\jhoccupant{Elenius}{\jhname[Johannes]{Elenius, Johannes} \& \jhname[Lydia]{Elenius, Lydia}}{1921--\allowbreak 1970}
Johannes Isaksson (Ruotsala), \textborn 24.12.1887 på Jungarå, gifte sig 23.09.1923 med Susanna Lydia Romar, \textborn 18.05.1894 på Romar. Som 2-årig flyttade han 1889 med sina föräldrar till Ruotsala och familjen tog detta som sitt efternamn. Efternamnet byttes senare till \jhbold{Elenius}.

1908 emigrerade Johannes till Michigan i USA och arbetade med skogsarbete, ett arbete som han också utförde i Alaska. Han antog amerikanskt medborgarskap och blev 1918 inkallad till armén där avsikten var att bli sänd till kriget i Europa. Emellertid insjuknade han i dubbelsidig lunginflammation och var så nära döden att hans kusin i Felch kallades till sjukbädden. Han överlevde, slapp kriget och 1920 kom han på besök till hemlandet för att ännu en gång få träffa sina åldrande föräldrar. I det skedet erbjöds han att överta halva hemmanet Mellangård, vilket skedde 1921. Det omfattade 0,0523 mtl och priset var 7000 mark. 1922 byggdes nytt fähus och 1924 ny bostadsbyggnad, som nu byggdes på landsvägens västra sida. 1923 hade han gift sig med Lydia Romar och familjen satsade på mjölkproduktion. Han var samtidigt ledamot i taxeringsnämnden och deltog som styrelsemedlem i lantmannagillet.

Under år 1944 inhyste gården krigsfången Vasilij, hemmahörande i Moskva. Banden till familjen stärktes, vilket gjorde det påtvingade avskedet smärtsamt.

Johannes problem med lungorna (astma) gjorde att de tidvis måste arrendera ut en del av hemmanet till Artur Jungar. Trots dessa problem överlevde han sin hustru med 20 år.
\begin{jhchildren}
  \item \jhperson{\jhbold{\jhname[Greta]{Elenius, Greta}} Elisabet}{23.06.1933}{}
  \item \jhperson{\jhname[Rolf Johan]{Elenius, Rolf Johan}}{25.08.1936}{15.09.1936}
\end{jhchildren}

Lydia \textdied 20.08.1955  ---  Johannes \textdied 14.07.1975



\jhbold{Mellangård på Ruotsala före 1924.}
Bostadsbyggnaden och ladugård för hemmanet stod före 1924 på östra sidan av landsvägen och beboddes av föregående släktled från 1889.\jhvspace{}


\jhoccupant{Ruotsala}{\jhname[Isak]{Ruotsala, Isak} \& \jhname[Lovisa]{Ruotsala, Lovisa}}{1889--\allowbreak 1921}
Isak Johansson Ruotsala, \textborn 16.01.1862 på Jungarå,  gifte sig 1883 med Lovisa Mattsdr. Sandberg, \textborn 08.04.1857 på Finskas hemman. De bodde till en början i Isaks barndomshem, men efter att Isak 1889 inropat 5/48 mtl på Ruotsala, av änkan Brita Jonasdr., \textborn 1834, flyttade familjen dit. Tillsammans med sin bror Henrik köpte de Mauriz von Essens lägenhet Koski på Tollikko för 20.000 mark, en lägenhet som i två delar tillhört von Essen från 1838 resp. 1858.

Isak var medlem i förmyndarnämnden, församlingens boställsnämnd och inte minst i kommunalnämnden. Se Ruotsala nr 17.

Lovisa \textdied 22.05.1942  ---  Isak \textdied 06.09.1944


Greta Back delar med sig av sina minnen:

\jhsubsection{Jungar by, mellan Böösasbäcken och Palobäcken på 1940-talet}

Som barn var man ej medveten om hur svårt det var på alla sätt. Vårt land var i krig från 1939-1944 så gott som hela tiden med uppehåll från våren 1940 till midsommar 1941. Det var brist på allt, men i en by, som mest bestod av jordbrukare, led man ej någon nöd. Frukter och kaffe och sånt som kom från utlandet fanns ej. Jag vet ej när jag åt de första apelsinerna eller bananerna f.ex. Sånt fanns ej under kriget. Rågen rostades och användes istället för kaffe.

Man odlade de vanligaste sädesslagen ; råg, korn o havre och potatis. Vete försökte man också odla för att baka vetebröd, men det var svårare att odla för det har ju längre växttid. Frosten kom ibland och tog skörden. Lin odlade man också i mindre skala. Det var mycket arbete innan det blev till garn. Det skulle rötas, brotas och klyftas. Sen sändes det till spinneriet för att bli till garn.

Det fanns ca 18 bönder i byn och alla hade några kor, ett par grisar och några får. Det fanns några som födde upp grisungar, som de sålde åt andra bönder, så alla hade sin hushållsgris och sina höns. Mjölken fördes till mejeriet i byn och förvandlades till smör, kärnmjölk och skummjölk. Skummjölken (klanon som vi kallade den) kom tillbaka och gavs åt småkalvarna. Nu kallas den för fettfri mjölk! De som ej hade jordbruk hämtade mjölken och smöret från mejeriet. Men de måste förstås ha köpkort, för ingenting fick man köpa utan kort. Där kommer folkförsörjningen in. Allting var ransonerat med kort och licenser. Det fanns kort för olika produkter med kuponger som skulle klippas när man köpte sin ranson. Man hade ofta ärende till folkförsörjningen. Där fick man tex. licens om man skulle ha nya skor m. m.  Skor var nog en sak som var svårt att få köpa.

Jag minns hur vi barn gick med pappersskor som hade träsulor. Det var ej så bra när det regnade och på vintern behövde man nog licens för att få bättre skor. Folkförsörjningskansliet var inrymt i Leonard Liljeqvist gård här i byn. Han var ledande för folkförsörjningen  i Jeppo. Han hade tidigare haft en liten butik i samma rum, som jag har lite minne av. Vi hade också butik i byn. Det var Handelslagets filial, SOK, som det hette på den tiden.

På jordbrukets område har det skett stora förändringar sen 40-talet. Det fanns inga maskiner, traktorer m. m.  Allt skedde med hästkraft, plöjning, harvning, slåtter m. m. Inga mjölkmaskiner, tvättmaskiner. I slutet av 40-talet kom de i samband med vattenledningen. Den första vattenledningen kom till byn 1948. Jag minns hur de borrade hål i stockar och genom den kom vatten till gårdarna. Den kom från Gunnarkangan och grävdes med handkraft. Nu är vi 40 personer i byn. 1(en) bondgård med mjölkproduktion. 1 (en) med diko-köttproduktion, 4 med potatisodling. På 40-talet fanns i byn över 100 personer och många barn.

Vi som var barn på 1940-talet hade alla våra fritidsaktiviteter och lekar i den egna byn. På vintrarna drog vi våra skidspår på åkrarna och åbacken blev vår hopp -och rännbacke. Vi gick till skola på Mietala småskola och Gunnar folkskola, 1 – 1½ km. Det var att plumsa i snön för det var ingen som plogade vägen på den tiden och vintrarna var kalla i början av 1940-talet.

Då våren kom var det mest utelekarna som gällde; ``strinta mora'', ”risti o rona”, bollspel, ``kallika'' m. fl.  På sommaren hade byn sin egen ``idrottsplan'' på Torrlandet. Det var ett samfällt område mellan Jungar-gårdarna och järnvägen. Där fanns en såg, så det fanns mjuk spån för höjd -och längdhopp m. m.  Det var främst pojkarna som använde sportplanen. Det fanns flera ställen vid ån där det gick att simma, men den bästa simstranden fanns på andra sidan ån. Vi kallade det för ``Grundet'' för det var så långgrunt där.

Nu var det så att alla bönder i byn hade åkerjord på Holmen mitt emot byn. Det var en lång väg till Holmen via Silvast med hästslåttermaskinen m. m. På sommaren lade bönderna ut en stegbro mellan stenarna i forsen så det gick att gå över. Den bron använde vi barn också då vi for till vår fina simstrand. Vi fick bara gå en liten bit norr om forsen. Det blev ett riktigt fritidsområde. En sommar hade lantbruksklubben 4H sin sommarutfärd dit med program, servering och simning. En annan aktivitet som var mycket viktig vid forsen var klädtvättningen. Man tog en stor gryta med sig som placerades mellan stenarna att värma vattnet i. Så hade man bunkar, såar och tvättbräde. Vatten fanns i ån. Hela byns kvinnor skötte på detta sätt tvätt på sommaren. Vi barn var naturligtvis med.

En sak som jag särskilt minns; Det var på våren 1944 då vi hos oss fick en ryssfånge till hjälp i jordbruket. Pappa hämtade honom från stationen i Jeppo, och han var hos oss hela sommaren. Han lärde sej alla göromål som hörde till både på åker och i fähus. I byn fanns det krigsfångar i 5-6 gårdar. Vår Vasili var en 25 år gammal Moskva-gosse. Visst tyckte min pappa att han skröt litet för mycket på de ryska ledarna. Han kunde lite finska o lärde sig lite svenska. Vi några ryska ord. Men vi trivdes nog bra med honom och han med oss tror vi, för det var ett svårt avsked då han skulle fara, då det blev vapenstillestånd på hösten.

Förutom i byn fanns det fångar i hela Jeppo. De flesta kunde ej cykla, så det var många gånger som vi hade roligt åt dem då de lärde sej cykla. Vår Vasili lärde sej också att cykla på min mammas cykel. Det var ej så roligt för henne, han nötte säkert upp cykelringar som det var svårt att få köpa. Fångarna  cyklade och träffa andra fångar. Vasili hade ett eget rum och tog också kamrater med till oss.  Då kom hösten och innan Vasili hade farit kom Kemijärvievakuerade till byn. Vi måste ta emot en familj och Vasili måste flytta från sitt rum och ge rum åt de evakuerade. Han fick ju bo i vårt finrum, med det tyckte han inte om.

Det var Elvi Keränen och hennes flickor Raili o Liisa Kunelius som kom till oss. Hennes man bodde också en tid hos oss och hans pojke Sauli. Jag minns att han tyckte om att vara ute med min pappa o köra med häst och släde. På 40-talet var det vanligt att man for med sitt vete till Voltti, för där fanns en kvarn som malde vete. Han var en gång med min pappa på en sådan resa.

Det var många hem i byn som tog emot Kemijärvifamiljerna. De bodde här till våren 1945 och vi barn fick finska kamrater. Vi fick lära oss finska. Raili var lite äldre än jag. Hon flyttade senare till Borås i Sverige och gifte sej där. Hon har varit till oss ett par gånger på genomresa, men nu har jag tyvärr ingen kontakt med henne.

1940-talet var en svår tid i Finland, men på ett vis gjorde det att människorna kom närmare varandra. Man kom samman i arbetet t.ex vid tröskningen. Det fanns tröskverk i byn som alla hade del i. Tröskdagen blev en samling för hela byn, en talkodag. Då bjöd husmor på gården på mat och kaffe vad det var för kaffe (kanske råg), men gott var det. Man brukade tala om ``tryskankalaase''. Vi barn fick också vara med, vi tumlade om i halmen, som kastades upp i ladan eller stacken. Sen fick vi också äta sen andra gjort det.

En sak som man också gjorde som hörde till kristiden var att man gjorde potatismjöl. Potatisen revs, sen sköljdes det med vatten och fick blötas. Jag vet inte riktigt hur det gick till, men potatismjölet blev på botten av bunken eller sån. Man odlade också sockerbetor som man försökte laga sirap av. Det blev ej så gott som den vi köper idag, men visst var den mörkbrun o lite söt då den kokades länge.

Detta var några händelser från barndomen som fastnat i mitt minne. Den som var vuxen på 1940-talet skulle säkert ha upplevt det på ett annat sätt. Som barn kunde man ej ta till sej allt som kriget förde med sej, men det var kanske lika bra.

Greta Back f. 23 juni 1933 på Ruotsala



\jhhouse{Ruotsala}{8:69}{Ruotsala}{13}{19}


\jhhousepic{205-05782.jpg}{Marvin Bränn}

\jhoccupant{Bränn}{\jhname[Marvin]{Bränn, Marvin}}{2009--}
Marvin Bränn, \textborn 20.06.1988 i Ytterjeppo, köpte huset av Holger och Valdine Liljeqvist arvingar år 2009. Han har fått lantbruksutbildning vid Lannäslund och är anställd som farmarbetare vid Fäboda.\jhvspace{}


\jhoccupant{Liljeqvist}{\jhname[Valdines arv.]{Liljeqvist, Valdines arv.}}{1980--\allowbreak 2008}
Syskonen Leif och Solveig Liljeqvist bodde tillsammans med mamman  Valdine efter Holgers död. Solveig skötte daghemsbarn i hemmet under flera år. Leif har varit intresserad av datafunktioner och har arbetat inom Mirka, innan han för en period om 4 år reste över till Sundsvall för att förkovra sig i ämnet. Återkommen till Vasa fick han småningom problem med ögonen, omskolade sig till väktare och fick tjänst i Karleby, där han kvarstod till pensioneringen.


\jhoccupant{Liljeqvist}{\jhname[Holger]{Liljeqvist, Holger} \& \jhname[Valdine]{Liljeqvist, Valdine}}{1945--\allowbreak 1980}
Holger gifte om sig 1950 med Elsa Valdine Sandvik, \textborn 30.08.1926 i Terjärv. Utöver arbetet på hemmanet började Holger arbeta på Keppo pälsfarm. Han ansvarade bl.a. för utplacering av fällor för att infånga djur som rymt ur sina burar. Holger var också postutdelare in på 1960-talet i det södra distriktet och han var kännspak med sin moped utrustad med postväskor och sin ``flygarluva'' av läder.
\begin{jhchildren}
  \item \jhperson{\jhname[Kaj Roger]{Liljeqvist, Kaj Roger}}{29.06.1950}{}
  \item \jhperson{\jhname[Mona Carita]{Liljeqvist, Mona Carita}}{29.06.1950}{}
  \item \jhperson{\jhbold{\jhname[Solveig]{Liljeqvist, Solveig}} Marianne}{03.04.1952}{}
  \item \jhperson{\jhname[Tage Thomas Mikael]{Liljeqvist, Tage Thomas Mikael}}{17.03.1955}{}
  \item \jhperson{Stefan \jhbold{Leif} Ole}{03.06.1956}{}
  \item \jhperson{\jhname[Gun Ann-Christine]{Liljeqvist, Gun Ann-Christine}}{04.08.1961}{}
\end{jhchildren}

Holger \textdied 24.02.1980  ---  Valdine \textdied i nov. 2008


\jhoccupant{Liljeqvist}{\jhname[Holger]{Liljeqvist, Holger} \& \jhname[Gerda]{Liljeqvist, Gerda}}{1939--\allowbreak 1945}
Holger Martin Liljeqvist, \textborn 29.06.1915, gifte sig 1934 med Gerda Emilia Forsman, \textborn 18.05.1916. De övertog 1/3 av Johan och Mina Forsmans hemman om 9 ha jord och 14 ha skog (se Ruotsala nr 24). De levde på sitt lilla jordbruk och på gården föddes 6 barn.
\begin{jhchildren}
  \item \jhperson{\jhname[Göran Martin]{Liljeqvist, Göran Martin}}{17.09.1934}{}
  \item \jhperson{\jhname[Tor Viking]{Liljeqvist, Tor Viking}}{06.01.1936}{27.02.1968}
  \item \jhperson{\jhname[Rolf Johan]{Liljeqvist, Rolf Johan}}{25.01.1939}{17.02.1948}
  \item \jhperson{\jhname[Etel Maria Birgitta]{Liljeqvist, Etel Maria Birgitta}}{27.04.1941}{}
  \item \jhperson{\jhname[Fredrik Vilhelm]{Liljeqvist, Fredrik Vilhelm}}{02.02.1943}{}
  \item \jhperson{\jhname[Kerstin Ingegerd Marita]{Liljeqvist, Kerstin Ingegerd Marita}}{28.02.1944}{}
\end{jhchildren}

Göran började som liten bo hos sin morfar och mormor. Gerda insjuknade och dog 25.11.1945.



\jhhouse{Nykvist}{8:100}{Ruotsala}{13}{190}


\jhoccupant{Nyqvist}{\jhname[Anders]{Nyqvist, Anders} \& \jhname[Maria]{Nyqvist, Maria}}{}
Anders Nykvist, \textborn 24.06.1877, gifte sig med Maria Lovisa, \textborn 27.07.1880. Anders och Maria bodde i ett hus på östra sidan om vägen strax efter avtaget upp till Svartbackan. Huset revs på 1960-talet av Uno Gunell. Efter Marias död 12.01.1930 levde Anders ensam en tid innan han gifte sig med Hilma Alexandra Nybonde, \textborn 07.08.1892 från Pedersöre. Hon flyttade till Jakobstad 08.10.1945 och äktenskapet upplöstes 16.10.1945. Senare flyttade hon till Sverige där hon på nytt gifte sig 04.03.1950.

På det lilla hemmanet livnärde de sig på 3-4 kor. Anders var en storväxt och kraftig karl som alltid gick med en pipa i munnen oberoende om den var tänd eller inte. I hans 2:a äktenskap föddes inga barn.
\begin{jhchildren}
  \item \jhperson{\jhname[Anders Wilhelm]{Nyqvist, Anders Wilhelm}}{05.02.1904 i Amerika}{}
  \item \jhperson{\jhname[Johannes Arthur]{Nyqvist, Johannes Arthur}}{22.07.1906}{}
  \item \jhperson{\jhname[Alfred Edvin]{Nyqvist, Alfred Edvin}}{18.02.1912}{}
  \item \jhperson{\jhname[Emil Walfrid]{Nyqvist, Emil Walfrid}}{06.07.1914}{}
  \item \jhperson{\jhname[Gustaf Sigfrid]{Nyqvist, Gustaf Sigfrid}}{17.12.1919}{}
  \item \jhperson{\jhname[Ellen Maria]{Nyqvist, Ellen Maria}}{28.11.1922}{26.07.1923}
\end{jhchildren}

Anders \textdied 30.10.1957.



\jhhouse{Solbacka}{7:32}{Jungar}{13}{20}


\jhhousepic{206-05783.jpg}{Håkan och Lenelis Cederström}

\jhoccupant{Cederström}{\jhname[Håkan]{Cederström, Håkan} \& \jhname[Lenelis]{Cederström, Lenelis}}{1975--}
Håkan Gustav Cederström, \textborn 07.04.1951 i Ytterjeppo, gifte sig 12.07.1975 med Lenelis Lindgren, \textborn 21.02.1955 på Jungar. År 1975 köpte makarna fastigheten av Esko Hautamäki. Han hade köpt fastigheten av Olof och Gretel Laxen, men aldrig bott i den. Istället hade den hyrts ut till bl.a. Sune Jungerstam och Herman och Paula Broo. Efter köpet har Håkan och Lenelis utvidgat och grundligt renoverat bostaden. Ekonomiebyggnaden har fungerat som värmecentral för bostaden och lager för Håkans rörinstallationsfirma.

Håkan genomgick bilmekanikerutbildning i Jakobstad 1966--69. Därefter arbetade han på Boris Lindéns bilverkstad i Silvast 1969--71. Läsåret 1972/73 skaffade han sig rörmokarutbildning vid Korsnäs kurscentrals VVS-linje. 1973--80 hade han anställning på Heikius Rör Ab i Oravais. Egen VVS firma startade han 1980 och har i huvudsak haft närområdet som sitt verksamhetsfält. År 1979 startades också en rävfarm, som numera är nerlagd.

Lenelis skaffade sig hjälpskötarutbildning vid Ekenäs Sjukvårdsskola år 1975. Hennes huvudsakliga gärning har hon ändå utfört inom församlingen, först för Jeppo församling och efter sammanslagningen av Nykarlebynejdens församlingar, hos Nykarleby församling.  Kansliarbetet har fallit på hennes lott och i perioder också som ledare av dagklubben. ``Källan'' är ett begrepp som i Jeppo förknippas med den hjälporganisation som Lenelis i tiden var en av initiativtagarna till, en verksamhet som ännu är vital även om målen växlat med förändringarna i världen.
\begin{jhchildren}
  \item \jhperson{\jhname[Linda Maria]{Cederström, Linda Maria}}{17.02.1977}{}
  \item \jhperson{\jhname[Jonas Gustav]{Cederström, Jonas Gustav}}{23.07.1980}{}
  \item \jhperson{\jhname[Sebastian Karl-Gustav]{Cederström, Sebastian Karl-Gustav}}{28.06.1991}{}
\end{jhchildren}


\jhoccupant{Laxén}{\jhname[Olof]{Laxén, Olof} \& \jhname[Gretel]{Laxén, Gretel}}{1958--\allowbreak 1966}
Sven Olof Laxén, \textborn 26.06.1933 på Jungar, gifte sig 1957 med Gretel Ingegerd Dahlkar, \textborn 20.01.1937 i Vörå. Olof utbildade sig till målare, ett arbete som var efterfrågat under den efterkrigstida uppbyggnadsperioden och fortsättningsvis är. Han och familjen köpte sig detta hem på Svartbackan av Valter och Rakel Jungerstam 1958. Familjen bodde här fram till 1966. Olof var förutom en uppskattad målare också en skicklig schackspelare. Han blev bl.a. finsk mästare i korrespondens schack år 1967.
\begin{jhchildren}
  \item \jhperson{\jhname[Rosita Maria]{Laxén, Rosita Maria}}{1958}{}
  \item \jhperson{\jhname[Mona Christina]{Laxén, Mona Christina}}{1963}{}
\end{jhchildren}

Familjen har bott på ett flertal andra platser i Jeppo innan de den 10.7.1973 flyttade till Vörå.


\jhoccupant{Jungerstam}{\jhname[Walter]{Jungerstam, Walter} \& \jhname[Rakel]{Jungerstam, Rakel}}{1949--\allowbreak 1956}
Walter Edvin Jungerstam, \textborn 21.10.1916 på Jungar gifte sig 18.06.1947 med Hjördis Rakel Helena Häggström, \textborn 31.05.1923 i Nedervetil. De tog beslutet att bosätta sig på Svartbackan, som under flera hundra år varit en bosättningsplats, men där bosättningen småningom tunnats ut. År 1948/49 uppfördes både bostad och ladugård. De vintertida
transportsvårigheterna gjorde att de beslöt att flytta ner till åbacken på den hemmansdel på Jungar som han tilldelats efter att hemgården delades med brodern Åke Jungerstam. Det skedde 1956 och fastigheten på Svartbackan såldes (se Jungar nr 27 och 11).



\jhhouse{Nygård}{8:100}{Ruotsala}{13}{21}


\jhhousepic{207-05784.jpg}{Benny Gunell}

\jhoccupant{Gunell}{\jhname[Benny]{Gunell, Benny}}{2006--}
Benny Fredrik Gustav Gunell, \textborn 19.07.1973 på Ruotsala övertog 2002 hemmanet av sina föräldrar. Han är utbildad bioenergi-ingenjör och har sedan 2001 fungerat som lantbrukssekreterare i Nykarleby med stationering i Jeppo Kommuns tidigare kansli. Han fortsätter brukande av lägenheten med potatisodling  och lever ogift, bosatt på radhusområdet i Jeppo, medan föräldrarna bor kvar i det hus som står på Ruotsala.


\jhoccupant{Gunell}{\jhname[Boris]{Gunell, Boris} \& \jhname[Gunborg]{Gunell, Gunborg}}{1978--\allowbreak 2006}
Boris Gustav Gunell, \textborn 29.09.1946 på Ruotsala, gifte sig 29.07.1972 med Gunborg Kerstin Birgitta Forsbacka, \textborn 22.09.1949 från Rökiö, Vörå. De övertog Boris' hemgård om 12 ha odlad jord och 40 ha skog. Mjölkproduktionen fortsatte på gården fram till 1997. Samtidigt arbetade Boris på Keppo pälsdjursfarm till 1982. År 1997 fick han anställning på Mirka, där han arbetade fram till pensionen.

Boris gick 1963/64 i Evangeliska folkhögskolan och här framkom tydligt hans musikaliska intresse och begåvning. Han började traktera gitarr och breddade vartefter förmågan att hantera flera olika instrument. Tillsammans med de andra jeppopojkarna Sten Sandberg, Nils Strand och Bjarne Finskas bildades kort därefter popbandet The Teddybears. Efter den perioden har han fortsatt med musicerandet i många	olika former med flera olika instrument och i flera musikgrupper i olika sammanhang och glatt många Jeppobor och andra. Gunborg har efter mjölkproduktionens upphörande arbetat på Jeppo Potatis fram till pensioneringen. Innan giftermålet arbetade hon som fabrikssömmerska. Huset renoverades år 1972.
\begin{jhchildren}
  \item \jhperson{\jhbold{\jhname[Benny]{Gunell, Benny}} Fredrik Gustav}{19.07.1973}{}
  \item \jhperson{\jhname[Peter Andreas]{Gunell, Peter Andreas}}{23.01.1977}{}
\end{jhchildren}


\jhoccupant{Gunell}{\jhname[Uno]{Gunell, Uno} \& \jhname[Gerda]{Gunell, Gerda}}{1940--\allowbreak 1978}
Uno Leander Gunell, \textborn 11.01.1912 på Ruotsala gifte sig 02.07.1939 med Gerda Viola Jungarå\textborn 23.07.1916 på Jungarå. Som 19-åring gick Gerda i Evangeliska Folkhögskolan, som då var stationerad på Keppo Gård. Det var läsåret 1935/36. Makarna har hela sitt verksamma liv varit bönder på hemmanet med mjölkproduktion som huvudsaklig inkomstkälla.
\begin{jhchildren}
  \item \jhperson{\jhname[Inger Viola Margareta]{Gunell, Inger Viola Margareta}}{23.08.1943}{}
  \item \jhperson{\jhbold{\jhname[Boris]{Gunell, Boris}} Gustav}{29.09.1946}{}
\end{jhchildren}

Uno \textdied 28.04.1990 ---  Gerda, som när detta skrivs (2016) skall fylla 100 år, är nu den äldsta levande Jeppobon och har genom åren varit en aktiv person i nykterhetsrörelsen och i Marthaföreningen, ofta som sekreterare.

Gerda blev 100 år + 6 dagar, \textdied 29.07.2016.



\jholdhouse{Den gamla gården på Nygård}{8:100}{Ruotsala}{13}{121}


\jhoccupant{Gunell}{\jhname[Henrik \& Katarina]{Gunell, Henrik \& Katarina} \& \jhname[Ida]{Gunell, Ida}}{1906--\allowbreak 1940}
Henrik Mattsson Gunell, \textborn 03.03.1882 på Gunnar, gifte sig år 1901 med Katarina Jakobsdotter Forsgård, \textborn 29.01.1882 på Mietala. De for till Amerika efter giftermålet och där föddes 3 barn. 1906 återkom familjen och samma år köptes 5/48 mtl av Erik Heikkilä och hans hustru Sanna på Ruotsala. Köpekillingen var 11.100 mk och de erhöll då ca 50 ha varav ca ½ var odlat. Nu byggdes också ett nytt hus.

Emellertid dog hustrun Katarina den 15.09.1918 i tuberkulos. Efter några år, 1921, gifte han sig med hennes lillasyster Ida Johanna, \textborn 23.06.1889. I det äktenskapet föddes det 2 barn. Henrik var aktiv i Jungar Andelsmejeris styrelse en kort tid.
\begin{jhchildren}
  \item \jhperson{\jhname[Ellen Irene]{Gunell, Ellen Irene}}{25.04.1902}{}
  \item \jhperson{\jhname[Henrik Elmer]{Gunell, Henrik Elmer}}{24.01.1904}{}
  \item \jhperson{\jhname[Signe Katarina]{Gunell, Signe Katarina}}{02.04.1906}{}
  \item \jhperson{\jhname[Jenny Cecilia]{Gunell, Jenny Cecilia}}{06.10.1908}{}
  \item \jhperson{\jhbold{\jhname[Uno Leander]{Gunell, Uno Leander}}}{11.01.1912}{}
  \item \jhperson{\jhname[Eira Johanna]{Gunell, Eira Johanna}}{19.04.1923}{}
  \item \jhperson{\jhname[Eskil Edvin]{Gunell, Eskil Edvin}}{26.08.1928}{}
\end{jhchildren}

Ida \textdied 19.02.1969  ---  Henrik \textdied 02.05.1972

Under tiden hade Henriks yngre bror Anders Villiam emigrerat till USA i början på 1920-talet och där gift sig 1923 med Maria Albertina Henriksdr., \textborn 21.04-1895 på Levelä i Nyk.lk. När de återkom från USA övertog de ½ av Henriks och Villiams föräldrars,  Matts och Sannas hemman. Redan år 1933 sålde de sin ½ lägenhet till brodern Henrik som nu hade den av Erik Heikkilä köpta lägenheten plus ½ av hemgården.

Deras yngsta bror Johannes Joel, \textborn 24.10.1906, hade i sin tur övertagit den andra ½  av föräldragården, men år 1929  överlämnade han sin andel till sin äldre syster Anna Gustava och emigrerade till Canada, där han senare gifte sig med Taimi Vilhelmina Aalto. Anna var gift med Johannes Strand och de flyttade nu in i den långa mangårdsbyggnadens södra ända.


\jhoccupant{Strand}{\jhname[Anna]{Strand, Anna} \& \jhname[Johannes]{Strand, Johannes}}{1929--\allowbreak 1945}
Anna Gustava Gunell, \textborn 04.07.1896 på Ruotsala, gifte sig 1927 med Johannes Andersson Strand, \textborn 09.12.1886 på Jungar. Johannes hade i 2 perioder om sammanlagt 10 år arbetat i Ironwoods gruvor i USA . Som så många andra hade han knäckt sin hälsa i denna ohälsosamma miljö och återkommen till Jeppo övertog han tillsammans med sin hustru ½ av hennes hemgård på Ruotsala.

Efter att han avlidit 1945, övergick hemmanet i enda sonen Mannes ägo och familjen bodde kvar i huset. 1970 byggde Manne ett nytt hus längre söderut och på östra sidan av landsvägen (se nr 30) och mor Anna följde med i flytten, som skedde 1972.
\begin{jhchildren}
  \item \jhperson{\jhname[Saga Verna Alfhild]{Strand, Saga Verna Alfhild}}{12.12.1925}{}
  \item \jhperson{\jhbold{\jhname[Manne]{Strand, Manne}} Johannes}{18.07.1927}{}
  \item \jhperson{\jhname[Rut Ann-Mari]{Strand, Rut Ann-Mari}}{02.02.1929}{}
  \item \jhperson{\jhname[Tense Viola]{Strand, Tense Viola}}{02.08.1932}{}
  \item \jhperson{\jhname[Berit Alice]{Strand, Berit Alice}}{29.06.1933}{}
  \item \jhperson{\jhname[Birgitta Valborg]{Strand, Birgitta Valborg}}{05.02.1937}{}
  \item \jhperson{\jhname[Gun-Lis Viva Susanna]{Strand, Gun-Lis Viva Susanna}}{02.02.1941}{}
\end{jhchildren}

Johannes \textdied 17.07.1945  ---  Anna \textdied 22.11.1991


\jhoccupant{Gunell}{\jhname[Matts]{Gunell, Matts} \& \jhname[Sanna]{Gunell, Sanna}}{1892--\allowbreak 1927}
Matts Mattsson Gunell, \textborn 09.02.1852 på Bärs, gifte sig 1878 med Sanna Henriksdr., \textborn 13.08.1861 på Gunnar. De bodde till en början på hennes hemgård, men via köp av en lägenhet om 19/192 mtl, på  Lassila, som de sålde ett par år senare, reste Matts till USA. Efter några år, 1892, återkommer han och köper en lägenhet om 5/96 mtl på Ruotsala av änkan Sofia Ruotsala. 1896 förstoras hemmanet med en lika stor lägenhet inköpt av Erik Forsman. Detta blir boplatsen för familjen och där den får sin försörjning (nr 121).
\begin{jhchildren}
  \item \jhperson{\jhname[Ida Maria]{Gunell, Ida Maria}}{10.09.1879}{}
  \item \jhperson{\jhbold{\jhname[Henrik]{Gunell, Henrik}}}{03.03.1882}{}
  \item \jhperson{\jhname[Matts Leander]{Gunell, Matts Leander}}{06.08.1888}{}
  \item \jhperson{\jhname[Sanna Sofia]{Gunell, Sanna Sofia}}{28.04.1893}{}
  \item \jhperson{\jhname[Jenny Emilia]{Gunell, Jenny Emilia}}{14.10.1894}{}
  \item \jhperson{\jhbold{\jhname[Anna Gustava]{Gunell, Anna Gustava}}}{04.07.1896}{}
  \item \jhperson{\jhname[Elna]{Gunell, Elna}}{25.10.1898}{}
  \item \jhperson{\jhbold{\jhname[Anders Villiam]{Gunell, Anders Villiam}}}{10.08.1900}{}
  \item \jhperson{\jhname[Ellen Irene]{Gunell, Ellen Irene}}{05.04.1903}{}
  \item \jhperson{\jhbold{\jhname[Johannes Joel]{Gunell, Johannes Joel}}}{24.10.1906}{}
\end{jhchildren}
Matts \textdied 13.08.1926  ---  Sanna \textdied 13.08.1927

Den hemmansdel som inköptes av Erik Forsman ägdes på 1880-talet av dennes far Jakob, \textborn 1818 och mor Lisa, \textborn 1815.



\jhhouse{Nylund}{7:15}{Jungar}{13}{22}


\jhoccupant{Palosaari}{\jhname[Susanne]{Palosaari, Susanne} \& \jhname[Markku]{Palosaari, Markku}}{1989--}
Senny Nylund gav den 28.10.1989 lägenheten till sin syster Jennys äldsta barnbarn, Susanne f. Wärnman, gift med Markku Palosaari. Makarna bor i Tammerfors och använder huset för sommarvistelse i Jeppo.\jhvspace{}


\jhhousepic{208-05785.jpg}{Nylunds, numera sommarvistelse}

\jhoccupant{Nylund}{\jhname[Senny]{Nylund, Senny}}{1954--\allowbreak 1989}
Senny Hjördis Jakobsdr. Nylund, \textborn 12.02.1913 på Jungar. Hon var yngst av systrarna och hennes lillabror Bruno, född tio år senare, blev endast 10 månader gammal. Hon gick 1929/30 i Evangeliska folkhögskolan och efter krigen gick hon i Korsholms husmoderskola 1945/46. Under kriget fungerade Senny en tid som kanslist på folkförsörjningskansliet och en kort tid på Jeppo Kraft som kanslist. Hon var också en tid expedit på Varma-butiken i Silvast. Hon verkade därefter som föreståndare för handelslagets filial i Jungar fram till 1956. Det var under denna tid, 18.05.1954, som hon löste in fastigheten av dödsboet.

Efter 1956 blev hon sjukhusbiträde på Vasa Centralsjukhus fram till pensioneringen 1973. Hennes mor Hilda, som dog 1970, bodde sina sista år hos henne i Vasa. Senny levde ogift och  dog  25.04.1997. Under denna tidsperiod bodde Ole och Dagmar Norrgård i huset som hyresgäster, innan de byggde sitt hus på Mietala.

Sanfrid och Svea Broo bodde här från 1968 innan de köpte huset 41 på Böös.


\jhoccupant{Nylund}{\jhname[Jakob]{Nylund, Jakob} \& \jhname[Hilda]{Nylund, Hilda}}{1922--\allowbreak 1954}
Jakob Johansson Nylund, \textborn 26.07.1884 på Finskas, gifte sig 1907 med Hilda Katarina Johannesdr., \textborn 18.05.1888 på Jungar. Sin gamla stuga köpte de från Tollikko och uppförde den på ett litet soldattorp som hörde till Hildas hemgårds hemman. Ansökan inlämnades 22 okt 1920 och formell legogivare var Johan Jungar. År 1922, i samband med torparlagen, inlöste de torpet och bodde nu på egen mark om 27,8 ar. Lagfart beviljades 20.12.1929 (ett sådant soldattorp finns emellertid inte upptaget i den förteckning som uppgjorts av Ulf Smedberg. Jungar hemman nr 6 hade sitt soldattorp tillsammans med hemman nr 21 Gunnar, nr 31 Krögarn, nr 5 Ekolla (Ekola) och nr 13 Kojonen, och soldattorpet  låg på Gunnar hemmans mark.) Det är därför något oklart med torpets status.

Efter dottern Sennys födelse 1913 reste Jakob till USA och arbetade där i 6 år. När han kom hem 1919 var hälsan förstörd av ``gruvsjukan''.

Tack vare åren i USA kunde Jakob med sina pengar köpa ett mindre åkerskifte, som senare skulle bidra till familjens försörjning. Det skomakaryrke han lärt sig i unga år kom nu väl till pass när krafterna svek, men redan efter 5 år hemmavid avled han.

Hilda stod nu ensam med 4 döttrar. Hon var i unga år en aktiv medlem i den år 1890 grundade nykterhetsföreningen.
\begin{jhchildren}
  \item \jhperson{\jhname[Dagny Adele]{Nylund, Dagny Adele}}{09.04.1908}{}
  \item \jhperson{\jhname[Jenny Linnea]{Nylund, Jenny Linnea}}{24.04.1909}{}
  \item \jhperson{\jhname[Fanny Hilda Alice]{Nylund, Fanny Hilda Alice}}{08.10.1910}{}
  \item \jhperson{\jhbold{\jhname[Senny]{Nylund, Senny}} Hjördis}{12.02.1913}{}
  \item \jhperson{\jhname[Bruno Jakob Johannes]{Nylund, Bruno Jakob Johannes}}{06.08.1923}{}
\end{jhchildren}

Jakob \textdied 02.03.1924  ---  Hilda \textdied 22.04.1970



\jhhouse{Alåkern}{7:57}{Jungar}{13}{23}


\jhhousepic{209-05786.jpg}{Vivan och Johan Back}

\jhoccupant{Back}{\jhname[Vivan]{Back, Vivan} \& \jhname[Johan]{Back, Johan}}{1986--}
Vivan Elisabet Jungar, \textborn 15.12.1955 på Jungar, gifte sig 11.06.1977 med Johan Bror Fredrik Back, \textborn 11.07.1951 på Back. Vivan är född och uppvuxen på lägenheten. Hon är utbildad barnmorska och kosmetolog och har arbetat som barnmorska på Malmska Sjukhuset fram till sin sjukpensionering. Johan har arbetat på Keppo herrgårds jordbruk, vaktmästare på Jeppo skola och brandstation och Haldins bilmåleri fram till 1986 då de övertog Vivans hemgård.

År 1990 byggde han ut den tidigare ladugården och ekonomiebyggnaden med en avdelning för svinproduktion. På gården har också potatisodling varit en bärande gren. Hälsoproblem har gjort att produktionen nu är nedlagd och jorden har sålts.
\begin{jhchildren}
  \item \jhperson{\jhname[David Johan]{Back, David Johan}}{06.04.1981}{}, tradenom, Böös nr 40
  \item \jhperson{\jhname[Jonna Elisabeth]{Back, Jonna Elisabeth}}{07.02.1985}{}, kosmetolog, Böös nr 43
  \item \jhperson{\jhname[Elin]{Back, Elin}}{30.06.1989}{}, kock, närvårdare
  \item \jhperson{\jhname[Daniel]{Back, Daniel}}{28.07.1991}{}, fordonsmekaniker
\end{jhchildren}


\jhoccupant{Jungar}{\jhname[Lars]{Jungar, Lars} \& \jhname[Doris]{Jungar, Doris}}{1965--\allowbreak 1986}
Lars Johan Jungar, \textborn 31.12.1928 på Jungar, gifte sig 06.06.1952 med Doris Elisabet Lybäck, \textborn 20.09.1932 i Kronoby. De träffades i Evangeliska folkhögskolan läsåret 1949/50 och år 1965 övertog familjen Lars' hemgård på Jungar. Den utgjordes då av 19 ha odlad jord och 25 ha skog. På gården utvecklades efter en tid fårskötsel i något större skala, vilket var nytt och ovanligt.

Redan 1957 hade Lars börjat sköta Jeppo församlings kantorstjänst, en tjänst som han innehade i 34 år till 1991. Han fungerade också som församlingens kyrkvärd och ekonom. Han var ledamot i kommunens nykterhetsnämnd, en tid ordf. för Evangeliska Ungas avdelning i Jeppo och ledamot i Ev.Ungas norra krets.

Doris var i början av 1970-talet ledamot i kommunala skolnämnden och 1987--\allowbreak 1993 församlingens dagklubbsledare. Doris \textdied 03.12.2000.
\begin{jhchildren}
  \item \jhperson{\jhname[Lars Peter]{Jungar, Lars Peter}}{04.04.1954}{}
  \item \jhperson{\jhbold{\jhname[Vivan]{Jungar, Vivan}} Elisabet}{15.12.1955}{}
  \item \jhperson{\jhname[Maria Susanne]{Jungar, Maria Susanne}}{27.01.1962}{}
\end{jhchildren}


\jhhousepic{210-05787.jpg}{``Sytningsstugan''}

\jhoccupant{Jungar}{\jhname[Artur]{Jungar, Artur} \& \jhname[Fanny]{Jungar, Fanny}}{1921--\allowbreak 1965}
Johan Artur Jungar, \textborn 22.10.1898 på Jungar, gifte sig 1922  med Fanny Maria Forss, \textborn 19.12.1899 på Fors. År 1917/18 hade Artur gått i Korsholms Lantmannaskola och läsåret 1921/22 hade både han och Fanny gått i den vandrande folkhögskolan Breidablick.

1921 övertog de Arturs föräldrars ½  hemman som delats mellan Artur och brodern Lennart. Arealen utgjordes av 19 ha odlad jord och 25 ha skog. Nu byggdes år 1923 ny bostad och driftcentrum en bit söderut och mjölkproduktionen var den huvudsakliga inriktningen. Artur var verksam inom en rad organisationer; kyrkofullmäktige och kommunalfullmäktige, lantmannagillet, lantbruksklubben, styrelsen för Jeppo-Oravais handelslag, ÖSP-lokalavd., direktionen för Gunnar folkskola.

Fanny var aktiv inom martharörelsen på orten. 1953 byggdes en ny ``sytningsbostad'', nr \jhbold{24}, på tomten för Artur och Fanny. Sonen Lars med familj flyttade då in i huvudbyggnaden byggd 1923, som efteråt renoverades. När ett nytt generationsskifte blev aktuellt flyttade Lars och Doris in i sytningsbostaden, där Lars som änkling tidvis bor kvar. Huset ägs av Johan och Vivan Back.

Gården med sitt hemman är en av de delade enheter som finns kvar i samma släktled sedan 1593 (se kedjan, bl.a. under gård nr 9, Stig-Johan Jungar) och hemmanet finns noterat så långt tillbaka som 1382. Vivian och Johan är nu de 19:e ägarna i denna kedja. Som ovan nämnts är den odlade jorden såld, men skogen kvarstår i släktens ägo.


\jhhouse{Ådahl}{7:30}{Ruotsala}{13}{25}


\jhhousepic{211-05788.jpg}{Vesa och Mia Kalliosaari}

\jhoccupant{Kalliosaari}{\jhname[Vesa]{Kalliosaari, Vesa} \& \jhname[Mia]{Kalliosaari, Mia}}{2007--}
Vesa Kalliosaari, \textborn 18.09.1985 i Jeppo, gifte 2013 sig med Mia, \textborn 15.09.1988. Vesa köpte fastigheten 2007 av Lea Ströms arvingar efter att hon avlidit. Han har renoverat och moderniserat huset. Vesa arbetar som snickare på VäriHärmä Oy medan Mia är laboratorieskötare  på sjukhuset i Jakobstad.
\begin{jhchildren}
  \item \jhperson{Jaakko A.}{27.09.2013}{}
  \item \jhperson{\jhname[Maija Matilda]{Kalliosaari, Maija Matilda}}{11.07.2016}{}
\end{jhchildren}


\jhoccupant{Ström}{\jhname[Lea]{Ström, Lea} \& \jhname[Gomer]{Ström, Gomer}}{1974--\allowbreak 2007}
Lea Jungerstam, \textborn 15.12.1915 på Kojonen, änka efter Viktor Jungerstam, ingick sitt tredje äktenskap med Gomer Ström, \textborn 16.09.1906 på Stenbacka, och han flyttade in i hennes hus på Jungar. Han hade emigrerat till USA 1934 och där gift sig med Elin Sofia Björkgren från Nedervetil. Efter hennes död 01.01.1973 återvände han till Jeppo och gifte sig med Lea den 20.12.1977. Efter Gomers död 13.08.1982 kvarstod fastigheten i Leas och senare dödsboet ägo fram till 2007.

Lea \textdied 28.05.2007 på Hagalund.


\jhoccupant{Jungerstam}{\jhname[Viktor]{Jungerstam, Viktor} \& \jhname[Lea]{Jungerstam, Lea}}{1954--\allowbreak 1974}
Viktor Eriksson Jungerstam, \textborn 02.02.1909, gifte sig 19.01.1941 med Lea Johanna Andersdr. Engström, \textborn 1915 (se ovan) som 1939 erhållit skilsmässa från Axel Gunn.

Axel hade rymt med hennes ringar, smycken och örhängen och rest till Sverige. Han återkom aldrig. Lea hade gift sig med Axel, \textborn 14.01.1902, den 31.03.1935 och samma år rest till USA. Därifrån hade de efter en kort tid blivit tvungna att återvända till Finland p.g.a. oklara resedokument. Tillbaka i landet bosatte de sig till en början hos Georg Strengell, Axels halvbror, och här föddes parets enda barn, sonen Runar Wilhelm Axelsson \jhbold{Gunn}, \textborn 01.04.1936. Efter denna period bodde familjen på flera platser bl.a. i Helsingfors, innan Axel försvann ur Leas liv.

Lea och Viktor köpte denna fastighet av Sigurd och Heldine Jungar 02.04.1954.
\begin{jhchildren}
  \item \jhperson{\jhname[Rolf Sune]{Jungerstam, Rolf Sune}}{02.07.1941}{}
  \item \jhperson{\jhname[Stig Harry]{Jungerstam, Stig Harry}}{21.10.1942}{}
  \item \jhperson{\jhname[Eivor Gunvi Denice]{Jungerstam, Eivor Gunvi Denice}}{08.10.1946}{}
  \item \jhperson{\jhname[Christel Viktoria]{Jungerstam, Christel Viktoria}}{04.07.1955}{}
\end{jhchildren}

Efter kriget arbetade Viktor på ``vedplanen'' vid stationen. Två månader innan yngsta dotterns födelse dog Viktor plötsligt den 14.05.1955. Lea stod nu ensam med 5 barn varav endast Runar var i arbetsför ålder och med en fastighet de köpt ett drygt år tidigare. Trots detta beslutade familjen att renovera huset och en kärv period vidtog. Lea plockade mossa åt William Backlund och startade ett litet hönseri i den befintliga ladugården tills dess att också den äldsta sonen Runar flyttade till Sverige. Också han förblev borta. Lea fick tidig anställning på nystartade Mirka vid Kiitola och fick senare arbete på Prevex i Nykarleby.


\jhoccupant{Jungar}{\jhname[Sigurd]{Jungar, Sigurd} \& \jhname[Heldine]{Jungar, Heldine}}{1938--\allowbreak 1954}
Sigurd Jungar, \textborn 1909, och Heldine, \textborn 1915 (se Fors, nr 98) bytte till sig fastigheten av Valdemar och Ellen Strengell 1938 i utbyte mot bageriet i Silvast. De bedrev nu jordbruk på den ½ hemmansdel han erhållit 23.02.1939 av sin far Daniel Jungar och byggde en mindre ladugård. Våren 1954 hade de bestämt sig för att sälja fastigheten och den 2 april undertecknades köpebrevet med Viktor och Lea Jungerstam. Makarna Jungar hade redan 1950 flyttat till Krylbo i Sverige där de köpt ett jordbruk.


\jhhousepic{S-J-Kalliosaari.jpg}{Kalliosaaris--Lea Jungerstams--Jungars--Strengells gård på Svartbacken 1910. Fr.v. Severus, Maurits, Valdemar (Valde) och Sanna Strengell (Åstrand)}

\jhoccupant{Strengell}{\jhname[Valdemar]{Strengell, Valdemar} \& \jhname[Ellen]{Strengell, Ellen}}{1935--\allowbreak 1938}
Valdemar Strengell, \textborn 06.09.1906, påbörjade efter hemkomsten från Canada 1935 flyttningen av hemgården från Svartbackan till s.k. Nyåkern, där den står idag. Tomten bytte han till sig av Georg Strengell, som i utbyte fick åkern mellan Svartbackan och tvärvägen (utfallet).

I samband med flytten ändrades och tillbyggdes stugan och fick därmed ett något annorlunda utseende. En kort tid efter giftermålet vid midsommaren 1937 med Ellen Forss, \textborn 13.11.1914 på Fors (se Fors, nr 102/400), bodde paret i stugan. Lars Jungar berättar för Bruno Strengell hur han en kväll när han levererade mjölk till makarna fick se något han inte sett varken förr eller senare. Radion spelade på hög volym och Ellen och Valde dansade Jeppomenuetten, endast två deltagare (menuetten dansas normalt på ett led med många deltagare)!

Den 03.09.1938 gjorde de fastighetsbyte med Sigurd och Heldine Jungar och flyttade till Silvast och övertog där bageriet (Fors, nr 98).



\jhhouse{Gammelgård}{7:76}{Jungar}{13}{26}


\jhhousepic{214-05791.jpg}{Göran och Mildrid Björkgård; tidigare Sven och Ellen Jungerstam}

\jhoccupant{Björkgård}{\jhname[Göran]{Björkgård, Göran} \& \jhname[Mildrid]{Björkgård, Mildrid}}{2011--}
Göran, \textborn 1939 och Mildrid, \textborn 1942, Björkgård från Nykarleby köpte den 16.12.2011 tomt och bostadsbyggnad på Gammelgård lägenhet. Jord och skog som tillhört sammanslutningen av Sven Jungerstams arvingar kvarstår i arvingarnas ägo. Huset står obebott.


\jhbold{1993--\allowbreak 2011}: Sven och Ellen Jungerstams arvingar (sammanslutning)


\jhoccupant{Jungerstam}{\jhname[Sven]{Jungerstam, Sven} \& \jhname[Ellen]{Jungerstam, Ellen}}{1940--\allowbreak 1993}
Sven Olof Jungerstam, \textborn 27.10.1919 på Jungar, gifte sig 14.01.1940 med Ellen Johanna Häggström, \textborn 23.06.1920 på Holm. I samband med giftermålet flyttar de nygifta in i detta hus som Svens storebror Sigurd börjat bygga efter sin hemkomst 1929 från USA, men inte hunnit flytta in i innan han dog 10.06.1930. Sven hade innan dess bott i föräldragården på åbacken vid Jungerstam. Nu skiftades också hemgården och Sven erhöll ¾ av denna, medan ¼ tillföll systern Gerda, senare gift med Konrad Perus.

Sven och Ellen fortsatte med jordbruket på sin andel av Jungar hemman med mjölkproduktion som huvudinriktning.
\begin{jhchildren}
  \item \jhperson{\jhname[Siv Lilian Marita]{Jungerstam, Siv Lilian Marita}}{17.02.1940}{}
  \item \jhperson{\jhname[Evy Sol-Britt]{Jungerstam, Evy Sol-Britt}}{16.07.1941}{}
  \item \jhperson{\jhname[Sven Håkan]{Jungerstam, Sven Håkan}}{07.06.1950}{}
\end{jhchildren}

Sven \textdied 22.06.1993  ---  Ellen \textdied 03.05.2013



\jhhouse{Jungerstam}{7:61}{Jungar}{13}{27}

\jhoccupant{Jungerstam}{\jhname[Christer]{Jungerstam, Christer} \& \jhname[Harriet]{Jungerstam, Harriet}}{1985--}
Christer Matts Valter Jungerstam, \textborn 28.02.1949, gifte sig 25.11.1978 med Gerd Harriet Sandberg, \textborn 01.10.1956 i Esse. Christer är pälsdjursfarmare och jordbrukare, hemgården övertogs år 1985. Lägenheten omfattar 13,61 ha odlad jord och 25 ha skog.

Harriet har hjälpskötarexamen, men deltar i arbetet med den stora farmningen, som är den helt dominerande driftsgrenen. Redan 1970 startades en minkfarm för att 1975 kompletteras med en rävfarm. Undan för undan har farmningen utvidgats och omfattar nu också en enhet på stadens farmområde. Detta gör den till den största farmen i denna del av kommunen.

\jhhousepic{212-05789.jpg}{Pontus Jungerstam}

1981 byggdes ny bostad för familjen (se nedan nr 28). Ladugården, som byggts 1957, hade 1973 tömts på djur och ombyggdes detta år till pälsningshus, men redan 1979 ansågs tiden vara inne för ett nytt hus för pälsningsarbetet på andra sidan landsvägen. Sonen Pontus arbetar också som utbildad fordonsmekaniker på farmen (och bor i det äldre huset).
\begin{jhchildren}
  \item \jhperson{\jhname[Jenny Ulrika]{Jungerstam, Jenny Ulrika}}{12.03.1979}{}, Pol.mag. Bor i H:fors
  \item \jhperson{\jhname[Thomas Jan Christer]{Jungerstam, Thomas Jan Christer}}{28:04.1982}{}, metallarbetare
  \item \jhperson{\jhname[Pontus Christoffer]{Jungerstam, Pontus Christoffer}}{13.08.1988}{}, fordonsmekaniker
\end{jhchildren}


\jhoccupant{Jungerstam}{\jhname[Walter]{Jungerstam, Walter} \& \jhname[Rakel]{Jungerstam, Rakel}}{1956–1985}
Walter Edvin Jungerstam, \textborn 21.10.1916, gifte sig 18.06.1947 med Hjördis Rakel Helena Häggström, \textborn 31.05.1923 i Nedervetil. År 1949 byggde makarna bostad och ekonomiebyggnad på Svartbackan, men 1956 delades hemgården med brodern Åke och byggnationen på Svartbackan såldes till Olof Laxéns familj 1958 och dessa bodde där fram till 1966 varefter de hyrde rum hos Astrid Bäckstrand vid Furubacken 1966-68.

Orsaken till försäljningen av fastigheten på Svartbackan var till stor del praktisk. Vägen, som korsade järnvägen, var problematisk för en jordbruksfastighet, speciellt vintertid. Mjölkkusken kom inte till gården för att hämta mjölken, utan gårdens folk var tvungna att själva forsla den ut till landsvägen eller direkt till Jungar Andelsmejeri. Vid rikliga snöfall och drivande snö blev vägen svårframkomlig till och med för hästar, för att inte tala om för de små traktorerna vid den tiden. En lösning måste till.

En ny bostad uppfördes nu mellan landsvägen och ån och året därpå 1957 stod en ny ladugård med tillhörande ekonomiedel under tak. Mjölkproduktion, spannmål och potatis var den traditionella driftsinriktningen på den 12 ha stora odlade arealen. Höns och grisar fick också samsas om utrymmet.

Rakel var utbildad mejerska och hade under utbildningen praktiserat både i Nedervetil och Terjärv innan hon fick anställning på Jungar Andelsmejeri. I och med generationsväxlingen 1985 började en ny era på lägenheten.
\begin{jhchildren}
  \item \jhperson{\jhbold{\jhname[Christer]{Jungerstam, Christer}} Matts Valter}{28.02.1949}{}
  \item \jhperson{\jhname[Ingmaj Carola Helena]{Jungerstam, Ingmaj Carola Helena}}{02.09.1954}{}
\end{jhchildren}

Walter \textdied 10.12.1996  ---  Rakel \textdied 01.11.2008



\jhhouse{Jungerstam}{7:61}{Jungar}{13}{28}


\jhhousepic{213-05790.jpg}{Christer och Harriet Jungerstam; huset byggt 1981}

\jhoccupant{Jungerstam}{\jhname[Christer]{Jungerstam, Christer} \& \jhname[Harriet]{Jungerstam, Harriet}}{1981--}
Christer Matts Valter Jungerstam, \textborn 28.02.1949, gifte sig 25.11.1978 med Gerd Harriet Sandberg, \textborn 01.10.1956 i Esse (se ovan). Makarna byggde 1981 ny bostad åt sig på samma tomt som Christers föräldrar år 1956/57 byggt såväl bostad som ladugård och ekonomiebyggnader. Den gamla bostaden fortsatte som bostad för föräldrarna.



\jhhouse{Nyåker}{8:97}{Ruotsala}{13}{29}


\jhhousepic{217-05900.jpg}{Kjell Liljeqvist och Annika Sundström}

\jhoccupant{Liljeqvist/Sundström}{\jhname[Kjell]{Liljeqvist/Sundström, Kjell} \& \jhname[Annika]{Liljeqvist/Sundström, Annika}}{1995--}
Fredrik Kjell Liljeqvist, \textborn 17.04.1969, lever sambo med Annika Sundström, \textborn 28.04.1971 från Jakobstad.  Kjell odlar hemmanet med potatis som specialitet och endast till en mindre del med vall och spannmål.

Annika har arbete på teknologiföretaget Beamex i Jakobstad, en arbetsplats hon varit trogen i över 20 år.

Barn: Ella Liljeqvist, \textborn 28.04.2005.


\jhoccupant{Liljeqvist}{\jhname[Göran]{Liljeqvist, Göran} \& \jhname[Martha]{Liljeqvist, Martha}}{1975--\allowbreak 1995}
Göran Martin Liljeqvist, 17.09.1934 på Ruotsala, gifte sig 1954 med Martha Stina Linnea Renvall, \textborn 23.10.1934 i Munsala. De har varit jordbrukare på Ruotsala och flyttade till denna plats och byggde nytt hus år 1975. Potatis och spannmål har varit driftsinriktningen. År 2002 överlät de huset till sonen Kjell och flyttade själva till en radhuslägenhet i Silvast.
\begin{jhchildren}
  \item \jhperson{\jhname[Marita Barbro Kristina]{Liljeqvist, Marita Barbro Kristina}}{02.07.1955}{}
  \item \jhperson{\jhname[Karin Gerda Linnea]{Liljeqvist, Karin Gerda Linnea}}{23.02.1958}{}
  \item \jhperson{\jhname[Gun-Britt Viola]{Liljeqvist, Gun-Britt Viola}}{12.03.1965}{}
  \item \jhperson{Fredrik \jhbold{Kjell} Göran}{17.04.1969}{}
\end{jhchildren}



\jhhouse{``Forsman''}{8:97}{Ruotsala}{13}{118}


\jhoccupant{Liljeqvist}{\jhname[Göran]{Liljeqvist, Göran} \& \jhname[Martha]{Liljeqvist, Martha}}{1954--\allowbreak 1975}
Huset byggt på 1880-talet, rivet 1977. Ända sedan litet barn bodde Göran Liljeqvist tillsammans med sin morfar och mormor Johan och Vilhelmina Forsman i detta hus. De var föräldrar till Görans mor Gerda, som var deras 10:e barn. Under Görans uppväxt bodde också hans moster Irene och morbror Arvid i samma hus. På andra sidan vägen bodde Görans pappa och mamma, Gerda och Holger Liljeqvist tillsammans med Görans yngre syskon. Gerda och Holger hade erhållit 1/3 av hemmanet om 14 ha skog och 9 ha jord. Vilhelmina dog 15.12.1944 och 25.11.1945 dog Gerda, hennes dotter.

Före sin död 27.10.1950 överlät morfar Johan 1/3 vardera till Irene och Arvid. Göran besökte 1952/53 Ev. Folkhögskolan i Vasa och där träffade han sin blivande hustru Martha (se ovan). Efter giftermålet 1954 köpte han sin moster Irenes 1/3 och före sin död 1960 överlät morbror Arvid sin andel.

Göran och Martha med sin familj hade nu ett hemman på 45 ha skog och 24 ha jord. Mjölkproduktion var huvudinriktning, men då både huset och ladugården började bli slitna, beslöt de att bygga en ny bostad ca ½ km söderut. 1975 var huset klart och de flyttade in. Efter en tid avstod de från mjölkkorna.


\jhoccupant{Forsman}{\jhname[Arvid]{Forsman, Arvid}}{1950--\allowbreak 1960}
Efter fadern Johans död var Arvid formell ägare till huset fram till 1960, då det överläts till Göran Liljeqvist.\jhvspace{}


\jhhousepic{Janne Forsman Jungarv.jpg}{Janne Forsmans}

\jhoccupant{Forsman}{\jhname[Johan]{Forsman, Johan} \& \jhname[Vilhelmina]{Forsman, Vilhelmina}}{1907--\allowbreak 1944 (1950)}
Johan Jakob Eriksson Forsman, \textborn 09.07.1880 på Ruotsala hemman, gifte sig 1899 med Maria Vilhelmina (Mina) Sandell, \textborn 07.03.1880 på Måtar hemman. Makarna övertog medelst gåvobrev av den 25.01.1907  0,0760 mtl av Ruotsala hemman eller ½ av föräldrarnas hemman. Den andra delen tillföll år 1913 brodern Anders, som sedermera emigrerade till Amerika, men vid ett hembesök sålde sin andel till brodern Johan.

Vid sidan av jordbruket innehade Johan ett antal förtroendeposter i kommunen och församlingen. Han var ledamot i kyrkofullmäktige, församlingens boställsnämnd, kvarnbolagsstyrelsen och i såväl Jungar som Jeppo Andelsmejeris styrelse.
\begin{jhchildren}
  \item \jhperson{\jhname[Erik Johannes]{Forsman, Erik Johannes}}{08.03.1900}{}
  \item \jhperson{\jhname[Signe Katarina]{Forsman, Signe Katarina}}{21.01.1902}{}
  \item \jhperson{\jhbold{\jhname[Irene]{Forsman, Irene}} Maria}{14.05.1906}{}
  \item \jhperson{\jhname[Helga Alice]{Forsman, Helga Alice}}{23.01.1908}{}
  \item \jhperson{\jhname[Lennart Vilhelm]{Forsman, Lennart Vilhelm}}{22.10.1909}{}
  \item \jhperson{Torsten \jhbold{Arvid}}{25.04.1912}{}
  \item \jhperson{\jhname[Johannes Alfred]{Forsman, Johannes Alfred}}{13.08.1913}{}
  \item \jhperson{\jhname[Hedvig Sofia]{Forsman, Hedvig Sofia}}{24.05.1915}{}
  \item \jhperson{\jhbold{\jhname[Gerda]{Forsman, Gerda}} Emilia}{18.06.1916}{}
  \item \jhperson{\jhname[Gunhild Linnea]{Forsman, Gunhild Linnea}}{01.02.1918}{}
  \item \jhperson{\jhname[Dagny Linnea]{Forsman, Dagny Linnea}}{11.02.1919}{}
  \item \jhperson{\jhname[Anna Angeli]{Forsman, Anna Angeli}}{13.12.1920}{}
\end{jhchildren}

Vilhelmina \textdied 15.12.1944  ---  Johan \textdied 27.10.1950


\jhoccupant{Forsman}{\jhname[Erik]{Forsman, Erik} \& \jhname[Sanna]{Forsman, Sanna}}{1896--\allowbreak 1907}
Den 15.04.1896 köpte Erik Jakobsson Forsman, \textborn 01.02.1855, gift med Sanna Samuelsdr. Roos, \textborn 18.12.1863 på Gunnar, hennes föräldrars lägenhet på Ruotsala. Hennes föräldrar, nämndeman  Samuel Jöransson Roos och hustrun Maria Henriksdr., hade nämligen år 1868 köpt lägenheten. Efter att Maria avlidit 1876, gifte Samuel om sig med Anna Gustafsdr. Åvist, \textborn 10.07.1841 i Purmo och syster till ``Ådala-Jepp''. De sålde lägenheten 1896 på auktion varvid  dottern Sanna med sin man Erik köpte den.
\begin{jhchildren}
  \item \jhperson{\jhbold{\jhname[Johan]{Forsman, Johan}} Jakob}{09.07.1880}{}
  \item \jhperson{\jhname[Hilda Maria]{Forsman, Hilda Maria}}{30.07.1885}{}
  \item \jhperson{\jhname[Anna Sofia]{Forsman, Anna Sofia}}{02.09.1887}{}
  \item \jhperson{\jhname[Erik Vilhelm]{Forsman, Erik Vilhelm}}{04.02.1890}{}
  \item \jhperson{\jhname[Anders Gustav]{Forsman, Anders Gustav}}{18.07.1893}{}
  \item \jhperson{\jhname[Anna Lovisa]{Forsman, Anna Lovisa}}{10.01.1896}{}
\end{jhchildren}

Samuel Jöransson Roos och hustrun Anna kvarlevde på lägenheten som sytningstagare. Det hus som revs 1977 är sannolikt byggt under Samuels tid. Anna \textdied 21.05.1907  --  Samuel \textdied 08.08.1913.

Erik \textdied 21.04.1912  --  Sanna \textdied 11.09.1913, bägge i lungsot.


\jhoccupant{Forsman}{\jhname[Jakob]{Forsman, Jakob} \& \jhname[Lisa]{Forsman, Lisa}}{1876--\allowbreak 1890}
Fadern till Erik Forsman, Jakob Johansson Forsman, \textborn 1818, med hustrun Lisa, \textborn 1815 köpte 11 jan. 1876  5/48 mtl av Ruotsala hemman av bönderna Simon Jakobsson och Karl Johansson, en hemmansdel som 1890 övertagits av sonen Erik.\jhvspace{}



\jhhouse{Banåker}{7:66}{Jungar}{13}{30}


\jhhousepic{215-05793.jpg}{Dan och Tina Back}

\jhoccupant{Back}{\jhname[Dan]{Back, Dan} \& \jhname[Tina]{Back, Tina}}{2002--}
Dan och Ann-Kristin Back köpte gården år 2002 av Manne Strands arvingar. Familjen bedriver jordbruk på Mellangård (karta 13, nr 18).\jhvspace{}


\jhoccupant{Strand}{\jhname[Manne]{Strand, Manne}}{1970--\allowbreak 2000}
Manne Strand, \textborn 18.07.1927, är född och uppvuxen i den gård (nr 21) som numera äges av Benny Gunell. Manne bodde där och bedrev jordbruk på sin hemmansdel, men 1970 byggde han ny bostad och flyttade ett  par hundra meter söderut. En ny ekonomiebyggnad uppfördes också. Hans mor Anna bodde tillsammans med honom fram till sin död 22.11.1991. Manne levde ogift och hade också förvärvsarbete utom gården. Han var bl.a snickare och timmerman på olika arbetsplatser. Länge var han också anställd på Mirka slippappersfabrik.

Manne \textdied 11.07.2000.
