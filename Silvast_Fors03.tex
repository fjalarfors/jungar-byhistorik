
<--- se KARTA nr 6 --->


%%%
% [house] Sparbanken 1
%
\jhhouse{Sparbanken 1}{4:194}{Silvast}{6}{81}

Styckad av stomlägenhet Nygård 4:9

%%%
% [occupant] Fastighets Ab
%
\jhoccupant{Fastighets Ab}{\jhname[Sparbanken Jeppo]{Fastighets Ab, Sparbanken Jeppo}}{2001--}
Sommaren 2001 köpte trädgårdsmästare Greger Anders Ruben	Nygård, \textborn 16.05.1962 på Silvast, aktierna i Fab Sparbanken. Vasa 	Andelsbank fortsatte som hyresgäst till juni 2014, då kontoret i Jeppo stängdes och kunderna fick vända sig till bankkontoret i Nykarleby. Efter att klubblokalen stängdes 2015 har bankrummet använts som samlings- och skolningslokal och mot ersättning av en del föreningar och grupper.

Historien som har lett oss hit presenteras nedan.


%%%
% [subsection] Från by med bank till banklös by
%
\jhsubsection{Från by med bank till banklös by}

\jhbold{Jeppo Sparbank	---   Vasa Andelsbank   ---	Fastighets Ab Sparbanken Jeppo}

Redan före kriget diskuterade de ekonomiskt tänkande jeppoborna behovet av en egen bank, en sparbank eller en andelskassa, utöver det dåvarande och redan mångåriga filialkontoret Helsingfors Aktiebank. Ofärdsåren gjorde emellertid att planerna tog form först år 1946, då man vid ett talrikt besökt ÖSP-möte den 9 maj tog frågan till diskussion. Beslutet blev att tillsätta en kommitté för att förbereda ärendet och föra det vidare.

En månad senare sammankallades till allmänt möte på Uf-lokalen. Solidariteten till Helsingfors Aktiebank var påtaglig, men en majoritet fick mötet att efter noggrant övervägande besluta anhålla hos kommunalnämnden att vidta åtgärder för bildande av en sparbank i Jeppo. Anhållan är daterad 27.6.1946, signerad av Edvin Bergman, Joel Sandberg, Ernst Westerlund, Haniel Sandberg och Lennart Jungar, därtill stödd av Jeppo Andelsmejeri och Jeppo Skogsandelslag. Den fick grönt ljus i såväl kommunalnämnd som i ett enhälligt -fullmäktige, som i sitt protokoll den 11.9.1946 skrev: ``Beslöts att stifta en sparbank i kommunen och anslogs för densamma såsom grundfond 50.000 mk.''

Kvarnarna malde snabbt. Den 29.10.1946 godkände kommunalfullmäktige förslaget till stadgar och den 20.12.1946 konstaterade samma församling att dessa godkänts av Finansministeriet. Vid det nämnda decembermötet utsågs de första principalerna.

Bankens första lokalitet blev från 10.2.1947, för en månadshyra på 600 mk, Johannes Lindéns gård i centrum (senare Jeppo Sko- och Läderaffär/Runar Källström, Kirsiläs skomakeriverkstad fram till 1.8.1983, etc.) på nuvarande Östra Jeppovägen 19 (se nr 78). Storstugan i söder inreddes till banklokal med behövlig lång disk, hyllor, stolar och kassaskåp. Lokalen var otrevligt kall. Hösten 1948 flyttade verksamheten in i ett varmare kaférum i Jeppo-Oravais Handelslag. Handelslaget ansåg sig redan i början av 1950 behöva utrymmet för egen verksamhet, varför banken blev tvungen att hastigt hitta en annan lokal.

På den tomt där banken nu finns, stod en gård (nr 381) tillhörande Helli Sarelin, som i rummet mot vägen drev en kemikaliebutik. Med henne kom styrelsen överens om att hyra de övriga rummen mot gårdssidan, på villkor att fru Sarelin fick fortsätta med sin rörelse. Senare samma vår beslöt vårmötet att köpa hela fastigheten om 0,0007 mtl, benämnd Nygård lägenhet RN:o 4:9 av Silvast hemman, för löseskillingen 850.000 mk. I den vevan uppmanades Sarelins handel med kemikalier att flytta ut med 2 veckors varsel.

\jhhousepic{Sparbanksvy 50-talet.jpg}{Sparbanken, nr 381, innan nybygget tog dess plats. Gaveln på ``Jeppoboden'', nr 380, t.v.}

Vid höstmötet 1950 beslöts bryta ut tomten och vid tillfälle utvidga den mot norr, fram till ungdomslokalens tomt. Utvidgningen kunde verkställas i samband med höstmötet 1953. Då godkändes inköpet av en 800 m$^2$ tilläggstomt till priset 170.000 mk. Därmed drogs rån i norr till den plats där den nu finns.

Under detta årtionde kom bankkunderna ofta med häst och behövde möjligheter att binda fast sitt ekipage medan de utförde sitt ärende. Därför anskaffades skyndsamt en stadig bom för ändamålet. I dagens värld har denna bytts mot asfalterade och uppskottade parkeringsfickor för bilar.

Med tiden utökade banken sin verksamhet med nya serviceformer och mera personal. Trängseln blev olidlig och styrelsen började sondera möjligheterna att bygga nytt på den befintliga tomten. År 1958 företogs en studieresa till Sydösterbotten. Den gjorde styrelsen klar över att ett nybygge låg väl inom möjligheternas gräns. Då även kommunen visade sig intresserad av hyreslokaler i ett planerat bankbygge, var det rätt tryggt att gå från tanke till handling. Styrelsemötet den 26.2.1959 och en senare besiktning under vårvintern kom fram till att markgrunden var säker. Man kunde därmed släppa planen på ett enplanshus i trä och gå in för en byggnad av sten i två våningar.

I april 1959 antog styrelsen bland tre inlämnade förslag den ritning som K-E Nyman uppgjort. Efter smärre ändringar godkändes dessa tillsammans med arbetsbeskrivningen den 11.5.1959. Nu kunde man utlysa byggprojektet på entreprenad. Flera anbud inkom. Den 13 juni avgjorde styrelsen dessa på följande sätt:

\begin{center}
  \begin{tabular}{l l r}
    \hline
    Uppdrag & Ansvarig & Kostnad \\ \hline
    Byggande & Georg Romar, byggm. & 11.443.000 mk \\
    Rör- och vattenanläggning & Johannes Fors & 1.147.000 mk \\
    Elinstallationer & Jeppo Kraft Alg & 448.000 mk \\
    Jordkabel & Jeppo Kraft Alg & 55.000 mk \\
    Ritningar & K-E Nyman & 200.000 mk \\
    Arbetsövervakning & o & 50.000 mk \\
    \hline
  \end{tabular}
\end{center}
Målaren Olov Laxén vidtalades att göra förslag till färgsättning.

För att säkerställa finansieringen upptogs ett 7 milj.mk stort lån hos SCAB. Efter ett ovanligt raskt byggande var huset inflyttningsklart redan den 11.2.1960. För 15 milj.mk, vilket var satt som kostnadstak, hade man fått ett modernt hus med adekvata bankutrymmen och sidorum, mottagningsrum för hälsovården, tre boningslokaler i övre våningen avsedda för bankdirektör, hälsosyster och barnmorska samt i källaren klubblokal, arkivrum och kök. Invigningsfest hölls den 15.10.1960.


\jhhousepic{037-05573.jpg}{``Banken'' är år 2016 fortfarande mitt i byn}

TILLBYGGNAD

Erfarenheten efter sex års användning visade på vissa brister. Man klagade på att klubbrummet i källaren var för litet, mörkt och dåligt ventilerat. Styrelsen väckte därför hösten 1966 förslag om tillbyggnad av ett större klubbrum i söder, med garage i bottenvåningen. Våren 1967 gav principalerna klartecken för tillbyggnad av 120 m2. Med Gunnar Kronqvists ritningar fullföljdes byggandet i egen regi med Georg Romar som ansvarig ledare. Totala kostnaden för bygget, med fullständig inredning, blev 65.000 mk. Resultatet utföll i en stor och ljus klubblokal, som fram till hösten 2014 varit mycket uppskattad och i flitig användning av föreningar o.a. Till bybornas besvikelse sattes den nu i annan funktion.


FRÅN SPARBANK TILL ANDELSBANK OCH INGEN BANK

År 1991, den 9.10, anslöt sig Jeppo Sparbank till \jhbold{Sparbanken Deposita}, som uppstått 1966 som ett resultat av en fusion av sparbankerna i Jakobstad och Nykarleby. Redan 31 januari 1992 anslöt sig Deposita av solidaritetsskäl till Sparbanken i Finland i ett försök att rädda den krisande sparbanksrörelsen. Det hjälpte inte. Kreditförluster och flykten från banken gjorde att statsrådet den 22.10.1993 beslöt sälja \jhbold{Sparbanken i Finland} i fyra lika stora delar till de övriga bankgrupperna i Finland, d.v.s. till Andelsbankerna, Föreningsbanken, Postbanken och KPO.

Redan följande morgon då bankfolket kom till jobbet hörde de telefaxerna knattra fram den av Finlands Bank uppgjorda fördelningen, som visade att kontoret i Jeppo tillförts \jhbold{Vasa Andelsbank}. Endast 1,5 timme senare stegade den nya vd:n Kaj Skåtar in i lokalen för att beskåda sitt nyförvärv och lugna personalen för vilken ``golvet plötsligt gett vika under fötterna''. Nu skulle skräplånen, som uppgick till ca 40 miljarder mark inom sparbanksgruppen, överföras till den nygrundade banken för detta ändamål, \jhbold{Arsenal}.

Jeppo-kontoret var filial under Vasa Andelsbank och hyresgäst i fastigheten tills det drygt 20 år senare stängdes i juni 2014 p.g.a. rationaliseringsåtgärder inom andelslaget. Stängningen möttes av omfattande men resultatlösa protester från byinvånarna. F.o.m. nu saknas ett fysiskt bankkontor på orten!

Sedan år 2001 äger bolaget Fastighets Ab Sparbanken Jeppo (Greger Nygård) bostads- och affärsfastigheten Sparbanken 1, R:no 4:194 i Jungar by.


DIREKTÖRER

Skötseln av bankens verksamhet anförtros verkställande direktören. Vanligen är det också den personen som är bankens ansikte utåt. Att man det stora ansvaret till trots har trivts på sin post visas av att enbart sex personer hunnit bekläda den:
\begin{enumerate}
  \item \jhname[Rakel Törnkvist (Romar)]{Törnkvist, Rakel},  1947
  \item \jhname[Lisbeth Ågren (Jungar)]{Ågren, Lisbeth},   1947-1951
  \item \jhname[Gustav Johansson]{Johansson, Gustav},   1952-1965
  \item \jhname[Johan Stenfors]{Stenfors, Johan},  1965-jan. 1993
  \item \jhname[Marit Leinonen]{Leinonen, Marit},  1993-apr. 1994
  \item \jhname[Ing-Britt Palm (Fors)]{Palm, Ing-Britt},  1994-2004
  \item \jhname[Ralf Bonde]{Bonde, Ralf}, 2005--> juni 2014, enhetschef fram till stängningen
\end{enumerate}


\jhbold{Anställda under olika konstellationer fram till 2014}
\jhname[Ulla]{Nordström, Ulla}, \jhname[Märtha]{Slangar, Märtha}, Palm Ingbritt, \jhname[Benita]{Bäckstrand, Benita}, \jhname[Margaretha]{Nybyggar, Margaretha}, Bonde Ralf, \jhname[Marita]{Holländer, Marita}, \jhname[Ingemo]{Mäntymäki, Ingemo}, \jhname[Ann-Maj]{Öst, Ann-Maj}, Stenfors Johan, \jhname[Carina]{Lassander, Carina}, \jhname[Barbro]{Enqvist, Barbro}, \jhname[Carita]{Sandström, Carita}, Johansson Gustav, \jhname[Lisbeth]{Ågren, Lisbeth}, Törnqvist Rakel, \jhname[Aidi]{Jungar, Aidi}.


\jhbold{INHYSTA VERKSAMHETER}

Många verksamheter, av vilka några nämndes ovan, har tidvis varit inhyrda i husets markplan. De har här funnit en lämplig verksamhetsadress.

1994-2014
\jhbold{Vasa Andelsbank} hyrde bankkontoret och klubblokalen den 14.03.1994 och flyttade från den egna fastigheten, Andelsbanken 4:199, nr 92, till Sparbankshuset. Föreningar och privata personer fick fortsätta använda klubblokalen gratis.

Norra lokalen och lilla rummet i söder, 1991-2016:
2013-2016   Gästarbetare från Estland och Ukraina,
2008-2013   Ann-Christin Dahlkvists Frisörsalong,
2003-2008   Lokalförsäkring,
1991-2002   Lokalförsäkring, norra lokalens västra rum (1968-)
1991-2002   Bokföringsbyrå Barbro Stenfors norra lokalens	östra rum (f.o.m. 1986, se under Sparbankens tid, nedan)
2003-2006   Bokföringsbyrå Barbro Stenfors, lilla rummet vid klubblokalen.
2006-2009   Jodilonna, Jonna Wallin, lilla lokalen invid klubblokalen
2015-2015   Hans Kronlund använde lokalen som kontor.

Hyresgäster i gamla huset och nedre våningen i det nya bankhuset under Jeppo Sparbanks tid. Från starten 1960 och till nutid, samt även delvis efter överlåtelsen.


1968--2016:

Efter att Hälso- och Rådgivningsbyrån flyttat ut 1968 flyttade \jhbold{Jeppo-Nykarleby Försäkringsförening} från ett rum inne i banklokalen till nordvästra rummet mot vägen. Från 2003 använde försäkringsbolaget även det östra rummet.	Lokalförsäkring stängde sitt kontor i Jeppo 2008.

\jhbold{ÖSP-skattetjänst/John Lassén} öppnade ett kontor i nordöstra rummet mot bäcken samtidigt som försäkr.föreningen tog plats i sitt. År 1981 gick John Lassén i pension. Se mera om John Lassén, och Gun-Lis Lassén, nr 95, på Fors hemman. Päivi Nyholm övertog skötseln av ÖSP-skattetjänst 1981-1986. Barbro Stenfors hade startat en \jhbold{Bokföringsbyrå} i ett rum därhemma och 1986 övertog hon ÖSP-skattetjänst, jordbrukarnas bokföring och flyttade till bankens nordöstra rum. Den 06.12.2002 flyttade Bokföringsbyrån till det lilla rummet vid klubblokalen. År 2006 gick Barbro Stenfors i pension och sålde verksamheten till Nykarleby Bokföringsbyrå, numera Norlic.

I rummet mot banklokalen hade Linnea Fagerlund sin \jhbold{damfrisering}, som 1973-1975 övertogs av Siv Strengell och därefter av Gun-Lis Lassén 1976. Därefter gjordes lokalen om till bankdirektörsrum.

I april 2008 hyrde Ann-Christine Dahlkvists \jhbold{frisörsalong} den norra lokalens båda rum. Den 28.02.2013 avslutade hon sin frisörsalong, utan att någon ny frisör övertog verksamheten. Ann-Christine hade i januari börjat sina studier till närvårdare. Se mera Höglund 4:106, nr 77. Norra lokalen har därefter varit ledig, förutom vissa perioder, då utländska gästarbetare använt lokalen för övernattning.

1968--1991:

\jhbold{Jeppo Skogandelslag och Skogsvårdsförening}. Det lilla rummet i klubblokalhuset hyrdes 1968, då tillbyggnaden blev inflyttningsklar, av skogsförman Bernhard Fogström för Jeppo Skogsandelslag. Fogström hade tidigare från 1960 använt, för Jeppo Skogsandelslags behov, ett rum i bankens källare. År 1991 gick Fogström i pension och Skogsandelslaget och även Skogsvårdsföreningen flyttade till Andelsringen i f.d. Postens lokaler. Postens verksamhet hade flyttat till Nykarleby.

1960--1968:
När det nya bankhuset i februari 1960 blev färdigt flyttade \jhbold{Jeppo Hälso- och Rådgivningsbyrå} till nedre våningens norra lokal. Hälsosyster var då Thea Fogström och barnmorska Lisbeth Pantolin. På hösten 1968 flyttade Hälso- och Rådgivningsbyrån till det nya	kommunalhuset på Fors hemman, nr 112.


\jhbold{HYRESGÄSTER/BOENDE}

SÖDRA LOKALEN, DIREKTÖRSBOSTADEN:
\begin{enumerate}
  \item 	maj 2014--  : I maj 2014 hyrde merkonom Richard Nordlund, \textborn 28.12.1993 i Esse, den södra lokalen. Richard arbetar på Jeppo Potatis Ab. Se mera nr 51. I januari 2016 blev studerande Pernilla Marianne Lindell, \textborn 10.05.1991 i Larsmo, sambo med Richard.
  \item 2010-2014: Tom lokal?
  \item jan 2009-mar 2010: På nyårsaftonen 2008 flyttade Marika till den södra lokalen, där hon bodde januari 2009-mars 2010, då hon flyttade till Jakobstad. Marika studerar till närvårdare, får sin examen i december 2016. Se mera nr 51.
  \item 1993-1999: Målare Maunu Tapani Hahto, \textborn 1961 i Jurva, och hans hustru tandskötare Inger Heldine, född Bergman, \textborn 1963 på Lassila, flyttade i maj 1993 till den södra lokalen.
  Barn;	Jenny Ida Maria,	\textborn 1991 i Tusby och	Heidi Jonna Viola,	\textborn 1993 i Vasa
  Skilsmässa 1997. Inger och barnen flyttade i aug. 1999 till en lärarbostad på Jeppo skola. Jenny är barnträdgårdslärare, gift Sorjonen på Kaup, och har två barn. Heidi är merkonom, socionom och arbetar som handledare för flyktingbarn på Evangeliska Folkhögskolan i Vasa. Hon är sambo med Janne Nylund. Inger är gift med Pentti Rantala \textborn 1959 i Kittilä. Han arbetar på Mirka i Jeppo och de bor på JS Fastighets Ab på Silvast, nr 57.
  \item jan 1982-apr 1993: Efter 1981 användes en del av södra lokalen som kaffe- och sammanträdesrum för personalen.
  \item dec 1977-dec 1981: Sven och Christina Simons (se nr 90). 1978 flyttade familjen Simons till den södra lokalen och 1981 till sitt nybyggda hus i Nykarleby.
  \item nov 1965-dec 1977: 	Den 20.11.1965 blev merkant Johan Uno Stenfors, \textborn 1937 i Övermark, bankdirektör för Jeppo Sparbank. Han flyttade med sin hustru merkonom Barbro, född Karls, \textborn 1940 i Oravais, till lokalen. Barbro fick tjänst 1969 som bokförare på Oy Mirka Ab. 1973 fusionerades Mirka med moderbolaget Oy Keppo Ab och Barbro flyttade till kontoret på Keppo. 1978 flyttade huvudkontoret för Keppo-bolagen till Oravaisfabrik, som då blev Barbros arbetsplats. 1980 startade Barbro en egen bokföringsbyrå och 1981 hyrde hon nordöstra rummet av banken. 1978 kunde familjen Stenfors flytta in i sitt nybyggda hus Älvbranten Rnr 2:53. Se mera, karta 2, nr 16.
  \item feb 1960-nov 1965: Gustav och Gun-Britt Johansson samt barnen Bo, Ulf och Susann kunde i februari 1960 flytta in på övre våningen i det nya husets södra bostadslokal. I november 1965 blev Gustav bankdirektör på Oravais Sparbank och familjen flyttade till Oravais.
\end{enumerate}

NORRA LOKALEN:
\begin{enumerate}
  \item 2009-2017: Under de senaste åren har flera ukrainska familjer bott kortare eller längre tider i den norra lokalen; 2016-	Panchenko Yeven <-- 2014-2015	Dudenko Mykola <-- 2009-2014	Familjen Gudenko Oleksndr.
  \item 1993-2008: Markku Tapio Kalliosaari, \textborn 20.05.1969 på Romar hemman, nr 28, bodde 1993-2008 i lokalen. Markku arbetar på Oy KWH-Mirka Ab. Han köpte år 2008 Eva och Paul Grahns hus på Purmovägen, se nr 32 på Romar hemman.
  \item 1988-1992: Sommaren 1988, genast efter Asplund, flyttade Stig och Ann-Christin Julin in i norra lokalen. Stig Julin \textborn 15.02.1963 på Mjölnars hemman och Ann-Christin \textborn 30.01.1966 i Vörå. Stig är lastbilschaufför och är anställd hos trafikant J. Sandström och Ann-Christin på Oy KWH-Mirka Ab:s försäljningskontor. Deras son Tobias föddes 23.05.1991, då de bodde på banken. I juli 1992 flyttade de till sitt nya hus på Nylandsvägen, nr 39 på Romar hemman.
  Under en kortare tid bodde Joakim Bärs och hans sambo från Australien i norra lokalen.
  \item 1982-1988: Boris Asplund, \textborn 1959 på Fors hemman på Holmen och Ann Slangar, \textborn 1960 på Slangar, hyrde den norra lokalen 1982-1988. Boris arbetade på Mirka och Ann studerade till merkonom i Jakobstad och senare till ekonom i Vasa och de gick skilda vägar. Boris emigrerade till Sverige 1988.
  \item dec 1977-1981: 	Tage Fors, \textborn 1949 på Finskas hemman och hans hustru tandskötare Ann-Christin, född Eriksson, \textborn 1954, hyrde den norra lokalen 1978-1981. Barn;	Jonas	\textborn 1980 och	Nicklas	\textborn 1982	Tage blev företagare och övertog 1981 hemgården och bilreparationsverkstaden med bensinförsäljning på Finskas.
  \item 1968-1981: Sven Simons, \textborn 1939 i Oravais och hans hustru Christina, född Sundell, \textborn 12.01.1943 på Fors hemman, hyrde den norra lokalen när familjen Pantolin 1968 flyttade till den nya kommunalgården. Sven var anställd på Nykarleby-Jeppo Försäkringsförenings Jeppo kontor och Christina på Jeppo Telefoncentral som ansvarig telefonist. Barn;	Peter \textborn 11.09.1967	--  \textdied 2015 i Stockholm, Jan	\textdied 02.03.1975. År 1978 flyttade familjen Simons till den södra lokalen.
  \item feb 1960-nov.1968: Den 11.02.1960 flyttade barnmorska Lisbeth Pantolin och hennes 	man byggmästare Hemming Pantolin in i norra bostadslokalen. Lis-Beth är född Rönnlund \textborn 1927 i Övermark och Hemming \textborn 1922 i Korsnäs. Lisbeth fick tjänst som barnmorska i Jeppo 1958 och Hemming som byggmästare i Vörå. Då de kom till Jeppo hyrde Lisbeth och Hemming övre våningen av Jungbo 4:96, nr 63. De adopterade en dotter, Maria Elisabeth \textborn 30.04.1961. År 1968 flyttade familjen till den nya 	kommunalgården på Fors, nr 112. De byggde 1975 ett bostadshus i Oravais. Lisbeth gick i pension 1992. Hemming dog 26.01.1993 och Lisbeth dog 23.08.2008. De är begravna i Oravais.
\end{enumerate}

MELLERSTA LOKALEN:
\begin{enumerate}
  \item 2011-2017: Johan och Emma Sjölind. Företagare Johan Gustav Sjölind, \textborn 10.07.1990 på Holm hemman, är utbildad artesan/finsnickare. I dag är han företagare inom mark-och jordbyggnadsbranschen. Den 24.08.2014 gifte han sig med klasslärare Emma Maria Karolina, född Söderlund, \textborn 24.06.1991 i Jakobstad. De har	en son Helmer Gustav \textborn 05.07.2015. I december 2015 köpte de John och Lea Lasséns fastighet, nr 95, på Fors hemman. Bostadshuset grundrenoverades och rumsindelningen förändrades. Se mera under nr 95.
  \item 2009-2010: Arbetsledare David Johan Back, \textborn 06.04.1981 på Jungar hemman och hans hustru hälsovårdare Annika Maria Terese, född Eklöv, \textborn 13.06.1980 i Lillby, Purmo, hyrde lokalen 01.03.2009. De gifte sig 08.08.2009 i Purmo. Se Böös, karta 12, nr 40.
  \item 2008-2010: 	Hösten 2008 flyttade Marika Nordlund-Nybyggar, \textborn 16.01.1970 i Esse, in. Redan på nyårsaftonen 2008 flyttade Marika till den södra lokalen, där hon bodde januari 2009-mars 2010, då hon flyttade till Jakobstad. Se mera nr 51.
  \item 1996-2008: 	Helge Gustav Bergman, \textborn 28.09.1933 på Lassila hemman, flyttade som änkling i februari 1996 till lokalen. Helges dotter Carina med familj flyttade från samma lokal till hemgården på Lassila. Helge var en social och aktiv föreningsmänniska med många förtroendeuppdrag. Styrelseordf. för Jeppo Potatis Ab 1976-1990, styrelsemedlem för Nykarleby-Jeppo Försäkringsförening 1966-1997, styrelsemedlem 1965-2007 varav ordförande under många år i Jeppo Hembygdsförening. 1966 blev han grundande medlem i Lions och fullmäktigeledamot i Jeppo församling 1986-2002, samt många kommunala förtroendeuppdrag och i andra föreningar under kortare perioder. Sommaren 2008 blev trapporna till andra våningen ett problem för Helge och han flyttade till vaktmästarbostaden på brandstationen (nr 111) och därifrån efter några månader till Hagalund. Helge \textdied 08.03.2009.
  \item 1980-1996: Merkonom Carina Bergman, \textborn 15.02.1960 i Lassila, och Glenn Lassander, \textborn 11.03.1959 i Kovjoki, hyrde den mellersta lokalen i oktober 1980-1996. Carina hade tjänst på Jeppo Sparbank, och Glenn på Kovjoki Snickeri.
  Barn;	Tina \textborn 17.04.1985 och	Nina \textborn 30.04.1987
  Tina är utbildad frisör och har sin frisörsalong på hemgården i Lassila. Nina är en känd sångerska och musikpedagog. Hon har sin verksamhet i Stockholm. År 1996 flyttade familjen till Carinas hemgård. Carinas mor Rose-Maj dog 1994 och far Helge bytte bostad 1996 med familjen Lassander.
  \item 1979-1980: Björn Palm, \textborn 1945 i Helsingfors och hans hustru Ingbritt, född Fors, \textborn 1941 på Finskas, hyrde 1979-1980 den mellersta lokalen. Ingbritt var då banktjänsteman på Jeppo Sparbank och blev på 1990-talet enhetschef på Vasa Andelsbank, som då hyrde banklokalen i nedre våningen. Björn arbetade på Mirka som inköpschef, men är numera pensionär.
  Barn; Kenneth, \textborn 1980, är tradenom och arbetar på Mirka och bor med sin familj i Nykarleby. Björn och Ingbritt köpte 1983 av Oy Keppo Ab det gamla kommunalhuset på Holmen.
  \item 1973-1979: Från Tånabacken flyttade 1973 änkefru Helga Enlund, \textborn 1905 på Gunnar, till den mellersta lokalen. Helga var mor till Margit Julin, född Hägglund, \textborn 1929 på Tånabacken och till Rolf Enlund, \textborn 1941. Helga flyttade 1979 till sin son Rolf i Jakobstad. Hon dog 1980.
  \item nov 1968-1973: Caj-Erik Isidor Fagerlund, \textborn 23.08.1943 i Jakobstad och hans hustru Linnea Marianne, född Åkermark, \textborn 25.08.1943 i Purmo, hyrde 1968-1973  den mellersta lokalen. Linnea hade damfrisering i nedre våningen. Deras dotter Carita Vivan Marianne, ``Tita'' \textborn 12.03.1963 i Purmo, är utbildad sjukhuslaborant och arbetar på Malmska i Jakobstad. Familjen Fagerlund kom till Jeppo, Keppo 1965. Caj blev färdig skogstekniker i december 1964. Efter tiden på Keppo och	en kortare arbetstid för Suomen Gullvik flyttade familjen Fagerlund till Jakobstad. Caj dog 28.08.2007.
  \item 1960-nov 1968: 1960 fick hälsosyster Thea Margaretha Backa, \textborn 1935 i Oravais, tjänst som hälsosyster i Jeppo. Hon flyttade i februari in i	bostadslokalen. Thea gifte sig i Oravais den 18.09.1960 med Bernhard	Vilhelm Fogström, född 1928 i Hirvlax, Munsala. Bernhard fick 1960 	tjänst som skogsförman på Jeppo Skogsandelslag. Kontor öppnades i 	bankens källarvåning, där det fanns till 1968, då det flyttades upp till rummet vid klubblokalen. Theas och Bernhards barn;	Britt-Marie	\textborn 19.03.1961, Mårten Johan \textborn 10.04.1966, Mats Anders	\textborn 10.04.1966.	År 1968 flyttade familjen till den nya kommunalgården på Fors hemman, nr 112. De byggde 1976 ett egnahemshus vid älvstranden på Grötas hemman, nr 36.
\end{enumerate}

1950-1991:
I början av år 1950 hyrde och efter vårmötet 1950 köpte Jeppo Sparbank fastigheten med den gamla kemikalieaffärs- och 	bostadsbyggnaden. Lagfart erhölls 20.09.1950 på 0,0007 mantal	av skattelägenhet Nygård 4:9.



%%%
% [oldhouse] Tomtområde 1
%
\jholdhouse{Tomtområde 1}{4:9}{Silvast}{6}{381}

Utgör 0,0007 mtl av stomlägenhet Nygård 4:9

%%%
% [occupant] Jeppo
%
\jhoccupant{Jeppo}{\jhname[Sparbank]{Jeppo, Sparbank}}{1950-1960}
Banken kom att använda fastigheten under 10 års tid innan den förverkligade planerna på nybygge och kunde riva det gamla huset. I början av år 1952 blev Stig Gustav Valdemar Johansson, \textborn 31.01.1929 i Oravais, bankdirektör för Jeppo Sparbank. Han flyttade till övre våningen på den gamla Sparbanken. Gustav hade kommit med föräldrar och systrar till Jeppo-Oravais Handelslag 1941. År 1953 gifte sig Gustav med Evy Gun-Britt, född Sandström, \textborn 24.12.1929 i Jeppo.
\begin{jhchildren}
  \item \jhperson{\jhname[Mats Bo-Gustav]{Johansson, Mats Bo-Gustav}}{10.02.1954}{}
  \item \jhperson{\jhname[Ulf Stig Henrik]{Johansson, Ulf Stig Henrik}}{18.06.1957}{}
  \item \jhperson{\jhname[Susanna Elisabet]{Johansson, Susanna Elisabet}}{29.05.1963}{}, föddes i nya lokalen
\end{jhchildren}
Bo och Ulf föddes när de bodde i gamla bankhuset. I februari 1960 fick de flytta in på övre våningen i det nya husets södra bostadslokal.


%%%
% [occupant] Helli
%
\jhoccupant{Helli}{\jhname[Sarelin]{Helli, Sarelin}}{1938-1950}
Helli Maria Sarelin, \textborn 19.07.1898 i Åbo, köpte fastigheten 1938. Hon var gift med Edvin, lärare i Bärs folkskola. Hellis mor, \jhname[Mimmi Laquist]{Laquist, Mimmi}, \textborn 04.03.1867 i Kristinestad, flyttade till Jeppo och öppnade en kemikalieaffär i nedre våningen, där hon även hade sin bostad. Butiken fanns i ett rum ut mot vägen. Övre våningen hyrdes ut.

När Helli blev änka 07.03.1946, flyttade hon till huset i slutet av året och bodde där med dottern Solveig Helena Sarelin, \textborn 14.08.1946 på Heikfolk, Bärs folkskola. Hellis äldre döttrar, Gunvor och Svea stannade på Bärs.

\jhhousepic{Sparbanken 1950.jpg}{Sparbanken på sin tredje och slutgiltiga adress}
I början av 1950 hyrde Helli nedre våningens bostadsrum till den då husvilla sparbanken mot löfte om att kunna fortsätta sin rörelse i butiken. När hon samma vår gick med på att sälja hela fastigheten, fick hon endast två veckor på sig att flytta ut. År 1951 emigrerade Helli och Solveig till Sverige. ``Mimmi'' Maria Vilhelmina dog 30.09.1952.

Hyresgäst i övre våningen 1936 till slutet av 1946 var massör	Aina Juliana Berg, \textborn 26.08.1880 i Sysmä. Hon flyttade till Johanssons hus, nr 380a, då Hellin med nyfödda Solveig Sarelin bosatte sig i huset.


%%%
% [occupant] Jungell
%
\jhoccupant{Jungell}{\jhname[Anders]{Jungell, Anders} \& \jhname[Olivia]{Jungell, Olivia}}{1938}
Anders och Olivia Jungell köpte fastigheten 1938, men sålde den vidare efter några veckor.\jhvspace{}


%%%
% [occupant] Sundell
%
\jhoccupant{Sundell}{\jhname[Einar]{Sundell, Einar} \& \jhname[Helfrid]{Sundell, Helfrid}}{1927-1938}
Einar Johannes Sundell, \textborn 19.08.1905 på fastighet Bageri 4:23 av 	skattelägenhet Bäck 4:16 på Silvast och hans hustru Helfrid, född Jungar, \textborn 14.02.1905 på Jungar hemman, köpte 1927 0,0007 mantal av lägenhet Nygård 4:9. Säljare var Isak och Anna Stenbacka. Lagfart på mantal erhöll de 01.03.1929. De byggde ett affärs- och bostadshus i två våningar. Einar hade en cykel- och radioaffär, samt verkstad i nedre våningen.

Hyresgäst i nedre våningen 1936-1938 var barberare Enlund.
Ang. Sundell, se mera om fastigheten Bageri 4:23, figur 72.



%%%
% [house] Svinvallen I
%
\jhhouse{Svinvallen I}{20}{Fors(samfällighet)}{6}{82}


\jhhousepic{046-05579.jpg}{Paul och Carita Sandström}

%%%
% [occupant] Sandström
%
\jhoccupant{Sandström}{\jhname[John]{Sandström, John}}{2007--}
Transportföretagare John Paul Alfred Sandström, \textborn 17.02.1958, när familjen bodde på denna lägenhet av Fors (Lillsilvast) 3 skattehemman. John övertog barndomshemmet år 1994, då han blev sambo med Sisko Birgit Karin Väisänen, \textborn 16.04.1962 i Ylistaro. Birgits barn Berit Bäckman, \textborn 1980 och Tero Bäckman, \textborn 1986 på Keppo, flyttade med sin mamma till Ruckusvägen 14. John och Birgit vigdes 30.07.2010.
\begin{jhchildren}
  \item \jhperson{\jhname[Kim Krister]{Sandström, Kim Krister}}{16.11.1994}{}
  \item \jhperson{\jhname[Camilla Charlotta]{Sandström, Camilla Charlotta}}{24.05.1997}{}
\end{jhchildren}

John är utbildad maskintekniker och övertog efter sin far Paul transportföretaget P \& J Sandström, som har flera anställda. Birgit har sedan 1980-talet arbetat på KWH-Mirka Ab:s fabrikslager i Jeppo och numera på färdigvarulagret i Oravais. Kim har studerat logistik på yrkeskolan i Närpes och Camilla studerar inom sjukvården. John och Birgit byggde år 2008 ett nytt bostadshus på andra sidan av Ruckusvägen på Bäcklid 3:96 på Fors hemman, nr 89.

Bostadshuset på Svinvallen står idag tomt.


%%%
% [occupant] Sandström
%
\jhoccupant{Sandström}{\jhname[Paul]{Sandström, Paul} \& \jhname[Carita]{Sandström, Carita}}{1954-2007}
Den 24.11.1954 övertog Paul Alfred Sandström, \textborn 24.06.1926 på Fors hemmans samfällighet, lägenheten med byggnader. Säljare var Alfred Sandströms dödsbo. Han flyttade med sin mor Ida från sin bostad på fastighet Väster 4:98, som han hade sålt till sin bror Runar Sandström. Han renoverade bostadshuset och gjorde ändringar i uthuset. Den 24.07.1957 vigdes Paul med Carita Magdalena, \textborn Sandberg 23.03.1933 på Skog hemman. Lastbilschaufför Paul blev i unga år egen företagare inom transportbranschen. Firman växte och utvecklades kontinuerligt med flera grustag och anställda chaufförer på transportbilar och grävmaskiner. Carita var banktjänsteman och bokförare på Jeppo Sparbank. Paul och Carita gjorde en tillbyggnad och grundrenoverade bostadshuset på 1980-talet.
\begin{jhchildren}
  \item \jhperson{\jhbold{\jhname[John Paul Alfred]{Sandström, John Paul Alfred}}}{16.02.1958}{}
  \item \jhperson{\jhname[Lena Birgitta]{Sandström, Lena Birgitta}}{09.03.1963}{}
\end{jhchildren}
Lena vigdes i Jeppo kyrka den 07.04.1984 med Kennet Hägg, \textborn 20.03.1959 i Komossa, Oravais.

Paul och Carita flyttade 1994 som pensionärer till sin lokal nr 3 i Bostads Ab Gökbrinken på Stationsvägen 17.
Paul \textdied 19.09.2007  ---  Carita \textdied 13.01.2009.


%%%
% [occupant] Sandström
%
\jhoccupant{Sandström}{\jhname[Vilhelm]{Sandström, Vilhelm} \& \jhname[Rut]{Sandström, Rut}}{1945-1954}
År 1945 köpte krigsveteran Vilhelm Sandström, \textborn 1915 på Fors hemmans samfällighet, en del av lägenhet Nord 4:55 av stomlägenhet Norrgård 4:8 av Silvast hemman. Säljare var Vilhelms föräldrar Alfred och Ida. Vilhelm och hans hustru Rut Mari Bernice, \textborn Janfelt 26.01.1913 i Nedervetil, bosatte sig i Vilhelms barndomshem. De var jordbrukare på sin del och brödernas andel av lägenhet Nord 4:55. De adopterade en pojke, Helge, \textborn 25.03.1950 i Gamlakarleby. Vilhelm och Rut köpte 1953 en jordbrukslägenhet i Bonäs, Nykarleby och familjen flyttade dit. Rut dog 1998, sonen Helge 2007 och Vilhelm 2009 i Nykarleby.

Under en kortare tid var bostaden uthyrd till Arthur Stoor från Pensala och hans hustru Signe, född Almberg, med dotter Benita.


\jhhousepic{Svinvallen.jpeg}{Hus nr 382, vitt; Alfred och Ida Sandström}

%%%
% [occupant] Sandström
%
\jhoccupant{Sandström}{\jhname[Alfred]{Sandström, Alfred} \& \jhname[Ida]{Sandström, Ida}}{1915-1945}
Ida Johanna Isaksdotter, \textborn 24.07.1889 på Silvast hemman, gifte sig 31.01.1915 med stationskarl Alfred Sandström, \textborn 06.09.1891 i Larsmo. De byggde sitt hem (nr 382) på Fors hemmans samfällighet, nuvarande Svinvallen I R:nr 20. Den 28.02.1925 köpte de av bonden Karl-Henrik Storgård skattelägenhet Nord 4:55 om 0,0082 mantal av stomlägenhet Norrgård 4:8. De byggde fähus och stall på bostadstomten.
\begin{jhchildren}
  \item \jhperson{\jhname[Alice Linnea]{Sandström, Alice Linnea}}{08.07.1915}{14.08.1982}, Västerås
  \item \jhperson{\jhname[Elna Johanna]{Sandström, Elna Johanna}}{07.12.1916}{14.12.2006}, Jeppo
  \item \jhperson{\jhname[Johan Vilhelm]{Sandström, Johan Vilhelm}}{30.07.1918}{16.04.2009}, Nyk.by
  \item \jhperson{\jhname[Gerda Adele]{Sandström, Gerda Adele}}{17.09.1920}{03.02.2013}, Eskilstuna
  \item \jhperson{\jhname[Gertrud Lovisa]{Sandström, Gertrud Lovisa}}{03.10.1922}{07.06.1924}, Jeppo
  \item \jhperson{\jhname[Anders Gunnar]{Sandström, Anders Gunnar}}{05.10.1924}{10.08.1944}, Jeppo
  \item \jhperson{\jhbold{\jhname[Paul Alfred]{Sandström, Paul Alfred}}}{24.06.1926}{19.09.2007}, Jeppo
  \item \jhperson{\jhname[Jarl Uno]{Sandström, Jarl Uno}}{05.06.1928}{}
  \item \jhperson{\jhname[Runar Isak]{Sandström, Runar Isak}}{05.04.1932}{14.06.1972}, Bonäs
  \item \jhperson{\jhname[Brita Anita]{Sandström, Brita Anita}}{27.03.1934}{22.06.1937}, Jeppo. En ``långflakasläde'' stjälpte över Brita.
\end{jhchildren}


Alfred Sandström \textdied 15.05.1951  ---  Ida \textdied 17.02.1975



%%%
% [house] Bäckström-Bäcklid
%
\jhhouse{Bäckström-Bäcklid}{3:96-3:97}{Fors (Bäck)}{6}{89, 89a}


\jhhousepic{047-05580}{John Sandström och Birgit Väisänen}

%%%
% [occupant] Sandström
%
\jhoccupant{Sandström}{\jhname[John]{Sandström, John} \& \jhname[Birgit]{Sandström, Birgit}}{2008--}
År 2008 byggde John Sandström och sambo Birgit Väisänen ett nytt bostadshus på en del av fastigheten Bäckström 3:96, där verkstaden, som byggts av Stig och Torsten Sundell fanns. Verkstaden revs och på 	samma område byggdes bostadshuset. John och Birgit med barnen Kim och Camilla flyttade in hösten 2008. Se mera om familjen i gamla bostadshuset, nr 82, Svinvallen I R:nr 20, på andra sidan av Ruckusvägen.


%%%
% [occupant] Sandström
%
\jhoccupant{Sandström}{\jhname[P]{Sandström, P} \& \jhname[J Kb]{Sandström, J Kb}}{1987--}
Paul Sandström och sonen John köpte 2002 en del av fastigheten Bäckström 3:96, där tegelverkstaden (nr 389) fanns. Göran Stenvall fortsatte hyra	verkstaden för sin bilmåleriverksamhet till 2004, då han blev pensionär. Medlemmar i Jeppo Lions Club använde verkstaden under ca 2 år, då de 	byggde båten Mörkrädd. År 1987 köpte Paul och John fastigheten Bäcklid 3:97 på Fors hemman. Säljare Nykarleby stad. De byggde en större bil- och maskinhall, 26 x 12 m och med en höjd på 4,5 meter, samt en dränerad plan runt omkring byggnaden, nr 89a.


%%%
% [occupant] Sundell
%
\jhoccupant{Sundell}{\jhname[Ulf]{Sundell, Ulf} \& \jhname[Ann-Britt]{Sundell, Ann-Britt}}{ 1981-1989}; Stig/Torsten	1961-1974
Ulf och Ann-Britt Sundell övertog 1981 fastigheten, Bäckström 3:96, då de köpte Ulfs barndomshem, se figur 90. Den ljusa verkstaden med tegel- och fönsterväggar byggdes på 1960-talet av Stig och Torsten Sundell. Torsten var en uppfinnare och vissa produkter tillverkades i 	verkstaden. Stig installerade kylmaskiner m.m. Stig emigrerade till Stockholm 1973. Torsten dog 1974 och fastigheten ägdes av dödsboet 1975--1981. Den 01.05.1973 hyrde Göran Stenvall verkstaden för sin bilmåleriverksamhet.


%%%
% [occupant] Källman
%
\jhoccupant{Källman}{\jhname[Edvard]{Källman, Edvard} \& \jhname[Ida]{Källman, Ida}}{1923-1953}
På nuvarande bostadstomt fanns tidigare en torkria, som tillhört bonden	Viktor Boman, som 1923 sålde 0,0123 mantal av skattelägenhet Bäck 3:11 på 	Fors hemman till Edvard och Ida Källman. Torkrian revs i början av	1950-talet.



%%%
% [house] Bäckström
%
\jhhouse{Bäckström}{3;96}{Fors (Bäck)}{6}{90}


\jhhousepic{045-05578.jpg}{Matts och Marina Stoor}

%%%
% [occupant] Stoor
%
\jhoccupant{Stoor}{\jhname[Matts]{Stoor, Matts} \& \jhname[Marina]{Stoor, Marina}}{2002--}
Servicemontör Staffan Matts Helge Stoor, \textborn 12.05.1969 i Jakobstad,	och hans hustru, merkant  Marina Ann-Sofi, \textborn Nyblom 24.03.1970 i Jakobstad, köpte 12.11.2002 fastigheten Bäckström 3:96 på 3470	m$^2$  vid Ruckusvägen 11 på Fors hemman. För att förstora tomten köpte de 11.10.2004 av Ole Sundell den icke utbrutna tomtdelen på 896 m$^2$ av Bäckström 3:96, där Källmans hus funnits, men som nu är rivna.

Matts arbetar på Sport- och maskin Johan i Nykarleby centrum och 	Marina som hemhjälpare inom hemservicen i Jeppo och Pensala.	Barn:	Jimmy Matts Alexander, \textborn 17.09.1997	och Janina Ida-Marina,	\textborn 31.01.2002.	Jimmy fick examen 03.06.2016 i el-. och automationsteknik från 	Optima i Jakobstad. Janina går i Carleborgsskolan.


%%%
% [occupant] Sundell
%
\jhoccupant{Sundell}{\jhname[Ann-Britt]{Sundell, Ann-Britt}}{1989-2002 + Ulf 1981-1989}
Maskinmontör Ulf Torsten Sundell, \textborn 12.02.1956 på fastigheten Bäckstöm 3:96, och hans hustru Ann-Britt, \textborn Kåla 31.10.1956 i Karleby, övertog Ulfs barndomshem 1981. Ulf och Ann-Britt gifte sig 1979. Ulfs mamma Ingrid bodde tillsammans med dem. År 1982 förstorade de bostadsbyggnaden med tre rum, så att Ingrid fick eget kök och kammare.
\begin{jhchildren}
  \item \jhperson{\jhname[Johanna Sofie]{Sundell, Johanna Sofie}}{05.05.1980}{}, storkökskock, g. Emet, i Umeå
  \item \jhperson{\jhname[Jessica Annamarie]{Sundell, Jessica Annamarie}}{19.12.1983}{}, formgiv. fr. Konstsk., i Sthlm
  \item \jhperson{\jhname[Jonny Ulf Matias]{Sundell, Jonny Ulf Matias}}{13.01.1988}{}, plastmek. på Prevex, i J:stad
\end{jhchildren}

År 1988 insjuknade Ingrid och hennes ben amputerades. Hon blev kvar på Nykarleby sjukhus, där hon var glad, nöjd och tacksam. Ingrid dog 	1999. Ulf och Ann-Britt skildes 1989. Ann-Britt övertog fastigheten och	hon och barnen bodde kvar i huset till 2002, då de flyttade till Nykarleby och sålde fastigheten. Ulf fick en ny familj i Nykarleby, Forsby. Han är numera gift med Monica Lindell, född Isaksson 1959 i Monå, och installerar värmepumpar m.m.


%%%
% [occupant] Sundell
%
\jhoccupant{Sundell}{\jhname[Torsten db]{Sundell, Torsten db}}{1974-1981}
\jhvspace{}


%%%
% [occupant] Sundell
%
\jhoccupant{Sundell}{\jhname[Torsten]{Sundell, Torsten} \& \jhname[Ingrid]{Sundell, Ingrid}}{1932-1974}
Ingrid Maria Källman, \textborn 1911 i Jeppo, gifte sig 26.12.1932 med Torsten Valdemar Sundell, \textborn 18.06.1913 på Silvast hemman, lägenhet Bageri 4:23, nr 72. Deras första hem blev i Ingrids föräldrars gamla hus, den röda stugan i ändan på uthusraden. År 1937 byggde 	Torsten och Ingrid sitt nya hus i tegel med rappad fasad på samma område. Bostaden bestod av 3 rum och kök och källarutrymmen under hela huset. När de äldre flickorna växte upp användes även ``röstugon'' och morföräldrarnas ``gulstugon''. Under kriget bodde evakuerade 	familjer från Hogland och Viborg och i slutet av kriget Kemijärvibor i 	bostadshusen på tomten.
\begin{jhchildren}
  \item \jhperson{\jhname[Ulla Marianne]{Sundell, Ulla Marianne}}{08.03.1933}{}, till Kanada 1957, gm Ralf Högberg fr Ytterjeppo
  \item \jhperson{\jhname[Eva Maj-Lis]{Sundell, Eva Maj-Lis}}{08.09.1935}{}, Nykarleby, hush.lärare, gm Heimer Dahlskog fr Kronoby
  \item \jhperson{\jhname[Ole Johan Valdemar]{Sundell, Ole Johan Valdemar}}{15.06.1937}{2016}, Smedsby, spec.tandtekniker, gm Christina Nordgren
  \item \jhperson{\jhname[Stig Otto Algot]{Sundell, Stig Otto Algot}}{03.02.1939}{}, kylmaskinsmontör, sambo Sthlm
  \item \jhperson{\jhname[Lisa Christina]{Sundell, Lisa Christina}}{12.01.1943}{}, telefonist, gm Sven Simons, barnträdg.lär. Nkby (nr 81)
  \item \jhperson{\jhname[Helena Ulrika]{Sundell, Helena Ulrika}}{11.10.1950}{}, tandsköterska, gm Erik Blomqvist, Kantlax
  \item \jhperson{\jhbold{\jhname[Ulf Torsten]{Sundell, Ulf Torsten}}}{12.02.1956}{}, installatör, gm Monica Lindell
\end{jhchildren}

I källaren startade Torsten 1937 en glass-och läskedrycksfabrik. När barnen blev större deltog de i arbetet. Eva Dahlskog berättar; glasflaskorna tvättades i en stor balja med sodalösning, var flaska för sig med flaskborste, fick rinna av och sköljdes och spolades inuti med en vattenstråle och fick därefter rinna av. När läskedryckerna tillverkades ``tappade'' Torsten kolsyrat vatten i flaskorna, Ingrid hällde ett mått saft i flaskan och barnen stod i rad, fyllde på i flaskan om det behövdes,	knäppte igen, satte på etiketter, och ``plocka'' i korgen. Det fanns 25 flaskor i korgen, 10 korgar hörde till en omgång. Korgarna lyftes ut i axelhöjd genom fönstret, ett av barnen drog undan korgarna. Leveransen 	skedde med handkärra.

Till glasstillverkningen togs is från älven. Isblocken kördes med häst och förvarades under en hög av sågspån. Isen hackades och blandades 	med salt för att få kyla till glassen. Glassen såldes från glasskärra bl.a. vid Sundells bageri. Glass sändes också med buss och tåg till andra orter. Så här var det före kylmaskinens tillkomst. Torsten bedrev även caférörelsen invid bageriet under några år. Barnens kamrater fick arbete som glassförsäljare och även vid tillverkningen av glass och läskedrycker. Skribenten sålde två somrar glass från kärran och sommaren 1948 var jag med i tillverkningen. För lönen köpte jag en gitarr från Anderséns fabrik i Bennäs.

\jhhousepic{Tv T Sundells 1937, E o I Kallmans roda grd m uthus o nya gula th 1930.jpg}{Torsten Sundell 1937 tv och Källmans röda resp. gula hus 1930 th, karta 6, 390a-b}

Torsten och Ingrid byggde 1954 en övre våning på huset med 5 små sovrum och badrum. Den gamla uthusraden revs och den röda stugan såldes till Lappland. En större ekonomibyggnad uppfördes vid bäcken. Torsten slutade med glass- och läskedryckstillverkning. Torsten började arbeta som montör. Han installerade mjölk- och kylmaskiner. När Silvast	Vattenandelslag bildades var han med och installerade vattenledningar. Torsten var en uppfinnare och tillsammans med sonen Stig byggde de på 1960-talet en verkstad, vars väggar bestod av stora fönster och tegel. 	Byggnaden är numera riven och på platsen finns John Sandströms nya 	bostadshus. På 1960-talet anställdes Torsten på Fiskfryseriet i Nykarleby. Varje veckoslut och helg tillbringade familjen i Jeppo.


%%%
% [occupant] Källman
%
\jhoccupant{Källman}{\jhname[Edvard]{Källman, Edvard} \& \jhname[Anna-Sofia, Ida]{Källman, Anna-Sofia, Ida}}{1923-1963}
Mjölnaren Johan Edvard Källman, \textborn 27.03.1884 på Jungarå, gifte sig med Anna-Sofia Jungarå, \textborn 1883 på Jungarå. Paret emigrerade till USA. Efter några år återvände de till Finland. Edvard arbetade då som mjölnare i kvarnen på Jungarå. Anna-Sofia dog 07.05.1908.
\begin{jhchildren}
  \item \jhperson{\jhname[Ester Sofia]{Källman, Ester Sofia}}{01.11.1904 i USA}{}
  \item \jhperson{\jhname[Artur Edvin]{Källman, Artur Edvin}}{15.09.1905 i USA}{}
  \item \jhperson{\jhname[Valter Johannes]{Källman, Valter Johannes}}{30.01.1908}{18.04.1908}, född i Jeppo
\end{jhchildren}

År 1909 gifte sig Edvard med Ida, född Nybyggar 23.06.1883 på Nybyggar hemman. Edvard fortsatte några år som mjölnare på Jungarå och Tolikko kvarnar.
\begin{jhchildren}
  \item \jhperson{\jhname[Ingrid Maria]{Källman, Ingrid Maria}}{05.03.1911}{}
  \item \jhperson{\jhname[Hjördis Emilia]{Källman, Hjördis Emilia}}{21.05.1915}{}
\end{jhchildren}
Den 04.07.1923 köpte Edvard och Ida Källman 0,0123 mantal av Bäck 3:11 benämnda lägenhet på Fors hemman. Säljare var bonden Viktor Boman och hans omyndiga barn. Edvard arbetade då som mjölnare och sågställare på Silvast kvarn. Edvard var också en skicklig murare och en mycket anlitad fiolspelman. Edvard och Ida byggde en mindre stockstuga	med uthus, som gick mot bäcken, och hade ett fähus för två kor och får. På lägenheten längre mot järnvägen fanns Bomans tröskria, där 	Sandström nu har nytt bostadshus. År 1930 byggde de ett nytt bostadshus närmare bäcken. Bostadshusen kallades ``Röstugon'' (nr 390a) och 	``Gulstugon'' (390b). Lägenheten styckades i tre olika tomter, 3:96, 3:97 och 	3:98, som tilldelades de tre flickorna. Det gula huset skrev Edvard och Ida till Ingrids son Ole Sundell.

Ida \textdied 21.09.1953 --- Edvard \textdied 07.02.1963.

Ester förblev ogift och bodde med föräldrarna. Hon arbetade som affärsbiträde 1926-1935 hos bagerifirman Otto Sundell, då hon övergick till Jeppo-Oravais Handelslag, senare Andelsringens huvudaffär på Silvast, där hon arbetade till sin pensionering. År 1973 flyttade Ester till en mindre lokal i dåvarande Andelsbankens hus, nr 92. Det gula bostadshuset revs 1992. Den röda stugan och den gamla uthusraden hade Torsten och Ingrid rivit tidigare. Ester Källman sålde sin tomt till Ingrids dotter Christina. Ifrågavarande fastighet, nr 87, ägs i dag av Bernt Norrback, som uppfört en verkstad, BN-Motor, på området.

Ester \textdied 09.07.1991.

Artur reste 1928 till Olympia, Wash., där han hade en morbror. Ingrid gifte sig med Torsten Sundell och de byggde sitt hem på Ingrids föräldrars tomt. Hjördis gifte sig med Valter Fagerholm från Forsby i Nykarleby.



%%%
% [house] Bäckåker
%
\jhhouse{Bäckåker}{3:98 (893-408-3-98)}{Fors}{6}{87}


\jhhousepic{051-05894.jpg}{BN-Motor}

%%%
% [occupant] BN-Motor
%
\jhoccupant{BN-Motor}{\jhname[Norrback Bernt]{BN-Motor, Norrback Bernt}}{2009--}
Bernt Norrback (se Fors nr 44) köpte lägenheten, stor 5284 m$^2$, den 12.07.2009  av Christina Simons (född Sundell). Efter fordonsmekanikerexamen från Optima hade Bernt anställning på Mirka fram till 2008. Att skruva bilar på egen tid intresserade alldeles speciellt.

Som 20-åring hade Bernt påbörjat sin verksamhet som hyresgäst	hos John Sandström i den bilmåleriverkstad som Göran Stenvall använt sig av, se nr 89. Då verkstaden revs för att ge rum för Johns egnahemshus, flyttade Bernt in till gamla Union (Grötas nr 11) för en kort tid och därefter till Leif Häggbloms pälshus (Gunnar nr 34b); Leif hade då lagt ner farmen. Under sina år på denna plats, dvs 2007-08, mognade	planerna på en alldeles egen verkstad och nybygge i Silvast.

Efter utförd grundundersökning av KS-Geokonsult och pålning av Roland Norrgård, byggdes verkstaden av betongelement från Tara-Element, Nedervetil, och med hjälp av Morits Nygård som träkarl för formar och innertak. Byggnaden på 150 m$^2$ blev klar år 2010. I den utförs	motorrenoveringar, metallarbeten och bilreparationer. En ekonomibyggnad från Jeppo Biogas, ett f.d. truckgarage, har senare placerats på tomten.



%%%
% [house] Vikinga
%
\jhhouse{Vikinga}{Vikinga 3:34 och Vikinga 1 3:40}{Fors}{6}{88}


\jhhousepic{099-05637.JPG}{Vy från bäcken}

%%%
% [occupant] Hietala
%
\jhoccupant{Hietala}{\jhname[Seppo]{Hietala, Seppo} \& \jhname[Sylvi]{Hietala, Sylvi}}{1985--}
Seppo Jorma Aatos, \textborn 24.05.1959 i Ylihärmä, gifte sig 1976 med Sylvi, \textborn	20.11.1957,	född Kalliosaari från Alahärmä.
\begin{jhchildren}
  \item \jhperson{\jhname[Katja Kristiina]{Hietala, Katja Kristiina}}{16.12.1976}{}, gift Peltonen/Jeppo Potatis
  \item \jhperson{\jhname[Annika Katariina]{Hietala, Annika Katariina}}{19.01.1982}{}, gift Keski-Petäjä/Jorois kommun
  \item \jhperson{\jhname[Juha Jorma Aatos]{Hietala, Juha Jorma Aatos}}{28.04.1988}{}, processkötare/KWH Mirka, (karta 5, nr H2)
\end{jhchildren}

Paret flyttade från Alahärmä till Jeppo år 1975 och till nuvarande hus år 1985. Samtliga barn är födda i Jeppo. Seppo jobbar som maskinskötare på KWH Mirka. Fru Sylvi har varit packare på samma företag, är numera pensionär.

Huset, som i dag mäter 90 m$^2$, byggdes 1942 av Haniel Orrholm med fru Anna. Gårdsbyggnaden, nr 88b, uppfördes 2006 av Seppo Hietala.


%%%
% [occupant] Hietamäki
%
\jhoccupant{Hietamäki}{\jhname[Erkki]{Hietamäki, Erkki} \& \jhname[Regina]{Hietamäki, Regina}}{1978-1985}
Erkki Matti, \textborn 28.12.1953 i Ylihärmä, gifte sig år 1978 med 19-åriga Regina Karita, \textborn 22.03.1959, född Dahlström på Jungarå.
\begin{jhchildren}
  \item \jhperson{\jhname[Tanja Heidi]{Hietamäki, Tanja Heidi}}{14.11.1978}{}
  \item \jhperson{\jhname[Juha-Matti]{Hietamäki, Juha-Matti}}{20.12.1983}{}
\end{jhchildren}
Gården köptes in år 1978 och blev parets första hem. Regina har arbetat på Sundells bageri och Erkki som chaufför på järnvägsbil. Erkki köpte Anders Nymans gård på Grötas år 1985 (Grötas nr 112) och flyttade ensam dit.


%%%
% [occupant] Dahlström
%
\jhoccupant{Dahlström}{\jhname[Alfons]{Dahlström, Alfons}}{1964-1978}
Ungkarlen Alfons köpte gården i samband med att brodern Olof övertog förädrahemmanet på Jungarå. Som nya ägare bodde han egentligen aldrig på gården, endast tidvis. Hans adress var i Nykarleby. Däremot beboddes huset av Alfons' föräldrar, Anders och Gerda (karta 17, nr 122), fram till pappans död 1969. Mamman, som flyttat till en pensionärsbostad i centrum, dog där år 1980.

Alfons gifte sig sedermera med Marja-Liisa. Han lät i början på 1970-talet riva det gamla fähuset jämte gårdsbastun.


%%%
% [occupant] Dahlström
%
\jhoccupant{Dahlström}{\jhname[Olof]{Dahlström, Olof} \& \jhname[Sirkka]{Dahlström, Sirkka}}{1963-1964}
Olof, \textborn 06.10.1940, gift med Sirkka, \textborn 22.05.1938. Se karta 4, nr 50, där paret är bosatt sedan 5.5.1989. Paret var gårdsägare enbart ett år, varefter de sålde lägenheten till Olofs bror Alfons.


%%%
% [occupant] Rintala
%
\jhoccupant{Rintala}{\jhname[Fina]{Rintala, Fina}}{1959-1964}
Fina var hyresgäst i huset, som hon i olika omvarv delade med sonen Oiva och barnbarnen Lasse, Rainer och Juhani innan hon flyttade till Hellin Strengells hus, granne till Silfvast. - Jfr karta 5, nr 354a.


%%%
% [occupant] Orrholm
%
\jhoccupant{Orrholm}{\jhname[Haniel]{Orrholm, Haniel} \& \jhname[Anna]{Orrholm, Anna}}{1942-1959}
Jakob Haniel Jakobsson, \textborn 14.09.1896 i Jeppo Silvast, gifte sig den 27.02.1929 med Anna Sofia, \textborn 12.08.1905, född Nygren från Oravais. Efter giftermålet bodde Haniel och Anna i Oravais, där barnen är födda.
\begin{jhchildren}
  \item \jhperson{\jhname[Elsbeth Sofi]{Orrholm, Elsbeth Sofi}}{07.03.1929}{12.12.2007 i Torshälla}, g. m. G. Sandström
  \item \jhperson{\jhname[Max Per Jakob]{Orrholm, Max Per Jakob}}{30.06.1931}{30.9.2011 i Eskilstuna}
  \item \jhperson{\jhname[Karl-Erik]{Orrholm, Karl-Erik}}{11.07.1935}{}, bor i Vålberg, Värmland
  \item \jhperson{\jhname[Frej Viking]{Orrholm, Frej Viking}}{10.03.1940}{}, bor i Eskilstuna
\end{jhchildren}
Haniel började som ung hjälpa till vid Jungells gästgiveri (nr 380) i Silvast med att skjutsa resande, först med häst, sedan med bil. Han hade en av de första personbilarna i Jeppo. Han var även mycket intresserad av träslöjd och blev känd som duktig ``träkarl''. När han gift sig med Anna flyttade familjen till Björneborg, där han fick tjänst som vaktmästare vid Björneborgs svenska samskola. Efter en tid återvände de till Jeppo, där han fick arbete som finsnickare vid Kiitola karosserifabrik. Senare var han en tid förman vid järnvägens bränsle-/vedförråd i Jeppo.

Haniel och Anna köpte tomten för 15000 mk den 13.10.1941 av Jeppo kommun och byggde huset år 1942 med kök, ett rum, vindsrum och hel källare. Därtill uppfördes fähus, en gårdsstuga med ett rum samt skild bastubyggnad. I slutet av 1950-talet hade alla barnen emigrerat till Sverige.

Haniel \textdied 27.5.1957  ---  Anna \textdied 10.11.1959



%%%
% [house] Mellanåkern
%
\jhhouse{Mellanåkern}{4:196}{Silvast}{6}{93}

Styckad av stomlägenhet Nygård 4:9

\jhhousepic{035-05570.jpg}{Kurt och Mona Stenvall}

%%%
% [occupant] Stenvall
%
\jhoccupant{Stenvall}{\jhname[Kurt]{Stenvall, Kurt} \& \jhname[Mona]{Stenvall, Mona}}{1986--}
Kurt Peter Einar Stenvall, \textborn 11.09.1960 på Nygård 4:9 av Silvast hemman, och Mona Marja-Kristina Sandström, \textborn 12.05.1963 i 	Jakobstad, köpte 21.07.1986 fastigheten Mellanåkern 4:196 av Doris Lundvik och Dorthy Gunnar. Mellanåkern utgör en del av	stomlägenhet Nygård 4:9 (numera 4:208). Den 03.09.1986 gjordes en bytesaffär med Vasa Andelsbanks grannlägenhet, som utgör en del av stomlägenhet Bäckstrand 4:33, Andelsbanken 4:199. Vasa Andelsbank fick en del mot vägen av Mellanåkern och Kurt och Mona förstorade	tomten mot älven. De grundrenoverade båda våningarna i	bostadshuset samt uthuset 1987-1988.

Kurt och Mona vigdes i Jeppo	kyrka den 8 april 1988. Den 29.04.2003 förstorade de tomten, då de köpte tilläggsmark av Vasa Andelsbank, en triangel mot Mejeriet av fastigheten Andelsbanken 4:199. Tomten är c:a 0,6 ha.	Bostadshuset förstorades med en tillbyggnad i en våning 2007.

Kurt Stenvall är utbildad elingenjör och är VD för Jeppo Kraft Andelslag och Jeppo Biogas Ab, som han varit initiativtagare till och dragare av. Mona är ekonom och har arbetat som bokförare och redovisare i 30 år på	KWH-Plast i Jakobstad, numera KWH Plast Schur Flexibles Oy
\begin{jhchildren}
  \item \jhperson{\jhname[Alexandra Marie-Leone]{Stenvall, Alexandra Marie-Leone}}{27.09.1988}{}
  \item \jhperson{\jhname[Christoffer Carl Jonathan]{Stenvall, Christoffer Carl Jonathan}}{05.01.1996}{}
\end{jhchildren}
Kurt har varit styrelsemedlem i Jeppo Uf, SFP, Nkby stads tekniska samt social- \& hälsovårdsnämnder, ordf. och styrelsemedlem i JUO. År 2008 utsågs han till Årets jeppobo.

Mona har verkat som styrelsemedl./kassör i Jeppo Uf, Hem- och skolaföreningen i lågstadiet och ordf. i högstadiet. Hon har också suttit i många nämnder i Nkby stad samt i Jeppo FRK-avd.

Alexandra blev student 2007 och 2011 restonom från Mellersta Öb:s yrkeshögskola. 2014 fick hon examen MS in Tourism Studies från Östersund Mittuniversitet. Påbyggnadsstudier i Växjö Univ. 2014/2015 i 6 månader. Efter andra arbetsvikariat fick hon 20.11.2016 vikariat som kanslist inom Pedersöre MI.

Christoffer blev el- och automationsmontör från Optima våren 2015. Efter avtjänad värnplikt i december 2015, studerar han energi- och miljöteknik vid Yrkeshögskolan Arcada i Helsingfors.


%%%
% [occupant] Lundvik Doris
%
\jhoccupant{Lundvik Doris}{\jhname[Gunnar Dorthy]{Lundvik Doris, Gunnar Dorthy}}{1978-1986}
Då Doris' och Dorthys mor Edit Westin avled 1977 erhöll Doris Lundvik och Dorthy Gunnar vid arvsskifte 01.09.1978 fastigheten Mellanåkern. Deras far Ivar fortsatte som änkling att bo i huset. Efter några år flyttade han till Hagalund servicehem. Dorthys son Göran Gunnar bodde i huset tills det såldes 1986. På tomten mot älven hade Göran bikupor för produktion av honung. Se Lilland 4:36, karta 5, nr 79 samt Gunnar nr 27.


%%%
% [occupant] Westin
%
\jhoccupant{Westin}{\jhname[Ivar]{Westin, Ivar} \& \jhname[Edit]{Westin, Edit}}{1970-1978}
Den 28.04.1970 köpte Anders Ivar Westin, \textborn 21.04.1906 i Ironwood, USA, och hans hustru Edit Maria, född Jungarå, \textborn 29.09.1905 	på Jungarå,	fastigheten Mellanåkern, som utgjorde en del av stomlägenhet Nygård 4:9. Säljare var Anna Henriksson, Åke och Erik Stenvall. Ivar och Edit flyttade som pensionärer från sitt hus på Grötas, där de haft ett litet jordbruk (se Grötas, karta 10, nr 101).


%%%
% [occupant] Stenvall
%
\jhoccupant{Stenvall}{\jhname[Elis]{Stenvall, Elis} \& \jhname[Aina]{Stenvall, Aina}}{1946-1968}
Elis och Aina Stenvall ärvde 1/6 av lägenhet Nygård 4:9 efter deras son Waldemar, som stupade 1943 i fortsättningskriget. I samband med att sonen Paul 20.02.1951 köpte Nygård 4:9 av sina föräldrar och syskon, avskildes Mellanåkerskiftet om c:a 0,5 ha jämte därpå befintliga byggnader till en självständig lägenhet. År 1946 hade Elis och Aina byggt ett bostadshus och uthus på området. I medlet av juni 1955, då sonen Paul gifte sig, flyttade Elis och Aina till sitt nya bostadshus. De hade en ko i ett litet fähus. Aina dog plötsligt i en stroke på kvällen den 24.05.1961 i hemmet. Elis levde ensam och klarade sig själv i huset tills han efter två dagar på sjukhus dog 19.04.1968, dagen efter sin 79-årsdag. Han hade på dagen då han blev sjuk sågat av ett äppelträd, som skymde utsikten till landsvägen.


\jhbold{Hyresgäster}:

Från november 1968-juli 1970 hyrde Bruno och Siv \jhbold{Strengell} huset. Siv hade sin damfrisering i en del av vardagsrummet. I köket fanns kallvatten- och slaskledning. För kundernas hårtvätt värmde hon på en	vedspis i köket vatten, som hälldes i ett ämbar. Se Fors, nr 102.

1953--1955 hyrde Alma \jhbold{Lillkungs} styvmor, Elin Sirén, född Ståhlberg i Oravais, huset. Hon fick bo kvar i vardagrummet en kortare tid, när Elis och Aina flyttade in.

Den 01.03.1948 hyrde disponent Anders Evert \jhbold{Brandt}, \textborn 19.03.1922 och hans hustru Saga Olga Linnéa, född Lomas, \textborn 02.12.1913, bostadshuset fram till den 31.12.1952. De gifte sig samma år som Evert blev ny disponent för Jeppo Kraft Andelslag och paret fick sitt första barn.
\begin{jhchildren}
  \item \jhperson{\jhname[Stig Björn Anders]{Stenvall, Stig Björn Anders}}{24.08.1948}{}, bor i Jakobstad
  \item \jhperson{\jhname[Hans Evert]{Stenvall, Hans Evert}}{1955}{}, föddes efter flytten, bor i Jakobstad
\end{jhchildren}
Evert avgick som disponent på andelslaget den 28.10.1952. Familjen flyttade till Kronoby.



%%%
% [house] Verkstaden
%
\jhhouse{Verkstaden}{3:70}{Fors}{6}{94}


\jhhousepic{036-05571.jpg}{Erik Sundqvist och Camilla Viklöv}

%%%
% [occupant] Sundqvist \& Viklöv
%
\jhoccupant{Sundqvist \& Viklöv}{\jhname[Erik]{Sundqvist \& Viklöv, Erik} \& \jhname[Camilla]{Sundqvist \& Viklöv, Camilla}}{2017--}
Den 01.09.2017 flyttar Anders och Ann-Kristin tillbaka till Nykarleby och sonen Erik med sambo \jhname[Camilla Viklöv]{Viklöv, Camilla}, \textborn 14.12.1994 i Kronoby, tar över lägenheten. Camilla studerar nu sista året vid Utbildningslinjen för barnträdgårdslärare, fakulteten för pedagogik och välfärdsstudier vid Åbo Akademi. Erik är rörmontör på NKAB.


%%%
% [occupant] Sundqvist
%
\jhoccupant{Sundqvist}{\jhname[Anders]{Sundqvist, Anders} \& \jhname[Ann-Kristin]{Sundqvist, Ann-Kristin}}{1997--2017}
Erik Anders Erland, \textborn 26.08.1955 i Nykarleby, gifte sig den 28.12.1991 med Märta Ann-Kristin, \textborn 14.11.1961, född Björk från Karleby, Såka.
\begin{jhchildren}
  \item \jhperson{\jhname[Emma Kerstin Marita]{Sundqvist, Emma Kerstin Marita}}{13.02.1993}{}, stud.merkonom och datanom
  \item \jhperson{\jhbold{\jhname[Erik Anders]{Sundqvist, Erik Anders}}}{02.08.1995}{}, vvs-montör
\end{jhchildren}

Innan familjen slog sig ner i Jeppo bodde Anders och Ann-Kristin i Nykarleby. Anders är lagerarbetare vid KWH Mirka i Jeppo och Ann-Kristin har bl.a. arbetat som projektledare vid Optima samkommun i Jakobstad innan hon övergick till funktionen som verksamhetsledare på Psykosociala föreningen Contact rf på samma ort.

Husets bottenvåning är numera inredd för boende med bl.a. sovrum, bastu och tvättstuga. Entrén har byggts om till inglasat uterum.


%%%
% [occupant] Björkqvist
%
\jhoccupant{Björkqvist}{\jhname[Stig]{Björkqvist, Stig} \& \jhname[Camilla]{Björkqvist, Camilla}}{1995-1997}
Stig-Erik Ivar, \textborn 11.02.1960, gifte sig 20.06.1992 med Camilla Ann-Christine Strengell, \textborn 11.12.1971 i Jeppo.
\begin{jhchildren}
  \item \jhperson{\jhname[Emilia Josefine]{Björkqvist, Emilia Josefine}}{04.10.1988}{}, stud. i Aberdeen
  \item \jhperson{\jhname[Oscar Ivar Jonathan]{Björkqvist, Oscar Ivar Jonathan}}{22.01.1996}{}
  \item \jhperson{\jhname[Filip Erik Cesar]{Björkqvist, Filip Erik Cesar}}{23.04.2002}{}
  \item \jhperson{\jhname[Alexander Paul Lucas]{Björkqvist, Alexander Paul Lucas}}{08.08.2008}{}
\end{jhchildren}

Efter flytten från Jeppo har familjen bott i Esbo och Maastricht. Nu finns bopålarna i Kyrkslätt. Stig jobbar som Business controller för Rettig Group och Camilla som barnträdgårdslärare för Kyrkslätt stad.


%%%
% [occupant] Lindén
%
\jhoccupant{Lindén}{\jhname[Gurli]{Lindén, Gurli}}{1988-1995}
Se nedan! Efter separationen från Boris blev Gurli och Eva Stina Byggmästar	boende i huset. Gurli fortsatte sitt etablerade och framgångsrika författarskap. Under den här tiden utkom hon med:
\begin{enumerate}
  \item 1989	``Framtid'', Författarnas Andelslag
  \item 1990	``Pyramid'', Författarnas Andelslag
  \item 1993	``Grodan'', Författarnas Andelslag
\end{enumerate}


Andelslagets verksamhet var under en stor del av 70- och 80-talet stationerad i huset. Eva Stina inledde här sitt författarskap.


%%%
% [occupant] Lindén
%
\jhoccupant{Lindén}{\jhname[Boris]{Lindén, Boris} \& \jhname[Gurli]{Lindén, Gurli}}{1962-1988}
Boris Runar, \textborn 07.06.1940 i Jeppo, gifte sig den 02.04.1961 med Gurli Ingegärd Kronqvist, \textborn 18.01.1940, från Öja.
\begin{jhchildren}
  \item \jhperson{\jhname[Thomas]{Lindén, Thomas}}{01.09.1961}{}
  \item \jhperson{\jhname[Andreas]{Lindén, Andreas}}{08.05.1964}{}
  \item \jhperson{\jhname[Johanna]{Lindén, Johanna}}{09.06.1967}{}
\end{jhchildren}

Boris utnyttjade verkstaden i huvudsak som bilverkstad. Gurli hade syateljé i hemmet under 60-talet. Senare, år 1970, flyttade verkstadsverksamheten till Boris' nybyggda bensinstation och verkstad Union (Grötas nr 11), där också annat nytänkande tog form och där man får anta att ett frö till företaget BSB Mekan grodde.


\jhbold{På hyra eller arrende}:

\jhbold{Enlund} Valdemar \& Berit, 1955-1960

Anders Lennart Valdemar Enlund, \textborn 27.09.1925 i Kengo, gifte sig den 08.07.1950 med Berit Maria Elenius, \textborn 02.09.1926 i Jungar, Jeppo.
\begin{jhchildren}
  \item \jhperson{\jhname[Lisen]{Lindén, Lisen}}{31.05.1951}{}, bor i Terjärv
  \item \jhperson{\jhname[Britten]{Lindén, Britten}}{04.12.1956}{}, bor i Jeppo, se Fors nr 131
  \item \jhperson{\jhname[Anders]{Lindén, Anders}}{30.11.1966}{}, bor i Närpes
\end{jhchildren}
\jhname[Valdemar och Berit]{Enlund, Valdemar \& Berit}, som bodde på Holmen, hyrde gården av Vilhelm Björklund. Tillsammans fortsatte Valdemar och hans bror Bror med samma typ av arbeten som redan förekommit i verkstaden. År 1960, då Valdemar fick anställning som bilskollärare vid Haldin \& Rose i Jakobstad, flyttade familjen bort från orten.


%%%
% [occupant] Björklund
%
\jhoccupant{Björklund}{\jhname[Vilhelm]{Björklund, Vilhelm} \& \jhname[Elsa]{Björklund, Elsa}}{1954-1962}
Jarl Vilhelm, \textborn 17.07.1906 i Jeppo, gifte sig den 19.08.1934 med Elsa Erika Lindström, \textborn 06.10.1912 i Vasa, från Jeppo.
\begin{jhchildren}
  \item \jhperson{\jhname[Karin]{Björklund, Karin}}{1936}{}
  \item \jhperson{\jhname[Gustav]{Björklund, Gustav}}{1942}{}
\end{jhchildren}

Den 08.09.1954 köpte Vilhelm den 600 m$^2$ stora tomten från lägenheten Nybonde 3:56 för sin verkstad. I verkstaden utförde han ensam all slags reparationer på traktorer, lantbruksmaskiner och bilar, eller enkelt sagt, allt det som folk kom med som skulle åtgärdas. Vilhelm saknade sin ena hand efter en olycka som barn, den var endast en liten klump längst ut på armen, men han behärskade ändå sitt hantverk mycket skickligt.

Ett utrymme i bostaden användes som frisörsalong för Elsa. Familjen flyttade så småningom till Sverige, där Vilhelm och Elsa också slutade sina dagar.

Vilhelm \textdied 1982  ---  Elsa \textdied 2007


%%%
% [occupant] Westerlund
%
\jhoccupant{Westerlund}{\jhname[Ernst]{Westerlund, Ernst} \& \jhname[Agnes]{Westerlund, Agnes}}{1945-1954}
Ernst Eric, \textborn 15.11.1913 i Jeppo, gifte sig den 10.10.1939 med Agnes Emilia, \textborn 21.01.1914, född Lindfors på Silvast (se 359).
\begin{jhchildren}
  \item \jhperson{\jhname[Greta]{Westerlund, Greta}}{02.09.1942}{}
  \item \jhperson{\jhname[Gunnevi]{Westerlund, Gunnevi}}{27.08.1947}{}
  \item \jhperson{\jhname[Gun-Britt]{Westerlund, Gun-Britt}}{11.04.1949}{}
\end{jhchildren}

Ernst och Agnes övertog denna del av lägenheten Nybonde 3:56 den 6.2.1945, men torde knappast ha bott i det nybyggda huset, eftersom de redan då hunnit etablera sig på Timmerbackholmen (karta 9, nr 127). Därför arrenderade de ut och senare sålde det till Vilhelm Björklund för hans ändamål.


%%%
% [occupant] Westerlund
%
\jhoccupant{Westerlund}{\jhname[Eric]{Westerlund, Eric} \& \jhname[Amanda]{Westerlund, Amanda}}{1938-1945}
Eric Jakobsson, \textborn 18.05.1888, gifte sig den 02.02.1913 med Amanda Irene Jakobsdotter Nybonde, \textborn 29.07.1892 i Soklot.
\begin{jhchildren}
  \item \jhperson{\jhbold{\jhname[Ernst Eric]{Westerlund, Ernst Eric}}}{15.11.1913}{13.12.1992}, i USA
  \item \jhperson{\jhname[Bruno Johannes]{Westerlund, Bruno Johannes}}{26.06.1919}{}
  \item \jhperson{\jhname[Edna Amanda]{Westerlund, Edna Amanda}}{08.06.1922}{}, gift Eklöv
\end{jhchildren}

Eric och Amanda byggde huset åren 1937--38. Övre våningen byggdes som bostad medan markvåningen, uppförd i rött tegel, inreddes för 	företagsändamål. Det sägs att man under kriget tillverkade potatismjöl i husets  bottenvåning. Då Eric dog redan 13.10.1940, flyttade Amanda tillbaka till den ursprungliga bostaden (nr 396).

Amanda \textdied 07.07.1984



%%%
% [house] Nybonde
%
\jhhouse{Nybonde}{3:126}{Fors}{6}{394}

\jhhousepic{Ture och Mia Westerlund.jpeg}{Ture och Mia Westerlund vid norra husgaveln under björkarnas lövverk}

%%%
% [occupant] Westerlund
%
\jhoccupant{Westerlund}{\jhname[Ture]{Westerlund, Ture} \& \jhname[Mia]{Westerlund, Mia}}{1923 – 1995}
Ture Evert Jakobsson Westerlund, backstugusittare,  \textborn 30.04.1897, var son till landspolis Jakob Westerlund. Han gifte sig med Anna Irene Eriksdr., \textborn 24.11.1894, som var dotter till ``Stöipas-Erik'' (se karta 5, nr 376). 17 år gammal flyttade hon år 1911 till Jakobstad som piga, men återkom 1913 och arbetade vid Sundells bageri. Hon gifte sig 29.07.1917 med Ture.

Makarna fick dottern \jhperson{\jhname[Ragnhild Anna Irene (Nanna)]{Westerlund, Nanna}}{13.01.1921}{}, men modern Anna dog redan följande dygn, den 14.01.1921.

\jhhousepic{Turebutelj.png}{Tures limonadflaska med rar etikett och patentkapsyl}

Nanna gifte sig som vuxen med Gunnar Holm och familjen livnärde sig en tid på jordbruk innan de i början av 1960-talet emigrerade till Seattle, USA. Ture gifte om sig 30.07.1922 med Maria (Mia) Elisabeth Rönnlund, \textborn 24.08.1903 i Nykarleby.
\begin{jhchildren}
  \item \jhperson{\jhname[Tage Rolf Gustav]{Westerlund, Tage Rolf Gustav}}{05.02.1934}{}
  \item \jhperson{\jhname[Alf E. Evert]{Westerlund, Alf Erling Evert}}{06.03.1941}{15.06.1941}
\end{jhchildren}

Rolf emigrerade till USA och gifte sig där med Margot Sandberg, \textborn 19.11.1935, som tillsammans med föräldrarna Joel och Jenny Sandberg år 1955 emigrerat till Seattle, USA (se bl.a. Silvast, karta 5, nr 60).

Under sin ungdomstid i Jeppo var Rolf en uppskattad målvakt i JIF:s fotbollslag, som kanske blev styvmoderligt behandlat. Bäckman på  Keppo fick laget att entusiasmeras med en energisk gåpåaranda och under en tid var fotbollsmatcherna publika tillställningar och jublet steg mot skyn, inte enbart när hemmalaget gjorde mål, utan också när Rolf framgångsrikt lyckades freda målet.

Den tomt på vilken Ture Westerlund byggde familjens hus i början av 1920-talet, tillhörde hans hemgård och fanns mitt emot Ungdomslokalen. I ett litet uthus experimenterade han och påbörjade tillverkning av lemonad i liten skala. Egna försäljningsetiketter var en självklarhet, vilket framgår av bifogad bild. Han får i fortsättningen benämningen ``läskedrycksfabrikant''. Ryktet förtäljer att han också bryggde öl under namnet ``Dunkel'', men den historien har inte kunnat bekräftas. Knappast hittar vi här upprinnelsen till uttrycket att det är ``mörkt i Jeppo''!

Ture skaffade också bil i ett tidigt skede och bedrev taxiverksamhet vid sidan av lemonadtillverkningen. Efter en taxikörning till Nykarleby sjukhus på kvällen den 17.02.1952, besökte han restaurang von Döbeln för en kopp kaffe innan hemfärden och där drabbades han av hjärtslag, sjönk ihop vid bordet och dog.

Huset revs 1995 av Lennart Gustafsson och dess vindstrappa har funnit vägen till ett fritidshus i Oravais. Tures svärdotter Margot Westerlund i USA står som nuvarande ägare till lägenheten.

Ture \textdied 17.02.1952 --- Maria (Mia) \textdied 05.09.1994 efter många år på åldringshem i Nykarleby.



%%%
% [house] Dahllund
%
\jhhouse{Dahllund}{3:42}{Fors}{6}{396}


\jhhousepic{Amanda W.jpg}{Amandas hus, ur Sv. Öb:s bebyggelse i ord och bild, Borås 1965}

%%%
% [occupant] Jeppo Kraft
%
\jhoccupant{Jeppo Kraft}{\jhname[Alg]{Jeppo Kraft, Alg}}{2001--}
I april 2001 köpte Jeppo Kraft Andelslag den andra halvan av Dahllund 3:42 av Kapiteeli Oy och kom i och med detta att äga hela lägenheten. Amandas bostadshus revs hösten 2002.


%%%
% [occupant] Jeppo Sparbank
%
\jhoccupant{Jeppo Sparbank}{\jhname[Arsenal]{Jeppo Sparbank, Arsenal}}{1987-2001}
Jeppo Sparbank hade införskaffat lägenheten med tanke på ett ev. nybygge, som ändå inte kunde förverkligas då turbulensen i en krisande sparbanksrörelse uppstod i början på 1990-talet (jfr nr 81). I efterföljande villervalla övertog skräpbanken Arsenal både Sparbanks- och Dahllundslägenheten och sålde den sistnämnda vidare till Jeppo Kraft Alg år 2001.


%%%
% [occupant] Dödsbo
%
\jhoccupant{Dödsbo}{\jhname[Westerlund]{Dödsbo, Westerlund}}{1984-1987}
\jhvspace{}


%%%
% [occupant] Westerlund
%
\jhoccupant{Westerlund}{\jhname[Eric]{Westerlund, Eric} \& \jhname[Amanda]{Westerlund, Amanda}}{1912-1984}
Eric Jakobsson, \textborn 18.05.1888, gifte sig den 02.02.1913 med Amanda Irene Jakobsdotter Nybonde, \textborn 29.07.1892 i Soklot. Jfr ovan med Fors nr 94.
\begin{jhchildren}
  \item \jhperson{\jhbold{\jhname[Ernst Eric]{Westerlund, Ernst Eric}}}{15.11.1913}{13.12.1992}, i USA
  \item \jhperson{\jhname[Bruno Johannes]{Westerlund, Bruno Johannes}}{26.06.1919}{}
  \item \jhperson{\jhname[Edna Amanda]{Westerlund, Edna Amanda}}{08.06.1922}{}, gift Eklöv
\end{jhchildren}

Eric beskrivs som bonde på Fors på del av faderns hemman (5/96 mantal), som övertogs 1912 från Jakob och h h Kristina Westerlund. Eric och Amanda uppförde bostadshuset med två rum och kök år 1913 i trä under tegeltak. Samma år byggdes gårdsbyggnaden i trä under pärttak. Mellan bostaden och uthuset grävdes en vattenbrunn som gav bra dricksvatten ännu in på 70-talet, vilket de närmaste grannarna med tacksamhet vindade upp och utnyttjade.

Nya företagsplaner slog så småningom rot och åren 1937-38 lät paret bygga ett nytt hus på angränsande tomt (nr 94, Verkstaden 3:70). Då Eric dog en kort tid efteråt, flyttade Amanda tillbaka till den ursprungliga bostaden.

Eric \textdied 13.10.1940  ---  Amanda \textdied 07.07.1984.


%%%
% [occupant] Westerlund
%
\jhoccupant{Westerlund}{\jhname[Jakob]{Westerlund, Jakob} \& \jhname[Kristina]{Westerlund, Kristina}}{1910-1912}
Jakob Westerlund, \textborn 25.10.1850 på Biggas, son till Jan Jakobsson Biggas, gift 11.10.1874 m. Kristina Eriksdotter Pet, \textborn 04.11.1854 på Pet.
\begin{jhchildren}
  \item \jhperson{\jhname[Joel]{Westerlund, Joel}}{1875}{}, till Amerika
  \item \jhperson{\jhname[Viktor]{Westerlund, Viktor}}{1877}{}, till Nykarleby 1897
  \item \jhperson{\jhname[Johannes]{Westerlund, Johannes}}{1880}{}, till Amerika
  \item \jhperson{\jhname[Hanna]{Westerlund, Hanna}}{1883}{1951}, gift Sjöblom
  \item \jhperson{\jhname[Wilhelm]{Westerlund, Wilhelm}}{1886}{}, till Amerika, g m jeppoflicka (1 barn)
  \item \jhperson{\jhbold{\jhname[Eric]{Westerlund, Eric}}}{1888}{}, bn på Fors (5/96 mtl)
  \item \jhperson{\jhname[Emil]{Westerlund, Emil}}{1890}{}
  \item \jhperson{\jhname[Ingrid Maria]{Westerlund, Ingrid Maria}}{1892}{}
  \item \jhperson{\jhname[Gustav]{Westerlund, Gustav}}{1894}{}
  \item \jhperson{\jhname[Ture Evert]{Westerlund, Ture Evert}}{1897}{1952}, taxi, limonadtillverkning i Silvast
\end{jhchildren}
Jakob var byskollärare i Korsholm 1870--72. Familjen flyttade via Nykarleby till Stenbacken år 1886/-88, där de bodde fram till 1910, då de flyttade till Silvast. Jakob tjänstgjorde som landspolis med många förtroendeuppdrag i kommun och församling. Ägare till en del av Fors skattehemman Rno 3:8; svårt att veta var de bodde under dessa två år.

Jakob \textdied 08.03.1914  --- Kristina \textdied 14.01.1936



%%%
% [house] Blom
%
\jhhouse{Blom}{3:65}{Fors}{6}{95}


\jhhousepic{041-5876.jpg}{Johan och Emma Sjölind}

%%%
% [occupant] Sjölind
%
\jhoccupant{Sjölind}{\jhname[Johan]{Sjölind, Johan} \& \jhname[Emma]{Sjölind, Emma}}{2016--}
Paret Sjölind köpte dödsboet i december 2015 och har sedan dess renoverat huset från grunden och därvid utfört det mesta arbetet på egen hand. Det blev inflyttningsklart i juli 2017, då nedre våningen kunde tas i bruk. Övre våningen, fasad- och tomtarbeten åtgärdas efter hand.
\begin{jhchildren}
  \item \jhperson{\jhname[Helmer Gustav]{Sjölind, Helmer Gustav}}{05.07.2015, hus nr 81}{}
  \item \jhperson{\jhname[Valter]{Sjölind, Valter}}{05.09.2017}{}
\end{jhchildren}
Johan, \textborn 1990, är företagare inom mark- och jordbyggnadsbranschen och Emma, \textborn 1991, har tjänst som lärare på Jeppo-Pensala skola. Se bakgrund, hus 81, mellersta lokalen.


%%%
% [occupant] Dödsbo
%
\jhoccupant{Dödsbo}{\jhname[Lassén]{Dödsbo, Lassén}}{2012-2015}
Under den här perioden utnyttjades lägenheten sommartid som gemensam samlingsplats för syskonen med familjer.\jhvspace{}


%%%
% [occupant] Lassén
%
\jhoccupant{Lassén}{\jhname[John]{Lassén, John} \& \jhname[Lea]{Lassén, Lea}}{1957-2012}
John Eliel Lassén, \textborn 09.09.1915 i Esse, gifte sig med Lea Alice Johanna Anderssén, \textborn 08.07.1921, från Måtars.
\begin{jhchildren}
  \item \jhperson{\jhname[Lars Christer]{Lassén, Lars Christer}}{25.10.1948}{} Bor i Uppsala, Sverige
  \item \jhperson{\jhname[Ann-Christine]{Lassén, Ann-Christine}}{07.05.1951}{} Bor i Larsmo
  \item \jhperson{\jhname[Ulla Rose-May]{Lassén, Ulla Rose-May}}{22.09.1952}{03.08.1987 i Stockholm}
  \item \jhperson{\jhname[Gun-Lis Charlotta]{Lassén, Gun-Lis Charlotta}}{30.01.1954}{29.05.2015 i Stockholm}
  \item \jhperson{\jhname[Hans Johan]{Lassén, Hans Johan}}{29.11.1956}{14.4.1957}
\end{jhchildren}

John och Lea bodde ursprungligen på Måtar men hade jobb i Silvast. År 1957 köpte de lägenheten av William och Johanna Sjöbloms dödsbo efter att ha bott på hyra det första året. Paret drev under åren 1952/53 -- 1968 gemensamt diversehandeln ``Jeppoboden'' (karta 5, nr 380) på nuvarande Östra Jeppovägen, norra delen av UF:s tomt. Lea hade redan i unga år arbetat på Lainas kafé, gått en kort frisörsutbildning och efteråt arbetat som frisörska med tillhörande kemikaliabutik i s.k. Jakobssons hus (nr 370), granne till järnvägsstationen.

John sårades allvarligt i vinterkriget. Under fortsättningskriget var han som krigsveteran på hemmafronten och tjänstgjorde som effektiv soldatgosseledare i Jeppo skyddskår. Efter kriget fungerade han också som arméns materialförvaltare i Jeppo och hade då att hantera kinkiga ärenden.

I det civila blev John känd som mycket idrottsintresserad och var bl.a. god långdistanslöpare. Han syntes därför ständigt och aktivt med vid olika tävlingsarrangemang både vinter och sommar. Föreningsintresset ledde till att han blev s.k. chartermedlem i Lions Club Jeppo och han axlade ordförandeskapet för byggnadskommittén då Uf-lokalen renoverades 1974--1976. John var även länge verksam som ÖSP:s skattebokförare sedan jordbruket övergick från arealbeskattning till kvittobaserad anteckningsskyldighet/bokföring år 1968, ett arbete han skötte med plikttrohet fram till sin pensionering.

John \textdied 13.01.2000  ---  Lea \textdied 20.02.2012


%%%
% [occupant] Dödsbo
%
\jhoccupant{Dödsbo}{\jhname[Sjöblom]{Dödsbo, Sjöblom}}{1954-1957}
\jhvspace{}


%%%
% [occupant] Sjöblom
%
\jhoccupant{Sjöblom}{\jhname[Hanna]{Sjöblom, Hanna} \& \jhname[William]{Sjöblom, William}}{1917-1954}
Johanna (Hanna) Westerlund, \textborn 26.07.1883, gifte sig 05.11.1905 med Karl William Sjöblom (Dalas), \textborn 04.03.1883 på Harald i Forsby.
\begin{jhchildren}
  \item \jhperson{\jhname[Elis William]{Sjöblom, Elis William}}{26.07.1906}{}
  \item \jhperson{\jhname[Elmer Erik]{Sjöblom, Elmer Erik}}{03.08.1913}{14.12.1959}
  \item \jhperson{\jhname[Gerda Elise]{Sjöblom, Gerda Elise}}{08.05.1915}{}
  \item \jhperson{\jhname[Elvi]{Sjöblom, Elvi}}{}{}, på Åland
  \item \jhperson{\jhname[Else Valdine]{Sjöblom, Else Valdine}}{08.07.1924}{}
\end{jhchildren}

William var i Amerika 1902--04, 1907--13 och 1916--21. Paret hade den 28.6.1917 via testamente tagit över en 2/96 mtl av Kristina 	Westerlunds (änka efter landspolisen Jakob Westerlund) lägenhet av Fors skattehemman Rnr 3:8, mot en ersättning om 3.600 mk, livstids sytning och 4 övriga nedtecknade villkor. På platsen byggde de sitt bostadshus och ladugård. Senare uppförde sonen Elmer ett eget hus på hemmanet, tvärs över ån, på Holmen (karta 6, nr 128), men då han kom på obestånd, såldes det år 1953 till Paul och Marta Broo. Elmers svårigheter gjorde att hans familj med barnen Florens, Roger och Bjarne bodde hos Hanna och William i många år.

I början av 1951 hyste ladugården två hästar och tre kor. I bouppteckningen upptogs även lägenheterna Bäckstrand 4:33, Bäck 3:25, Heimhagen 4:59 och Storbackhagen 4:26 i Jungar by samt Kilen 3:36 i Överjeppo by.

Hanna \textdied 06.01.1951  --- 	William \textdied 07.02.1954



%%%
% [house] Westlund
%
\jhhouse{Westlund}{3:19}{Fors}{6}{128}


%%%
% [occupant] Jeppo Kraft
%
\jhoccupant{Jeppo Kraft}{\jhname[Alg]{Jeppo Kraft, Alg}}{2017--}
Andelslaget införskaffade lägenheten den 20 januari 2017 som en investering för framtida planer.\jhvspace{}


\jhhousepic{PBro.jpeg}{Paul och Martta Broo}

%%%
% [occupant] Broo
%
\jhoccupant{Broo}{\jhname[Paul]{Broo, Paul} \& \jhname[Martta]{Broo, Martta}}{1953-2013}
Paul Broo, \textborn 11.02.1926, gift 24.04.1949 med Martta Sofia Välimäki, \textborn 15.03.1928, köpte 12.08.1953 fastigheten av Karl William Sjöblom. Paul har arbetat bl.a. på Keppo pälsdjursfarm. Han har varit en utpräglad friluftsmänniska som ägnat mycken tid åt jakt och fiske.

Paul deltog ofta i jakt av fasaner med doktor Alftan, som hade en egen jaktstuga på åkanten vid Prästas (se Jungarå, nr 24). Som visad uppskattning för sällskapet fick Paul en vävd rya föreställande bengaliska tigrar i skogsmiljö av doktorn. - Men, sade Paul, de gånger president Kekkonen var här som Alftans jaktkompis, då rymdes jag inte med!

Martta skötte i huvudsak hemmets sysslor.
\begin{jhchildren}
  \item \jhperson{\jhname[Pirkko]{Broo, Pirkko}}{20.09.1950}{}, gift Alaharju
  \item \jhperson{\jhname[Rainer]{Broo, Rainer}}{16.02.1953}{16.01.1996}
\end{jhchildren}

Martta \textdied 26.12.1991  ---  Paul \textdied 20.03.2013


%%%
% [occupant] Sjöblom
%
\jhoccupant{Sjöblom}{\jhname[Karl William]{Sjöblom, Karl William}}{1946-1953}
Karl William byggde år 1946 tillsammans med sin son Elmer huset på sin mark på Holmen, där tomten utgjorde ca 25 ar. Sonen Elmer bodde till en början i huset, men huset beboddes senare av Oiva Rintala med familj och av Alf och Anja Ludén innan det såldes till Paul och Martta Broo år 1953.



%%%
% [house] Kilen - Jeppo Kraft
%
\jhhouse{Kilen - Jeppo Kraft}{3:128}{Fors}{6}{96, 96a-c}


\jhhousepic{038-05574.jpg}{Jeppo Kraft Alg}

%%%
% [occupant] Jeppo Kraft
%
\jhoccupant{Jeppo Kraft}{\jhname[Andelslag]{Jeppo Kraft, Andelslag}}{1983--}
Sedan 1983 är Jeppo Kraft Andelslag verksamt på Kiitolavägen 1. Andelslaget rörelse hade vuxit och fick behov av både större kontor och en lämplig butikslokal. Man kom överens om att köpa lägenheten från Oy Mirka Ab.

Eran inleddes med att man rev upp de illa medfarna golven i alla rum. Grunden innanför stenfoten fylldes med grus varpå ny betongplatta göts och monterades värmekabel i golvet. Väggarna blev förstärkta och satta i gott skick. Likaså blev vvs-installationerna sanerade och moderniserade för att tjäna den nya verksamheten i huset. På vinden tillkom mötesrum i ett senare skede, antagligen 1998.

Genom markbyte den 20.6.1984 med rågrannen Christer \& Runa Fors flyttades tillfarten till verkstaden nere vid ån. Det lilla förvaringshuset för byns handdrivna brandspruta i s.ö. hörnet av s.k. Rijäälån hade avlägsnats redan tidigare. Sista gången handsprutan kom till användning var år 1952 då Christer själv tillsammans med storebror Håkan höll på att ställa till storbrand hemma i höladan. Den nya vägdragningen tjänade båda parter. Den gamla gårdsbyggnaden, som tjänat som enkelt lager, revs bort och gav plats för en betydligt större lagerlokal, \jhbold{96a}, under plåttak och med fasader av korrugerad rödmålad plåt. En del av utrymmet har senare avdelats och hålls varm för att förvarad elektronik inte skall ta skada.

I dag arbetar disponenten jämte 2-3 personer i kontoret-butiken. Styrelsen o.a. drar nytta av de mötesresurser huset erbjuder på övre 	våningen. Andelslagets montörer huserar i huvudsak utgående från verkstaden i hus 96b, ``kraftstationen'' och magasinbyggnaden, 96c.

S.k. mjölnarstugan (nr 397a) revs 1992 på Alf Julins försorg. År 1998 avslutades det 50-åriga arrendet av den tomt varpå den stått och där Silvast kvarn ännu står kvar. Den är nu införlivad med Kvarnbacken 3:140.


---> \jhbold{Kraft}, R:nr 3:63,	Kraftstationen, \jhbold{96b}

Bygget av ``en större'' kraftstation på andelslagets nordvästra tomthörn vid älven påbörjades 1941 som en följd av svårigheter att 	säkerställa tillräcklig eldistribution, då varken Esse Elektro Kraft eller Nykarleby kraftverk lovade leverans av tilläggsström. Byggplatsen var densamma som hade använts av Böös mejeri 1913-33. P.g.a. krig och rådande materialbrist, kunde man installera den nya suggasmotorn först senhösten 1944 för att tas i bruk vid instundande årsskifte. Nu trodde man sig ha sluppit ifrån den ``sedvanliga mörkläggningen'' i samband med låga vattenflöden. Reservkraftkällan visade sig likväl vara otillräcklig samtidigt som den slukade ansenligt med bränsle, varav fanns endast begränsad tillgång. Inte förrän i slutet på 1948 kunde man ordna tillräcklig kapacitet genom sammankoppling med riksnätet via Nykarleby kraftverk.

\jhhousepic{040-05576.jpg}{``Kraftstationen'', nr 96b}

Fram till 1956 hade andelslaget saknat kontorsrum. Det kansliarbete som skulle göras, utfördes hemma hos disponenten och senare i s.k. mjölnarstugan (397a). Nämnda år inreddes kanslilokalitet i det rum där suggasmotorn stått. Motorns stora svänghjul av gjutjärn måste sprängas sönder för att kunna tas ut genom dörröppningen. Tilltaget spräckte fönster både i kraftstationen och i grannen Elis Stenvalls hus, nr 93. Kansliet blev snart för litet. År 1968 tillbyggde andelslaget därför en tidsenlig lokal med kontorsrum, skilt rum för disponenten, matsal för arbetarna och en lagerlokal för allehanda tillbehör. Förstoringen gav andrum för femton år framöver.

År 1993 införskaffade andelslaget en Wärtsilä-dieselgenerator av typ peak load att köra ner toppeffekterna med. Den finns installerad vid 	stationen. På senare år har generatorn inte använts på grund av dålig lönsamhet till följd av höga bränslepriser.

I dag används byggnaden för montörsstaben som verkstad, lagerlokal för installationsmaterial, arbetscentral, m.m. Här finns numera också den tekniska knutpunkten för öppna fibernätet NU-net i Jeppo.


---> \jhbold{Mejeriet}, R:no 3:16, 	Magasinbyggnaden, \jhbold{96c}

Huset är en låg uthuslänga i trä under tegeltak. Den torde vara byggd i samband med att mejeriet från Böös flyttade hit 1913, d.v.s. byggnaden har 100 år på nacken, se bild nedan. Jfr även texten under Matts Källman, karta 5, nr 350. Uthuset tjänar i dag som lagerutrymme. På bilden syns också gaveln av mjölnarstugan, 397a.

\jhpic[pic: boosmejeri]{Boosmejeri 1928.JPG}{Mejeriet från Böös står där nuvarande verkstad/kraftstation finns.}


%%%
% [occupant] Keppo
%
\jhoccupant{Keppo}{\jhname[Oy/Ab]{Keppo, Oy/Ab}}{1976/77-1983}
Året 1968 hade i alla avseenden inneburit ett rekordår för Mirka, som kunde uppvisa en försäljningsökning med hela 89\%. Antalet anställda ökade från 20 till 47. Exporten drog. Den goda utvecklingen fortsatte under hela 1970-talet och krävde ständiga nyrekryteringar. Oy Keppo Ab köpte fastigheten för det växande företaget Mirka för att användas som hyreslägenheter för anställda. Typen av genomgångsbostad har gjort att flera familjer och enskilda bott en viss tid här medan planerna för andra boendelösningar mognat och förverkligats. Dessa har bl.a. varit:
\begin{enumerate}
  \item Grahn Paul och Eva
  \item Nikkanen Pekka och Margit + Jani
  \item Två familjer Bäckman
\end{enumerate}
Efter de ovan nämnda har bostäderna bebotts av många andra hyresgäster. Under senare delen av den eran fick huset rykte om sig att vara tillhåll för både äventyrligt och lösaktigt leverne.


%%%
% [occupant] Wikström
%
\jhoccupant{Wikström}{\jhname[William]{Wikström, William} \& \jhname[Sigrid]{Wikström, Sigrid}}{1967-1976/77}
William Wikström, ``Peeda-Viljam'', (Grötas nr 102) köpte fastigheten sedan Mellersta Österbottens Andelskassa flyttat ut. Det är oklart vilka hans egentliga planer var med anskaffningen, men han började med att hyra ut utrymmena för boende. Hans första hyresgäster bör ha varit Leo Hinkkanen, \textborn 01.11.1944, med fru Siv, \textborn 21.03.1944, född Jungarå, vilka flyttade in i februari 1967 och bodde här till januari 1968. Under den tiden föddes dottern Regina. Sex år senare föddes dottern Sonja på nya adressen, Keppovägen 74.

I enbart cirka ett halvt års tid, från mars 1968 till hösten samma år, har ``Jeppo Blomma och Kemikalia - Jepuan Kukka ja Kemikalia'' med unga ägarinnan Sinikka Alainen, senare gift med Klas Forslund, hyrt in sig med sin nya verksamhet i s.k. butiksdelen. Företagets namn ger en tydlig vink om produktsortimentet och tillhandahållna tjänster. Etableringen blev en realitet som kom att bli kortvarig av olika orsaker.

William \textborn 11.9.1902--\textdied 24.4.1976   ---   Sigrid \textborn 17.10.1906--\textdied 19.9.1984


%%%
% [occupant] Mell.Öb:s
%
\jhoccupant{Mell.Öb:s}{\jhname[Andelskassa]{Mell.Öb:s, Andelskassa}}{1953-1967}
Mellersta Österbottens Andelskassa stod i behov av filialkontor i Jeppo och fastnade i sitt köpval för det centralt belägna nymanska huset i Silvast. Med små åtgärder kunde det inrättas för sin nya funktion.

Eliel Brunell avslutade stegvis sin butiksrörelse ``Jeppo lanthandel'' i 	Modéns hus, som han drev åren 1932-1962 invid Stationsvägen (hus nr 375). Den 01.08.1953 blev Eliel kontorsföreståndare för Andelskassan. Tillsammans med fru Ines och mellersta sonen Rolf flyttade han in i fastighetens bostad år 1961 och bodde där till 1966(-67).
Familj:
\begin{enumerate}
  \item Eliel, \textborn 11.11.1904 i Nkby lkm, Ytterjeppo, \textdied 12.03.1973
  \item Ines, \textborn	19.12.1909 	'' ''   ''    ''   , Markby/Dalabacka, \textdied 08.05.2010
  \item Göran, \textborn 04.02.1936 i Jeppo, bodde inte i huset
  \item Rolf, \textborn	07.03.1940 i Jeppo, \textdied 11.04.2011
  \item Viking, \textborn	20.12.1941  i Jeppo, bodde inte i huset
\end{enumerate}
Andelskassan verkade i huset fram till år 1967, då den flyttade in i sin nya kontorsbyggnad på Östra Jeppovägen 12 B, hus 92. Eliel, Ines och Rolf flyttade år 1968 till Nykarleby.


%%%
% [occupant] Jeppo
%
\jhoccupant{Jeppo}{\jhname[Fabriker]{Jeppo, Fabriker}}{1952-1953}
Den 11.6.1952 köptes fastigheten av Jepuan Tehtaat-Jeppo Fabriker, som stod som ägare enbart ett år. Jeppo Fabriker sålde fastigheten redan 12.6.1953. Under detta år bodde familjen Erik och Henna Söderström med barnen Lars, Ulla, Carina Jan-Erik (Erkki) och Kerstin (Keti) i huset innan den flyttade till huvudbyggnaden på Kiitola, där Erik var chef för garveriet.


%%%
% [occupant] Nyman
%
\jhoccupant{Nyman}{\jhname[Hugo]{Nyman, Hugo} \& \jhname[Flora]{Nyman, Flora}}{1934-1952}
Hugo Mathias, \textborn 03.02.1899 i Oravais, gifte sig med Flora Elisabet Ståhlberg, \textborn 22.05.1899 i Seattle. Floras familj flyttade senare till Komossa. Flora och Hugo Nyman köpte s.k. Kilen 3:51 på 9 ar den 31.05.1934 av Emil och Ida Fors. De byggde ett hus i rött tegel på tomten och bedrev kolonialvaruhandel där fram till 22.10.1948.
\begin{jhchildren}
  \item \jhperson{\jhname[Doris Marita]{Nyman, Doris Marita}}{05.06.1926}{15.07.1988}, g. Sirén, till Vasa. Dotter Monica Sirén-Aura
  \item \jhperson{\jhname[Gun Maj-Lis]{Nyman, Gun Maj-Lis}}{13.04.1929}{12.06.2012}, g. m. Paul Storbacka, Gamlakarleby
  \item \jhperson{\jhname[Frank Bernhard]{Nyman, Frank Bernhard}}{18.02.1937}{}, gift med Ritva, nu i Sverige
\end{jhchildren}
Hugo dog i ung ålder, han blev bara 39 år gammal. Flora fortsatte ändå att sköta handeln med hjälp utifrån och av sina uppväxande döttrar. Då verksamheten avslutades och huset skulle avyttras, flyttade Flora till Vasa, där hon levde resten av sitt liv, till 73 års ålder.

Hugo \textdied 21.11.1938  ---  Flora \textdied 01.11.1972. Båda är jordfästa i Jeppo.



%%%
% [house] Silvast bro
%
\jhhouse{Silvast bro}{4:}{Fors}{6}{Kiitolavägen 15-26}

\jhhousepic{Brobyggare vid Fors 1909.JPG}{Brobyggare vid Fors 1909}

%%%
% [occupant] Nykarleby
%
\jhoccupant{Nykarleby}{stad}{}
Turerna kring byggandet av järnbetongbron vid Fors respektive Mjölnars (Kiitola) beskrivs i boken ``Historik över Jeppo'' på sidan 315. År 1907 inleddes planeringsarbetet. När man väl kommit fram till i vilket material de skulle byggas, skrevs kontrakt med byggnadsbyrå V. Virkkula i Uleåborg. Emedan brokistor redan fanns på plats vid Kiitola, färdigställdes den bron år 1910 till en kostnad om 32.000 mk. Bron vid Fors behövde ett år till på sig och blev klar 1911 till ett vederlag av 42.000 mk. Båda broarna var bland de första i sitt slag i Finland.

\jhhousepic{Abron Silvast.jpg}{Silvast åbro (ev 1920-talet) på den tidigare huvudleden för busstrafiken Jakobstad---Vasa.}

Silvastbron kom att få en spännvidd på 47,5 m med en totallängd på 60 m. Bredden på vardera broarna blev 5 m, vilket vid ett tillfälle visade sig räcka till för möte mitt på brolocket, då varken Haldins busschaufför eller den lokala ``smala-Mossi''-diton vägrade stanna och invänta sin tur vid överfarten. Ända till 1970, då nya åbron vid skolan blev körbar, stod järnbetongbroarna vid Fors och Mjölnars för buss- och den s.k. tyngre trafikens logistiska möjligheter. Hundra år senare börjar de båda stiliga konstruktionerna längta efter ordentligt underhåll.




%%%
% [house] Kvarnbacken
%
\jhhouse{Kvarnbacken}{3:140}{Fors}{6}{97, 97a-e}


\jhhousepic{044-05875.jpg}{Fjalar och Mayvor Fors}

%%%
% [occupant] Fors
%
\jhoccupant{Fors}{\jhname[Fjalar]{Fors, Fjalar} \& \jhname[Mayvor]{Fors, Mayvor}}{1995--}
Fjalar Rainer, \textborn 11.04.1949 i Jeppo, agronom, gift 26.07.1975 med Mayvor Margareta Lillqvist, \textborn 10.03.1949, laborant, jordbrukardotter från Purmo. Sedan januari 1995 har paret Fors ordnat livet på denna lägenhet, Fjalars barndomshem och avstamp ut i världen, efter att först ha planerat, arrangerat, ritat, flyttat, sanerat och	byggt till den röda stugan på Kraftgränd 4 (nr 97a). Orsaken till flytten från Tottesund, Maxmo, var arbetsrelaterad.
\begin{jhchildren}
  \item \jhperson{\jhname[Joakim Gustav Andreas]{Fors, Joakim Gustav Andreas}}{16.02.1982 i Vasa}{}, dipl.ing. på Bambuser, familj i Dalby, Lund
  \item \jhperson{\jhname[Daniel Rainer Alfred]{Fors, Daniel Rainer Alfred}}{14.08.1985 i Vasa}{}, dipl.ing., processtyrning
\end{jhchildren}

Efter sin examen har Fjalar haft tjänster vid Korsholms skolor (linjeföreståndare 8 år), Finlands Pälsdjursuppfödares Förbund (Maxmo försöksfarm för pälsdjursforskning - farmchef 5 år), Lannäslunds lantbruksskolor -- senare Lannäslunds skolor, Lannäslundskolan Optima resp. Optima Lannäslund - (rektor 19 år), Projektkoordinator (Optima 4,5 år).	Pensionerad i dec. 2013. Under yrkeskarriären var Fjalar turvis både ordf. och sekr. i Österbottens Agronomer r.f., han väckte liv i och var ordf. i Svenska lantbrukslärarföreningen, styrelsemedlem och kassör i Maatalousalan oppilaitosten rehtorit r.y./ Luonnonvara-alan johtajat r.y. och startade år 2008 personalföreningen OptiPop r.f. på Lannäslund.

Mayvor har arbetat som laborant vid Schaumans fabriker och Östanlid sjukhus i Jakobstad, preparator vid Tekniska högskolan i Otnäs, laborant vid utvecklingslaboratoriet vid Dickursby färgfabriker och sakkunnig vid uppstarten av Solf Centralbutiks spannmålslaboratorium, kontorist på Maxmo Försöksfarm samt byråfunktionär vid länsmanskansliet i Vörå.

Kvarnbacken 3:140 bildades medels köp 05.08.1995 av delar av Broända 3:109 och Strand 3:130, inom vilka den tidigare Mejerivägen 3:14 ingått. Området är enhetligt då fastigheten i dagsläget omsluts av Kiitolavägen, Kraftgränd och Lappo å. På lägenheten, som omfattar ca 8500 m$^2$, finns i dag:
\begin{enumerate}
  \item en kvarnbyggnad (nr 97d, uppförd före sekelskiftet 1800-1900),
  \item ett timrat bostadshus från 1800-talet (nr 97a), som flyttats 1960 och 1994, restaurerats 1994 på ny grund på Kraftgränd 4. Sedan jan. 1995 bor Hildur Fors i detta hus.
  \item ladugårdsbyggnad (1935, nr 97e),
  \item gårdsbyggnad, garage (ca 1950, nr 97b),
  \item stock-/fritidshus (1999, nr 97c), samt
  \item bostadshus i vitt kalksandtegel (nr 97, uppfört 1961--62).
\end{enumerate}


\jhhousepic{042-05577.jpg}{Huset flyttat och restaurerat -94, står på samma plats som f.d. ``mjölnarstugan'', nr 397a}

Fram till i dag pratar gårdsfolk och andra om ``Krymboas'', ett namn härlett ur benämningen ``kronobyboas'', efter Jakob Jakobsson, som ägde hela Fors hemman 1699--1709, då man refererade till detta område (se nedan samt nr 100 och 400).

Ett landmärke för gården var flaggstången, som restes av Gustaf	Fors och Leo Julin fredag kväll den 9 oktober 1936. Det var viktigt att få upp den i tid före bröllopet mellan syster Anna Fors och Leo Julin, vilket firades söndagen den 11 oktober. - Till vigseln anlände 800	gäster, som kommit till kyrkan p.g.a. ryktet att Anna skulle bära den sällan sedda ``storkronan''. Den kom ändå inte till användning p.g.a	dess tilltagna tyngd. - Den slanka furan hade med viss möda forslats genom	krökarna från Kampasskogarna, liggande på dubbla stöttingpar. Längden var då 22 m; genom åren förkortad från ``skatan och loman''  i två	repriser till ca 18 m. Flaggstången blickade över Silvast i 76 års tid, fram till natten 17-18 september 2012, då en kraftig stormby lade den till ro över fähustaket.

\jhhousepic{Gymnastik3.jpg}{Medaljtyngda Joanna Alho, Milla Pitkäaho, Sofie Nylund och framför dem Alisa (Niemelä) Forslund och Daniela Nygård. Alisa blev år 2014 vald till Finlands mannekäng varefter hon senare också segrade i The International Modeling and Talent Association-tävlingen i New York.}

Den 13.01.1997 startade paret Fors och Stig-Olof Lillqvist, Mayvors	bror från Purmo, ``Kvarnbackens gymnastikförening r.f.'' (KGF) för att ge	egna och andras barn ordnade möjligheter till nyttig fysisk aktivitet. Då byn saknade goda utrymmen för redskapsgymnastik, tog paret itu med att vintern 1997--98 ändra den gamla	skulltorken/höladan till en värmeisolerad fungerande gymnastikhall,	``Arcas'' (nr 97e), med tillgång till alla de redskap som fordras för träning och tävling i denna disciplin.

Som mest hyste föreningen ett 50-tal aktiva barn från närregionen. Bland dem hittas flera talangfulla utövare och stolta medaljörer på ÖIDm- och FSGm-nivå. För sitt engagemang utsågs Mayvor år 2002 till ``Årets Jeppobo''. Under verksamhetssäsongen sept.-- juni var Arcas i användning 3 kvällar/vecka fram till hösten 2008. Under 2009 lades KGF i vila i väntan på yngre ledare.

Verksamheten inom tredje sektorn har också i övrigt intresserat på olika sätt. Fjalar blev medlem i Lions Club Jeppo hösten 1995 och har innehaft olika poster i den, bl.a. två perioder som president och fem dito som sekreterare. År 2003 var han med och initierade	nystart av ``Jeppo byaråd r.f.--Jepuan kyläyhdistys r.y.'', sedan hösten 2016 med nya namnet ``Jeppo byaförening r.f.'', vars ordförande han varit 2003--06 och 2009--17. I anslutning till detta fungerade han en tid som ordförande i ``Svenska Österbottens byar rf'', som 2009 sammangick med dagens ``Aktion Österbotten''.

Som en direkt koppling till nämnda post kom makarna att bli kraftigt engagerade i planeringen och förverkligandet av tjärdalsprojektet ``Tjära Mor'' år 2010. Byns talkoanda visade sig från sin bästa sida.  Satsningen blev dokumenterad i en utgåva i text- och bild. Fjalar och Mayvor fullföljde även 2013--2014 byarådets plan att anlägga en vandringsled i Jeppo. ``Trådi'' kunde invigas 1 maj 2014 och paret fick senare Nykarleby stads idrottsdiplom ``Årets talkoarbetare 2014''. -- Fjalar har medverkat till tillkomsten av denna bok.



%%%
% [house] Broända gård
%
\jhhouse{Broända gård}{3:6 + 3:15}{Fors}{6}{397 \& 97}


\jhhousepic{Broanda vid Fors 1950-t.jpeg}{Gustaf och Hildur Fors}

%%%
% [occupant] Fors
%
\jhoccupant{Fors}{\jhname[Gustaf]{Fors, Gustaf} \& \jhname[Hildur]{Fors, Hildur}}{1945-1995 (-79)}
Gustaf Alfred, \textborn 10.04.1919, gifte sig den 21.10.1945 med Hildur Anna Maria Engström, \textborn 17.10.1925, från Jeppo (Kojonen hemman, Lassila).
\begin{jhchildren}
  \item \jhperson{\jhname[Per-Håkan Gustaf]{Fors, Per-Håkan Gustaf}}{18.09.1946}{}, instrumentmekaniker, -planerare, Nautor, J:stad
  \item \jhperson{\jhbold{\jhname[Christer]{Fors, Christer}} Anders Alfred}{30.09.1947}{}, LD-tekniker, jordbrukare
  \item \jhperson{\jhbold{\jhname[Fjalar]{Fors, Fjalar}} Rainer}{11.04.1949}{}, agronom, rektor
  \item \jhperson{\jhname[Kerstin Maria]{Fors, Kerstin Maria}}{04.05.1951}{}, grafiker vid Pedlex Norsk Skoleinformasjon, Oslo
\end{jhchildren}
Gustaf och Hildur övertog 31.12.1945 en relativt modern lantbrukslägenhet (Broända gård) som Gustaf själv varit med om att bygga upp sedan tonåren, i gott samspel med sina fyra systrar, sin far Emil och mor Ida. Gustaf var intresserad av jordbruket, vilket bl.a. syns i 4H, där han år 1932 vann pokalen i sädesodling, och bekräftas ytterligare av att han gärna genomförde utbildningen i Gamla Vasa vid Korsholms lantmannaskola och tog med sig nyheter från den. Det medförde att han tidigt testade majsodling och initierade tillbyggnaden av AIV-torn i anslutning till det 10 år gamla fähuset.

Gustafs medverkan i kriget 1939--44 blev, liksom för många andra i samma ålder, en ovälkommen parentes för den drivkraft som fyllde ynglingens väsen. Äventyret höll på att få ett snabbt slut då han redan på nionde dagen vid fronten sårades när en explosiv kula träffade gevärskolven och drev splitter in i vänster hand. Att komma helskinnad från fronten var ingen självklarhet för honom eller för dem som väntade där hemma.

Lägenheten omfattade vid övertagandet ca 79 ha; trädgård 0,02 ha, åker 26,50 ha, skog 45,00 ha och övrig jord 7,48 ha. Av åkermarken var ca 7 ha täckdikad. Den relativt nya ladugården hade rum för 12 klavbundna mjölkkor, nödiga kättplatser, 3 hästar, en handfull s. k. baconsvin och några får för gårdens behov. Hönsskötseln hade avslutats tio år tidigare.

Vid arvsskiftet löstes systrarna Jenny, Gemima, Anna och Etel ut, varvid Mellanåkern, mellan mejeriet och Stenvall, samt en hustomt vid Harisloon, se nr 43, styckades från gården.

Hemmanet utvecklades i takt med tiden. Inriktningen var mjölk- och foderproduktion, vilket bl.a. medförde behov av omändringar i befintliga gårdsbyggnader. Kalluftstork för spannmål inrättades ovanför redskapslidret (kärrladan). Höladan, mjölboden och vedlidret gav plats för en skulltork för hö. För att ge rum för dubbelt fler kor än vad ursprunget var planerat för, gjordes nödvändiga omdisponeringar i fähuset. Får, svin och hästar hade utmönstrats. Mekaniseringen gjorde fortgående landvinningar, uttryckligen i och med inköpet av den första traktorn, en Ferguson (``Grålle'') år 1952.

Gustav engagerade sig under sin mest aktiva tid i organisationsliv och 	samhällsbygge. Han hann vara ordförande en period i 4H. I Jeppo Uf var han ordförande genast efter kriget, 1945. Senare satsade han mycket tid på ordförandeposten i flera andra sammanhang; Jeppo Lantmannagille, Jeppo lokalavd. av Österbottens Svenska Producentförbund, Österbottens Kötts förvaltningsråd (1962-)1973--79, Hälsovårdsnämnden i Jeppo, i Jeppo kommunalfullmäktige 1966--72, byggnadskommittén för Jeppo Kommunal- och hälsogård samt Jeppo Hembygdsförening 1981--87.

Hildur har under alla år skött hemmet och varit en uthållig medarbetare i jordbruket samt trägen, intresserad medlem i Jeppo Marthaförening. Mode, inredning, hus och hem och läsning har varit andra favoritområden.

Vid generationsskiftet 12.09.1979 till Christer och Runa Fors (se Fors nr 45 och 127), behöll Gustaf och Hildur den outbrutna delen av tomten mellan älven och ladugården på vilken bostadshuset står.

Gustaf \textdied 23.11.1992 (alzheimer)


\jhhousepic{Broanda ForsIMGP2476.jpg}{Infarten t. gården ca 1935 (gård 397). T.h. gaveln av sytningsstugan (nu gård 97a)}

%%%
% [occupant] Rijf/Fors
%
\jhoccupant{Rijf/Fors}{\jhname[Emil]{Rijf/Fors, Emil} \& \jhname[Ida]{Rijf/Fors, Ida}}{1911/-23 – 1945}
Emil Alfred Alfredsson Rijf/Fors (Elfstrand), \textborn 15.05.1883, gift 02.02.1908 med Ida Gustava Pettersdr Holm, \textborn 10.01.1886, i Jeppo. Lysning 23.11.1907 i närvaro av brudgummen och bonden Johan Eriksson Grahn. Efter giftermålet bodde makarna som inhyses 1908--1912 hos Brita Kajsa Holm och flyttade därefter till Fors. Piga på Fors var vid sekelskiftet 1800/1900 Hilma Elisabet Gustafdr. Ahola.

Emil var född i Alahärmä. Enligt uppgift var han två gånger till USA, 1902--04 och 1905. Den andra gången, i sällskap med Anders Jungarå och Johan Henriksson Grötas, reste han för att leta reda på sin far. Han hittade honom på en skogscamp i Ashland, Wisconsin, där Katarina Sandberg (gift Sikström; kallad ``Klockar-Katrin'') kunde sammanföra fadern med	sonen. De två hade arbetat en tid på samma camp utan att veta om varandra. Emil lär ha gett sin far ett kok stryk för att han lämnat familjen i sticket och beordrade honom därpå att resa hem.
\begin{jhchildren}
  \item \jhperson{\jhname[Jenny Gustava]{Fors, Jenny Gustava}}{01.03.1909 på Holm}{07.07.1989}, g. Karlsson, Sverige
  \item \jhperson{\jhname[Gemima Alfrida]{Fors, Gemima Alfrida}}{08.07.1912 på Fors}{06.06.1999}, gift Gustafsson
  \item \jhperson{\jhname[Anna Viola]{Fors, Anna Viola}}{29.10.1916 på Fors}{08.05.1967}, gift Julin
  \item \jhperson{\jhbold{\jhname[Gustaf Alfred]{Fors, Gustaf Alfred}}}{10.04.1919 på Fors}{23.11.1992}
  \item \jhperson{\jhname[Etel Karin Alice]{Fors, Etel Karin Alice}}{28.02.1924 på Fors}{22.07.2009}, gift Albäck
\end{jhchildren}
Det är oklart i vilket skede Emil ändrade sitt efternamn från Rijf, via Johansson i Amerika, till Fors, men det bör ha skett ca 1910. År 1911 köpte Emil av sin mor halva Strand skattehemman 3:15 och ärvde den andra halvan. Tolv år senare, 1923, köpte han hela Broända no 3:6 av sin mor. De hittills sambrukade lägenheterna gick under namnet \jhbold{Broända gård}. Nytt fähus för 12 kor byggdes år 1935; det står ännu kvar.

Emils yngre bror Johan (``Jukka'', ``Jan'', ``Krymboas-Janne'') var	förmodligen tilltänkt som blivande hemmansägare. Liksom många andra gjorde han sin egen emigrantresa till Amerika, där han råkade ut för ett gruvras (i Vancouver-trakten) tillsammans med en Munsala-yngling. Olyckan höll honom instängd med sin döende kamrat i tre dygn, vilket kom att prägla Johans mentala hälsa i fortsättningen. Han skickades hem, där hans mor måste kvittera ut honom på farstutrappan. Johan bodde på gården fram till sin död 1954.

Ida \textdied 08.08.1934  ---	 Emil \textdied 22.03.1955


%%%
% [occupant] Grötas/Rijf
%
\jhoccupant{Grötas/Rijf}{\jhname[Sanna Lisa]{Grötas/Rijf, Sanna Lisa} \& \jhname[Alfred]{Grötas/Rijf, Alfred}}{1904-1911/-23}
Sanna Lisa Karls(Johans-)dotter Grötas,\textborn 12.10.1857, gift med Alfred Johansson Rijf, \textborn 01.12.1856 i Alahärmä, kom till Jeppo 1887.
\begin{jhchildren}
  \item \jhperson{\jhname[Johannes Alfreds. Rijf]{Rijf, Johannes Alfreds.}}{08.12.1881}{} i Markkula, Alahärmä
  \item \jhperson{\jhbold{\jhname[Emil Alfreds. Rijf/Fors]{Rijf/Fors, Emil Alfreds.}}}{15.04.1883}{} i Alahärmä
  \item \jhperson{\jhname[Hilda Maria Alfredsdr. Rijf]{Rijf, Hilda Maria Alfredsdr.}}{25.03.1885 i Sverige}{}, g. Moisio
  \item \jhperson{\jhname[Johan Valfrid Alfreds. Rijf]{Rijf, Johan Valfrid Alfreds.}}{10.06.1886 i Sverige}{29.11.1954}
\end{jhchildren}
Sanna Lisa var uppvuxen på Fors hemman. Giftermålet förde henne till Markkula i Alahärmä, där svärfadern Johan Rijf, ingift från Hirvlax, hade ett litet torp. Torpet förslog inte att försörja familjen, så han drygade ut utkomsten med sin goda händighet som bettsmed.

Sanna Lisa och Alfred flyttade snart efter Emils födelse till Sverige, där de förtjänade sitt levebröd i några års tid. Då  beskedet från Jeppo kom, att hemmanet på Fors överförts på Sanna Lisa den 22.8.1904, styrdes flyttlasset tillbaka till Finland. Lägenheten på 7/96 mantal av Fors skattehemman (Broända) var en gåva av Sanna Lisas föräldrar, därför att hennes bror, kallad \jhname[``Stöipas-Erik'']{Stöipas-Erik} (se också karta 5, nr 376), inte var betrodd arvtagare p.g.a. sitt missbruk av alkohol.

Maken Alfred, en hetlevrad person, lämnade år 1902 sin familj för ett mera äventyrligt liv i Amerika. Han återkom, men drog ånyo iväg i	februari 1905, denna gång med dottern Hilda i sitt följe. Tilltaget	verkade inte vara sanktionerat till alla delar och vistelsen blev heller	inte långvarig, för sonen Emil reste efter, fann honom och såg till att	han kom hem igen för att ta sitt ansvar för familjen och hemmanet.


\jhpic{BroandaForsGasthausMGP2473.jpg}{Gården (nr 397) tjänade som gästgiveri ca 1891-1928. Den bistod med 	ungefär 200 skjutsar per år. En gäst har sänt sitt tackkort: ``Til minde av ingeniør Gustav Keugl''. På bilden ses Johan Valfrid Alfredsson Rijf, okänd NN, Gustav Keugl, Sanna Lisa Karls-(Johans)dotter Rijf, Jenny Fors 3 år, smeden/gjutaren Karl Johan Abrahamsson Rijf, Ida Gustava Fors. År 1912.}

Sanna Lisa \textdied 22.02.1927  ---  Alfred \textdied 23.09.1925



%%%
% [house] Broända och Strand
%
\jhhouse{Broända och Strand}{3:6 och 3:15}{Fors}{6}{397 \& 397a}

Broända gård blir summan av lägenheterna Broända och Strand.

%%%
% [occupant] Fors
%
\jhoccupant{Fors}{\jhname[Karl Johan]{Fors, Karl Johan} \& \jhname[Greta Lisa]{Fors, Greta Lisa}}{1877/94-1904}
Karl Johan Abrahamsson, \textborn 17.07.1834 (i Grötas, karta 11, nr 127), gift 1857 med Greta Lisa Johansdotter Sundlin, \textborn 09.09.1833 i Soklot på Mannfors.
\begin{jhchildren}
  \item \jhperson{\jhbold{\jhname[Sanna Lisa Karlsdotter Grötas]{Fors, Sanna Lisa Karlsdotter Grötas}}}{12.10.1857}{} i Grötas
  \item \jhperson{\jhname[Karl Johannes Karlsson Fors]{Fors, Karl Johannes Karlsson Fors}}{19.04.1860}{21.06.1867}
  \item \jhperson{\jhname[Erik Karlsson Fors]{Fors, Erik Karlsson Fors}}{19.02.1863}{24.03.1933}, gick under namnet Stöipas-Erik
  \item \jhperson{\jhname[Anna Sofia Karlsdotter Grötas]{Fors, Anna Sofia Karlsdotter Grötas}}{07.08.1868}{10.08.1868}
  \item \jhperson{\jhname[Gustav]{Fors, Gustav}}{17.04.1870}{}
  \item \jhperson{\jhname[Anna Lovisa Karlsdotter Grötas]{Fors, Anna Lovisa Karlsdotter Grötas}}{08.07.1874}{}, g.m. \jhname[Anders Forsblom]{Forsblom, Anders}
\end{jhchildren}

Karl Johan A. var en utomordentligt skicklig gjutare med gott rykte. I hans produktsortiment ingick bjällror, klockor, m.m. Han var även rutinerad, god smed. Sonen Stöipas-Erik lärde sig smedens yrke. Han bodde nära den plats där Johan Modén, och efter honom Eliel Brunell, senare bedrev butiksverksamhet (se nr 75-76, 375 och 376). Erik skulle egentligen ha fått hemmanet, men på grund av hans problem med alkoholen gav föräldrarna lägenheten på Fors till dottern Sanna Lisa, gift med Alfred Rijf.

Karl Johan innehade lägenheten Broända från år 1877. \jhbold{Han blev m.a.o. den person som förenade lägenheterna Strand och Broända}, varefter hemmanet hållits i samma släkte.

Karl Johan \textdied 28.02.1918 (på Fors)  ---  Greta Lisa \textdied 04.03.1900 (på Fors)



%%%
% [house] Broända
%
\jhhouse{Broända}{3:6}{Fors}{6}{397}


%%%
% [occupant] Forss
%
\jhoccupant{Forss}{\jhname[Matts Nilsson]{Forss, Matts Nilsson}}{1865-1877}
Matts, \textborn 03.04.1835 i Överjeppo, var en av arrendatorerna till kvarnplatsen. Han blev måg i huset via giftermål med Caisa Thomasdr., \textborn 04.02.1838, som är dotter till Thomas Johansson Forss (se nedan).
\begin{jhchildren}
  \item \jhperson{\jhname[Mathias]{Forss, Mathias}}{19.12.1861}{}
  \item \jhperson{\jhname[Anders]{Forss, Anders}}{14.03.1864}{}
  \item \jhperson{\jhname[Johannes]{Forss, Johannes}}{20.07.1866}{}
  \item \jhperson{\jhname[Jakob]{Forss, Jakob}}{18.01.1869}{}
\end{jhchildren}


%%%
% [occupant] Forss
%
\jhoccupant{Forss}{\jhname[Thomas Johansson]{Forss, Thomas Johansson}}{1838-1865}
Thomas, \textborn 13.06.1810 i Jungar by, gift med Anna Andersdr., \textborn 05.09.1811 i Jungar by.
\begin{jhchildren}
  \item \jhperson{\jhname[Johan Thomasson]{Forss, Johan Thomasson}}{06.10.1832}{}
  \item \jhperson{\jhname[Maria Thomasdr.]{Forss, Maria Thomasdr.}}{10.06.1834}{}
  \item \jhperson{\jhbold{\jhname[Caisa]{Forss, Caisa}}}{04.02.1838}{}
  \item \jhperson{\jhname[Elias]{Forss, Elias}}{09.01.1841}{}
  \item \jhperson{\jhname[Anders]{Forss, Anders}}{08.05.1845}{}
  \item \jhperson{\jhname[Simon]{Forss, Simon}}{04.08.1848}{}
  \item \jhperson{\jhname[Erik]{Forss, Erik}}{25.12.1850}{}
  \item \jhperson{\jhname[Thomas Thomasson]{Forss, Thomas Thomasson}}{05.02.1853}{}, se Strand 1886--1886
\end{jhchildren}


%%%
% [occupant] Forss
%
\jhoccupant{Forss}{\jhname[Johan Thomasson]{Forss, Johan Thomasson}}{1826-1838}
Johan, \textborn 12.10.1784 i Härmä, hustru Maria Eriksdr., \textborn 10.01.1790 i Wexala. Jfr innehavarna Johan och Maria med Fors, nr 400.

Barn: \jhperson{Maja Lisa Johansdr.}{25.09.1827}{}, flyttade till Överjeppo.

Hustrun Maria \textdied 25.03.1834 --- änklingen Johan \textdied 04.09.1864.

Som änkling hade han bl.a. hjälp av drängen Johan Simonsson Haapaluhta, \textborn 30.11.1827 med hustru Lisa Gabrielsdr., \textborn 05.03.1822. Parets son Johannes föddes \textborn 15.09.1864. Efter husbonden Johans död 1864 flyttade drängfamiljen till Oravais.



%%%
% [house] Strand
%
\jhhouse{Strand}{3:15}{Fors}{6}{397a}


%%%
% [occupant] Arrendator
%
\jhoccupant{Arrendator}{\jhname[Jeppo Kraft]{Arrendator, Jeppo Kraft}}{1872-1998}

\jhbold{Jeppo Kraft Alg och dess föregångare} hade flera olika byggnader och aktiviteter på denna lägenhet. Hänvisas till kapitlet ``Silvast kvarn'', som belyser verksamheten på området.


\jhbold{Jeppo Kraft Alg}, 1948 – 1992/-98

Andelslaget använde huset som kontor fram till 1956 och senare även som paus-/matrum till 1965 och därefter lager för allehanda elmaterial. Alf Julin skötte om att huset revs år 1992.


\jhbold{Ekström Emil \& Hilda}, 1936-1948

Emil Ekström var disponent på Jeppo Kvarn och Sågverksandelslag 1920-1948 (se Silvast kvarn) men flyttade in i huset först 1936 då han sålt sitt hus på Ruotsala (se Ruotsala, karta 13, nr 13).


%%%
% [occupant] Rijf/Fors
%
\jhoccupant{Rijf/Fors}{\jhname[Emil]{Rijf/Fors, Emil}}{1911-1945}
Emil sambrukade lägenheten Strand med lägenheten Broända enligt tidigare inslagen linje. Morfar Karl Johan bodde sannolikt i detta hus till 1918 så som nedan nämnda sytningsbrev stipulerade.\jhvspace{}


%%%
% [occupant] Rijf
%
\jhoccupant{Rijf}{\jhname[Sanna Lisa]{Rijf, Sanna Lisa}}{1904-1911}
Sanna Lisa skötte lägenheten tillsammans med Alfred fram till 1911,	då hon sålde hälften till sonen Emil och överlät andra halvan i arv.\jhvspace{}


%%%
% [occupant] Fors
%
\jhoccupant{Fors}{\jhname[Karl Johan]{Fors, Karl Johan}}{1894-1904}
Karl Johan Abrahamsson \& Greta Lisa Johansdotter (se ovan) köpte lägenheten Strand på 7/96 mantal av Fors skattehemman 3 och kom således att föra samman Strand och Broända lägenheter. År 1904 skrev Karl-Johan över lägenheten på dottern Sanna Lisa mot sytning fram till sin död, 1918.


%%%
% [occupant] Heikkilä
%
\jhoccupant{Heikkilä}{\jhname[Erik Johansson]{Heikkilä, Erik Johansson}}{1894-1894}
Erik, från Alahärmä, var i borgen för föregående innehavare Mats Männikkö. Mats rymde från sitt hemman en natt och sålunda kom Heikkilä att få det, men sålde det vidare samma år.\jhvspace{}


%%%
% [occupant] Männikkö
%
\jhoccupant{Männikkö}{\jhname[Matts]{Männikkö, Matts}}{1887-1894}
Matts, \textborn 30.04.1835 i Alahärmä, och h.h. Valborg Eriksdr., \textborn 24.06.1844 i Alahärmä, köpte lägenheten av folkskolläraren Anders Lundén år 1887.
\begin{jhchildren}
  \item \jhperson{\jhname[Susanna]{Männikkö, Susanna}}{02.10.1870}{}
  \item \jhperson{\jhname[Matts]{Männikkö, Matts}}{06.02.1875}{}
  \item \jhperson{\jhname[Valborg Vilhelmina]{Männikkö, Valborg Vilhelmina}}{07.02.1880}{}
  \item \jhperson{\jhname[Gustaf]{Männikkö, Gustaf}}{25.03.1882}{}
  \item \jhperson{\jhname[Johan Emil]{Männikkö, Johan Emil}}{24.07.1886}{}
\end{jhchildren}
Matts försvann plötsligt med familjen, oklart precis när, men det noteras att Valborg flyttade till Helsingfors den 04.06.1894. Inga uppgifter om när huset, s.k. mjölnarstugan, byggdes finns att tillgå. Förmodligen var det en 1800-talsbyggnad. Den revs år 1992 av Alf Julin.


%%%
% [occupant] Lundén
%
\jhoccupant{Lundén}{\jhname[Anders]{Lundén, Anders}}{1886-1887}
\jhname[Anders]{Lundén, Anders}, \textborn 15.05.1853 i Nedervetil, gift med Emilia Fredrika Häggman, \textborn 02.03.1861 i Gamlakarleby. Han kom som lärare till Jeppo privata folkskola 10.11.1882 och stannade till sin död, \textdied 17.07.1887. Familjen, till vilken hörde sonen Anders Joel, \textborn 17.08.1883 i Jeppo, bodde på skolan. Kort efter Anders' död flyttade änkan med son till Gamlakarleby den 22.08.1887.

Under år 1887 övergick skolan, trots rätt stort motstånd, i kommunal ägo och endel lärare begärde i den processen en lönejustering för att stanna kvar; Anders var en av dem.


%%%
% [occupant] Fors
%
\jhoccupant{Fors}{\jhname[Thomas Thomasson]{Fors, Thomas Thomasson}}{1886-1886}
Thomas, sannolikt \textborn 05.02.1853 och son till Thomas och Anna Forss (se Broända). Han är antecknad som torpare och inhyseshjon på Broända hemman i 1880 års mantalslängd. Om han har varit kortvarig ägare till Strand är oklart.\jhvspace{}


%%%
% [occupant] Lassila
%
\jhoccupant{Lassila}{\jhname[Jakob Jakobsson]{Lassila, Jakob Jakobsson}}{1877-1886}
Jakob, \textborn 26.07.1835, gifte sig med Kajsa Isaksdr., \textborn 08.02.1844. Jakob var mjölnare på Silfvast kvarn (se 97d). Han flyttade till Finskas kvarn år 1887.
\begin{jhchildren}
  \item \jhperson{\jhname[Johan Jakob]{Lassila, Johan Jakob}}{20.01.1867}{}, blev mjölnare på Keppo
  \item \jhperson{\jhname[Anna Lovisa]{Lassila, Anna Lovisa}}{08.10.1871}{}
  \item \jhperson{\jhname[Maria]{Lassila, Maria}}{24.03.1878}{}
  \item \jhperson{\jhname[Ida]{Lassila, Ida}}{10.02.1882}{}
  \item \jhperson{\jhname[Henrik]{Lassila, Henrik}}{04.11.1886}{}
  \item \jhperson{\jhname[Amanda]{Lassila, Amanda}}{12.08.1888}{}
\end{jhchildren}



%%%
% [subsection] Fors hemman - Lillsilfvast 1 mtl
%
\jhsubsection{Fors hemman - Lillsilfvast 1 mtl}
Lillsilfvast får så småningom namnet Fors efter en utveckling som kan spåras från Matts Nilsson 1592 via Hans \& Susanna Jakobsson 1719, ett hopp över Stora ofreden 1713--21 och andra ofärdsår fram till nyare tider. Se inledningen på kapitlet Silvast och Fors, hemman Nr 4 och 3!



%%%
% [house] Silvast Kvarn - Kvarnbacken
%
\jhhouse{Silvast Kvarn - Kvarnbacken}{3:140}{Fors}{6}{97d}

\jhhousepic{043-05891.jpg}{Ett fortsatt blickfång med anor från 1873. Plåttak modell 2013 bidrar till ytterligare livstid.}

%%%
% [occupant] Fors
%
\jhoccupant{Fors}{\jhname[Fjalar]{Fors, Fjalar} \& \jhname[Mayvor]{Fors, Mayvor}}{1998 --}
Vårt lands forsrika älvar lärde människorna tidigt att utnyttja den kraft som vattnets rörlighet och tyngd levererar. Genom att anlägga vattenverk gick det att övergå från handkvarnar till vattendrivna kvarnar och från tjärbränning till mera lönsam sågverksamhet och försäljning av virke till skeppsbyggnad.

En färd längs Lappo å/Nykarleby älv i Jeppo bjuder på flera intressanta upplevelser, oberoende av om man rör sig utmed vägarna, stränderna eller på älven. Att det strömmande vattnet en gång i tiden valde att dela sig i två fåror genast innanför Jeppos gräns mot Härmä, för att åter rinna ihop sju kilometer längre nedströms, har genom tiderna skapat s.a.s. dubbelt goda förutsättningar för ortsbefolkningen att utnyttja dess oroliga styrka och givmildhet.

Kvarnverksamheten har av naturliga orsaker en lång historia i Jeppo, liksom i kringliggande bygder. På 1550-talet omnämns den första kvarnen i Jeppo, då kronan började uppbära skatt eller kvarntull för kvarnarna. År 1572 hittas t.ex. \jhname[Madz Madzon]{Madzon, Madz} på Sylvastholmen i kvarntullslängden. Skrivningen antyder att kvarnen fanns på västra sidan av älven. I Pedersöre storsockens historia får man veta att år 1559 uppbars 2-6 öre för varje kvarn av sex bönder i Jeppo. Bara fem år senare betalade 2 stora och 7 mindre kvarnar i Jeppo kvarntull.

Under de århundraden som gått, har ett otal kvarnar blivit byggda i ortens älvfåror eller straxt intill och efter en relativt kort tid försvunnit, förstörda av islossning eller ödelagda av brand, för att snart byggas upp igen på ungefär samma plats. Under livligaste och senare hälften av 1800-talet verkade 17 kvarnar i Jeppo (notering: Gustaf Fors). På en sockenkarta från 1846 finns en kvarn inritad på den plats där \jhname[Daniel Fors]{Fors, Daniel} senare, 2015 och 2017, hittat två kvarnstenar o.a. tecken på gammal kvarnverksamhet vid västra stranden i Silvast fors. Var det här Madz Madzon hade hållit till? Från medlet av 1850-talet vet vi om att kvarnar funnits vid Keppo, Back-Måtar, Lavast, Kiitola, Finskas, Jungarå, Tollikko, Silta-bäcken, Palo-forsen, Jungar, Grötas, Silvast, Romars-bäcken, Heikfolk-bäcken, Bärs (ångkvarn) och Lussi (väderkvarn). Av dessa är Silvast kvarn den enda som ännu står kvar -- tyst och stilla, drömmande om tider som flytt -- som ett påtagligt vittne om hur vattenkraften på olika sätt tjänat bygden.

Vattensågarna inträdde på arenan på 1730-talet. På 1800-talet räknas de till 6 st i Jeppo (notering: Gustaf Fors). Deras betydelse avtog i och med att ångkraften blev allmän på 1870-talet. Ortens egen ångsåg, den 7:e av sågar, (Fors, nr 406) är exempel på detta.


KVARNPLATSEN

Den 1.12.1873 gav guvernören över Vasa län (Nikolaistads län 1855-1917), Carl Gustaf Fabian Wrede, rätt att uppföra en tullmjölkvarn med två par stenar på lägenheten Strand skattehemman no 3:15, strax invid södra rågränsen mot Broända no 3:6. Tydliga lämningar, se ovan, visar att en kvarn funnits även på västra sidan av älven under någon tid, några tiotal meter nedströms nuvarande kvarn.

Enligt en egendomsbeskrivning, gjord av \jhname[Gustaf Fors]{Fors, Gustaf}, tecknades det första arrendekontraktet för kvarnplatsen den 14.12.1872. Arrendatorerna enl. gamla handlingar hos G.F. har varit:
\begin{enumerate}
  \item 14.12.1872 till 49 års tid: Gustav Danielsson Bärs och backstugukarlen Jakob Jakobsson Lassila (senare Finskas).
  \item 27.04.1886: Thomas Romar och Daniel Lövbacka
  \item 29.01.1887: Daniel Eriksson Knock m.fl. inropar av avlidne lanthandlaren Karl Östman
  \item 1889: Isak Karlsson Lavast köper av handelsföreståndaren Samuel Perämäki dennes del i kvarnstället. Rätten till fri malning för Fors hemman återkom 1890.
  \item 26.09.1891: Matts Johansson Fors har velat ensam tillskansa sig äganderätten till kvarnstället vid Silvast-forsen.
\end{enumerate}
Man kan med fog säga att platsen under drygt 140 år gjort god samhällsnytta. Men inte alltid utan gnissel. Den har varit begärligt föremål för många olika planer, verksamheter, åsiktsyttringar, stridande viljor och växlande ägointressen. Kvarnen ägdes år 1888 av Samuel Perämäki. Han sålde den följande år till Isak Lavast, som åter bildade bolag år 1890.


VERKSAMHETEN VÄXER

Vid sekelskiftet byttes vattenhjulet mot två turbiner, då kvarnägarna Johan Back och Anders Montin samtidigt fick rätt att till anläggningen foga en såg och träullsfabrik. Sågen uppfördes år 1902 och träullsfabriken år 1904. Den senare byggdes strax norr om sågen, men var i verksamhet endast några år.

År \jhbold{1915 installerades en elektrisk generator}, som gav den närmaste omgivningen efterlängtad elbelysning - en modernitet starkt på kommande. Under år 1920 hade både kvarnen och dammen genomgått en grundlig reparation. Den tidigare byggnaden av stock ersattes med en ny på resvirke. Nedre utloppsrännan fördjupades och dammen förstärktes med kilstenar. Anläggningen Silvast kvarn- och sågbolag bör alltså ha varit i gott skick då nästa fas i dess historia inleddes.

Jeppo kvarn- och sågverksandelslag, som bildades 5 april 1920,  köpte in Silvast kvarn- och sågbolag medels godkännande vid extra stämma den 3 april år 1921. Säljare var Johannes (Johan) Forss, Johan Jungar och Edvard Källman. I övrigt får man mycket lite till livs från de första årens protokoll. Man kan anta att det var nu som frimalningsrätten, som var en del av köpesumman, tillföll säljarna.

Trots att andelslagets stadgar inte direkt nämner elektricitetsverk, torde dess ledning ha förstått det stora behovet och betydelsen för Jeppo. En utbyggnad inleddes och den 15 juli 1921 kan man läsa i Österbottniska Posten: ``Elektrifieringen av Jeppo fortskrider med rask takt. Sedan ett andelslag, som bildats inom kommunen, övertagit Silvast kvarn och såg, har A/B Agros ombetrotts utföra elektricitetsverkets ombyggnad till högspänningssystem. Arbetet härmed har fortskridit så långt, att de höga stolparna i snörräta linjer redan är uppsatta ända till Lussi norrut och cirka 5 km söderut från kraftkällan. Med all sannolikhet kommer hela Jeppo kommun instundande höst att erhålla såväl elektricitetsbelysning som elektricitetskraft för jordbruksändamål.''  Elförbrukningen under de tre första decennierna steg snabbare än utbyggnadstakten, vilket föranledde en notis i ortstidningen den 29.10.1934: ``Det är stor skillnad på affär och affärsmän. Belysningsandelslaget går i åratal och grubblar om inköp av en 40 hästars petroleummotor, medan privata personer för en spottstyver inköpa ortens största vattenkraftanläggning (Kiitola), samt slå an i samma bransch.''

Först när ledningsnätet förenades med Nykarleby kraftverks högspänningsnät har elektricitetsbehovet i Jeppo blivit helt tillfredsställt.

\jhpic{Silvast kvarn ca 1911.JPG}{År 1911 är huskropparna vid kvarnen många. Silvast bro är nybyggd. I fonden ses ångsågens skorsten.}

Den gamla sågen hölls i funktion t.o.m. 1953. År 1954 flyttades den till nuvarande plats närmare järnvägen, till en tomt inköpt av Jeppo kommun 29.12.1952. Åtgärden väckte ingående diskussioner och stark opposition under hela byggnadsskedet, se karta 7, nr 109. I det följande behandlas inte denna utveckling, utan i huvudsak kvarnverksamheten, som kom att fortgå i ytterligare nästan ett kvarts sekel.


KVARNEN MAL OCH MAL

År 1928 köpte man en skalmaskin från Malmö kvarnstensfabrik och året därpå insattes ett aspirationsskåp (för dammfiltrering) av svensk tillverkning. I februari 1930 godkände en extra stämma styrelsens förslag om inköp av nya turbiner med regulatorer för att få bättre kraft vid kvarnen och därmed en förbättrad ström för belysning. Beslutet innebar att man tackade nej till inköp av hjälpström från Kiitola kraftverk med de villkor som ställdes. För att tillmötesgå kundernas önskemål inköpte man en grynvalsmaskin år 1934.

I syfte att bättre kunna betjäna sina kunder verkställdes år 1953 en grundligare reparation av turbinerna liksom andra omändringar i kvarnen. Den \jhbold{första elektriska motorn} installerades för att driva ett stenpar. På oktoberstämman samma år uttrycktes tillfredsställelse över de gjorda förbättringarna. Redan följande år drevs två stenpar på samma sätt och snart även det tredje.

År 1962 brast transmissionsaxeln pga sättningar i grunden till följd av att vatten i samband med vårfloden började söka sig igenom på landsidan av turbinrännan. Grunden förbättrades och generatorn flyttades till västra sidan av kvarnen. Sättningen i nordöstra hörnet kvarstår ännu i dag.

Lönsamheten för kvarnen var beroende av vattenståndet i älven och tillgången på mäld, som i sin tur grundade sig på skördeårets resultat. Därtill spelade konkurrensen från andra kvarnar en viss roll för årsresultatet. På grund av vattenbrist stod kvarnen stilla 15.11.1941--9.4.1942, vilket måste ha inverkat menligt på lönsamheten. Också den stigande elförbrukningen på 1930- och 1940-talen påverkade kvarnrörelsen oförmånligt, eftersom all vattenkraft gick åt att producera elström.

Enligt räkenskaperna för 1935 steg kvarnens bruttoinkomst till 118.698 mk för att två år senare ligga på 43.846 mk. Som orsak till nedgången anges även två nya kvarnar i grannkommunerna och en nybyggd kvarn på Kiitola. År 1937 anslöt sig andelslaget som medlem i Österbottens norra kvarnägareförening, som gjorde upp enhetliga frakttariffer för ett 20-tal medlemskvarnar. Ifrågavarande år erlades frakt i pengar enligt följande:

\begin{center}
  \begin{tabular}{l r}
    \hline
    Råg- och kreatursmjöl & 7  p/kg \\
    Sammanmalning av korn	&	10 '' \\
    Korn, skrätt &	15 '' \\
    Råg, skrätt	&	10 '' \\
    \hline
  \end{tabular}
\end{center}
Som en följd av försämringar i penningvärdet, steg mäldtaxan oupphörligt och fastställdes år 1948 till följande belopp, vilka var något lägre än de av Öb:s norra kvarnägareförening föreslagna:
\begin{center}
  \begin{tabular}{l r}
    \hline
    Råg & 1 mk/kg \\
    Råg, putsad & 1,5 '' \\
    Fodermjöl & 1,0 '' \\
    Korn, skrätt & 2,0 '' \\
    Havre- och korngryn & 3,0 \\
    \hline
  \end{tabular}
\end{center}

Mängden spannmål, som kvarnen förmalat under året 1.7. – 30.6, uppgick för nedanstående redovisningsperioder till:
\begin{center}
  \begin{tabular}{l l r}
    \hline
    Redov.period & Mängd & Ökning/minskning \\
    1947-48	&	454.889 kg & + 25,0\% \\
    1948-49	&	670.393 '' & + 15,0\% \\
    1949-50	&	742.353 '' & + 10,7\% \\
    1950-51	&	659.928 '' & - 11,0\% \\
    1951-52	&	624.533 '' & -  5,3\% \\
    1952-53	&	599.415 '' & -  5,3\% \\
    1953-54	&	925.894 '' & + 54,4\% \\
    \hline
  \end{tabular}
\end{center}
Mängden låg på sistnämnda nivå ännu i ett par års tid innan den började dala till följd av att nya privata (traktordrivna) kvarnar blev allt mera allmänna ute på gårdarna. Under årens lopp har kvarnens stenar malt spannmål åt ortens och grannkommunernas bönder. Kvarnkammaren fyllde en viktig terapeutisk funktion efter kriget för dem som behövde bearbeta sina svåra upplevelser. Utöver detta har samhällsförbättrande diskussioner, gärdsgårdskunskap, skvaller, skryt och socialt umgänge satt sig stadigt i kvarnens väggar, bakom vars oskyldiga panel man kunde finna buteljer för en medhavd styrketår att döda tiden med medan man väntade på sin mäld.

År 1977 genomfördes den sista malningen vid Silvast kvarn. Förmodligen var det ``nostlgimalning'', för samlingen mäldkvitton hade sinat totalt redan två år tidigare.


FRIMALNINGEN

Frågan om frimalningsrätten, som uppstod i samband med ägarbytet 1920--21 vid andelslagets kvarn, har lett till oenighet med vederbörande ägare, då också obehöriga verkar att under årens lopp ha tillskansat sig fördelen. Från 1 oktober 1948 sade styrelsen upp frimalningsrätten för dem som inte kunde förete papper på rätten till frimalning. Åtgärden resulterade i rättegång, som slutade vid hovrätten. Denna fastställde häradsrättens för andelslaget negativa utslag i ärendet. Frimalningen förblev en vagel i kvarnägarens öga under återstående kvarnverksamhet.


MJÖLNARE OCH DISPONENTER

Genom åren har många olika personer skött hanteringen av kvarntekniken och den praktiska delen av malningen. Räknat från starten av Jeppo kvarn- och sågverksandelslag år 1920 har dessa varit:
\begin{enumerate}
  \item 1920-1939		Johan Vesterlund
  \item 1939-1943		Edvard Källman, även sågställare
  \item 1943-1948		Valdemar Strengell
  \item 1948-1972		Karl Svanbäck
  \item 19xx-19yy   William Almberg
  \item 1972-1977		Sven-Olof Forsbacka
\end{enumerate}
Också andra tillfälliga mjölnare har varit anställda under olika tider.

Disponenter:
\begin{enumerate}
  \item 1920-1948		Emil Ekström
  \item 1948-1952		Evert Brandt
  \item 1952-1996		Alf Julin
  \item 1996-1998	  Kurt Stenvall
\end{enumerate}


DAMMBYGGNADEN

Påfrestningarna från vatten och is har under årens lopp gått hårt åt konstruktionerna och fordrat att man genomfört reparationer av olika slag. År 1927 byggde man om kilnäbben och dessförinnan hade man åtgärdat själva dammen med kilstenar från Palo-forsen. År 1929 hade man undersökt möjligheterna att höja dammen med en halv meter, men kunde inte verkställa planen pga oklara skiftesförhållanden bland dem som berördes. Åtgärden kom därmed att förfalla denna gång.

Först i och med länsstyrelsens utslag den 4 december 1950 fick andelslaget rätt att dämma upp vattnet i älven med en halv meter högre än tidigare. Förhöjningen fordrade en genomgripande ombyggnad av dammen för att den skulle klara alla påfrestningar som en högre konstruktion utsätts för. Kilstenar från prästgårdens gamla fähus inropades på auktion för 7.000 mk. Transporten till platsen kostade 25.000 mk, sedan en talkokörning inte gav önskat resultat. Dammbygget utfördes sommaren 1955 under planering och övervakning av byggmästare August Österholm från Pedersöre.

I samband med att den nya vägdragningen till Pensala-Oravais blev aktuell och en ny vägbro skulle byggas över älven vid Jeppo skola år 1970,  framkom att planerarna inte i tillräcklig grad tagit hänsyn till högvatten vid vårflöden och isproppar. För att eliminera risken att is skulle nå undre sidan av den nya bron och åstadkomma förstörelse, tvingades andelslaget att spränga ner dammen till den nivå den hade före 1955!


KVARNTOMTEN

Frågan om arrenderätten till andelslagets kvarn- och sågtomt har åstadkommit en del tvister inför häradsrätten. Ett nytt arrendekontrakt för tomten tillfredsställde inte ledningen för andelslaget, som med en motivering om hur stor allmän betydelse andelslaget har, beslöt att hos vederbörande myndighet anhålla om expropriation av ifrågavarande område. Gustav Fors, som en kort tid innan hade övertagit Broända gård, höll fast vid sin äganderätt och anlitade advokat Christer Boucht till sin hjälp. Schismen slutade med en förlikning den 25 september 1948, som utmynnade i ett för båda parter godtagbart nytt arrendekontrakt på 50 år.

Malningsverksamheten vid kvarnen hade sedan länge (1977) upphört då kontraktet löpte ut hösten 1998. Utrymmet hade då i 20 års tid använts som lager för Jeppo Krafts olika tillbehör för elinstallationer. Kvarnens ursprungliga syfte var satt på undantag och förlängt arrende var inte aktuellt. Nämnda år övergick byggnaden därmed till den utbrutna och år 1996 nybildade lägenheten Kvarnbacken 3:140, som i dag ägs av Fjalar och Mayvor Fors.

SILVAST KVARN; några sammanfattade nerslag i en betydande historia
\begin{center}
  \begin{tabular}{l p{0.8\textwidth}}
    \hline
    1.12.1873: &	Guvernören över Vasa län ger rätt att uppföra tullmjölkvarn med 2 par	stenar. \\
    Sekelskiftet: &	Byte av vattenhjul mot 2 turbiner. Rätt till såg (1902) och träullsfabrik (1904). \\
    1915: &	Första generatorn installeras \= 25 hkr (\~15 kVA) dynamomaskin. \\
    1920: &	Kvarnbyggnaden i stock ersattes med resvirkeskonstruktion och dammen förbättras samt ``juoåpan'' (nedre utloppsrännan) fördjupas. \\
    1921: & Dynamomaskinen byts mot asynkronlindad generator, en 50 kVA-A/B Agros. Kvarnen har nu 3 stenpar och 2 skrädverk \\
    1928: &	Skalmaskin skaffas fr. Malmö kvarnstensfabrik. \\
    1929: &	Aspirationsskåp installeras (aspirator \= skapar vacuum). \\
    1934: &	Grynvalsmaskin inköps. \\
    1948:	&	Arrendekontrakt på tomten för 50 år uppgörs med markägaren Gustaf Fors. \\
    1953: &	Turbinreparationer, omändringar i kvarnen. Den första elmotorn installeras för att driva ett stenpar. Sista året för sågning på denna plats. \\
    1955: &	Dammen höjs med ½ m (kilstenar från prästgårdens fähus). \\
    1962: &	Översvämning med åtföljande grundsänkning på landsidan av turbinrännan, vilket ledde till axelbrott. Generatorn måste flyttas. \\
    1977: &	Sista mäldkvittona skrivs ut (förmodligen s.k. nostalgimalning). \\
    1978-98: & Kvarnen används enbart som lagerlokal för Jeppo Kraft Andelslag. \\
    1998:	&	Arrendekontraktet utgår. Kvarnen införlivas med Kvarnbacken 3:140, ägare Fjalar \& Mayvor Fors. Ägarna försöker med små resurser årligen utföra smärre reparationer och underhåll. \\
    2013:	&	17.7 – 8.8 byte av läckande takplåt till ny dubbelfalsad 0,6 mm plåt, utfört av firman Ena Golv \& Tak, Jakobstad. \\
    \hline
  \end{tabular}
\end{center}



%%%
% [house] Lillkungs
%
\jhhouse{Lillkungs}{3:94}{Fors}{6}{98, 98a}

\jhhousepic{039-05575.jpg}{Familjen Havryliuk bor i huset}

%%%
% [occupant] Havryliuk
%
\jhoccupant{Havryliuk}{\jhname[Andrii]{Havryliuk, Andrii} \& \jhname[Tetyana]{Havryliuk, Tetyana}}{2013--}
Den ukrainska familjen Havryliuk hade anlänt till Jeppo, köpte fastigheten 2013 och flyttade in följande år. Man har därefter utfört vissa omändringar i fasaderna, bl.a. med tilläggsisolering och ny panel på tre väggar. Fönster mot Jungarvägen har också lagts fast för att minska trafikbuller och insyn. Andrii, \textborn 01.05.1980, har arbete på Jeppo Lantgris, medan Tetyana, \textborn 17.02.1980, f.n. (2017) är studerande på Kredu - Kristliga folkhögskolan i Nykarleby.
\begin{jhchildren}
  \item \jhperson{\jhname[Yaroslav]{Havryliuk, Yaroslav}}{16.06.2009}{}
  \item \jhperson{\jhname[Oleksandr]{Havryliuk, Oleksandr}}{02.02.2011}{}
  \item \jhperson{\jhname[Denis]{Havryliuk, Denis}}{02.12.2016}{}
\end{jhchildren}
De två äldsta barnen går i Jeppo-Pensala skola. Enligt egen utsago trivs familjen bra i sin nya tillvaro.


%%%
% [occupant] Elenius
%
\jhoccupant{Elenius}{\jhname[Uno]{Elenius, Uno} \& \jhname[Aune]{Elenius, Aune}}{1969-2013}
Uno, \textborn 08.09.1912, gifte sig 1939 med Aune Siiponen, \textborn 23.06.1916 i Pelkkala. (Gunnar nr 124)
\begin{jhchildren}
  \item \jhperson{\jhname[Per]{Elenius, Per}}{1940}{}
  \item \jhperson{\jhname[Tor]{Elenius, Tor}}{1941}{}
  \item \jhperson{\jhname[Nils]{Elenius, Nils}}{1943}{}
  \item \jhperson{\jhname[Bengt]{Elenius, Bengt}}{1948}{}
\end{jhchildren}
Uno och Aune köpte fastigheten 1969. Busstationen som verkat i byggnaden kvarhölls endast en tid efter ägarbytet. I utrymmena öppnades istället \jhbold{tygaffären} ``Midinette'' som förestods av Tuulikki Elenius, Uno och Aunes svärdotter. Där har senare även annan butiksverksamhet försökt etablera sig, dock med kortvarig historia. Bostaden har stått i sonen Tors ägo och hyrts ut, men har nu sålts till familjen Havryliuk från Ukraina.

\jhbold{Hyresgäst;}
Under tiden 2009--2012 hyrde familjen Vovk huset. Svetsare Serhii Vovk, \textborn 08.10.1970, och hans hustru Larysa, \textborn 21.04.1974, båda från Ukraina, flyttade sedan med sina barn till Stationsvägen. Se mera under hus nr 62, karta 5.

%%%
% [occupant] Lillkung
%
\jhoccupant{Lillkung}{\jhname[Alf]{Lillkung, Alf} \& \jhname[Alma]{Lillkung, Alma}}{1956-1969}
Alf Sigurd Lillkung, \textborn 30.01.1915 gift med Alma Johanna, \textborn 09.03.1911 från Oravais, köpte fastigheten den 06.02.1956. Bageriet lades nu ner och fastighetens nedre våning inreddes till \jhbold{skrädderi}, som Alf efter fastighetsköpet flyttade från det inhyrda utrymmet hos Olga Jungerstam i närheten av stationen. Småningom försämrades lönsamheten, verksamheten avslutades och Alf och Alma flyttade till Oravais. År 1957 byggdes den ekonomibyggnad som nu står på tomten.

Efter flytten hyrdes fastigheten ut till Börje, \textborn 10.10.1937, och Anneli, \textborn 21.10.1940, Julin, som öppnade en välkommen \jhbold{busstationsverksamhet} för bussbolaget Haldin \& Rose under 1960-talets första hälft. Som en av de första tog Börje värvning i FN:s fredsbevarande styrkor från Finland och enrollerades i Cypern-krisen 1964 tillsammans med Leif Kula från Jeppo. Börje var FN-soldat 10.4.--10.8.1964 och flyttade efter en tid till sina fäldrars hus på Holmen.

Almberg Manne och Kerttu hyrde fastigheten en tid av Lillkungs, men flyttade ut 1969 då gården såldes och bytte ägare.

Med Alf \& Alma bodde barnen Carey, Nils, Rainer och Solveig.


%%%
% [occupant] Strengell
%
\jhoccupant{Strengell}{\jhname[Valdemar]{Strengell, Valdemar} \& \jhname[Ellen]{Strengell, Ellen}}{1938-1956}
Valdemar och Ellen Strengell köpte huset inklusive bageriet 03.09.1938 och fortsatte bageriverksamheten. De hade tidigare bott på Jungar (se Ruotsala nr 26) men gjorde en bytesaffär med Sigurd och Heldine Jungar, som istället nu flyttade till Strengells fastighet på Jungar. Under de sista åren i huset bodde också Ellens föräldrar Edvard och Ida Forss tillsammans med dem mellan åren 1948 och 1956.
Barn: jfr Fors, karta 7, nr 102

Den 6 febr.1956 såldes fastigheten till Alf Lillkung.


%%%
% [occupant] Jungar
%
\jhoccupant{Jungar}{\jhname[Sigurd]{Jungar, Sigurd} \& \jhname[Heldine]{Jungar, Heldine}}{1936-1938}
Sigurd Jungar, \textborn 24.10.1909, gifte sig 28.06.1936 med Jenny Heldine Lundvik, \textborn 13.12.1915 på Kaup. De köpte bageriet i vägkorsningen på Fors år 1936 av Evald och Signe Nygård och övertog verksamheten, men 03.09.1938 avyttrade de fastigheten och rörelsen till Valdemar och Ellen Strengell. 1955 flyttade Sigurd och Heldine till Sverige.
\begin{jhchildren}
  \item \jhperson{\jhname[Lilly Ann-Britt]{Jungar, Lilly Ann-Britt}}{19.12.1937}{}
  \item \jhperson{\jhname[Gretel Heldine]{Jungar, Gretel Heldine}}{07.07.1943}{}
  \item \jhperson{\jhname[Bo Erik Johan]{Jungar, Bo Erik Johan}}{10.10.1945}{}
\end{jhchildren}


%%%
% [occupant] Nygård
%
\jhoccupant{Nygård}{\jhname[Evald]{Nygård, Evald} \& \jhname[Signe]{Nygård, Signe}}{1920-talet}
Evald Nygård, \textborn 19.03.1901 i Markby, gifte sig 28.06.1922 med Signe Maria Broo, \textborn 11.04.1899.

I början av 1920-talet köpte makarna en bostadstomt av Bäck 3:25 av Fors skattehemman. De bodde en tid på Keppo gård där Evald arbetade och deras första dotter Ragni föddes. Efter några år byggdes \jhbold{bageri}, inkluderande bostad, på den köpta tomten. Efter ett antal verksamma år såldes fastigheten till Sigurd och Heldine Jungar och Evald och Signe flyttade till Jakobstad, där Evald fick anställning som montör och busschaufför på Haldin \& Rose. Ragni fick arbete på Strengbergs fabrik. Sonen Bengt arbetade en tid på Jakobstads Tidnings tryckeri, men öppnade efter en tid en egen bensinstation.
\begin{jhchildren}
  \item \jhperson{\jhname[Ragni Maria]{Nygård, Ragni Maria}}{13.10.1923}{i nov 1996}
  \item \jhperson{\jhname[Bengt Fride Bernhard]{Nygård, Bengt Fride Bernhard}}{06.11.1926}{}
\end{jhchildren}



%%%
% [house] Löv
%
\jhhouse{Löv}{3:24}{Fors}{6}{399}


%%%
% [occupant] Kangas
%
\jhoccupant{Kangas}{\jhname[Hjalmar]{Kangas, Hjalmar} \& \jhname[Alma]{Kangas, Alma}}{1950-tal-1964}
Hjalmar Kangas, \textborn 28.08.1894 i Vörå gifte sig 23.05.1920 med Alma Johanna Sihvonen, \textborn 16.06.1892 i Åbo. Paret bodde som hyresgäster i huset i slutet av 1950-talet och fram till 60-talet. De var ett försynt och strävsamt pensionärspar. När Hjalmar dog 16.02.1964 flyttade Alma den 14 juli samma år till Tammerfors. Hur det kom sig att dessa två kom till Jeppo har vi inte kunnat utröna.

Efter Hjalmar och Alma har huset inrymt ett flertal kortvariga gäster som p.g.a. husets dåliga skick gärna flyttade bort så fort som möjligt. Också under tidigare perioder av husets historia har ett flertal hyresgäster passerat revy.

Uthuslängan, som stod med norra gaveln mot landsvägen, var ett populärt och synligt affischställe i centrum av Silvast och här fick man se kandidater till såväl riksdag som president trängas skuldra vid skuldra, efter en tid slitna av både snöplog och snålblåst.

Fastigheten, som kallades ``Lövskos'', efter Löv, revs på 1980-talet.


%%%
% [occupant] Sandberg
%
\jhoccupant{Sandberg}{\jhname[Edit]{Sandberg, Edit} \& \jhname[Elmer]{Sandberg, Elmer}}{1924-1980-talet}
Edit Julia Löv, \textborn 29.12.1901 i USA, gifte sig år 1924 med Jakob Elmer Sandberg, \textborn 19.01.1902 på Back. De övertog fastigheten efter Edits föräldrar. De övertog också ¼ av Elmers föräldrars hemman med 15 ha odlad jord och 20 ha skog på Ojala, där de bosatte sig. Åren 1926-27 var Elemer i Kanada på förtjänstarbete. Efter drygt 30 år hemmavid reste han på nytt 1961, denna gång till Kalifornien och Edit följde efter samma år. Efter fem år återvände de nu för gott. De bodde endast de första åren efter giftermålet i huset, i övrigt vid Ojala.
\begin{jhchildren}
  \item \jhperson{\jhname[Emry Elof]{Sandberg, Emry Elof}}{01.06.1924}{}
  \item \jhperson{\jhname[Erland Levi Johannes]{Sandberg, Erland Levi Johannes}}{03.09.1928}{}
  \item \jhperson{\jhname[Edna Marita]{Sandberg, Edna Marita}}{23.05.1932}{}
\end{jhchildren}


%%%
% [occupant] Löv
%
\jhoccupant{Löv}{\jhname[Gustav]{Löv, Gustav} \& \jhname[Anna Lovisa]{Löv, Anna Lovisa}}{1896-1924}
Gustav Löv, \textborn 14.06.1867 gifte sig den 05.11.1896 med Anna Lovisa Johannesdotter, \textborn 12.09.1873 på Ruotsala. Anna Lovisa hade 10.10 .1896 flyttat till Nykarleby. Hennes föräldrar Johannes och Sanna, födda 1848 resp. 1846 bodde på Ruotsala, men flyttade tydligen till Amerika 1892. Hennes man Gustav är antecknad född i Alahärmä.
\begin{jhchildren}
  \item \jhperson{\jhname[Hilda Lovisa]{Löv, Hilda Lovisa}}{20.09.1900}{}
  \item \jhperson{\jhbold{\jhname[Edit Julia]{Löv, Edit Julia}}}{29.12.1901}{}
  \item \jhperson{\jhname[Gustav Selim]{Löv, Gustav Selim}}{17.01.1904}{}
  \item \jhperson{\jhname[Ester Susanna]{Löv, Ester Susanna}}{24.02.1907}{}
\end{jhchildren}

Efter barnens födelse reste också Gustav till Amerika och dog där 1907. Anna Lovisa gifte om sig den 09.06.1912 med Thomas Vilhelm Lindström, \textborn 05.06.1878.
Barn: Evert Vilhelm Lindström, \textborn 02.12.1912--\textdied 28.07.1985



%%%
% [house] Varma-huset
%
\jhhouse{Varma-huset}{3:95}{Fors}{6}{391}


\jhhousepic{Varma.JPG}{F.d. V.Bomans gård, Varma-butik och handarbetsaffär.}

%%%
% [occupant] Stenbacka
%
\jhoccupant{Stenbacka}{\jhname[Alf]{Stenbacka, Alf} \& \jhname[Maija]{Stenbacka, Maija}}{1970-1975}
Alf och Maija Stenbacka (se nr 91) övertog fastigheten efter hans föräldrar. I medlet av nov. 1967 öppnade Maija en handarbets- och textilaffär, Han-Tex, i fastighetens tidigare butiksutrymmen. Affären var centralt placerad sedan tidigare, stående på vägkanten i korsningen till Oravaisvägen. Men sommaren 1971 skulle vägen breddas. Alla de björkar som tidigare gett vägsträckning dess prägel av allé skulle nu huggas ned, och på sommaren får Maija besök av VoV:s personal, som meddelar att butikstrappan måste avlägsnas för att inte hamna ut på vägen. I samband med detta avslutas verksamheten i fastigheten och huset rivs 1975.


%%%
% [occupant] Stenbacka
%
\jhoccupant{Stenbacka}{\jhname[William]{Stenbacka, William}}{1935-1970}
Fastigheten köptes 1935 av William Stenbacka (se nr 91). Andelslaget VARMA hyrde fastigheten i slutet av 1930-talet och inrättade här en kolonialvaruaffär hörande till OTK-gruppen. Med åren började man hantera ett allt bredare varusortiment; bl.a. konstgödsel, hö och bränsle. Här har genom åren flera  olika föreståndare med sin personal strävat efter att hålla jeppoborna med varor av alla de slag och här fanns också post att avhämtas innan systemet med utdelning till enskilda hushåll startade. Einar Strand och Senny Nylund har här anlitats vid behov, likaså Margit Julin, Ragni Björkqvist, Gundel Juslin, gift Lillström och Gunborg Enström, gift Stenvall arbetade här som expediter under olika perioder. Längst engagerad var Gundel Juslin/Lillström i 10 år, 1957--67.

Föreståndare har genom åren varit:
\begin{enumerate}
  \item John Helsing, 1938-1947, med avbrott under kriget
  \item Sigurd Norrgård, 1947-1956
  \item Gunnar Almberg, 1956-1967
\end{enumerate}

VARMA avslutade verksamheten år 1967.


%%%
% [occupant] Centrallaget Ägg
%
\jhoccupant{Centrallaget Ägg}{\jhname[Produkt m.b.t.]{Centrallaget Ägg, Produkt m.b.t.}}{1932-1935}
Den 31 okt. 1932 sålde barnen till Viktor Boman , Agnes och Karl, fastigheten inkluderande bostadsbyggnad och uthusbyggnad till det nygrundade Centrallaget Ägg Produkt m.b.t. i Jeppo. Företaget köpte och sålde medlemmarnas äggproduktion och tanken var att samordna både utbudet och försäljningen, men verksamheten blev kortvarig och fastigheten såldes 1935 till William Stenbacka, som under tiden ett nytt hus byggdes, bodde i huset (se nr 91) och därefter hyrde ut huset som affärslokal till VARMA.


%%%
% [occupant] Nylund
%
\jhoccupant{Nylund}{\jhname[Brita]{Nylund, Brita} \& \jhname[Johan]{Nylund, Johan}}{1903-1908}
Brita Sofia Frilund, \textborn 04.10.1879 i Jeppo, flyttade som ung  till Nykarleby i febr. 1897, där hon tydligen träffade Johan Herman Nylund, \textborn 14.07.1873, och gifte sig med honom 04.06.1897.

Efter att Britas far Johan Karlsson Frilund hade avlidit 1903 kom makarna till Jeppo och övertog gården och hemmanet. Johan Herman reste nu till Amerika som så många andra, men hörde inte av sej. Hustrun Brita reste efter för att leta efter honom, men fick så  småningom beskedet att han dött i en skogscamp utan att någon kommit sej för att meddela detta. Brita reste hem. Något senare fick hon meddelande från sin stora syster Anna Lovisa i Amerika att hennes barn insjuknat i lungsot och hoppades Brita kunde komma och sköta hennes barn. Brita reste igen till Amerika och som 27-åring återvände hon än en gång till det tomma huset. Syster Ida hade gift sej och flyttat till Pensala, de andra syskonen hade emigrerat. Nu fick hon ta hand om egendomen.


%%%
% [occupant] Boman
%
\jhoccupant{Boman}{\jhname[Brita]{Boman, Brita} \& \jhname[Viktor]{Boman, Viktor}}{1908-1918}
Hon gifte sej nu den 17.05.1908 med den 6 år yngre Viktor Boman, \textborn 03.02.1885 på Ojala som son till Isak Eriksson Boman, som flyttat till Ojala från Nykarleby. Britas yngre bror Emil, som också emigrerat, återvände nu hem som invalid efter att ha förstört sin fot i skogsarbete. Han sålde sin andel av hemgården och dog 2 febr. 1911. Brita och Viktor fick tre efterföljare.
\begin{jhchildren}
  \item \jhperson{\jhbold{\jhname[Agnes Sofia]{Boman, Agnes Sofia}}}{21.01.1909}{}, gift med Eino Klaavo i Pornais
  \item \jhperson{\jhbold{\jhname[Karl Viktor]{Boman, Karl Viktor}}}{01.05.1910}{}, till Kanada 1927
  \item \jhperson{\jhname[Edit Susanna]{Boman, Edit Susanna}}{01.03.1913}{}, gift med Axel Backlund, Vasa
\end{jhchildren}

Brita dog den 9 juli 1918, oklart om i spanska sjukan eller struptuberkulos.


%%%
% [occupant] Boman
%
\jhoccupant{Boman}{\jhname[Viktor]{Boman, Viktor} \& \jhname[Ida]{Boman, Ida}}{1918-1932}
Viktor gifte om sig 18.03.1923 med Ida Dahl, \textborn 18.09.1881 i Munsala. Den 26 juli 1923 flyttade hon till Jeppo. Den 30 maj 1924 tog de ut betyg för resa till Amerika, men reste de? I vilket fall som helst emigrerade deras son Karl Viktor 15 maj 1927 till Kanada och den 28 febr. 1939 hade Viktor och Ida sålt sin egendom och flyttat till Munsala där Viktor dog 05.11.1962 och Ida 07.08.1968.


%%%
% [occupant] Frilund
%
\jhoccupant{Frilund}{\jhname[Johan]{Frilund, Johan} \& \jhname[Maja-Lisa]{Frilund, Maja-Lisa}}{1880-1903}
Huset byggdes på 1880-talet av Johan Karlsson Frilund, \textborn 09.07.1840 i Munsala. Han hade gift sig med Maja Lisa Andersdr., \textborn 05.08.1842 i Jeppo.
\begin{jhchildren}
  \item \jhperson{\jhname[Erik Johan]{Frilund, Erik Johan}}{07.03.1870}{}, till Amerika
  \item \jhperson{\jhname[Anna Lovisa]{Frilund, Anna Lovisa}}{10.03.1872}{}, till Amerika
  \item \jhperson{\jhname[Johanna Katarina]{Frilund, Johanna Katarina}}{21.04.1874}{07.10.1894}
  \item \jhperson{\jhname[Ida Maria]{Frilund, Ida Maria}}{30.07.1876}{}, gift Andersson, Pensala
  \item \jhperson{\jhbold{\jhname[Brita Sofia]{Frilund, Brita Sofia}}}{04.10.1879}{}, gift Nylund/Boman
  \item \jhperson{\jhname[Anders Wilhelm]{Frilund, Anders Wilhelm}}{05.01.1882}{}, till Amerika
  \item \jhperson{\jhname[Jakob Emil]{Frilund, Jakob Emil}}{14.04.1886}{}, till Amerika
\end{jhchildren}
I familjen bodde också en timmerman/snickare kallad ``Krymbo-Janne'', som sannolikt var en nära släkting.
Johan \textdied 11.04.1903  ---  Maja Lisa \textdied 05.07.1889



%%%
% [house] Stenbacka
%
\jhhouse{Stenbacka}{3:95}{Fors}{6}{91, 91a-b}

\jhhousepic{096-05635}{Familjen Kruhliak, tidigare Alf och Maija Stenbacka}

%%%
% [occupant] Kruhliak
%
\jhoccupant{Kruhliak}{\jhname[Yuliia]{Kruhliak, Yuliia} \& \jhname[Viacheslav]{Kruhliak, Viacheslav}}{2016--}
Yuliia Kruhliak, \textborn 26.05.1987, och Viacheslav Krugliak, \textborn 11.08.1986, båda ukrainare, köpte fastigheten 2016 av Alf Stenbackas dödsbo. Mannen kom till Jeppo i slutet av 2013 och frun med dotter i januari 2014. De olika skrivningarna av makarnas efternamn beror på att myndigheten, som skrev ut passen, har stavat namnet på olika sätt. Krugliak är m.a.o. en lika god rättskrivning.
\begin{jhchildren}
  \item \jhperson{\jhname[Arina]{Kruhliak, Arina}}{03.08.2012}{}
\end{jhchildren}
Viacheslav har från första början varit anställd som svinskötare hos Kenneth Lindén medan Yuliia studerat språk i två år på Kredu och nu studerar till närvårdare på Yrkesakademin i Bennäs. Med det och sin 6-åriga bakgrund som sjukskötare på urologisk avdelning i Ukraina, har hon haft möjlighet att sedan mars 2017 fungera som inhoppare på Nykarleby sjukhem. Dottern Arina är glad över gemenskapen på daghemmet och familjen trivs bra i Jeppo.


%%%
% [occupant] Stenbacka
%
\jhoccupant{Stenbacka}{\jhname[Alf]{Stenbacka, Alf} \& \jhname[Maija]{Stenbacka, Maija}}{1970-2016}
Alf Vilhelm Stenbacka, \textborn  04.02.1935, gifte sig 26.08.1962 med Anna Maija Savolainen, \textborn 25.07.1942 i Uleåborg. De övertog fastigheten i början av 1970-talet.

Alf fick handelsutbildning och utexaminerades som ekon. mag. Han har hela sitt verksamma liv arbetat på ekonomisidan inom företaget Österbottens Trä i Jakobstad, först på bokföringen och sedan som fältgranskare inom ett distrikt, som sträckte sig från Närpes till Brahestad, vilket innebar många körkilometer. Han hade också övertagit lastbilstrafiken efter sin far William. Den fortsatte till 1980 då verksamheten lades ner. Som chaufförer hade fungerat Terho Ekola, Veikko Kontio och sist Manne Almberg. Bilarna användes vid entrepenader i VoV:s verksamhet. Under de perioder chauffören var ledig hoppade Alf själv in som chaufför. Han gick i pension från Österbottens Trä 1993.

Maija har sedan nov. 1967 haft en handarbets- och tygaffär i grannfastigheten (se nr391) och sedan 1972 kortvariga anställningar på John och Gunilla Nymans affärer, dels inom matvarusidan och dels på lantbruksvarusidan. Hon har också haft korta inhopp i Mirkakontoret och i ortens Kemikalia-affär. Sedan 1977 har hon  haft anställning på Nykarleby-Jeppo försäkringsförening, till en början på halvtid, men så småningom på heltid. Sedan föreningens kontor flyttat till Nykarleby 1981 blev hon ansvarig för kontoret i Jeppo fram till pensionen 2006, varefter filialen indrogs 2007. Maija bor numera i Åbo.
\begin{jhchildren}
  \item \jhperson{\jhname[Ann-Christine]{Stenbacka, Ann-Christine}}{01.03.1963}{}, tandläkare, bosatt i Åbo
  \item \jhperson{\jhname[Björn Johan Vilhelm]{Stenbacka, Björn Johan Vilhelm}}{12.06.1965}{}, redaktör på Vasabladet
\end{jhchildren}

Alf \textdied 07.12.2012


%%%
% [occupant] Stenbacka
%
\jhoccupant{Stenbacka}{\jhname[William]{Stenbacka, William} \& \jhname[Alva]{Stenbacka, Alva}}{1935-1970}
Johan William Stenbacka, \textborn 02.09.1902 i Alahärmä, gifte sig 17.06.1934 med Alva Elise Andersson, \textborn 19.06.1907 i Gamlakarleby. Hon var dotter till banmästare Johannes Andersson. Williams far, Jakob Eriksson Stenbacka hade kommit från Munsala 18.11.1905 och var då gift med Mina Joh.dr., \textborn 25.09:1874 i Alahärmä. De hade tydligen varit till Amerika där deras äldsta son Jakob Edvin föddes 03.04.1901.

William föddes 1902 i Alahärmä. Han blev 1920 dräng hos Amanda Mattsdr.(Måtar) Nylund, vars man Gustaf Mattsson Måtar avlidit 17.10.1917. År 1921 flyttade han till Back där han fanns bosatt som ba{c{k{s{t{}.karl fram till början av 1930-talet. Han flyttar därefter till Keppo, där han får arbete som arbetsledare och också lastbilschaufför fram till 1943.

Den 2 maj 1935 köper han tomten av Centrallaget Ägg Produkts i Silvast och bor i det hus som använts av dem (se nr 391), tills han byggt detta hus åt familjen. Efter kriget anskaffade William egen lastbil och började livnära sig på transporter. Han blev småningom entrepenör för Väg o Vatten och skötte om landsvägarna i Jeppo såväl sommar som vintertid. Torra somrar var det viktigt att dämpa dammet på grusvägarna med regelbunden bevattning av vägytan och hans pump med plats för påfyllning av vattentanken var flera år placerad nere vid Silvast kvarn. William idkade även taxitrafik i Jeppo. Under kriget hade Vellamo Halme frisörssalong i huset.

Barn: \jhbold{Alf} Vilhelm, \textborn 04.02.1935.

William \textdied 22.01.1970  ---  Alva \textdied 14.03.1972



%%%
% [house] Sikström
%
\jhhouse{Sikström}{3;93}{Fors}{6}{99, 99a}


\jhhousepic{149-05704.jpg}{Markku Yliaho och Sandra Sirén}

%%%
% [occupant] Yliaho
%
\jhoccupant{Yliaho}{\jhname[Markku]{Yliaho, Markku}}{2010--}
Markku Yliaho, \textborn 09.08.1985, köpte fastigheten 22.12.2010. Han arbetar på Mirka och har grundligt renoverat huset och garaget. Han är sambo med Sandra Sirén, \textborn 08.12.1989 i Pensala.


%%%
% [occupant] Strengell
%
\jhoccupant{Strengell}{\jhname[Bo]{Strengell, Bo}}{1987-2010}
Bo Rune Strengell, \textborn 19.04.1956 erhöll fastigheten som gåva av sina föräldrar Valde och Ellen Strengell den 19 okt. 1987. Föräldrarna fortsatte att bo kvar och efter att Valde avlidit 1993 bodde Ellen ensam kvar tills hon flyttade till Hagalund. Huset hyrdes då ut till Linda Sjö.


%%%
% [occupant] Strengell
%
\jhoccupant{Strengell}{\jhname[Valde]{Strengell, Valde} \& \jhname[Ellen]{Strengell, Ellen}}{1973-1987}
Valdemar och Ellen Strengell köpte den 17 mars 1973 fastigheten av kantor K.J. Sikströms arvingar. Den kom att tjäna som deras pensionärsbostad efter att hemmanet överlåtits till Bruno och Siv Strengell, se karta 7, nr 102.


%%%
% [occupant] Sikström
%
\jhoccupant{Sikström}{\jhname[K.J.]{Sikström, K.J.} \& \jhname[Katarina]{Sikström, Katarina}}{1944-1973}
Karl Johan Sikström, \textborn 1884, och hustrun Ida Katarina, \textborn 1882, köpte 14 juni 1944 fastigheten av Olivia Jungell i den fastighetscirkulation som också reglerade äganderätten till Ida Katarinas fastighet på andra sidan landsvägen och som nu köptes av Leander och Ellen Sandberg. Olivia kom därefter att leva i det husets 2:a våning. (se nr 407)

Efter att Sikströms köpt huset fortsatte han sin kantorsgärning fram till 1956 och huset blev allmänt kallat till ``Sikströms hus''. Ida Katarina dog 1957 och Karl Johan 1960. Huset kom nu att stå tomt en kort tid innan det hyrdes ut till ett flertal personer fram till 1973: Hjalmar och Alma Kangas, Börje och Anneli Julin, Volmar och Benita Strand och Karl-Erik och Ing-Britt Forss.


%%%
% [occupant] Jungell
%
\jhoccupant{Jungell}{\jhname[Anders]{Jungell, Anders} \& \jhname[Olivia]{Jungell, Olivia}}{1938-1944}
Anders Jungell, \textborn 26.07.1883 på Skog, gifte sig med Olivia Pettersdr. Ersfolk från Övermark. 23 mars 1938 köpte de fastigheten av Selma Sandlin efter att de sålt det gästgiveri på Silvast som de byggt och verkat i fram till slutet av 1930-talet, då gästgiverierna slutat fylla sin ursprungliga funktion. Tåg, bussar och bilar hade tagit över.


%%%
% [occupant] Sandlin
%
\jhoccupant{Sandlin}{\jhname[Selma]{Sandlin, Selma}}{1936-1938}
Selma Sandlin, \textborn 13.03.1893, köpte huset 1936 av skomakaren Eljas Keto-oja och bebodde det endast en kort tid innan hon sålde det till Anders och Olivia Jungell.


%%%
% [occupant] Keto-oja
%
\jhoccupant{Keto-oja}{\jhname[Eljas]{Keto-oja, Eljas} \& \jhname[Anna]{Keto-oja, Anna}}{1926-1936}
Eljas Eemeli Eljasson Keto-oja. \textborn 30.08.1899 i Kuortane, gifte sig 05.09.1926 med Anna Johanna Salo, \textborn 04.09.1907 i Jeppo. Eljas hade anlänt till Jeppo 12 aug. 1926 från Kuortane och han var verksam som en av flera skomakare i socknen vid denna tid. 1926 torde han ha byggt detta hus för att efter 10 år flytta med familjen till Nurmo 7 dec.1936.
\begin{jhchildren}
  \item \jhperson{\jhname[Elias Einari]{Keto-oja, Elias Einari}}{24.12.1926}{}
  \item \jhperson{\jhname[Anita Anna Viola]{Keto-oja, Anita Anna Viola}}{01.01.1932}{}
\end{jhchildren}



%%%
% [house] Olofsborg
%
\jhhouse{Olofsborg}{3:122}{Fors}{6}{107, 107a}


\jhhousepic{097-05636.JPG}{Greger Nygård}

%%%
% [occupant] Nygård
%
\jhoccupant{Nygård}{\jhname[Greger]{Nygård, Greger}}{1998--}
Greger Anders Ruben Nygård, \textborn 16.05.1962 är utbildad trädgårdsmästare. Han köpte fastigheten av sina föräldrar 15.07.1998. Greger äger en servicefirma under namnet Nyträ, som utför ett brett spektrum av serviceuppgifter; plantering, gräsklippning, plogning och röjning av snö, sandning av vägar och cykelvägar, uppdrag med mindre grävmaskiner m.m.

Greger har också satsat i fastighetsägande genom delägarskap i den tidigare kommungården och självständigt ägande av tidigare Sparbankens fastighet och tidigare brandstationen. Bostadslokalerna hyrs ut och har bl.a. under 2015--17 delvis fungerat som lokaler för språkundervisning och boende för irakiska asylsökande vid Oravais flyktingförläggning, då behovet i samband med Irak- och Syrienkrisen plötsligt växte.


%%%
% [occupant] Nygård
%
\jhoccupant{Nygård}{\jhname[Ruben]{Nygård, Ruben} \& \jhname[Ragnborg]{Nygård, Ragnborg}}{1975-1998}
Ruben Olof Nygård, \textborn 06.05.1934 gifte sig 11.09.1960 med Ragnborg Maria Sandås, \textborn 27.03.1937. Makarna köper fastigheten efter Leander Sandbergs död 1975 och flyttar in på självständighetsdagen den 6 dec.

Ruben är utbildad VVS-montör och efter diverse mindre arbeten här hemma reser han till Sverige och arbetar inom en billackeringsfirma, senare på en såg och slutligen med rörarbeten innan han kommer hem 1959. Ett av de första arbetena hemmavid blir att hugga ner den skog som stod på den tomt mittemot farbror Mannes verkstad där Jeppo - Oravais Handelslag planerar att bygga en filial i Gränden! I och med fusionen med Nykarleby Handelslag och bildandet av den nya Andelsringen förfaller tanken.

1960 får han anställning som svetsare på Purmo Produkt och samma år anställs han som chaufför på Andelsringen, ett arbete han har till 1965, då han får anställning som rörarbetare vid det påbörjade bygget av en ny centrumskola. Efter att skolan står klar 1966 väljs han till dess första vaktmästare. 1973 söker han sej till snickeriföretaget Westwood i Nykarleby, där han stannar till 1993, då han pensioneras.

Ragnborg var utbildad barnskötare. Hon fick arbete som städare på den nya skolan 1966 och kvarstod i tjänst till 1992. Ragnborg dog 16.12.2003
\begin{jhchildren}
  \item \jhperson{\jhbold{\jhname[Greger]{Nygård, Greger}} Anders Ruben}{16.05.1962}{}
  \item \jhperson{\jhname[Ulrika Maria]{Nygård, Ulrika Maria}}{23.03.1965}{}
  \item \jhperson{\jhname[Patrik Ruben Johannes]{Nygård, Patrik Ruben Johannes}}{08.03.1967}{}
\end{jhchildren}


%%%
% [occupant] Sandberg
%
\jhoccupant{Sandberg}{\jhname[Leander]{Sandberg, Leander}}{1968-1975}
Mathias Leander Sandberg, \textborn 18.05.1906 började bygga huset i slutet av 1960-talet för sin pensionärstillvaro. Bostadsytan är ca 95 m$^2$. Samtidigt rev han det gamla huset på tomten, som varit familjens hemviste sedan 1944 (se nr 407).

Leander \textdied 22.08.1975



%%%
% [oldhouse] Karlsborg
%
\jholdhouse{Karlsborg}{3:122 (3:18)}{Fors}{6}{407}


%%%
% [occupant] Sandberg
%
\jhoccupant{Sandberg}{\jhname[Leander]{Sandberg, Leander} \& \jhname[Ellen]{Sandberg, Ellen}}{1944-1972}
Mathias Leander Engelbrekt Sandberg, \textborn 18.05.1906 på Back, gifte sig den 15.05.1932 med Ellen Katarina Joh.dr. Jungell, \textborn 05.01.1908 på Skog. Ellen gick som ung i Kristliga folkhögskolan i Nykarleby.

Den 14 juni 1944 köpte Leander och Ellen fastigheten av klockaren K.J. Sikström och hans hustru Katarina  m.fl. Köpet ingick i en ganska komplicerad fastighetsrotation där Ellens farbrors änka Olivia Jungell samtidigt säljer Jeppo sockens sista gästgiveri på Bäckstrand 3:25 till K.J. Sikström och hans hustru Katarina, som efter giftermålet med sin 1:a man Jakob Eliel Sandberg, \textborn 14.03.1879, i egenskap av dödsbo ägt Karlsborg 3:18. Jakob hade avlidit 17.11.1910 och Katarina hade gift om sig med K.J. Sikström den 05.01.1913.

Leander och Ellen, som ett par månader efter fastighetsförvärvet säljer sitt hus, som 1938 byggts vid stationen, påbörjar nu bygget av en ny ekonomiebyggnad och ladugård. Vid sidan av odlandet av den hemmansdel, som han övertagit på Ojala, är Leander också en trägen snickare och timmerman i bygden och samarbetar ofta med sin tvillingbror Johannes. Olivia Jungell, som blivit änka efter maken Anders död i febr. 1942, hade i samband med fastighetsrotationen tillförsäkrats boende och hon bodde på husets övre våning.

Efter att hustrun Ellen och Olivia avlider 1961 fortsätter Leander att bo i det stora huset. Den 3 september 1963 drar ett ovanligt häftigt åskväder över nejden och blixten slår ner i huset. Fastän det börjar brinna på ett par ställen, kan brandkåren snabbt få branden under kontroll och det blir mest stora vattenskador. I medlet av 1960-talet beslutar han sej för att bygga ett mindre hus på samma tomt och i samma veva påbörjas rivningen av fastighetens övriga byggnader, vilket avslutas ca 1970.
\begin{jhchildren}
  \item \jhperson{\jhname[Magda Katarina]{Sandberg, Magda Katarina}}{17.11.1933 i gästgiveriet på Silvast}{}
  \item \jhperson{\jhname[Ingmar Matias]{Sandberg, Ingmar Matias}}{04.11.1937}{}
\end{jhchildren}

Ellen \textdied 05.08.1961  ---  Leander \textdied 22.08.1975


%%%
% [occupant] Sikström
%
\jhoccupant{Sikström}{\jhname[Katarina]{Sikström, Katarina} \& \jhname[Karl-Johan]{Sikström, Karl-Johan}}{1913-1944}
Ida Katarina Mattsdr. Sandberg, \textborn 18.05.1882 på Finskas, ingår sitt 2:a äktenskap den 05.01.1913 med klockaren Karl-Johan Sikström, \textborn 21.01.1884 i Petalax. Karl-Johan anlände 25.11.1911 från Petalax till Kaup, sedan han valts till kantororganist i Jeppo församling, en tjänst han kom att inneha till 1956! Efter giftermålet flyttade han in i det hus, som Katarinas första make, Jakob Sandberg, uppfört på Fors nr 3 (se nr 99) i början av 1900-talet.
Barn: Johannes Immanuel Sikström, \textborn 05.02.1914

Ida Katarina \textdied 03.09.1957  ---  Karl Johan \textdied 21.09.1960


%%%
% [occupant] Sandberg
%
\jhoccupant{Sandberg}{\jhname[Jakob]{Sandberg, Jakob} \& \jhname[Katarina]{Sandberg, Katarina}}{1900-1913}
Jakob Eliel Johansson Sandberg, \textborn 14.03.1879, gifte sig med Katarina Sandberg (se ovan) \textborn 1882 och bosatte sej på Stenbacken. Därifrån flyttade familjen till Fors och ett nytt hus byggdes. Det är något oklart när paret besökt Amerika, men åtminstone deras 2:a barn är fött där och Katarina arbetade som mattillredare på en skogscamp, där flera jeppobor slet för brödfödan. Det torde ha varit någon gång mellan 1900--1905.

Jakob dog 17.11.1910 på Fors och Katarina stod ensam med barnskaran.
\begin{jhchildren}
  \item \jhperson{\jhname[Jakob William]{Sandberg, Jakob William}}{26.12.1900}{}
  \item \jhperson{\jhname[Elmer Edvin]{Sandberg, Elmer Edvin}}{05.03.1903 i Amerika}{}
  \item \jhperson{\jhname[Agnes Maria]{Sandberg, Agnes Maria}}{04.11.1905}{}
  \item \jhperson{\jhname[Signe Katarina]{Sandberg, Signe Katarina}}{03.07.1908}{}
  \item \jhperson{\jhname[Gerda Alice]{Sandberg, Gerda Alice}}{29.03.1910}{}
\end{jhchildren}
