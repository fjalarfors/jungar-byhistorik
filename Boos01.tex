\jhchapter{Bösas, hemman Nr 6}



Gammalt folk lär ha berättat att båtrester hittats nära älven och på basen av dessa berättelser har en del forskare undrat om det kan vara rester av gamla Hansa-koggar. För flera hundra år sedan var ju vattenståndet mycket högre än nu och det kunde ha medgett att mindre handelsskepp lätt kunde seglat upp längs älven ända till Jeppo. Sådana händelser skulle vara orsaken till att en del hemmansnamn i Jeppo enligt forskare skulle ha tyskt påbrå; bl.a. Kaup och Böös. Böös skulle då komma från personnamnet Böse och ha en synonym i namnet Buse. Detta enligt Karsten.

Bösas hemman gränsar i norr till Grötas och före 1831 var hemmansnummern nr 24. Gränsen dem emellan går österut från landsvägen i höjd med Bernhard Fogströms bostadstomt, men det område mellan landsväg  och ån som hör till Grötas fortsätter söderut i jämnhöjd med Ulla Linders bostadstomt. I söder gränsar Bösas till Jungar hemman och Bösasbäcken kan väl sägas vara den huvudsakliga gränsen, men några bostadstomter och ett par mindre ägofigurer hörande till Bösas hemman finns söder om bäcken. Här byggdes också Jungar Andelsmejeri 1928. I likhet med på Grötas hemman, där andra hemmansnummer finns insprängda geografiskt, finns också på Bösas hemman samma företeelse. Nu är det Grötas hemman som finns insprängt i Bösas nummer i närheten av Bösas bäck, vilket framgår av gårdsbeskrivningarna.

Till Bösas hör också den plats som kallas ``Ekbacken'',(alt. Eikbackan el. Eitbackan) en skogsdunge som hamnade bakom järnvägen när den drogs 1885. Här fanns i tiden en samling av torp och backstugusittare, vars spår kan skönjas än i dag. Släkten Jungner i Kronoby har sina fäder härifrån.

Hemmanet har ägts av bl.a. Jöns Olofsson 1582--\allowbreak 1583, Simon Andersson 1586, Simon Jönsson 1588--\allowbreak 1618, Simon Eriksson 1647--\allowbreak 1673, Erik Simonsson och Mats Simonsson 1675--\allowbreak 1686,  mågen Esaias Sigfridsson 1688--\allowbreak 1709 med hu.Susanna Mattsdr och tillsammans med medåbon mågen Tomas Mattsson och hu. Brita Mattsdr. 1696--\allowbreak 1700, sonen Tomas Esaiasson 1710--\allowbreak 1711 varefter änkan 1712--\allowbreak 1713 och efter 1723 Henrik Esaiasson med hu. Lisa Knutsdr. År 1783 finns antecknat ägarna Mats med 5/12 mtl och Johan och Anders med 5/24 mtl vardera. År 1850 bodde 73 personer på Bösas.

På gamla skifteskartor framträder den plats där soldattorpsängar anvisades av Bösas hemman, som hade ansvar för soldattorpet Nr 140 tillsammans med Keppo, Mietala och Karckaus. Det gick under namn som Mätas, Bränbacka och från 1775 som Keppos. Ängarna fanns ca 1.5 km österut från älven i bortre hörnet av Bösas hemman. Vid granskning påpekades att Bösas äng skulle rödjas och utvidgas ``emedan then är måsa gången och thes utom otillräckelig''. Ladorna var i stånd. Själva torpet låg strax söder om Palobäcken på Mietala hemmans mark (se närmare under Mietala).

Efter att på 1950-talet ännu haft 8 brukningsenheter igång på det ursprungliga Bösas hemmanet, finns idag inte en enda kvar. All odlad  mark är antingen  utarrenderad eller såld.

Bösas hemman omfattas av karta \jhbold{nr 12}.

KARTA nr 12 hit --->



\jhhouse{Nyåker}{6:25}{Böös}{12}{37, 37a}


\jhoccupant{Nylind}{Gretel \& Åke \& Solveig Nyman}{1981--}
Greta Nylind, Åke Nylind och Solveig Nyman äger idag Nyåker lägenhet enligt gåvobrev av föräldrarna år 1981.
Martti Aho har arrenderat åkern. Gården har stått obebodd i många år.


\jhhousepic{171-05734.jpg}{}

\jhoccupant{Nylind}{Jürgen \& Sign}{1966--\allowbreak 1981}
Jürgen Nylind, \textborn 31.03.1905 på Böös hemman, gifte sig 07.07.1929 med Signe Jungerstam, \textborn 15.05.1903 på Jungar hemman.

\begin{jhchildren}
  \item \jhperson{Paul}{19.04.1930}{}
  \item \jhperson{Tor}{20.03.1932}{}
  \item \jhperson{Clas}{09.02.1934}{}
  \item \jhperson{\jhbold{Solveig}}{25.02.1940}{}
  \item \jhperson{\jhbold{Greta}}{26.06.1941}{}
  \item \jhperson{\jhbold{Åke}}{25.06.1946}{}
\end{jhchildren}

Uppgifter om Jürgen och Signes familj finns under Böös nr 52. Till denna gård flyttade de 1966.
Jürgen dog 06.06.1996, Signe dog 27.06.1997.


\jhoccupant{Nylind}{Paul \& Gretel}{1958--\allowbreak 1966}
Paul Nylind, \textborn 19.04.1930 på Böös hemman, gifte sig 21.06.1952 med Gretel Backlund, \textborn 27.02.1931 i Socklot.

\begin{jhchildren}
  \item \jhperson{Viveca}{23.04.1953}{}, gift Backlund
  \item \jhperson{Helén}{15.08.1955}{}, gift Kronholm
  \item \jhperson{Carina}{01.08.1961}{}, gift Bonde
  \item \jhperson{Monica}{18.03.1963}{}, gift Lindholm
\end{jhchildren}

År 1958 sålde Pauls föräldrar en del av hemmanet till Paul och hans hustru Gretel. På denna mark norr om Böös byggde Paul och Gretel åt sig ett bostadshus, där de bodde till år 1966, då Jürgen och Signe överlät hemmanet och den gamla gården vid Böös åt Paul. Denna gård samt åkern, som han köpt år 1958, överlät Paul och Gretel i sin tur år 1966 åt Pauls föräldrar, som då flyttade in.



\jhhouse{Storgård}{6:6}{Böös}{12}{38, 38a}


\jhhousepic{173-05736.jpg}{}

\jhoccupant{Linder}{Ulla}{2011--}
Ulla Linder, fil.mag,  \textborn 31.12.1963 på Böös. Ulla har arbetat och bott på olika orter både utomlands och i Finland. Hon flyttade tillbaka till hemgården i Jeppo år 2003. Hon har arbetat som frilans åren 2004--\allowbreak 2012, i Nykarleby lokaltv, med flyktingar och som skribent. Hon var koordinator för ett EU-projekt om flyktingintegration i Jakobstadsnejden 2012-14, efter det som företagare och språklärare för flyktingar.

Ulla har renoverat det gamla fähuset och där inhyst kattpensionat. Hon har varit aktiv i bl.a Jeppo byaråd, samt som förtroendevald i fullmäktige perioden 2012-16. Årets Jepobåo 2015. Ulla bor tillsammans med sonen Daniel Ben Erik, \textborn 04.05.2006, på hemgården. Lägenheten består idag av ca 17 ha åkermark, 19 ha skog. Marken är utarrenderad.


\jhoccupant{Linder}{Johannes \& Salli}{1947--\allowbreak 2011}
Johannes Linder, \textborn 10.11.1910 på Böös, gifte sig 1962 med Salli Marjatta Perkiömäki, \textborn 25.11.1923 i Vörå.
Barn: Britt \jhbold{Ulla} Kristin, \textborn 31.12.1963

Johannes Linder övertog hemmanet 1947 tillsammans med sin bror Alfred, född 17.01.1913, död 27.10.1993. Hemmanet omfattade 65 ha, varav 20 ha var odlad jord. Johannes deltog i både vinter- och fortsättningskriget. Han blev skadad och som en följd av krigsskadan amputerades ena benet då han var 85 år.
Johannes \textdied 17.08.2002  --  Sally \textdied 16.07.2011.


\jhoccupant{Linder}{Erik \& Anna Lovisa}{1903--\allowbreak 1947}
Erik Nilsson Linder, \textborn 03.10.1874 på Böös hemman, gift med Anna Lovisa Mietala, \textborn 01.08.1878 på Mietala hemman (Mietala 109)

\begin{jhchildren}
  \item \jhperson{Erik Ivar}{11.04.1904}{10.09.1904}
  \item \jhperson{Ellen}{12.07.1905}{}, gift Lindbäck
  \item \jhperson{Signe}{23.07.1907}{31.07.1907}
  \item \jhperson{Irene}{01.11.1908}{02.01.2011 (Tollikko 1)}, gift Nyström
  \item \jhperson{\jhbold{Johannes}}{16.11.1910}{}
  \item \jhperson{\jhbold{Alfred}}{17.01.1913}{}
  \item \jhperson{Ida Linnea}{19.02.1915}{07.12.1937}
  \item \jhperson{Erik Vilhelm}{10.01.1917}{12.11.1939 vid Leipäsuo}
  \item \jhperson{Runar}{14.07.1919}{21.06.2001 (Silvast 122)}
\end{jhchildren}

Efter ett antal år i gruvarbete i Amerika och Afrika, köpte  Erik sin gamla hemgårdslägenhet av brodern Matts Vilhelms sterbhus och blev jordbrukare på den. Erik Linder uppförde nuvarande bostadsbyggnad år 1909. Ekonomibyggnaden uppfördes av Johannes Linder år 1936. Erik besvärades  av den sk gruvsjukan. Han  dog 9 juni 1932. Anna Lovisa skrev över hemmanet till Johannes och Alfred år 1947. Hon dog 7 december 1951.


\jhoccupant{Nilsson}{Matts \& Greta}{1891--\allowbreak 1903}
Matts Vilhelm Nilsson, \textborn 22.01.1867 på Böös, gift med Greta Sofia Gustafsdotter, \textborn 26.09.1865 på Heikfolk hemman.

\begin{jhchildren}
  \item \jhperson{Gustaf Alarik}{12.02.1892}{}
  \item \jhperson{Susanna Lydia}{20.05.1893}{}
  \item \jhperson{Johan Arthur}{18.03.1895}{}
  \item \jhperson{Ester Sofia}{20.10.1897}{}
\end{jhchildren}

Matts Vilhelms far Nils Mattsson var född på Måtar och köpte Böös lägenhet 30 september 1875 och överlämnade denna till Matts och Greta 4 november 1891. Dessa drabbades dock av sjukdom och Matts Vilhelm dog 13.07.1899, Greta dog 13.03.1903. Barnen omhändertogs av morbror Daniel på Heikfolk.


\jhoccupant{Mattsson}{Nils \& Johanna}{1855--\allowbreak 1891}
Nils Mattsson Böös, \textborn 18.06.1830 på Kaup, gift med  Johanna Johansdotter Mjetala, \textborn 01.11.1840 på Mietala.

\begin{jhchildren}
  \item \jhperson{Matts}{06.11.1859}{}
  \item \jhperson{\jhbold{Johan Jakob}}{21.02.1861 (Böös 39)}{}
  \item \jhperson{Wilhelm}{12.06.1863}{1864}
  \item \jhperson{Maria Lovisa}{03.04.1864}{}
  \item \jhperson{\jhbold{Matts Wilhelm}}{22.01.1867}{}
  \item \jhperson{Sofia}{01.06.1869}{1869}
  \item \jhperson{Mina}{01.06.1869}{}
  \item \jhperson{Johanna Sofia}{17.08.1872}{}
  \item \jhperson{\jhbold{Erik}}{03.10.1874}{}
  \item \jhperson{Anders Gustaf}{24.06.1877}{}
  \item \jhperson{Alfred}{20.08.1881}{}
  \item \jhperson{Leander}{24.08.1884}{}
\end{jhchildren}

Nils och Johanna fick lägenheten på Böös i gåva av Nils' föräldrar 1855, år 1870 köpte de 5/96 mtl av bonden Johan Mattsson Böös. De överlämnade ena halvan 24 april 1885 till sonen Johan Jakob, den andra halvan 4 nov 1891 till sonen Matts Wilhelm. Efter Matts 				Wilhelms död köpte brodern Erik lägenheten av dödsboet.

I gåvobrevet som Nils Mattsson Böös undertecknat 1891, där han överlämnar 5/48 mantal av Böös hemman till sonen Matts Vilhelm förbehålls en torplägenhet till sonen Erik ``vid tågmynningen i Bäckåkern'', samt odlingsåkrar på flere olika platser, totalt 33 kappland. Erik reste dock till Amerika och kom inte att behöva sin torplägenhet. Sytningen till Nils och Johanna fördelades så att Johanna fick sin sytning från Nygårdska delen av hemmanet medan Nils fick sin sytning från Linders del av hemmanet.

Nils Mattsson \textdied 13.07.1899  --  Johanna \textdied 14.12.1909


\jhoccupant{Mårtensson}{Matts \& Maria}{1845--\allowbreak 1875}
Matts Mårtensson, \textborn 10.09.1799 på Måtar, vigd 12.12.1823 med  Maria Mattsdotter, \textborn 10.03.1800 i Socklot.
\begin{jhchildren}
  \item \jhperson{Matts}{26.01.1825 på Måtar}{16.05.1825}
  \item \jhperson{Anders}{02.05.1826    ''}{}
  \item \jhperson{Maja Cajsa}{28.11.1827    ''}{}, g. m. torparen Carl Petter Wikktelin
  \item \jhperson{\jhbold{Nils}}{18.06.1830 på Kaup}{}
  \item \jhperson{Anna}{21.11.1841 på Stenbacka}{}, gift med Simon Blat
  \item \jhperson{Erik}{03.07.1844 på Stenbacka}{11.10.1847 på Böös}
\end{jhchildren}
I gåvobrevet, där Matts och Maria överlåter 5/48 mantal av Böös hemman till sonen Nils Mattsson, beskrivs detaljerat vad som ingår i sytningen till föräldrarna, bl.a ska Nils Mattsson bygga och förfärdiga en stuga med farstu och kammare samt fähus och foderlada. Är detta månne samma torplägenhet som senare var tänkt för Erik Nilsson?

Matts och Maria köpte hemmanet år 1845 av Henrik Mattsson Böös och hans hustru Lisa Isaksdotter. De senare hade hemmanet ett år. De hade köpt det år 1844 av Isak Isaksson Böös den yngre och hans hustru Anna Jakobsdotter.


\jhoccupant{Isaksson}{Isak \& Anna, Isak \& Sanna}{-}
Husbn Isak Isaksson den yngre, \textborn  06.02.1815 på Böös, \textdied 06.08.1861 , hustru 1. Anna Jakobsdr, hustru 2. Sanna Lisa Johansdr, \textborn 1824 på Finskas.
\begin{jhchildren}
  \item \jhperson{Sanna Lisa}{15.07.1840}{}
  \item \jhperson{Isak Isaksson}{03.03.1844}{1869}
  \item \jhperson{Jacob}{02.07.1847}{}
  \item \jhperson{Anna}{04.11.1849}{}
  \item \jhperson{Caisa}{26.07.1852}{}
  \item \jhperson{Johannes}{14.07.1853}{1864}
  \item \jhperson{Anders}{1858}{1863}
  \item \jhperson{Matts}{1861}{1863}
\end{jhchildren}
År 1856 var Isak Isaksson Böös d.y och Matts Mårtensson gemensamma ägare till 5/24 dels mantal av Böös hemman.



\jhhouse{Sunnanäng}{6:20}{Böös}{12}{39, 39a}

\jhoccupant{Lassander}{Peter \& Ulla-Stina}{1989--}
Peter Lassander, \textborn 1961 i Kovjoki, gifte sig 1985 med Ulla-Stina Karlsson-Lassander, \textborn 1960.
\begin{jhchildren}
  \item \jhperson{Victoria}{1989}{}, ergoterapeut, gift Sandin
  \item \jhperson{Joachim}{1991}{}, el.montör på Jeppo Kraft
  \item \jhperson{Christoffer}{1995}{}, studerande
\end{jhchildren}

\jhhousepic{174-05737.jpg}{}

Peter arbetar som specialsnickare vid Westwood, Ulla-Stina arbetar som klasslärare vid Socklot skola. Ulla-Stina har också utbildat sig till massör och har eget företag, Bösas Massage och Hantverk. Familjen köpte gården av Agda Nygård år 1989.


\jhoccupant{Nygård}{Valfrid \& Agda}{1953--\allowbreak 1989}
Valfrid Nygård, \textborn 06.09.1914 på Böös hemman, gift med Agda Fagerholm, \textborn 27.12.1926 på Grötas. Efter några år i Amerika övertog Valfrid år 1953 sina föräldrars lägenhet på Böös. Denna omfattade 55 ha, varav 25 ha var odlad jord.

På 1960-talet hade paret 1 häst, 11 kor och 17 ungdjur. Valfrid uppförde husbyggnaden 1954, samma år som de gifte sig. Den uppfördes på samma plats som den gamla gården, vilken revs. Under byggnadstiden bodde Valfrids föräldrar och syskon i gård nr 41, som blev föräldrarnas stadigvarande bostad i fortsättningen. Ekonomibyggnaden är från 1946. Timret från den rivna gården köptes av Armas Niittysalmi, som av detta uppförde ett garage för bussar vid sitt hem i Voltti.

År 1975 sålde Valfrid och Agda den odlade jorden till staten för att utdelas åt jordbehövande grannar. Valfrid dog 07.03.1986. Agda bor i gård nr 29 på Grötas.


---> \jhbold{Nygård} på Böös hemman, karta 12 nr \jhbold{139}

\jhhousepic{Boos139-NygardEliel-.jpg}{}

\jhoccupant{Nygård}{Eliel \& Elin}{1911--\allowbreak 1953}
Johan Jakob Eliel Jakobsson Nygård, \textborn 12.08.1892 på Böös, gift med Elin Augustsdotter Sjö, \textborn 10.01.1892 på Böös.
\begin{jhchildren}
  \item \jhperson{\jhbold{Valfrid}}{06.09.1914}{}
  \item \jhperson{Ragnar}{01.04.1916 (Romar 13)}{}
  \item \jhperson{Lennart}{05.06.1917 (Tollikko 6)}{}
  \item \jhperson{Torsten}{26.02.1919}{}, till Jakobstad
  \item \jhperson{Elvi}{06.10.1922}{}, gift Eriksson
  \item \jhperson{Hilding}{24.07.1929 (Silvast 53)}{}
  \item \jhperson{Hjördis}{15.09.1932}{}, gift Eklund (Silvast 17)
\end{jhchildren}
År 1911 övertog Eliel föräldrarnas lägenhet om 5/48 mantal på Böös hemman. Eliel var flere gånger till USA på arbetsförtjänst. I Jeppo fanns många hästintresserade, Eliel var en av dem. Han hade flera stambokshästar, med vilka han deltog i utställningar och tävlingar. I februari 1947, då termometern visade 37 minusgrader, deltog han bl.a i Svenska Österbottens hästavelsförenings sto- och fölutställning i Jeppo och fick två ston godkända för intagning  i stamboken.

Eliel \textdied 09.12.1968  --  Elin \textdied 20.10.1965


\jhoccupant{Nilsson}{Johan \& Hanna, Johan \& Maria}{1885--\allowbreak 1911}
Johan Jakob Nilsson, \textborn 21.02.1861 på Böös hemman, gift med Maria 	Jakobsdotter, \textborn 17.06.1860 i Purmo. Maria dog 30.04.1895  och Johan gifte om sig med Hanna (Johanna) Isaksdotter, \textborn 04.02.1859 i Jeppo.
\begin{jhchildren}
  \item \jhperson{Maria Johanna}{01.02.1883}{}
  \item \jhperson{Ida Sofia}{20.05.1885}{}
  \item \jhperson{Anna Lovisa}{02.06.1887}{21.04.1893}
  \item \jhperson{Hilda Katrina}{03.01.1890}{18.12.1891}
  \item \jhperson{\jhbold{Johan Jakob Eliel}}{12.08.1892}{}
  \item \jhperson{Anders Leander}{29.04.1895}{}
  \item \jhperson{Hilma}{22.08.1897}{}
  \item \jhperson{Ester Irene}{30.09.1900}{}
\end{jhchildren}
Jakob Nilssons föräldrar överlämnade ena halvan av sin lägenhet	till sonen Matts Wilhelm (nuv. Linder) och den andra halvan till Johan.			Johans andel var 5/48 dels mantal av Böös hemman.	Sytningen till föräldrarna delades också upp så att Johan Jakob ansvarade för sytningen till modern, Matts Wilhelm till fadern.

Jakob \textdied 30.01.1911  --  Hanna \textdied 21.11.1936


\jhoccupant{Mattsson}{Nils \& Johanna}{1875--\allowbreak 1891}
Nils Mattsson Böös, \textborn 18.06.1830 på Kaup, gift med  Johanna Johansdotter Mjetala, \textborn 01.11.1840 på Jungar.
\begin{jhchildren}
  \item \jhperson{Matts}{06.11.1859}{}
  \item \jhperson{\jhbold{Johan Jakob}}{21.02.1861 (Böös 39)}{}
  \item \jhperson{Wilhelm}{12.06.1863}{1864}
  \item \jhperson{Maria Lovisa}{03.04.1864}{}
  \item \jhperson{\jhbold{Matts Wilhelm}}{22.01.1867}{}
  \item \jhperson{Sofia}{01.06.1869}{1869}
  \item \jhperson{Mina}{01.06.1869}{}
  \item \jhperson{Johanna Sofia}{17.08.1872}{}
  \item \jhperson{\jhbold{Erik}}{03.10.1874}{}
  \item \jhperson{Anders Gustaf}{24.06.1877}{}
  \item \jhperson{Alfred}{20.08.1881}{}
  \item \jhperson{Leander}{24.08.1884}{}
\end{jhchildren}
Nils och Johanna köpte lägenheten på Böös 1875, överlämnade ena halvan 24 april 1885 till sonen Johan Jakob, den andra halvan 4 nov 1891 till sonen Matts Wilhelm. För sonen Erik reserveras torplägenhet vid ``tågmynningen i Bäckåkern''.

Nils Mattsson \textdied 13.07.1899  --  Johanna \textdied 14.12.1909
Se närmare Linders gård (Böös 38).



\jhhouse{Arra}{6:61}{Böös}{12}{40, 40a}


\jhhousepic{175-05738.jpg}{}

\jhoccupant{Back}{David \& Annika}{2010--}
David Back, \textborn 06.04.1981 på Jungar, nr 23, gifte sig 08.08.2009 med Annika Eklöf, \textborn 13.06.1980 i Lillby, Purmo. David är utbildad
tradenom och arbetar som arbetsledare på KWH Mirka. Annika är	utbildad hälsovårdare och arbetar som skolhälsovårdare på Social-	och hälsovårdsverket i Jakobstad.

\begin{jhchildren}
  \item \jhperson{Melanie Maria Terese}{18.08.2011}{}
  \item \jhperson{Minelle Maria Elisabeth}{16.12.2014}{}
\end{jhchildren}
Tomten till huset köptes av Kennet Lindén. Huselementen är från Teri Hus, men det övriga har David och Annika byggt själv. Inflyttning			i huset skedde år 2010.



\jhhouse{Albrant}{6:43}{Böös}{12}{41, 41a}


\jhoccupant{Broo}{Hermans dödsbo}{1988--}
Herman Broo, \textborn 03.07.1945, och Paula Broo, \textborn 02.02.1947, köpte gården 30.12.1988 av Hermans föräldrar. Vid gården hade Herman och hans son Jari en timmersvarv och tillverkade timmerbyggnader, möbler o.a. Gården har stått obebodd sedan Sanfrid och Sveas tid där. Efter Hermans död är dödsboet ägare.


\jhhousepic{177-05740.jpg}{}

\jhoccupant{Broo}{Sanfrid \& Svea}{1973--\allowbreak 1988}
Sanfrid Broo, \textborn 03.10.1910 på Gunnar, gift med Svea Jungarå, \textborn 24.09.1908 på Jungarå hemman. Sanfrid och Svea köpte denna gård av Hilding Nygård. År 1988 flyttade de till en lägenhet i Kommunalgården. Mer om Sanfrids familj under Böös 151 och Jungarå 19.

Svea \textdied 20.05.1990  --  Sanfrid \textdied 03.04.2001.

Hyresgäster 1969--\allowbreak 1972: Bertel och Marja-Liisa Julin (Romar 41) hyrde gården av  Hilding Nygård år 1969 t.o.m 22.01.1972, då de flyttade till Jakobstad.


\jhoccupant{Nygård}{Eliel \& Elin}{1954--\allowbreak 1969}
Johan Jakob Eliel Jakobsson Nygård, \textdied 12.08.1892 på Böös, gift med Elin Augustsdotter Sjö, \textdied 10.01.1892 på Böös. Uppgifter om Eliels familj under Böös, gård nr 139.

Eliel byggde denna gård år 1953 med tanke på någon av sönerna. Gården planerades ursprungligen med två ingångar och var tänkt för två familjer. År 1954, när hemgården revs och Valfrid byggde ny gård på samma ställe, bodde hela familjen här. Ingen av Valfrids syskon blev kvar och gården kom sedan att bli Eliels och Elins stadigvarande bostad. Efter deras död kom gården i sonen Hildings ägo. Han hyrde ut den till Bertel och Marja-Liisa Julin och senare sålde han den till Sanfrid och Svea Broo.

Eliel \textdied 09.12.1968  --  Elin \textdied 20.10.1965



\jhhouse{Rögårds}{6:118}{Böös}{12}{42, 42a}


\jhhousepic{176-05741}{}

\jhoccupant{Anderson \& Rögård}{Kaj \& Linda}{2010--}
Kaj Anderson, \textborn 1981 i Jeppo, sambo med Linda Rögård, \textborn 1985, från Pensala.

Barn: Klara, \textborn 23.10.2012

Kaj har arbetat som processkötare vid KWH Mirka, men sedan 2014  är	han pälsdjursfarmare på heltid. Linda arbetar som sjukskötare vid Nykarleby sjukhem. På lägenheten, som omfattar 5000 m2, uppförde paret  år 2010 bostadsbyggnaden samt år 2011 ett garage. Tomten köpte de av Kennet Lindén.



\jhhouse{High Chaparall}{6:66}{Böös}{12}{43, 43a}


\jhhousepic{178-05742.jpg}{}

\jhoccupant{Wallin}{Sebastian \& Jonna}{2004--}
Sebastian Wallin, \textborn 03.10.1984 i Kenya, gifte sig 2004 med Jonna Back, \textborn 07.02.1985 på Jungar hemman, nr 23.
\begin{jhchildren}
  \item \jhperson{Liam Artur}{27.01.2008}{}
  \item \jhperson{Lucia Elisabeth}{12.12.2009}{13.12.2009}
  \item \jhperson{Leon Artur}{08.03.2011}{}
\end{jhchildren}

Sebastian är barn till missionärerna Alf och Mona Wallin och därför	uppvuxen i Kenya. Han kom som 9-åring till Finland. Utexaminerad fordonsmekaniker år 2003. Sebastian arbetar på Mirka samt har en verkstad på sidan om arbetet. Den egna firman startade han 2004.

Jonna blev utexaminerad kosmetolog/fotvårdare 2004. Hon har haft egen skönhetssalong, som upphörde 2010. Hon har arbetat på Norrvalla Rehabcenter, men sedan februari 2015 har hon tjänst som fotvårdare på Nykarleby sjukhem. Sebastian och Jonna köpte huset i juni 2004  av Terho Ekola.


\jhoccupant{Ekola}{Terho \& Gun-Lis}{1982--\allowbreak 2004}
Terho Ekola, \textborn 26.08.1943 på Ekola hemman i Alahärmä, gift med Gun-Lis Jakobsson, \textborn 19.04.1947 på Tålamods i Vörå.
\begin{jhchildren}
  \item \jhperson{Rita}{16.01.1967}{}
  \item \jhperson{Margita}{17.01.1970}{}
  \item \jhperson{Lisa}{18.03.1985}{}
\end{jhchildren}
GunLis och Terho köpte huset år 1982 och flyttade in på nyårsafton. Terho har varit egen företagare och skött om mjölktransport till mejerier. Gun-Lis är utbildad sömmerska och sydde enligt beställning i hemmet. Kurserna som hon höll i olje- och porslinsmålning vid arbetarinstitutet upptog så småningom allt mer av hennes tid. Gun-Lis flyttade till Sverige och Terho sålde gården år 2004.


\jhoccupant{Sunngren}{Karl \& Linnea}{1962--\allowbreak 1982}
Karl Johansson Salmela, senare med efternamnet Sunngren, \textborn 14.02.1916 i Kimo by i Oravais. Han var gift med Linnea Sandås, \textborn 16.10.1914 på Böös hemman.
\begin{jhchildren}
  \item \jhperson{Göran}{29.03.1941}{}
  \item \jhperson{Ralf}{02.09.1942 (adopterad av Emil och Juliana Elenius)}{}
  \item \jhperson{Maj-Len}{30.04.1944}{}, gift Mattfolk
  \item \jhperson{Ulla-Britt}{29.05.1947}{}, gift Ahlbäck
  \item \jhperson{Lars-Ole}{28.12.1949}{}
  \item \jhperson{Tage}{08.06.1957}{}
\end{jhchildren}
Linnea var utbildad mejerska och Karl var bl.a i järnvägsarbete. År 1962 övertog de Juliana och Emil Elenius lägenhet på Böös. Enligt boken Svenska Österbottens bebyggelse från 1965 fanns på gården 1 häst, 12 kor, 10 ungdjur, 1 svin samt traktordrift. Arealen var 45 ha, varav odlad jord 18,12 ha. Bostadsbyggnaden uppfördes 1962 men har renoverats av senare ägare. År 1982 såldes gården till Terho och GunLis Ekola.



---> \jhbold{Bösas}	Böös hemman, Karta 12, nr \jhbold{143}

\jhoccupant{Elenius}{Emil \& Juliana}{1913--\allowbreak 1962}
Emil Isaksson Elenius, \textborn 24.07.1894 på Ruotsala hemman, gift med Juliana Sandberg, \textborn 30.04.1893 på Finskas hemman. Året innan giftermålet hade Emils far köpt Olivia och Johan Elenius lägenhet på Böös. Lägenheten, som omfattade 1/12 dels mantal, fick	paret nu överta. De byggde upp ett mönsterhemman med högklassig djurbesättning och fick som erkännande motta Lantbrukssällskapets pris. Emil var med vid grundandet av Jungar Andelsmejeri, vars disponent och kassör han var i många år. Han var också medlem i kommunfullmäktige, ordörande i skattenämnden samt ledamot i andra nämnder och kommittéer.

Makarna var starka pådrivare av en evangelisk folkhögskola. De fick inga egna barn, men adopterade Ralf Sunngren år 1947. Ralf omkom i en trafikolycka 16.10.1961. Ralfs föräldrar övertog lägenheten medan Emil och Juliana behöll för egen del två rum i mangården.

Emil \textborn 02.01.1974  --  Juliana \textdied 20.12.1978


\jhoccupant{Elenius}{Johan \& Olivia}{1909--\allowbreak 1912}
Johan Mattsson, senare med efternamnet Elenius,  \textborn 06.04.1890 på Gunnar hemman, gift med Olivia Slangar, \textborn 10.07.1888 på Slangar.
\begin{jhchildren}
  \item \jhperson{Dagny}{26.10.1910}{17.11.1924}
  \item \jhperson{Gerda}{26.05.1912}{}, gift med Bertel Sarelin
\end{jhchildren}
År 1909 köpte Johan och Olivia en 1/12 mantals lägenhet på Böös av barnlösa Maria och Anders Böös. Det blev dock ingen framtid för dem som jordbrukare. Lungsoten angrep deras liv så att de blev tvungna att redan 1912 sälja lägenheten. Familjen flyttade till Olivias hem.

Johan \textdied 19.04.1914  --  Olivia \textdied 24.07.1914. Olivias föräldrar tog hand om de föräldralösa barnen.


\jhoccupant{Jakobsson}{Anders \& Maria}{1883--\allowbreak 1909}
Bn Anders Jakobsson Böös, \textborn 25.05.1856 i Jeppo, gift med  Maria Isaksdotter, \textborn 23.06.1857.

Barn: Emmy Nylund, \textborn 20.12.1897, gift med Leonard Liljeqvist

Anders och Maria hade inga egna barn, men fick förmånen att ta hand om Emmy, då hon var 4 år. Troligen adopterade de henne senare, eftersom de i handlingar skriver ``till vår adoptivdotter..''.

Bonden Anders Jakobsson Böös erhöll år 1883 genom arvsförening hälften av  hemmanet. Hans far, Jakob Andersson, hade muntligt på sin dödsbädd önskat att hemmanet skulle delas mellan Anders och systern Anna Sanna Jakobsdotter. År 1886 kom Anders att få hela hemmanet, då han på auktion köpte andra häften av systern samt hennes man, Johan Andersson Böös. Anna Sanna Jakobsdotter Böös eller Östman ansökte 1887 om ett arrendekontrakts inteckning i Anders Jakobsson Böös hemman.

År 1909 sålde Anders och Maria en del av hemmanet till Johan och Olivia Elenius, år 1922 sålde de resten av hemmanet till Emmy och Leonard Liljeqvist.

Maria bodde till sin död i ena ändan av gården. Hon var känd som Kock-Mari och var ofta huvudkock vid bröllop.
Anders \textdied 07.02.1925  --  Maria \textdied 15.07.1938


\jhoccupant{Andersson}{Jacob \& Anna}{1869--\allowbreak 1883}
F.d torparen Jacob Andersson Löfgren, \textborn 13.08.1827 på Böös, vigd 9 maj 1852 med Anna Sanna Johansdotter Finskas, \textborn 11.08.1828.
\begin{jhchildren}
  \item \jhperson{Johan Jakob}{10.12.1852}{1853}
  \item \jhperson{Maria}{11.03.1854}{1856}
  \item \jhperson{\jhbold{Anders}}{25.05.1856}{}
  \item \jhperson{Sanna Lisa}{22.12.1858}{}
  \item \jhperson{Anna Sanna}{23.04.1861}{}
  \item \jhperson{Maria Lovisa}{30.08.1865}{}
  \item \jhperson{Johanna}{22.11.1867}{}, gift med Johan Johansson Jungarå
  \item \jhperson{Isak}{28.11.1872}{}
\end{jhchildren}
Jacob Andersson är skriven som torpare på Böös under åren 1850--\allowbreak 1870. 8 augusti 1869 köpte han detta 5/24 mantals hemman av Finlands Hypoteksförening. Hemmanet hade ägts av Jakob Jakobsson Böös. Redan år 1872 sålde han hälften till Gustaf Böös (gård nr 44). Den andra hälften stannade i Jakob Böös ägo. Jacob \textdied 07.07.1882. Han hade kort före sin död meddelat sin yttersta vilja angående hemmanet. Enligt bouppteckningen var hemmanet 5/48 mantal och det fanns 2 hästar, 6 kor, 2 kalvar, 4 får och 1 lamm på gården. Anna Sanna \textdied 11.05.1896.


\jhoccupant{Böös}{Jakob \& Sanna}{1843--\allowbreak 1869}
Jakob Jakobsson Böös, \textborn 03.04.1821 på Böös, vigd 25.06.1843 med Sanna Lisa Elenius, \textborn 08.07.1823 på Tollikko hemman. Jakobs syster Sanna Lena gifte sig samtidigt med Anders Jakobsson Jungar.
\begin{jhchildren}
  \item \jhperson{Johan Jakob}{10.02.1844}{}, jordbrukare på Kaukos
  \item \jhperson{Maria}{01.02.1847}{}, g. Lindström (Silvast 78)
  \item \jhperson{Albertina}{01.11.1850}{}
  \item \jhperson{Isak}{26.08.1853}{}, till USA
  \item \jhperson{Anna Susanna}{03.06.1856}{28.10.1856}
  \item \jhperson{Vilhelm}{07.02.1858}{20.03.1858}
  \item \jhperson{Sofia}{06.12.1860}{}, till USA
\end{jhchildren}
Jakob mottog av sina föräldrar ett 5/24 dels mantals hemman på Böös. Han klarade dock inte av att sköta räntor och avbetalningar på lånet och år 1869 såldes hemmanet på exekutiv auktion. Sanna Lisa dog 04.05.1863. På 1870-talet var Jakob skriven bland lösa personer i Jungar by. Från år 1880 bodde han hos dottern Maria, som var bondhustru på Silvast.

\jhoccupant{Böös/Andersson}{Jakob \& Maria}{1810--\allowbreak 1843}
Jakob Andersson,  \textborn 25.12.1785 på Böös, gift med Maria Eriksdotter, \textborn 18.09.1791.
\begin{jhchildren}
  \item \jhperson{Anna}{04.09.1811}{}
  \item \jhperson{Maja Lovisa}{20.01.1817}{1818}
  \item \jhperson{\jhbold{Jacob}}{03.04.1821}{}
  \item \jhperson{Sanna Lena}{14.03.1823}{}
  \item \jhperson{Caisa Lisa}{18.11.1826}{29.09.1827}
  \item \jhperson{Anders}{14.12.1828}{07.09.1830}
\end{jhchildren}
År 1810 finns 4 hemman på Böös: Isak Gustavsson 5/24, Henrik Andersson 5/48, Isak Andersson 11/48 och Jakob Andersson 5/24.



\jhhouse{Södergård}{6:56}{Böös}{12}{44, 44a-b}


\jhoccupant{Cederström}{Håkan \& Lenelis}{2006--}
Håkan Cederström,  \textborn 07.04.1951 i Ytterjeppo samt hustrun Lenelis Lindgren, \textborn 21.02.1955 på Jungar, köpte huset 30.12.2006 av Niclas Sandås.\jhvspace{}

\jhoccupant{Sandås Niclas}{Raul \& Boris}{1995--\allowbreak 2006 och 1985--}
Boris och Raul Sandås köpte gården 8 augusti 1985 av Arne och Vivi Södergårds barn Jan-Erik och Kristina. Hemmanet hade sålts 1970. Raul blev senare ensam ägare till gården och efter att Niclas övertagit hemmanet 1995, blev han också ägare till denna gård.\jhvspace{}


\jhhousepic{179-05743.jpg}{}

\jhoccupant{Södergård}{Arne \& Vivi}{1945--\allowbreak 1985}
Arne Södergård, \textborn 01.12.1917 på Böös hemman, gift med Vivi Blomqvist, \textborn 20.06.1921 på Pått hemman i Pensala by.
\begin{jhchildren}
  \item \jhperson{Jan-Erik}{01.02.1943}{19.04.2013}, till Åland
  \item \jhperson{Kristina}{19.03.1947}{}, gift 1 Mäkelä, 2 Knuts i Munsala
  \item \jhperson{Nils}{25.10.1951}{24.01.1985}
\end{jhchildren}

Gården har tillhört samma släkt sedan 1800-talet och övertogs av Arne år 1945. Arealen var 54,94 ha, varav odlad jord 15,14 ha. På mitten av 1960-talet hade man 500 höns. Bostadsbyggnaden  uppfördes 1910 av Arnes far Anders. Den är timrad och har plåttak. År 1948 uppförde Arne ekonomibyggnaden av tegel och trä under pärttak.

Arne \textdied 05.03.1962  --  Vivi \textdied 05.05.1983.


\jhoccupant{Södergård}{Anders \& Anna L}{1900--\allowbreak 1945}
Anders Gustafsson, senare med efternamnet Södergård, \textborn 10.02.1871 på Böös hemman, gift med Anna Lovisa Andersdotter, \textborn 08.11.1873 på Grötas hemman.
\begin{jhchildren}
  \item \jhperson{Ingrid}{21.07.1901}{10.10.1973}, gift Knuts i Ytterjeppo
  \item \jhperson{Anna Viola}{10.10.1903}{30.05.1906}
  \item \jhperson{Anders Joel}{22.11.1905}{26.04.1923}
  \item \jhperson{Elma}{01.03.1908}{26.01.1939 (Mietala 118)}, gift Elenius
  \item \jhperson{Emilia}{07.07.1911}{09.09.2001 (Silvast 101)}, gift Löv
  \item \jhperson{\jhbold{Arne}}{01.12.1917}{}
\end{jhchildren}
Anders uteblev från militäruppbådet och flydde till USA, där han arbetade i skogen. När han år 1900 återvände till Finland hade han sparat ihop så mycket pengar att han kunde köpa gården kontant av sin broder Johan Jakob. Anders uppförde en ny bostadsbyggnad år 1910. Sannolikt har det funnits en äldre gård på lägenheten, som Anders far byggt eller ev. flyttat torpbyggnaden (Grötas 116), där hans föräldrar bott. Äldre personer på Böös har berättat att gården tidigare varit närmare landsvägen, men att den flyttats till nuvarande plats och byggts till. Om detta skett år 1910 eller om detta gäller den gård som äldre generationer bott i är oklart.

Anders \textdied 22.01.1958  --  Anna Lovisa \textdied 26.07.1937


\jhoccupant{Böös/Enkvist}{Johan \& Maria}{1892--\allowbreak 1900}
Johan Jakob Gustafsson Böös, senare med efternamnet Enkvist, \textborn 	22.08.1866 på Skog, gift med Maria Gustavsdotter Bjon, \textborn 12.05.1875 i Nykarleby.
\begin{jhchildren}
  \item \jhperson{Anders Gustav}{06.08.1893}{07.11.1893}
  \item \jhperson{Jenny Elvira}{25.09.1894}{19.02.1969}, gift Sigfrids
\end{jhchildren}
Johan Jakob erhöll år 1892 av sina föräldrar deras lägenhet , som omfattade 5/48 dels mantal. Året därpå köpte han av Matts Anders Gustafsson Jungarå 5/72 dels mantal (hälften av hans hemman). Med köpet följde sytningsavtal gällande änkan Anna Maria Andersdotter, tidigare ägare till Matts Jungarås hemman. Johan emigrerade till Amerika 1904. Familjen återkom senare till hemlandet och bosatte sig i Nykarleby.


\jhoccupant{Jakobsson Böös}{Gustav \& Maria}{1872--\allowbreak 1892}
Torparsonen Gustaf Jakobsson Böös, \textborn 12.04.1841 på Böös, gift 1866 med Maria Johansdotter Skog, \textborn 13.08.1841 på Skog. Barn födda, första på Skog, de senare på Böös:
\begin{jhchildren}
  \item \jhperson{\jhbold{Johan Jakob}}{22.08.1866}{}
  \item \jhperson{Sanna Sofia}{20.01.1869}{10.12.1948}, till USA, g. Blomqvist
  \item \jhperson{\jhbold{Anders Gustav}}{10.02.1871}{}
  \item \jhperson{Emil}{03.01.1873, (Böös 144 )}{}
  \item \jhperson{Anna Lovisa}{04.02.1875}{6.9.1957}, till USA, g. Bloomquist
  \item \jhperson{Axel}{23.07.1877}{13.9.1959}, (Enqvist) till Nykarleby
  \item \jhperson{Fredrik}{18.06.1879}{}, till USA, rånmördad på hemresan
  \item \jhperson{Maria Johanna}{18.12.1885}{12.09.1951 i Jakobstad}, gift Bergvall
\end{jhchildren}
Gustaf Böös	 hade år 1872 köpt en 5/48 mantals lägenhet på Böös. Denne överlämnade han 28.11.1892 till äldsta sonen Johan Jakob, som			dock drygt 7 år senare sålde lägenheten till brodern Anders. Gustaf hade köpt hälften av Jakob Andersson Böös hemman. Jakob behöll halva hemmanet i egen ägo och blev således kvar i mangården. Se närmare Böös nr 143.

Gustaf Böös \textdied 24.11.1930  --  Maria \textdied 19.04.1886


---> \jhbold{Böös/torp} på Böös hemman, Karta 12, nr \jhbold{125}

Enligt uppgifter skulle Gustaf Böös´ sytningsstuga funnits här.



\jhhouse{Sandås}{6:54}{Böös}{12}{45, 45a-c}


\jhhousepic{180-05744.jpg}{}

\jhoccupant{Pik-UP}{Ab}{2013--}
Aktiebolaget Jeppo Pik-UP Ab köpte Sandås lägenhet år 2013. Niklas bodde kvar i huset till år 2014, då han flyttade till Nykarleby. Hans Kronlund, som är delägare i företaget, flyttade i juni 2015 med sin familj till lägenheten. Hans Kronlund, \textborn 22.05.1958 på Gunnar hemman, sambo med Gun-Helen Häggback, \textborn 26.03.1973 i Malax.
\begin{jhchildren}
  \item \jhperson{Jako}{30.01.2010}{}
  \item \jhperson{Elias}{26.08.2011}{}
  \item \jhperson{Hanna}{28.01.2014}{}
\end{jhchildren}
Närmare om Hans Kronlund under Gunnar, gård nr 21.


\jhoccupant{Sandås}{Niclas}{1995--\allowbreak 2013}
Niclas Sandås,  \textborn 05.04.1971 på Böös, sambo med Åsa Kulla, \textborn 1973 i Oravais.

Barn: Emma, \textborn 10.09.1997.

Samboförhållandet upphörde och Niclas gifte sig år 2004 med Ingela Källbacka, född i Sverige. Frånskild. Niclas övertog jordbruket efter sin far år 1995. Han sålde  skogen till Bo-Göran Finskas, jorden till Kennet Lindén. Gården såldes år 2013.


\jhoccupant{Sandås}{Raul \& Greta}{1964--\allowbreak 1995}
Raul Sandås, \textborn 03.09.1939 på Böös, gift med Greta Eklund, \textborn 01.11.1934 på Stenbacka.
\begin{jhchildren}
  \item \jhperson{Camilla}{13.11.1968}{}
  \item \jhperson{\jhbold{Niclas}}{05.04.1971}{}
\end{jhchildren}
På grund av sin fars död kom Raul att vid unga år överta ansvaret för hemgården. Han utvecklade och förstorade sitt jordbruk och var en av de större mjölkproducenterna i Jeppo. Han specialiserade sig på spannmålsodling och mjölkproduktion och hade i början på 80-talet ca 30 mjölkkor. Höbärgningen slutade de med redan på 70-talet. Hela vallarealen skördades som AIV/ensilage. I mitten på 80-talet övergick han till uppfödning av slaktdjur.

År 1965 byggde Raul en ny egnahemsgård, där också ingick en liten lägenhet för mamma Hjördis. Han hade flera förtroendeuppdrag, bl.a. som ledamot i ÖSP:s styrelse och nämndeman vid jorddomstolen. Greta var före giftermålet butiksbiträde på Jeppo-Oravais Handelslag. Greta \textdied 28.01.1984. Raul gifte om sig år 1987 med Helena Westermark, \textborn 27.06.1938 i Korsnäs, men giftermålet slutade i skilsmässa.

Raul \textdied 03.10.2003.


---> \jhbold{Sandås} på Böös hemman  		Karta 12,  nr \jhbold{145}


\jhoccupant{Sandås}{Edvin \& Hjördis}{1936--\allowbreak 1964}
Edvin Sandås, \textborn 07.12.1911 på Böös, gifte sig 14.06.1936 med Hjördis	Finskas, \textborn 10.04.1912 på Finskas hemman.
\begin{jhchildren}
  \item \jhperson{Ragnborg}{27.03.1937}{}, gift Nygård (Silvast  107)
  \item \jhperson{\jhbold{Raul}}{03.09.1939}{}
  \item \jhperson{Boris}{18.11.1945}{}, tekniker, bor i Kvevlax
\end{jhchildren}
Edvin var endast 10 år när fadern dog i ``miinasjukon'' och dessutom äldsta sonen. Han kom följaktligen tidigt att ta ansvar för hemmanet på Böös. Hans stora mål och intresse var att få det förstorat. Genom köp och nyodling ökade han den odlade arealen från ca 12 till 25 ha odlingsmark. År 1939 sålde Sanna Sandås och Edvins syskon 6/72 dels mantal till Edvin och Hjördis.

Edvin blev inkallad i militärtjänst 16.03.1940 och var vid den s.k. Hangöfronten då Hangöudd befriades från ryssarna. Han blev hemförlovad 25.09.1944. Under dessa år sköttes hemmanet av hustrun Hjördis. Edvin hann också ta del i bygdens värv. Han var ordförande i styrelsen för Jeppo Andelsmejeri några år samt aktiv inom grävmaskinsfirman Jeppo Gräv, bl.a som direktör.

Edvin \textdied 05.10.1964  --  Hjördis \textdied 26.10.2007


\jhhousepic{Boos145-EmilSandas (1).jpg}{}

\jhoccupant{Sandås}{Emil \& Sanna}{1906--\allowbreak 1936}
Emil Gustafsson, senare med efternamnet Sandås, \textborn 03.01.1873 på Böös, gifte sig 29.07.1906 med Sanna Andersdotter Grötas, \textborn 18.05.1879 på Grötas.
\begin{jhchildren}
  \item \jhperson{Ester}{16.05.1907}{}, ogift
  \item \jhperson{Edith}{05.05.1909}{}, gift Smeds i Åvist
  \item \jhperson{\jhbold{Edvin}}{07.12.1911}{}
  \item \jhperson{Linnea}{16.10.1914}{}, gift Sunngren (gård nr 43)
  \item \jhperson{Agnes}{19.04.1918}{}, gift Elenius (Tollikko 8)
  \item \jhperson{Svea}{09.05.1921}{16.06.1922}
\end{jhchildren}

Bröderna Anders och Emil  uteblev från uppbådet och reste till USA. När brodern Anders återvände år 1900 köpte han ett hemman om 5/48 mtl. Emil återvände år 1906 och gifte sig med Sanna, som var syster till brodern Anders hustru. Emil köpte av brodern 1/3 av hans hemman, dvs 5/72 mtl av Böös. Detta bestod av 24 ha skogsmark och 12 ha odlingsmark. Troligen är det Emil som byggde gårds- och 				ekonomibyggnaderna på lägenheten. Brodern Anders bodde med sin familj i hemgården.

Emil tog ut flyttningsbevis en andra eller tredje gång 15.03.1915, då han troligtvis reste till Sydafrika för att jobba i guldgruvor. Han hade då 4 minderåriga barn. Emil var några år ledamot i styrelsen för Jungar andelsmejeri. Han dog 04.12.1922 i sviterna av ``miinasjukan''. Sanna dog 13.04.1942.

Jeppobor, som jobbade i Sydafrika under denna tid, samlade in pengar till Jeppo altartavla. Emils namn finns nedtecknat på baksidan av altartavlan.



\jhhouse{Ekbacka}{5:22}{Grötas (obs!)}{12}{46, 46a}


\jhoccupant{Mietas}{Jordbrukssam.}{2006--}
Per-Erik Forsgård ( Mietala, nr 11) köpte lägenheten år 2006. Idag ägs den av Mietas Jordbrukssammanslutning, dvs Per-Erik äger hälften och Per-Eriks söner hälften. Gården är obebodd.\jhvspace{}


\jhoccupant{Sandlin}{Stefan}{2004--\allowbreak 2006}
Stefan Sandlin, \textborn 28.08.1951 på Böös, gift med Anna-Maija Liesto, \textborn 25.10.1945 i Jyväskylä. Stefan har arbetat sedan 1984 som forskare vid Teknologiska Forskningscentralen VTT AB (tidigare Statens tekniska forskningscentral VTT) i Otnäs/Esbo och gick i pension den 01.09.2016. Anna-Maija har en lägre rättsexamen från Helsingfors universitet. Hon har arbetat som sekreterare vid Finansministeriet, senare som arbetsmarknadssakunnig vid statens arbetsmarknadsverk vid samma ministerium. Stefan och Anna-Maija är bosatta i Esbo.


\jhhousepic{185-05751.jpg}{}

\jhoccupant{Sandlin}{Emil \& Hjördis}{1944--\allowbreak 2004}
Emil Sandlin, \textborn 05.05.1914 på Grötas, gift med Alina Aino Åkerholm, \textborn 26.07.1919 i Korsholm. Äktenskapet slutade i skilsmässa. Emil ingick nytt äktenskap med Hjördis Källman, \textborn 08.04.1915 på Ruotsala.

Barn med Aino: Marita Viola, \textborn 08.10.1945. -- Barn med Hjördis: \jhbold{Stefan}, \textborn 28.08.1951.

År 1944 övertog Emil den andra halvan av sina föräldrars lägenhet på Grötas hemman. Emils far Jakob Sandlin (Grötas 27) bygggde huset 			 år 1936.

Emil \textdied 07.02.2004  --  Hjördis \textdied 01.05.2001


\jhbold{Hyresgäst åren 1936 – 1944:}
Huset hyrdes ut tills sonen Emil gifte sig och flyttade in: ``Jott'' Johannes Jungerstam, \textborn 28.08.1901 på Jungar, hyrde huset åren 1936--\allowbreak 1944. Johannes var agrolog till utbildningen. I Jeppo kommer man mest ihåg honom som lantbruksklubbledare samt arbetsledare vid iordningsställandet av den nya begravningsplatsen. Han kom aldrig att få någon fast tjänst, men hade många tillfälliga tjänster och uppdrag, bl.a var han timlärare vid Evangeliska Folkhögskolan på Keppo gård läsåren 1933-36. Under och efter kriget gav han licenser på spik och cement. Han var godeman vid lantmäteriförättningar samt ordförande i kolonisationsnämnden. Han höll otaliga föredrag i olika lantbruksfrågor runt om i Svenskfinland.

Johannes bodde därefter på Jungar, i Silvast samt Vasa. Han dog 04.02.1965.



\jhhouse{Sjö}{6:83}{Böös}{12}{47, 47a samt 48}


\jhhousepic{182-05748.jpg}{Hus nr 47}

\jhoccupant{Sjö}{Hugo \& Saima}{1938--\allowbreak 1980}
Hugo Sjö, \textborn 03.06.1905 på Böös hemman, gift med Saima Jungell, \textborn 22.12.1908 på Tollicko hemman.

Barn: Margot, \textborn 22.07.1938

År 1937 övertog Saima och Hugo en tredjedel av hans föräldrars lägenhet på Böös. Den omfattade då 44 ha, varav 11 ha var odlad jord. Bostadsbyggnaden uppfördes år 1938 av bröderna Sjö, den renoverades år 1970. Ekonomibyggnaden uppfördes 1940--\allowbreak 1941 av bröderna Sandvik. Hugo hade under många år styrelseuppdrag i Jungar Andelsmejeri.

Hugo \textdied 04.08.1985  --  Saima \textdied 11.03.2002


\jhhousepic{181-05746.jpg}{Hus nr 48}

\jhoccupant{Rausk}{Margot \& Raoul}{1979--}
Margot Sjö, \textborn 22.07.1938 på Böös hemman, gifte sig 01.08.1980 med Raoul Rausk, \textborn 04.05.1933 från Österby i Oravais. År 1979 övertog Margot och Raoul sina föräldrars lägenheter på Böös och i Österby. På Böös hemman finns 11 ha jord och 18 ha skog. Bostadshuset byggdes på samma tomt som Margots föräldrahem (gård nr 47). Huset uppfördes av Heikius Hus år 1982.\jhvspace{}



\jhhouse{Sjö}{6:78}{Böös}{12}{49, 49a}


\jhoccupant{Kronlund}{Erik \& Ebba}{1998--}
Erik Kronlund, \textborn 12.05.1945 i Vasa, gift med Ebba Nylund, \textborn 25.02.1944 i Nykarleby lk. Erik och Ebba är ägare till Verner Sjös hus sedan 05.08.1998, då de köpte huset av Marita Granskog och Johan Felix Söderlunds dödsbo. Huset används som lager.\jhvspace{}


\jhhousepic{186-05752.jpg}{}

\jhoccupant{Granroth/}{Söderlund}{- 1998}
Lägenheten såldes efter Verners bortgång. Huset såldes till Marita Granroth, \textborn 1939 och Johan Felix Söderlund. Johan Felix Söderlund var född 1913, dog 1997. Paret bodde aldrig i huset. Marita sålde huset efter att Felix dött.\jhvspace{}

\jhoccupant{Sjö}{Verner \& Hjördis}{1937--\allowbreak 1982}
Verner Sjö,  \textborn 05.05.1900 på Böös, gifte sig med Hjördis Lövmark, \textborn 08.02.1917 på Löv hemman i Nykarleby lk. Barn: Alf Johannes, \textborn 21.02.1953--\textdied 21.02.1953.

Verner var några år i Kanada. När han kom hem övertog han en tredjedel av föräldrarnas lägenhet på Böös. Troligen byggde han själv med brödernas hjälp sitt hus. Som modell för huset hade han J.A. Jungarås hus vid Prästas. Vid sidan av sitt jordbruk på 40 ha, varav 10 ha odlad jord, var Verner arbetschef vid Jeppo skogsandelslag sedan starten 1931 till år 1960. Han ansvarade för ledningen av andelslagets sågrörelse åren 1958--\allowbreak 1965. Han hade också förtroendeuppdrag i lantmannagillet, Jungar andelsmejeri, jaktvårdsföreningen samt som sparbanksprincipal.

Hjördis \textdied 09.06.1958  --  Verner \textdied 20.04.1982



\jhhouse{Annaslund}{6:95}{Böös}{12}{50, 50a}


\jhhousepic{184-05749.jpg}{}

\jhoccupant{Strengell}{Glenn \& Christina}{2008--}
Glenn och Christina Strengell köpte Sjös gård år 2008 av Birger och Anna Sjös ättlingar. Gården renoverades och har använts som
inkvarteringsstuga sedan 2013 i Jeppo Gäststugors regi. Den kallas ``Annahuset''. Folk från hela världen i behov av kortare eller längre vistelse i nejden har inkvarterats på Böösas.

Vintertid har Christina medicinsk fotvårdsmottagning i huset. Själva hemmanet har idag flere ägare. Glenn Strengells familjeuppgifter under Jungar 3.


\jhoccupant{Sjö}{Birger \& Anna}{1969--}
Birger Sjö, \textborn 25.09.1915 på Böös hemman, gifte sig 11.12.1938 med Anna Julin, \textborn 21.05.1919 på Mjölnar hemman. Uppgifter om Birger och Anna Sjö under gård nr 150, där familjen bodde de första 30 åren. År 1969 byggde Birger denna gård och familjen flyttade in. Detta torde ha varit det första elemenhuset, som uppfördes i Jeppo.

Birger \textdied 04.04.1988  --  Anna \textdied 18.10.2007


---> \jhbold{Sjö} på Böös hemman	Karta 12,  nr \jhbold{150}


\jhhousepic{Boos150-BirgerSjo-.jpeg}{}

\jhoccupant{Sjö}{Birger \& Anna}{1938--}
Birger Sjö, \textborn 25.09.1915 på Böös hemman, gifte sig 11.12.1938 med Anna Julin, \textborn 21.05.1919 på Mjölnar hemman.
\begin{jhchildren}
  \item \jhperson{Tore}{02.04.1940}{}, ingenjör, till Sverige
  \item \jhperson{Torhild}{02.12.1943}{}, barnsköterska, gift Wikar
  \item \jhperson{Maj-Britt}{13.01.1946}{22.01.2016}, merkonom, gift Pöyhönen
  \item \jhperson{Maggi}{12.09.1950}{}, privatföretagare, gift Söderlund
  \item \jhperson{Gerd}{18.10.1953}{15.07.1987}, merkonom, gift Snåre
\end{jhchildren}

Birger övertog 1/3 av hemgården på Böös. Den omfattade 36 ha. I mitten på 1960-talet hade de 12 kor och 8 ungdjur. Birger var aktivt med i närsamhället och innehade många förtroendeuppdrag. Bostadsbyggnaden var uppförd av Birgers farfar Johan. År 1969 byggde Birger ny bostad åt familjen. Ekonomibyggnaden är byggd år 1952 i trä och sten. Kilstenarna från den gamla fähusbyggnaden finns kvar. Där är inristat årtalet 1896.

Birger \textdied 04.04.1988  --  Anna \textdied 18.10.2007


\jhoccupant{Sjö}{Hugo \& Saima}{1934--\allowbreak 1938}
Sjö Hugo, \textborn 03.06.1905 på Böös hemman, gift med Saima Jungell, \textborn 22.12.1908 på Tollicko hemman. Saima och Hugo bodde de första åren i Hugos föräldrahem. Mera om Hugo och Saima under Böös 47.


\jhoccupant{Sjö}{August \& Katarina}{1897--\allowbreak 1938}
August Johansson Sjö, \textborn 16.12.1870 på Böös hemman, gift med Katarina Mattsdotter, \textborn 26.01.1871 på Gunnar hemman. Katarinas bror Isak skriver i sin dagbok från 1891: ``Katarina ymsade till Sjöas den 28 Februari, hon hade sex lass, två koddor, fyra får och ett lamm, fem tunnur råg, en soffa, två stolar, ett bord, en birong, en väfstol och en mjölksjula af bleck och en toalett.''
\begin{jhchildren}
  \item \jhperson{Elin}{10.01.1892}{}, gift Nygård (Böös 139)
  \item \jhperson{\jhbold{Verner}}{05.05.1900}{}, (Böös 49)
  \item \jhperson{Aurora}{09.08.1901}{}, till Jakobstad
  \item \jhperson{Saima Irene}{06.06.1903}{18.10.1903}
  \item \jhperson{\jhbold{Hugo}}{03.06.1905}{}, (Böös 47)
  \item \jhperson{Ester}{03.08.1907}{19.09.1922}
  \item \jhperson{Edit}{27.04.1909}{28.07.1958}
  \item \jhperson{\jhbold{Lennart}}{16.04.1911}{}, mejerist (Böös 130)
  \item \jhperson{Jenny}{02.01.1913}{}, gift Åman
  \item \jhperson{Signe}{16.05.1914}{}, gift Dahl
  \item \jhperson{\jhbold{Birger}}{25.09.1915}{}
\end{jhchildren}
Steg för steg kom Katarina och August att bli ägare till sitt relativt stora jordbruk på Böös hemman. 1893 köpte de Matts Jungarås 5/72 mantals lägenhet. August var åren 1894-99 på arbetsförtjänst i USA . Åren 1897, 1902 och 1904 blev de av olika omständigheter ägare av hans föräldrars lägenhet, som omfattade 120 ha.

August Sjö var ledamot i Jeppo lantmannagille samt dess föreståndare ett antal år. Dessutom tillhörde han styrelsen för Jeppo kvarn och  handelslaget.

August \textdied 22.04.1921  --  Katarina \textdied 23.10.1944


\jhoccupant{Hansson/Sjö}{Johan \& Kaisa}{1880--\allowbreak 1897}
F.d. Skarpskytten Johan Hansson, senare Silvast, senare Sjö, \textborn 10.08.1834 i Terjärv, gift 27.02.1855 med bndr Anna Brita Gustafsdr Nybyggar, \textborn 16.02.1836, \textdied 15.01.1859. Johan gifte om sig 16.02.1862 med bndr Kaisa Andersdotter Skog, \textborn 23.06.1838, från Jungar by.

Barn i första giftet: Maria Sofia Silvast, \textborn 06.09.1855
I andra giftet:
\begin{jhchildren}
  \item \jhperson{Johannes}{17.09.1863}{04.10.1863}
  \item \jhperson{Johannes}{05.12.1864}{}, till Amerika
  \item \jhperson{\jhbold{August}}{16.12.1870}{}
\end{jhchildren}
Johan och Katrina/Caisa var skrivna på Lussi, Skog och Nybyggar de första åren . År 1880 köpte de av Karl Henriksson Svahnbäck en 5/36 			dels mantals lägenhet på Böös. Svahnbäck sålde hälften av sitt hemman till Johan, hälften till Johan Mattsson Ryss (uppgifter om Ryss under Böös gård nr 132). I avtalet med Johan Sjö står att södra boningsstugan medföljer samt hälften av hemmanets hus och byggnader. Rian på södra sidan om mangården skulle dock ensam tillhöra Johan Sjö, rian på norra sidan om mangården skulle tillhöra Johan Mattsson. Dessa två kom nu att tillsammans äga 1/3 av Böös skattehemman.

Johan byggde en ny, ståtliga bondgård. Johan Sjös namn förekommer ofta då det gällde kommunens representanter till möten med andra kommuner. I gåvobrevet, där han överlät 5/54 dels mantal till sin sonhustru Katarina år 1897, finns flere villkor, som var ovanliga på den tiden, bl.a skulle 500 mk ges Jeppo kommun för ``församlingens värnlösa fattiga barns uppfostran'' efter Johans död. Hans önskemål om begravningsplats var också ovanlig i.o.m att han var aktivt med då gravgården i Jeppo byggdes, men ville bli begraven på holmskiftet, som han ägde. Detta var troligen inte möjligt. Hans gravplats finns i Nykarleby.

Kaisa \textdied 12.02.1899  --  Johan \textdied 04.05.1911


\jhoccupant{Jakobsson}{Karl \& Maria}{1877--\allowbreak 1880}
Karl Jakobsson Böös/Svanbäck, \textborn 05.05.1825, gift med Maja Lisa Jakobsdotter, \textborn 03.02.1831.
\begin{jhchildren}
  \item \jhperson{Anders Alfred}{02.10.1855 i Pedersöre}{}, tvilling
  \item \jhperson{Johan Eduard}{02.10.1855 i Pedersöre}{}, tvilling
  \item \jhperson{Arvid Alexander}{04.12.1867 i Tohmajärvi}{}
  \item \jhperson{ Wendla}{22.03.1870}{}
  \item \jhperson{Oscar}{28.11.1872}{}
\end{jhchildren}
Karl inropade på auktion 17 oktober 1877 hemmanet på Böös, som tidigare ägts av änkan Sanna Johansdotter Svahnbäck. Karl sålde i sin tur hemmanet 23 mars 1880 till Karl Henriksson Svanbäck, som redan efter 7 månader sålde det vidare till Johan Sjö och Johan Jakob Mattsson Ryss. Karl Svanbäck dog 30.11.1896 som inhyses på Böös.


\jhoccupant{Svanbäck}{Johan \& Sanna}{- 1877}
Uppsyningsmannen Johan Svanbäck, \textborn 1817 i Pedersöre, gift med Sanna Helena Johansdotter, \textborn 20.02.1815 i Pedersöre.
\begin{jhchildren}
  \item \jhperson{Johan}{1836 i Pedersöre}{}
  \item \jhperson{Anna Lena}{1850 på Keppo}{}
  \item \jhperson{Maria Lovisa}{1856 på Keppo}{}
  \item \jhperson{Charlotta}{1858 på Keppo}{}
\end{jhchildren}
Då Johan köpte hemmanet på Böös, arbetade han vid Keppo såg. Efter mannens död sålde Sanna hemmanet genom frivillig auktion. Denna hemmansdel är en del av det hemman som Isak Andersson ägde år 1840, men som senare splittrades (Böös nr 152).



\jhhouse{Nykvist}{5:12}{5:12 (obs!)}{12}{51, 51a}


\jhhousepic{183-05750.jpg}{}

\jhoccupant{Broo}{Rurik \& Tyyne}{1955--}
Rurik Broo, \textborn 11.02.1926 på Gunnar, gift med Tyyne Kekki, \textborn 29.07.1922 på Hiitola i Viborgs län.

Barn: Marianne, \textborn 21.05.1951, gift Perkola, bor i Esbo.

Familjen bodde i Sverige, men kom hem 1955, köpte Nykvist lägenhet av brodern Sanfrid. Rurik rev den lilla gården som fanns på tomten, påbörjade byggandet av en ny gård av tegel. Han rev Tollikko kvarn och använde en del av rivningsmaterialet i nybygget. Familjen bodde 		 ännu några år i Sverige, men kom tillbaka och flyttade in i den nybyggda gården. Rurik har varit diversearbetare. Han arbetade en tid tillsammans med brodern Elof med byggnadsarbete, han har varit i skogsarbete samt många år som farmarbetare. Jakt och fiske har varit Ruriks stora fritidsintressen.

Tyyne \textdied 20.04.1993  --  Rurik \textdied 13.10.2016


---> \jhbold{Nykvist}   R:nr 5:12		på Grötas hemman  	 Karta 12    nr \jhbold{151}


\jhoccupant{Broo}{Sanfrid \& Svea}{1937--\allowbreak 1955}
Sanfrid Broo, \textborn 03.10.1910 på Gunnar hemman, gift med Svea Jungarå, \textborn 24.09.1908 på Jungarå hemman.
\begin{jhchildren}
  \item \jhperson{Sally}{26.09.1937}{}, gift Dahlström (Romar 35)
  \item \jhperson{Mary}{19.12.1941}{}, gift Risikko
  \item \jhperson{Herman}{03.07.1945}{}, (Grötas 12)
\end{jhchildren}
Sanfrid och Svea köpte Nykvist lägenhet på Böös 23.01.1937 av ``Skomakar-Eiriks'' dotter Saima Lindkvist. Sanfrid fick sin utkomst som diversearbetare på järnvägen, i skogen m.m.	Han deltog också i flottningen av virke längs Nykarleby älv. Han arbetade åtta år på vedplanen vid Jeppo station. Där pågick arbetet i tre skiften med ca 20 anställda. Han har berättat att 86 kubikmeter var den största vedmängd han försett tågen med under ett 8 timmar långt skift. Han satt i kommunfullmäktig åren 1952-54.

År 1955 köpte de andra halvan av Sveas hemgård på Jungarå och flyttade dit (Jungarå  19 ). Denna  gård sålde de år 1955 till Rurik och Tyyne Broo. Rurik rev gården och byggde nytt hus på samma tomt.


\jhhousepic{Boos151-Nyqvist.jpg}{}

\jhoccupant{Nykvist}{Erik \& Elisabet}{1889--\allowbreak 1937}
Erik Isaksson Nykvist, \textborn 20.07.1866 på Böös, gifte sig 06.12.1888 med Elisabet, Liisa, Isaksson Bro, \textborn 16.06.1861 på Ryss.
\begin{jhchildren}
  \item \jhperson{Isak Vilhelm}{17.09.1889}{24.02.1905}
  \item \jhperson{Johannes Eliel}{02.02.1892}{21.04.1933}
  \item \jhperson{Anders}{04.01.1895}{04.01.1895}
  \item \jhperson{Hilda Katrina}{04.01.1895}{04.01.1895}
  \item \jhperson{Katarina Erika}{07.02.1897}{24.01.1905}
  \item \jhperson{Maria Elvira}{08.08.1899}{24.01.1905}
  \item \jhperson{Erik Joel}{20.08.1902}{23.01.1905}
  \item \jhperson{Saima Elisabet}{28.05.1907}{}, gift Lindkvist
\end{jhchildren}
Erik var skomakare liksom sin far och farfar. År 1922 löste han in det 12,57 kappland stora backstuguområdet tillhörande Gustaf Wikström 	och Nikolai Grötas. Åtta (8) barn föddes varav 4 dog i scharlakansfeber år 1905. Man kan bara ana vad de enskilda mänskorna hade att gå igenom då farsoterna härjade.

Erik \textdied 09.11.1935  --  Liisa \textdied 27.03.1925



\jhhouse{Nylind}{6:119}{Böös}{12}{52, 52a}


\jhhousepic{188-05755.jpg}{}

\jhoccupant{Nylind}{Gretel}{2002--}
Paul Nylind, \textborn 19.04.1930 på Böös hemman, gifte sig 21.06.1952 med Gretel Backlund, \textborn 27.02.1931 i Socklot.
\begin{jhchildren}
  \item \jhperson{Viveka}{23.04.1953}{}, gift Backlund
  \item \jhperson{Helén}{15.08.1955}{}, gift Kronholm
  \item \jhperson{Carina}{01.08.1961}{}, gift Bonde
  \item \jhperson{Monica}{18.03.1963}{}, gift Lindholm
\end{jhchildren}
Lantbrukslägenheten såldes år 2012 till Christina och Glen Strengell. Före det hade dottern Viveka och hennes man Torbjörn Backlund hand om jordbruket under åren efter Pauls död. Paul \textdied 23.02.2002. Gretel bor kvar i gården som pensionär.


\jhoccupant{Nylind}{Paul \& Gretel}{1966--\allowbreak 2002}
Paul och Gretel övertog år 1966 Pauls föräldrars lägenhet på Böös, som omfattade 27 ha odlad jord och 39 ha skog. År 1967 byggde Paul ny gård, varvid den gamla gården revs. Paul var även lastbilschaufför, bl.a anställd vid Andelsringen i Jeppo. Från 1979 till sin pensionering hade han egen lastbilstransportfirma.


---> \jhbold{Nylind}   R:nr 6:25		på	Böös hemman   Karta    nr \jhbold{152}


\jhoccupant{Nylind}{Jürgen \& Signe}{1930--\allowbreak 1966}
Jürgen Nylind, \textborn 31.03.1905 på Böös hemman, gifte sig 07.07.1929 med Signe Jungerstam, *15.05.1903 på Jungar hemman.
\begin{jhchildren}
  \item \jhperson{\jhbold{Paul}}{19.04.1930}{}
  \item \jhperson{Tor}{20.03.1932}{}, arbetat som sjukhusvaktmästare i Sverige
  \item \jhperson{Clas}{09.02.1934}{}, arbetat som bl.a skoldirektör
  \item \jhperson{Solveig}{25.02.1940}{}, studentmerkonom, gift Nyman
  \item \jhperson{Greta}{26.06.1941}{}, socionom
  \item \jhperson{Åke}{25.06.1946}{}, ingenjör, arbetat som yrkesskollärare
\end{jhchildren}
År 1930 flyttade Jürgen och Signe in i bondgården på Böös. Jürgens föräldrar bodde ännu några år i halva gården, men efter något år byggde Isak åt sig och sin fru en sytningsstuga (nuvarande Jeppo Gäststuga) och flyttade dit.

År 1933 övertog Signe och Jürgen hans föräldrars lägenhet på Böös, som omfattade 62 ha, varav 23 ha var odlad jord. Vintertid körde Jürgen på förtjänst ut virke ur skogen, ibland med två hästar. Han deltog i både vinterkriget och fortsättningskriget. Förtroendeuppdrag hade han i Jungar Andelsmejeri samt taxeringsnämnden.

Signe var aktiv inom Marthaföreningen. År 1958 sålde Jürgen och Signe 0.0094 mantal av hemmanet till sonen Paul och hans hustru. På denna mark norr om Böös byggde Paul  ett bostadshus åt sig och sin familj (gård nr 37). År 1966 överlät Jürgen och Signe hemmanet och den gamla gården på Böös till Paul och Gretel. Paul och Gretel  överlät i sin tur sin gård till Jürgen och Signe.

Jürgen \textdied 06.06.1996  --  Signe \textdied 27.06.1997


\jhoccupant{Nylind}{Isak \& Anna L}{1891--\allowbreak 1933}
Isak Andersson, senare med efternamnet Nylind, \textborn 08.02.1862 på Böös hemman, gift med Anna Lovisa Henriksdotter, \textborn 31.08.1869 på Gunnar hemman.
\begin{jhchildren}
  \item \jhperson{Ingrid}{04.08.1895}{}, till Sverige, gift Delldén
  \item \jhperson{Hilma}{22.02.1897}{}, till USA år 1917
  \item \jhperson{Anders}{15.04.1899}{31.10.1899}
  \item \jhperson{Ester}{12.09.1900}{10.11.1902}
  \item \jhperson{Elmer}{14.12.1902}{21.12.1930}
  \item \jhperson{\jhbold{Jürgen}}{31.03.1905}{}
  \item \jhperson{Gerda}{14.10.1907}{}, till Sverige, gift Olsson
  \item \jhperson{Dagny}{13.03.1911}{}, till Sverige, gift Näslund
\end{jhchildren}
År 1883 reste Isak Andersson till Amerika, men återvände efter en tid. År 1891 köpte han 1/9 mantals lägenhet på Böös hemman av Rosina Axelsson. I köpeavtalet ingick förbindelse att ansvara för sytning till ``sytningskvinnan Lovisa Isaksdotter Böös samt till hennes dotter Susanna Andersdotter'' (Isaks mor och syster). Redan i USA planerade Isak hur hans gård skulle se ut. Av ritningen, som han sände	hem till Isak Liljeqvist, finns en kopia i Åke Nylinds ägo. Isak gav i uppdrag åt Liljeqvist att påbörja byggandet. Eventuellt stod gården	färdig när Isak återvände till hemlandet. Gården revs av sonsonen Paul. Gården hade en stor vindsvåning med vinkeltak både mot landsvägen och mot gårdssidan. På vindsvåningen fanns ett rum inredd till bostad. Gården som fanns på lägenheten före Isak lät bygga sitt hus, har enligt uppgift funnits på ungefär samma plats, som den nya gården finns idag.

Isak \textdied 03.10.1946  --  Anna Lovisa \textdied 19.05.1964


\jhoccupant{Axelsson}{Rosina}{1870--\allowbreak 1891}
Axelsson Änkan Rosina Axelsson, \textborn 05.02.1839 på Ojala hemman, gift med sjökapten Matts Julius Axelsson, \textborn 15.06.1840 i Nykarleby. Den 31 mars 1870 köpte sjökaptenen 1/9 mantal av Böös hemman av Anders Jakobsson Böös. Axelssons bodde aldrig på Böös, utan det är troligt att Anders och Anna Lovisa bodde kvar i gården. Julius dog 30.05.1873 och änkan sålde ifrågavarande hemman till Anders Böös' son Isak.


\jhoccupant{Jakobsson}{Anders \& Anna}{1856--\allowbreak 1870}
Anders Jakobsson Rundt, \textborn 22.11.1831 på Bärs, gift med Anna Lovisa Isaksdotter, \textborn 03.05.1837 på Böös.
\begin{jhchildren}
  \item \jhperson{Jacob}{24.09.1857}{25.07.1858}
  \item \jhperson{Gustaf August}{12.02.1860}{}, till USA
  \item \jhperson{\jhbold{Isak}}{08.02.1862}{}
  \item \jhperson{Maria Lovisa}{04.06.1864}{}, till USA
  \item \jhperson{Anders}{16.09.1866}{}, till USA
  \item \jhperson{Susanna}{09.05.1869}{1926}
  \item \jhperson{Johanna}{31.05.1871}{}, till USA år 1892
  \item \jhperson{Johannes}{15.05.1874}{}, som måg till Finskas (Högbjörk)
  \item \jhperson{Ida}{14.09.1876}{}
\end{jhchildren}
Anders kom som måg till Böös och makarna fick överta hälften av hustruns föräldrars hemman. Det  är troligt att Anders av ekonomiska orsaker hamnade att sälja sitt hemman. Efter att ha vistats några år i Amerika, kom sonen Isak hem och köpte tillbaka sina föräldrars hemman.

Anders \textdied 1876-87  --  Anna-Lovisa \textdied år 1919


\jhoccupant{Isaksson}{Isak \& Maja}{- 1853}
Isak Isaksson Böös d.ä., \textborn 14.01.1810 på Böös hemman, gift med Maja Michelsdotter, \textborn 19.01.1811 på Slangar.
\begin{center}
  \begin{tabular}{l l l | l l l}
    Namn & Född & Död & Namn & Född & Död \\
    \hline
    Greta & 1829 & 1830 & Sofia & 1840 & 1847 \\
    \jhbold{Isak} & 01.02.1831 & - & Caisa & 1842 & 1846 \\
    Maja & 1832 & 1833 & Sanna Lisa & 1845 & - \\
    Gustaf & 1834 & 1835 & Jacob & 1847 & 1848 \\
    Wilhelm & 1836 & 1836 & Johan & 1849 & 1868 \\
    \jhbold{Anna Lovisa} & 03.05.1837 & Henrik & 1852 & - \\
  \end{tabular}
\end{center}
Isak fick sitt hemman efter fadern och senare, 1837 och 1854, köpte han bröderna Johan, Henriks och Jacobs andel. Brodern Jacob hade fått en torplägenhet från hemmanet, som han senare gav till sin dotter Greta emot sytning och dagsverk åt hemmansägaren (gård nr). Fadern hade redan 9 februari 1828 upplåtit hemmanet till sönerna Johan Henrik, Isak och Jacob.

Isak och Maja skiljer sig år 1856. Maja får 1/3 av hemmanet. Isaks son, senare Isak Liljeqvist, fick troligen Isaks andel eftersom han i mantalslängder innehar 2/9 mantal, medan Anna Lovisa och hennes man Anders innehar 1/9. Isak Liljeqvist torde ha bott kvar i hemgården som bonde, men flyttat till gård nr 131 efter att ha sålt sin hemmansdel. Isaks andel köptes av  Svanbäck och kom senare att köpas av Johan Sjö (gård nr 50, 137)

Isak Isaksson d.ä  \textdied 1876-87  --  Maria \textdied 08.10.1897.


\jhoccupant{Andersson}{Isak \& Greta}{1834--\allowbreak 1841}
Isak Andersson Böös, \textborn 27.12.1779, gift med Greta Henriksdotter, \textborn 09.09.1778.
\begin{center}
  \begin{tabular}{l l | l l}
    Namn & Född & Namn & Född \\
    \hline
		\jhbold{Henrik Joh,} & 09.08.1802 & \jhbold{Isak} & 14.01.1810 \\
		Maria & 29.06.1804 & Caisa & 19.09.1812 \\
		Anders & 22.12.1805	&	Greta Stina & ? .10.1814 \\
		Mathias & 21.03.1807 & Anna &	31.10.1815 \\
		Isak & 12.09.1808 \textdied 01.01.1809 & \jhbold{Jakob} & 22.07.1817 \\
		- & - & Matts &	12.01.1820 \\
  \end{tabular}
\end{center}
Isak Andersson fick del av farfaderns hemman redan år 1787, men fick vänta några år tills han blev myndig. År 1811 utökades egendomen med 5/48 dels mantal av Böös, då han köpte detta av brodern Matts Andersson Böös. Han hade också köpt 5/24 mantal av Henrik Andersson Böös. Henriks far , Anders Henriksson, hade köpt 5/24 mantal år 1788 av Johan Mattsson Böös, Isaks farbror.

Isak Andersson \textdied 03.02.1855  --  Greta Henriksdotter \textdied 17.05.1856


\jhoccupant{Andersson}{Matts \& Brita}{}
Matts Andersson var född 1722 och var gift med Brita Michelsdotter född 1724. Enligt ett dokument var han åren 1781--\allowbreak 1787 skriven som ägare till 5/12 mantal. År 1787 skriver Matts över hälften av hemmanet åt  sonen Johan, andra hälften åt sonen Anders barn, Matts Andersson och Isak Andersson. Sonen Anders hade hastigt dött år 1783.

Matts Andersson \textdied 13.08.1802  --  Brita Michelsdotter \textdied 27.01.1784



\jhsubsection{Torp på Böös hemman}

Kartblad 12   nr \jhbold{117}

TimbeKais:	På Eliel Nygårds/Boris Sandås karta finns ett torp där ``TimbeKais'' bott. Från kyrkböcker har det varit svårt att hitta en person som kunde stämma överens med det namnet. Kyrkböcker från åren 1862--\allowbreak 1886 saknas pga branden vid prästgården 1886, så eventuellt kan denna TimbeKais varit skriven på Böös dessa år.


Kartblad 12   nr \jhbold{118}

\jhoccupant{Böös}{Lovisa}{1880--\allowbreak 1895}{1903--\allowbreak 1925}
Lovisa Böös, \textborn 04.09.1846 på Böös. Hon var ogift. Hon livnärde sig genom att arbeta som piga. Lovisa Böös torp är placerat utgående från en karta baserad på Eliel Nygårds minnen.

Lovisa var dotter till sockenskomakaren Henrik Mattsson Kaup och är skriven tillsammans med familjen ännu 1870. År 1875 bor hon tillsammans med brodern Isak Henriksson, men år 1880 är de skrivna i skilda hushåll. I kyrkböcker är antecknat ``medellös'' efter Lovisas namn. I november 1895 flyttar hon till Nykarleby, där hon arbetar fram till 1903, då hon flyttar tillbaka till Böös. Lovisa \textdied 08.12.1925.


\jhoccupant{Mattsson Kaup}{Henrik \& Lisa}{1836--\allowbreak 1880}
Henrik Mattsson Kaup, \textborn 16.8.1806 i Socklot, vigdes 17.9.1836 med Lisa Isaksdotter Böös, \textborn 05.12.1812 i Jeppo. Henrik var sockenskomakare, två av barnen blev också skomakare.
\begin{jhchildren}
  \item \jhperson{\jhbold{Matts}}{17.08.1838 (Böös 136)}{}
  \item \jhperson{\jhbold{Isak}}{03.03.1841 (Böös 119)}{}
  \item \jhperson{Erik}{03.04.1843}{}, gift med Anna Lovisa Hansdotter Kaup
  \item \jhperson{\jhbold{Lovisa}}{04.09.1846}{08.12.1925}, ogift
  \item \jhperson{Cajsa}{19.12.1848}{02.06.1849}
  \item \jhperson{Johan}{03.07.1850}{11.07.1855}
  \item \jhperson{Henrik}{10.11.1852}{10.02.1853}
  \item \jhperson{Susanna}{10.08.1854}{01.04.1855}
\end{jhchildren}
Henriks familjs torplägenhet fanns under samma hemman som dottern Lovisas, dvs det hemman som sedan delades till Nygård och Linder. Henrik torde ha dött 1875 – 1880, Lisa 1870 – 1875.


Kartblad 12   nr \jhbold{119}

\jhoccupant{Sandberg}{Joel \& Jenny}{1931--\allowbreak 1939}
Joel Edvardsson Sandberg, \textborn 24.01.1914 på Ojala hemman, gifte sig 09.12.1934 med Jenny Andersdotter Åstrand, \textborn 26.07.1912
\begin{jhchildren}
  \item \jhperson{Margot}{19.11.1935}{}, gift Westerlund, bor i Arizona, USA
  \item \jhperson{Siv}{27.08.1940}{}, gift Mc Glothlen
\end{jhchildren}
Familjen bodde en tid på Böös. Jenny, som var hårfrisörska, tog emot sina kunder i huset. I maj 1939 köpte de Åstrand lägenhet i Jungar och flyttade dit. På 1940-talet revs gården av Lennart och Tor Julin. De byggde upp den på nytt på Mjölnars till sytningsbostad för Ida Julin, \textborn 1882. Hon bodde i stugan till sin död 1962. Stugan revs ca 1980.


\jhoccupant{Winkvist}{Henrik \& Anna}{1894--\allowbreak 1931}
Henrik Isaksson Böös, också med efternamnet Åvist eller Winkvist, \textborn 04.09.1862 i Jeppo, vigd 05.11.1890 med Anna Lisa Joh. Henr.dtr, \textborn	12.11.1869 i Purmo. Paret hade inga barn. I kyrkböckerna tituleras Henrik skomakare, men på gravstenen i Jeppo står byggmästare. Han torde ha varit i arbete vid olika byggnadsarbeten. Bl.a finns ett arbetsintyg bevarat från Tenala. Arbetsintyget är daterat 1894 av Karl Henriksson, skeppsbyggmästare: ¸¸..Timmermannen Henrik Isaksson har under ... monars Tid Arbetat på Orflax varf å Tenala Bromarvs församling...''. Vid byggandet av Gunnar folkskola bjöds uppförandet av skolhuset ut på entrepenad. Henrik Isaksson Böös gav det förmånligaste anbudet. Enligt uppgift var Henrik också en pådrivare för att få ett bönehus till Jeppo.

Efter att torparlagen trätt i kraft, ansökte Henrik om inlösen av det sk Ekbacken benämnda området. Henrik bodde dock inte på området och han fick inte inlösningsrätt p.g.a. att nämnda område innehades på basen av annat legoförhållande än backstuguområden. Arrendekontraktet skulle utgå hösten 1920 och Henrik avstod då från området mot en ersättning av 50 mk. På det område som stugan fanns, har Henrik haft arrendekontrakt sedan 1894.

Henriks farfar var sockenskomakare Henrik Kaup. Henriks far (Isak Henriksson) samt bror (Erik Nyqvist) var också skomakare.

Henrik \textdied 30.03.1929  --  Anna Lisa \textdied 14.11.1944


\jhoccupant{Henriksson}{Isak \& Lovisa}{1869--\allowbreak 1896}
Isak Henriksson Böös, \textborn 03.03.1841, var skomakare liksom fadern Henrik. Isak gifte sig 22.6.1862 med Lovisa Eriksdotter Nyby. Familjen bodde först i Munsala men flyttade senare tillbaka till Jeppo. 1.4.1895 dog Lovisa. Isak gifte om sig med Maria Henriksdotter 			Ojala, \textborn 09.06.1843 på Mietala. Maria hade ett barn från tidigare: Johan Jakob, \textborn 15.04.1876. Tillsammans med Isak födde hon ytterligare tre.
\begin{jhchildren}
  \item \jhperson{\jhbold{Henrik}}{04.09.1862 i Munsala}{}
  \item \jhperson{\jhbold{Erik}}{20.07.1866 i Nykarleby (Böös 151)}{}
  \item \jhperson{Isak}{13.09.1869 i Jeppo}{10.06.1891}
\end{jhchildren}
Familjen bodde under sin tid i Jeppo förutom på Böös, på Ruotsala, Silvast och Jungar hemman. Eftersom både sonen Henrik och Isak var skrivna under samma hemman så har de troligen haft samma torplägenhet.

Isak \textdied 31.10.1917  --  Maria \textdied 29.03.1922


Kartblad 12   nr \jhbold{121}

På denna plats funnits ett torp. Eliel visste inte vem som bott i detta.


Kartblad 12   nr \jhbold{124}


\jhoccupant{Liljeqvist}{Wilhelm \& Sanna}{1893--\allowbreak 1908}
Wilhelm Isaksson, \textborn 02.10.1858 på Böös, gift med Sanna Sofia Pettersdotter Häggman, \textborn 28.12.1868 i Munsala.
\begin{jhchildren}
  \item \jhperson{Susanna Elvira}{22.09.1893}{}, g. Andersson i Amerika
  \item \jhperson{Elmer Vilhelm}{26.05.1895}{}, till Amerika
  \item \jhperson{Elna}{18.09.1897}{29.06.1900}
  \item \jhperson{Elna}{01.03.1902}{}, g. Vesterholm
  \item \jhperson{Eldor}{23.07.1908}{}, till Vasa
  \item \jhperson{Elias}{17.04.1910}{}, till Storkyro
\end{jhchildren}
Familjen vistades några år i Amerika, men återkom till Jeppo. År 1908 flyttade de till Korsholm och år 1935 vidare till Vasa. Gården torde ha rivits 1916--\allowbreak 1917. Enligt uppgift av personer från Böös användes virket då bönehuset byggdes.

Wilhelm \textdied 06.03.1942 i Vasa  --  Sanna \textdied 08.12.1958 i Vasa


\jhoccupant{Liljeqvist}{Isak \& Sanna}{1875--\allowbreak 1908}
Isak Isaksson Liljeqvist,  *01.02.1831 på Böös, gifte sig 09.05.1852 med Sanna Lisa Eriksdotter,  *04.09.1833 i Jeppo.
\begin{jhchildren}
  \item \jhperson{Isak}{01.02.1855, lanthandlare på Jungar}{}
  \item \jhperson{Maria}{19.10.1856, gift Jaskari}{}
  \item \jhperson{Wilhelm}{02.10.1858}{}
  \item \jhperson{Sanna}{27.03.1860}{08.01.1861}
  \item \jhperson{Anna Sofia}{25.11.1861, till Helsingfors, g. Pälli}{}
  \item \jhperson{Erik}{20.05.1864}{14.02.1865}
  \item \jhperson{Johannes}{23.03.1866, handelsman, ägde bryggeri i Lappo}{}
  \item \jhperson{Gustav}{05.06.1869, bokhållare, handelsföreståndare}{}
  \item \jhperson{Erik}{10.07.1872, till USA}{}
  \item \jhperson{Ida Johanna}{20.10.1875, till USA}{}
\end{jhchildren}
Isak kommer från Nylinds/Sjös hemman (Böös 152). Han fick 2/9 mantal av föräldrarna. I mantalslängder 1860, 1865 och 1870 är han skriven som bonde, men finns fr.o.m 1875 som statsdräng. Orsaken till att han sålt hemmanet är okänd, men det är troligt att familjen i samband med försäljningen flyttat till detta torp. Isaks hemmansdel kom senare att innehas av Johan Sjö. Inga uppgifter finns om Isak uppfört torpet.

Isak \textdied 1885--\allowbreak 1890  --  Sanna \textdied 05.11.1908


Kartblad 12   nr \jhbold{132}

Enligt uppgift av flere personer från Böös, har här funnits ett torp. Torpet torde ha rivits i början av 1900-talet. Närheten till Sjös gård indikerar att torpet hört till Sjös hemman, eventuellt kan detta också varit Johan Sjös sytningsstuga.


\jhoccupant{Ryss}{Johan \& Anna M}{1887--\allowbreak 1909}
Johan Jakob Mattsson Ryss, \textborn 12.12.1837 på Ryss gifte sig 30.06.1873 med Anna Maja Andersdotter Romar, \textborn 28.01.1846 på Romar.
\begin{jhchildren}
  \item \jhperson{Matts}{06.03.1873}{12.05.1873}
  \item \jhperson{Margareta}{18.05.1874}{}, g. Bergquist i USA
  \item \jhperson{Josefina}{10.09.1876}{19.03.1897}, ogift
  \item \jhperson{Anders}{18.09.1879}{07.10.1879}
  \item \jhperson{Johannes}{20.03.1882}{09.11.1902 i gruvolycka i USA, ogift}
  \item \jhperson{Ida Maria}{27.10.1885}{}, g. Söderlund i USA
\end{jhchildren}
Johan och Anna Majas tid som bönder på Böös blev kort. Johan dog 25.04.1887 och Anna Maja säljer det 5/36 dels mantal stora hemmanet till Matts Anders Gustafsson Jungarå år 1890. I köpekontraktet finns förbehåll och villkor för att säkra Anna Majas och barnens uppehälle och utkomst. Anna Maja förbinder sig också att årligen utföra 6 dagsverken åt den nya ägaren (se gård nr 150). Efter att ha sålt hemmanet torde Anna Maja och barnen bott i denna stuga.

Yngsta dottern Ida Maria, som emigrerat till Amerika, var på besök i Jeppo 1908-09. Anna Maja, som var ensam kvar i Finland, följde med 	Ida Maria till Amerika. Äldsta dottern Margareta hade dött 1909 och Anna Maja behövdes för hennes tre små barn, 4 mån, 6 år och 8 år.


\jhoccupant{Klockars}{Matts \& Anna}{1858--\allowbreak 1870}
Matts Eriksson Klockars, \textborn 21.01.1813 på Klockars. Bonde på faderns hemman, som han sålde 1836 och flyttade till svärfadern på Finskas. Flyttade därifrån 1857. Gift med Anna Lisa Jakobsdotter Finskas, \textborn 18.08.1805.
\begin{jhchildren}
  \item \jhperson{Lisa Matilda}{02.07.1836}{10.05.1853 genom drunkning}
  \item \jhperson{Matts}{02.02.1842}{29.11.1843}
  \item \jhperson{Jakob}{04.07.1844}{}
\end{jhchildren}


Kartblad 12   nr \jhbold{133}
Denna gård hörde till Nylinds hemman och har därför troligen bebotts av sytningsfolk, backstuguhjon och andra som var knutna till gården. Den enda som vi med säkerhet vet att har bott här var Selma Sandlin. Hon bodde från och med 1922 ett par år på Böös (Grötas, nr 25). Gården fanns kvar ännu då Signe Nylind flyttade till Böös. Då bodde en kvinna där som sålde bulla. Ungdomarna brukade köpa bulla av henne då de kom hem från dansen.

Sytningsänkan Maria Mickelsdotter är fram till sin död 08.10.1897 skriven under sitt tidigare hemman, eventuellt kan hon ha bott i denna gård.



\jhhouse{Bastugälo}{6:7}{Böös}{12}{123}


\jhhousepic{Boos123-Niittysalmi.jpeg}{}

\jhoccupant{Niittysalmi}{Amalia}{1914--\allowbreak 1969}
Amalia Gabrielsdotter Hirsimäki, \textborn 10.12.1890 i Alahärmä, gift med Akseli Kaappo Niittysalmi, \textborn 30.01.1886 i Virdois, \textdied 24.07.1921.

Amalia gifte om sig år 1935 med arbetaren Heikki Jalmari Alén, \textborn 17.03.1905 i Soini. I samband med 1943 års anteckningar om den arbetsföra befolkningen finns både Heikki Alén och Eino Niittysalmi antecknade som skogsarbetare hos Jeppo Skogsandelslag.
\begin{jhchildren}
  \item \jhperson{Sylvi Ester}{13.07.1912 i Jeppo}{05.02.1929}
  \item \jhperson{Tekla}{23.09.1920}{}
  \item \jhperson{Armas}{16.06.1922}{}, busschaufför
  \item \jhperson{Werna Anna Elisabeth}{11.04.1924}{}
  \item \jhperson{Eino Eriksson Heiskari}{10.03.1926}{25.08.1952}
  \item \jhperson{Toivo Erik}{10.03.1926}{}
  \item \jhperson{Veikko Ilmari}{04.06.1928}{}
  \item \jhperson{Elvi Susanna}{10.09.1929}{15.04.1958}
\end{jhchildren}

Den 16 juni 1922 höll legonämnden möte hos Amalia, som ansökt om att få lösa in backstuguområdet ``Bastugälo'' tillhörande Eliel Nygård. Man bestämde lego-områdets gränser genom att slå ner pålar i alla hörn. Amalia hade att uppvisa skriftligt kontrakt från 1914. Kontraktet hade gjorts med arbetaren Axel Salminen (Akseli Niittysalmi). Priset för området bestämdes och dagsverksskyldigheten upphörde genast i.o.m kontant betalning.

Amalias äldsta barn gick i svensk skola, medan de yngre gick i den finska skolan. Amalia var en aktiv pådrivare av finsk undervisning åt barnen. Efter Amalias död fick Elvi Susannas son Reijo Vesanen gården i arv. Han sålde den på 70-talet till romer, som sålde gården vidare till Jeppo kommun. Kommunen sålde gården till tidigare jordägaren, Valfrid Nygård. Ragnar Nygård med sonen Leif-Ole rev gården.

Heikki Jalmari \textdied 14.02.1960  --  Amalia \textdied 24.05.1963


\jhoccupant{Stam}{Anna Brita}{1872--\allowbreak 1915}
Änkan Anna Brita Andersdotter Stam, \textborn 02.07.1838 i Larsmo, flyttade till Böös ett par år efter maken Johan Eriksson Stams död 15.09.1870. Familjen hade bott på Mietala (Mietala 113), Johan med sin första fru på Böös, gård nr 137.

Sonen Anders Johansson, \textborn 03.07.1863 och hans hustru Anna Lovisa Gustafsdotter Renvaktar, \textborn 06.10.1866 i Nykarleby är också skrivna på Böös. Änkan och sonen med familj är skrivna under Eliel Nygårds hemman. Anna Stam \textdied 01.11.1915.


\jhoccupant{Wicktelin}{Carl \& Maja}{1850--\allowbreak 1870}
Torparen Carl Petter Pehrsson Wicktelin, \textborn 25.10.1825 i Ytterjeppo, vigdes 25.07.1848 med Maja Caisa Mattsdotter Bösas, \textborn 28.11.1827.
\begin{jhchildren}
  \item \jhperson{Erik}{24.12.1849 }{}
  \item \jhperson{Matts}{10.03.1852}{}
  \item \jhperson{Maria}{22.09.1855}{}
  \item \jhperson{Karl Gustaf}{13.11.1858}{1861}
  \item \jhperson{Johannes}{26.08.1861}{}
  \item \jhperson{Jakob Wilhelm}{1865}{1868}
\end{jhchildren}
Torparen Carl Pehrsson (Pettersson) var skriven under samma bonde som Anna Stam, men om han bott i just detta torp är oklart. Sonen 			Matts var timmerman, gifte sig med Anna Lovisa Jakobsdotter, Stenbacka-smedens dotter. De flyttade till  Helsingfors, där Matts tog 			namnet Blomqvist.



\jhhouse{Pettersborg}{6:13}{Böös}{12}{129}


\jhoccupant{Perä}{Niilo \& Niina}{1951--\allowbreak 1956}
Niilo Johannes Perä, \textborn 03.03.1924 i Alahärmä, gift med Niina Uschanoff, \textborn 18.01.1927 från Åbo. Niina var grekisk katolik, men kyrklig vigsel är antecknad 24.06.1951.
\begin{jhchildren}
  \item \jhperson{Aino Ritva Orvokki}{27.05.1952 i Jeppo}{}
  \item \jhperson{Tauno Johannes}{27.05.1953 i Jeppo}{}
  \item \jhperson{Eila Inkeri}{11.08.1954}{}
\end{jhchildren}
Niilo, som var synskadad, livnärde sig som borstarbetare. Gården köpte han av Edvard och Hilda Sandbacka år 1951. Edvard hade ärvt den efter sin syster Katrina. Den 05.06.1956 flyttar familjen till Pargas. Jeppo kommun köper området och låter riva gården.


\jhoccupant{Mietala}{Katrina}{1909--\allowbreak 1949}
Katrina Simonsdotter Mietala, \textborn 06.12.1872 på Mietala. Kallades allmänt ``Skrädda-Katrin'', i mantalslängder hade hon benämningen sömmerska. Hon livnärde sig på att sy kläder åt folk. Hon flyttade till Böös nån gång mellan 1905 och 1910. Torparkontrakt med Anders Södergård är undertecknat 21 augusti 1909. Hon bodde före det på Mietala.

Katrina hade tagit ut betyg till Amerika år 1917, men emigrerade inte. Eventuellt inverkade torparlagen på beslutet att stanna. År 1920 beslöts att Katrina får lösa in det 2 kappland stora backstuguområdet i s.k. Pettersborgsgärdan. Området ägdes då av bonden Anders 				Gustafsson Böös (Södergård). Katrina \textdied 12.08.1949.


\jhoccupant{Östman}{Karl \& Sanna Lisa}{1875--\allowbreak 1888}
Änkan Sanna Lisa Isaksdotter eller Östman, \textborn 15.07.1840, innehar området enligt kontrakt från 5 februari 1887. Mannen, lanthandlare Karl Östman hade dött 23.12.1886 och kontraktet gjordes på nytt med änkan. I bouppteckningshandlingarna efter maken finns flere sidor med diverse handelsvaror, ss tyger, sidenband, pipor, kryddor, fickur. Bland kvarlåtenskapen nämns en mindre boningsstuga, 4 lador samt en vattenkvarn. Medan boningsstugan var värd 150 mk, är 2000 mk den summa som uppgivits för kvarnen. Platsen för denna kvarn är okänd.

Texten i kontraktet som gjorts med Anders Jakobsson Böös, är exakt samma som i Katrina Mietalas. Enligt kontraktet är det 2 kappland stora området beläget i ``norra hörnet av den så kallade Pettersborgs gärdan''. Det är därför troligt att det är samma torp som Katrina bodde i. Enligt kyrkböcker flyttar Sanna Lisa med barnen till Alahärmä 1888.
\begin{jhchildren}
  \item \jhperson{Wendla Maria}{01.10.1876}{}
  \item \jhperson{Ida Sofia}{15.05.1879}{}
  \item \jhperson{Anna Lovisa}{30.06.1884}{25.02.1887}
\end{jhchildren}


\jhoccupant{Forsman}{Jakob \& Lisa}{1863--\allowbreak 1876}
Sjömannen Jakob Johansson Forsman, \textborn 14.01.1818 samt hustru Lisa Mattsdotter, \textborn 21.02.1815.
\begin{jhchildren}
  \item \jhperson{Maria}{25.03.1844}{}
  \item \jhperson{Jakob}{29.11.1852, gift med Anna Lovisa Dahl på Mietala}{}
  \item \jhperson{Erik}{01.02.1853, g. med Maja Samuelsdr. Roos}{}
\end{jhchildren}
Familjen hade först sitt hem på Grötas, flyttade sedan till Böös, där Jakob först var skriven som torpare, sedan en kort tid som bonde. År 1870, då Johan Mattsson säljer sitt hemman till Nils Mattsson, skriver han in i köpebrevet torparen Jakob Forsmans rättighet till den torplägenhet som han innehade under Mattssons hemmansdel. Osäkert om denna torplägenhet var på Pettersborgs gärdan, men Östmans och Forsmans var skrivna under samma hemman. År 1871 köpte Forsman ett 5/96 mantals hemman av Karl Eriksson Grötas, år 1873 var Jakob och Nils Mattsson åboer. Var familjen bott under denna tid är oklart.

År 1876 köpte Jakob ett hemman på Ruotsala och familjen flyttade dit.



\jhhouse{Dahlström}{6:37}{Böös}{12}{130}


\jhoccupant{Dahlström}{Olof \& Sirkka}{1967--\allowbreak 1981}
Dahlström Olof, \textborn 1940, på Jungarå, gift med Sirkka Sipponen, \textborn 1938 i Alahärmä. Den 11.05.1967 köpte Olof Dahlström lägenheten med byggnader. Lägenheten var 0,205 ha stor. Sirkka och Olof bodde i stugan 1967--\allowbreak 1981. Gården brann ner i januari 1981. Mera info om Olofs familj under Silvast nr 50.


\jhoccupant{Sjö}{Lennart \& Jenny}{1937--\allowbreak 1967}
Lennart Sjö, \textborn 16.04.1911 på Böös (Böös 150 ), gift med mejerskan Jenny Nylund, \textborn 24.04.1909 på Jungar (Jungar 22 ).
\begin{jhchildren}
  \item \jhperson{Gun-Viol}{16.02.1941, gift Wärnman}{}
  \item \jhperson{Håkan}{*05.12.1943}{}
\end{jhchildren}
Lennart arbetade åren 1928--\allowbreak 1936 vid Jungar Andelsmejeri . Han blev anställd som hjälp åt montören, när de började installera maskiner i det nybyggda mejeriet och fortsatte som maskinist till september 1936 då han åker som praktikant till Närpes Andelsmejeri. Åren 1937 – 38 			gick han i mejeriskola i Vasa. Efter mejeriskolan har han arbetat vid bl.a Munsala Andelsmejeri (1939 – 56), Kållby Andelsmejeri (1956 – 60), Jakobstads mjölkcentral (1960 – 76). Jenny hade tjänst vid Jungar och Hirvlax mejerier.

Lennart köpte marken av sin mor 22.01.1937 och Lennart och hans bröder uppförde gården, som var tänkt för Lennarts familj. Lennart kom dock aldrig att bo i gården utan den hyrdes ut.

Personer på Böös kommer ihåg att dessa hyresgäster bott i gården:
\begin{enumerate}
  \item 1966 – 1967	Orrholm Ellen (Silvast 372)
  \item 1965 – 1966  Familjen Ruben och Ragnborg Nygård (Silvast 107)
  \item 1962 – 1964	Familjen Harry och Inga Ljung (Skog 1)
  \item 1955 – 1958	Manne Bergman, \textborn 30.09.1923, g. m. Elise Romar, \textborn 28.12.1927.  Barn: Lars-Erik, \textborn 05.02.1956. Familjen flyttade till Nykarleby 1957--\allowbreak 1958
  \item 1953  - 1955  Familjen Lea och Viktor Jungerstam ( Ruotsala 26)
  \item 1952--\allowbreak 1953	Bankdirektör Gunnar Wadström, \textborn 30.05.1895 , blev  sjukpensionerad år 1952 och flyttade till Böös. Sonen Johan gick i 	Kristinestads Samlyceum och blev student samma vår som pappan dog. Wadström sköttes av Ester Sandås samt Wadströms syster, Aina Wadström, \textborn 15.09.1887, som pensionerades vid denna tid. Hon flyttade till Böös för att sköta om sin bror. Aina 			arbetat som översköterska vid SÖD i Vasa, som fungerade som krigssjukhus. Hon hade erhållit II klass Frihetsmedalj av Mannerheim 1944 samt 1939-40 års krigsminnesmedalj med Fältmarskalk Mannerheims dagorder 1940. Gunnar Wadström dog 05.06.1953.
  \item 1941--\allowbreak 1944	Familjen Gustafsson Lennart och Gemima. Dottern Denice kommer ihåg att hon och brodern Stefan stod vid vägen vid sitt nya hem och såg den tyska armén, när den kom förbi på väg till Jakobstad (Fors 43).
  \item 1938 – 1939   Familjen Törnqvist bodde ca ett år i gården. Närmare uppgifter om dessa har inte hittats.
\end{enumerate}}



\jhhouse{Ekbacken}{?}{Grötas (obs!)}{12}{134}

\jhoccupant{Holmlund}{Johan \& Kaisa}{1877--\allowbreak 1903}
Backstugusittaren Johan Hansson Holmlund, \textborn 03.03.1835 i Purmo gift 02.10.1870 med Kaisa Eriksdotter Draka/Grötas, \textborn 20.06.1839  i Ytterjeppo.
\begin{jhchildren}
  \item \jhperson{Johanna}{15.10.1872, till Nykarleby 17.04.1893}{}
  \item \jhperson{Johannes}{18.12.1874}{}
  \item \jhperson{August}{10.07.1876}{}
  \item \jhperson{Maria Lovisa}{14.05.1878, till Nykarleby 04.05.1895}{}
\end{jhchildren}
Familjen var hela tiden skriven på Grötas, eventuellt bodde den första tiden hos Kaisas familj. När de flyttat till Ekbacken är oklart, men 22 juli 1891 har arrendekontrakt med bonden Johan Eriksson Grötas  skrivits. Senare, dvs 3 mars 1902 övertar Jeppo församlings fattigvårdsstyrelse inteckningen, enär paret då intagits i fattigvården. Alla barnen torde senare ha emigrerat till USA.

 Kaisa \textdied 21.01.1892  --	Johan \textdied 22.05.1903


\jhoccupant{Grötas/Eriksdotter}{Maria}{1870--\allowbreak 1877}
Maria Eriksdotter Grötas, ``Eikback Maj'', \textborn 25.12.1847 på Grötas, bodde på Ekbacken, eventuellt i den stuga som Holmlund sedan flyttade in i. Denna stuga fanns på Grötas hemman.

Maj var ogift. Hon tjänade som piga, men flyttade senare in hos sin syster Anna på Skog. Om orsaken var att hjälpa systern i arbetet eller om orsaken var problem med egen hälsa är okänt.  Åren 1873-82 är hon skriven både på Böös och på 	Skog. Eventuellt har hon dött i slutet på denna period. Hon finns inte med år 1883.



\jhhouse{Ekbacken}{?}{Böös}{12}{136}

\jhoccupant{Henriksson}{Matts \& Anna}{1866--\allowbreak 1907}
Skomakare Matts Henriksson Böös, \textborn 17.08.1838 på Böös. Hustru Anna Stina Mattsdotter Sämskar, \textborn 14.07.1841 i Larsmo. Anna Stina dog i lungsot 15.03.1895. Matts gifte om sig 18.03.1900 med Maria Eriksdotter, \textborn 25.12.1848 i Socklot.
\begin{jhchildren}
  \item \jhperson{Anna-Lisa}{16.10.1866}{09.02.1896}
  \item \jhperson{Maria Lovisa}{03.08.1869}{1969}, g. m Johannes Böös
  \item \jhperson{Johanna (Hanna)}{27.07.1871}{1908}, gift Wallin
  \item \jhperson{Axel Henrik}{15.10.1873}{12.08.1953}
  \item \jhperson{Matts Wiktor}{20.07.1876}{24.05.1966}, till Kronoby
  \item \jhperson{Wendla}{21.06.1879}{23.02.1898}
  \item \jhperson{Hilda Katarina}{08.10.1887}{}, gift 1. Johnson, 2. Nelson
\end{jhchildren}
Familjen bodde i en stuga på Ekbacken. Det berättas att det var en enkel, liten stuga med farstu. Det var inte så vanligt att torpare hade häst, men en häst hade Matts och hästen hade man att stå i farstun. Hästen hade han kanske skaffat efter första hustruns död. Vid bouppteckning efter henne är nämnt endast 1 ko och 4 får. Utöver diverse hushållssaker är nämnt en stuga, uthus och arrendejord.

Matts Henriksson var torpare under Nylinds hemman. I USA gjord släktutredning om dottern Hanna sägs fadern (Matts) varit timmerman, som varit på byggnadsarbeten bl.a i St Petersburg. På vintrarna fanns det inte arbete och många reste till St Petersburg för kortare eller längre perioder. I kyrkböckerna tituleras Matts dock skomakare och torpare.

Alla barnen, förutom Wiktor och Wendla, emigrerade till Amerika. Efter att barnen flyttat hemifrån och föräldrarna dött flyttades stugan till Silvast. Där uppfördes den på nytt, troligen vid Stationsvägen, som går bredvid järnvägsspåret.

Matts dog 07.05.1907, Maria flyttade två år senare till Skog, där hon bodde tills hon dog 3 februari 1920.


\jhoccupant{Nahkala}{Gustaf \& Maja}{1840--\allowbreak 1870}
Gustaf Mattsson Nahkala, \textborn 01.01.1793 i Kauhava, gift med Maja Caisasdotter, \textborn 13.12.1798 i Överjeppo.

Barn:	Lovisa, \textborn 05.09.1835

Gustaf Mattsson är i mantalslängder skriven som torpare samt inhyses under Nylinds hemman under minst 30 års tid. Eventuellt bodde de i detta torp.

Maja \textdied 10.01.1867, Gustafs dödsdatum har inte hittats.



---> \jhbold{Böös} hemman				      nr \jhbold{136a}

\jhoccupant{Löfgren}{Jacob \& Anna}{1850--\allowbreak 1870}
Torparen Jacob Andersson Löfgren, \textborn 13.08.1827 på Böös, vigd 9 maj 1852 med Anna Sanna Johansdotter Finskas, \textborn 11.08.1828.
\begin{jhchildren}
  \item \jhperson{Johan Jakob}{10.12.1852}{1853}
  \item \jhperson{Maria}{11.03.1854}{1856}
  \item \jhperson{Anders}{25.05.1856}{}
  \item \jhperson{Sanna Lisa}{22.12.1858}{}
  \item \jhperson{Anna Sanna}{23.04.1861, gift med Johan Andersson}{}
  \item \jhperson{Maria Lovisa}{30.08.1865}{}
  \item \jhperson{Johanna}{22.11.1867}{}
\end{jhchildren}
Jacob Löfgren övertog torpstället efter sin far, som var som torpare under Nylinds (Isak Isaksson) hemman. I torpkontraktet står bl.a ``Gårdstomten och päronlandet får han på vestra sidan om vintervägen till sex kappland.'' Det har funnits en vinterväg tvärs genom Ekbacken, men beskrivningen stämmer inte överens med de tre gårdar, som fanns där fram till 1900-talet. Den av dessa tre som är mest sannolik är Matts Henrikssons gård, men eventuellt har här funnits flere gårdar. - År 1870 köpte Jacob ett hemman på Böös och blev bonde (Böös 143).


\jhoccupant{Löfgren}{Anders \& Lisa}{1822--\allowbreak 1850}
Anders Andersson Löfgren, \textborn 23.04.1796 på Gästgivars i Munsala, gifte sig 15.07.1821 med Lisa Greta Andersdotter Böös, \textborn 02.08.1795. Lisa Greta var bonddotter på ett av de större hemmanen på Böös. Bröderna som övertog hemmanet överlät en torplägenhet till svågern Anders Löfgren och deras syster Lisa Greta.
\begin{jhchildren}
  \item \jhperson{Anders}{02.12.1821}{1823}
  \item \jhperson{Karolina}{25.09.1823}{}
  \item \jhperson{Johan Henrik}{11.03.1826}{}
  \item \jhperson{Jacob}{13.08.1827}{}
  \item \jhperson{Brita Lisa}{08.08.1830}{}
  \item \jhperson{Anna Sanna}{07.12.1836}{}
\end{jhchildren}
Anders \textdied 11.04.1842  --  Lisa Greta \textdied 12.10.1846



\jhhouse{Ekbacken}{?}{Böös}{12}{137}

Boris Sandås har ritat en karta utgående från Eliel Nygårds minnen av obesuttna på Böös. Enligt den kartan ska en man, ``Pellas Jepp'' bott här. Ännu långt in på 1900-talet var Pellas lindon en plats, där man samlades för att fira valborg. Det berättas också om att ungdomar ordnat danser där. Enligt Agda Nygård (Grötas 29) har hennes far berättat om en ``urmakare'' som bodde här. Då han reparerat en klocka lär han ha sagt att ``his klåckon ho gaar fast man sko gräva neer en fleir 	aalnar ondi jåolen''.

Vem denna Jepp eller vem denna urmakare var är fortfarande oklart. Torpare 1824--\allowbreak 1859 som placerats här har torpkontrakt med Nylinds förfäder och i torpkontraktet skrivs om ``tegar på västra sidan om 	vintervägen, norr om Anders Löfgrens torp...''. Inget av de tre torp som vi kunnat placera på Ekbacken finns på västra sidan om 	vintervägen. Eventuellt har det funnits flere torp här.


\jhoccupant{Kihakoski}{Petter \& Mina}{1865--\allowbreak 1913}
Simon Petter Johansson Kihakoski, \textborn 13.12.1823 i Purmo, gift med Wilhelmina Eriksdotter, \textborn 23.06.1842 i Pedersöre (Simons 2. hustru).
\begin{jhchildren}
  \item \jhperson{Petter Gustaf}{30.10.1870}{}
  \item \jhperson{Sofia Wilhelmina}{01.06.1872}{}
  \item \jhperson{Maria Lovisa}{24.01.1878}{}
  \item \jhperson{Ida Johanna}{20.02.1879}{}
  \item \jhperson{Wendla}{03.08.1882}{}
\end{jhchildren}
Simon Petter dog 09.02.1892. Den 1 april 1899 bjöd Mina Böös ut en stuga på Ekbacken. Eventuellt är det denna Mina, men hon är skriven också efter detta på Böös. Hon dog 01.11.1913 . Dottern Sofia är år 1917 skriven som piga på Romar. Dottern Maria Lovisa och hennes dotter Ida Maria, \textborn 31.03.1905 flyttade till Jungar.


\jhoccupant{Levelius}{Johan \& Lisa}{1861--\allowbreak 1865}
Torparen Johan Henriksson Levelius, \textborn 1834, vigd år 1856 i Alahärmä med Lisa Israelsdotter Frändi, \textborn 1837, de hade 6 barn då de kom från Alahärmä år 1861. De flyttade tillbaka efter ett par år på Böös.


\jhoccupant{Eriksson}{Johan \& Greta}{1855--\allowbreak 1859}
Drängen Johan Eriksson Kjeppo, senare Stam, \textborn 25.05.1837 på Lavast hemman, kom som måg till Böös i.o.m giftermålet 28.03.1858 med Greta Jakobsdotter Bösas, \textborn 01.03.1837. Greta var dotter till förra bonden, senare torparen Jakob Isaksson Böös. År 1855 överlät han sin torplägenhet till sin dotter.

Greta dog i barnsäng 03.09.1858. Johan flyttade till Grötas med sonen, som föddes samma dag som Greta dog. Sonen Johan Jakob blev endast 2 år, 8 månader.

Torparen Johan gifte om sig 14.01.1861 med Anna Brita Andersdotter Mjetala, \textborn 02.07.1838 i Larsmo (Mietala, gård 113).


\jhoccupant{Isaksson}{Jacob \& Anna}{1841--\allowbreak 1855}
Jacob Isaksson, \textborn 22.07.1817 på Böös, gift med Anna Brita Simonsdr, \textborn 04.12.1806, \textdied 30.06.1854.
Barn:	 \jhbold{Greta},  *01.04.1837

Jacob hade fått halva hemmanet av sin far Isak Andersson, men sålde sin andel till brodern. Torplägenheten fick han 1841 av fadern.


\jhoccupant{Dahlkarl}{Daniel \& Sanna}{1833}
Torparen Daniel Danielsson Dahlkarl, \textborn 28.07.1798 i Munsala, gift med Sanna Eriksdotter, \textborn 03.12.1808 i Munsala.
Barn:	Daniel, \textborn 05.12.1832

Isak Andersson hade sålt denna torplägenhet till sin bror Henrik 1824. Henrik hade sålt den vidare till Daniel Dahlkarl, som innehade torpet några år. Daniel sålde torpstället tillbaka till Isak år 1833.


\jhoccupant{Andersson}{Henrik}{1824--}
Henrik Andersson, köpte torplägenheten av sin bror Isak Andersson år 1824 (Böös 152). Angående torpet står ingenting i kontraktet var det är beläget, endast att husen ingår. Till åker får ``torparen fyra tegar på västra sidan om vintervägen, norr om Anders Löfgrens torp...''. Åtminstone torde åkern funnits på Ekbacken, där det tidigare fanns en 	vinterväg, som gick genom skogen.



\jhhouse{Jungar Filial}{6:15}{Böös}{12}{54}


\jhhousepic{189-05756.jpg}{}

\jhoccupant{Aalto}{Markku \& Seija}{1987--}
Markku Aalto, \textborn 1961 i Jyväskylä, gift med Seija Kyllönen, \textborn 1965 i Kuhmo. De kom till Jeppo 1987 och köpte fastigheten. Deras barn har alla växt upp i det hus som tidigare utgjort den gamla butikslokalen. Markku arbetar på ELHO-fabriken i Bennäs med tillverkning av lantbruksmaskiner och Seija är anställd på Jeppo Potatis, stationerad på enheten i Voltti.

\begin{jhchildren}
  \item \jhperson{Timo}{1983, arbetar på MH-Color i Vasa}{}
  \item \jhperson{Teija}{1985}{},  på Jeppo Potatis
  \item \jhperson{Teemu Markku}{1986}{}, på Voltin Väri
  \item \jhperson{Tiina Suvi}{1993}{}, är vuxenstuderande
  \item \jhperson{Toni Markku}{1995}{}, arbetar på Ab Mirka Oy
\end{jhchildren}


\jhoccupant{Jeppo-Oravais}{handelslag}{1932--\allowbreak 1982}
Jeppo-Oravais handelslags butiksfilial byggdes 1932, renoverades 1970, lades ner 1982.

Filialen var en  utbyggd service, som verkställdes efter att Jeppo Handelslag 1926 övertagit Oravais Andelshandel, 1927 Ytterjeppo filial av Nykarleby handelslag och 1933 Pensala filial från Munsala andelshandel. År 1932 stod Jungar by i tur och en byggnad i funkisstil byggdes.

Filialen har betjänat den södra delen av Jungar by i 50 år innan den stängde. Parallellt började också handelslagets butiksbil komplettera servicen. Rätt många personer har genom åren fungerat som personal i filialen och en komplett lista kan vi inte presentera men bl.a.följande personer känner vi till:
\begin{center}
  \begin{tabular}{l l | l}
    Sven Åke Forss, Vörå & 1939-46 & - \\
    Senny Nylund & 1946-50 & - \\
    Ellen Forss, Oravais & 1950-53 & - \\
    Senny Nylund & 1953-55 & Kurt Betlehem \\
    Valdemar Back & 1955-58 & Ingmar Sandvik \\
    Erik Frilund & 1958-62 & Gunhild Eklöv \\
    Sven Simons & 1962-64 & Kristina Elenius \\
    Börje Andersson & 1964-73 & Kerstin Häggblom \\
    Greger Forsbacka & 1973-78 & Ulla-Britt Sunngren \\
    Gretel Kronqvist & 1978-82 & Marianne Broo \\
  \end{tabular}
\end{center}



\jhhouse{Mejeriet}{6:90}{Böös}{12}{154}


\jhoccupant{Jeppo}{Pälsberederi}{1952--\allowbreak 1955}
Gunnar Norrback, \textborn 1914 (se Fors nr 19), byggmästare från Nykarleby lk. köpte fastigheten 1952 tillsammans med Wilhelm Åkermark, dipl.ing från Munsala. De hade startat företaget Ab Jeppo Pälsberederi Oy för skinnberedning efter att Jeppo Fabriker Ab vid Kiitola gått i konkurs. Som historien från Kiitola visar, var konkurrensen hård på denna marknad och efter en tid beslöt de lägga ned tillverkningen.


\jhoccupant{Prevex}{Ab/Oy}{1955--\allowbreak 1956}
Istället bildade de med Sven Nyman, \textborn 1916 i Purmo, och Manne Bergman, \textborn 1923, från Jeppo ett nytt bolag; ``Ab Prevex Oy'', som ett av de tidigaste företagen i landet att satsa på tillverkning av plastprodukter. Den 31 mars 1955 undertecknades bolagsordningen. En extruder (strängpress) beställdes från Larsmo Grävmaskinbolag och  tillverkningen av plaströr började. Att börja med en så helt ny produkt var inte lätt, varken tekniskt eller finansiellt och företaget balanserade ofta på en knivsegg.

Norrback och Åkermark sålde sitt aktieinnehav till bolaget och drog sej ur. Sven Nyman och Manne Bergman fortsatte tillsammans med en ny aktionär, Werner Nyström, att  driva företaget med en näst intill omänsklig arbetsinsats. Just som de ``fått upp farten'' brann fastigheten 28 dec. 1956. Till all lycka lyckades de rädda en nyinköpt men ännu oförsäkrad extruder ur den häftiga branden. De beslöt flytta produktionen till Nykarleby i nya utrymmen, för att där så småningom utvecklas till ett framgångsrikt företag.

Den 22 maj 1989 avslutade en ny eldsvåda fastighetens livstid definitivt. Den hade då stått tom en lång tid. Under den tid Ab Prevex Oy var verksam i fastigheten bodde både Åkermarks familj och Manne Bergman tidvis i mejeriets övre våning.


\jhoccupant{Jungar}{Andelsmejeri}{1928--\allowbreak 1952}
Den 17.10.1913 bildades Jungar Andelsmejeri. Detta som ett uttalat missnöje över att det mejeri som ca 10 år verkat på Böös flyttat till Silvast som granne till Silvast kvarn. En träbyggnad uppfördes nu för ändamålet på en annan tomt än denna. (se Ruotsala nr …)

1928 insåg man att det behövdes en ny och större byggnad. Planer för ett nytt mejeri av tegel startade och placeringen blev på denna av Isak Nylind inköpta tomt. Nopsa tegel med plåttak blev materialet. Det nya mejeriet väckte intresse för anslutning från byarna  Skog, Romar, Silvast och Grötas, men hårda anslutningsvillkor dödade intresset. Mejeristyrelsen var dock aktiv på flera plan. 1935 bildades en kontrollförening och den ännu oprövade insemineringstekniken var uppe till diskussion redan 1944.

1942, under brinnande krig, togs frågan om en samgång med Jeppo andelsmejeri upp till diskussion utan att beslut fattades. 1944 dök frågan upp på nytt, men inte heller nu kom man vidare. Men på våren 1951 tillsattes en kommitté och efter 3 på varandra följande sammanträden beslöt man slutligen den 11.05.1951 att sälja  mejeriet till Jeppo Andelsmejeri och nedlägga driften.

Under mejeriets driftstid beboddes bostadslägenheterna  på övre av bl.a. disponenten Birger Stenbäck, \textborn 1916, med hustru Anna, \textborn 1920, och sonen Leif, \textborn 1946. När mejeridriften lades ner 1951, flyttade han till Sydösterbotten och mejerskan Verna Strand till Silvast och Jeppo Andelsmejeri. Övriga mejerskor som  bott på mejeriet är bl.a. Etel Häll och Hjördis Häggström.



\jhhouse{Mangelstugan på Böös}{6:110}{Böös}{12}{53}


\jhhousepic{191-05757.jpg}{Mangelstugan}

\jhoccupant{Tiden}{från}{1949--}
När behovet av detta hus för Jungar Andelsmejeri minskade i betydelse, bildade intressenter från hemmanen på Böös, Jungar och Ruotsala ett ``mangelbolag''. Initiativet till anskaffandet av en mangel hade tagits 1936, men den saknade ett lämpligt utrymme. Den var i alla fall placerad i en liten bod hos Daniel Jungar i väntan på en bättre lösning. Blickarna riktades nu mot mejeriets stuga, avsedd för tillfälliga praktikanter. En teckningslista för intresserade cirkulerade i slutet av 1949. In på det nya året verkställdes köpet och köpesumman på 6000 mark erlades. Huset bestod av två rum och i ett av dem placerades mangeln och viktlådan fylldes på med stenar. Nu var den klar för användning i eget utrymme.

Gunnar marthakrets' porslinsservis och Jungar kalasservis fick också plats i huset. Servisen har sedan splittrats och förvaras delvis i församlingshemmets källare. Däremot står mangeln kvar, redo att när som helst användas om behov uppstår. Också en vävstol står monterad och väntar på sin användare. Den flyttades från kommunalgårdens källare tillsammans med de andra vävstolarna, som ägdes av Marthaföreningen, sedan kommunalgården bytt ägare 2011. Det fältkök som i tiden köpts av Jungar kalasservis, har också fått sin plats i huset. Fältköket lånades oftast ut när Missionslägret samlades i Jeppo. Tråkigt nog har tiden gått ifrån denna självklara plats för skapande samvaro och gemenskap för byns kvinnor. Att Marthaföreningen kunnat använda husets utrymmen har till alla delar varit av godo. Huset har med tiden renoverats, fått nytt tak och står väl rustat att möta framtiden.


\jhoccupant{Jungar}{Andelsmejeri}{1915--\allowbreak 1949}
Efter att Jungar Andelsmejeri återuppförts efter en kort session vid Jeppo Andelsmejeri i Silvast 1913, byggdes år 1915 denna stuga för att kunna hysa de praktikanter och annan personal, som mejeriet var i behov av. Åren ända fram till tiden efter krigen fyllde den denna uppgift, men småningom ändrade arbetskraftsituationen karaktär och behovet av ett sådant hus minskade. Därför kunde mejeriets styrelse bifalla ``mangelbolagets'' önskan om att få köpa huset 1949.

Kontrollassistenter som bott i detta hus är bl.a. Norna Slotte och Jenni Holmqvist.


\jhhouse{``Lovisastugan''}{6:86}{Böös}{12}{55}


\jhhousepic{192-05758.jpg}{Lovisastugan}

\jhoccupant{Strengell}{Christina \& Glenn}{2000--}
Christina och Glenn Strengell (se Jungar nr 5) köpte stugan år 2000 av Clas och Sonja Nylind. Christina hade drömt om gårdsbruksturism och kunde i och med köpet och renoveringen av stugan förverkliga drömmen om ``att ta hem världen''.

Stugan grundrenoverades från golv till tak. Vatten och avlopp drogs in, WC och dusch installerades och huset utrustades till en modern bostad, men med en mormorsstugas charm. Kakelugn och vedspis lämnades kvar i huset som ett pittoreskt inslag. I trädgården planterades bärbuskar, körsbärs- och äppelträd. En vit syrenhäck planterades runtomkring huset, samt midsommarrosor invid ytterdörren. Knappast kunde hon eller någon annan tro att så många turister från världens alla hörn skulle hitta till Jeppo. De första gästerna bokades in i mars 2005. De kom från USA. Internets sökmotorer gjorde det möjligt att hitta inkvartering i Jeppo!

I början bestod gästerna mest av emigranter på sommarbesök i barndomstrakterna, men i och med att nöjesparken Power Park i Alahärmä öppnade, ökade behovet av inkvartering i nejden markant. Många turister som kör mellan Helsingfors och Lappland övernattar ``halvvägs'' i Jeppo. De flesta gästerna har kommit från Finland, men många har rest långa sträckor innan de hittat inkvartering i Jeppo, t.ex från Hong Kong, Macao, Tanzania, Malaysia, Venezuela, Italien, Grekland, Australien, Tyskland m.fl. Stugan är öppen året om.

Gästerna har trivts i Jeppo och recensionspoängen är hög; 9,0 (2015)och 9,1 (2016). Inkvarteringsverksamheten kallades i början för Jeppo Gäststuga, men när inkvarteringstjänsten år 2008 utökades med Birger och Anna Sjös hus i Bösas på Ekbackavägen, behövdes mera preciserade namn på stugorna. Stugorna fick namnet ``Lovisastugan'' och ``Annastugan'' efter namnet på sina först kvinnliga ägare. Verksamheten behövde också ett mera internationellt namn. Det blev Jeppo Guesthouses, men även Jeppo Gäststugor och Jepuan Vierasmökit används beroende på sammanhang och gäster.


\jhoccupant{Nylind}{Clas \& Maria}{1980--\allowbreak 2000}
När Clas Isak Nylind, \textborn 09.02.1934, gift 1958 med Sonja Maria Tallbäck, \textborn 11.06.1937 i Kirkland Lake, Kanada, men uppväxt i Oravais, övertog fastigheten av dödsboet, efter Isak och Anna-Lovisa Nylind,var den ganska nergången. År 1985 lades nytt yttertak på både hus och uthus. Innerväggarna i bostaden fixades till och fasaderna putsades.

Sommartid vistades de ofta i sin lilla stuga, men sin yrkesgärning utförde de huvudsakligen i södra Finland. Clas är utbildad folkskollärare, men skaffade sig också kompetens för tjänst som skoldirektör. Till en början var han lärare vid en folkskola i Lovisa 1956-59, därefter lärare vid medborgarskolan i Pedersöre 1959-62 och på nytt i Lovisa 1962-73. Från 1973 var han t.f. skoldirektör för både svenska och finska skolor i Lovisa, men från 1974 fick han ordinarie tjänst som skoldirektör för Lovisa, Lappträsk, Liljendal, Mörskom, Pernå och Strömfors kommuner.

Sonja har bankutbildning och har bl.a. varit anställd på Sparbanken för Östra Nyland.


\jhoccupant{Nylind}{Isak \& Lovisa}{1934--\allowbreak 1964}
Isak Andersson (Nylind), \textborn 08.02.1862 på Böös, gift 1894 med Anna-Lovisa Henriksdr., \textborn 31.08.1869 på Gunnar, byggde år 1934 denna stuga till sin sytningsstuga efter att ha överlåtit hemmanet till sonen Jũrgen och hans hustru Signe (se Böös nr.......). Den bestod av kök, en kammare, farstu och en liten farstukammare. På vinden fanns ett litet rum med låg takhöjd där bl.a. barnbarnen kunde övernatta. Den övriga vinden var kallvind. Stugan var tidstypiskt röd med vita knutar.

En vinter under krigstiden bodde en evakuerad person i den lilla farstukammaren. Det är oklart om personen var från Karelen eller Kemijärvi. Där fanns då en vedspis och en säng, men antagligen var det en kall boplats då väggarna i just den kammaren bara utgjordes av oisolerad brädvägg. I övrigt var byggnaden av timrad stock.

Isak \textdied 03.10.1946  -- Anna-Lovisa \textdied 19.05.1964
Efter Anna-Lovisas död har fastigheten varit dödsbo fram till 1980.
