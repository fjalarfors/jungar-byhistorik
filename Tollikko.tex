\jhchapter{Tollikko, hemman Nr 11}

Namnet Tollikko kommer troligen från finskan och hänvisar till ett förfallet eller uselt boningshus. Det kan också betyda att någonting gått fel eller att man ``moka''. Gamla sägner har en  fantasifull förklaring: Till Tollikko kom en finngubbe vandrande. Så lade han ner konten på en plats och sade - ``Tässä nytt rupeen, tulkoon sitten töllikko tai tollikko'', (här börjar jag nu, det må sedan bli ett kyffe eller en tossa) och så fick platsen sitt namn!

Tollikko hemman har funnits med i mantalslängderna sedan 1570-talet. Gårdarna stod nära älven, där också den gamla vägen gick. Gårdstomterna på södra delen av Tollikko är högre belägna och har troligtvis varit bebodda en längre tid. Den uppodlade arealen var inte stor på 1700-talet, men redan 1708 nämns om en husbehovskvarn på Tollikko, vars plats man ännu kan finna spår av. Ängsmarker fanns det mera av vid den tiden.

Att också Tollikkoborna hade del i Loilax fiskevatten redan i mitten av 1500-talet, vittnar en omfattande bosättning redan då. En eldsvåda, som drabbade Matts Andersson 1730, visar på en riklig bebyggelse, endast 10 år efter Stora ofreden då det mesta blev skövlat och uppbränt. Enligt ett protokoll i Nykarleby tingslag förlorade Matts genom eldsvåda 1 sätesstuga, 1 liten gammal stuga, 1 bodstuga med förstuga, 1 liten källare med bod, 1 stall med loft över porten, 1 liten gammal bod. Det visar på en ansenlig mängd hus på ett hemman.

Genom århundraden har torparna på Tollikko varit många. Torparboställena fanns förutom i Pelkkala och Kihakoski också närmare. Ett ställe var Nörrbackan. Det talas också om Forshaga torp, Stenmossens torp och Vikmans backstuguområde, där flera stugor var samlade.

Tollikko hörde till soldattorp Nr 137, som genom tiden kallats Lafwast (1737), Hampbacka (1752) och Möllnars 1775. Soldatroten bestod av hemmanen Möhlnars (nr 7), Finskas (nr 29), Öfwerlafwast (nr 11) och Tollikko (nr 22). Själva torpet anges ha stått söder om Finskas gård, men Tollikko bestod ängar på ``Mjetar måsan'' med en uppskattad avkastning om 3 skrindor hö.

Industrialiseringen startade vid Keppo på 1700-talet och under detta århundrade inköptes ¼ av Tollikko hemman till Keppo. Orsaken var närmast tillgången på energi från den fors som fanns på hemmanets mark. Den sågs som en potential. 1786 såldes Tollikko och Keppo till Bengt Magnus Björkman, som redan då ägde Orisbergs herrgård i Sydösterbotten. Dennes son Lars adlades senare till Björkenheim. Tollikko följde med Keppo gårds alla ägarbyten ända fram till år 1899. Då bröts sammanlänkningen.

Under 1800-talet bytte de övriga delarna av Tollikko hemman ägare många gånger. Efter 1860 var ägarna ättlingar till släkten Elenius och efter 1899 har hela Tollikko hört till samma släkt.

Under 1900-talets första hälft gjordes försök att genom dammhöjning öka kraften i den då nybyggda kvarnen och en kort tid drev kvarnen också en generator som gav belysning i stugorna i området.

Ännu på 1950-talet fanns 7 aktiva brukningsenheter på Tollikko. I dag återstår 3 st.

Tollikko hemman omfattas av vidstående karta nr \jhbold{16}.


---> KARTA HIT


\jhsubsection{Lägenheter på Tollikko}
