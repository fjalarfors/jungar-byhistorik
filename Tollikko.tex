\jhchapter{Tollikko, hemman Nr 11}

Namnet Tollikko kommer troligen från finskan och hänvisar till ett förfallet eller uselt boningshus. Det kan också betyda att någonting gått fel eller att man ``moka''. Gamla sägner har en  fantasifull förklaring: Till Tollikko kom en finngubbe vandrande. Så lade han ner konten på en plats och sade - ``Tässä nytt rupeen, tulkoon sitten töllikko tai tollikko'', (här börjar jag nu, det må sedan bli ett kyffe eller en tossa) och så fick platsen sitt namn!

Tollikko hemman har funnits med i mantalslängderna sedan 1570-talet. Gårdarna stod nära älven, där också den gamla vägen gick. Gårdstomterna på södra delen av Tollikko är högre belägna och har troligtvis varit bebodda en längre tid. Den uppodlade arealen var inte stor på 1700-talet, men redan 1708 nämns om en husbehovskvarn på Tollikko, vars plats man ännu kan finna spår av. Ängsmarker fanns det mera av vid den tiden.

Att också Tollikkoborna hade del i Loilax fiskevatten redan i mitten av 1500-talet, vittnar en omfattande bosättning redan då. En eldsvåda, som drabbade Matts Andersson 1730, visar på en riklig bebyggelse, endast 10 år efter Stora ofreden då det mesta blev skövlat och uppbränt. Enligt ett protokoll i Nykarleby tingslag förlorade Matts genom eldsvåda 1 sätesstuga, 1 liten gammal stuga, 1 bodstuga med förstuga, 1 liten källare med bod, 1 stall med loft över porten, 1 liten gammal bod. Det visar på en ansenlig mängd hus på ett hemman.

Genom århundraden har torparna på Tollikko varit många. Torparboställena fanns förutom i Pelkkala och Kihakoski också närmare. Ett ställe var Nörrbackan. Det talas också om Forshaga torp, Stenmossens torp och Vikmans backstuguområde, där flera stugor var samlade.

Tollikko hörde till soldattorp Nr 137, som genom tiden kallats Lafwast (1737), Hampbacka (1752) och Möllnars 1775. Soldatroten bestod av hemmanen Möhlnars (nr 7), Finskas (nr 29), Öfwerlafwast (nr 11) och Tollikko (nr 22). Själva torpet anges ha stått söder om Finskas gård, men Tollikko bestod av ängar på ``Mjetar måsan'' med en uppskattad avkastning om 3 skrindor hö.

Industrialiseringen startade vid Keppo på 1700-talet och under detta århundrade inköptes ¼ av Tollikko hemman till Keppo. Orsaken var närmast tillgången på energi från den fors som fanns på hemmanets mark. Den sågs som en potential. 1786 såldes Tollikko och Keppo till Bengt Magnus Björkman, som redan då ägde Orisbergs herrgård i Sydösterbotten. Dennes son Lars adlades senare till Björkenheim. Tollikko följde med Keppo gårds alla ägarbyten ända fram till år 1899. Då bröts sammanlänkningen.

Under 1800-talet bytte de övriga delarna av Tollikko hemman ägare många gånger. Efter 1860 var ägarna ättlingar till släkten Elenius och efter 1899 har hela Tollikko hört till samma släkt.

Under 1900-talets första hälft gjordes försök att genom dammhöjning öka kraften i den då nybyggda kvarnen och en kort tid drev kvarnen också en generator som gav belysning i stugorna i området.

Ännu på 1950-talet fanns 7 aktiva brukningsenheter på Tollikko. I dag återstår 3 st.

Tollikko hemman omfattas av vidstående karta nr \jhbold{16}.


<--- se KARTA nr 16 --->


\jhsubsection{Lägenheter på Tollikko}



\jhhouse{Nyström}{25}{Tollikko}{16}{1, 1a}


\jhhousepic{252-05834.jpg}{Gunvi och Stig Jacobsson}

\jhoccupant{Jacobsson}{Stig \& Gunvi}{1969--}
Gunvi Marianne, \textborn 21.07.1948 född Nyström, gift 1968 med Stig Kaj Valter, \textborn 24.12.1947 Jacobsson från Jussila.
\begin{jhchildren}
  \item \jhperson{\jhbold{\jhname[Glen]{Jacobsson, Glen}} Kenneth}{20.03.1969}{}, ekonomiemagister
  \item \jhperson{\jhname[Nancy Jeanette]{Jacobsson, Nancy Jeanette}}{22.04.1972}{}, ped.mag. Gift Ackrén
  \item \jhperson{\jhbold{\jhname[Mats]{Jacobsson, Mats}}}{03.01.1980}{}, elmontör
\end{jhchildren}
Gunvi och Stig övertog hennes föräldrars hemman 1969. Bostaden 	byggde de 1970 åt sig, också Gunvis föräldrar bodde en tid i huset. 1973 påbörjades potatisodling. 1974 byggdes ett potatislager. 1980 byggdes en nötköttsladugård. 1982 på börjades överflyttningen av jordbruket till naturenlig odling, som det då hette, utan kemiska bekämpningsmedel och konstgödsel. Gården var då med de första i Österbotten för den inriktningen.

I början var efterfrågan på naturenliga produkter ganska liten. Tanken på mottot ``vi äger inte marken utan har den till låns av kommande generationer'' underlättade situationen under de första åren. I början på 1990-talet ökade intresset bland konsumenterna för ekologiska produkter, som det numera kallas. För att bredda försäljningen inleddes morotsodling 1993. I slutet av 90-talet kom även partiaffärerna med i spelet, något som vi väntat på i 20 år.

Stig, utbildad merkonom, har arbetat 7 år på Keppo, var med och startade Jeppo potatis, VD i 6 år från starten 1976. Stig var med som representant från Jeppo vid bildandet av IF Minken 1975, I styrelse och sektion var han sedan i 20 år, skidsektionens ordförande några år. Gunvi har gått i Evangeliska folkhögskolan och i yrkesskolan i Jakobstad, sedan giftermålet arbetat på hemgården.

År 2000 blev Glen och Mats delägare i lägenheten, tre år senare blev bröderna ägare till hela lägenheten. Under åren har lägenheten vuxit från 9 ha odlat 1969 till 140 ha idag. Ett skogsskifte köptes av den lägenhet på Ojas som deras morfar Joels föräldrar ägt och där han föddes. Av skiftet är idag 40 ha nyodlat. Jacobssons Gårdsbrukssammanslutning äger numera halva Tollikko hemman.

I packeriet finns tre robotliknande maskiner. En som packar potatis en packar morötter och en viker papperslådor. Den senaste utbyggnaden av lagren skedde 2016, då med 1200 m$^2$. Idag odlas morötter på 20 ha och potatis på 35 ha. Utvecklingen har varit möjlig tack vare konsumenternas ökade intresse för de ekologiska produkterna. Stigs förhoppning är att jordbruket utvecklas och blir mer ekologisk i framtiden. a = fähus, potatis- och morotslager.


\jholdhouse{Gamla huset Nyström}{25}{Tollikko}{16}{101, 101a}


\jhhousepic{Tollikko 101A.png}{Joel och Irene Nyström}

\jhoccupant{Nyström}{Joel \& Irene}{1945--\allowbreak 1969}
Joel, \textborn 25.08.1897, gift 1946 med Anna Irene, \textborn 01.11.1908, född Linder på Böös.

Barn: \jhbold{Gunvi} Marianne, \textborn 21.07.1948, gift Jacobsoon.

Joel var med i frihetskriget och i fortsättningskriget. Som 49-åring gifte han sig. Joel och Irene byggde sig en ny mangård 1946 och ekonomiebyggnad 1949. De bedrev mjölkproduktion som de flesta andra i grannskapet.

Irene hade gått i Korsholms husmoderskola åren 1931-32. Efter att dottern och mågen byggt ny gård på samma tomt, flyttade Joel och Irene in där. Joel dog 17.04.1990 och Irene som 102 åring 02.01.2011. a = fähus, boda, foderlada.



\jhhouse{Tollikko}{11:47}{Tollikko}{16}{2, 2a}


\jhhousepic{253-05835.jpg}{Per och Birgitta Elenius}

\jhoccupant{Elenius}{Per \& Birgitta}{1973--}
Per Johannes, \textborn 29.02.1940 på Gunnar, gift 1964 med Birgitta Saga Sofia \textborn 16.05.1944, född Södersved, Pensala.

Barn: Vyn Susann \textborn 27.05.1965, banktjänsteman, gift Nymark.

Per och Birgitta bodde åren 1965-72 i Sverige, Per arbetade på traktorfabriken Bolinder Munktell i Eskistuna och senare som busschaufför. Efter hemkomsten har han jobbat på Lindéns verkstad och senast som montör på BSB i Nykarleby. Birgitta har varit butiksbiträde i olika butiker i Jeppo i 20 års tid.

Lägenheten köpte de av fadern Uno Elenius 1971. Uno i sin tur köpte lägenheten på auktion 1966 av Sven Nyströms dödsbo. Bostaden de bor i byggde de åren 1973-74.  a = f.d. fähus numera lager och garage.



\jholdhouse{Nyström, inkl. hyresg.}{11:47}{Tollikko}{16}{102, 102a}


\jhhousepic{Tollikko 102A.png}{Sven och Joel Nyström}

\jhoccupant{Elenius}{Uno}{1966--\allowbreak 1971}
Uno köpte en del jord och bostaden och ekonomibyggnaden på auktion efter Svens död. Huset hade han uthyrt i repriser. Husen och tomtmarken såldes 1971 till Per Elenius.

Hyresgäster på gården:
\begin{enumerate}
  \item Bo och Marita Norrgård, 1966-okt. 1968
  \item Helmi och Kuisma Koski, 1968--\allowbreak 1969
  \item Nils och Anna  Elenius, 1970
\end{enumerate}
Eino Broos kiosk (se under Gunnar 29b) var placerad nära infarten till denna lägenhet.

\jhoccupant{Nyström}{Sven}{1932--\allowbreak 1963}
Sven, \textborn 15.08.1911, född Nyström. Han övertog halva hemmanet efter fadern. Andra halvan fick Joel. Han förblev ogift. Tillsammans med brorsonen Guy tillverkade de betongrör och taktegel. Däremellan var han byggnadsarbetare på Keppo. Sven dog 15.08.1963. a = uthusbyggnad.\jhvspace{}


\jhoccupant{Nyström}{Joel}{1932--\allowbreak 1946}
Efter några år och två gånger till Canada, stannade Joel hemma och blev ägare till andra delen av hemmanet, som bestod av 17 ha, varav odlad jord 9 ha. Efter kriget byggde han \jhbold{gård nr 101}, se ovan.



\jholdhouse{Tollikko}{11:34}{Tollikko}{16}{103}


\jhhousepic{Tollikko 103A.png}{Tollikko nr 103}

\jhoccupant{Nyström}{Anna-Lovisa \& Johannes}{1906--\allowbreak 1932}
Anna-Lovisa, \textborn 14.09.1873, född Gunnar (Jungarå), gift 1891 med Johannes Andersson (Tollikko), \textborn 08.10.1869 på Tollikko.
\begin{jhchildren}
  \item \jhperson{\jhbold{\jhname[Joel]{Nyström, Joel}}}{25.08.1897}{17.04.1990}
  \item \jhperson{\jhname[Hilda]{Nyström, Hilda}}{16.02.1899}{21.02.1917}
  \item \jhperson{\jhname[Johannes]{Nyström, Johannes}}{31.08.1900}{12.11.1986}
  \item \jhperson{\jhname[Anders]{Nyström, Anders}}{18.07.1904}{21.06.1922}
  \item \jhperson{\jhname[Leander]{Nyström, Leander}}{20.01.1906}{08.11.1974 i Vancouer}
  \item \jhperson{\jhname[Zaida]{Nyström, Zaida}}{24.01.1908}{08.02.2000}
  \item \jhperson{\jhname[Runar]{Nyström, Runar}}{27.09.1909}{27.09.1991}
  \item \jhperson{\jhbold{\jhname[Sven]{Nyström, Sven}}}{15.08.1911}{15.08.1963}
  \item \jhperson{\jhname[Anna Linnea]{Nyström, Anna Linnea}}{12.08.1913}{18.08.2002, gift Hällmark}
  \item \jhperson{\jhname[Ida Maria]{Nyström, Ida Maria}}{19.08.1914}{14.12.2007, gift Frände, Vörå}
\end{jhchildren}
Anna-Lovisa gifte sig som 18-åring med Johannes Tollikko. Den 11.02.1892 köpte Johannes och hans bror Anders av modern, änkan Maja-Lena Andersdotter 1/8 dels mantal av Tollikko för 3325 mk. Därefter reste Johannes till Amerika på arbetsförtjänst. Under hans bortavaro sålde Anna-Lovisa år 1895 gården till Thomas Lindström för 2000 mk. År 1899 sålde Lindström, som också förvärvat Anders' del (han dog 1892) av hemmanet, hela 1/8 dels mantal av Tollikko till Anna-Lovisas far, Matts Gunnar.
Efter hemkomsten från Amerika köpte Johannes en lägenhet på Ojala hemman. Där föddes de tre äldsta barnen. Efter en tid sålde de detta hemman och flyttade till Anna-Lovisas föräldrahem på Gunnar, 1906 då de av hennes bror Joel Elenius fick köpa 1/6 del av hans hemman för 3500 mk.

Anna-Lovisa \textdied 10.05.1916 i lungsot och Johannes blev ensam med sina många barn, på en gång både mor och far. Han ändrade namnet till \jhbold{Nyström} på 1920-talet. En tid var han också hönsfarmare och sålde ägg på Nykarleby torg. Johannes \textdied 02.06.1953.


\jhoccupant{Eriksson (Tollikko)}{Anders \& Maja-Lena}{1873--\allowbreak 1906}
Anders Gustaf Eriksson Böös (Tollikko), \textborn 05.10.1839, gift med Maja-Lena Andersdotter, \textborn 06.09.1838-191?.
\begin{jhchildren}
  \item \jhperson{\jhname[Erik]{Eriksson (Tollikko), Erik}}{25.11.1861}{24.03.1895}
  \item \jhperson{\jhname[Anders]{Eriksson (Tollikko), Anders}}{15.11.1864}{08.04.1892}
  \item \jhperson{\jhname[Maria]{Eriksson (Tollikko), Maria}}{16.12.1866}{}
  \item \jhperson{\jhname[Johannes]{Eriksson (Tollikko), Johannes}}{08.10.1869}{02.06.1953}
\end{jhchildren}
Anders Gustaf Tollikko köpte av Samuel Roos 1873 1/8 del av Tollikko 	hemman. Samuel igen hade varit i borgen för sin bror Gabriel och 1860 övertagit hans hemman för 675 rubel, andra borgesmannen Isak Jungarå tog över andra 1/8 delen. Gabriel, eller Gabb, och Maj fortsatte som torpare på samma hemman. Sista tiden som backstugosittare.

Gabb \textborn 15.12.1834--\textdied 17.12.1919 och Maj \textborn 14.12.1835--\textdied 10.02.1919.

Kan nämnas att Anders Eriksson (Tollikko) föräldrar var Erik	Mattsson Böös och Brita Stina Andersdotter Forsberg, syster till Anna Helena, \jhbold{gård 20}.



\jhhouse{Jacobsson}{11:55}{Tollikko}{16}{6}


\jhhousepic{254-05836.jpg}{Glen och Camilla Jacobsson}

\jhoccupant{Jacobsson}{Glen \& Camilla}{2003--}
Glen, \textborn 20.03.1969, gift 11.06.2005 med Camilla, \textborn 15.10.1972, född Sacklén.
\begin{jhchildren}
  \item \jhperson{\jhname[Johannes Glen Valter]{Jacobsson, Johannes Glen Valter}}{30.06.2004}{}
  \item \jhperson{\jhname[Signe Anni Linnea]{Jacobsson, Signe Anni Linnea}}{12.05.2006}{}
  \item \jhperson{\jhname[Willhelm Caj Erik]{Jacobsson, Willhelm Caj Erik}}{19.05.2010}{}
\end{jhchildren}

Glen köpte tomten till sitt hus av Erik Elenius och byggde huset 2003. Camilla har haft frisörssalong i huset en tid, arbetar numera på jordbruket och i potatis- och morotspackeriet.

Glen var framgångsrik skidlöpare i yngre dar med många medaljer från Fss-mästerskapen och en silvermedalj i Fm för herrar under 20 år. Numera är Glen ordf. i Jeppo IF och hela familjen engagerad i träningar och tävlande.



\jhhouse{Tommos}{11:31}{Tollikko}{16}{4, 4a}


\jhhousepic{256-05839.jpg}{Curt och Monica Elenius}

\jhoccupant{Elenius}{Curt \& Monica}{1970--}
Curt Gunnar, \textborn 02.08.1947, gift 1971 med Monica Heldine Charlotta, \textborn 25.10.1948, född Thel i Soklot.
\begin{jhchildren}
  \item \jhperson{\jhname[Johan Mikael]{Elenius, Johan Mikael}}{08.01.1973}{}, elingenjör på Jeppo Kraftandelslag
  \item \jhperson{\jhname[Peter Joakim]{Elenius, Peter Joakim}}{19.05.1975}{}, husbyggare med egen firma
  \item \jhperson{\jhname[Fredrik Curt]{Elenius, Fredrik Curt}}{21.12.1983}{}, elmontör
\end{jhchildren}

Curt arbetade en tid i Sverige som kranförare, återvände varefter makarna blev jordbrukare 1970. Parallellt har Curt jobbat som byggnadsarbetare. De slutade med mjölkproduktionen och satsade på grisproduktion och potatisodling. Samtidigt var Curt rävfarmare.

Monica har arbetat på Jeppo potatis och på den egna gården. Curt och Monica byggde nytt bostadshus 1976. Åkermarken såldes till Glen och Mats Jacobsson 2014. a = fähus.



\jhhouse{Tommos}{11:31}{Tollikko}{16}{5}


\jhhousepic{255-05837.jpg}{Elenius C och M, sommarbostad}

\jhoccupant{Elenius}{Curt \& Monica}{1996--}
Curt är ägare till gamla hemgården sen föräldrarna avlidit. Huset är sommarbostad för systern Gun-Britt och Rickard Holster.\jhvspace{}


\jhoccupant{Elenius}{Gunnar \& Adele}{1935--\allowbreak 1966}
Gunnar Johannes, \textborn 19.07.1906, gift 1935 med Adele Maria, \textborn 13.11.1911, född Ström på Stenbackan.
\begin{jhchildren}
  \item \jhperson{\jhname[Gunn-Britt Maria]{Elenius, Gunn-Britt Maria}}{23.11.1936, gift Holsteri Sverige}{}
  \item \jhperson{\jhname[Ruth Magnhild Helena]{Elenius, Ruth Magnhild Helena}}{24.04.1940, gift Nordling, Jakobstad}{}
  \item \jhperson{\jhname[Vivian Kristina]{Elenius, Vivian Kristina}}{23.07.1945, gift Axelsson, Sverige}{}
  \item \jhperson{\jhbold{\jhname[Curt]{Elenius, Curt}} Gunnar}{02.08.1947}{}
\end{jhchildren}
Gunnar gick i Korsholms lantmannaskola läsåret 1928-29. Bostaden byggdes 1933 och ekonomibyggnaden året därpå. Samma år som äktenskapet ingicks, övertog han en tredjedel av föräldrahemmanet av föräldrarna. Jordbruket bestod av 15 ha åker och lika mycket skog. Bröderna Lennart, Gunnar och Birgers mor dog 18.10.1914.

I ett testamente från 12.10.1914 skrivs följande: ``Jag undertecknad förklarar här min yttersta vilja vara att sedan jag 	med döden avgått min efterlämnade egendom skall delas sålunda att mina söner Lennart, Gunnar och Birger skola envar erhålla en tredjedel af den fasta och lösa egendomen sedan sonen Manne Ragnar erhållit två tusen mark och dottern Ellen Katarina erhållit femhundra mark utöver de ettusen mark hon är försäkrad till i bolaget Kataja. Förenämnda egendom och belopp skall till envar utbetalas ur boet vid af dem ingånget äktenskap eller myndig ålder. -- Jeppo den 12 okt 1914, Hilda Johanna Elenius. ( bomärke).''

Sedermera blev Gunnar och Birger ägare till en tredjedel var och till Lennart köpte fadern Södergård hemman. Det blev som Hilda bestämde i testamentet fast barnen var endast 3-9 år gamla när hon dog.



\jholdhouse{Krooks}{11:12}{Tollikko}{16}{105}


\jhoccupant{Krooks}{Vilhelm \& Maria Adolfina}{1921--\allowbreak 1936}
Maria Adolfina, \textborn 03.03.1891, född Johansdotter, \textdied 	16.04.1934.	Vilhelm dog redan tidigare.

Barn: \jhname[Inga Wilhelmina]{Krooks, Inga Wilhelmina} \textborn 22.05.1922 i Jeppo, utflyttad till Jakobstad 26.12.1938. Gift Junnila.

I januari 1921 flyttade familjen in ett hus på Johan Vikmans backstuguområde (området fanns där nuvarande stugan nr 11 	finns.) Legonämnden ansåg att han brukat området för kort tid för att kunna inlösa densamma, varför legogivaren erbjöd ett annat område om 2 kappland för ``ervärdelig'' tid, en bit norrut längs landsvägen. Dit flyttades huset och boden 1921.

Vid fastställande av fastan (lagfarten) den 23.02.1926, varom erfars att arbetaränkan Maria Kroks och hennes två omyndiga barn då erhållit det första och enda uppbådet på Krook lägenhet. Vilhelm hade dött innan de fick fasta på lägenheten. År 1934 dog modern och inga omhändertogs av andra släktingar.

Efter att vårdnadsnämnden övertagit vårdnaden, beslöts att övertaga familjens kvarlåtenskap bestående av Krook 11:12 benämnda bostadstomt. Daterat 13.01.1936. Huset flyttat till Pelkkala 1935. Ivar Jungell köpte lägenheten av kommunen 1951 för 5000 mark.



\jholdhouse{Byggning 104}{11:12}{Tollikko}{16}{104}

\jhhousepic{Tollikko 104.jpg}{Tollikko-Jungarå gemensamma brandspruta}

\jhnooccupant{}
På platsen 104 fanns byggnaden för brandsprutan, gemensam för Tollikko och Jungarå. Normalt var dylika byggnader enkla uthus eller skjul. Emedan bild på huset saknas, får läsaren använda sin fantasi och nöja sig med objektet som det hyste.\jhvspace{}



\jhhouse{Nygård}{11:39}{Tollikko}{16}{3}

\jhoccupant{Elenius}{Erik \& Ann-Britt}{2000--}
Erik och Ann-Britt är gårdens nuvarande ägare. Huset bebos sommartid av Eriks systrar Dorrit och Eva och deras familjer.\jhvspace{}


\jhhousepic{259-05841.jpg}{Lennart och Agnes Nygård; Erik Elenius}

\jhoccupant{Nygård}{Lennart \& Agnes}{1944--\allowbreak 1975}
Lennart Eliel, \textborn 05.06.1917, gift 1944 med Anna Agnes Julia, \textborn 21.05.1924, född Elenius. Agnes och Lennart övertog en tredjedel av hennes föräldrars hemman på Tollikko. Det omfattade 23 ha varav 10 ha odlad jord.
\begin{jhchildren}
  \item \jhperson{\jhname[Gunvor Margareta]{Nygård, Gunvor Margareta}}{19.04.1946, gift Eklund bor i Närpes}{}
  \item \jhperson{\jhname[Gunda Marianne]{Nygård, Gunda Marianne}}{27.11.1949, gift Adams bor i Nykarleby}{}
  \item \jhperson{\jhname[Karin Elisabet]{Nygård, Karin Elisabet}}{31.10.1959, gift Mattjus i Purmo}{}
\end{jhchildren}

\jhhousepic[pic:boda]{Boda fr 1727.jpg}{Den gamla bodan från 1727}

Lennart blev svårt sårad i kriget med rakt ben som följd, men var ändå aktiv jordbrukare. Han var också ett antal år mejerist vid Jungar och Jeppo andelsmejeri. År 1975 såldes hemmanet till Erik och Ann-Britt Elenius, också hemgården såldes till Erik. a = fähus, foderlada, b = garage, c = boda från 1727, se bild.

Lennart \textdied 26.10.1988 --- Agnes \textdied 13.05.2016\jhvspace{}

\jhoccupant{Elenius}{Joel \& Hilda Johanna}{1904--\allowbreak 1944}
Matts Joel, \textborn 08.02.1884, född Gunnar senare Elenius, gift 1903 med Hilda Johanna, \textborn 08.07.1885, född Jungarå.
\begin{jhchildren}
  \item \jhperson{\jhname[Joel Lennart]{Elenius, Joel Lennart}}{08.10.1904}{07.02.1973}
  \item \jhperson{\jhname[Gunnar Johannes]{Elenius, Gunnar Johannes}}{19.07.1906}{01.04.1995}
  \item \jhperson{\jhname[Ellen Katarina]{Elenius, Ellen Katarina}}{23.04.1908}{20.07.1959}
  \item \jhperson{\jhname[Anders Birger]{Elenius, Anders Birger}}{24.02.1911}{31.05.2002}
  \item \jhperson{\jhname[Manne Ragnar]{Elenius, Manne Ragnar}}{04.03.1914}{24.01.1915}
\end{jhchildren}


Gift 2:a gången med Anna Lovisa, \textborn 22.02.1890, född Jungar.
\begin{jhchildren}
  \item \jhperson{\jhname[Erik Edvin]{Elenius, Erik Edvin}}{17.11.1917}{11.03.1940, stupade i strid}
  \item \jhperson{\jhname[Ragnar Osvald]{Elenius, Ragnar Osvald}}{04.03.1920}{20.07.1920}
  \item \jhperson{Anna \jhbold{Agnes} Julia}{21.05.1924}{13.05.2016}
\end{jhchildren}
Joel gick i Kronoby folkhögskola åren 1902-03. Han gifte sig som nittonåring. Joel och Hilda fick 1/4 dels mantal av Tollikko hemman genom gåvobrev 16.12.1904 av sin far Matts Gunnar. 1906 sålde han en tredjedel till sin syster Anna-Lovisa. År 1928 köpte han Södergård hemman till äldsta sonen Lennart. Joel var många år ledamot och ordförande i kommunfullmäktige i Jeppo, ledamot i lantmannagillet och skoldirektioner. Han dog i sviterna av lunginflamation 03.01.1935. Anna bodde vidare i huset tillsammans med dottern och hennes familj. Hon dog 21.04.1979.


\jhoccupant{Gunnar}{Matts \& Kajsa}{1873--\allowbreak 1904}
Matts Gunnar fick 1/4 dels mantal av Tollikko hemman genom testamente av sin far 22.02.1873, också en del av Gunnar hemman. Matts bodde hela tiden på Gunnar hemman.

Matts \textdied 02.11.1911  --- 	Kajsa \textdied 16.03.1934


\jhoccupant{Jungarå}{Isak \& Caisa}{1860--\allowbreak 1873}
Gabriel Roos ägde 1/8 dels mantal av Tollikko hemman som han köpte 29.11.1856 för 950 rubel silver, 09.03.1858 köp av 1/16 del för 325 rubel, 28.03.1859 sålde han 1/8 del, 16.12.1859 köp av en annan 1/8 del för 880 rubel. Eftersom Isak var i propieborgen för skulderna till Rådman Lybäck för sin halvbrorsson tillsammans med Samuel Roos (broder), övertog de hemmansdelarna. Före Gabriel Roosas hemmansaffärer fanns åtminstone fem ägare på tio år. 1847 sålde prästsonen Johan Thomasson Elenius hemmansdelarna och flyttade till Smedbacka hemman i Nykarleby.

När bodan \ref{pic:boda} byggdes 1727 var en Matts Andersson och hustrun Margareta bönder på Tollikko hemman, de återkom till Tollikko efter stora ofreden 1723. Jfr ovan under Lennart och Agnes Nygård.

Samma Matts Andersson drabbades av eldsvåda 07.10.1730 och förlorade en sätesstuga värd 18 daler, en liten gammal stuga 8 daler, en ny främmandstuga med tillbehör 60 daler, en bodstuga med förstuga 12 daler, en liten källare med bod 6 daler, ett stall med loft över porten 25 daler, en liten gammal bod 3 daler, 6 tunnor 24 kappar spannmål 54 daler, 8 skrindar halm 16 daler, en egendom värd 202 daler kopparmynt gick upp i rök.



\jhhouse{Strand}{11:59}{Tollikko}{16}{7, 7a}


\jhhousepic{257-05840.jpg}{Anders och Charlotta Elenius}

\jhoccupant{Elenius}{Anders \& Charlotta}{2014--}
Anders Håkan, \textborn 30.06.1984, gift 08.08.2009 med Charlotta Maria Elisabet, \textborn 11.08.1984, född Näse.
\begin{jhchildren}
  \item \jhperson{\jhname[Iselinn Naema]{Elenius, Iselinn Naema}}{29.09.2011}{}
  \item \jhperson{\jhname[Eddie Erik Wilhelm]{Elenius, Eddie Erik Wilhelm}}{26.08.2014}{}
  \item \jhperson{\jhname[Alwa Agnes Elisabet]{Elenius, Alwa}}{12.07.2017}{}
\end{jhchildren}
Anders och Charlotta har renoverat och förstorat gården med en våning till. Familjen har flyttat in i sitt nya hem i slutet av 2016. Anders är bilmontör och arbetar i Jakobstad. Charlotta är utbildad utvecklingspsykolog och teolog, arbetar på Härmän kuntokeskus som psykolog.


\jhoccupant{Elenius}{Erik \& Ann-Britt}{1966--\allowbreak 2014}
Erik Håkan, \textborn 07.10.1946, gift 30.06.1968 med Ann-Britt Helena, \textborn 06.01.1947, född Paulin i Vörå.

Barn: \jhbold{Anders} Håkan, \textborn 30.06.1984, bilmontör

Erik har gått i Lannäslunds lantmannaskola 1964--65. Efter giftermålet fortsatte de med mjölkkor, ändrade senare till slaktnöt samt potatisodling. Erik började köra ut virke med skogstraktor i slutet på 1960-talet och höll på med det till pensioneringen. Ann-Britt har arbetat på Jeppo potatis och som butiksbiträde och på den egna gården. Lägenheten förstorades med köp av Lennart och Agnes Nygårds del, och var därefter ca 53 ha totalt. Bostadsbyggnaden byggdes 1983. Åkermarken såldes till Glen och Mats Jacobsson 2010. a = fähus, potatislager.



\jhhouse{Strand}{11:59}{Tollikko}{16}{8, 8a}


\jhhousepic{258-05843.jpg}{Birger och Agnes Elenius; Erik Elenius}

\jhoccupant{Elenius}{Birger \& Agnes}{1937--\allowbreak 1966}
Anders Birger, \textborn 24.02.1911, gift 1939 med Agnes Katarina, \textborn 19.04.1918, född Sandås, Böös.
\begin{jhchildren}
  \item \jhperson{\jhname[Eva Birgitta]{Elenius, Eva Birgitta}}{30.01.1941}{}, gift Henriksson, bor i Sverige
  \item \jhperson{\jhname[Dorrit Marita]{Elenius, Dorrit Marita}}{14.01.1943}{}, gift Wägar, bor i Sverige
  \item \jhperson{\jhbold{\jhname[Erik]{Elenius, Erik}} Håkan}{07.10.1946}{}
\end{jhchildren}

Birger gick i Korsholms lantmannaskola läsåret 1932-33. Han övertog en tredjedel av föräldrajordbruket 1937, byggde bostaden samma år och ekonomibyggnaden året därpå. Birger och Agnes bedrev mjölkproduktion. Erik och Ann-Britt bodde på vindsvåningen tills de flyttade in i eget hus 1983. Agnes dog 13.04.1989 och Birger bodde de sista åren på Nykarleby sjukhem, han dog 31.05.2002. a = fähus, foderlada. Huset revs sommaren 2017.



\jhhouse{Jungell}{11:50}{Tolliko}{16}{9}


\jhhousepic{Tollikko 9.jpg}{Svens arbetsresultat är njutbart}

\jhoccupant{Jungell}{Sven}{1993--}
Sven Anders, \textborn 11.05.1947,  köpte gården 1993 av broder Ragnar när de flyttade bort. Sven är byggnadsarbetare och friluftsmänniska som helst cyklar till arbetet sommar som vinter. Han håller sin stora tomt i perfekt skick och nere vid älven har han anlagt broar, stensättningar och stigar. Han är en ivrig vinterbadare. Tvillingbrodern Lars bodde några år i samma stuga innan han dog.

Sven har varit en mycket uppskattad byggnadsarbetare i nejden. Han har aldrig behövt söka efter arbete --- arbetet har sökt efter honom.


\jhhousepic{260-05844.jpg}{Sven Jungell}

\jhoccupant{Jungell}{Ragnar \& Britta}{1986--\allowbreak 1993}
Tor Ragnar, \textborn 17.12.1944, gift 1982 med Britta Kerstin, \textborn 26.12.1950, född Bergström från Övermark.
\begin{jhchildren}
  \item \jhperson{\jhname[Marcus Benjamin]{Jungell, Marcus Benjamin}}{17.02.1987}{}
  \item \jhperson{\jhname[Carl David]{Jungell, Carl David}}{03.02.1991}{}
\end{jhchildren}
Huset byggdes 1986 är ett Honka stockhus som uppfördes på tomten med utsikt över forsen i älven.

Ragnar hade tillsammans med brodern Erik övertagit lägenheten 1979 som då omfattade 61 ha varav 29 ha odlat. På gården bedrevs mjölkproduktion, och Ragnar var särskilt intresserad av att genom avelsarbete förbättra besättningar. 1982 förstorades fähuset till 50 kor i lösdrift. Vid sidan av jordbruksarbetet skrev Ragnar dikter och gav ut flera böcker, senare skulpterade han djur, från små till stora älgar.

Britta är utbildad hum. kand. och hade tjänst bl.a. som lärare i Nykarleby. Familjen flyttade från Jeppo 1993 till olika orter för att till slut stanna och bosätta sig i Närpes.



\jhhouse{Koski}{11:52}{Tollikko}{16}{10, 10a-b}


\jhoccupant{Jungell}{Mikael}{2013--}
Mikael, \textborn 22.04.1983 i Jeppo, övertog jordbruket 2013 och har potatisodling som huvudnäring. Forna fähuset är omgjort till potatislager.\jhvspace{}


\jhhousepic{261-05846.jpg}{Mikael Jungell sedan 2013}

\jhoccupant{Jungell}{Erik \& Sonja}{1978--\allowbreak 2013}
Erik, \textborn 07.04.1953 i Jeppo, gift 1982 med Sonja,\textborn 13.08.1956, född Backström från Esse.
\begin{jhchildren}
  \item \jhperson{\jhbold{\jhname[Mikael]{Jungell, Mikael}} Andreas}{22.04.1983}{}
  \item \jhperson{\jhname[Isabella Nina Maria]{Jungell, Isabella Nina Maria}}{17.06.1985, frisör/kosmetolog}{}
  \item \jhperson{\jhname[Charlotta Sabina Veronica]{Jungell, Charlotta Sabina Veronica}}{20.05.1987, sömmerska, tradenom}{}
  \item \jhperson{\jhname[Daniel Sebastian Manuel]{Jungell, Daniel Sebastian Manuel}}{10.01.1990, svetsare}{}
  \item \jhperson{\jhname[Florian Kim Jonathan]{Jungell, Florian Kim Jonathan}}{29.03.1992, svetsare}{}
  \item \jhperson{\jhname[Victoria Jasmine]{Jungell, Victoria Jasmine}}{01.06.1996}{}
  \item \jhperson{\jhname[Natalie Simone Iselin]{Jungell, Natalie Simone Iselin}}{02.02.2001}{}
\end{jhchildren}

Erik övertog jordbruket tillsammans med brodern Ragnar 1978. 	Ekonomibyggnaden förstorades 1981 med plats för 50 mjölkkor och bl.a. datastyrd utfodring. 1993 övertog Erik \& Sonja hela lägenheten. Han renoverade hemgården 1995 och byggde ett separat pannrum och garage. Mjölkproduktionen och djurhållningen avslutades år 2002. I hemgården bodde också brodern Lars, som var pälsdjursuppfödare, senaste tiden bodde han i \jhbold{hus nr 9}, han dog 11.07.2003.


\jhoccupant{Jungell}{Ivar \& Aili}{1943--\allowbreak 1978}
Ivar, \textborn 14.03.1913 på Tollikko, gift med Aili Amalia, \textborn 19.09.1916, född Sipponen.
\begin{jhchildren}
  \item \jhperson{\jhname[Lisbet Helena]{Jungell, Lisbet Helena}}{17.09.1943}{09.02.2008}, hon var sjukskötare och hade arbetat förutom i Finland i Danmark, Schweiz och sedan 1997 i Melbourne och Sydney, Australien
  \item \jhperson{Tor \jhbold{Ragnar}}{17.12.1944}{}
  \item \jhperson{\jhname[Lars Henrik]{Jungell, Lars Henrik}}{10.05.1947}{11.07.2003}
  \item \jhperson{\jhname[Sven Anders]{Jungell, Sven Anders}}{11.05.1947}{}
  \item \jhperson{Hans \jhbold{Erik}}{07.04.1953}{}
\end{jhchildren}
1932--33 var Ivar elev i Evangeliska folkhögskolan vid Keppo. Ivar och Aili övertog föräldrarnas hemman 1943 sedan de köpt ett hemman på Holmen åt brodern Birger. Makarna ägnade sig åt mjölkproduktion och byggde 1954 en ny ladugård i tegel.


\jhoccupant{Jungell}{Henrik \& Hilda}{1907--\allowbreak 1943}
Henrik, \textborn 07.10.1880 på Jungarå, gift med Hilda Maria, \textborn 09.08.1883, född Julin.
\begin{jhchildren}
  \item \jhperson{\jhname[Johannes Artur]{Jungell, Johannes Artur}}{25.09.1907}{14.02.1908}
  \item \jhperson{\jhname[Saima Emilia]{Jungell, Saima Emilia}}{22.12.1908}{11.03.2002}, gift med Hugo Sjö
  \item \jhperson{\jhname[Henrik Birger]{Jungell, Henrik Birger}}{31.05.1910}{24.11.1970}
  \item \jhperson{Tor \jhbold{Ivar}}{14.03.1913}{20.10.1990}
  \item \jhperson{\jhname[Hellin Maria]{Jungell, Hellin Maria}}{21.01.1915}{14.10.1928}
  \item \jhperson{\jhname[Venny Irene]{Jungell, Venny Irene}}{12.03.1917}{}, gift med Uno Sandberg
\end{jhchildren}
Henrik föddes Jungarå men tog senare namnet Jungell, efter några år i Amerika kom han hem och köpte 1907 tillsammans med sin äldre bror Isak 1/4 dels mantal av Tollikko hemman av deras farbroder Matts Gunnar. Henrik och Hilda bodde första tiden efter giftermålet i den gamla patricier-mangården, men byggde en ny gård åt sig 1926. Henrik sålde huset till Paul Ståhl, som rev, flyttade och timrade upp huset i Silvast, gård nr 380a. Henrik var farbror till Pauls hustru Hilda Sofia Jungarå, \textborn 12.05.1898, \textdied 05.11.1924. En annan gård flyttades också från Tollikko till Ruotsala till Jacob Nylund, till vars hustru Henrik var morbror, gård nr 22, karta 13.


\jholdhouse{Gamla gården Koski}{11: }{Tollikko}{16}{110, 110a-c}


\jhoccupant{Gunnar}{Matts}{1899--\allowbreak 1906}
Matts Gunnar köpte Tollikko hemman 1899, brukade gården men bodde på Gunnar. I och med köpet var han ägare till hela Tollikko nr 11 och delar av Gunnar hemman nr 10.


\jhoccupant{von Essen}{Otto Henrik}{1876--\allowbreak 1899}
Otto Henrik von Essen hade boenderätt i patricierhuset och bodde där med sin familj, hustrun Amanda och 4 döttrar Judith, Dagmar, Saima och Linda åren 1899-26.03.1907. Familjen flyttade därefter till Nykarleby.

Possessionaten Otto Henrik von Essen gift med Amanda Emilia, född Krank. Ca 1870 flyttade familjen von Essen till Tollikko på grund av brand vid Keppo. Enligt sägner byggdes patriciergården av halva Elenius prästgård, som ombyggdes på 1850-talet.

Otto von Essen arbetade mycket för \jhbold{skolväsendet} i Jeppo kommun. Han hade många medarbetare men också motståndare för byggandet av en folkskola. Då ingenting gjordes, anhöll han om att få hyra sin gård på Tollikko som folkskola. Han fick bifall om öppnandet, året var 1878. Det blev den första folkskolan i Jeppo. Den första läraren var Henrik Backlund 1878--79, han blev dimitterad från seminariet i Nykarleby som första årskull. Andra läraren var Johannes Sjöblad, dimitterad 1879, var lärare åren 1879--82, senare fick han tjänst på järnvägen. Enligt Hilda Jungarå hölls handarbetesundervisningen vid ``adesto''.

Denna del av Tollikko har hört till Keppo sedan 1700-talet. \jhname[Bengt Magnus Björkman]{Björkman, Bengt Magnus} från Stockholm köpte hemmanet 1786. Han var ägare till Orisberg blev senare adlad och hette Björkenheim. År 1828 köpte Nykarleby-borgaren och handlaren Gustaf Adolf Lindqvist Keppo och Tollikko. Under hans tid var det mycket som förföll.


\jhoccupant{von Essen}{Carl Otto}{1838--\allowbreak 1876}
Carl Otto von Essen och Hustru Anna Kristina, född Snellman, köpte Keppo och Tollikko år 1838. Under hans tid rustades Keppo upp, detsamma gällde säkert också Tollikko, som hela tiden sköttes av drängar och torpare. Carl Otto dog 1876 och sonen kom tillbaka till Keppo med familjen några år tidigare och fortsatte med 	verksamheten. a = fähus, b = bastu, c = ria.



\jhhouse{Elenius}{11:19}{Tollikko}{16}{11, 11a-c}


\jhhousepic{262-05847.jpg}{Magnus och Kristina Elenius}

\jhoccupant{Elenius}{Magnus \& Kristina}{1974--\allowbreak 2013}
Magnus, \textborn 20.08.1953 i Jeppo, gift sig den 12.12.1975 med Kristina, \textborn 04.12.1958, född Sandell i Kronoby. Magnus övertog, efter att gått i Korsholms lantmannaskola åren 1972-73, jordbruket 1974. Tillsammans med hustrun Kristina bedriver de mjölkproduktion.
\begin{jhchildren}
  \item \jhperson{Steve \jhbold{Jan-Olav}}{01.04.1976}{}, se gård nr 13
  \item \jhperson{\jhname[Terese Mikaela]{Elenius, Terese Mikaela}}{01.03.1978}{10.04.1978}
  \item \jhperson{\jhname[Ulrika Anne-Maria]{Elenius, Ulrika Anne-Maria}}{17.01.1979, gift Päivärinta}{}
  \item \jhperson{\jhname[Matias Dan Mikael]{Elenius, Matias Dan Mikael}}{27.05.1983}{}
  \item \jhperson{\jhname[Jenny Sofia]{Elenius, Jenny Sofia}}{25.07.1985, gift Aarrekangas}{}
  \item \jhperson{\jhname[Emma Kristina]{Elenius, Emma Kristina}}{08.04.1988}{}, sambo Anders Backlund
  \item \jhperson{\jhname[Ida Maria]{Elenius, Ida Maria}}{15.09.1989, gift Alvik}{}
  \item \jhperson{\jhname[Tobias Emanuel]{Elenius, Tobias Emanuel}}{22.06.1993}{}
  \item \jhperson{\jhname[Daniel André]{Elenius, Daniel André}}{16.05.1996}{4.05.2015 i trafikolycka}
\end{jhchildren}
Ekonomibyggnaden har förstorats i flera omgångar, 1998 förstorades ladugården med plats för 50 kor. En separat verkstadsbyggnad byggdes år 1993. Magnus och Jan-Olav inledde stegvis generationsväxling 2003. Från år 2013 övergick hela hemmanet i Jan-Olavs och Maarets ägo. a = fähus,foderförråd, b = gårdsverkstad, c = maskinlada.


\jhoccupant{Elenius}{Leo \& Helmi}{1945--\allowbreak 1974}
Leo byggde bostaden 1969, gård nr 11. När Magnus övertog hemmanet 1974 bodde de i gårdens östra flygel. År 1982 flyttade Leo och Helmi till en lägenhet i Silvast. Brodern Rune bodde ytterligare några år i huset förrän han också flyttade till lägenhet i Silvast.



\jholdhouse{Gamla gården Elenius}{11:19}{Tollikko}{16}{111, 111a-d}


\jhhousepic{Tollikko 111.jpg}{Leo och Helmi Elenius}

\jhoccupant{Elenius}{Leo \& Helmi}{1945--\allowbreak 1969}
Leo, \textborn 09.04.1916 på Tollikko, gift 1945 med Helmi Irene, \textborn 04.10.1923, född Grägg i Vörå.
\begin{jhchildren}
  \item \jhperson{\jhname[Leo Runar Johannes]{Elenius, Leo Runar Johannes}}{28.08.1946}{02.03.1950}
  \item \jhperson{\jhname[Mary Solveig Irene]{Elenius, Mary Solveig Irene}}{29.04.1948, gift Överfors}{}
  \item \jhperson{\jhname[Rune Christer Leander]{Elenius, Rune Christer Leander}}{11.11.1950}{}
  \item \jhperson{\jhbold{\jhname[Magnus]{Elenius, Magnus}} Olav Emanuel}{20.08.1953, gård nr 11}{}
  \item \jhperson{\jhname[Stefan Carl-Gustav]{Elenius, Stefan Carl-Gustav}}{29.05.1959}{}
\end{jhchildren}
Leo och Helmi övertog hemgården efter giftermålet. Den omfattade 68 ha, varav 30 ha odlad jord. 1951 byggde de en ny ladugård i tegel för 16 kor. Leo har tillhört kommunalfullmäktige och olika nämnder. I hushållet bodde också fadern Leander till sin död 17.08.1969. 1969 byggdes på tomten en ny gård, nr 11. Hus nr 111 revs 1977.


\jhoccupant{Elenius}{Leander \& Hilda}{1911--\allowbreak 1945}
Leander, \textborn 06.02.1886 på Jungarå. Sedan tog fadern Isak namnet Ruotsala, dit familjen Isak Elenius flyttade när fadern Johan Jungarå köpte ett 5/48 dels hemman på Ruotsala till äldsta sonen 1889. Leander gifte sig 1906 med Hilda Maria, \textborn 22.01.1888, född Sandberg på Finskas.
\begin{jhchildren}
  \item \jhperson{\jhname[Isak]{Elenius, Isak}}{06.04.1907}{21.05.1907}
  \item \jhperson{\jhname[Göta]{Elenius, Göta}}{11.09.1908}{24.04.1924}
  \item \jhperson{\jhname[Gerda]{Elenius, Gerda}}{21.01.1911}{21.12.1983, gift med Axel Sjöberg}
  \item \jhperson{\jhname[Hjördis]{Elenius, Hjördis}}{14.12.1913}{13.12.1947, gift med Tor Valfrid Julin}
  \item \jhperson{\jhbold{\jhname[Leo]{Elenius, Leo}}}{09.04.1916}{04.12.2000}
  \item \jhperson{\jhname[Judit]{Elenius, Judit}}{30.05.1918}{}, gift i Sverige med Karl-Ivar Johansson
  \item \jhperson{\jhname[Eskil]{Elenius, Eskil}}{16.05.1920}{17.05.1920, samma dag som sin mor}
  \item \jhperson{\jhname[Irja Lyydia]{Elenius, Irja Lyydia}}{16.05.1921}{23.10.2008} Modern var Selma Huhta, \textborn 27.01.1895, som arbetade på gården.
\end{jhchildren}
Efter att Leander och Hilda Maria gift sig bodde de första åren i hemgården på Ruotsala. 1907 köpte hans föräldrar Tollikko och Koski hemman tillsammans med brodern Henrik av Matts Gunnar, han i sin tur var farbror till Henrik och Leanders förälder Isak. Matts Gunnar köpte hemmanen Koski och Elenius av Gustav Mauriz von Essen, son till Otto Henrik år 1899. Hemmanen, som var 134 ha, varav 53 ha odling, delades i två delar. Leander och Hilda blev jordbrukare på Tollikko 	1911 och då byggdes mangården. Hilda Maria dog samma dag som yngsta barnet Eskil, 17.05.1920, och Leander blev änkling med 5 barn.


\jhoccupant{Elenius}{Isak \& Lovisa}{1907--\allowbreak 1911}
Isak och Lovisa brukade hemmanen på Tollikko och Ruotsala, men bodde hela tiden på Ruotsala.


\jhoccupant{Gunnar}{Matts \& Kajsa}{1899--\allowbreak 1907}
Matts och Kajsa bodde på Gunnar hemman, ägde från förr halva Tollikko och efter köpet av von Essen ägde han hela Tollikko hemman.


\jhoccupant{von Essen}{Carl-Otto \& Anna Kristina}{1863--\allowbreak 1899}
Carl-Otto blev ägare till 1/8 dels mantal av Tollikko hemman år 1863 och då ägde han redan sen 20 år tillbaka en annan 1/8 del av samma hemman. Han bodde på Keppo och andra skötte jordbruket. På 1870-talet blev sonen Otto Henrik ägare till hemmanet.


\jhoccupant{Olika}{bönder}{1847--\allowbreak 1858}
Åren 1847-56 ägdes 1/8 del av Tollikko hemman av flera olika bönder bl.a. av Michel Jakobsson Tollikko. Gabriel Roos köpte hemmansdelen 29.11.1856 och sålde den 1858 till Elias Roos (inte släkt med Gabriel), som var gift med Otto von Essens syster Anna Ottilia. Paret ägde och bodde på Tollikko ca 5 år. Efteråt övergick hemmanet i Ottos ägo.


\jhoccupant{Elenius}{Johan \& Maria}{1820--\allowbreak 1847}
Johan, äldsta sonen till kaplanen Thomas Elenius, \textborn 19.05.1795 på Lillollas kaplansboställe, gifte sig 1815 med Maria Tollikko, \textborn 10.09.1798.
\begin{jhchildren}
  \item \jhperson{\jhname[Sanna Lisa]{Elenius, Sanna Lisa}}{08.07.1823}{04.05.1863, g m Jacob Böös}
  \item \jhperson{\jhname[Anna Sofia]{Elenius, Anna Sofia}}{29.04.1826}{02.09.1891, g m 1 Simonsson, 2 m Sarelin}
  \item \jhperson{\jhname[Maja Albertina]{Elenius, Maja Albertina}}{09.09.1828}{05.12.1916, g m Henrik Eriksson på Gunnar}
  \item \jhperson{\jhname[Johan Jacob]{Elenius, Johan Jacob}}{08.10.1831}{11.12.1832}
  \item \jhperson{\jhname[Isak]{Elenius, Isak}}{07.12.1833}{13.06.1915, g m Sanna Lisa Högbacka}
  \item \jhperson{\jhname[Thomas Jacob]{Elenius, Thomas Jacob}}{05.11.1836}{14.01.1868}
  \item \jhperson{\jhname[Erik Gustaf]{Elenius, Erik Gustaf}}{09.10.1839}{13.03.1887, med och byggde Jeppo kyrka}
  \item \jhperson{\jhname[Greta Mathilda]{Elenius, Greta Mathilda}}{04.03.1843}{10.03.1908, g m Johan Kaup}
\end{jhchildren}
Efter moderns Susanna Elenius död 1816 fick Johan via testamente överta Sandbacka hemman, som Thomas köpte 1806 för 333 riksdaler silver. Johan och Maria sålde hemmanet 1819 till doktor Topelius för 1000 rubel. Året därpå köpte han 1/8 del mantal av Tollikko hemman av Anders Johansson. Familjen var hela tiden bosatt på Tollikko (var huset stod är obekant). År 1837 köpte han av svågern Johan Eriksson ena halvan av sin hustrus hemgårds hemman på 1/8 dels mantal för 500 	riksdaler, två år senare köpte han andra halvan av samma hemman. År 1847 sålde Johan bort alla hemmansdelar på Tollikko och i stället köpte han Smedbacka hemman i Nykarleby för 1200 rubel. Som häradsdomare och prästson var han en man i upphöjd ställning i samhället. Tio år senare gick hemmanet honom ur händerna efter att han gått i borgen för andra. Sista åren bodde de hos dottern på Stenbacka.

Johan \textdied 16.01.1868 --- Maria \textdied 24.10.1868



\jhhouse{Elenius}{11:19}{Tollikko}{16}{13, 13a}


\jhhousepic{268-05849.jpg}{Jan-Olav och Maaret Elenius}

\jhoccupant{Elenius}{Jan-Olav \& Maaret}{2005/2013--}
Jan-Olav, \textborn 01.04.1976 i Jeppo, gifte sig 25.09.2004 med Tiina Maaret, \textborn 02.03.1977, född Autio från Virdois.
\begin{jhchildren}
  \item \jhperson{\jhname[Hugo Vilhelmi]{Elenius, Hugo Vilhelmi}}{16.02.2005}{}
  \item \jhperson{\jhname[Senja Katarina]{Elenius, Senja Katarina}}{24.04.2007}{}
  \item \jhperson{\jhname[Helmer Leander]{Elenius, Helmer Leander}}{23.01.2009}{}
  \item \jhperson{\jhname[August Olav]{Elenius, August Olav}}{25.10.2010}{}
  \item \jhperson{\jhname[Ilona Juliaana]{Elenius, Ilona Juliaana}}{05.04.2013}{}
\end{jhchildren}

Jan-Olav och Maaret byggde en ny bostad 2005 och ekonomibyggnad med pannrum. Samma år uppfördes på hemgårdstomten en nytt fähus med plats för 60 kor med varm lösdrift. Familjen bedriver mjölkproduktion med robotmjölkning. Magnus och Jan-Olav ingick stegvis generationsväxling 2003, jfr gård nr 11. År 2013 övertog Jan-Olav och Maaret hela lägenheten. Genom köp har storleken stigit till 180 ha varav odlad areal är 80 ha.

Vardera makarna har lantbruksutbildning, bl.a. från Lannäslund. Maaret är därtill utbildad närvårdare och jobbar på gården. A= garage, pannrum.



\jhhouse{Sandgropsbackan}{11:10}{Tollikko}{16}{14, 14a}


\jhhousepic{267-05854.jpg}{Maj-Lis Simanainen}

\jhoccupant{Simanainen}{Maj-Lis}{1998--}
Maj-Lis, \textborn 15.12.1937 gift 1958 med Eino, \textborn 04.08.1934, född Simanainen är numera delägare i sterbhuset.  Huset används som sommarhus av familjen. Maj-Lis och Eino flyttade till radhus i Jeppo 1996, maken Eino dog 05.11.2012. Han var tidigare farmchef i Kimo och då bodde familjen där.\jhvspace{}


\jhoccupant{Jungarå}{Artur \& Maria}{1934--\allowbreak 1998}
Artur, \textborn 25.05.1896, gift 1932 med Maria Emilia, \textborn 19.05.1907, född Ojala på Kampas hemman.
\begin{jhchildren}
  \item \jhperson{\jhname[Anders Bruno]{Jungarå, Anders Bruno}}{24.06.1932}{20.08.2016, bodde i Deje Sverige}
  \item \jhperson{\jhbold{\jhname[Maj-Lis]{Jungarå, Maj-Lis}} Johanna}{15.12.1937}{}, gift Simanainen
\end{jhchildren}
Artur skulle ha fått bli jordbrukare på halva föräldrahemmanet, men tackade nej och istället för jordbruksyrket livnärde han sig som diversearbetare bl.a skogsarbetare. Artur var vakt vid Kiitola fabrik under krigstiden, han var också mjölnare på Tollikko kvarn. Han besvärades flera år av nedsatt hälsa och dog 27.12.1961. Maria fortsatte att bo i stugan efter makens död, ända tills hon flyttade in i pensionärslägenhet i centrum. Hon dog 24.02.1998. a = lider, 	sommarbostad.


\jhoccupant{Mattsson}{Sanna-Maija}{1920--\allowbreak 1934}
Fabriksarbetaränkan Sanna-Maja Mattsson löste in sin innehavande Sandgropsbacka benämnda backstugoområde av Leander Elenius den 3 juli 1920.\jhvspace{}



\jhhouse{Pikkumäki}{11:11}{Tollikko}{16}{106, 106a}


\jhhousepic{Timperis fahusgrund.jpg}{Timperis fähusgrund är en fröjd för ögat}

\jhoccupant{Timmer}{Oskari}{1915--\allowbreak 1948}
Juho Oskari, \textborn 09.11.1880, född Kortesjärvi, gift med Lisa, \textborn 05.02.1862, Hermansdotter. Han flyttade till Jeppo 1915, bodde här i två år och sen tio år i Kortesjärvi för att sen återvända till Jeppo år 1927.

Tomten 11:11 bildades av lägenheten 11:6 som kolonisationslägenhet 1921 och tomten fick fastebrev 30.09.1925. Timperis stuga hade bara ett timrat rum och farstu, storleken var 3 x 4 meter, där fanns öppen spis, ett skåp, en säng, ett bord två stolar och en bänk.

Oskar var en godhjärtad och snäll person, men något barnslig. Han hade det svårt i livet på grund av handikapp; han var bl.a. låghalt. Han spelade även fiol, men vad han spelade visste ingen. På kvällarna samlades ungdomar hemma hos Oskar. Golvet i huset var så glest att man kunde se råttor genom springorna, ibland blev det råttjakt med spetsade trädkäppar. Han hämtade hem kärnmjölk från mejeriet en gång i veckan gående. Kärnmjölken hämtades i en ``fjälung'' som rymde 15-20 l. En gång i tiden hade han också häst och kärra, som han forslade sina kvastar med. Han band 2 olika sorters kvastar, inne- och utekvastar. Utekvastarna bestod av björkris, innekvastarna  bestod av kråkris. Under senare tid hade han ingen häst, så han forslade sina kvastar på dragkärra. Hans varumärke var mycket jämna och vackra kvastar.

Timperi bjöd emellanåt på en fyllevisa: ``Tämän kylän flikat he kävelivät niillä nirskuvat nirunarukengillä, eikä nyt kannata ilveirellä, ne on tienattu reisillä.''

Oskars hustru hade bullaförsäljning (skorpor). Lisa dog 11.11.1933. När hälsan blev sämre flyttades Oskar till åldringshemmet (såggården). ``Han flyttades till sin nya vårdare Brita Backlund '', står det i kommunalvårdsnämndens papper. Oskar dog 07.06.1956. Redan 1938 övertog kommunen hans lägenhet. Ivar Jungell köpte tillbaka tomten av Jeppo kommun 1951 för 5000 mark.

Stugan revs av Paul och Manne Laxen 1948 och flyttades till deras hemgårdstomt, där den restes på nytt och blev ungkarlslya. a = fähus.



\jhhouse{Tollikko kvarn, inkl. boende}{11:20}{Tollikko}{16}{107, 107a-b}


\jhhousepic{Tollikko 107.jpg}{Tollikko kvarn}

Enligt gamla dokument fanns på Tollikko hemman vattenkvarn redan 1708 och ägdes av en Matts Andersson. Kvarnen var placerad 300 m nedströms från den nyare Tollikko kvarn.

Den nya Tollikko kvarn byggdes av byggmästar Törnqvist år 1882 åt N F Nordman och Johan Jungarå, som då var ägare, Samtidigt byggdes en stendamm i huvudfåran för att styra och uppdämma vattnet. Distriktingenjör F.W. 	von 	Willebrand granskade anläggningen 18 okt 1882 och gav utlåtande med vissa villkor, bl.a. bestämdes tilloppsrännans bredd till 8 fot, liksom avloppsrännans bredd till 8,5 fot. Fallhöjden var ca 5 fot. Vid en ny granskning 1887 hade ändringar	godkänts så fallhöjden var ca 6 fot och bredden på avloppsrännan 6 m med sidolutning = 1: 11/2. Detta ansågs inte medföra olägenheter. På tomten byggdes också fähus, stall, kvarnstuga och vedlider. Kvarnen bestod av tre par stenar och grynverk.

\jhhousepic{Karta fr Tollikko fors 1882}{Tollikko fors 1882}

Ägarna Joel Elenius, Henrik Jungell, Gustav Jungarå, Isak Jungarå och Erik Elenius ansökte 7 mars 1922 om tillstånd att bygga en fast damm över huvudfåran av älven med en så stor fallhöjd som möjligt. För flottningens skulle inrättas en 2 m bred öppning som kunde stängas. Fallhöjden mättes av Ingenjör August Saxberg på kalkerduk med plankarta och nödiga höjder och profiler. Enligt avmätningar av alla botten- och vattenhöjder skulle fallhöjden bli nästan 3 meter. Den genomgående vattenmassan vid synetillfället 15 aug 1922 var 7 m3, vilket var lägre än normalt, största vattenmängden var beräknad till 170 m3/sekund.

Den 12 april 1923 kom utlåtande i ärendet och resultatet var sämre än tillståndet från 1887, och ingen fast damm fick byggas. Ägarna var inte nöjda, men någon ändring trots besvär blev det inte. Ännu i början av 1930-talet var äredet aktuellt.

1887 erhöll Henrik Andersson Kojonen koncession för verket.

1898 fick Erik Elenius Isak Jungarås andel genom testamente, han sålde delen 1903. 1908 avled Nils Fredrik Nordman och hans del sköttes av bankdirektör Erik Söderström som konkursbo.

1914-18 fick Leander Elenius och Henrik Jungell kvarnhyra av Vilhelm Sandström. Också Antti Ruotsala och Karl Mågas är inblandade.

1918 står Gustav Jungarå m.fl. som ägare.

1920 fanns 1 st vattenturbin 20 hk, 1 elektrisk motor 5 hk. Elbelysning till 75 lampor.

1925 brann kvarnen men byggdes upp på nytt.

Från 1930-talet var ägarna Henrik Jungell och Leander Elenius.
a = kvarnstuga, b = fähus/lider, c = pärthyvel/såg.

Personer som bott i kvarnstugan och mjölnare på Tollikko kvarn:
\begin{enumerate}
  \item Anders och Matts Tyni
  \item Vilhelm Sandström
  \item Edvard Källman
  \item Carl Gustav Bro \textborn 20.10.1886 hustru Hilda Irene \textborn 20.12.1899.
  \item Artur och Maria Jungarå, mjölnare bodde där 1932-34.
  \item Jukka Karkaus mjölnare efter kriget, installerade cirkesåg, pärthyvel drogs med egen motor, sonen Osmo fortsatte till mitten av 1950-talet.
  \item Sommaren 1945 bodde Lennart Laxèn med familj där.
  \item Hösten 1945 en Kemijärvkvinna Kemppainen och barnen Arvo och Sointu.
  \item 1946 kom karelare dit som hette Alexanteri och Maria Simanainen med barnen Teuvo, Eino och Pirkko.
  Alexanteri \textborn 1898 i Suitsamo dog i Jeppo 1952. Han var Mjölnare/ sågställare/ pärthyvlare.
  Maria \textborn 20.09.1903 född Ikonen dog 1981. Deras barn: Teuvo Kalevi \textborn 15.12.1931--\textdied 1949---- Eino Olavi \textborn 04.08.1934--\textdied 05.11.2012---- Lea Pirkko \textborn 06.01.1943
  \item På 1950-talet inhystes även skogsdikare
\end{enumerate}
Stugan revs av Rurik och Lauri Broo 1960-61.


\jhsubsection{Stockflöitaren på 1940 talet}

Paul Laxén  minns 1940-talsflöitaren.

Man kan säga att flottningens förarbete började redan på vintern när skogshuggarna fällde träd och kapade dom till olika längder och barkade en del på stället. Vissa forslades till älven som lagringsplats t.ex Rambelbackan, som var mittemot Karkkas, men den stora lagringsplatsen var vid Krouvi, där det barkades stora mängder.

\jhhousepic{Stockflottning 1950-t.JPG}{Ibland var älven fullmatad med virke på väg nedströms}

Det fanns 3 olika typer av barkning randbarkning halvbarkning och prima barkning. Randbarkning gjordes vid vedhuggning om klabben var tjockare än 5 cm för att den skulle torka och inte ruttna. Vid halvbarkning, som var lättare, fick det vita under barken lämnas kvar på propsen. Vid primabarkningen skulle även allt vitt under barken bort. Barkningen var ett tungt jobb. När barkningen var gjord, fick staplarna (pinona) vänt tills flottningen började. När flottningen började var det bara att slänga propsen i älven som ibland var besvärligt på grund av avståndet till älven. Ibland gjordes rännor ända ned till älven, liksom en fåra som man blötte så att propsen gled bättre i rännan, rännorna byggdes antingen av props eller av bräder. När rännan byggdes av props var den tjockaste propsen längst från älven och tunnaste propsen längst ned mot älven för att inte propsarna som skickades i rännan skulle fastna. Rännan i props var byggd sektionsvis av 3 props, varav den tunnaste propsen var i bottnet och dom två tjockare propsen var en på var sida av den tunna propsen. Propsrännan låg på backen och plankrännan var en bit upp i luften på bockar. Rännorna flyttades varteftersom propsarna var samlade för flottning.

Ibland kunde man se olika bokstäver i propsänden, för det mesta såg man bokstaven S. Kanske var betydelsen av detta någon samflottning för lönsamheten.

Älven var vid flottningstiden ett bra transportsystem, stora mängder transporterade sig själv om än dock stockflöitarna hjälpte till. Bra tanke av honom som kom på idén att använda älven och dess resurs som transportsystem, miljövänligt var det också. Visst sjönk det en del virke i älven, men folk kom på idén att dragga upp det sjunkna virket sedan flottningen (flöitån) hade passerat och med detta fick folk ved så man klarade sig över vintern och ett år framåt. Även jag draggade upp många kubik virke. Man skulle föra båten tvärs över ån för att hitta sjunkna virket på älvens botten. Man så att säga kände sig för med kexet.

Vid draggningen användes ett kex med 6,5 meter långt skaft som man slog fast i det sjunkna virket. När virket satt så stadigt fast i bottenleran att man inte fick loss virket, då fick man göra en ögla runt kexskaftet så djupt som man nådde, andra ändan av öglan band man fast i åran, åran måst man hålla stadigt över aktern och gå till den andra ändan av båten så att man erhöll en hävarma som lyfte upp virket eftersom man inte orkade själv lyfta upp det. Om stocken var så lång att den ej gick att ta in i båten, så band man stocken fast i hottorn (årklykan), man släpade stocken på detta sätt med båten. När båten var fullastad rodde man till stranden för lossning.

\jhpic{Flottning 1943.JPG}{Flottning 1943: Viisan Salo, Erik Haga, Eskil Forslund, Ingvald Berg och Erik Grandén}

Om själva stockflöitana.

Det kom några gubbar i sina piikkobååtar (flöitabååtana) och satte ut bommar. Den första var vid Pauhus, där dom satt dubbelbommar för trycket var enormt på grund av virkesvikten i stora mängder som tryckte på ovanför bommarna. Bommarna var fastsatta i en stor stock vars ända var fastsatt i marken på älvstanden. Bommarna var bundna med flätade (tvinnade) granrötter eller granslanor. Slanorna var ``basade'' för att kunna tvinnas så att dom inte gick av. Andra bomstället var vi Krouvi. Dom stängde av västra älvfåran (Keppos ååen) med sina bommar. Tredje bomstället var vid Losso, älvfåran till Karkkas Jukkas kvarn så att stock och props inte skulle flyta till Jukkas kvarn.

Innan bommarna öppnades vid Pauhus infann sig flöitarna vid Tollikko kvarn. Då blev det liv och rörelse i byn. Flöitarna var från Lappo, Kauhava, Nykarleby samt någon enstaka från Jeppo. Flöitare från Jeppo var Rurik Broo, Sven Nyström och kanske någon till. När propsar och stock började komma placerade sig flöitarna ut på strategiska ställen där det brukade stockas (patas) upp. I början på Tollikko-forsen hade dom en öppning på ca 4m så virke slapp flytande vidare. Där hade dom en stock över öppningen så flöitarna kunde ta sig över till andra sidan älven. Men hur det nu var skulle även jag till andra sidan, men att gå över på en våt och hal stock så vet man hur det gick, jag trillade i forsen mellan stock och props, ca 5 m längre ned kom jag upp med huvudet mellan props och stock. Det var svårt att komma upp, porilana och stockarna tryckte på hela tiden medstöms. Men jag hade lyckan med mig och kanske lärde mig något av denna händelse.


Några episoder under tiden då flöjtarna var vid Tollikko.

När dom första propsarna (porilana) flöt förbi, fick vi ungar bråttom till älven för att snatta varsin propps (klabb), som var 1 m lång och som gömdes i diken med gräs på så flöitaren inte skulle hitta dom. Sen när flöitån var förbi hämtade vi dom gömda propsarna (porilana) vilka vi använde som livboj när vi simmade. Man flöt bra med dom placerade under bröstet. En gång kom en flöitare i flöitabååtin, men hur det nu var så kantrade båten i forsen och flöitarn hade drunknat om inte Johannes Jungarå hade räddat honom. Johannes stod på rätt plats i rätt tid.

En ung flöitare ville att min bror Manne skulle komma med honom till sitt vaktarpass i forsen. Men då Manne ville bort från flöitarens vaktpass släppte flöitaren inte bort honom. Manne for dit på dagen och följande dag slapp Manne bort. Manne var hemskt yr då han kom hem vattnet forsade hela tiden för ögonen.

Flöitarna festade ibland rikligt, då blev det slagsmål, kanske hade någon blått öga och svullna kinder. Man vet ju inte vad flöjtarna tänkte om oss ungarna, som skrattade och talade svenska, bland annat var Ingmar Dahlström och jag där på kvarngården. En flöitare tyckte troligtvis inte om oss, eftersom han började springa efter oss. Ingmar sade att flöitarn har kniv i handen men jag såg ingen kniv. Vi hoppade över järdesgården mot Timperis-stugan och klarade oss den gången. Jag har frågat av Ingmar på senare år om flöitarn verkligen hade kniv och Ingmar sade ja. Många år senare då jag hade tjänst vid järnvägen med många olika uppdrag och kommenderingar med män från Lappo, Kauhava och Härmä, då kände jag igen en man från mina unga år, så jag frågade om han varit flöitare och bott i en kvarnstuga i Tollikko i Jeppo? Det har jag sade han. Hur vet du det? Då sade jag om han kommer ihåg då han sprang efter två pojkar som hoppade över järdesgården. Han tittade en stund på mig och så sade han att det kommer han inte ihåg. Kanske ville han inte det?

En flöitare, troligtvis från Nykarleby, sade till oss ungar ``flessi pela pannån päärona påtra i gryitån'', troligen kunde han och förstod lite svenska. Det blev lugnt i byn som förut när flöitån drog vidare efter ca 2-3 veckor. Stockflöitasi pågick i hela Lappo älv i hela dess längd.

Vid ett tillfälle när jag och Manne draggade efter stockar, simmade en orm mot oss och skulle upp i båten. Manne högg av ormen mot båtsidan när den skulle upp i båten. Vid ett annat tillfälle, när jag var med far och draggade timmer, lossnade en sjunken props från kexet från älvbottnet och slog i båtbottnet. Konstigt att en sedan tidigare sjunkt props började flyta.


\jhhouse{Tollikko}{11:19}{Tollikko}{16}{108}


\jhoccupant{Tollikko}{Abraham \& Maja-Lena}{}
Abraham Johansson, \textborn 14.11.1826--\textdied 188?, gift med Maja-Lena, \textborn 	30.03.1823--\textdied 189?, född Storstara.
\begin{jhchildren}
  \item \jhperson{\jhname[Maria]{Tollikko, Maria}}{1860}{}
  \item \jhperson{\jhname[Anna-Lovisa]{Tollikko, Anna-Lovisa}}{16.05.1862}{13.07.1941, gift med Gustav Bro}
\end{jhchildren}
Barnbarn: 8 st bl.a. Fanny gift Laxén. Abraham var statsdräng vid Keppo på 1860-70 talet. Flyttade till Tollikko.


\jhoccupant{Tollikko}{Johan \& Brita}{}
Johan Johansson, \textborn 03.02.1780--\textdied 07.06.1856, gift med Brita, \textborn 	03.12.1785--\textdied 26.05.1853, Mattsdotter Tollikko.
\begin{jhchildren}
  \item \jhperson{\jhname[Brita-Lisa]{Tollikko, Brita-Lisa}}{1805}{}, gift Ruotsala
  \item \jhperson{\jhname[Johan]{Tollikko, Johan}}{17.08.1807}{}
  \item \jhperson{\jhname[Matts]{Tollikko, Matts}}{18.09.1810}{}
  \item \jhperson{\jhname[Erik]{Tollikko, Erik}}{03.03.1814}{1841}
  \item \jhperson{\jhname[Anders]{Tollikko, Anders}}{10.08.1816}{}
  \item \jhperson{\jhname[Maja-Sofia]{Tollikko, Maja-Sofia}}{19.12.1818}{}, gift Gunnar
  \item \jhperson{\jhname[Isak]{Tollikko, Isak}}{14.06.1821}{}
  \item \jhperson{\jhname[Jacob]{Tollikko, Jacob}}{10.10.1823}{}
  \item \jhperson{\jhbold{\jhname[Abraham]{Tollikko, Abraham}}}{14.11.1826}{188?}
\end{jhchildren}



\jhhouse{Krooks}{11:19}{Tollikko}{16}{112}


\jhoccupant{Krooks}{Nestor \& Laina}{1937-195?}
Nestor Adolf, \textborn 14.05.1907, född Krooks i Oravais, gift med Laina, \textborn 	08.04.1915, född Palmu i Alahärmä.
\begin{jhchildren}
  \item \jhperson{\jhname[Uuno Johannes]{Krooks, Uuno Johannes}}{02.08.1937}{}
  \item \jhperson{\jhname[Sisko Kyllikki]{Krooks, Sisko Kyllikki}}{14.11.1938}{}
  \item \jhperson{\jhname[Reijo Matias]{Krooks, Reijo Matias}}{19.09.1946}{}
  \item \jhperson{\jhname[Heino]{Krooks, Heino}}{10.04.1951}{2016}
\end{jhchildren}
Nestor förtennade kopparpannor, han spelade också dragspel. På bröllopet mellan Nestor och Laina i mitten av 1930-talet dansades det och spelades grammafon så golvet gungade till långt in på småtimmarna, berättar Paul, då 3-4 år gammal som var upplyftad på ett skåp.

Nestor \textdied 25.08.1960  ---  Laina \textdied 02.02.1969


\jhoccupant{Krooks}{Gustav \& Tilda}{1909--\allowbreak 1937}
Gustav Isaksson, \textborn 03.06.1878, född Krooks i Alahärmä, gift med Anne Matilda, \textborn 30.08.1873 i Oravais.
\begin{jhchildren}
  \item \jhperson{\jhname[Anna Sofia]{Krooks, Anna Sofia}}{20.01.1895}{02.04.1912}
  \item \jhperson{\jhname[Teodor Ingvar]{Krooks, Teodor Ingvar}}{10.04.1904}{1973, till Umeå Sverige}
  \item \jhperson{\jhbold{\jhname[Nestor]{Krooks, Nestor}} Adolf}{14.05.1907}{25.08.1960}
\end{jhchildren}
Gustav och Matilda Krooks flyttade från Oravais till Jeppo 1909. Alla barn var födda i Oravais. 1913 emigrerade Gustav eller Gust till Amerika. Han var i USA:s militärtjänst under första världskriget 1917--\allowbreak 1918, bodde i Butte, Montana. I familjen bodde också fosterflickan Sonja Backman? Tilda koppade och satte leder på plats åt folk som hade krämpor,  kallades för ``kupparitilda''. 	Matilda \textdied 10.04.1937.

\jhpic{krooks militarpass.jpg}{Gustav Krooks militärpass}



\jhhouse{Mattisbacken}{11:56}{Tollikko}{16}{17}


\jhhousepic{269-05853.jpg}{Ida och Sebastian Alvik}

\jhoccupant{Alvik}{Ida \& Sebastian}{2012--}
Ida Maria Elenius, \textborn 15.09.89, gift 15.06.2013 med Sebastian Alvik, \textborn 13.06.1988 i Kållby. Sebastian är utbildad elmontör och arbetar på Jeppo Kraftandelslag. Ida är utbildad sjukskötare, har arbete på Nykarleby sjukhem.
\begin{jhchildren}
  \item \jhperson{\jhname[Ellinore]{Alvik, Ellinore}}{01.11.2012}{}
  \item \jhperson{\jhname[Oskar]{Alvik, Oskar}}{17.10.2014}{}
\end{jhchildren}


\jhoccupant{Elenius Matias}{Annika Jussila}{2009--\allowbreak 2012}
Matias Dan Mikael, \textborn 27.05.1983, och Annika Jussila, \textborn 15.06.1987. Paret byggde år 2009 ett ``Heikkius hus'' åt sig på hemgårdshemmanets åker, en bit från älven. Matias och Annika separerade och Matias flyttade till lägenhet på Älvvägen, Silvast.
