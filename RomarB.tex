<--- se KARTA nr 3 --->



\jhhouse{Furubacken (Dalbacka)}{2:126 Elfstrand}{Romar}{3}{20}


\jhhousepic{090-05904.jpg}{Lars och Nanna Romar, numera Håkan Romar}

\jhoccupant{Romar}{Håkan}{2005--}

Håkan Romar, \textborn 31.10.1965, övertog lägenheten år 2005 efter sina föräldrar Lars och Nanna. På lägenheten odlas för närvarande spannmål.


\jhoccupant{Romar}{Lars \& Nanna}{1959--\allowbreak 2005}
Lars Gustav Romar, \textborn 13.01.1940, gifte sig 1964 med Nanna Dahlkar, \textborn 20.10.1938 i Jörala by, Vörå. De övertog hemmanet 1959.
\begin{jhchildren}
  \item \jhperson{\jhbold{\jhname[Håkan Lars Erik]{Romar, Håkan Lars Erik}}}{31.10.1965}{}
  \item \jhperson{\jhname[Ulf Christian]{Romar, Ulf Christian}}{07.05.1969}{}
\end{jhchildren}

På lägenheten har hållits kor, höns och svin. Potatisodling har också varit viktig, speciellt i form av stärkelsepotatis. Lars transporterade också i unga år mjölk till mejeriet med traktor längs en rutt på östra sidan av älven. Han har varit intresserad av samhällsfrågor och varit ledamot av bl.a. stadsstyrelsen i Nykarleby förutom flera föreningsuppdrag såsom i ungdomsföreningen, SFP:s lokalavd. styrelse, socialnämnden, skolplaneringskomissionen, sparbanksprincipal, grundskolrådet, lionsklubben och pensionärsföreningen.


\jhoccupant{Romar}{Selim \& Ellen}{1944--\allowbreak 1959}

Strax efter krigets slut, den 11.12.1944 skiftades Elfstrands lägenhet R:nr 2:27 på Romar mellan bröderna Selim, Georg och Birger Romar. Selim hade efter giftermålet 1933 odlat 1/2 hemmanet.

Selim, \textborn 11.01.1900, gifte sej 02.07.1933 med Ellen Elenius, \textborn 09.05.1905. Makarna beslöt efter giftermålet att flytta bopålarna till Furubacken öster om järnvägen, där familjen byggde sitt bostadshus och senare ekonomiecentrum med ladugård och uthus. År 1959 påbörjades utbyggnaden av ladugården men då dog Selim plötsligt och ansvaret fick övertas av sonen Lars.

Selim fungerade som disponent för Jeppo Andelsmejeri i 10 års tid. Han var under en tid ordförande för ÖSP avdelningen i Jeppo, ledamot i Jeppo Kraft Andelslags styrelse och ledamot i skattenämnden.
\begin{jhchildren}
  \item \jhperson{\jhname[Eva]{Romar, Eva}}{07.09.1934}{}
  \item \jhperson{\jhname[Lars]{Romar, Lars}}{13.01.1940}{}
\end{jhchildren}

Selim \textdied 07.07.1959  ---  Ellen \textdied 07.10.1982


\jhhouse{Furubacken}{2:15}{Romar}{3}{21, 321}

\jhhousepic{Hogdahl nr 21.jpg}{Högdahls fritidsbostad, nr 21}

\jhoccupant{Högdahl}{Kaj \& Anne}{2015--}
År 2015 blev Kaj Esa Allan Högdahl, \textborn 08.07.1953 på Furubacken, och hans hustru Anne Ahlvik, \textborn 1954, ensam ägare till Furubacken	2:15 på Romar hemman i Jungar by. Kajs faster \jhname[Aidi Jungar]{Jungar, Aidi} överlät då sin andel till dem. Tidigare hade Kaj vid arvsskifte erhållit sin far Torvalds andel. Huset används som fritidsbostad. Kaj är speciallärare i en finsk grundskola i Jakobstad där han bor med sin familj. Hus nr 321 är rivet. Kaj och Anne har tre barn.
\begin{jhchildren}
  \item \jhperson{\jhname[Heidi]{Högdahl, Heidi}}{1980}{}
  \item \jhperson{\jhname[Henrik]{Högdahl, Henrik}}{1984}{}
  \item \jhperson{\jhname[Heline]{Högdahl, Heline}}{1989}{}
\end{jhchildren}

\jhoccupant{Högdahl}{Anders, dödsbo}{1958--\allowbreak 2015}

Anders Högdahl dog tragiskt 27.09.1958, då han hamnade under tåget.	Han var efter korna till Hästängen på andra sidan järnvägen. Han var då 77 år gammal. Hans hustru Alina bodde kvar i husets norra del till sin död den 05.05.1971.

Sonen Torvald och hans hustru Anja, född Perä \textborn 1927 i Voltti, Härmä, bodde i den södra ändan av huset. Torvald arbetade från unga år som	stationskarl på SJ i Jeppo.
\begin{jhchildren}
  \item \jhperson{\jhname[Raija]{Högdahl, Raija}}{22.12.1949}{} på Furubacken i Jeppo
  \item \jhperson{\jhbold{\jhname[Kaj]{Högdahl, Kaj}}}{08.07.1953}{} på Furubacken i Jeppo
  \item \jhperson{\jhname[Karita]{Högdahl, Karita}}{19.?.1955}{} på Furubacken i Jeppo
  \item \jhperson{\jhname[Rainer]{Högdahl, Rainer}}{02.02.1963}{} i Jakobstad
\end{jhchildren}

Torvald läste på finska till stationskarl och konduktör och fick fina betyg. År 1955 flyttade familjen till Jakobstad och han var en lugn och omtyckt förman på godsstationen och också som konduktör. Anja arbetade på ett tvätteri. Torvald dog i en hjärtattack den 14.08.1994 på	Furubacken. Han skulle ha fyllt 68 år i september.

Raija är utbildad köksa och kokerska och gift med Ulf Carlstedt i Esse.	Hon arbetar nu i Esse församlingshem med olika uppgifter. Karita är 	gift Kruskopf och bor i Reso utanför Åbo. Hon är utbildad barndagvårdare och arbetar i en barnträdgård. Rainer som är född i Jakobstad gifte sig för några år sedan och han arbetar på UPM. Efter att familjen flyttade till Jakobstad användes Furubacken	som fritidshus. Torvald byggde en stockstuga söder om den gamla stugan, som numera är riven. I sin ungdom bodde Kaj en tid på Furubacken och arbetade på Mirka i Jeppo.


\jhoccupant{Högdahl}{Anders \& Alina}{1916--\allowbreak 1958}
Den 30.03.1916 köper Anders Thomasson Romar (senare Högdahl),	\textborn 02.06.1881 på Romar, Furubacken på 0,5 ha av Romar hemman. I köpesumman 950 mark	ingår uthusrad och byggnadsmaterial för husbyggnad. Den 23.02.1925 erhåller han lagfart på Furubacken benämnda parcell R:nr 2:15 i Jungar by. År 1922 gifter sig Anders med Alina Katarina Lindén, \textborn 08.03.1895 på Kasakbacken i Purmo. Anders hade vistats några år i USA. Han byggde ett bostadshus i 1½ plan med fyra rum i nedre våningen och två sommarrum på vinden (nr 321). Det stod på samma plats som där första Uf-lokalen brann ner 1911.

Kanske hans stora intresse för trädgård väcktes i Amerika. Han sprängde och plockade stenar på tomten och byggde en stenmur omkring hela området, som ännu idag kan beundras.

\jhhousepic{Alina o Anders Hogdahl.jpg}{Alina och Anders Högdahl på Furubacken. Torvald i famnen, Anders' syster, gift Niemi, med döttrarna Agnes och Ester}

Anders odlade mandelpotatis, morötter, rödbetor, kål mm. Han planterade prydnadsbuskar, jordgubbar, äppelträd, hallon, krus- och vinbärsbuskar och fick därav lite inkomster. Högdahl var postbärare 1916--\allowbreak 1934 och kallades Post-Ant. Posten anställde den person, som tog emot arbetet till den lägsta 	summan. 1935 hade Gunnar Orrholm det lägsta beloppet och Anders blev utan arbete som postbärare. Alina och Anders fick tre barn.
\begin{jhchildren}
  \item \jhperson{\jhname[Ernst Fride Erland]{Högdahl, Ernst Fride Erland}}{14.04.1923}{02.11.1925}
  \item \jhperson{\jhname[Torvald Algot Fride]{Högdahl, Torvald Algot Fride}}{30.09.1926}{14.08.1994}
  \item \jhperson{\jhname[Aidi Myrtle Carita]{Högdahl, Aidi Myrtle Carita}}{28.03.1928}{}
\end{jhchildren}

Förutom egna barn tog Alina och Anders hand om Anders' döda systers barn, \jhname[Agnes och Ester Niemi]{Niemi, Agnes, Ester o. Enok}. Deras bror Enok fick växa upp på barnhemmet i Nykarleby och som fosterbarn i Jussila. Som vuxen fick han arbete på SJ:s vedplan i Jeppo och bodde då på Furubacken. Barnens far, Jaakko Niemi, försvann när deras mor dog. Anders och Alinas dotter Aidi Jungar berättar att de hade det mycket fattigt då. De hade 4 kor, några får, höns ibland och kanske någon griskulting. De hade byggt ett nytt fähus och uthus med tre avdelningar, en avdelning som 	vedlider det mellersta med spannmålslårar och det tredje för redskap. Senare byggde Anders en fin bastu. Anders hade tidigare, eller före köpet	av Furubacken, köpt ett jordområde, ca en ha på andra sidan järnvägen av skattelägenhet Norrgård 4:8 av Silvast hemman. Lägenheten infördes i Jordregistret 1915 som Högdahl 4:154. Anders arrenderade av sin bror Tomas Skog lite jord, den så kallade Storhagen, och av sin bror Johannes Bäckstrand för 100 år lite jord nära Furubacken.

De hade ingen häst och inga jordbruksmaskiner. För att få åkrarna plogade och harvade anlitades bönderna på Romar. Istället gjorde Alina dagsverken hos dem. Under kriget i början av 1940-talet fick Anders arbete på SJ:s vedplan vid Vattentornet. Då gick tågloken med 	ånga och det behövdes ved och vatten. Aidi kommer ihåg att hon ibland hjälpte sin far med att såga ved och lasta veden på en kärra. Familjen fick det ekonomiskt bättre då. Aidi 	kunde gå i Nykarleby Samskola, hon fick bo hos sin gifta kusin Rakel och hennes man Verner Sundell de två första åren. Efter samskolan, där hon fick ett gott avgångsbetyg, fortsatte hon ett år i Folkakademin i Borgå, där hon lärde sig bokföring. Hennes första tjänst blev kassör på Jeppo Sparbank vid starten 1946.

\jhhousepic{Furubacken 1930.jpg}{Samul-Anttas t.v och Roos t.h.(nr 22)}

Aidi gifte sig 1949 med Per Evert Jungar, född 22.11.1923 på Romar hemman. Per fick arbete på Wärtsilä, och de flyttade till Jakobstad. Aidi har arbetat på Pedersöre Handelslags kontor, några år på Skattebyrån samt 33 år som bokförare och redovisare på Jakobstads Skeppsstuveri och R.M. Labbart Oy vid hamnen i Jakobstad. Aidi och Per fick tre barn, Dan Per Olav, \textborn 22.02.1949, utbildad ekonom och CGR revisor, Ulf Hans Anders, \textborn 07.01.1958, dog hastigt 16.09.1960, och Ann-Cathrine Maria, \textborn 13.09.1962, statsvetare och doktor i statsvetenskap från Uppsala universitet, forskar och undervisar som docent och lektor i Söderströms Högskola i Huddinge utanför Stockholm.

Under början av 1940-talet hyrde Maija och Jussi Aaltonen ett rum på Furubacken. Aidi och Torvald lärde sig tala finska då.


\jhoccupant{Lind}{Astrid}{1915--\allowbreak 1916}

Den 10.10.1915 säljer Jeppo Ungdomsförening området till konstapel Alfred Linds omyndiga dotter \jhbold{Astrid} Elisabeth Lind för 200 mk.


\jhoccupant{Jeppo Ungdomsförening}{nr 321}{1909--\allowbreak 1915}
Den 07.01.1909 köper Jeppo Ungdomsförening för 200 mk området på Furubacken, ca 0,5 ha. Säljare \jhname[Johan och Kajsa Romar]{Romar, Johan \& Kajsa}. Ungdomsföreningen förbinder sig att inhägna området. Genom att tigga stockar, pengar och dagsverken byggdes en ungdomslokal på 150 m$^2$, till en kostnad på 14000 mk. Ungdomslokalen kunde invigas den 07.11.1909.

Glädjen blev kort. På söndagsmorgonen den 20.10.1911, kl. 04.30, såg man från Romar att lokalen brann. Huset brann ner till grunden och värdet uppskattades till 15.000 mk. Brandförsäkringar fanns på sammanlagt 12.000 mk. Fr.o.m. 27.12.1909 och fram till branden blev kafferummet i ungdomslokalen en naturlig plats för kommunalstämman i Jeppo. En ny ungdomslokal uppfördes 1912 i centrum av Silvast, på skattelägenhet Bäckstrand 4:33 (Silvast, nr 80).


\jhhouse{Furubacka}{2:26}{Romar}{3}{22}


\jhhousepic{089-05628.jpg}{Fritidsstuga; Doris Sandberg och Marita Nordling}

\jhoccupant{Sandberg \& Nordling}{Doris \& Marita}{1960--}
Marita Ahlström, \textborn 15.07.1938, gifte sig 16.11.1958 med Evald Nordling, \textborn 26.09.1934. År 1960 övertog Doris, \textborn 28.06.1937, gift med Stig Sandberg, tillsammans med syster Marita fastigheten på Furubacken. Huset har länge stått obebott.

Systrarna är födda i Purmo dit deras mor \jhname[Ester Bäckstrand]{Bäckstrand, Ester}, \textborn 18.12.1905, flyttade efter att hon gift sig med Viktor Vilhelm Ahlström 11.11.1936. Han var född 10.06.1906 i Purmo. I äktenskapet föddes två  döttrar: Doris och Marita. Makarna livnärde sig på ett litet jordbruk med några kor, får och höns. Viktor hade också förtjänstarbete som snickare. Han dog redan 24.10.1946 och Ester flyttade 1947 tillbaka till Jeppo med sina döttrar till detta hus, som de köpte av Johannes Roos arvingar. Huset var byggt av Roos från Grötas i början av 1900-talet. I huset bodde i samband med köpet 1947, Hugo och Linnea Saha som hyresgäster.

Jordinnehavet i Purmo bibehölls en tid och Ester försörjde sin familj med några kor och fick hjälp från sin hemgård på Bäckstrand med mycket av det praktiska arbetet.

Ester Elvira \textdied 13.12.1992



\jhhouse{Trafikant}{2:87}{Romar}{3}{23}


\jhhousepic{088-05627.jpg}{Siv-Britt Furu}

\jhoccupant{Furu}{Siv-Britt}{2010--}
Siv Britt föddes 1954 i Kronoby, Jeussen-Nabben. Vid 6 års ålder flyttade familjen till Drycksbäck. Sedan 1980-talet har hon varit sambo med Krister Bäckstrand.

Siv-Britt är utbildad kock från yrkesskolan. På 1970-talet fick hon arbete på sjömansskolan och  har arbetat på Viking-Line som kallskänka. Därefter kom hon tillbaka till Jakobstad och arbetade i Terrazzo-huset innan hon anställdes av Herrmans Måleri i Sandsund.

1979 kom hon till Jeppo och hade en kort anställning på Mirka innan hon anställdes som skolköksa vid centrumskolan i Jeppo, en anställning hon innehaft sedan dess. Efter omorganiseringen av stadens köksfunktioner har det skett en koncentration till skolans kök och idag framställer hon tillsammans med sina medarbetare ca 200 portioner/dag.


\jhoccupant{Bäckstrand}{Krister}{2003--\allowbreak 2010}
Krister föddes 03.08.1951. Han genomgick folk- och medborgarskolan i Jeppo. Dagen efter skolavslutningen tog han anställning vid Keppo pälsfarm. Där arbetade han tills farmen lades ner 1992 för att därefter överflyttas till Mirka. Efter en svår sjukdom avled Krister den 06.04.2010.


\jhoccupant{Bäckstrand}{Agda \& Astrid}{1945--\allowbreak 2003}
Astrid Susanna föddes 06.03.1913 och systern Agda 02.12.1915. Huset byggdes åt de ogifta systrarna i början av 1950-talet. Efter att Agda gift sig med \jhname[Helge Björk]{Björk, Helge} och flyttat till Sverige uthyrdes den del av huset som tillhört henne och kom genom åren att inhysa flera olika personer, bl.a. Lars och Nanna Romar, Roland Laxén, familjen Olof Laxén, ``Aalton-Jussi'' m.fl.
Agda var mejerska.

Astrid arbetade länge på Källströms skomakarverkstad, senare på Keppo och Mirka.
\begin{jhchildren}
  \item \jhperson{{Krister}}{03.08.1951}{06.04.2010}
\end{jhchildren}

Agda \textdied 21.01.1992  ---  Astrid \textdied 10.02-2003


\jhhouse{Lugnet}{2;42}{Romar}{3}{24, 24a}


\jhhousepic{087-05626.jpg}{Mona och Alf Wallin}

\jhoccupant{Wallin}{Mona \& Alf}{1997--}
Alf Mikael Martin Wallin, \textborn 16.04.1956 i Oravais, och Mona Marie Klemets från Tjöck, \textborn 16.01.1958, gifte sig den 23.02.1980. Den 26.08.1997 köpte de fastigheten av Göran och Kerstin Elenius.

Alf utbildade sig till skogstekniker/skogsförman och arbetade som sådan åt W. Schaumann cellulosafabrik i Jakobstad fram till mars 1981, när han tillsammans med Mona kallades till missionärer av Svenska Lutherska Evangeliföreningen (SLEF)för arbete i Kenya. Mona och Alf reste till en början till London för skolning och på sensommaren samma år avskildes de för missionsuppdraget och reste till Kenya.

Makarna verkade som missionärer fram till 1994. Från 1995 har Alf varit anställd av SLEF som ungdomssekreterare med ansvar för Lägerområdet Klippan i Munsala fram till 2010 med kortare inhopp på missionsfältet. Under perioden 2011--\allowbreak 2012 arbetade Alf och Mona halva året i Kenya och andra halvan som ansvariga för Klippan. Från 2013 övergick de att på heltid arbeta som missionärer i Kenya.

Mona utexaminerades som diakonissa 1979. Hon arbetade i Oravais församling ca 1 år innan hon tillsammans med Alf åkte ut till missionsfältet där hennes kunskaper i sjuk- och familjevård kom väl till pass. Från 1997 har hon arbetat som diakonissa i Nykarleby församling, varvat med uppdrag i Kenya.

Makarna reste ut på en ny 2-års period i början av december 2015.
\begin{jhchildren}
  \item \jhperson{\jhname[Tobias Alf Martin]{Wallin, Tobias Alf Martin}}{04.12.1982}{} i Nairobi, gick kl 1-5 i Kenya. Ingenjör.
  \item \jhperson{\jhname[Sebastian Hilding]{Wallin, Sebastian Hilding}}{03.10.1984}{} i K:stad, gick kl 1-3 i Kenya. Bilmekaniker.
  \item \jhperson{\jhname[Ron Johan]{Wallin, Ron Johan}}{26.08.1988}{} i Kismu, Kenya. Servitör/restaurangkock.
\end{jhchildren}


\jhoccupant{Elenius}{Göran \& Kerstin}{1977--\allowbreak 1997}
Alf Göran Lennart, \textborn 18.09.1948, gifte sig 14.11.1970 med Kerstin Eva Margareta Juthbacka från Nykarleby, \textborn 27.10.1947. Makarna övertog fastigheten 1977 av dödsboet efter framlidne Lennart Elenius, Görans far. Boningshuset genomgick en grundlig renovering 1981, varvid en av de första jordvärmepumparna i Jeppo installerades.

Göran utbildade sig till byggmästare inom väg-och vattenbyggnadsteknik. Efter utexaminering fick han omedelbart anställning inom statliga Väg och Vatten. År 1988 flyttade familjen för en tid till Sverige och fastigheten stod tom till 1989. Under denna tid fungerade Göran som övervakare, anställd av Svenska Vägverket, vid byggandet av bron mellan Holmsund och Obbola. Hösten 1989 återkom familjen delvis på barnens begäran då skolan i Finland upplevdes stå för bättre undervisning. Dottern Anne Marie gick första året i gymnasiet i Sverige och sonen Patrik sista året i högstadiet.

År 1994, 28 juni, godkände Vägverkets centralavdelning planen på byggandet av Replotbron, vilket innebar starten på byggandet av Finlands hittills längsta bro. Göran utsågs till övervakare av bygget, som kom att sträcka sig över flera år. Den långa arbetspendlingen mellan Jeppo och Replot bidrog till beslutet att sälja huset i Jeppo och flytta närmare Vasa, till Singsby i Korsholm.

Efter brons färdigställande blev Göran chef för Vasa vägmästardistrikt med ansvar för området från Oravais till Jakobstad. Göran fick sjukpension 1996.

Kerstin har arbetat inom Andelsringen i Nykarleby fram till 1970, varefter hon drev en egen skobutik i Nykarleby fram till 1979. Därefter har hon genomfört vuxenstudier inom vårdsektorn och bl.a. arbetat som vårdare på Fylgiahemmet i Vasa fram till pensioneringen 2010.
\begin{jhchildren}
  \item \jhperson{\jhname[Anne Marie Angela]{Elenius, Anne Marie Angela}}{07.05.1971}{}, lärare vid Vasa Övningsskola, gift Kronholm.
  \item \jhperson{\jhname[Patrik Kristian]{Elenius, Patrik Kristian}}{06.07.1973}{}, skogsbruksingenjör, Tenala
\end{jhchildren}


\jhoccupant{Elenius}{Lennart \& Göta}{1951--\allowbreak 1977}
Joel Lennart, \textborn 08.10.1904, gifte sig med sin första hustru Elma Maria Södergård från Böös, \textborn 01.03.1908. De fick en dotter, Doris Viola, \textborn 24.07.1929, gift Björkell i Esse. Elma Lovisa dog 26.01.1939. Familjen bodde fram till dess på Södergård lägenhet vid Mietala.

Lennart gifte om sig den 29.11.1939, dagen innan vinterkriget startade, med Göta Alice Forss, \textborn 12.03.1916. Under kriget bodde Göta i sin hemgård på Fors tillsammans med den växande barnskaran. Den 20.09.1951 köpte Lennart och Göta av Lantbruksministeriets kolonisationsavdelning från Löte lägenhet R;nr 2:41 utstyckade bostadstomt, en s.k. frontmannaparcell som fick namnet Lugnet 2:42. Året därpå uppfördes huset och senare garaget.

Före kriget ägde Lennart en lastbil tillsammans med Lennart Gustafsson, en lastbil som beslagtogs av försvaret. Under kriget fungerade bägge som chaufförer, men endast tidvis med den egna bilen. Efter kriget körde Lennart också taxi innan han på nytt kom att äga en lastbil, med vilken han tillsammans med Johannes Eklöv utförde mestadels grustransporter. Grävmaskiner för lastning fanns inte, varför all pålastning måste ske för hand med skyffel.

Göta kom att arbeta på Kiitola, dels under kriget och därefter under den tid som pälsberederiet var verksamt där. Hon arbetade också länge på Café Funkis. Fram till pensionen 1981 arbetade hon på Mirka.
\begin{jhchildren}
  \item \jhperson{\jhname[Gun-Lis Maria]{Elenius, Gun-Lis Maria}}{28.05.1941}{}, arbetat i Sverige, bor nu på Åland
  \item \jhperson{\jhname[Ann-Lis Margareta]{Elenius, Ann-Lis Margareta}}{24.02.1944}{}, varit barnskötare vid Tölö barnhem, g. Brunnsberg
  \item \jhperson{Alf \jhbold{Göran} Lennart}{18.09.1948}{},
\end{jhchildren}

Lennart \textdied 07.02.1973  ---  Göta \textdied 29.12.2004


\jhhouse{Sivula}{2:43}{Romar}{3}{25, 25b}


\jhhousepic{086-05625.jpg}{Reijo Vesanen, tidigare Annanolli}

\jhoccupant{Vesanen}{Reijo}{2014--}
Husägaren Reijo Juhani Vesanen bor i Jakobstad. Han besöker platsen under sommarhalvåret.\jhvspace{}


\jhoccupant{Annanolli}{Kauko}{1985--2014}
Kauko, som föddes i ``Grötas Selmas'' hus, övertog efter föräldrarnas död fastigheten och innehade den till 2014. Som ung började han arbeta på Keppo, fortsatte sedan som postiljon för att därefter flytta till Jakobstad och där arbeta vid järnvägen. Han var banvakt och skötte t.ex. järnvägsbommen vid Frams korsning. Fram till försäljningen använde han huset ofta som sommarbostad.


\jhoccupant{Annanolli}{Juho \& Helmi}{1945--1985}
Juho Viljami Annanolli föddes i Himango, \textborn 23.03.1910, gift med Helmi Maria (Kennola), \textborn 01.07.1909. Den 3 mars 1945 erhöll Juho(Jussi) en s.k. frontmannaparcell längs Nylandsvägen, utbruten från Löte lägenhet R:nr 2:23, varefter huset byggdes.

Jussi hade som ung varit sjöman och senare arbetat på Asevarikko i Gamlakarleby. Efter kriget och giftermålet med Helmi arbetade han på järnvägsstationens vedplan i Jeppo fram till sin död.
\begin{jhchildren}
  \item \jhperson{\jhname[Kauko Juhani]{Annanolli, Kauko Juhani}}{08.04.1946}{feb. 2014}
  \item \jhperson{\jhname[Regina Ameli]{Annanolli, Regina Ameli}}{09.05.1950}{}
\end{jhchildren}

Juho \textdied 09.10.1966  ---  Helmi \textdied 19.11.1985



\jhhouse{Tallbo}{2:44}{Romar}{3}{26, 26a-b}


\jhhousepic{085-05624.jpg}{Gunnevi Niskanen}

\jhoccupant{Niskanen}{Gunnevi \& Vesa}{1997--}
År 1997 köpte Gunnevi Birgitta Niskanen ( f. Granbacka från Kronoby), \textborn 16.06.1941, gift 1980 med Vesa Kyösti Niskanen, \textborn 07.03.1944 i Säyneinen nära Kuopio, den fastighet som bebyggts av Hugo och Linnea Saha. Gunnevi är en av de först utbildade familjedagvårdarna. Hon har emellertid största delen av sitt yrkesliv arbetat inom livsmedelssektorn och tillsammans med sin man bl.a. upprätthållit ett arbetsplatskök i Sandsund, Pedersöre med både stationära kunder och fördelning ända till Esse och Munsala.

Vesa är utbildad restaurangkock. Utöver ovan nämnda arbetsplatskök har han också fungerat som ansvarig för Juthbacka restaurangkök tillsammans med sin fru.


\jhoccupant{Saha}{Hugo \& Linnea}{1959--\allowbreak 1997}
Hugo Klemes Saha, \textborn 30.10.1919 i Alahärmä, gifte sig med Mabli Linnea Gunvor Asplund, \textborn 14.03.1918 i ``Ruckuskroken''. Begreppet Ruckus kommer från Pensala-Aspnäsområdet. Systrarna \jhname[Gerda Dahlström]{Dahlström, Gerda} och Linnea Sahas mor flyttade därifrån med efternamnet Asplund och bosatte sej vid vägkröken och från detta har platsen fått sitt namn. Om man svänger österut från tidigare Sparbankens tomt, är man inne på Ruckusvägen och drygt 200 m längre fram finns den brant svängande Ruckuskroken.

Paret fick inga gemensamma barn, men på fastighet 2:44 har Linneas dotter Sirkka Liisa växt upp. Hon var född Kosola efter moderns första man, som dog 1944.

1957 fanns färdiga ritningar till huset, men först efter att lagfart erhållits den 14.04.1959 påbörjades bygget på den skogbevuxna tomten, som ursprungligen varit en s.k. frontmannaparcell och ägts av Johanna och Johannes Fagerholm med R:nr 2:41.

Hugo var till en början ``dagaman'', men kom så småningom in i skogsarbete, där han med energi och skicklighet vann mycken respekt. Linnea var under krigstiden anställd vid Kiitola ammunitionsfabrik och arbetade senare länge på Café Funkis i Silvast.

Barn: Sirkka  \textborn 07.09.1942

Linnea \textdied 05.06.1998  ---  Hugo \textdied 06.06.1999



\jhhouse{Björkhagen}{2:45}{Romar}{3}{27, 27a-b}

\jhhousepic{084-05623.jpg}{Nils och Helena Forss}

\jhoccupant{Forss}{Nils \& Helena}{1981--}
Nils, \textborn 04.01.1952, gifte sig 1978 med Helena Backlund från Lövö, \textborn 24.03.1950. Den 16 juni 1981 inköptes tomten, som utgjort en frontmannaparcell, av \jhname[Alf och Alma Lillkung]{Lillkung, Alf}, \textborn 30.01.1915 resp. \textborn 09.03.1911.

Husbygget startade genast och var inflyttningsklart i  jan. 1983. Garaget byggdes 1984.

Nils har genomgått den 2 åriga maskinlinjen vid Lannäslund Lantbruksskolor. Till en början arbetade han i 5 års tid vid Hägglunds verkstad i Jakobstad, därefter en tid vid Union, ägd av Boris Lindén i Jeppo. Från 1974 arbetade han vid Keppos verkstad och tidvis som chaufför. 1986 flyttade han arbetsplats till Bröderna Sjöholms farmverkstad och arbetade där fram till 1992. Från denna tidpunkt jobbade han med rostfria produkter hos Borgmästars Svets i Bennäs. 1994 anställdes han av Mirka i Oravais som montör och reparatör på företagets verkstad.

Helena har genomgått kockutbildning vid Yrkesskolan i Jakobstad, men har arbetat på Jakobstads spetsfabrik och Rukka Regnkläder i Jakobstad. Mellan 1991 och 1996 var hon köksbiträde vid Jeppo skola. Från 1997 har hon varit anställd på Mirka i Jeppo, varifrån hon pensionerades våren 2015.
\begin{jhchildren}
  \item \jhperson{\jhname[Jan]{Forss, Jan}}{29.11.1978}{}
  \item \jhperson{\jhname[Nina]{Forss, Nina}}{09.09.1980}{}
\end{jhchildren}



\jhhouse{Löte}{2:60}{Romar}{3}{28, 28a}


\jhhousepic{091-05630.jpg}{Nils och Anna-Kaisa Elenius}

\jhoccupant{Elenius}{Nils \& Anna Kaisa}{1974--}
Nils Elenius, \textborn 31.08.1943, och hans hustru Anna Kaisa, född Kalliosaari \textborn 04.06.1947 i Ala-Härmä, köpte den 23.03.1974 lägenheten av bröderna Toivo och Tarmo Kalliosaari. På tomten fanns då ett äldre bostadshus jämte uthus.

År 1996 byggde Nils och Anna Kaisa ny bostadsbyggnad och flyttade in i denna 1997. Därefter revs en del av uthusbyggnaden och den resterande delen byggdes om till förråd och garage. Nu revs också den gamla bostadsbyggnaden.

Nils och Anna Kaisa gifte sig 1970 och flyttade till Tollikko. Därefter bodde de vid Kiitola fram till 1976 när renoveringen av den år 1974
inköpta bostaden var klar.
\begin{jhchildren}
  \item \jhperson{\jhname[Tom Göran]{Elenius, Tom Göran}}{16.11.1970}{}, gift m. Annika (Lillqvist), bor i Purmo
  \item \jhperson{\jhname[Mats Johannes]{Elenius, Mats Johannes}}{17.09.1978}{}, gift m. Sofia (Holmäng), bor i Purmo
\end{jhchildren}
Nils har arbetat på Keppo pälsdjursfarm, handhaft intransport av mjölk till mejeriet, tillsammans med sin bror Tor varit täckdikningsentrepenörer på 1970-talet och sedan år 1989 fram till pensioneringen 2007 arbetat på KWH-Mirka. Till en början på huvudmaskinen och sedan som arbetsledare inom produktionen.

Anna Kaisa började arbeta på Kleemola stickerifabrik i Härmä. Från 1970 började hon arbeta på Mirkas Kiitolafabrik. Efteråt har hon arbetat som hemvårdare och barnskötare åt kommunen.


\jholdhouse{Romar}{2:23}{Romar}{3}{328}

\jhoccupant{Kalliosaari}{Toivo \& Tarmo}{1959--\allowbreak 1974}
Den 31 januari 1959 köpte bröderna Kalliosaari den gamla bostaden och uthusbyggnaden av Elmer Sandberg, Leander Sandberg och Edvin Jungell. Här bosatte sig Toivo med familjen. Toivo, \textborn 01.11.1930, gift med \jhname[Elli Eliisa Perämäki]{Perämäki, Elli E.} från Alahärmä, \textborn 12.07.1930, \textdied 01.12.1978. Toivo har arbetat på järnvägen, tidvis i ``ståpparåikkån''.

Barn; Marko \textborn 20.05.1969

Tarmo har också arbetat på järnvägen, men flyttade till Bennäs.


\jhhousepic{Lote Romar 28,jpeg}{Johannes och Johanna Fagerholm byggde huset ca 1928--1929}

\jhoccupant{Trojka}{Elmer \& Leander \& Edvin}{1956--\allowbreak 1959}
Tillsammans med Edvin Jungell köper Elmer och Leander Sandberg av en ren tillfällighet lägenheten på auktion. De arbetade på uppförandet av Jeppo Handelslags spannmålsmagasin på stationsområdet när de en dag på hemväg från arbetet med cykel passerar platsen för auktionen. De tycker att budgivningen går trögt och besluter sig för att gemensamt sätta fart på auktionen. Plötsligt finner de sig vara ägare till hela fastigheten utan att egentligen menat det! Fastigheten hyrs ut till Toivo och Tarmo Kalliosaari, vilka sedermera köper den.
\jhperson{\jhname[Edvin Jungell]{Jungell, Edvin}}{18.11.1920}{22.07.2011}
\jhperson{\jhname[Elmer Sandberg]{Sandberg, Elmer}}{05.03.1903}{1987 i USA}. Gravsatt i Jeppo
\jhperson{\jhname[Leander Sandberg]{Sandberg, Leander}}{18.05.1906}{22.08.1975}


\jhoccupant{Sjöholm}{Gunvor}{1955--\allowbreak 1956}
Gunvor Sjöholm från Jakobstad erhöll lägenheten jämte byggnader i arv/gåva av Johannes och Johanna Fagerholm mot sytning. Hon bodde aldrig på fastigheten.


\jhoccupant{Fagerholm}{Johannes \& Johanna}{1929--\allowbreak 1954}
Johannes, \textborn 04.04.1873, och Johanna (Romar), \textborn 27.12.1887, hade enligt uppgift uppdragit åt Elias Fors, ``Ellagubben'', att upptimra bostaden år 1929. Det skedde på annan plats, men den nya timran plockades ner och flyttades hit. Makarna hade år 1928 sålt sin ägande Slangar lägenhet till \jhname[Simon Julin]{Julin, Simon} och efter detta startat bygget. Platsen tillhörde Löte lägenhet R:nr 2:23 av Romar hemman och hade tillhört Johannas föräldrar. Se Romar, karta 2, nr 313.

Paret Fagerholm hade fått 3 barn, vilka alla dog i tuberkulos och de dog således barnlösa. Under de sista decennierna av sina liv sålde de flera bostadstomter,  bl.a. längs älven och längs vägen mot Purmo.

Johannes \textdied 23.12. 1952  ---  Johanna \textdied 24.02.1954



\jhhouse{Björken}{2:94}{Romar}{3}{29}, utstyckad fr. Parken R:nr 2:59


\jhhousepic{093-05690.jpg}{Karin och Esko Kalijärvi}

\jhoccupant{Kalijärvi}{Karin \& Esko}{1969--}
Karin Margareta Kalijärvi, född Pelmas \textborn 08.10.1938, köpte den 6 september 1969 tomten av Edvin Jungell m.fl.(se Romar nr 28). Hon gifte sig 1966 med Esko Armas Kalijärvi från Kortesjärvi, \textborn 07.09.1940.

Huset började byggas 1971 0ch var klart för inflyttning 1972. Det byggdes i egen regi och som timmermän fungerade Johannes och Gösta Stenbacka, Niilo Pitkäranta (Eskos halvbror) från Kortesjärvi och Karins far Artturi Pelmas. År 1985 förstorades huset med en tilläggslokal och sonen Jorma tillsammans med sin hustru stud.merk. Birgitta Knutar, \textborn 1964, bodde här tilsammans med barnen Tony, \textborn 1990
och Marie, \textborn 1992. Makarna skiljde sej senare och Birgitta gifte sig med Guy Kronqvist (se nr 46).

Karin har i 3 års tid arbetat på Keppo och därefter i nästan 40 år som postiljon i Jeppo där posten i ur och skur, men också i vackert väder, förmedlats till adressaten.

Esko blev föräldralös vid 1 års ålder. Han kom till Jeppo i och med byggandet av Jeppo skola. Därefter fick han arbete som kontrollör (lastbilsräknare) vid byggandet av vägen genom Jeppo centrum 1970. Senare har han fram till pensioneringen varit anställd av VR.
\begin{jhchildren}
  \item \jhperson{\jhname[Tuula]{Kalijärvi, Tuula}}{07.09.1964}{} i Härmä
  \item \jhperson{\jhname[Jari]{Kalijärvi, Jari}}{28.11.1966}{} i Jeppo
  \item \jhperson{\jhname[Jorma]{Kalijärvi, Jorma}}{05.01.1968}{} i Jeppo
\end{jhchildren}

Esko \textdied 22.03.2011.


\jhhouse{Villa}{2:38}{Romar}{3}{30, 30a-b}


\jhhousepic{092-05691.jpg}{Peter och Merja Dahlström}

\jhoccupant{Dahlström}{Peter \& Merja}{1997--}
Den 26.10.1997 köpte Peter Dahlström, \textborn 18.10.1963, och hans hustru Merja Tepponen från Vasa, \textborn 12.01.1958, fastigheten av Karl-Erik Haglund. Peter Johan arbetar på Jeppo Potatis Ab, medan Merja Anneli till en början arbetade som ladugårdsavbytare och senare som anstaltsvårdare på Hagaborg äldreboende i Nykarleby centrum i mer än 20 år. De gifte sig 1988 och har följande barn:
\begin{jhchildren}
  \item \jhperson{\jhname[Maria Susanna]{Dahlström, Maria Susanna}}{27.08.1986}{}
  \item \jhperson{\jhname[Kim Mikael]{Dahlström, Kim Mikael}}{12.11.1987}{}
  \item \jhperson{\jhname[Joni Peter]{Dahlström, Joni Peter}}{12.11.1987}{}
  \item \jhperson{\jhname[Tina Isabella]{Dahlström, Tina Isabella}}{17.08.1990}{06.09.1990}
  \item \jhperson{\jhname[Mika]{Dahlström, Mika}}{25.08.1991}{}
\end{jhchildren}


\jhoccupant{Haglund}{Karl-erik}{1989--\allowbreak 1997}
Karl-erik, \textborn 30.05.1951, erhöll den 11.10.1989 fastigheten som gåva av Svante Haglunds dödsbo. Lagfartsansökan inlämnades 23.11 samma år. Karl-erik har varit lastbils- och långtradarchaufför under många år fram till pensioneringen. Han bor nu på nr 86, karta 5.


\jhoccupant{Dödsbo Haglund}{Svante \& Ester}{1980--\allowbreak 1989}
Efter Esters död 1980 har dödsboet gemensamt förvaltats fram till 1989 då det överlämnades som gåva till sonen Karl-erik.\jhvspace{}


\jhoccupant{Haglund}{Svante \& Ester}{1963--\allowbreak 1980}
I samband med att Karl Sunngren och hans hustru Linnea övertog Emil och Juliana Elenius jordbruk vid Böös (se nedan), köpte Svante och Ester Haglund denna fastighet Villa 2:38 år 1963. Svante Jakob Edvard Haglund föddes den 22.10.1915 och hans hustru Ester Skog föddes den 09.05.1918. Svante och Ester arbetade till en början med jordbruket på Esters hemgård vid Skog. Efter några år skiftade Svante om till att bli agent inom försäkringsbranschen och senare i livet fortsatte han som målare.
\begin{jhchildren}
  \item \jhperson{\jhname[Bo Anders]{Haglund, Bo Anders}}{28.04.1948}{}
  \item \jhperson{\jhname[Stig Göran]{Haglund, Stig Göran}}{18.04.1949}{}
  \item \jhperson{\jhbold{\jhname[Karl-erik]{Haglund, Karl-erik}}}{30.05.1951}{}
  \item \jhperson{\jhname[Barbro Ester Katarina]{Haglund, Barbro Ester Katarina}}{30.08.1953}{}
  \item \jhperson{\jhname[Elvi Heldine Elisabeth]{Haglund, Elvi Heldine Elisabeth}}{30.04.1958}{}
\end{jhchildren}

Svante \textdied 23.11.1996  ---  Ester \textdied 23.08.1980


\jhoccupant{Sunngren}{Karl \& Linnea}{1945--\allowbreak 1963}
Tillsammans med flera andra undertecknade Karl Salmela, senare  med efternamnet Sunngren, den 3 mars 1945 ett köpebrev med Johanna och Johannes Fagerholm om inköp av en frontmannaparcell från Löte R:nr 2:23 av Romar skattehemman. Han hade 20.10.1940 gift sig med Anna Linnea Sandås. Huset byggdes medan Karl arbetade på stationen.
\begin{jhchildren}
  \item \jhperson{\jhname[Leif Göran]{Sunngren, Leif Göran}}{29.03.1941}{}
  \item \jhperson{\jhname[Maj-Len Benita]{Sunngren, Maj-Len Benita}}{30.04.1944}{}
  \item \jhperson{\jhname[Ulla-Britt Linnea]{Sunngren, Ulla-Britt Linnea}}{29.05.1947}{}
  \item \jhperson{\jhname[Lars-Ole]{Sunngren, Lars-Ole}}{28.12.1949}{}
  \item \jhperson{\jhname[Tage Johannes]{Sunngren, Tage Johannes}}{08.06.1957}{}
\end{jhchildren}

Karl Sunngren arbetade på Statens järnvägars vedplan tillsammans med den stab arbetare som kontinuerligt skulle förse ångloken med bränsle under den intensiva tiden med krigsskadeståndets betalande, som hårt ansträngde järnvägens kapacitet. Han avslutade arbetet 1963 då familjen flyttade till Böös.

Karl \textdied 25.04.1995  ---  Linnea \textdied 27.03.1987



\jhhouse{Lillskog}{2:36}{Romar}{3}{31, 31a-b}


\jhhousepic{083-05622.jpg}{Harry Bro}

\jhoccupant{Dödsbo}{Bro Lauri \& Eila}{1981--}
Lauri Johannes Bro, \textborn 04.05.1914. Han gifte sig 26.12.1942 med Eila Suomela, \textborn 13.07.1921 från Kortesjärvi efter kriget. Den 5 februari 1948 gjorde han ett ägobyte med sin bror \jhname[Edvin Bro]{Bro, Edvin} och erhöll då denna tomt, som ursprungligen utstyckats som frontmannaparcell åt Edvin 3.3.1945, men inte bebyggts av denne. Tomten hade tillhört Johannes och Johanna Fagerholms ägande lägenhet Löte R:no 2:23 av Romar skattehemman. Området omfattade ca 5000 m$^2$.

Byggandet av bostaden påbörjades genast våren 1948 och huset stod klart samma år. Ekonomibyggnaden uppfördes 1950. Innan bostaden stod färdig bodde familjen på Gunnar hemman i Saikkonens hus, granne med Lauris far, Jakob Bro.

Lauri hade fungerat som bl.a. stridsstaffett under kriget och var känd bland sina frontkamrater som en sällsynt orädd person. Efter kriget livnärde han sig med slakteriverksamhet i de egna utrymmena och senare som ombud för andra slakterier.
\begin{jhchildren}
  \item \jhperson{\jhname[Harry]{Dödsbo, Harry}}{11.10.1945}{}
  \item \jhperson{\jhname[Paula Marjatta]{Dödsbo, Paula Marjatta}}{1951}{}, gift Isomäki, Jakobstad
  \item \jhperson{\jhname[Henrik Allan]{Dödsbo, Henrik Allan}}{27.03.1953}{}, sambo m. Katri Sandström i Jakobstad
\end{jhchildren}

På lägenheten bor den äldste i syskonskaran, Harry. Han har livnärt sig med minkfarmning och som farmarbetare. Han har också varit anställd som lantarbetare i trakten.

I unga år var han SFI:s bästa spjutkastare och innehade SFI:s rekord i spjutkastning 70.02 m. Det kastet gjordes i en kamp i Umeå 1967 och var den första gången en svenskösterbottning kastat över 70 m. Han var samtidigt JIF:s stora affischnamn idrottssammanhang under sin mest aktiva tid på 1960-talet.

Lauri \textdied 16.01.1981  ---  Eila \textdied 31.03.1997



\jhhouse{Sundell}{2:30}{Romar}{3}{32}


\jhhousepic{082-05619.jpg}{Marko och Marina Kalliosaari}

\jhoccupant{Kalliosaari}{Marko \& Marina}{2008--}
Marko Kalliosaari, \textborn 20.05.1969, gift med \jhname[Marina Cederström]{Cederström, Marina}, \textborn 11.03.1974. Fastigheten med bostadshus köptes i nov. 2008 av Paul Grahns dödsbo.

Marko har bl.a. arbetat åt svinfarmaren \jhname[Johan]{Slangar, Johan} och åt pälsdjursfarmaren Christer Jungerstam. Sedan 1999 har han arbetat på Ab Mirka Oy. Marina har bl.a arbetat på Kalles Konditori i Jakobstad  och har för tillfället tjänst inom hemservicen i Nykarleby.
\begin{jhchildren}
  \item \jhperson{\jhname[Amanda Cederström]{Cederström, Amanda}}{28.05.1997}{}, examen i båtbyggnad 2016
  \item \jhperson{\jhname[Matilda Söderbacka]{Söderbacka, Matilda}}{11.09.2001}{}
  \item \jhperson{\jhname[Matias Söderbacka]{Söderbacka, Matias}}{16.01.2004}{}
\end{jhchildren}
Hennes barn, Amanda Cederström, \textborn , .


\jhoccupant{Grahn}{Paul \& Eva}{1961--\allowbreak 2008}
Paul Grahn, \textborn 22.03.1929 i Jeppo, gifte sig 12.08.1956 med Eva Romar, \textborn 07.09.1934 i Jeppo. Den obebyggda tomten köptes år 1961 av Erland och Maire Sundell. Huset stod inflyttningsklart i febr. 1963.

Paul började sin arbetsbana på Handelslaget och flyttade sedan till Nedervetil som filialföreståndare i Murik. 1956 började han vid Sarins på Finskasbacken och efter att ortens skolor centraliserats 1966 startade han en privat bybutik i den tidigare småskolan på Finskas. Som biträde fungerade Saga Sandvik. Butiken stängde 1977 och Paul fick tjänst som hälsoinspektör i Oravais. Efter ett antal år flyttade han till samma tjänst i Nykarleby och gick i pension år 1992.

Eva började 1952 som bokförings- och kassabiträde vid Jeppo Kraft andelslag efter \jhname[Senny Nylund]{Nylund, Senny}. Senare övertog hon uppgifterna som kassörska och bokförare för andelslaget. Efter att sonen Mikael fötts 1962 hjälpte hon i 10 års tid till vid butiken på Finskas tills hon 1972 fick tjänst inom biblioteket, en anställning hon innehade fram till sin pensionering 1997.
\begin{jhchildren}
  \item \jhperson{\jhname[Peter]{Grahn, Peter}}{06.09.1959}{02.04.2013}
  \item \jhperson{\jhname[Mikael]{Grahn, Mikael}}{26.11.1962}{}, skogsbrukstekniker
  \item \jhperson{\jhname[Lena]{Grahn, Lena}}{21.01.1970}{}, gift Grahn-Söderström
\end{jhchildren}


\jhoccupant{Sundell}{Erland}{1944--\allowbreak 1961}
Erland Sundell inköpte tomten den 12.06.1944 av Johannes och Johanna Fagerholm, men bebyggde aldrig området, utan sålde det år 1961 till Paul och Eva Grahn.


\jhhouse{Tegelbacken-Bäckstrand}{2:10}{Romar}{3}{33, 33a-b}


\jhhousepic{079-05695.jpg}{Conny Bäckstrand och Pia Jakobsson}

\jhoccupant{Bäckstrand}{Conny}{2012--}
Conny Bäckstrand, \textborn 03.07.1985, sambor med Pia Jakobsson på det driftcentrum han övertog 2012 i samband med ett delvis genomfört generationsskifte från sina föräldrar Tom och Benita Bäckstrand. Utöver driftcentrum medföljde också en del av lägenhetens åker och skogsareal. Conny har inriktat sig på spannmålsproduktion och är vid sidan av denna också grävmaskinsentrepenör.


\jhoccupant{Bäckstrand}{Tom \& Benita}{1979--\allowbreak 2012}
Tom, \textborn 10.01.1954, gifte sig 05.07.1975 med Benita Finne, \textborn 17.02.1953. De övertog lägenheten 31.12.1979 av Toms föräldrar Edvin och Ruth Bäckstrand. Till en början fortsatte de med lägenhetens mjölkproduktion, men övergick efter en tid till äggproduktion.

I samband med EU-inträdet 1995 försämrades lönsamheten kraftigt för äggproduktionen och den lades ner. Sedan dess har på lägenheten varit inriktad på spannmålsproduktion. Tom har utöver skötseln av lägenheten därefter kompletterat inkomsterna som diversearbetare.

Benita, till utbildningen merkonom, har haft arbete på Andelsringen i Nykarleby och på Föreningsbanken i Nykarleby.  Hon anställdes av Jeppo Sparbank den 16.11.1976 och var i Sparbankens tjänst fram till den 18.10.1993 då Sparbanken Finland, till vilket kontoret i Jeppo då hörde, slaktades på statsmaktens försorg. Kontoret överfördes i samband med den händelsen till Vasa Andelsbank och Benita överfördes som ”gammal arbetstagare” till den nya organisationen.
\begin{jhchildren}
  \item \jhperson{\jhname[Anette]{Bäckstrand, Anette}}{05.08.1979}{}, tradenom, g. Karlsson i Munsala
  \item \jhperson{\jhname[Caroline]{Bäckstrand, Caroline}}{29.03.1983}{}, se Skog nr 2
  \item \jhperson{\jhbold{\jhname[Conny]{Bäckstrand, Conny}}}{03.07.1985}{}, vuxenutbildning till landsbygdsföretagare
\end{jhchildren}


\jhoccupant{Bäckstrand}{Edvin \& Ruth}{1946--\allowbreak 1979}
Edvin Bäckstrand, \textborn 05.02-1923, började bygga huset på Tegelbacken 1946. Det uppfördes av nytt timmer som avverkats året innan och fått torka. Timran uppfördes av timmermannen/snickaren Uno Storbacka, Ruth´s far, sedermera Edvins svärfar. Ladugårdens murade del började byggas hösten 1946 och därefter uppfördes hölada, porten och garaget. Edvin gifte sig 04.07.1948 med Ruth Storbacka,  \textborn 17.08.1923.
\begin{jhchildren}
  \item \jhperson{\jhbold{\jhname[Tom]{Bäckstrand, Tom}}}{10.01.1954}{}
  \item \jhperson{\jhname[Åsa]{Bäckstrand, Åsa}}{29.03.1956}{}
\end{jhchildren}

Tegelbacken, där Edvin och Ruth valde att bosätta sig, hyste vid den tidpunkten ingen annan bosättning. Däremot hade Tegelbacken under krigsåren på 1940-talet varit föremål för uppbevaring av vapen och ammunition i de 4 baracker som uppförts på områdets östra del. En barack var avdelad för vaktmanskapet, som dygnet runt gick vakt runt det taggtrådsingärdade området. På området fanns också en löpgrav och en korsu (se kartskissen, nr 333). Barackerna tömdes och revs omedelbart efter krigsslutet i enlighet med stilleståndsavtalet och kontrollkommissionens arbete. Kvar i terrängen kan ännu spåras små lämningar av anläggningarna.

Den 15 januari 1975 var olyckan framme. I samband med bostadens renovering uppstod en eldsvåda som delvis förstörde huset. Kvar stod
stockväggarna, men med nya krafter renoverades huset på nytt och gav nu plats förutom för Ruth och Edvin, också sonen Toms familj.

Edvin och Ruth var inriktade på mjölkproduktion. Edvin arbetade tidvis som montör på Jeppo Kraft. Han var många år ordförande för Jeppo Lantmannagille, som speciellt på 1950 och-60talet hade en livlig verksamhet. Ruth har vid sidan av arbetet på lägenheten varit en föreningsmänniska med många järn i elden. Hon var under kriget engagerad i Lottarörelsen och har bl.a. varit mångårig ordförande för Jeppo Hembygdsförening.

Edvin \textdied 02.07.2005

Ruth har bott i Älvlidens radhus i byns centrum innan hon flyttade till Florahemmet i Nykarleby.



\jhhouse{Metsäpirtti}{2:35}{Romar}{3}{34, 34a-b}


\jhhousepic{081-05618.jpg}{Jonas Forsbacka och Phatcharapon Tonsura}

\jhoccupant{Forsbacka}{Jonas}{2014--}
Jonas Mats Greger Forsbacka, \textborn 1986, köpte fastigheten 2014 av Ralf och Katrina Wallis. Jonas är född i Oravais, men flyttade med sina föräldrar Greger och Kristina Forsbacka till Jeppo när han var 1 år gammal. Han har examen från Optimas  maskin- och metallinje där han gick 2001--\allowbreak 2003. Efteråt påbörjade han studier vid Tekniska skolan i Vasa och i samband med detta fick han kombinerat sommarjobb och praktik vid Jeppo Potatis. Detta resulterade i att han småningom avbröt studierna och år 2011 fick fast anställning vid företaget. Han arbetar inom vad som benämns ``lådläggning'', vilket innebär den del av processen där den färdiga produkten läggs i lådor för kylning och vidare transport till kunden.

Jonas bor tillsammans med \jhname{Phatcharapon Tonsura}, \textborn 1981 i Thailand. Hon kom till Finland i dec. 2006. och har högskole-examen inom IT och marknadsföring. Också hon arbetar på Jeppo Potatis inom tillverkningsprocessen med vacumförpackning inkluderande kvalitetskontroll. År 2010 fick hon fulltidsanställning inom företaget.


\jhoccupant{Wallis}{Ralf \& Katarina}{2007--\allowbreak 2014}
Ralf, \textborn 23.07.1963 i Karleby, och Katarina (Lillkvist), \textborn 06.12.1964 i Maxmo, köpte fastigheten år 2007 när de flyttade från Jakobstad. Ralf har arbetat som vicedirektör för centret för livslångt lärande vid yrkeshögskolan Novia, Åbo Akademi i Vasa. Katarina arbetade som utvecklingskoordinator vid Optima samkommun, Jakobstad. Paret flyttade till södra Sverige (2013) för att överta ledningen av en folkhögskola. Ingendera av barnen har bott på fastigheten.
\begin{jhchildren}
  \item \jhperson{\jhname[Frank Nyman]{Wallis, Frank Nyman}}{1987}{}
  \item \jhperson{\jhname[Anne]{Wallis, Anne}}{1989}{1991}
  \item \jhperson{\jhname[Anne-Lie]{Wallis, Anne-Lie}}{1993}{}
  \item \jhperson{\jhname[Jenny Wallis]{Wallis, Jenny Wallis}}{1994}{}
  \item \jhperson{\jhname[Lotta Wallis]{Wallis, Lotta Wallis}}{1996}{}
\end{jhchildren}


\jhoccupant{Julin}{Thomas \ Monica}{1992--\allowbreak 2007}
Thomas, \textborn 03.08.1964 och Monica(Huggare), \textborn 22.12.1964, köpte fastigheten år 1992 av Johannes Eklövs sterbhus. Thomas arbetar på Mirka i Jeppo sedan 1983. Fritiden har han bl.a.offrat på engagemang i Jeppo Bya(råd)-förening och en tid som styrelseledamot i Röda Korset.

Monica arbetar som närvårdare inom socialväsendet i Nykarleby. Före tjänstgöringen i Nykarleby hade hon tjänst som hemvårdare i Pedersöre. Hon i sin tur visar sitt intresse för hembygdsfrågor som styrelsemedlem i Hembygdsföreningen, Jeposmettona och Källan-föreningen, som är en hjälporganisation bl.a. till Polen.
\begin{jhchildren}
  \item \jhperson{\jhname[Simon]{Julin, Simon}}{11.10.1993}{}
  \item \jhperson{\jhname[Robin]{Julin, Robin}}{13.02.1997}{}
  \item \jhperson{\jhname[Sandra]{Julin, Sandra}}{23.02.2001}{}
\end{jhchildren}


\jhoccupant{Eklöv}{Johannes \& Greta}{1948--\allowbreak 1992}
Johannes, \textborn 29.12.1914, köpte tomten av Oiva och Anna-Lisa Rintala den 12 jan. 1948. Johannes gifte sig samma år med Margareta Finnila, \textborn 14.03.1924 på Draka hemman, och samma år uppfördes huset och därefter ekonomiebyggnaden.

Johannes var chaufför i hela sitt liv, förutom åren 1952-54, då familjen bodde i Skärblacka, Sverige och Johannes arbetade på ett pappersbruk. Under dessa år bodde Paul och Gretel Nylind i det nybyggda huset. Under krigen körde han både lastbil och personbil. Han sjukpensionerades från Andelsringen 1981.

Greta arbetade bl.a. på Jeppo Andelsmejeri, som städerska på olika byggen och pensionerades också hon från Andelsringen, ortens största affär. Efter att huset sålts 1992 flyttade Greta till Gökbrinken, senare till Älvliden 10, bst 6, och år 2005 till Hagaborg.

Barn: Manfrid, \textborn 1949.

Han har genomgått yrkesskola och Vasa Tekniska läroanstalt och arbetat på Haldins karosserifabriker i Vasa och Maxmo. Manfrid var senare privatföretagare med ansvar för Finncements terminaler i Jakobstad och Vasa. Gift 1972 med Gudrun(Anderssén) född 1950, sjukskötare som varit ansvarig företagshälsovårdare i Jakobstad. Hon har också kandiderat i riksdagsval för socialdemokraterna. De bor i Bennäs och har två barn; Marcus född 1974, ingenjör i prod.ekonomi och Benny född 1977, lektor i engelska.

Manfrid berättar ang. \jhbold{``Greta'' Margareta Eklöv:}

På 50--60-talet hade några damer en syjunta i ``Silvastin''. Dit hörde Ellen Silfvast, Gemima Gustafsson, Hildur Fors, Svea Lindström, Göta Elenius , Anni Forss, Emilia Löf, Ellen Strengell, Lea Lassèn, Greta Eklöv, troligen också Meri Kronlöf. Jag minns hur roligt de hade. Syjuntan var också på resa bl.a till Ruovesi 1963 då en del gubbar samt barn var med.

Våren 1950 skulle min pappa köpa barnvagn till den nyfödda. Grannen Ekström beställde också en om det fanns att köpa. Pappa frågade priset på en barnvagn, frågade därefter vad det kostar om han tar två st. Han köpte 2 st men förklarade inte åt försäljaren vem som skulle ha den andra.

På 50-talet fanns det många barn på Alexandersgatan. Vi lekte olika lekar, sparkade fotboll på skolplan, tränade friidrott på sportplanen, drog cyklarna under tågen som stod stilla. På höstarna var det spännande att springa med strålkastare i mörkret och även titta in i något fönster. Vi var också och tittade på TV, ``Måndagsposten''  och senare kom Bröderna Cartwright. Holger Sandqvist skaffade den första tv-apparaten till vår gata. 1960 hade grannen Forslunds skaffat TV, så jag var där och såg på OS från Rom alla dagar.

Jag kommer ihåg att det var roligt att få vara med i mangelstugan då Ellen Silfvast och mamma manglade tillsammans. Då hade de kaffe o limonad med.

Jag har också fina minnen från höbärgningen och dyngkörandet med Ellen, Vilhelm och Lars Silfvast. Jag fick köra traktor och nu som då tog Ellen upp ``pastilde'' ur förklädet. Kommer också ihåg hur Lauri Bro tog hand om alla barn. Det blev ofta trångt i hans Ifa lastbil så det blev att byta om, men alla fick vara med. Vi körde hö, mossa, slakter, slaktavfall. Pojkarna var gärna med honom i slakteriet för att se hur det går till att avliva olika djur, pistol i huvudet eller ``muokan i huvu''.

Johannes \textdied 02.04.1981  ---  Margareta (Greta) \textdied 11.01.2013


\jhoccupant{Rintala}{Oiva \& Anna-Liisa}{1945--\allowbreak 1948}
Oiva, \textborn 22.09.1918 , gift med Anna-Liisa Pukkinen, \textborn 26.03.1918 i Pieksämäki, köpte den 3.3. 1945 tomten av Johannes och Johanna Fagerholm, ägare till Löte lägenhet R:nr 2:23. Den utstyckades som frontmannaparcell. Oiva och Anna-Lisa bebyggde inte tomten, utan sålde den till Johannes Eklöv 12 jan. 1948.



\jhhouse{Björklid}{2:34}{Romar}{3}{35, 35a}


\jhhousepic{078-05616.jpg}{Alf Dahlström och Päivi Mäkinen}

\jhoccupant{Dahlström}{Alf}{1996}
Alf Dahlström, \textborn 31.08.1960, köpte fastigheten 1996 av sina föräldrar Ingmar och Sally Dahlström. Alf har ända sedan 18 års ålder fungerat som lastbilschaufför för flera olika arbetsgivare. Ett av de längsta anställningsförhållandena har han haft med Nykarleby Trafik Central. Efter 19 år bakom ratten har han sedan 2001 arbetat på Jeppo Potatis Ab. Han har varit sambo med Päivi Mäkinen, \textborn 1964, från Ala-Härmä.Tillsammans har de tre döttrar.
\begin{jhchildren}
  \item \jhperson{\jhname[Heidi Dahlström]{Dahlström, Heidi Dahlström}}{03.04.1993}{}
  \item \jhperson{\jhname[Emma Dahlström]{Dahlström, Emma Dahlström}}{08.04.1996}{}
  \item \jhperson{\jhname[Ida Dahlström]{Dahlström, Ida Dahlström}}{11.03.1999}{}
\end{jhchildren}


\jhoccupant{Dahlström}{Ingmar \& Sally}{1963--\allowbreak 1996}
Ingmar Dahlström, \textborn  29.12.1937, gift med Sally Broo, \textborn 26.09.1937, köpte fastigheten av Helge och Karin Forslund år 1963. Dessa flyttade då till Gamlakarleby. Ingmar har arbetat  på Keppo pälsdjursfarm som byggnadsarbetare därifrån han sjukpensionerades 1979.

I unga år var Ingmar en av JIF:s mest framgångsrika idrottsmän och samlade massor av poäng som kulstötare i de återkommande kamperna mellan idrottsföreningarna i nejden. Sally har under en lång följd av år varit barndagvårdare i Jeppo för många av traktens barn.
\begin{jhchildren}
  \item \jhperson{\jhname[Stig]{Dahlström, Stig}}{21.01.1958}{}
  \item \jhperson{\jhname[Regina]{Dahlström, Regina}}{22.03.1959}{}
  \item \jhperson{\jhbold{\jhname[Alf]{Dahlström, Alf}}}{31.08.1960}{}
  \item \jhperson{\jhname[Peter]{Dahlström, Peter}}{18.10.1963}{}
  \item \jhperson{\jhname[Sten]{Dahlström, Sten}}{04.02.1966}{}
\end{jhchildren}


\jhoccupant{Forslund}{Helge \& Karin}{1951--\allowbreak 1963}
Den 2.6. 1951 köpte järnvägstjänstemannen Helge Forslund och hans hustru Karin (f.Grahn) fastigheten av montören Verner Ekström och hans hustru Inga.

Helge Torvald, \textborn 31.01.1930, gifte sig med Karin Gran, \textborn 12.11.1930, den 20.03.1949. Som framgår ovan arbetade Helge på järnvägen, till en början som förrådsarbetare och senare som tjänsteman. Karin fungerade som skolköksa på kommunens medborgarskola till dess att de 1963 sålde fastigheten och flyttade till Gamlakarleby den 29.1.1964. Under några månader bodde familjen som hyresgäster i husets övre våning tillsammans med Helges mor Maria Lovisa Johansdotter, \textborn 05.19.1886. Också Helges far, järnvägsarbetaren Jakob Emanuel Forslund,  \textborn 18.9.1884, bodde med familjen tills han dog den 21.12.1959.
\begin{jhchildren}
  \item \jhperson{\jhname[Stig Helge Johan]{Forslund, Stig Helge Johan}}{08.08.1949}{}, bor i Västervik, Sverige
  \item \jhperson{\jhname[Bo Erik Anders]{Forslund, Bo Erik Anders}}{29.06.1952}{}
  \item \jhperson{\jhname[Solveig Karin Maria]{Forslund, Solveig Karin Maria}}{07.08.1954}{}
  \item \jhperson{\jhname[John]{Forslund, John}}{}{}, född i Gamlakarleby efter flytten från Jeppo
\end{jhchildren}


\jhoccupant{Ekström}{Verner \& Inga}{1945--\allowbreak 1951}
Den 3.3.1945 köpte Verner och Inga Ekström Björkliden benämnda tomt av Johannes och Johanna Fagerholm tillhörande Löte R:nr 2:23 av Romar skattehemman för att utstyckas som frontmannaparcell. Makarna Ekström uppförde bostaden 1945 och uthuset året därpå. De fick två barn födda på lägenheten. Verner arbetade  som elmontör.



\jhhouse{Sandkulla}{2:33}{Romar}{3}{36}


\jhhousepic{077-05615.jpg}{Kjell och Ann-Britt Norrgård}

\jhoccupant{Norrgård}{Kjell \& Ann-Britt}{1984--}
Kjell Karl-Gustav Norrgård, \textborn 17.01.1954 gifte sig  med Ann-Britt Strandell, \textborn 26.12.1955. Den 25.04.1984 köpte makarna fastigheten, som utgjorde Kjells hemgård. Den har blivit grundrenoverad år 1987.

Kjell har sedan 1974 arbetat på Jeppo Kraft och därav en lång tid med s.k. linjearbete, d.v.s. ansvaret för att hålla distributionsnätet i skick enligt tidens krav. Men i minst lika hög grad har han varit involverad i det omfattande installationsarbete som Mirkas ständiga expansion medfört.

Ann-Britt är utbildad textiltekniker i Sverige, men har arbetat på Abeko och sedan några år in på 2000-talet på Nykarleby sjukhem.
\begin{jhchildren}
  \item \jhperson{\jhname[Malin Anna-Maria]{Norrgård, Malin Anna-Maria}}{23.02.1980}{}
  \item \jhperson{\jhname[Mats Johannes]{Norrgård, Mats Johannes}}{20.04.1983}{}
\end{jhchildren}


\jhoccupant{Norrgård}{Sigurd \& Verna}{1947--\allowbreak 1948}
Sigurd Johannes Norrgård, \textborn 25.12.1910, gifte sig 1940 med Verna Ingeborg Granvik, \textborn 24.12.1915 i Terjärv. Sigurd bodde tillsammans med sina bröder Evert och Lennart på ``Nörrgåln'' i Mietala. Han arbetade som butiksbiträde på Nymans kolonialvaruaffär i Silvast dåtida landsvägskorsning till Oravais, tills vinterkriget bröt ut och han inkallades. Han var fem år i krigstjänst.

Under tiden hade familjen vuxit. Dottern Gun föddes i Terjärv 1940 och när dottern Christina föddes 1943 hade familjen flyttat till Jeppo och hyrt in sig i familjen Backlunds hus vid Purmovägen (se nr 41).

Efter krigets slut köptes en s.k. frontmannaparcell vid Purmovägen den 3 mars 1945 av Johannes och Johanna Fagerholms hemman och byggandet av ett nytt hus kunde börja. Huset var inte helt klart hösten 1947, men familjen kunde flytta in i ett eget hem som upplevdes ``som ett himmelrike på jorden''.

Efter kriget fick han anställning som chaufför i Friis åkeriföretag i Gamlakarleby och deltog i och med detta i evakueringen av områden i norra Finland. Han berättar hur lastbilarna, som drevs med gengas, kunde ställa till med problem under uppdragen. En gång får han en transport från en gård och i den ingick också någon ko och det hö som fick plats för att säkra djurens foder för någon dag. På väg bort började människor de körde förbi vifta med armarna och peka bakåt mot lastbilsflaket. När han stannade för att ta reda på vad som stod på, såg han hur höet på lasset fattat eld! Gnistor från gengasaggregatet hade antänt höet, men branden släcktes utan att vålla desto större skada. Efter evakueringen vände trafiken småningom i motsatt riktning. Återuppbyggandet tog vid.

Från 1947--56 arbetade Sigurd som föreståndare för OTK:s Varmafilial i Silvast. Därefter fick han arbete som chaufför vid Jeppo-Oravais Handelslag/Andelsringen fram till pensioneringen. Sigurd var en aktiv jägare och var nästan varje helg ute i skogen på jakt efter harar de tider som jakt var tillåten.

Verna praktiserade för mejerskeutbildning i Vörå och Jeppo, men när det var dags för mejeriskolan stängdes denna p.g.a. kriget och utbildningen avbröts. Hon kom därefter att arbeta hemma som sömmerska och vart efter åren gick utökades familjen. De sista åren levde hon med äldsta dottern Gun på Sandbergs.
\begin{jhchildren}
  \item \jhperson{\jhname[Gun Inga-Britt Helena]{Norrgård, Gun Inga-Britt Helena}}{05.11.1940}{}
  \item \jhperson{\jhname[Ulla Christina Ingeborg]{Norrgård, lla Christina Ingeborg}}{22.12.1943}{}
  \item \jhperson{\jhname[Eivor Ann-Sofi]{Norrgård, Eivor Ann-Sofi}}{22.07.1950}{}
  \item \jhperson{\jhname[Verna Anna Elisabeth]{Norrgård, Verna Anna Elisabeth}}{01.10.1952}{}
  \item \jhperson{\jhbold{\jhname[Kjell Karl-Gustav]{Norrgård, Kjell Karl-Gustav}}}{17.01.1954}{}
\end{jhchildren}

Sigurd \textdied 09.05.1983  ---  Verna \textdied 03.05.2003



\jhhouse{Tomtebo}{2:72}{Romar}{3}{37, 37a}

\jhhousepic{080-05694.jpg}{Per-Erik och Marlene Lindgren}

\jhoccupant{Lindgren}{Per-Erik \& Marlene}{1977--}
Den 18.04.1977 undertecknade Per-Erik och Marlene Lindgren köpebrevet som gjorde dem till ägare av Tomtebo R:nr 2:72, tidigare tillhörande Bäckstrand lägenhet R:nr 2:10 av Romar skattehemman. Området ägdes då av Edvin Bäckstrand m.fl och var vid tillfället ännu oskiftat mellan de tidigare ägarna. Tomten blev 2700 m² stor och utgjordes till stor del av det område som under senaste krig fungerat som vapen och ammunitionsdepå, inrymd i de baracker som byggts på området (se skissen Romar hemman nr 333).

Per-Erik Lindgren, \textborn 07.06.1946 i Monå, gifte sig den 30.06.1973 med Louise Marlene Sikström, \textborn 29.04.1950 i Monäs.

Per-Erik, till utbildningen  merkonom, började med att arbeta på Munsala Handelslag, för att därefter fortsätta på Nykarleby Båtvarv. Efter att det företaget gått i konkurs startade han tillsammans med en kompanjon ett eget företag, L \& L-Produkter, som tillverkade karosser och andra komponenter av glasfiber. 1974 anställdes han på Mirka AB som exportassistent, ett företag han varit trogen fram till pensioneringen i maj 2012.

Under dessa år har han, förutom att arbeta som exportassistent, också varit exportchef, kundservicechef och logistikchef, d.v.s han har deltagit i företagets dynamiska utveckling under hela denna tid. På fritiden har han också under flera år varit ordförande för Jeppo ungdomsorkester och sommartid är det segling som gäller. Han är också mångårig medlem i LC Jeppo och ledamot i Jeppo pensionärsförenings styrelse.

Marlene blev klar med sin sjukskötareutbildning 1971. Till en början fick hon jobb på Malmska sjukhuset och senare på Östanlid sjukhus. År 1973 blev hon hälsovårdare och började arbeta på Wilhelm Schaumanns fabriker i Jakobstad som företagshälsovårdare. Fr.o.m. 1976 hade hon samma uppgifter inom Nykarleby hälsovårdscentral fram till år 1989 då hon flyttade till Jeppo rådgivningsbyrå. 1994 avlade hon magisterexamen i hälsovård och fick därefter arbete som föreståndare på Hagalund pensionärscenter i Nykarleby. Mellan åren 1999 och 2010 var hon ledande skötare på Nykarleby Hälsovårdscentral. Från 2010 fram till sin pensionering 2012 fungerade hon som ledande skötare i det nya interkommunala social-och hälsovårdsverket i Jakobstadsnejden med Jakobstad som värdkommun.
\begin{jhchildren}
  \item \jhperson{\jhname[Kim]{Lindgren, Kim}}{28.10.1974}{}, bor i Oravais, underhållschef på Mirka, Jeppo
  \item \jhperson{\jhname[Ann]{Lindgren, Ann}}{22.06.1976}{}, bor i Sandsund, förskolelärare
\end{jhchildren}



\jhhouse{Sandqvist}{2:98}{Romar-Löte}{3}{38}


\jhhousepic{076-05692.jpg}{Alf och Benita Andersson}

\jhoccupant{Andersson}{Alf \& Benita}{1984--}
År 1984 köpte Alf Rune Vilhelm Andersson, \textborn 21.06.1954 i Munsala och hans hustru Märtha Benita, \textborn Nyberg 29.11.1955 i Larsmo, fastigheten Sandqvist 2:98 vid Nylandsvägen 88. Alf arbetade 1976--\allowbreak 1987 som försäljare på Andelsringens butiksbil och Benita började 1979 som butiksbiträde på Andelsringen, numera Sale/KPO. De flyttade 1978 till Jeppo och fick sitt hem i Andelsringens hus. Alf arbetade några år på Marino i Bennäs före anställningen på KWH Mirka år 1992. Alf och Benita har renoverat uthusbyggnaden, som nu används som garage och verkstad.
\begin{jhchildren}
  \item \jhperson{\jhname[Kaj Stefan]{Andersson, Kaj Stefan}}{21.03.1981}{} i Jeppo, bilmek., rävfarmare, bor på Böös nr 42
  \item \jhperson{\jhname[Thomas Jan Mikael]{Andersson, Thomas Jan Mikael}}{05.05.1984}{}, bor på Åkerv., arbetar på Mirka
  \item \jhperson{\jhname[Jessica Ann-Louise]{Andersson, Jessica Ann-Louise}}{30.10.1992}{}, stud. i Vasa, sambo i Nykarleby
\end{jhchildren}


\jhoccupant{Norrgård}{Kjell \& Ann-Britt}{1981--\allowbreak 1984}
Elektriker Kjell Karl-Gustav Norrgård, \textborn 1954 på granntomten Sandkulla, och hans hustru Ann-Britt Carola, \textborn Strandell 1955 i Purmo, köpte fastigheten, Sandqvist 2:98 den 24.06.1981. De grundrenoverade bostadshuset. Kjell arbetar som elmontör på Jeppo Kraft Andelslag. Ann-Britt är utbildad sömmerska och arbetade då som tillskärare på Abeko i Nykarleby och från och med 2007 arbetar hon som lokalvårdare på Nykarleby Sjukhem.
\begin{jhchildren}
  \item \jhperson{\jhname[Malin]{Norrgård, Malin}}{23.02.1980}{}, ekon.mag., bor i Purmo
  \item \jhperson{\jhname[Mats]{Norrgård, Mats}}{20.04.1983}{}, se Romar, karta 2, nr 15
\end{jhchildren}

1984 övertog Kjell hemgården Sandkulla och familjen flyttade dit, se nr 36.


\jhoccupant{Dödsbo}{Hyresgäster}{1980--\allowbreak 1981}
Efter att Ellen Sandqvist och sonen Holger avlidit hösten 1979 hyrde Rita Janström och Veijo Mäkinen med sonen Mika Henrik Tapani född 09.03.1979, nedre våningen av bostadsbyggnaden, bastu och nödiga förvaringsutrymmen i uthusbyggnaden.


\jhoccupant{Sandqvist}{Alexander \& Ellen}{1943--\allowbreak 1979}
Gustav Alexander Sandqvist, \textborn 26.10.1896 på Svartbacken på Jungar hemman, och hans hustru Ellen Johanna, \textborn 08.07.1903 i Markby, köpte 1942 en av bröderna Isak och Joel Rönnqvist upptimrad gård för bortflyttning, som fanns på Stenholmen på Lavast hemman. Isak Rönnqvist hade stupat i kriget.

Alexander och Ellen hade i början på 1930-talet mist sin gård och butik på Silvast, R:nr 4:21, nr 62.  De köpte en skogstomt vid Nylandsvägen på Löte lägenhet R:nr 2:23, av Romar hemman 2. Lagfart erhölls 03.03.1945 på Sandqvist R:nr 2:98. Den timrade gården nedmonterades och fördes med hästslädar till tomten. Trots kriget och materialbrist blev huset delvis klart, så att familjen kunde flytta in till julen 1943. Äldsta sonen Ragnar var i kriget ute vid Svir och väntade med iver på sin första permission för att få se det nya hemmet, men det fick han aldrig. Alexander gräftade för hand upp skogstomten och de byggde uthus med bastu och tvättstuga samt ett mindre fähus, lada och förråd.

\jhhousepic{Romar nr 38 AlexanderS.jpg}{Ellen och Alexander Sandqvist}

Ellen och Alexander köpte 1945, 0,9310 ha av Löte skattelägenhet 2:46. Dessutom hade de inköpt ängsmark i Lill-Jinjärvi, samt höbärgade säv och gräs på Jinjärvi träsk och kunde då ha 2 kor och några får i fähuset. De yngre flickorna vallade korna på vägrenar samt samlade upp hö, som fallit på marken vid järnvägens lastbryggor. Vid auktion av Löte skattelägenhet inropade sonen Holger 07.08.1956, ett åkerområde på andra sidan av landsvägen, Tegelåker 2:63, om 1,4 ha.

Alexander arbetade som sotare och diversearbetare samt på 1950-talet som lantbrevbärare på södra rutten av Jeppo, när postiljonarbetet delades på två personer. Före och under kriget jagade och fiskade Alexander och sönerna Ragnar och Holger, bland annat sålde de en hel del ekorrskinn. Familjen var flitiga bärplockare. Lea kommer ihåg att en dag plockade hon och hennes far 38 liter åkerbär på plogade, icke brukade ängar vid Lill-Jinjärvi.

På 1930-talet ställde Ellen upp som kalaskock på bröllop och begravningar samt tvättade kläder vid älven åt ``herrskapsfolket'' på Silvast. År 1941 fick hon tjänst som postiljon i Jeppo, vilket innebar förutom att sortera posten, att föra ut den i alla väder på cykel och vintertid på sparkstötting runtomkring hela Jeppo, över 20 km varje vardag. Dessutom hörde till arbetet att sända och ta emot postsäckar vid morgon-, kvälls- och nattposttågen. Förutom tidningar, brev och postpaket till Jeppo, kom posten till Oravais, Munsala och Ytterjeppo via Jeppo. Postkärrorna var ofta fullastade med säckar. För att klara av arbetet hjälpte familjen henne ofta.  I medlet av 1950-talet delades postvägen i norra och södra rutten. Alexander anställdes på den södra delen. 1957 blev det tre linjer, då Silvast-området fick hembäring av posten till dörren och Bengt Strengell anställdes. Ellen gick i pension 1961 och Alexander hade slutat tidigare.
\begin{jhchildren}
  \item \jhperson{\jhname[Ragnar Gustav]{Sandqvist, Ragnar Gustav}}{24.10.1924}{27.01.1944} vid Svir
  \item \jhperson{\jhname[Holger Alfred]{Sandqvist, Holger Alfred}}{13.09.1926}{08.09.1979} i Jeppo
  \item \jhperson{\jhname[Hjördis Johanna]{Sandqvist, Hjördis Johanna}}{12.02.1928}{29.03.2007} i Arboga
  \item \jhperson{\jhname[Svea Marita]{Sandqvist, Svea Marita}}{12.11.1929}{29.01.2009} i Nykarleby
  \item \jhperson{\jhname[Lea Elisabet]{Sandqvist, Lea Elisabet}}{28.06.1934}{}
  \item \jhperson{\jhname[Eva Anita]{Sandqvist, Eva Anita}}{05.06.1936}{}
\end{jhchildren}

Ragnar, Holger, Hjördis och Marita föddes när familjen ägde och bodde på lägenhet Sandqvist 4:21 på Silvast hemman. Lea och Eva föddes på lägenhet Dahlbo 4:127 på Silvast, när familjen hyrde bostad där. Som tonåringar hade Hjördis, Marita och Eva tjänst efter varandra på Jeppo telefoncentral. Ragnar arbetade på Sundells Bageri då han blev inkallad för militärtjänstgöring i Dragsvik sommaren 1943. I början av oktober 1943 sändes han till fronten vid Svir, där han stupade efter tre månader i januari 1944. Holger blev inkallad och placerades som vakt vid militärområdet i Gamlakarleby. Holger blev stationskarl i Gamlakarleby och senare arbetade han på Keppo minkfarm i Jeppo och Sjöholms minkfarm i Ytterjeppo och bodde då med föräldrarna. Endast några dagar före sin 54-årsdag dog Holger i lungcancer. Hjördis gift med järnvägstjänsteman Leif Nyman, född 1925 i Bennäs, flyttade till Bennäs 1947 och därifrån flyttade familjen 1971 till Arboga, Sverige. De har 5 barn, Ragnhild, Gunvor, Richard, Astrid och Regina, alla födda då de bodde i Pedersöre, Bennäs.

\jhhousepic{Romar 38 vigsel.jpg}{Kvällen före Maritas och Erlands bröllop 5.6.1949}
Marita gift med målare Erland Sundqvist, född 1927 i Nykarleby, bosatte sig och byggde hus i Nykarleby. De har 5 barn, Birgitta, Kristina, Anders, Monica och Jan. Lea studerade till merkonom och blev kvar i Jeppo, då hon gifte sig med jordbrukare Paul Stenvall, född 1915 på Silvast hemman. Lea arbetade 1,5 år på Jouper-Produkt Ab i Nykarleby och 45 år inom KWH-koncernen, därifrån hon pensionerades som koncernens ekonomidirektör. Eva gift med försäljare Dan Östman, född 1935 i Esse, bor i eget bostadshus i Ytteresse. De har 2 söner, Ulf och Nils. Eva har arbetat tidvis som posttjänsteman i Esse och Kållby samt som städerska på skolan i Ytteresse.

Alexander \textdied 10.10.1973  ---   Ellen \textdied 27.10.1979, några veckor efter sonen Holger.



\jhhouse{Högskog}{2:118}{Romar}{3}{33c-d}


\jhhousepic{075-05613.jpg}{Tom och Benita Bäckstrand}

\jhoccupant{Bäckstrand}{Tom \& Benita}{2012--}
Tom, \textborn 10.01.1954, gifte sig 05.07.1975 med Benita Finne, \textborn 17.02.1953 i Markby. I samband med delvis genomförd generationsväxling 2012 utstyckades tomten och ett nytt hus av modell Eteletär från Älvsby hus uppfördes samma år. En garagebyggnad har också uppförts på tomten. Byggnaderna fungerar nu som paret Bäckstrands nya hem efter flytten från Tegelbacken. Se mera under Romar hemman nr 33.


\jhhouse{Stentorp}{2:102}{Romar}{3}{39, 39a-b}

\jhhousepic{074-05612.jpg}{Stig och Ann-Christin Julin}

\jhoccupant{Julin}{Stig \& Ann-Christin}{1992--}
Från Högskog lägenhet R:nr 2:99 köpte Stig Göran Julin, \textborn 15.02.1963, och hans hustru Marita Ann-Christin, \textborn 30.01.1966 i Bertby, Vörå, det område som ägdes av Tom och Benita Bäckstrand för att utstyckas som ny bostadstomt. Köpebrevet undertecknades den 06.04.1992 för ett område på 5000 m². Ett Gullringshus om 124 m² uppfördes snabbt och var klart för inflyttning redan på sommaren. Samtidigt uppfördes även garaget medan en jordkällare färdigställdes senare.

Stig är lastbilschaufför sedan 21 års ålder, men han startade sin karriär som chaufför redan som 15-åring med att köra hjullastare vid byggnadsbyrå H.Backlund där han senare fortsatte som lastbilschaufför. Han har också kört lastbil vid Mirkas slippappersfabrik vid Kiitola. En tid var han också chaufför åt Dan Sandås Transport från Pedersöre innan han fortsatte med att köra för Anita Kulas livsmedelsbutik, Fyrens
snabbköp, i Jeppo centrum. Sedan flera år har han haft anställning hos transportföretaget John Sandström.

År 1978 startade hans far Börje, tillsammans med brodern Bertel Julin, Julins Pälsfarm. När Börje senare dog 15.11.1987 övertog Stig hans ägarandel tills dess att farmen såldes 2011.

Ann-Christin hade sin första anställning på Vörå Sparbank fram till 1986 då hon fick anställning på Mirka som exportsekreterare, senare har hon haft arbetsuppgifter inom logistiken.
\begin{jhchildren}
  \item \jhperson{\jhname[Tobias]{Julin, Tobias}}{23.05.1991}{}
  \item \jhperson{\jhname[Daniela]{Julin, Daniela}}{31.10.1994}{}
\end{jhchildren}


\jhhouse{Silvast vattenandelslag}{2:90}{Romar}{3}{40}

\jhnooccupant{}

\jhbold{1947 – 2011}

Andelslaget bildades 5 juli 1947. Drivande kraft var \jhname[Gunnar Wadström]{Wadström, Gunnar} som var bankdirektör på HAB. Första styrelsen bestod av Torsten Forss ordf. , E.V. Jungerstam viceordf. I.V. Björklund, Leander Sandberg och Vilhelm Silfvast.


\jhhousepic{Vattentorn.jpg}{Vattentornet Anno 2007}

Styrelsen körde ner till Syd.Österbotten för att på ort och ställe stifta bekantskap med fungerande vattendistribution. Plats för pumphuset hyrdes på Fors hemman vid ån strax norr om tvättstranden. Stockar anskaffades och transporterades till ett område bakom Jarl Lövs hus. Stockarna skulle vara möjligast kvistfria för att minska risken för läckage i framtiden. Ett specialiserat företag för borrning av stockarna
anställdes och borrningen kunde genomföras. Diken, tillräckligt djupa, grävdes ner till 120 cm och kilometervis med stockar sattes ner och monterades ihop. 1948 togs ledningarna i bruk för det åvatten som nu levererades till bostad och ladugård. Vattnet användes som  vatten för kreaturen och för disk och rengöring. Dricksvatten fick fortsättningsvis hämtas ur brunnar med god vattenkvalitet.

Önskan om att få ett gott dricksvatten direkt ur kranen gjorde att andelslaget började utröna möjligheten att genom djupborrning hitta gott vatten nere vid ån. Våren 1964 kontaktades Wilhelm Norrgård  från Oravais och åtog sej uppdraget. Med spänning väntade man på  resultatet; skulle man hitta vatten och och tillräcklig mängd? Vatten hittades och i tillräcklig mängd, men det var salt! Man provpumpade en tid för att se om salthalten skulle avta och försvinna, men det hjälpte inte. Nu måste andelslaget se sej om efter en ny plats.

Man kom överens om ett tomtköp ett stycke upp längs Nylandsvägen med Birger och Lars Romar. Borrningsaggregatet flyttades dit och gav resultat. Den 30 aug 1964 kunde andelslaget konstatera att vatten av god kvalitet hade hittats och provpumpning startade för att utröna om mängden räckte till. Hålet på 112 m:s djup gav 60000 l per dygn Bedömningen var att det var tillräckligt.

10 nov. gavs fullmakt åt styrelsen, med Gunnar Almberg som ordf. och Eliel Brunell som kassör, att genast starta byggandet av ett vattentorn. 1350 m nytt plaströr drogs under järnvägen till det befintliga stocknätet i närheten av UF-lokalen. Wilhelm Norrgård som borrat, byggde också tornet på 42000 liter och efter en del problem med läckage kunde tornet fyllas och det nya vattnet med självtryck rinna ut i nätet. Tornet som var 9 m högt stod på en kulle som också gav ett naturligt fall på 9 m ner till bebyggelsen, varför trycket på 1,8 kg/cm2 kunde uppmätas i ledningarna. Allt detta blev klart till julen 1964. År 1967 påbörjades bytet till moderna plaströr och samtidigt anslöts nya konsumenter, i första hand längs Nylandsvägen.

Anslutningen av nya konsumenter, där den nya  skolan var en betydande faktor, gjorde att trycket upplevdes som otillräckligt. Beslut om att höja tornet fattades och 30 maj 1985 lyfts tornet upp på 10 m höga pelare. Trycket blev bättre, men kvaliteten började av andra orsaker gå ner, även om den mycket goda kvaliteten i initialskedet ledde till stridigheter huruvida det var för bra till djuren!

Med tiden började kvalitetsförsämringen förutsätta reningsåtgärder. Investeringskostnaderna för att råda bot på detta bedömdes så höga att andelslagets styrelse började undersöka möjligheterna att ansluta sig till Keppo Vattenandelslag. Lösningen blev att medlemmarna kunde ansluta sig som s.k. ``gamla medlemmar'' till Keppo Vatten, som efter ett år övertog ledningsnätet.

Silvast Vattenandelslag avslutade verksamheten 2011, men kvarstår som juridisk person och äger fortsättningsvis tornet och tomten. I den sista styrelsen satt: Ralf Häggblom, Greger Nygård, Kjell Norrgård, Göran Näs och Staffan Stenvall.



\jhhouse{Täppan}{2:27}{Romar}{3}{41} Tomtens storlek 2204 m²


\jhhousepic{095-05634.jpg}{Bertel och Marja-Liisa Julin}

\jhoccupant{Julin}{Bertel \& Marja-Liisa}{1987--}
Bertel Julin, \textborn 15.09.1944 i Jeppo, gifte sig 10.11.1968 med Marja-Liisa Hautala, \textborn 18.01.1944 i Ala-Härmä.
\begin{jhchildren}
  \item \jhperson{\jhname[Peter]{Julin, Peter}}{28.11.1974}{}, se Romar nr 42
  \item \jhperson{\jhname[Petrina]{Julin, Petrina}}{05.06.1980}{}, musikpedagog, g. Lindqvist i Kronoby
\end{jhchildren}
Bertel Per Erik Julin erhöll lagfart på Täppan 2:47 den 22.04.1986 genom partiellt arvsskifte efter sin far Walfrid Julins död 27.08.1976. Den 07.05.1986 beviljades byggnadstillstånd för bostaden och inflyttningen skedde 1987. Den bostad som tidigare stått på tomten (se mera nedan) revs innan byggandet av det nya huset påbörjades. Uthuset som fanns på stället revs 1988. På tomten hade stått Silvast sista rökbastu.

Bertel och Marja-Liisa hade efter giftermålet bott ca 10 år i Jakobstad därifrån de i september 1980 flyttade till Jeppo för att tillsammans med Bertels bror Börje Julin driva det nystartade bolaget Julins Pälsfarm. Till en början hyrde de in sig i bostadsaktiebolaget Gökbrinken (Felix Julins lokal) innan de 1987 flyttade in i eget hus.

Bertel Julin har fungerat som VD för Jakobe Betongstation. Utöver detta har han haft många förtroendeuppdrag, bl.a: Styrelseordf. för Nyko Frys och dess dotterbolag Feora Ab och Granlunds Farmtillbehör Ab, styrelsemedlem i Ostro Center Ab, styrelsemedlem i Jeppo Potatis Ab och styrelseordf. i dess dotterbolag Jeppo Voltti Ab, styrelseordf. i Jeppo Kraft andelslag, nämndeman i tingsrätten under 16 års tid som representant för pälsfarmarna.

Marja Liisa har varit tekn.biträde vid Ala-Härmä Apotek och vid 1:a Apoteket i Jakobstad. Efter flytten till Jeppo har hon främst arbetat deltid på den gemensamt ägda pälsfarmen fram till sin pensionering då farmområdet såldes.


\jholdhouse{Täppan}{2:47}{Romar}{3}{341}

\jhoccupant{Julin}{Walfrid \& Annie}{1965--\allowbreak 1986}
År 1965 köpte Walfrid och Anni Julin Täppan R:nr 2:47 med tillhörande bostadsbyggnad och uthus av lärarinnan \jhname[Julia Dahlström]{Dahlström, Julia}. Walfrid, \textborn 05.02.1917, gifte sig med Anni Stoor, \textborn 12.01.191 .

Walfrid har hela livet arbetat som timmerman och deltog bl.a. vid byggandet när kraftverksdammen vid kvarnen i Silvast förstärktes och höjdes i mitten på 1950-talet. Han deltog också vid byggandet av Sundells bageri och vid uppförandet av Villa Keppo. Anni arbetade bl.a. en tid vid Mirkas fabriksenhet vid Kiitola.

Walfrid \textdied 27.06.1986  ---  Anni \textdied 03.01.2004


\jholdhouse{Kiltäppan}{2:22, om 0,0003 mtl av Romar skattehemman}{Romar}{3}{341}


\jhoccupant{Dahlström}{Julia}{1952--\allowbreak 1965}
Den 31.07.1952 köper Julia Dahlström den bebyggda tomten av Elis och Anna Sofia Backlund. Genom diverse ägobyten på området mellan syskonen Backlund (se nedan) erhåller slutligen Julia Dahlström den södra delen av de sammantagna tomterna Fridkulla 2:19 och Kiltäppan 2:22. Styckningen är slutförd 22.12.1960 och Kiltäppan 2:22 omvandlas nu från mantal 0,0003, till ett arealbestämt område och får namnet \jhbold{Täppan} R:nr 2:47 och tillfaller i sin helhet Julia Dahlström.

Julia Dahlström föddes 05.08.1885 i Lappfjärd och fungerade som småskolelärarinna i bl.a. Lassila by. Hon flyttade till Vasa 26.04.1965.

\jhhousepic{Romar 341.jpg}{Sotaren Alexander Sandqvist rensar piporna i hus nr 341}

Under \jhname[Julia Dahlström]{Dahlström, Julia} tid som ägare av 0,0003 mantal av Kiltäppan 2:22, sedermera Täppan 2:47, bodde flera personer som hyresgäster i huset. I första hand \jhname[Lissi Bergström]{Bergström, Lissi}, som var Julia Dahlströms sambo. Här bodde också \jhname[Ann-Mari Lundberg]{Lundberg, Ann-Mari}, som var byns, ja kanske hela kommunens \jhbold{fotograf} och stod för fotografering av människor som ville ha foton av olika orsaker, bl.a. passfoton. Hon föddes i Maxmo 21.04.1895 och flyttade till Ekenäs 11.04.1956. Här bodde Paul och Marta Bro som nygifta sedan våren 1949. För Alf och Anja Ludén var fastigheten också en välkommen lösning efter giftermålet.

Omedelbart efter krigets slut 1944, hyrde Sigurd och Verna Norrgård halva huset för sin familj i väntan på att det egna huset ett par hundra meter längre fram längs Nylandsvägen skulle färdigställas.


\jholdhouse{Kiltäppan}{2:22 , 0,0006 mantal av Romar skattehemman}{Romar}{3}{342}


\jhoccupant{Backlund}{Ingeborg \& Johannes}{1942--\allowbreak 1952}
När Ida Matilda Backlund, \textborn 28.08.1868 i Korsholm, hustru till bonden Johan Backlund, \textborn 12.02.1865, avlider den 4 juni 1942, tillfaller Fridkulla skattelägenhet 2:19 om 0,0016 mantal och Kiltäppan 2:22 om 0,0003 mantal, bägge delar av Romar skattehemman 2 i Jungar by av Jeppo socken, förutom änklingen också barnen: Edit Emilia, Elis Edmund, Erik Edvin Backlund samt Ida Ingeborg Åkerholm och Ellen Elisabet Hedman.

1946 erhålles lagfart på ½ Fridkulla lägenhet 2: 19 och ½ Kiltäppan 2:22 av spinnmästare Elis Backlund, gårdsägare Erik Edvin Backlund, lärarinnan Ida Ingeborg Åkerholm och bondehustrun Ellen Elisabet Hedman.

Johan Adolf Backlund avlider 02.01.1949. Genom arvsförening 1949 och köp erhåller Ida Ingeborg jämte hennes make, författaren \jhname[Johannes Åkerholm]{Åkerholm, Johannes}, lagfart på 27/40 mantal av Fridkulla 2:19 och 27/40 mantal av Kiltäppan 2:22. För att kunna sälja området till \jhname[Julia Dahlström]{Dahlström, Julia} måste ytterligare en överenskommelse komma till stånd mellan intressenterna. Den 24.07.1952 gör man ett ägobyte där Ingeborg och Johannes Åkerholm överlåter sina andelar i Kiltäppan 2:22 till Elis och Anna Sofia Backlund i utbyte mot att dessa i sin tur överlåter sina andelar i Fridkulla 2:19. Nu kan försäljning av Kiltäppan 2:22 till Julia Dahlström ske, vilket också skedde den 31.07.1952.

\jhname[Ingeborg Åkerholm]{Åkerholm, Ingeborg}, småskolelärarinna, var en lång tid verksam vid Gunnars småskola. Som ovan framgår var hon gift med Johannes Åkerholm från Maxmo. Han har varit författare till några böcker; bl.a. Folket i Lövbrunnen, Walter går vilse. Han sysslade också en tid med taxiverksamhet. (se Mietala 111)


\jholdhouse{Kiltäppan}{2:22 ,  0,0006 mantal av Romar Skattehemman}{Romar}{3}{341}


\jhoccupant{Backlund}{Johan \& Ida}{1925--\allowbreak 1942}
Vid skiftet av Skog nr 1 och Romar nr 2 skattehemman i Jungar by av Jeppo kommun, vilket avslutades 1925, styckades Löte R:nr 2:12 i Löte Rno 2:21 ägd av Joel Mattsson Romar och Kiltäppan  R:nr 2:22 ägd av Johan Backlund och dennes hustru Ida Matilda Backlund. Åren efter 1925 torde huset blivit uppfört. Som ovan framgår revs detta av Bertel Julin 1986.

Johan Backlund, bonde \textborn 12.02.1865, \textdied 02.01.1949 gift med Ida Matilda, \textborn 28.08.1868, \textdied 04.06.1942
\begin{jhchildren}
  \item \jhperson{\jhname[Edit Emilia]{Backlund, Edit Emilia}}{03.11.1892}{}, diakonissa
  \item \jhperson{\jhname[Ida Ingeborg]{Backlund, Ida Ingeborg}}{29.01.1897}{}, småskolelärarinna
  \item \jhperson{\jhname[Elis Edmund]{Backlund, Elis Edmund}}{26.06.1899}{}, spinnmästare
  \item \jhperson{\jhname[Ellen Elisabet]{Backlund, Ellen Elisabet}}{28.06.1905}{}, bondehustru
  \item \jhperson{\jhname[Erik Edvin]{Backlund, Erik Edvin}}{15.04.1911}{}
\end{jhchildren}
Makarna fick utöver ovan nämnda, 6 barn som dog i späd ålder: Artur, Ester, Helga, Olivia, Uno och Viola.


\jhhouse{Fridkulla}{2:19}{Romar}{3}{42, 42a}


\jhhousepic{094-05633.jpg}{Peter och Birgitta}

\jhoccupant{Julin}{Peter \& Birgitta}{2000--}
Peter Per-Erik Julin, \textborn 28.11.1974, gifte sig 22.05.1999 med Birgitta Pirjo Marianne Blomström, \textborn 04.09.1970 i Jakobstad. Byggnaden uppfördes år 2000 på den tomt som är placerad närmast järnvägen och uppgår till 3590 m².

Peter är regionchef på Havator Oy i västra Finland, specialiserad på lyftuppdrag. På fritiden är han intresserad av motionsidrott och har deltagit i några halvmaratonlopp. Birgitta är exportsekreterare på KWH-Mirka.
\begin{jhchildren}
  \item \jhperson{\jhname[Joakim Per-Erik]{Julin, Joakim Per-Erik}}{13.10.1999}{}
  \item \jhperson{\jhname[Kristoffer Per-Erik]{Julin, Kristoffer Per-Erik}}{25.12.2001}{}
\end{jhchildren}



\jholdhouse{Fridkulla}{2:19, Löte lägenhet}{Romar}{3}{342}

\jhpic{Soldater1941.jpg}{Marscherande trupp längs Nylandsvägen, förbi Åkerholms hus och Jungar folkskola. Året är 1941, foto Henrik Kennola.}

\jhoccupant{ Åkerholm}{Johannes \& Ingeborg}{1952--\allowbreak 1993}
Johannes Jakobsson Åkerholm, \textborn 16.04.1901 i Brudsund,Maxmo, gifte sig 25.03.1931 med Ida Ingeborg Joh.dr. Backlund, \textborn 29.01.1897 i Jeppo. Ingeborg var länge småskollärarinna vid Mietala där paret bodde i den småskola som uppförts efter att Jakob Forsgård den 16 febr. 1919 utarrenderat en tomt från Mietala skattehemman  till kommunen ``för en fast småbarnsskola''. Hon kom att verka där 1922--\allowbreak 1956 för att därefter flytta till sin hemgård på Löte hemman nr 2 invid Jungar folkskola som hon erhållit via arvsförening 1949.

Under tiden i Mietala (Mietala nr 111) hade Johannes påbörjat taxiverksamhet som så småningom avvecklades. Han var samtidigt författare och utgav några böcker som långt berörde skärgårdslivet, som han var uppväxt med i Maxmo. ``Folket i Lövbrunnen'' , ``Walter går vilse'' och ``Dunkel över havsbandet'', kan nämnas som exempel på hans författarskap. På senare år skrev han också artiklar i Österbottniska Posten och Jakobstads Tidning. För eleverna i Jungar skola på 1950 och -60-talen är han en känd profil i sin svarta basker när han hämtade ved ur lidret för husets uppvärmning. Huset revs i början av 1990-talet i samband att en ny underfart för landsvägen till Purmo grävdes under järnvägen med början 1992.
\begin{jhchildren}
  \item \jhperson{\jhname[Ole]{Åkerholm, Ole}}{13.09.1932}{05.12.1934}
  \item \jhperson{\jhname[Ole Johannes]{Åkerholm, Ole Johannes}}{28.10.1935}{20.10.1936}
  \item \jhperson{\jhname[Solbritt Ingeborg]{Åkerholm, Solbritt Ingeborg}}{09.02.1938}{}
\end{jhchildren}

Ingeborg \textdied 04.01.1962  ---  Johannes \textdied 21.01.1963


\jhoccupant{Backlund}{Johan \& Ida}{}
Johan Adolf Backlund, \textborn 28.08.1868, gifte sig med Ida Matilda, \textborn 12.02.1865 i Mustasaari (Korsholm) (se 341)\jhvspace{}


\jhhouse{Jungar folkskola}{893-408-2-13}{Romar}{3}{343}


\jhhousepic{Sparvback skola.jpg}{Jungar folkskola}

\jhoccupant{Folkskola}{1880}{1989}
Den 20.01.1879 godkände direktionen för Jungar folkskola den ritning som den jeppofödda läraren \jhname[Henrik Backlund]{Backlund, Henrik} gjort upp. I byggnaden, som skulle uppföras för skolan, planerades ett klassrum, en slöjdsal och fyra bostadsrum för läraren. Bröderna \jhname[Karl och Niklas Laggnäs]{Laggnäs, K. och N} från Munsala vann anbudstävlingen och vårvintern 1880 timrades huset upp på Laggnäs för att med utnyttjande av det sista slädföret transportera timran till den tomt på Sparvbacken (ibland används det snarlika ``Sparrbacken'') som bönderna Johan, \textborn 1834, och Gustav, \textborn 1838, Romar skänkt. Dragkamp fördes om platsen, men 2 dec 1879 hade direktionen bestämt att Sparvbacken blir skolans plats. Skolan blev färdig under sommaren 1880 och kunde invigas 25 oktober samma år.

Det hade funnits ett motstånd mot skolan. Tankens arbete ansågs enbart höra herrskapsfolket till. Men det fanns de som tänkte längre och arbetade för att skolan skulle bli kommunal, vilket den inte var till en början. Då också guvernören ställde sig på skolvännernas sida vad gällde styrningen av brännvinspengarna, ebbade småningom motståndet ut. Den 25 sept. 1887 godkände kommunalstämman övertagandet av skolan, byggnaden, uthus och inventarier. Lärare var till en början \jhname[Anders Lundén]{Lundén, Anders}. Eleverna började strömma till också från de gårdar där man tidigare varit emot skoletableringen och snart kunde man inte ta emot alla. Därför är det  med tillfredställelse vi kan citera Vasabladet 11 juni 1892: ``Ett hedrande kommunalstämmobeslut fattades i Jeppo vid kommunalstämman den 30 maj. Vid stämman beslöts nämligen att vid folkskolan skulle tillbyggas en vinkel innehållande en större lärosal, bostad för lärarinnan, samt en bagarstuga. Tillbyggnaden skall vara fullt färdig inom 3 år, men lärarinnans rum inom 2 år. Enligt ett tidigare beslut kommer folkskolan att i sommar brädfodras och målas. Då tillbyggnaden blir färdig, blir otvivelaktigt Jeppo folkskola bland de ståtligaste landsfolkskolebyggnader i Österbotten, och det är vackert så för en kommun på knappa 20 mantal.''

1909 uppgick elevantalet till 105 och följande år till 125 vilket ledde till att en andra lärarinna måste anställas; Emma Wistbacka. Från
tidigare arbetade John Svedberg och Lise Gardberg i skolan.

Trycket på skolan blev nu så stort att Gunnar skola byggdes 1911 och en del elever från de södra delarna kunde slussas dit, men då antalet elever i Jungar skola redan 1913 på nytt uppgick till 113 st, byggdes en 3:e skola på Bärs. 1919-20 hade elevantalet på nytt stigit till 107 och man blev tvungen att undervisa 1:a och 2:a klassen i skift. 1921 byggdes så den sista distriktskolan på Måtar och Jeppo hade då 4  skoldistrikt. Skolan utvidgades med utrymmen för småskolan och bostad för lärare i början av 1930-talet då också matsalen kom till.

Verksamheten i skolan fortgick nu någorlunda lugnt fram till 1939 då vinterkriget startade den 30 nov. Skolans utrymmen blev under krigsåren använda av krigsmakten och i första hand som en laddningsstation för granater. Här satt unga kvinnor och laddade dessa farliga tingestar och trotylpulvret som dammade av sig färgade deras hår rött. De färdigladdade granaterna transporterades sedan till de baracker som byggts ca 100 m nordost om Nylandsvägen på Tegelbacken och vaktades dygnet runt av soldater bakom taggtråd och skyttevärn (se nr 33 och 333). Militärpersonal var också stationerad vid skolan.

Efter kriget återgick verksamheten till det normala för skolan fram till 1966 då eleverna från 1-6:e klass fick flytta till den nya skolan.
Under tiden som folkskola har följande lärare verkat i byggnaden:
\begin{center}
  \begin{tabular}{l l l l}
    \hline
    \jhname[J. Sjöblad]{Sjöblad, J.} & 1879--82 & \jhname[Lisa Gardberg]{Gardberg, Lisa} & 1889--92\\
    \jhname[A. Lundén]{Lundén, Anders}  & 1882--87 & \jhname[Dagmar Holmén]{Holmén, Dagmar} & 1892--\allowbreak 1900\\
    \jhname[M. Vikman]{Vikman, M.} & 1885--86 & \jhname[Ellen Thors]{Thors, Ellen} & 1900--40\\
    \jhname[Joh. Svedberg]{Svedberg, Joh.} & 1887--95 & \jhname[Greta Bertlin]{Bertlin, Greta} & 1908--10\\
    \jhname[O.V. Holmén]{Holmén, O.V.} & 1895--99 & \jhname[Emma Wistbacka]{Wistbacka, Emma} & 1910--11\\
    \jhname[Fredrik Thors]{Thors, Fredrik} & 1899--\allowbreak 1940 & \jhname[Karin Sjöblom]{Sjöblom, Karin} & 1910--12\\
    \jhname[Sven Jungar]{Jungar, Sven} & 1940--66 & \jhname[Karin Sjöblom]{Sjöblom, Karin} & 1913--52\\
    \jhname[Agda Norrback]{Norrback, Agda} & 1952--56 & \jhname[Linda Lönnfors]{Lönnfors, Linda} & 1911--13\\
    \jhname[Ellen Nygård]{Nygård, Ellen} & 1956--66 & \jhname[Elna Sandberg]{Sandberg, Elna} & 2,5 år u. kriget + 1946--66\\ \hline
  \end{tabular}
\end{center}

Många av lärarna har bott i skolans bostäder, men vartefter det blivit mera allmänt att de byggde egna bostäder, kunde bostäderna också hyras ut till andra. Bl.a.familjen Alice Sandvik bodde här åren 1971--80.

Efter att klasserna 1--6 flyttat till den nya centrumskolan 1966, intogs skolbyggnaden av medborgarskolans elever, som tills dess hållit till i den s.k. ``Såggården'' (se nr 406). Lärare \jhname[Edvin Westin]{Westin, Edvin}, som också bott i medborgarskolans bostad i Såggården flyttade med och fick pension efter ett år. Ny föreståndare blev \jhname[Helge Hedman]{Hedman, Helge}. Andra lärare inkopplades också i medborgarskolans undervisning; \jhname[Gunda Hedman]{Hedman, Gunda} 1967--68 och ämneslärarna \jhname[Eva Dahlskog]{Dahlskog, Eva} i huslig ekonomi, \jhname[Rafael Rissanen]{Rissanen, Rafael} i manlig slöjd, \jhname[Siv Smeds]{Smeds, Siv}, \jhname[Kerstin Sved]{Sved, Kerstin} och \jhname[Helena Helsing]{Helsing, Helena} i kvinnligt handarbete. Kurser i barnavård kunde hållas i Sparbankens klubblokal med khd. hustrun \jhname[Ruth Sandell]{Sandell, Ruth} och här kunde det fort hända att undervisningsdockornas armar skruvades loss under obevakade ögonblick. Mycket hyss hörde normalt till medborgarskolans liv.

När den nya grundskolereformen tömt Jungar skola på elever, blev den en plats för diverse fritidsaktiviteter. Här höll bl.a. styrkelyftarna till i den tidigare matsalen, som efter diverse förbättringar användes fram till 1986, då en ny konditionshall i närheten av den nya skolan togs i bruk (se nr 106).

Jämsides med denna aktivitet startade den första barnträdgården/daghemmet i den tidigare föreståndarbostaden. Tack vare föräldrars aktivitet kunde man år 1972 köra igång verksamheten, som fungerade här fram till hösten 1984 då ett nytt daghem kunde öppnas i närheten av skolan (se nr 105).

Jungar skola användes också en tid som lagerlokal för Mirka. Marthorna kunde också använda utrymmen för sin uppbevaring av service, men stod de sista åren tom. Den skola som byggts 1880 revs hösten 1989.



\jhhouse{Tegelbacken vapendepå}{2:}{Romar}{3}{333}

\jhnooccupant{}

\jhpic{Tegelbacken-vapen 001.jpg}{Skiss över området för vapendepån uppgjord av Christer Fors på basen av Ruth Bäckstrands hågkomster 2013. Jfr med Romar hemman nr 33}
