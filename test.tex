\documentclass[b5paper,twoside,10pt]{book}
\author{Fjalar Fors \and Christer Fors
}
\title{Jungar by \\
gammal och ny
}
% Packages
\usepackage{polyglossia}
\setdefaultlanguage{swedish}
\usepackage{fontspec}
\usepackage{xunicode}
\usepackage{xltxtra}
\usepackage[parfill]{parskip}
\usepackage{calc}
\usepackage{etoolbox}
\usepackage{graphicx}
\usepackage[space]{grffile}
\usepackage{wrapfig}
\usepackage{url}
%\usepackage{datetime2}
%\usepackage{hyphenat}
%\usepackage{showframe}
\usepackage[makeindex]{imakeidx}
\usepackage[nottoc]{tocbibind}
\usepackage{enumitem}
\usepackage{longtable}
\usepackage{multicol}
\usepackage{marginnote}
\usepackage{nameref}
\usepackage{ifoddpage}
\usepackage{needspace}
\usepackage{lipsum}
\usepackage{titling}
\usepackage[nobottomtitles*]{titlesec}
\usepackage[format=plain,font=footnotesize,labelfont=bf,skip=0.5\baselineskip,figurename=Bild,labelsep=endash]{caption}
\usepackage{microtype}
%\usepackage[top=Bcm, bottom=Hcm, outer=Ccm, inner=Acm, heightrounded, marginparwidth=Ecm, marginparsep=Dcm]{geometry}
\usepackage[b5paper,top=15mm,bottom=15mm,outer=35mm,inner=15mm,marginparwidth=30mm,marginparsep=0.6em]{geometry}
\usepackage[
	unicode=true,
	colorlinks=true,
	urlcolor=black,
	linkcolor=black,
	citecolor=black,
	filecolor=black,
	pdfauthor={Fjalar Fors},
	pdftitle={Jungar byhistorik},
	pdfsubject={Byahistorik},
	pdfkeywords={Jeppo, Jungar, historik, byar, gårdar}
]{hyperref}
\usepackage{bookmark}

\setmainfont[
	Path = typsnitt/LinLibertine/,
	Extension = .otf,
	UprightFont = LinLibertine_R,
	BoldFont = LinLibertine_RZ, % Linux Libertine O Regular Semibold
	ItalicFont = LinLibertine_RI,
	BoldItalicFont = LinLibertine_RZI,
]{Libertine}
\setsansfont[
	Path = typsnitt/LinLibertine/,
	Extension = .otf,
	UprightFont = LinBiolinum_R,
	BoldFont = LinBiolinum_RB,
	ItalicFont = LinBiolinum_RI,
]{Biolinum}

\defaultfontfeatures{Mapping=tex-text}
% and/or - see comments
\defaultfontfeatures{Ligatures=TeX}

% Variables
\pagestyle{plain}

% Settings
\setlist{nosep}
\setlist[enumerate,1]{label=\arabic*), ref=\arabic*}
\setcounter{topnumber}{4}
\setcounter{bottomnumber}{4}
\setcounter{totalnumber}{10}

\newlength{\jhoccupantskip}
\setlength{\jhoccupantskip}{1\parskip}
\addtolength{\jhoccupantskip}{1\baselineskip}
%\setlength{\wrapoverhang}{\marginparwidth} % wrapfig intrusion on margin
%\addtolength{\wrapoverhang}{\marginparsep} % - " -
\newlength{\jhoccupantstart}
\newlength{\jhoccupantstop}

% Commands
\makeatletter
\newcommand{\gobblepars}{%
	\@ifnextchar\par%
		{\expandafter\gobblepars\@gobble}%
		{}
}
\makeatother
\newcounter{occupantcounter}
\newcounter{nextoccupantcounter}
\makeatletter
\newcommand{\jhoccupantlabel}[1]{%
\refstepcounter{occupantcounter}%
\immediate\write\@auxout{%
  \string\newlabel{occ:\arabic{occupantcounter}}{{1}{\thepage}{{\unexpanded{#1}}}{occupantcounter.\number\value{occupantcounter}}{}}%
}%
}
\makeatother
\newcommand{\jhnooccupant}[0]{\jhoccupantlabel{}}
\newcommand{\jhoccupant}[3]{%
	\needspace{4\baselineskip}%
	%\message{start \the\jhoccupantstart}%
	%\message{stop \the\jhoccupantstop}%
	%\invisiblesubsection{#1}%
	%\setlength{\jhoccupantstart}{\pagetotal}%
	%\leavevmode%
	\gobblepars%
	\checkoddpage%
	  \ifoddpage%
			\invisiblesubsection{#1}%
			\leavevmode%
			\marginnote{%
				\unskip\parbox[t]{0.95\marginparwidth}{\raggedright \textbf{#1} \\ #2 \\ #3}%
				\jhoccupantlabel{ --- #1, #2}%
			}%
		\else%
			\invisiblesubsection{#1}%
			\leavevmode%
			\marginnote{%
				\unskip\parbox[t]{0.95\marginparwidth}{\raggedleft \textbf{#1} \\ #2 \\ #3}%
				\jhoccupantlabel{ --- #1, #2}%
			}%
		\fi%
	%\marginnote{%
	%	\textbf{#1} \\ #2 \\ #3%
	%	\jhoccupantlabel{ --- #1, #2}%
	%}%
	\gobblepars%
	%\setlength{\jhoccupantstop}{\pagetotal}%
}
\newcommand{\jhchapter}[1]{%
	\chapter*{#1}%
	\addcontentsline{toc}{chapter}{\protect#1}%
}
\newcommand{\jhsection}[1]{\section*{#1}}
\newcommand{\jhsubsection}[1]{\subsection*{#1}}
\newcommand{\jhsubsubsection}[1]{\subsubsection*{#1}}
\newcommand{\jhhouse}[5]{%
	\Needspace{16\baselineskip}%
	\section*{#1}%
	\setcounter{nextoccupantcounter}{\theoccupantcounter+1}%
	\addcontentsline{toc}{section}{\protect#1\nameref{occ:\arabic{nextoccupantcounter}}}%
	\vspace{-1.5\baselineskip}
	R:nr #2 av #3 hemman Karta #4 nr \jhbold{#5}
}
\newcommand{\jholdhouse}[5]{%
	\subsection*{#1}%
	\vspace{-1\baselineskip}
	R:nr #2 av #3 hemman Karta #4 nr \jhbold{#5}
}
\newcommand{\jhname}[2][]{\notblank{#1}{#1}{#2}\index[namn]{#2}}
\newcommand{\jhperson}[3]{#1, \textborn #2\notblank{#3}{--\textdied #3}{}}
\makeatletter
\newcommand{\checkheight}[1]{%
  \par \penalty-100\begingroup%
  \setbox8=\hbox{#1}%
  \setlength{\dimen@}{\ht8}%
  \dimen@ii\pagegoal \advance\dimen@ii-\pagetotal
  \ifdim \dimen@>\dimen@ii
    \break
  \fi\endgroup}
\makeatother
\newcommand{\jhhousepic}[3][]{%
	%\Needspace{12\baselineskip}%
	\Needspace{0.52\textwidth}%
	\begin{wrapfigure}{O}{0.5\textwidth}
		\vspace{-1\baselineskip}
		\centering
			\ifstrempty{#2}{%
				\rule{0.48\textwidth}{0.32\textwidth}
			}{%
				\includegraphics[width=0.48\textwidth]{bilder/#2}
			}
			%\ifstrempty{#2}{}{{\footnotesize#2}}
			\caption{#3}
			\ifstrempty{#1}{}{\label{#1}}
	\end{wrapfigure}
}
\newcommand{\jhpic}[4][]{
	\begin{figure}[h]
		\centering
			\ifstrempty{#2}{%
				\rule{1\textwidth}{0.67\textwidth}
			}{%
				\ifstrempty{#4}{%
					\includegraphics[width=1\textwidth]{bilder/#2}
				}{%
					\includegraphics[width=#4\textwidth]{bilder/#2}
				}%
			}
			%\ifstrempty{#2}{}{{\footnotesize#2}}
			\caption{#3}
			\ifstrempty{#1}{}{\label{#1}}
	\end{figure}
}
\newcommand\invisiblesection[1]{%
	\refstepcounter{section}%
	\addcontentsline{toc}{section}{\protect\numberline{\thesection}#1}%
	\sectionmark{#1}
}
\makeatletter
\newcommand\invisiblesubsection[1]{%
%	\refstepcounter{section}%
	%\markboth{}
	\vskip \baselineskip
	%\addcontentsline{toc}{subsection}{\protect#1}%
	%\sectionmark{#1}
	\@afterheading%
}
\makeatother
\newcommand{\jhbold}[1]{\textbf{#1}}
\newcommand*{\jhvspace}[1][2]{\vspace{#1\baselineskip}}
\def\textborn {\char"2605\relax} %{\char"002A }
\def\textdied {\char"271D\relax} %{\char"2020 } %{\char"271D }
% Envs
\newenvironment{jhchildren}{%
	\begin{samepage}%
	\par\noindent Barn:%
	\begin{enumerate}[label=\roman*)]}
{%
	\end{enumerate}%
	\end{samepage}%
}

\makeindex[name=namn,title=Namn]

% Remove numbering/labeling from bib entries
\makeatletter
\renewcommand\@biblabel[1]{}
\makeatother

% Document
\begin{document}
\frontmatter

\maketitle

\tableofcontents

\mainmatter


\jhchapter{kapitelnamn}

\jhsection{sektionsnamn}

\jhsubsection{undersektionsnamn}

\jhhouse{husnamn}{registernummer}{hemman}{kartblad}{husnummer}

\jhoccupant{efternamn}{förnamn}{årtal}

\lipsum[1-1]

\jhhousepic{}{bildtext}

\lipsum[2-6]

\jhpic{}{bildtext}

\lipsum[7-8]

\jhoccupant{Westerlund}{Jakob \& Kristina}{1910--\allowbreak 1912}

Jakob Westerlund, \textborn 25.10.1850 på Biggas, son till Jan Jakobsson Biggas, gift 11.10.1874 m. Kristina Eriksdotter Pet, \textborn 04.11.1854 på Pet.
\begin{jhchildren}
  \item \jhperson{Joel}{1875}{}, till Amerika
  \item \jhperson{Viktor}{1877}{}, till Nykarleby 1897
  \item \jhperson{Johannes}{1880}{}, till Amerika
  \item \jhperson{Hanna}{1883}{1951}, gift Sjöblom
  \item \jhperson{Wilhelm}{1886}{}, till Amerika, gift med jeppoflicka (1 barn)
  \item \jhperson{\jhbold{Eric}}{1888}{}, bonde på Fors (5/96 mantal)
  \item \jhperson{Emil}{1890}{}
  \item \jhperson{Ingrid Maria}{1892}{}
  \item \jhperson{Gustav}{1894}{}
  \item \jhperson{Ture Evert}{1897}{1952}, taxi, limonadtillverkning i Silvast
\end{jhchildren}

Jakob var byskollärare i Korsholm 1870-72. Familjen flyttade via Nykarleby till Stenbacken år 1886/-88, där de bodde fram till 1910, då de flyttade till Silvast. Jakob tjänstgjorde som landspolis med många förtroendeuppdrag i kommun och församling. Ägare till en del av Fors skattehemman Rno 3:8; svårt att veta var de bodde under dessa två år.

Jakob \textdied 08.03.1914  -- 	Kristina \textdied 14.01.1936


\jhhouse{Blom}{3:65}{Fors}{6}{95}

\jhoccupant{Sjölind}{Johan \& Emma}{2016--}

Paret Sjölind köpte dödsboet i december 2015 och har sedan dess renoverat huset från grunden. Det beräknas bli inflyttningsklart sommaren	2017. Se bakgrund, hus 81.


\jhoccupant{Dödsbo}{Lassén}{2012--\allowbreak 2015}

Under den här perioden utnyttjades lägenheten som gemensam samlingsplats för syskonen med familjer.

\jhhousepic{}{}


\jhoccupant{Lassén}{John \& Lea}{1957--\allowbreak 2012}

John Eliel Lassén, \textborn 09.09.1915 i Esse, gifte sig med Lea Alice Johanna Anderssén, \textborn 08.07.1921, från Måtars.
\begin{jhchildren}
  \item \jhperson{Lars Christer}{25.10.1948}{} Bor i Uppsala, Sverige
  \item \jhperson{Ann-Christine}{07.05.1951}{} Bor i Larsmo
  \item \jhperson{Ulla Rose-May}{22.09.1952}{03.08.1987 i Stockholm}
  \item \jhperson{Gun-Lis Charlotta}{30.01.1954}{} Bor i Sverige
  \item \jhperson{Hans Johan}{29.11.1956}{14.4.1957}
\end{jhchildren}

\jhhousepic{../test_2407x1604.jpg}{}

John och Lea bodde ursprungligen på Måtar men hade jobb i Silvast. År 1957 köpte de lägenheten av William och Johanna Sjöbloms dödsbo efter att ha bott på hyra det första året. Paret drev under åren 1952/53 - 1968 gemensamt diversehandeln ``Jeppoboden'' (karta 5 nr 380) på nuvarande Östra Jeppovägen, norra delen av UF:s tomt. Lea hade redan i unga år arbetat på Lainas kafé, gått en kort frisörsutbildning och efteråt arbetat som frisörska med tillhörande kemikaliabutik i s.k. Jakobssons hus (37x), granne till järnvägsstationen.

John sårades allvarligt i vinterkriget. Under fortsättningskriget var han som krigsveteran på hemmafronten och tjänstgjorde som effektiv 	soldatgosseledare i Jeppo skyddskår. Efter kriget fungerade han också som arméns materialförvaltare i Jeppo och hade då att hantera kinkiga ärenden.

\jhhousepic{../test_2407x1604_noexif.jpg}{}

I det civila blev John känd som mycket idrottsintresserad och var bl.a. god långdistanslöpare. Han syntes därför ständigt och aktivt med vid olika tävlingsarrangemang både vinter och sommar. Föreningsintresset ledde till att han blev s.k. chartermedlem i Lions Club Jeppo. John var även verksam som skatterådgivare.

John \textdied 13.01.2000  --  Lea \textdied 20.02.2012



\end{document}

