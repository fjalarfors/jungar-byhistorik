\jhchapter{Romar, hemman Nr 2}

Romar – varifrån kommer det namnet? Som så ofta tvistar de lärda. Runar Nyholm har i en artikel i Vasabladet 24.6.1976 gett en spännande redogörelse över de lärdas åsikter. Enligt P.E. Ohls är namnet bildat av ordet romar= tala högröstat och kommer från ljudet i ån i närheten. Besläktade ord är åumar=ekar el. en ko som råmar. Jämförande ord lär finnas i Rombaksbotn i Norge och Hundalsälven, där ljudet är likt ett hundskall. Ralf Karsten menar å sin sida att namnet kan komma från personnamnet Hrodmarr. Han menar också att det namn, som i vardagslag länge använts om Romar, nämligen ”Fjössmans”, varit ett okvädningsord för en här boende tassig gubbe. Ortsborna anser dock att varför skulle ett helt hemmansnummer, ett av 1600-talets största, få namn efter en fjollig gubbe. Här bodde istället på 1600-talet en fjärdingsman och från detta har vi böjningen fjässmans som blivit fjössmans.

Romar utgjorde ursprungligen 1 mantal, men utökades på 1640-talet till 1 ½ mtl. Hemmanet ägdes 1551-1553 av Henrik Jönsson,  1556-1560 av Olof Henriksson, 1562-1580 av Henrik Jönsson, 1582-1603 av Simon Olsson, 1603-1622 av Hans Larsson, 1624-1629 av Hans Hansson, 1633-1673 av Hans Mikkelsson som är son till Mikkel Mattsson på Slangar där han 1630 i RL är antagen som fjärdingsman. Han är också antagen till fjärdingsman på Romar, 1675-1681 av Matts Hansson, fjärdingsman, 1683-1705 av mågen  Jakob Knutsson, 1709-1713 av Matts Hansson, måg till Mikkel Mattsson. Ett hemman utan namn uppgår i Romar 1640 omfattande ½ mtl. Med innehavare 1567-1580 av änkan Elin, 1592-1623 av Mårten Hindersson, 1624-1631 av Hans Markusson och 1634-1640 av Markus Jöransson.

Den av Matthias Wörman år 1740 uppgjorda kartläggningen upptar på sin karta endast en gård på Romar hemman, placerad mellan den smala landsvägen och ån, ganska exakt där Ralf och Svea Romars hus nr 10 står i dag.  1783 har hemmanet delats i 4 lika stora delar. I slutet av den svenska tiden i 1807 uppgjord personbok, är Romar det mest folkrika hemmanet i det som senare  skulle bli Jungar by.  Romar hade då 47 personer. Tack vare samhällsutvecklingen har på Romar hemmans mark bibehållits en god bosättning då bl.a. områden som hör till hemmanet längs Nylandsvägen tagits i anspråk för byggande. Antalet brukningsenheter på hemmanet har dock stadigt minskat under efterkrigstiden och idag finns endast 3 aktivt kvar. De tidigare lägenheternas jord är utarrenderad eller såld.

Skog hemman omfattas av kartorna 2 och 3.


KARTA KOMMER HIT


\jhsubsection{Lägenheter på Romar}

\jhhouse{ Åbacka}{2:91}{Romar}{2}{3, 3a}

\jhoccupant{Romar}{Erik}{1996 -}
Erik, \textborn 14.06.1951, övertog lägenheten år 1996. Han har fungerat som truckförare inom Jeppo Potatis AB nästan ända sedan företagets start år 1976. Innan dess arbetade han på Oy Mirka Ab mellan åren 1972-76. Han lever ogift.

Lägenheten omfattar förutom tomtmarken också 0,7 ha åker och 4,3 ha skog.


\jhoccupant{Romar}{Georg \& Vera}{1944-1996}
Georg, \textborn 08.07.1906, gift år 1941 med Vera Jåssis från Vörå \textborn 05.12.1915.
\begin{jhchildren}
  \item \jhperson{Greta}{1942}{}
  \item \jhperson{Helge}{1948}{}
  \item \jhperson{\jhbold}{Erik}{1951}{}
  \item \jhperson{Gunvor}{1955}{}
\end{jhchildren}

Georg och Vera uppförde bostaden 1949-50 efter att genom frivilligt köp år 1944 från stomfastigheten förvärvat området. År 1963
tillkom  ekonomibyggnaden.

Georg var byggmästare,utexaminerad 1930. Till en början hade han uppdrag i södra Finland, men flyttade efter några år tillbaka till Österbotten. Han blev småningom engagerad i många byggprojekt på orten. Han hade ansvaret för byggandet av Jeppo Andelsmejeri 1932/33 och  han var ansvarig byggmästare i byggandet av Jeppo Handelslags nya stora affärsfastighet i Silvast år 1942. Därtill uppförde han som entrepenör Jeppo Sparbanks nya byggnad i Silvast. Han fungerade också som övervakande byggmästare vid uppförandet av den nya centrumskolan i Jeppo i början av 1960-talet. Georg satt under 1960-talet med i kommunens byggnadsnämnd, varav en tid som dess ordförande.

Georg \textdied 10.02.1996  --  Vera \textdied 17.03.1990


\jhhouse{Elfstrand}{2:93}{Romar}{2}{8-8a}

\jhoccupant{Romar}{Anders \& Henrik}{2008 -}
Bröderna Anders och Henrik Romar övertog dödsboets fastigheter år 2008. Anders bor i Närpes och Henrik i Jakobstad.


\jhoccupant{Romar}{Birger \& Margit}{1944 -}
Paul Birger Romar, \textborn 22.11.1910 på Romar, gifte sig 10.08.1947 med Margit Lovisa Backlund, \textborn 05.12.1919 på Holmen. Birger hade 11.12.1944 övertagit lägenheten tillsammans med bröderna Selim och Georg. Lägenheten skiftades och Birger övertog ½ lägenheten med 14 odlat och 40 ha skog. Selim hade sedan början av 1930-talet brukat halva hemmanet. Birger och Margit byggde småningom ny ladugård, men hann ta den i bruk endast en kort tid innan Birger plötsligt avled när han hade kört kontrollassistentens utrustning till Lövbacken.

Efter en tid lades mjölkproduktionen ner och marken utarrenderades. Änkan Margit bodde kvar i bostaden en tid tillsammans med sönerna innan hon flyttade till sin bror Valter Backlund på Holmen, där också han blivit änkling 1969. Senare flyttade hon till en hyresbostad i centrum.

Birger var medlem i folkskoldirektionen, och vägnämnden förutom ledamot i Jeppo Sparbanks styrelse.

Margit besökte Ev. Folkhögskolan 1937/38 och fungerade som expedit vid Jeppo-Oravais Handelslag fram till giftermålet. Hon var länge verksam inom Martharörelsen och deltagit aktivt i flera körer och hon tyckte om att läsa dikter också i offentliga sammanhang.
\begin{jhchildren}
  \item \jhperson{\jhbold}{Paul Anders}{04.06.1949}{}
  \item \jhperson{\jhbold}{Henrik Gustav}{14,11,1953}{}
\end{jhchildren}

Birger \textdied 11.03.1968  --  Margit \textdied 28.10.2005


\jhoccupant{Romar}{Anders \& Ida}{1923-1944}
Anders Gustafsson Romar, \textborn 04.06.1877, gifte sig 14.05.1899 med Ida Maria Johansdr. \textborn 26.12.1877. Under hans tid som husbonde revs den gamla gården ner, som stått på åbacken alldeles nära ån tvärs över rån. Den hade delats med Johan Romars familj, men nu byggde de båda familjerna varsitt hus längs landsvägen år 1923 (se nr 10).
\begin{jhchildren}
  \item \jhperson{Gustav Selim}{11.01.1900}{}
  \item \jhperson{Sigrid Fredrika}{29.09.1903}{}
  \item \jhperson{Johannes Georg}{08.07.1906}{}
  \item \jhperson{Henrik Evald}{27.12.1908}{03.03.1940}, stupade
  \item \jhperson{\jhbold}{Paul Birger}{22.11.1910}{}
  \item \jhperson{Hjördis Katarina}{06.01.1918}{}
\end{jhchildren}

Anders \textdied 01.01.1953  --  Ida \textdied 04.09.1949


\jhhouse{Gamla gården}{2:93}{Romar}{2}{308}

\jhoccupant{Romar}{Gustav \& Brita}{1872 -}
Gustaf Andersson (Romar), \textborn 09.07.1838, gifte sig med Brita Kajsa Eriksdr. \textborn 13.01.1839. Makarna hade sitt jordbruk på åbacken vid Romar. Hemmanet delades sannolikt 1872 med brodern Johan, men fortsattes att brukas gemensamt.
\begin{jhchildren}
  \item \jhperson{Erik}{16.03.1863}{}, emigrerade till Amerika
  \item \jhperson{Henrik Gustaf}{29.05.1873}{28.12.1893}
  \item \jhperson{Anders}{04.06.1877}{}
\end{jhchildren}

Gustaf \textdied 11.06.1895  --  Brita \textdied 22.04-1898


\jhoccupant{Romar}{Anders \& Maria}{}
Anders  Thomasson Romar, \textborn 07.12.1812, gift med Maria Andersdr. Bärs, \textborn 26.05.1809.
\begin{jhchildren}
  \item \jhperson{Johan}{12.09.1834}{}, se 310
  \item \jhperson{Anders}{15.05.1836}{}
  \item \jhperson{Gustav}{09.07.1838}{}
  \item \jhperson{Erik}{16.04.1842}{}
  \item \jhperson{Matts}{28.03.1844}{}
  \item \jhperson{Cajsa Greta}{23.01.1848}{}
  \item \jhperson{Susanna}{12.02.1851}{}
\end{jhchildren}

Anders och Maria var bönder på ½  av föräldrarnas hemman som han delat med brodern Erik. Den 8 dec. 1854 köpte han tillsammans med Johan Thomasson Romar 55/576 mtl av Romar skattehemman nr 2 av bonden Gustav Eliasson Romar för 500 rubel silver och därtill sytning åt sytningsmannen Anders Romar och hans hustru Maria Jakobsdr.

Anders \textdied 16.05.1896  --  Maria  \textdied 02.11.1876 
