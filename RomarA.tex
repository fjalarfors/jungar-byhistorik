%%%
% [chapter] Romar, hemman Nr 2
%
\jhchapter{Romar, hemman Nr 2}

Romar – varifrån kommer det namnet? Som så ofta tvistar de lärda. Runar Nyholm har i en artikel i Vasabladet 24.6.1976 gett en spännande redogörelse över de lärdas åsikter. Enligt P.E. Ohls är namnet bildat av ordet romar  = tala högröstat och kommer från ljudet i ån i närheten. Besläktade ord är åumar  = ekar el. en ko som råmar. Jämförande ord lär finnas i Rombaksbotn i Norge och Hundalsälven, där ljudet är likt ett hundskall. Ralf Karsten menar å sin sida att namnet kan komma från personnamnet Hrodmarr. Han menar också att det namn, som i vardagslag länge använts om Romar, nämligen ``Fjössmans'', varit ett okvädningsord för en här boende tassig gubbe. Ortsborna anser dock att varför skulle ett helt hemmansnummer, ett av 1600-talets största, få namn efter en fjollig gubbe. Här bodde istället på 1600-talet en fjärdingsman och från detta har vi böjningen fjässmans som blivit fjössmans.

Romar utgjorde ursprungligen 1 mantal, men utökades på 1640-talet till 1 ½ mtl. Hemmanet ägdes 1551--\allowbreak 1553 av Henrik Jönsson, 1556--\allowbreak 1560 av Olof Henriksson, 1562--\allowbreak 1580 av Henrik Jönsson, 1582--\allowbreak 1603 av Simon Olsson, 1603--\allowbreak 1622 av Hans Larsson, 1624--\allowbreak 1629 av Hans Hansson, 1633--\allowbreak 1673 av Hans Mikkelsson, som är son till Mikkel Mattsson på Slangar, där han 1630 i RL är antagen som fjärdingsman. Han är också antagen till fjärdingsman på Romar, 1675--\allowbreak 1681 av Matts Hansson, fjärdingsman, 1683--\allowbreak 1705 av mågen Jakob Knutsson, 1709--\allowbreak 1713 av Matts Hansson, måg till Mikkel Mattsson. Ett hemman utan namn uppgår i Romar 1640 omfattande ½ mtl. Innehavare 1567--\allowbreak 1580 var änkan Elin, 1592--\allowbreak 1623 Mårten Hindersson, 1624--\allowbreak 1631 Hans Markusson och 1634--\allowbreak 1640 Markus Jöransson.

Den av Matthias Wörman år 1740 uppgjorda kartläggningen upptar på sin karta endast en gård på Romar hemman, placerad mellan den smala landsvägen och ån, ganska exakt där Ralf och Svea Romars hus nr 10 står i dag. 1783 har hemmanet delats i 4 lika stora delar ägda av änkan Susanna, Tomas, Johan Johansson och Johan Mattsson med 5/24 mtl vardera. I slutet av den svenska tiden i år 1807 uppgjord personbok, är Romar det mest folkrika hemmanet i det som senare skulle bli Jungar by. Romar hade då 47 personer. Tack vare samhällsutvecklingen har på Romar hemmans mark bibehållits en god bosättning då bl.a. områden som hör till hemmanet längs Nylandsvägen tagits i anspråk för byggande.

\jhhousepic[pic: broadress]{Brofonden.jpg}{S.k. Brofonden, okt. 1956}

Romar gårdsgrupp hade i tiden en egen kyrkbåt som var större till formatet än vanliga båtar. Den var avsedd för två roddare, men stakades ofta fram med långa störar. Sommartid var en kavelbro vanligtvis utlagd vid ``Färistranden'' för att förkorta vägen till kyrkan. Samtidigt fungerade den för nyttotransporter. Efter kriget startade lärarinnan Karin Sjöblom och diakonissan Backlund ett upprop, Brofonden, för att åstadkomma en fast broförbindelse mellan åstränderna, men initiativet rann ut i sanden.

Antalet brukningsenheter på hemmanet har dock stadigt minskat under efterkrigstiden och idag finns endast 3 aktiva kvar. De tidigare lägenheternas jord är utarrenderad eller såld.



Romar hemman omfattas av kartorna nr \jhbold{2---3}.


<--- se KARTA nr 2 --->


%%%
% [subsection] Romar kvarn
%
\jhsubsection{Romar kvarn} på Romar hemman, 18??--1905, (ungefärlig plats på karta nr 1)

Vid Romarsbäcken, något öster om landsvägen, fanns i tiden en kvarn, som upphörde med verksamheten omkring 1905. Senare användes vattenhjulet för att driva en pärthyvel. (Gustaf Fors)


%%%
% [subsection] Romar Mejeribolag
%
\jhsubsection{Romar Mejeribolag} på Romar hemman, 1896--?, (platsen okänd)

Den 17 febr. 1896 undertecknade sex bönder från Romar och Skog stiftelseurkunden för en mejerirörelse under benämningen Romar mejeribolag:

``Vi underskrifne hafva oss emellan ingått bolag för drifvande av mejerirörelse med följande vilkor:
\begin{enumerate}
  \item Rörelsen drifves med såkallad handseparator å Romar hemman i Jeppo kommun under benämning af Romar mejeri bolag.
  \item Hvar och en af oss bolagsmän tillskjuter etthundrafemtio mark.
  \item Bolagets affärer skötes af en för året å bolagsstämma vald disponent.
  \item I frågor rörande bolagets angelägenheter gälla flertalet af de närvarandes beslut.
  \item Bolagets böcker böra afslutas med juni månads utgång och räknas således räkenskapsåret från den 1 juli till 1 juli följande år, och böra räkenskaperna vara färdiga till granskning en månad derefter af bolaget eller den de dertill utser.
  \item All af rörelsen uppkommande vinst och förlust delas lika på en hvar bolagsman, dock får ingen vinstutdelning ske så länge bolaget har skuld på anlägningen eller rörelsen. Och varder detta kontrakt af oss i tillkallade vittnens närvaro underskrifvet och bestyrkt.
\end{enumerate}

Jeppo den 17 Februari 1896

Bönderna:
Johan Johansson Romar  E.g.h. -- Isak Eriksson Romar (Bomärke) --  Thomas Johansson Romar (Bom) -- Anders Johansson Skog  (Bom) -- Jonas Jungell (Bom) -- F.d. Bonden Johansson Romar (Bom)

Vittnen:
Fredrik Skog (Bom), Bonde i Jeppo -- Anders Fredriksson Skog(Bom), Dreng''

Rörelsen torde ha startat och mejeriet omnämns i skrifter, men var det stått placerat på Romar har fallit i historiens glömska. Dess verksamma tid blev ganska kort och mejeriet på Silvast med ångdriven separator blev destinationen för mjölken från Skog och Romar i början av 1900-talet.



%%%
% [subsection] Lägenheter på Romar
%
\jhsubsection{Lägenheter på Romar}



%%%
% [house] Åbacka
%
\jhhouse{Åbacka}{2:91}{Romar}{2}{3, 3a}


%%%
% [occupant] Romar
%
\jhoccupant{Romar}{\jhname[Erik]{Romar, Erik}}{1996--}
Erik, \textborn 14.06.1951, övertog lägenheten år 1996. Han har fungerat som truckförare inom Jeppo Potatis AB nästan ända sedan företagets start år 1976. Innan dess arbetade han på Oy Mirka Ab mellan åren 1972-76. Han lever ogift.

Lägenheten omfattar förutom tomtmarken också 0,7 ha åker och 4,3 ha skog.


\jhhousepic{003-05525.jpg}{Erik Romar}

%%%
% [occupant] Romar
%
\jhoccupant{Romar}{\jhname[Georg]{Romar, Georg} \& \jhname[Vera]{Romar, Vera}}{1944--\allowbreak 1996}
Georg, \textborn 08.07.1906, gift år 1941 med Vera Jåssis från Vörå \textborn 05.12.1915.
\begin{jhchildren}
  \item \jhperson{\jhname[Greta]{Romar, Greta}}{1942}{}
  \item \jhperson{\jhname[Helge]{Romar, Helge}}{1948}{}
  \item \jhperson{\jhbold{\jhname[Erik]{Romar, Erik}}}{1951}{}
  \item \jhperson{\jhname[Gunvor]{Romar, Gunvor}}{1955}{}
\end{jhchildren}
Georg och Vera uppförde bostaden 1949-50 efter att genom frivilligt köp år 1944 från stomfastigheten förvärvat området. År 1963 tillkom ekonomibyggnaden.

Georg var byggmästare,utexaminerad 1930. Till en början hade han uppdrag i södra Finland, men flyttade efter några år tillbaka till Österbotten. Han blev småningom engagerad i många byggprojekt på orten. Han hade ansvaret för byggandet av Jeppo Andelsmejeri 1932/33 och  han var ansvarig byggmästare i byggandet av Jeppo Handelslags nya stora affärsfastighet i Silvast år 1942. Därtill uppförde han som entrepenör Jeppo Sparbanks nya byggnad i Silvast. Han fungerade också som övervakande byggmästare vid uppförandet av den nya centrumskolan i Jeppo i början av 1960-talet. Georg satt under 1960-talet med i kommunens byggnadsnämnd, varav en tid som dess ordförande.

Georg \textdied 10.02.1996  ---  Vera \textdied 17.03.1990



%%%
% [house] Elfstrand
%
\jhhouse{Elfstrand}{2:93}{Romar}{2}{8-8a}


%%%
% [occupant] Romar
%
\jhoccupant{Romar}{\jhname[Anders]{Romar, Anders} \& \jhname[Henrik]{Romar, Henrik}}{2008--}
Bröderna Anders och Henrik Romar övertog dödsboets fastigheter år 2008. Anders bor i Närpes och Henrik i Jakobstad.\jhvspace{}


\jhhousepic{005-05528.jpg}{Anders och Henrik Romar}

%%%
% [occupant] Romar
%
\jhoccupant{Romar}{\jhname[Birger]{Romar, Birger} \& \jhname[Margit]{Romar, Margit}}{1944--}
Paul Birger Romar, \textborn 22.11.1910 på Romar, gifte sig 10.08.1947 med Margit Lovisa Backlund, \textborn 05.12.1919 på Holmen. Birger hade 11.12.1944 övertagit lägenheten tillsammans med bröderna Selim och Georg. Lägenheten skiftades och Birger övertog ½ lägenheten med 14 odlat och 40 ha skog. Selim hade sedan början av 1930-talet brukat halva hemmanet. Birger och Margit byggde småningom ny ladugård, men hann ta den i bruk endast en kort tid innan Birger plötsligt avled när han hade kört kontrollassistentens utrustning till Lövbacken.

Efter en tid lades mjölkproduktionen ner och marken utarrenderades. Änkan Margit bodde kvar i bostaden en tid tillsammans med sönerna innan hon flyttade till sin bror Valter Backlund på Holmen, där också han blivit änkling 1969. Senare flyttade hon till en hyresbostad i centrum.

Birger var medlem i folkskoldirektionen, och vägnämnden förutom ledamot i Jeppo Sparbanks styrelse.

Margit besökte Ev. Folkhögskolan 1937/38 och fungerade som expedit vid Jeppo-Oravais Handelslag fram till giftermålet. Hon var länge verksam inom Martharörelsen och deltagit aktivt i flera körer och hon tyckte om att läsa dikter också i offentliga sammanhang.

\begin{jhchildren}
  \item \jhperson{\jhbold{\jhname[Paul Anders]{Romar, Paul Anders}}}{04.06.1949}{}
  \item \jhperson{\jhbold{\jhname[Henrik Gustav]{Romar, Henrik Gustav}}}{14,11,1953}{}
\end{jhchildren}

Birger \textdied 11.03.1968  ---  Margit \textdied 28.10.2005


%%%
% [occupant] Romar
%
\jhoccupant{Romar}{\jhname[Anders]{Romar, Anders} \& \jhname[Ida]{Romar, Ida}}{1923--\allowbreak 1944}
Anders Gustafsson Romar, \textborn 04.06.1877, gifte sig 14.05.1899 med Ida Maria Johansdr. \textborn 26.12.1877. Under hans tid som husbonde revs den gamla gården ner, som stått på åbacken alldeles nära ån tvärs över rån. Den hade delats med Johan Romars familj, men nu byggde de båda familjerna varsitt hus längs landsvägen år 1923 (se nr 10).
\begin{jhchildren}
  \item \jhperson{\jhbold{\jhname[Gustav Selim]{Romar, Gustav Selim}}}{11.01.1900}{}
  \item \jhperson{\jhname[Sigrid Fredrika]{Romar, Sigrid Fredrika}}{29.09.1903}{}
  \item \jhperson{\jhbold{\jhname[Johannes Georg]{Romar, Johannes Georg}}}{08.07.1906}{}
  \item \jhperson{\jhname[Henrik Evald]{Romar, Henrik Evald}}{27.12.1908}{03.03.1940}, stupade
  \item \jhperson{\jhbold{\jhname[Paul Birger]{Romar, Paul Birger}}}{22.11.1910}{}
  \item \jhperson{\jhname[Hjördis Katarina]{Romar, Hjördis Katarina}}{06.01.1918}{}
\end{jhchildren}

Anders \textdied 01.01.1953  ---  Ida \textdied 04.09.1949



%%%
% [oldhouse] Gamla gården
%
\jholdhouse{Gamla gården}{2:93}{Romar}{2}{308}


\jhhousepic{Romars gamla hus.jpg}{Romars gamla gård, nr 308}

%%%
% [occupant] Romar
%
\jhoccupant{Romar}{\jhname[Gustav]{Romar, Gustav} \& \jhname[Brita]{Romar, Brita}}{1872--}
Gustaf Andersson (Romar), \textborn 09.07.1838, gifte sig med Brita Kajsa Eriksdr. \textborn 13.01.1839. Makarna hade sitt jordbruk på åbacken vid Romar. Hemmanet delades sannolikt 1872 med brodern Johan, men fortsattes att brukas gemensamt.
\begin{jhchildren}
  \item \jhperson{\jhname[Erik]{Romar, Erik}}{16.03.1863}{}, emigrerade till Amerika
  \item \jhperson{\jhname[Henrik Gustaf]{Romar, Henrik Gustaf}}{29.05.1873}{28.12.1893}
  \item \jhperson{\jhbold{\jhname[Anders]{Romar, Anders}}}{04.06.1877}{}
\end{jhchildren}

Gustaf \textdied 11.06.1895  ---  Brita \textdied 22.04-1898


%%%
% [occupant] Romar
%
\jhoccupant{Romar}{\jhname[Anders]{Romar, Anders} \& \jhname[Maria]{Romar, Maria}}{}
Anders  Thomasson Romar, \textborn 07.12.1812, gift med Maria Andersdr. Bärs, \textborn 26.05.1809.
\begin{jhchildren}
  \item \jhperson{\jhname[Johan]{Romar, Johan}}{12.09.1834}{}, se 310
  \item \jhperson{\jhname[Anders]{Romar, Anders}}{15.05.1836}{}
  \item \jhperson{\jhbold{\jhname[Gustav]{Romar, Gustav}}}{09.07.1838}{}
  \item \jhperson{\jhname[Erik]{Romar, Erik}}{16.04.1842}{}
  \item \jhperson{\jhname[Matts]{Romar, Matts}}{28.03.1844}{}
  \item \jhperson{\jhname[Cajsa Greta]{Romar, Cajsa Greta}}{23.01.1848}{}
  \item \jhperson{\jhname[Susanna]{Romar, Susanna}}{12.02.1851}{}
\end{jhchildren}

Anders och Maria var bönder på ½  av föräldrarnas hemman som han delat med brodern Erik. Den 8 dec. 1854 köpte han tillsammans med Johan Thomasson Romar 55/576 mtl av Romar skattehemman nr 2 av bonden Gustav Eliasson Romar för 500 rubel silver och därtill sytning åt sytningsmannen Anders Romar och hans hustru Maria Jakobsdr.

Anders \textdied 16.05.1896  ---  Maria  \textdied 02.11.1876



%%%
% [house] Bäckstrand
%
\jhhouse{Bäckstrand}{2:89}{Romar}{2}{9, 9a-b}


\jhhousepic{004-05527.jpg}{Olof och Gun-Lis Bäckstrand}

%%%
% [occupant] Bäckstrand
%
\jhoccupant{Bäckstrand}{\jhname[Olof]{Bäckstrand, Olof}}{1962--}
Olof Bäckstrand, \textborn 10.02.1927, gift 1972 med Gun-Lis Sten från Purmo, \textborn 23.04.1931.

Olof övertog  ½ hemmanet 1962. Den andra  halvan övertogs av hans bror Edvin. Hemmanet skiftades mellan bröderna 1984, efter att till dess fungerat enligt sämjobyte.

Det nya huset byggdes 1963. Föräldrarna Emil och Ester Bäckstrand flyttade in tillsammans med Olof, som fortsatte jordbruket med huvudsaklig inriktning på mjölkproduktion också efter att Gun-Lis anlänt till gården. Det gamla huset revs 1964.

Olof har varit aktiv i såväl kommunala som föreningssammanhang. Han har suttit som medlem i Jeppo kommunalfullmäktige och varit speciellt engagerad i Skogsvårdsföreningen, Skogsandelslaget, Lantmannagillet och Jeppo Församlings kyrkofullmäktige.


%%%
% [oldhouse] Bäckstrand hemmans gamla gård
%
\jholdhouse{Bäckstrand hemmans gamla gård}{2:}{Romar}{2}{309}


\jhhousepic{Backstrands gamla hus.png}{Bäckstrands gamla hus, nr 309}

%%%
% [occupant] Bäckstrand
%
\jhoccupant{Bäckstrand}{\jhname[Emil]{Bäckstrand, Emil} \& \jhname[Ester]{Bäckstrand, Ester}}{1932--\allowbreak 1962}
Emil, \textborn 31.12.1895, gifte sig 11.07.1920 med Ester Jungar, \textborn 17.05.1891. Emils far, Johan Romar, som senare antog efternamnet Bäckstrand, avled 1917 och efterlämnade änkan Maria med 6 barn. I och med arvsskiftet 1932 sålde hon sin ½ andel åt sonen Emil  med hustru Ester. Den andra halvan tillföll de övriga syskonen.

Hemmanet brukades som sterbhus fram till 1941, då Emil och Ester köpte tomtskiftet varefter största delen av den andra hemmansdelen stegvis köptes tillbaka av systrarna, börjande med utskogsskiftet.
\begin{jhchildren}
  \item \jhperson{\jhbold{\jhname[Edvin]{Bäckstrand, Edvin}}}{05.02.1923}{}
  \item \jhperson{\jhname[Gunnel]{Bäckstrand, Gunnel}}{08.07.1924}{24.09.1925} i difteri
  \item \jhperson{\jhbold{\jhname[Olof]{Bäckstrand, Olof}}}{18.02.1927}{}
  \item \jhperson{\jhname[Ragni]{Bäckstrand, Ragni}}{01.10.1928}{}
  \item \jhperson{\jhname[Sven]{Bäckstrand, Sven}}{18.03.1930}{19.08.1930} av svaghet
  \item \jhperson{\jhname[Lars Emil]{Bäckstrand, Lars Emil}}{04.07.1931}{12.07.1931} av svaghet
\end{jhchildren}

Emil \textdied 24.07.1965  ---  Ester \textdied 01.11.1974


%%%
% [occupant] Romar
%
\jhoccupant{Romar}{\jhname[Johan Th:son]{Romar, Johan Th:son} \& \jhname[Maria]{Romar, Maria}}{1898--\allowbreak 1932}
Johan Thomasson Romar, \textborn 01.05.1869, gift med Maria Hägglund, \textborn 27.07.1875. Maria och Johan antog i samband med påbörjat skifte på Romar hemmansnummer, efternamnet \jhbold{Bäckstrand}, som också kom att bli den nya hemmansdelens namn.
\begin{jhchildren}
  \item \jhperson{\jhbold{\jhname[Emil]{Romar, Emil}}}{31.12.1895}{}
  \item \jhperson{\jhname[Signe]{Romar, Signe}}{03.08.1891}{}, gift Jungell
  \item \jhperson{\jhname[Ester]{Romar, Ester}}{18.12.1905}{}, gift Ahlström
  \item \jhperson{\jhname[Elna]{Romar, Elna}}{23.03.1908}{}, gift Sandberg
  \item \jhperson{\jhname[Astrid]{Romar, Astrid}}{06.03.1913}{}
  \item \jhperson{\jhname[Agda]{Romar, Agda}}{02.12.1915}{}, gift Björk
\end{jhchildren}

Johan \textdied 06.05.1917  ---  Maria \textdied 04.09. 1955

Utöver sitt ägande mantal om 0,1132, förvaltade Johan samtidigt 0,0566 mtl av Romar hemman. På gården bodde också inhyses Jakob Niemi, \textborn 1884, med hustrun Lisa, \textborn 1883, och två döttrar.


%%%
% [occupant] Romar
%
\jhoccupant{Romar}{\jhname[Thomas Joh:son]{Romar, Thomas Joh:son} \& \jhname[Sanna]{Romar, Sanna}}{1871--\allowbreak 1898}
Thomas Johansson Romar, \textborn 06.01.1838, gift med Sanna Thomasdotter \textborn 25.10.1844.
Thomas blev ägare till hemmanet 1871.Han dog 03.01 1899. Hustrun dog 21.08.1925.
\begin{jhchildren}
  \item \jhperson{\jhname[Sanna Kajsa]{Romar, Sanna Kajsa}}{28.06.1867}{}
  \item \jhperson{\jhbold{Johan (Johannes)}}{01.05.1869}{}
  \item \jhperson{\jhname[Erik]{Romar, Erik}}{14.12.1871}{}
  \item \jhperson{\jhname[Thomas]{Romar, Thomas}}{15.03.1877}{}
  \item \jhperson{\jhname[Maria Sofia]{Romar, Maria Sofia}}{01.04.1879}{}
  \item \jhperson{\jhname[Anders]{Romar, Anders}}{01.06.1881}{}
  \item \jhperson{\jhname[Lisa Johanna]{Romar, Lisa Johanna}}{24.05.1883}{}
  \item \jhperson{\jhname[Anna Lovisa]{Romar, Anna Lovisa}}{01.08.1884}{}
\end{jhchildren}


%%%
% [occupant] Romar
%
\jhoccupant{Romar}{\jhname[Johan]{Romar, Johan} \& \jhname[Kajsa-Greta]{Romar, Kajsa-Greta}}{1845--\allowbreak 1871}
Johan Erik Thomasson Romar, \textborn 02.03.1815, gift den 05.05.1837 med Kajsa Greta Johansdotter(Fors), \textborn 30.08.1814 el. 1815.
Hemmanet ägdes tillsammans med brodern Anders. Se nedan. Johan dog 18.10 1871. Hustruns dödsdatum okänt.
\begin{jhchildren}
  \item \jhperson{\jhbold{\jhname[Thomas]{Romar, Thomas}}}{06.01.1838}{}
  \item \jhperson{\jhname[Anders]{Romar, Anders}}{}{}
  \item \jhperson{\jhname[Erik]{Romar, Erik}}{}{}
  \item \jhperson{\jhname[Johan]{Romar, Johan}}{}{}
\end{jhchildren}


%%%
% [occupant] Romar
%
\jhoccupant{Romar}{\jhname[Anders]{Romar, Anders} \& \jhname[Maria]{Romar, Maria}}{1845--\allowbreak 1871}
Hemmanet omfattande 19/48 skattemantal och 1/48 kronomantal ägdes tillsammans med brodern Johan. Se ovan. Anders Thomasson var född 07.12.1812 och dennes hustru Maria Andersdotter Bärs var född 26.03.1809. Hon dog 19.11.1876 och maken Anders 16.05.1896.
\begin{jhchildren}
  \item \jhperson{\jhname[Johan]{Romar, Johan}}{19.09.1834}{}
  \item \jhperson{\jhname[Anna Maja]{Romar, Anna Maja}}{12.02.1851}{}
\end{jhchildren}


%%%
% [occupant] Romar
%
\jhoccupant{Romar}{Thomas Thomasson}{}
Thomas Thomasson Romar, \textborn 04.04.1785, gift med Anna Andersdotter Silvast, \textborn 03.04.1790. Thomas och Anna övertog Thomas föräldrars hemman på Romar. omfattande 5/24 mantal.

Vid samma tid på 1820-talet ägde Matts Jespersson 5/24 mantal, Eric Carlsson, \textborn 13.04.1796, hu. Anna, \textborn 06.01.1788,  5/48 skatte mtl och 5/48 krono mtl och Henrik Hansson med hu. Brita samma mtl. fördelning (5/48 + 5/48) av Romar skattehemman.
\begin{jhchildren}
  \item \jhperson{\jhbold{\jhname[Anders]{Romar, Anders}}}{07.12.1812}{}
  \item \jhperson{\jhbold{\jhname[Johan Erik]{Romar, Johan Erik}}}{02.03.1815}{}
  \item \jhperson{\jhname[Thomas]{Romar, Thomas}}{}{}
  \item \jhperson{\jhname[Sanna]{Romar, Sanna}}{}{}
\end{jhchildren}

Thomas \textdied 11(14).02.1847  ---  Anna \textdied 24.04.1850


%%%
% [occupant] Romar
%
\jhoccupant{Romar}{\jhname[Thomas Th:son]{Romar, Thomas Th:son} \& \jhname[Karin Mattsdr.]{Romar, Karin Mattsdr.}}{}
Thomas Thomasson Romar, \textborn 09.12.1757, gift med Karin Mattsdotter från Munsala, \textborn 1762. Thomas och Karin övertog sannolikt en del av hemmanet efter Thomas föräldrar.

Barn bl.a.: Thomas, \textborn 04.04.1785.

Thomas Thomasson \textdied 1792  ---  Karin Mattsdotter \textdied 26.11.1803


År 1810, efter den ryska tidens början, fanns på Romar hemmansnummer 18 följande mantalsägare:
\begin{enumerate}
  \item Matts Jespersson med hustrun Lisa, 5/24 skatte mtl.
  \item Hans Thomasson med hustrun Anna,   5/24      ”
  \item Johan Carlsson med hustrun Lisa,   5/24      ”       + 5/48 krono mtl.
  \item Jakob Andersson med hustru Maria,  5/48      ”       + 5/48     ”
\end{enumerate}

År 1801, medan Finland ännu var en del av Sverige, förändrades mantalsvärdet för hemmanen och Romar hemmansnummer, som tidigare omfattat 1 ½ mantal, fick nu ett nytt värde, nämligen 5/6 mantal. Enligt detta nya värde fördelades mantalsinnehavet detta år enligt följande:
\begin{enumerate}
  \item Matts       5/24 mtl
  \item Thomas      5/24  ”
  \item Carl        5/24  ”
  \item John.       5/24  ”
  \item Totalt:   20/24 mtl = 5/6 mtl
\end{enumerate}

År 1793 omfattade Romar 5/12 skatte mtl + 5/12 krono mtl. De innehades enligt följande:
\begin{enumerate}
  \item Susanna , änka 5/24 mtl.
  \item Thomas         5/24  ”
  \item Johan          5/12  ”
\end{enumerate}

År 1783 innehades äganderätten av:
\begin{enumerate}
  \item Änkan Susanna    5/24  mtl
  \item Thomas           5/24   ”
  \item Johan Johansson  5/24   ”
  \item Johan Mattsson   5/24   ”
\end{enumerate}



%%%
% [house] Romar
%
\jhhouse{Romar}{2:115}{Romar}{2}{10, 10a}


%%%
% [occupant] Romar
%
\jhoccupant{Romar}{\jhname[Richard]{Romar, Richard}}{1999--}
Richard Romar, \textborn 28.01.1966, sambo med Anneli Granqvist från Molpe, Korsnäs.
\begin{jhchildren}
  \item \jhperson{\jhname[Robin Romar]{Romar, Robin}}{1997}{}
  \item \jhperson{\jhname[Kevin Romar]{Romar, Kevin}}{2002}{}
  \item \jhperson{\jhname[Adrian Romar]{Romar, Adrian}}{2007}{}
\end{jhchildren}
Richard köpte hemmanet 1999 av sina föräldrar Ralf och Svea Romar. Bosatt i Smedsby, Korsholm, sköter han jordbruket som bisyssla till sitt ordinarie arbete som transportchef för Scandic Trans. Den huvudsakliga driftsinriktningen är potatisodling. År 2007 brann den äldre ekonomibyggnaden ner, tillsammans med alla maskiner. Året därpå byggdes den upp på nytt och maskinparken förnyades för fortsatt verksamhet.

Anneli är ekonom och arbetar för Korsholms kommun.


\jhhousepic{007-05530.jpg}{Ralf och Svea Romar}

%%%
% [occupant] Romar
%
\jhoccupant{Romar}{\jhname[Ralf]{Romar, Ralf} \& \jhname[Svea]{Romar, Svea}}{1955--\allowbreak 1999}
Ralf Romar, \textborn 03.11.1932, gift med Svea Björkvik, \textborn 03.07.1934.
\begin{jhchildren}
  \item \jhperson{\jhname[Gun-May]{Romar, Gun-May}}{15.03.1955}{}, gift Pettersson, Uppsala, Kambo gård
  \item \jhperson{\jhbold{\jhname[Richard]{Romar, Richard}}}{28.01.1966}{}, bor i Smedsby
\end{jhchildren}
Ralf och Svea köpte ½ hemmanet 1955 av Ralfs föräldrar Edvin och Saima Romar. Den andra halvan köpte de  år 1969. Till en början bedrev Ralf och Svea mjölkproduktion, kombinerad med arbete utanför gården. Senare blev gården kreaturslös med inriktning på spannmål och potatis.


%%%
% [occupant] Romar
%
\jhoccupant{Romar}{\jhname[Edvin]{Romar, Edvin} \& \jhname[Saima]{Romar, Saima}}{1932-55/-69}
Edvin, \textborn 08.12.1899, gift med Saima Sandberg, \textborn 23.01.1905.
\begin{jhchildren}
  \item \jhperson{\jhname[Birgitta]{Romar, Birgitta}}{1929}{}, gift Mäenpää, barn: Helena och May-Gret
  \item \jhperson{\jhbold{\jhname[Ralf]{Romar, Ralf}}}{03.11.1932}{}
  \item \jhperson{\jhname[Ulla-Stina]{Romar, Ulla-Stina}}{18.08.1948}{}, gift Holländer, barn: Marika \& Mats
\end{jhchildren}

Hemmanet köptes  1932  av Edvins föräldrar Johan och Kajsa Romar. Som alla andra hade gården mjölkkor som huvudinriktning. Edvin och Saima skötte hemmanet fram till 1955, då det delades så att sonen Ralf och hustrun Svea erhöll ½ hemmanet.  Den återstående delen av hemmanet brukade de fram till 1969, då också den andra halvan inköptes av son och sonhustru.

Edvin \textdied 08.12.1974  ---  Saima \textdied 15.02.1993


%%%
% [occupant] Romar
%
\jhoccupant{Romar}{\jhname[Johan]{Romar, Johan} \& \jhname[Kajsa]{Romar, Kajsa}}{1882--\allowbreak 1932}
Johan Johansson, \textborn 28.04.1865, gift med Cajsa Heikfolk, \textborn 25.09.1963.
\begin{jhchildren}
  \item \jhperson{\jhname[Maria Fredrika]{Romar, Maria Fredrika}}{1889}{1978} i Vasa, gift Roos
  \item \jhperson{\jhname[Hilda Katarina]{Romar, Hilda Katarina}}{1891}{1967}, gift Videll
  \item \jhperson{\jhname[Susanna Lydia]{Romar, Susanna Lydia}}{1894}{1955}, gift Elenius
  \item \jhperson{\jhname[Ester Sofia]{Romar, Ester Sofia}}{1897}{1965}, gift Jungarå
  \item \jhperson{\jhbold{\jhname[Johan Edvin]{Romar, Johan Edvin}}}{08.12.1899}{08.12.1974}
  \item \jhperson{\jhname[Edit Irene]{Romar, Edit Irene}}{1902}{1938}, gift Julin
  \item \jhperson{\jhname[Gertrud Elisabet]{Romar, Gertrud Elisabet}}{1906}{1992}, gift Sandberg
\end{jhchildren}
Johan övertog hemmanet av sina föräldrar Johan Andersson Romar och hans hustru Maria i två repriser;  1882 och 1884. Alltför många år hann paret Johan och Kajsa inte helhjärtat ägna åt sitt jordbruk. Redan den 1 juli 1893, nyss fyllda 28 år, åtog han sig det svårskötta uppdraget som ordförande i kommunalnämnden, fattigvårdsnämnden m.m. Därtill kassörskapet för ett antal olika kassor förenade med dessa. Dessa uppdrag skötte han sedan i drygt 44 år fram till 1937, då han avgick av hälsoskäl.

Johan Romar var ``Kommunen'' personifierad. Han blev så inväxt i den, att hans tal och handlande gav intrycket att ingen person kunde vara mera mån om sin egendom än han var om kommunens. Han hade ingen hjälp av någon personal, förutom  att han så småningom fick rätt att avlöna hjälp vid skatteuppbörder. Efter hand hjälpte också någon av hans döttrar till. Hela tiden skulle han ha hand om verkställigheten av de kommunala funktionerna såsom sammanträden, protokollskrivning, kontakter med andra kommuner och inrättningar. Han inkasserade skatter, skötte utbetalningar och bokföring av en hel hop olika kassor m.m. Beaktar man att han skulle sköta dessa uppdrag i hemmet i någon vrå, kan man ana vilken enorm prestation det här var fråga om. Avlöningen var naturligtvis inte stor om man tar i beaktande allt det arbete han utförde.

En motsvarande effektiv prisvärd förvaltning kan inte frambringas idag. Det är naturligt att hans fotografi avtäcktes i kommunens sessionssal som ett tacksamhetsbevis, när han 1935 fyllde 70 år.

Under hans tid som husbonde genomfördes skiftesförrättning på Romar och den stora gård längs ån man tidigare delat med Anders Romar och dennes familj, revs ner. Den hade stått precis tvärs över rån mellan de bägge lägenheterna och också huset delades på tvärs. Man byggde nu varsin ny bostad nära landsvägen år 1923 och det är dessa gårdar som fortfarande står kvar. Den bostad som Johan och Kajsa uppförde för sin familj bebos idag av deras barnbarn Ralf Romar med hustru Svea.

Johan \textdied 11.01.1939  ---  Kajsa \textdied 26.04.1935


%%%
% [oldhouse] Romar
%
\jholdhouse{Romar}{2:}{Romar}{2}{310}

%%%
% [occupant] Romar
%
\jhoccupant{Romar}{\jhname[Johan A:son]{Romar, Johan A:son} \& \jhname[Maria E:dr]{Romar, Maria E:dr}}{1872--\allowbreak 1884}
Johan Andersson Romar, \textborn 12.09.1834, gift med Maria Eriksdr. Back, \textborn 03.02.1837.
\begin{jhchildren}
  \item \jhperson{\jhname[Anders]{Romar, Anders}}{20.10.1863}{19.02.1864}
  \item \jhperson{\jhbold{\jhname[Johan]{Romar, Johan}}}{28.04.1865}{}, ``Romar-Janne''
  \item \jhperson{\jhname[Maja Lisa]{Romar, Maja Lisa}}{17.12.1867}{}
\end{jhchildren}
Johan och Maria övertog ½ hemmanet av föräldrarna år 1872. Det utgjorde då 65/384 dels mantal. Andra halvan övertogs av brodern Gustav, sannolikt samma år. Bröderna med familjer bodde i samma stora hus på älvkanten och brukade jorden tillsammans.

Johan \textdied 15.05.1884  ---  Maria \textdied 24.11.1926


%%%
% [occupant] Romar
%
\jhoccupant{Romar}{\jhname[Anders Th.son]{Romar, Anders Th.son} \& \jhname[Maria A:dr.]{Romar, Maria A:dr.}}{1854--\allowbreak 1872}
Anders Thomasson Romar, \textborn 07.12.1812, gift m. Maria Andersdr. Bärs, \textborn 26.06.1809.
\begin{jhchildren}
  \item \jhperson{\jhbold{\jhname[Johan]{Romar, Johan}}}{12.09.1834}{}
  \item \jhperson{\jhname[Anders]{Romar, Anders}}{15.05.1836}{}
  \item \jhperson{\jhname[Gustav]{Romar, Gustav}}{09.03.1838}{}
  \item \jhperson{\jhname[Erik]{Romar, Erik}}{16.04.1842}{}
  \item \jhperson{\jhname[Matts]{Romar, Matts}}{28.03.1844}{}
  \item \jhperson{\jhname[Anna Maja]{Romar, Anna Maja}}{28.01.1846}{}, (Böös 132)
  \item \jhperson{\jhname[Cajsa Greta]{Romar, Cajsa Greta}}{23.01.1848}{}
  \item \jhperson{\jhname[Susanna]{Romar, Susanna}}{12.02.1851}{}
\end{jhchildren}
Anders och Maria var bönder på ½ av föräldrarnas hemman, som han delat med brodern Erik. Den 8 dec. 1854 inhandlade han tillsammans med Johan Thomasson Romar 55/576 mantal av Romar skattehemman Nr 2 av bonden Gustav Eliasson Romar för 500 rubel silver och därtill sytning åt sytningsmannen Anders Romar och hans hustru Maria Jakobsdr.

Anders \textdied 16.05.1896  ---  Maria \textdied 02.11.1876


%%%
% [occupant] Romar
%
\jhoccupant{Romar}{\jhname[Thomas Th:son]{Romar, Thomas Th:son} \& \jhname[Anna A:dr.]{Romar, Anna A:dr.}}{- 1854}
Thomas Thomasson Romar * 04.04.1785, gift med Anna Andersdotter Silvast född 03.04. 1790.
Thomas och Anna övertog Thomas föräldrars hemman på Romar.
\begin{jhchildren}
  \item \jhperson{\jhbold{\jhname[Anders]{Romar, Anders}}}{07.12.1812}{}
  \item \jhperson{\jhname[Erik]{Romar, Erik}}{}{}
\end{jhchildren}

Thomas \textdied 11.02.1847  ---  Anna \textdied 24.04.1850


%%%
% [occupant] Romar
%
\jhoccupant{Romar}{\jhname[Thomas Th:son]{Romar, Thomas Th:son} \& \jhname[Karin M:dr.]{Romar, Karin M:dr.}}{}
Thomas Thomasson Romar, \textborn 09.12.1757,  gift med Karin Mattsdotter från Munsala, \textborn 1762. Thomas och Karin övertog sannolikt hemmanet efter Thomas föräldrar.
\begin{jhchildren}
  \item \jhperson{\jhbold{\jhname[Thomas]{Romar, Thomas}}}{04.04.1785}{}
\end{jhchildren}

Thomas \textdied 1792  ---  Karin \textdied 26.11.1803



%%%
% [house] Backlund
%
\jhhouse{Backlund}{2:121}{Romar}{2}{14, 14a, 14c-d}


\jhhousepic{010-05675.jpg}{Leif-Ole Romar och Kristina Blomberg}

%%%
% [occupant] Romar
%
\jhoccupant{Romar}{\jhname[Leif-Ole]{Romar, Leif-Ole} \& \jhname[Blomberg Kristina]{Romar, Blomberg Kristina}}{1990--}
Leif-Ole, \textborn 16.12.1960, sambo med Kristina Blomberg från Purmo, \textborn 17.04.1962.
\begin{jhchildren}
  \item \jhperson{\jhname[Tobias]{Romar, Tobias}}{1988}{}
  \item \jhperson{\jhname[Nadja]{Romar, Nadja}}{1990}{}
\end{jhchildren}
Leif-Ole övertog år 1990 Backlund lägenhet 2:18, som då omfattade ca 24 ha odlad jord och ca 46 ha skogsmark. Han har under åren utökat odlingen till ca 110 odlad jord i Jeppo förutom odlad jord också i Purmo och Munsala och ca 110ha skog. Samtidigt har han fortsatt specialseringen av potatisodlingen, som uppgår till ca 50 ha . Spannmålsproduktion upptar resten av arealen. Han har strävat efter att hålla sig i framkant vad gäller odlingsteknik och kunnande.

Nytt boningshus byggdes 1989 på älvkanten väster om landsvägen. Hemgården öster om landsvägen kvarstår som sterbhus efter modern Gertruds frånfälle och bebos av fadern Bruno Romar. Se nr 11.

Leif-Ole är verksam inom en rad områden: Styrelsen för Jeppo Potatis, Jeppo IF och IF Minken. Han var också medlem i den första styrelsen för Jeppo Food. Sambon Kristina har varit anställd på kontoret vid Jeppo Potatis mellan åren 1993--\allowbreak 2007. För tillfället arbetar hon fr.om. 2009 på Arbetskliniken i Jakobstad.



%%%
% [house] Backlund
%
\jhhouse{Backlund}{2:121}{Romar}{2}{11}


\jhhousepic{006-05529.jpg}{Bruno och Gertrud Romar}

%%%
% [occupant] Romar
%
\jhoccupant{Romar}{\jhname[Bruno]{Romar, Bruno} \& \jhname[Gertrud]{Romar, Gertrud}}{1954--\allowbreak 1990}
Bruno, \textborn 03.08.1932, gift 1956 med Gertrud Ekström, \textborn 25.11.1932.
\begin{jhchildren}
   \item \jhperson{\jhname[Jan-Erik]{Romar, Jan-Erik}}{1957}{}
   \item \jhperson{\jhbold{\jhname[Leif-Ole]{Romar, Leif-Ole}}}{1960}{}
   \item \jhperson{\jhname[Alf-Håkan]{Romar, Alf-Håkan}}{1964}{}
   \item \jhperson{\jhname[Ann-Britt]{Romar, Ann-Britt}}{1966}{}
\end{jhchildren}
Bruno övertog ½ hemmanet av sina föräldrar Vilhelm och  Edit Romar år 1954, omfattande ca 16,5 ha odlad jord. Den andra halvdelen hade 1946 överlåtits till hans bror Jarl Romar. Ett nytt boningshus byggdes nu på hemgårdstomten år 1956. I likhet med flera andra den tiden, byggdes huset av hyvlade timmerstockar ställda på stående, tätade med pappersremsor. Råmaterialet till huset kom från två av gårdens rior som tills dess stått ett par hundra meter österut på kanten av Romarsbäcken.

Småningom revs den gamla ladugården och en ny modern båsladugård byggdes åren 1972-73 med rum för 24 mjölkkor och ungdjur. Efter att hustrun Gertrud avlidit 1977, fortsatte Bruno en kort tid med mjölkkorna innan han övergick till köttproduktion, som också den efter en kort tid övergavs för en mer specialiserad produktion av potatis, inklusive spannmålsproduktion.

Bruno var Jeppo Andelsmejeris styrelseordförande vid fusionen med Mejeriandelslaget Milka 1 jan. 1974. Han har också varit ordförande i Semesternämnden i Jeppo och medlem i idrottsföreningen.

Gertrud \textdied 30.01.1977


%%%
% [oldhouse] Romar
%
\jholdhouse{Romar}{2:}{Romar}{2}{311}


\jhhousepic{Bruno Romars gla hus.jpg}{Huset nr 311 revs 1996}

%%%
% [occupant] Romar
%
\jhoccupant{Romar}{\jhname[Vilhelm]{Romar, Vilhelm} \& \jhname[Edit]{Romar, Edit}}{1908--\allowbreak 1954}
Vilhelm Isaksson,\textborn 08.04.1891, gifte sig den 15.07.1917 med Edit Jungar, \textborn 04.11.1896.

Vilhelm  övertog  ½ hemmanet 1908 av sin mor, änkan Sanna Johansdotter Romar meddelst gåvobrev tillsammans med sin syster Sanna-Kajsa och dennes make Johan Backlund, som övertog den andra halvan av hemmanet. Vilhelm bodde i det  gamla boningshuset som stod med kortändan mot landsvägen tillsammans med sin mor och också efter giftermålet med Edit. Här växte familjen upp.

Som nygift hamnade Vilhelm till striderna vid Tammerfors under frihets- och inbördeskriget 1917--18. Ovetande om vad som tilldrog sig kring Tammerfors reste Edit höggravid till stridsområdet med mat åt Vilhelm! (Vilhelm förde dagbok under fälttåget). Allt slutade lyckligt. Senare i livet undrade Vilhelm om de som var med faktiskt stred på rätt sida?

Efter att systern Sanna-Kajsas make Johan Backlund i ett barnlöst äktenskap avlidit 1929, säljer Sanna-Kajsa sin hemmansdel tillbaka till Vilhelm 1929. Han förstorar ytterligare sitt hemman genom att köpa en del av Löte lägenhet på auktion 1938. Den hade ägts av Johanna och Johannes Fagerholm som utan arvingar då  avyttrade  delar av sitt hemman. Det som återstod såldes som tomter längs med ån och nästan all bebyggelse längs Purmovägen står på den mark som ägts av Johanna och Johannes Fagerholm. Marken styckades till parceller för hemvändande frontmän 1945. Deras byggnader på Löte revs, virket hyvlades upp och kom att användas i det hus som John Helsing byggde på åbacken 1945 (se nr 17).

Vilhelm var ordf. i styrelsen för Jeppo Sparbank och satt i styrelsen för Jeppo lokalavd. av ÖSP, Jeppo Lantmannagille, Jeppo Kvarn och Jeppo-Oravais Handelslag.
\begin{jhchildren}
  \item \jhperson{\jhbold{\jhname[Jarl]{Romar, Jarl}}}{21.06.1918}{}
  \item \jhperson{\jhname[Elsbeth]{Romar, Elsbeth}}{02.05.1925}{}
  \item \jhperson{\jhname[Ethel Elise]{Romar, Ethel Elise}}{22.12.1927}{}
  \item \jhperson{\jhbold{\jhname[Bruno]{Romar, Bruno}}}{03.08.1932}{}
\end{jhchildren}
Produktionen på hemmanet skedde på traditionellt sätt med kor och ungdjur.

Vilhelm \textdied 15.02.1965  ---  Edit \textdied 26.01.1986


%%%
% [occupant] (Romar)
%
\jhoccupant{Romar}{\jhname[Isak]{Romar, Isak} \& \jhname[Sanna]{Romar, Sanna}}{1872--\allowbreak 1908}
Isak Eriksson Back (Romar), \textborn 03.08.1848, gift 1872 med Sanna Johansdotter Romar, \textborn 02.01.1852.
\begin{jhchildren}
  \item \jhperson{\jhbold{\jhname[Sanna Kajsa]{Romar, Sanna Kajsa}}}{13.03.1876}{}
  \item \jhperson{\jhname[Maria Lovisa]{Romar, Maria Lovisa}}{05.02.1880}{}
  \item \jhperson{\jhname[Anna Johanna]{Romar, Anna Johanna}}{29.06.1887}{}
  \item \jhperson{\jhname[Johan Jakob]{Romar, Johan Jakob}}{12.02.1890}{18.02.1890}
  \item \jhperson{\jhname[Vilhelm]{Romar, Vilhelm}}{08.04.1891}{}
  \item \jhperson{\jhname[Ida Sofia]{Romar, Ida Sofia}}{09.03.1894}{13.03.1894}
  \item \jhperson{\jhname[Ellen Sofia]{Romar, Ellen Sofia}}{18.06.1895}{}
\end{jhchildren}
Isak som var född på Back i Överjeppo var måg i huset och övertog tillsammans med dottern i huset, Sanna, hemmanet i början på 1870-talet efter att husbonden Johan avlidit i okt. 1871.

Isak \textdied 16.07.1907  ---  Sanna \textdied 26.05.1922


%%%
% [occupant] Romar
%
\jhoccupant{Romar}{\jhname[Johan]{Romar, Johan} \& \jhname[Kajsa]{Romar, Kajsa}}{1846--\allowbreak 1872}
Johan Erik Thomasson, \textborn 02.03.1815, gift den 05.05.1837 med Kajsa Greta Johansdotter Fors, \textborn 30.08.1814 el 1815.
Barn: Sanna, \textborn 02.01.1852. Inalles föddes 9 barn i familjen.

Johan \textdied 08.10.1871  ---  Kajsa \textdied 03.01.1899


%%%
% [occupant] Romar
%
\jhoccupant{Romar}{\jhname[Thomas]{Romar, Thomas} \& \jhname[Anna]{Romar, Anna}}{1803--\allowbreak 1846}
Thomas Thomasson, \textborn 04.04.1785, gift med Anna Andersdotter (Lillsilvast), \textborn 03.04.1790. De fick 9 barn.
\begin{jhchildren}
  \item \jhperson{\jhname[Anders]{Romar, Anders}}{07.12.1812}{}
  \item \jhperson{\jhbold{\jhname[Johan Erik]{Romar, Johan Erik}}}{02.03.1815}{}
\end{jhchildren}
Thomas och Anna övertog hemmanet på Romar år 1803 av Thomas mor Karin, som varit änka sedan 1792. Thomas var då 18 år.

Thomas \textdied 11.02(14.02).1847  ---  Anna \textdied 24.04.1850


%%%
% [occupant] Romar
%
\jhoccupant{Romar}{\jhname[Thomas]{Romar, Thomas} \& \jhname[Karin]{Romar, Karin}}{- 1803}
Thomas Thomasson, \textborn 09.12.1757, gift med Karin Mattsdotter från Munsala, \textborn 1762. Thomas och Karin övertog sannolikt hemmanet av Thomas föräldrar.
\begin{jhchildren}
  \item \jhperson{\jhbold{\jhname[Thomas]{Romar, Thomas}}}{04.04.1785}{}
  \item \jhperson{m.fl.}{}{}
\end{jhchildren}

Thomas \textdied 1792  ---  Karin \textdied 26.11.1803



%%%
% [oldhouse] Backlund
%
\jholdhouse{Backlund}{2:11}{Romar}{2}{312}


\jhhousepic{Johan Backlunds gla hus.JPG}{Rivet 1956, nr 312}

%%%
% [occupant] Backlund
%
\jhoccupant{Backlund}{\jhname[Johan]{Backlund, Johan} \& \jhname[Sanna-Kajsa]{Backlund, Sanna-Kajsa}}{1908--\allowbreak 1955}
Sanna-Kajsa Romar, \textborn 13.03.1876, gifte sig 14.11.1897 med Johan Johansson Backlund, \textborn 07.01.1877 på Måtar.

Efter att Sanna-Kajsas far Isak Romar avlidit 16 juli 1907 överlåter änkan Sanna Johansdr. (se 311) meddelst gåvobrev 1908 hemmanet att delas lika mellan syskonen Vilhelm Romar och Sanna-Kajsa med sin make Johan Backlund. De andra systrarna blir utlösta.

Johan är äldsta barnet till backstugukarlen \jhname[Johan Mattson]{Mattsson, Johan} och dennes hustru Maria Lovisa från Måtar. År 1912 flyttar Johans syster Ida Katarina till Romar som piga åt makarna och senare, när hennes man Anders Gustaf Gustafsson rest till Amerika, flyttade hon för en tid till Skog som piga åt Anders Skog. Hon flyttade senare till Kassusbackan på Jungarå (se Jungarå, karta 17, nr 114), där hon livnärde sej på att ha en s.k. ``kamahandil'', d.v.s. en liten butik.

Huset byggs på samma tomt som Sanna-Kajsas barndomshem där brodern Vilhelm tillsammans med modern fortsätter att bo. Makarna Backlund förblir barnlösa och när Johan Backlund avlider 1929 väljer Sanna-Kajsa att sälja tillbaka sin hemmansdel till brodern Vilhelm 1929, men behåller huset under sin livstid.

När Bruno Romar skall bygga sig ett nytt hus för sin familj står det gamla huset i vägen för nybygget och rivs år 1956.



%%%
% [house] Ström
%
\jhhouse{Ström}{2:24 och 2:111}{Romar}{2}{12, 12a}


\jhhousepic{008-05532}{Tor och Gunvor Lindén}

%%%
% [occupant] Lindén
%
\jhoccupant{Lindén}{\jhname[Tor]{Lindén, Tor} \& \jhname[Gunvor]{Lindén, Gunvor}}{2001--}
Tor Mikael, \textborn 05.10.1934 på Kasackbacka i Purmo, gifte sig 18.08.1957 med Gunvor Sofia Elenius, \textborn 24.10,1932 på Jungar. Makarna köpte fastigheten 2001 av Nykarleby Stad och renoverade den. Som fritidshus används den tillsammans med barnen Maria, Sofia och Tomas och deras familjer.

Tor vistades under sin uppväxt ofta i Jeppo hos sin moster \jhname[Lina Högdahl ``Post-Antas Lina'']{Högdahl, Lina} kallad (se nr 21, 321). Efter fullgjorda studier vid Sibelius Akademin fick Tor sin första kantorstjänst i Lappträsk 1956. Efter avtjänad värnplikt och giftermålet med Gunvor stannade de i Lappträsk fram till juni 1963. Detta år tillträdde han som kantor i Terjärv och verkade där till oktober 1973 då flyttlasset gick till Jakobstad och kantorstjänsten inom Jakobstads svenska församling. Den 31.10 1997 avgick han med pension efter 24 år i denna tjänst.

Gunvor studerade på SOK:s kooperativa Handelsskola i Helsingfors efter att hon börjat i arbetslivet bl.a vid mejeriet i Jeppo. Efter avslutade studier fick hon sin första tjänst på Varubodens kontor vid Sockenbacka i Helsingfors. När familjen flyttat till Terjärv 1963 fick hon arbete vid Terjärv Sparbank och efter flytten till Jakobstad 1973 fick hon arbete på Sparbanken Deposita därifrån hon efter 19 tjänstår gick i pension.


%%%
% [occupant] Widell
%
\jhoccupant{Widell}{\jhname[Margareta]{Widell, Margareta}}{1967--\allowbreak 1997}
Iris Margareta föddes 21.06.1922. Hon var enda barnet och förblev ogift. Hon bodde på fastigheten hela sitt liv och hade arbete bl.a som städerska. Hon övertog fastigheten efter sina föräldrar. Före sin egen död hade hon inte upprättat något testamente och hennes egendom tillföll staten. Staden Nykarleby anhöll om att överta dödsboet, vilket beviljades.

Margareta \textdied 24.03.1997


%%%
% [occupant] Widell
%
\jhoccupant{Widell}{\jhname[Erik]{Widell, Erik} \& \jhname[Hilda]{Widell, Hilda}}{1921--\allowbreak 1967}
Erik Johansson Widell (Lillas), \textborn 02.11.1885, gifte sig 04.09.1921 med Hilda Katarina Romar, \textborn 04.03.1891. Samma år som paret gifte sig flyttade Erik från Lillas hemman till Romar och antog nytt efternamn; \jhbold{Widell}. Hans fru Hilda var ``Romar-Jannes'' dotter och tomten utbröts nu från stomlägenheten och huset och ladugården byggdes.

Erik var också på arbetsresa till USA som så många andra och gjorde sig synbarligen en god förtjänst då han enligt vad berättas torkade
dollarsedlar utplacerade längs bräder på vinden efter att dessa av någon anledning blivit våta.

Familjen livnärde sig främst med mjölkproduktion och kor av västfinsk ras. Konkurrensen mellan ayschire och västfinsk ras var länge aktuell i bygden.

Erik \textdied 17.07.1967  ---  Hilda \textdied 06.10-1967



%%%
% [house] Backlund
%
\jhhouse{Backlund}{2:18-2(2:120)}{Romar}{2}{13, 13a-b}


\jhhousepic{009-05534}{Gundel och Bjarne Nylund}

%%%
% [occupant] Nylund
%
\jhoccupant{Nylund}{\jhname[Gundel]{Nylund, Gundel} \& \jhname[Bjarne]{Nylund, Bjarne}}{1991--}
Gundel Marianne Romar, \textborn 24.03.1948, gifte sig 14.06.1969 med Bjarne Johannes Nylund, \textborn 25.10.1946 i Sundby, Pedersöre. De övertog hemmanet 1991. Gundels föräldrar bodde därefter kvar på fastigheten. Efter att Jarl avlidit 1993 flyttade Elsie till Nykarleby centrum.

Hemmanets odlade jord har avyttrats till Leif-Ole Romar, medan skogen kvarstår i makarnas ägo. Fastigheten har ett antal år stått tom, men numera vistas deras son \jhname[Benny Nylund]{Nylund, Benny}, \textborn 08.03.1972, lärare vid Sankt Olofs skola i  Åbo, ofta här under sina ledigheter.


%%%
% [occupant] Romar
%
\jhoccupant{Romar}{\jhname[Jarl]{Romar, Jarl} \& \jhname[Elsie]{Romar, Elsie}}{1946--\allowbreak 1991}
Jarl Isak Johannes Romar, \textborn 21.06.1918 gifte sig 05.08.1945 med Elsie Iris Lovisa Nygård \textborn 05.03.1924 på Hilli. De övertog ½ av Jarls hemgård (se nr 311) den 15.03.1946 och började genast bygga ett nytt driftcentrum ca 150 m söder om hemgården. Först restes en ny ladugård, som färdigställdes 1947 och året därpå stod också boningshuset klart. Makarna bedrev ett traditionellt lantbruk med mjölkproduktion som den huvudsakliga inkomstkällan.

Jarl besökte Korsholms Lantmannaskola omedelbart före kriget.Han blev efter kriget, där han sårades, livligt engagerad i samhällsfrågor. Han var ordf. för Andelsringens förvaltningsråd i mera än 10 år. Han satt i kommunalstyrelsen hela 1960-talet och efter kommunalfusionen i Nykarleby stadsfullmäktige åren 1975--85. Därtill ingick han i flertalet kommunala nämder och kommittéer, liksom på det interkommunala planet. I kyrkliga organ satt han mera än 30 år, varav 13 år som ordf. för förvaltningsnämden. Under hans tid uppfördes bl.a. begravningskapellet.

Elsie var som ung butiksbiträde i Lassila och i Nykarleby fram till giftermålet.
\begin{jhchildren}
  \item \jhperson{\jhbold{\jhname[Gundel Marianne]{Romar, Gundel Marianne}}}{24.03.1948}{}
  \item \jhperson{\jhname[Bengt Johan Isak]{Romar, Bengt Johan Isak}}{10.04.1952}{28.05.1962}, dog i trafikolycka
\end{jhchildren}

Jarl \textdied 21.11.1993  ---  Elsie \textdied 01.11.2007



%%%
% [house] Löte
%
\jhhouse{Löte}{2:121}{Romar}{2}{313}

%%%
% [occupant] Fagerholm
%
\jhoccupant{Fagerholm}{\jhname[Johanna]{Fagerholm, Johanna} \& \jhname[Johannes]{Fagerholm, Johannes}}{1928--\allowbreak 1948}
Sanna Johanna Fagerholm, f. Romar \textborn 27.12.1887, gifte sig 06.07.1913 med Johannes Fagerholm, \textborn 04.04.1873 på Slangar. Han var son till Matts Johanss. Slangar, \textborn 07.11.1843, och dennas hustru Sofia Mattsdr., \textborn 09.10.1848.

Efter giftermålet flyttade Johanna till Slangar och verkade där som husmor till 1928. Av en syskonskara på 6 dog nu hennes enda kvarvarande bror Karl Joel den 21.06.1928. I sitt eget äktenskap med Johannes hade hon redan förlorat flera barn och nu hade de endast sonen Jurgen, \textborn 10.07.1919, kvar.

\jhpic{Troskningstalko hos Johanna o Johannes Fagerholm vid Lotes 1932.jpg}{Tröskningstalko hos Fagerholm på Löte ca 1932. Paul Björkqvist längst t.v.}

Makarna sålde nu sin egendom på Slangar åt \jhname[Simon Julin]{Julin, Simon} och flyttade till ``Löte'' på Romar och övertog Johannas hemgård  ``Lötis''. Livet gick vidare, men efter att också sonen Jurgen avlidit av lungsot 11.05.1935, förefaller livslusten ha fått sig en knäck. Efterhand avyttrar de nu sin egendom. De säljer områdena  längs Purmovägen som utstyckas till frontmannaparceller. Likaså säljer de tomterna längs ån där bl.a.  Romar nr 16 och nr 17 nu står. Slutligen låter de uppföra ett nytt hus åt sig själva (nr 328) som de överlåter åt Gunvor Sjöholm från Jakobstad mot sytning. Hemgården och uthusen rivs i slutet av 1940-talet och ingenting avslöjar idag bosättningens forna plats på Romar.

Johanna \textdied 24.02.1954  ---  Johannes \textdied 23.12.1952


%%%
% [occupant] Romar
%
\jhoccupant{Romar}{\jhname[Matts]{Romar, Matts} \& \jhname[Maria]{Romar, Maria}}{-- 1918}
Bonden Matts Karl-Gustafsson Romar, \textborn 26.01.1858, gifte sig med Maria Mattsdr., \textborn 27.09.1861, senare i livet kallad ``Lötis-Maj''. Av deras 6 barn övertog sonen Karl Joel hemgården 1918, men han blev sinnessjuk och dog 21.06.1928. Hans syster, som gift sig till Slangar, kom nu hem och övertog hemgården tillsammans med sin man Johannes.

Matts \textdied 15.04.1918  ---  Maria \textdied 26.05.1943


%%%
% [occupant] Eriksson
%
\jhoccupant{Eriksson}{\jhname[Karl-Gustaf]{Eriksson, Karl-Gustaf} \& \jhname[Susanna]{Eriksson, Susanna}}{}
Karl-Gustaf Eriksson (Romar), \textborn 21.11.1827 på Romar, verkade som bonde på Löte tillsammans med sin hustru Susanna Mattsdr., \textborn 06.01.1826. De var föräldrar åt Matts Karl-Gustafsson Romar som senare övertog hemmanet tillsammans med ``Lötis-Maj''.

Karl-Gustaf \textdied 06.08.1865  ---  Susanna \textdied 10.07.1895



%%%
% [house] Matsas
%
\jhhouse{Matsas}{2:129}{Romar}{2}{15, 15a-b}
Tidigare Hästängen, nedan.


\jhhousepic{011-05535.jpg}{Mats Norrgård och Ida King}

%%%
% [occupant] Norrgård
%
\jhoccupant{Norrgård}{\jhname[Mats]{Norrgård, Mats} \& \jhname[Ida King]{Norrgård, Ida King}}{2012--}
Fastigheten Hästängen 2:104 köptes av Nils Jungars dödsbo den 05.03.2012. Mats, \textborn 20.04.1983 på fastighet 36, karta 3, Ida Marie King kommer från Esse, \textborn 18.10.1983.

Mats är utbildad kock, men har sedan 2004 arbetat som montör vid Jeppo potatis. Ida är utbildad massör och kosmetolog och arbetar sedan 2006 vid OMI-Clinic i Jakobstad.
\begin{jhchildren}
  \item \jhperson{\jhname[Noomi Sandra Alexandra]{Norrgård, Noomi Sandra Alexandra}}{01.09.2011}{15.03.2015}
  \item \jhperson{\jhname[Mio]{Norrgård, Mio}}{20.09.2015}{}
  \item \jhperson{\jhname[Loe]{Norrgård, Loe}}{04.02.2017}{}
\end{jhchildren}

%%%
% [oldhouse] Hästängen
%
\jholdhouse{Hästängen}{2:104}{Romar}{2}{15, 315}

%%%
% [occupant] Jungar
%
\jhoccupant{Jungar}{\jhname[Nils]{Jungar, Nils}}{2008--\allowbreak 2011}
Nils, \textborn 15.03.1929, levde ensam på fastigheten sedan hans syster Karin avlidit 2008. Se mera nedan.


%%%
% [occupant] Jungar
%
\jhoccupant{Jungar}{\jhname[Nils]{Jungar, Nils} \& \jhname[Karin]{Jungar, Karin}}{1991--\allowbreak 2008}
Skiftesavtal efter avlidna Ester och Evert Jungar undertecknades 26.01.1991 mellan deras barn. Syskonen Karin, \textborn 21.04.1922, och Nils, \textborn 15.03.1929, erhöll då bl.a. fastigheten med tillhörande bostad och ekonomiebyggnad.

Nils utbildade sig till agrolog vid Svenska Lantbruksläroverket och fick till en början anställning av Dräneringsföreningen i Nyland. Efter flera år återkom han till Jeppo och övertog det praktiska ansvaret för driften av hemgården (dödsbo sedan 1965) tillsammans med systern Karin, som stannat hemma. Han var samtidigt ett flertal år lärare vid Lannäslunds Lantbruksskolor.

Produktionsinriktningen var mjölkproduktion och tillsammans nådde syskonen anmärkningsvärda resultat och syntes ofta i statistiken över de bästa besättningarna inom sv. Österbottens Lantbrukssällskaps område. Besättningen utgjordes av ca. 15 kor. Under denna tid var han också ledamot i kommunalstyrelsen.

Efter kommunsammanslagningen 1975 fick han tjänsten som lantbrukssekreterare i den nybildade kommunen, en tjänst som han innehade till sin pensionering 1994. I sin tjänsteutövning var han besjälad av ett starkt rättvisetänkande.

Hembygdsföreningen stod honom varmt om hjärtat och otaliga är de timmar som han tillbringat på talkoarbete och på sina vandringar i skog och mark på jakt efter fornlämningar. En karta över dessa finns uppsatt i Hembygdsgården.

Karin har under hela sitt liv varit hemgården trogen och bidragit med sitt idoga arbete till dess förkovring.

Karin \textdied 30.07.2008  ---  Nils \textdied 25.08.2011


%%%
% [occupant] Jungar
%
\jhoccupant{Jungar}{\jhname[Evert]{Jungar, Evert} \& \jhname[Esters dödsbo]{Jungar, Esters dödsbo}}{1965--\allowbreak 1991}
Efter år 1965 avlidna Ester och år 1979 avlidna Evert, förvaltades hemmanet som ett dödsbo fram till skiftesavtalet 1991. Under denna tid påbörjades en lantmäteriförrättning som innebar en styckning och omkretsrågång mellan Evert Jungar och Edvin Romar, från vars stomlägenhet Hästängen lägenhet utstyckades. Sämjobyte hade varit rådande från 1931 fram till 1994, då förrättningen avslutades.


\jhhousepic{Evert J 1.JPG}{Vy från landsvägen, nr 315}

%%%
% [occupant] Jungar
%
\jhoccupant{Jungar}{\jhname[Evert]{Jungar, Evert} \& \jhname[Ester]{Jungar, Ester}}{1931--\allowbreak 1965}
Evert, \textborn 10.09.1899, gifte sig 1921 med Ester Romar, \textborn 15.01.1897. Brudparet gick från brudens hemgård på Romar till kyrkan för vigsel, tillsammans med hela brudföljet och spelmän över den flottbro, som var lagd sommartid vid ``Färistranden''.

\jhhousepic[pic:flottbro]{Evert o Ester Jungars brollop 1921}{Evert och Ester Jungars bröllopsfölje vandrar över flottbron 1921}

Efter giftermålet arbetade Evert en tid på hemgården vid Jungar innan han 29.08.1923 som så många andra reste till Amerika för att förtjäna pengar.

Han hade 1917 börjat studera vid Korsholms lantbruksskola och under denna tid deltog han frivilligt i frihets/inbördes/medborgarkriget. Han deltog i avväpningen av de ryska styrkorna i Vasa (som för en kort tid blev Finlands huvudstad), fortsatte till Gamlakarleby där samma sak upprepades. Därefter fortsatte resan norrut till Uleåborg, där han ovetande om situationen träffade på sin far Daniel. På vägen söderut insjuknade han i lunginflammation och vårdades antingen i Gamlakarleby eller Jakobstad. Något senare, ca 1920, blev han chef för skyddskåren i Jeppo och kunde med detta uppdrag bära en ståtlig värja.

Under vistelsen i Amerika blev han varnad av sina landsmän att ``vakta sig väl'' då det synbarligen fanns de som fått reda på att han varit skyddskårschef. Rådet var att resa bort från Hibbing för att undgå repressalier från ev. ``röda'' emigranter. 1928 återkom han från Amerika med kanske mindre pengar än vad han hoppats. Han var nu ca. ett år mjölnare vid Jungar och bodde i ett hus nära kvarnen med sin familj.

1931 köpte Evert och Ester hälften av hennes far ``Romar-Jannes'' hemman R:nr 2:25. Ett nytt hus byggdes (nr 315) med timmer som fällts på Fagerlandet,Jungar. Djuren var installade i grannarnas fähus till dess en gammal ria från Everts hemgård flyttades och fick tjäna som fähus, utvidgat med  ett utrymme för häst och svin, byggd av halmbalar som väggar.

Under krigen var Evert flera gånger till fronten med matleveranser åt Jeppobor stationerade vid Svir. Bl.a. var bagare Einar Sundell med på en resa och gjorde stor succé med sitt kunnande. Som tack för hjälpen förärades Evert en oljemålning skänkt av pojkarna vid Svir. Den överräcktes självständighetsdagen 6.12.1944.

Efter kriget byggdes en ny ladugård av självslaget cementtegel och 1972 flyttade familjen in i nytt boningshus.

En olycka som kunnat sluta illa inträffade en gång när han skulle med häst och kärra hämta hö från en ängslada. När han skulle kliva in i ladan snavade han och spetsades på en stör som trängde genom låret. Ensam som han var hade han stora svårigheter att få stören utdragen, men när han väl lyckades lämnade stören ett stort hål i låret. Hängslena tog han av sej och drog hårt om låret för att minska blodförlusten. Väl tillbaka på kärran manade han på hästen som hittade hem och svårt medtagen kunde Evert komma under vård.

Evert var intresserad av husdjursavel och var medlem i styrelsen för Jeppo Ayrshireförening. Hästar, hästavel och travtävlingar var också ett stort intresse. På isen vid Romar deltog han ibland i de travtävlingar som arrangerades.
\begin{jhchildren}
  \item \jhperson{\jhbold{\jhname[Karin]{Jungar, Karin}}}{21.04.1922}{30.07.2008}
  \item \jhperson{\jhname[Per]{Jungar, Per}}{22.11.1923}{05.09.2009}, arbetat som svarvare vid Wärtsilä, J:stad
  \item \jhperson{\jhbold{\jhname[Nils]{Jungar, Nils}}}{15.03.1929}{25.08.2011}
  \item \jhperson{\jhname[Hans]{Jungar, Hans}}{06.02.1938}{}, komm.sekr. i Larsmo och ekon.dir. vid kretssjukhuset, J:stad
\end{jhchildren}
Ester \textdied 1965  ---  Evert \textdied 1979



%%%
% [subsection] ``Färistranden'' och flottbron
%
\jhsubsection{``Färistranden'' och flottbron}

I Jeppobygden har det alltid funnits behov att röra sej tvärs över älven; emellanåt över dess bägge armar. Vintertid gick det någorlunda när isen var tillräckligt stark, vilken den oftast var. Efter islossningen fram till isläggningen på senhösten var det värre. Flottbroar och enkla broar lagda på bockar och stenar var tidiga lösningar på problemet. När de först prövats är omöjligt att säga, men det är säkert för flera hundra år sedan.

Efter att kyrkan byggts på Stenbacken och järnvägen ett kvartsekel senare etablerade sig med station på Silvast hemman och bebyggelsen där snabbt expanderade, blev behovet av en någorlunda fungerande bro under sommartid allt mer akut. Folk måste ju kunna ta sig till kyrkan utan att bli tvungna att hoppa från sten till sten i forsarna.

Hur länge en flottbro vid ``Färistranden'' funnits vet vi inte, men i den form den finns dokumenterad har det skett under 1800-talet. Flottbron var av naturliga skäl ganska smal. Endast en häst med kärra kunde rymmas åt gången och bärförmågan tålde kanske inte mer. Men på  ena sidan fanns en förhöjning, ”trottoar”, där de gående kunde röra sig utan att bli våta om fötterna. Tomma ekfat på båda sidorna bidrog till bärighet och stabilitet.

På vilket sätt bron underhölls finns inte berättat, men istället för att plocka sönder den inför vintern svängdes den in mot stranden på Romars sida och förankrades stadigt. Vid kraftiga vårflöden och islossningar blev den säkert illa åtgången, men efter översyn och eventuella reparationer svängdes den på nytt ut över ån när vattenståndet normaliserats.

Den var i användning fram till sommaren 1930, d.v.s. ännu nästan 20 år efter att betongbroarna vid Fors och Mjölnars (Kiitola) byggts. Det var helt enkelt närmare till kyrkan både från Skog, Romar, Silvast och Fors´ gårdsgrupper. På den något suddiga bild \ref{pic:flottbro} som presenteras, ser vi Evert Jungar och Ester Romars brudfölje passera bron den 10 sept 1921. Se närmare Romar nr 15.

Vid Färistranden var det också populärt att bada och otaliga är de barn som här under varma sommarkvällar svalkat och roat sig i älvens vatten.

\jhhousepic{Simstranden Romar 1938.jpeg}{Hjördis Romar och Signe Jungell övervakar simfantasterna vid Romar omkr. 1938}

Det har tidvis funnits krafter för att här bygga en permanent bro över älven. Bl.a. bildades det en broförening efter krigen för insamling av pengar till ett brobygge. En egen speciell minnesadress, som kunde köpas i samband med jordfästningar, skulle ge en grund för projektet. Den syns i bild \ref{pic: broadress}. Inflationen var ändå så stark att insamlingarna inte förmådde hålla jämna steg, även om eldsjälarna med diakonissan Edit Backlund och lärarinnan Karin Sjöblom i spetsen gjorde sitt bästa. När Edit Backlund dog 22.02.1970 lades verksamheten ner och de medel som insamlats, ca 8000mk, donerades som en grundplåt för församlingshemmet vid kyrkan.



%%%
% [house] Älvbranten
%
\jhhouse{Älvbranten}{2:53}{Romar}{2}{16, 16a}


\jhhousepic{012-05538.jpg}{Johan och Barbro Stenfors}

%%%
% [occupant] Stenfors
%
\jhoccupant{Stenfors}{\jhname[Johan]{Stenfors, Johan} \& \jhname[Barbro]{Stenfors, Barbro}}{1976--}
Johan Uno, \textborn 24.02.1937 i Övermark, gift 1963 med Barbro Helena Karls, \textborn 11.05.1940 i Oravais.
\begin{jhchildren}
  \item \jhperson{\jhname[Anne-Marie Helena]{Stenfors, Anne-Marie Helena}}{1964}{}
  \item \jhperson{\jhname[Johan Krister]{Stenfors, Johan Krister}}{1967}{1978}, omkom i trafikolycka
  \item \jhperson{\jhname[Dan-Ove Johan Mikael]{Stenfors, Dan-Ove Johan Mikael}}{1979}{}
\end{jhchildren}

Johan och Barbro köpte tomtmarken 1976 av \jhname[Verna och Lars-Erik Häll]{Häll, Verna, Lars-Erik}, änka och son till framlidne VD för Jeppo Oravais Handelslag, \jhname[Sven Häll]{Häll, Sven}. Tomten hade 15.06.1949 inhandlats av Sven Häll, \textborn 15.11.1911 i Esse, i akt och mening att användas som boplats, men inte blivit utnyttjad för detta ändamål. Området hade ursprungligen tillhört lägenheten Löte R:nr 2:41 och sålts av dess dåvarande ägare Johannes och Johanna Fagerholm.

Johan och Barbro uppförde huset som elementhus 1977 och det blev inflyttningsklart till julen samma år.

Före ankomsten till Jeppo var Johan verksam inom byggföretaget Helsing \& Ollil i Vasa och som ombudsman för Småföretagarförbundet i Sv. Österbotten. Johan har under åren 1965--\allowbreak 1991 fungerat som VD för Jeppo Sparbank. Efter fusionen med Sparbanken Deposita i Jakobstad var han bankdirektör vid kontoret i Jeppo fram till 1993. Efter ett års sjukskrivning gick han i pension 1994.

Johan har fungerat som offentligt köpvittne och häradsdomare.	Likaså har han varit invald i Nykarleby stadsfullmäktige. Därtill har	han aktivt medverkat vid startandet av Konstskolan i Nykarleby,	Jeppo Potatis, Jeppo Food, Silvast Antennandelslag. Också föreningslivet har legat honom varmt om hjärtat. Han var en av chartermedlemmarna då Lions Club Jeppo startades 1966 och var klubbens första president.

Barbro har arbetat som bokförare vid Vasa Sparbank och huvudbokförare vid Mirka, Keppo. Sedan 1984 har hon varit egen företagare som bokförare åt lantbrukare och företagare fram till pensioneringen 2006.



%%%
% [house] Åbacka
%
\jhhouse{Åbacka}{2:31}{Romar}{2}{17, 17a-b}


\jhhousepic{013-05539.jpg}{Pekka och Johanna Kalliosaari}

%%%
% [occupant] Kalliosaari
%
\jhoccupant{Kalliosaari}{\jhname[Pekka]{Kalliosaari, Pekka} \& \jhname[Johanna]{Kalliosaari, Johanna}}{1997--}
Pekka Kalliosaari, \textborn 1971 i Alahärmä, gifte sig 1999 med Johanna, \textborn 1975 i Kauhajoki. De köpte fastigheten av Edgar och Hjördis Eklund 1997. Med stort intresse och energi har de sedan dess renoverat huset från källare till taknock.

Pekka fick 1989 anställning som köttskärare vid Snellmans slakteri när det ännu fanns på Skatan i Jakobstad. Då blev han arbetstagare nr 68. Nu har företaget över 1000 anställda! Under tiden har han avancerat till produktionschef inom företaget. Johanna var hemma när barnen var små, men har fr.o.m. 2001	anställning som caféföreståndare på Kuntokeskus i Härmä.
\begin{jhchildren}
  \item \jhperson{\jhname[Mia]{Kalliosaari, Mia}}{1995}{}, stud. till sjuksköterska
  \item \jhperson{\jhname[Pinja]{Kalliosaari, Pinja}}{1995}{}, stud. intern. business på YH
  \item \jhperson{\jhname[Jutta]{Kalliosaari, Jutta}}{1998}{}
\end{jhchildren}


%%%
% [occupant] Eklund
%
\jhoccupant{Eklund}{\jhname[Edgar]{Eklund, Edgar} \& \jhname[Hjördis]{Eklund, Hjördis}}{1961--\allowbreak 1997}
Edgar Eklund, \textborn 13.11.1934, gift 22.07.1961 med Hjördis Nygård, \textborn 15.09.1932. De köpte fastigheten 1961 av John och Adele Helsing som vid den tidpunkten hade flyttat till Yttermark.

Edgar född på Stenbacken utbildade sig till elmontör och var under sin aktiva tid i Jeppo länge anställd som montör vid Jeppo	Kraft. Han 	råkade i slutet av 1950-talet ut för en svår arbetsolycka med kraftig elström genom kroppen under linjearbete vid Centralvägen och blev 	hängande i el-linjen tills strömmen kunde brytas. Han tillfrisknade.

I början av 1960-talet fick han anställning vid Schaumanns fabriker i Jakobstad och råkade otroligt nog ut för en liknande olycka, trots att också där enligt uppgift strömmen var bortkopplad, och där gnistbildningen tände eld på hans arbetskläder. Också nu klarade han sig utan livshotande skador.

Hjördis arbetade till en början som butiksbiträde vid John Lasséns kolonialvaruaffär innan hon tog anställning vid Gleisners butik i Nykarleby centrum. Efter en tid fick hon anställning vid Andelsringens huvudaffär i Jeppo såväl på affärens kött- som tygavdelning.

I medlet av 80-talet flyttade de till Nykarleby centrum och bodde på 	somrarna kvar i sitt hus i Jeppo innan de slutligen 1997 sålde det.
Barn:  \jhname[Sonja]{Eklund, Sonja}, \textborn 15.03.1963, reseplanerare på Ingves \& Svanbäck.


%%%
% [occupant] Helsing
%
\jhoccupant{Helsing}{\jhname[John]{Helsing, John} \& \jhname[Adele]{Helsing, Adele}}{1944--\allowbreak 1961}
John Helsing, \textborn 01.12.1915 i Vexala, gift 1937 med Adele Alexandra Andersson, \textborn 29.05.1918, också hon från Vexala. Den 12.06.1944  köpte de tomten som utstyckades från Löte hemman Nr 2:12 vid Harisloon. Uppförandet av huset startade genast med timmer som 	köpts och tidigare utgjort bostad för ``Lötis Maja'' vid Romar. Timret hyvlades upp och timrades stående enligt ny teknik för att senare 	rappas och vitkalkas. Inflyttningen skedde hösten 1945.

John hade genomgått en 1-årig utbildning vid Korsholms Lantbruksskolor och fick anställning till en början vid Varma andelslag i	Vexala för att senare flytta till Varma i Jeppo som föreståndare. Han sårades allvarligt 1942 vid Svirfrontens Schemenskiavsnitt	under fortsättningskriget och befriades från fronttjänst. Efter att ha tillfrisknat fortsatte han sin tjänstgöring vid Varma i Jeppo. År 1947	fick han anställning som lagerföreståndare vid Jeppo-Oravais Handelslag, en tjänst han innehade till hösten 1960 då han fick tjänst som 	föreståndare vid Yttermark Andelshandel.

1971 gick flyttlasset tillbaka till Jeppo, där han övertog posten som	direktör efter avlidne Sven Häll för tidigare nämnda Jeppo-Oravais	Handelslag, som nu bytt namn till Andelsringen. Här stannade han	fram till sin pensionering 1978, varefter han tillsammans med hustrun Adele flyttade till Nykarleby centrum.

Adele hade i unga år gått i sylära hos fru Elsa Kennola i Jeppo.	Sömnad på förtjänst blev ändå aldrig riktigt aktuell, men däremot	stickning med stickmaskin. Tröjor, pullovrar och kalsonger stickades fram till flytten till Yttermark. Därtill hade hon ansvaret för familjens ko, kalv och höns, som fanns i den lilla ladugården på åkanten.
\begin{jhchildren}
  \item \jhperson{\jhname[Ulla Johanna]{Helsing, Ulla Johanna}}{05.09.1937}{02.03.2004}, född i Vexala*)
  \item \jhperson{\jhname[Kita Mari-Ann]{Helsing, Kita Mari-Ann}}{09.01.1939}{}, född i Vexala
  \item \jhperson{\jhname[Alf Erik]{Helsing, Alf Erik}}{18.03.1941}{}, i Jeppo
  \item \jhperson{\jhname[Anna Christina]{Helsing, Anna Christina}}{24.09.1944}{}, i Jeppo
  \item \jhperson{\jhname[Tea Ingegerd Susanna]{Helsing, Tea Ingegerd Susanna}}{09.12.1947}{}, i Jeppo
  \item \jhperson{\jhname[Kjell Anders Johan]{Helsing, Kjell Anders Johan}}{02.02.1959}{}, i Jeppo
\end{jhchildren}

*)Födelsetiden är i kyrkboken angiven till den 28.09.1937, men är	enligt familjen felaktig, vilket berett henne besvär under hela hennes	levnad. Hon dog i Tumba, Sverige den 02.03.2004.

John Helsing \textdied 18.12.1989  ---  Adele Helsing \textdied 21.08.2008
